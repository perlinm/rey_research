\documentclass[nofootinbib,notitlepage,11pt]{revtex4-2}

%%% linking references
\usepackage{hyperref}
\hypersetup{
  breaklinks=true,
  colorlinks=true,
  linkcolor=blue,
  filecolor=magenta,
  urlcolor=cyan,
}

%%% header / footer
\usepackage{fancyhdr} % easier header and footer management
\pagestyle{fancy} % page formatting style
\fancyhf{} % clear all header and footer text
\renewcommand{\headrulewidth}{0pt} % remove horizontal line in header
\usepackage{lastpage} % for referencing last page
\cfoot{\thepage~of \pageref{LastPage}} % "x of y" page labeling


%%% symbols, notations, etc.
\usepackage{physics,braket,bm,amssymb} % physics and math
\renewcommand{\t}{\text} % text in math mode
\newcommand{\f}[2]{\dfrac{#1}{#2}} % shorthand for fractions
\newcommand{\p}[1]{\left(#1\right)} % parenthesis
\renewcommand{\sp}[1]{\left[#1\right]} % square parenthesis
\renewcommand{\set}[1]{\left\{#1\right\}} % curly parenthesis
\newcommand{\bk}{\Braket} % shorthand for braket notation
\renewcommand{\v}{\bm} % bold vectors
\newcommand{\uv}[1]{\bm{\hat{#1}}} % unit vectors
\newcommand{\av}{\vec} % arrow vectors
\renewcommand{\d}{\text{d}} % for infinitesimals
\renewcommand{\c}{\cdot} % inner product

\usepackage{dsfont} % for identity operator
\newcommand{\1}{\mathds{1}}

\newcommand{\x}{\text{x}}
\newcommand{\y}{\text{y}}
\newcommand{\z}{\text{z}}

\newcommand{\D}{\mathcal{D}}
\renewcommand{\O}{\mathcal{O}}
\newcommand{\T}{\mathcal{T}}

%%%%%%%%%%%%%%%%%%%%%%%%%%%%%%%%%%%%%%%%%%%%%%%%%%%%%%%%%%%%%%%%%%%%%%
\begin{document}

\title{Multi-level generalizations of collective spin dynamics and
  squeezing}%
\author{Michael A. Perlin}%
\date{\today}

\maketitle
\thispagestyle{fancy}

%%%%%%%%%%%%%%%%%%%%%%%%%%%%%%%%%%%%%%%%%%%%%%%%%%%%%%%%%%%%%%%%%%%%%%
\section{Uniform interactions and the Generalized Gell-Mann matrices}

Under the frozen-mode approximation, SU($n$)-symmetric uniform
all-to-all two-body interactions between fermions on a lattice take
the form\cite{perlin2019effective}
\begin{align}
  H_{\t{int}} = \f{u}{2} \sum_{p,q,\mu,\nu}
  \p{c_{p\mu}^\dag c_{q\nu}^\dag c_{q\nu} c_{p\mu}
    + c_{q\mu}^\dag c_{p\nu}^\dag c_{q\nu} c_{p\mu}},
\end{align}
where $u\equiv U/L$ is the two-body on-site interaction energy $U$
divided by the number of lattice sites $L$, and $c_{p\mu}$ is the
annihilation operator for a fermionic mode with quasi-momentum $p$ and
internal state (e.g.~nuclear spin) $\mu$.  Defining the spin operators
$s_{\mu\nu}^{(p)}\equiv c_{p\mu}^\dag c_{p\nu}$ for each
quasi-momentum $p$, we can alternately write
\begin{align}
  H_{\t{int}} = \f{u}{2} \sum_{p,q,\mu,\nu}
  \p{s_{\mu\mu}^{(p)} s_{\nu\nu}^{(q)} - s_{\mu\nu}^{(p)} s_{\nu\mu}^{(q)}}
  = \f{u}{2}\p{N^2 - \sum_{p,q,\mu,\nu} s_{\mu\nu}^{(p)} s_{\nu\mu}^{(q)}},
\end{align}
where $N$ is the total number of particles.  In the absence of
coherence between sectors of different particle number, we can neglect
the $\sim N^2$ term and simply write
\begin{align}
  H_{\t{int}}
  = - \f{u}{2} \sum_{p,q,\mu,\nu} s_{\mu\nu}^{(p)} s_{\nu\mu}^{(q)}.
  \label{eq:H_int_start}
\end{align}
In the case of SU(2), at this point we would expand the spin operators
$s_{\mu\nu}^{(p)}$ in terms of Pauli operators in order to convert
$H_{\t{int}}$ into an SU(2) spin Hamiltonian.  To generalize this
procedure to SU($n$), we define the {\it generalized Gell-Mann (GGM)
  operators}\cite{hioe1981level, bertlmann2008bloch} for
$j,k,\ell\in\set{0,1,\cdots,n-1}$ with $k<j$ and $\ell>0$:
\begin{align}
  \lambda_0 \equiv \sqrt{\f{2}{n}}~ \1,
  &&
  \lambda_\ell \equiv \sqrt{\f{2}{\ell\p{\ell+1}}}
  \p{\sum_{n=0}^{\ell-1}\op{n} - \ell \op{\ell}},
  \label{eq:GGM_diag}
\end{align}
\begin{align}
  \lambda_{jk,+} \equiv \op{j}{k} + \op{k}{j},
  &&
  \lambda_{jk,-} \equiv i\p{\op{j}{k} - \op{k}{j}},
  \label{eq:GGM_off_diag}
\end{align}
where $\1$ is the identity operator.  The GGM operators in
\eqref{eq:GGM_diag} and \eqref{eq:GGM_off_diag} coincide exactly with
the Pauli operators in the case of $n=2$, generally provide a complete
basis for operators on the Hilbert space of an $n$-level system, and
satisfy the orthonormality relation
\begin{align}
  \bk{\lambda_a,\lambda_b}
  \equiv \tr\p{\lambda_a^\dag\lambda_b}
  = 2\delta_{ab}.
  \label{eq:GGM_inner}
\end{align}
Denoting a vector of all GGM operators by $\v\lambda$, we can make the
remarkable simplification
\begin{align}
  H_{\t{int}}
  = -\f{u}{4} \sum_{p,q} \v\lambda^{(p)} \c \v\lambda^{(q)}
  = -u \v\Lambda \c \v\Lambda,
  \label{eq:H_int_GGM}
\end{align}
where $\O^{(p)}$ denotes the action of $\O$ on the degrees of freedom
at quasi-momentum mode $p$, and
$\v\Lambda\equiv\f12\sum_p\v\lambda^{(p)}$ is a collective SU($n$)
spin vector.

%%%%%%%%%%%%%%%%%%%%%%%%%%%%%%%%%%%%%%%%%%%%%%%%%%%%%%%%%%%%%%%%%%%%%%
\section{Transition operators}

The GGM operators in \eqref{eq:GGM_diag} and \eqref{eq:GGM_off_diag}
provide a convenient operator basis to describe dynamics obeying
SU($n$) symmetry.  An external driving field addressing $n$-level
nuclear spins, however, will generally violate SU($n$) symmetry.  Such
a field obeys the symmetries of the {\it polarization} or {\it
  transition operators} defined by\cite{kryszewski2006alternative,
  bertlmann2008bloch}
\begin{align}
  T_{JM} \equiv \sqrt{\f{2J+1}{2I+1}} \sum_{\mu,\nu}
  \bk{I\mu;I\nu,JM} \op{\mu}{\nu},
  \label{eq:trans_ops}
\end{align}
where $J\in\set{0,1,\cdots,n-1}$ and $M\in\set{-J,-J+1,J}$ index a
total spin and its projection onto a quantization axis;
$I\equiv\p{n-1}/2$ is the maximal angular momentum of a spin-$n$
system; $\mu,\nu\in\set{-I,-I+1,\cdots,I}$ index projections of an
$n$-level nuclear spin onto a quantization axis; and
$\bk{I\mu,JM;I\nu}$ is a Clebsch-Gordan coefficient.  Similarly to the
GGM operators in \eqref{eq:GGM_diag} and \eqref{eq:GGM_off_diag}, the
transition operators in \eqref{eq:trans_ops} provide a complete basis
for the space of operators on the Hilbert space of an $n$-level
system, and satisfy the orthonormality relation
\begin{align}
  \bk{T_{JM},T_{J'M'}}
  \equiv \tr\p{T_{JM}^\dag T_{J'M'}}
  = \delta_{JJ'} \delta_{MM'}.
\end{align}
Written in terms of the transition operators, the interaction
Hamiltonian in \eqref{eq:H_int_start} takes the form
\begin{align}
  H_{\t{int}} = -\f{u}{2} \sum_{p,q} {\v T^{(p)}}^\dag \c \v T^{(q)}
  = -\f{u}{2} \v\T^\dag \c \v\T,
  \label{eq:H_int_trans}
\end{align}
where $\v T$ is a vector of all transition operators in
\eqref{eq:trans_ops} and $\v\T\equiv\sum_p\v T^{(p)}$ is a collective
transition vector.

%%%%%%%%%%%%%%%%%%%%%%%%%%%%%%%%%%%%%%%%%%%%%%%%%%%%%%%%%%%%%%%%%%%%%%
\section{Drive operators}

Unlike the GGM operators in \eqref{eq:GGM_diag} and
\eqref{eq:GGM_off_diag}, the transition operators in
\eqref{eq:trans_ops} are not self-adjoint.  In order to express
Hamiltonians in terms of self-adjoint operators, we define the {\it
  drive operators}
\begin{align}
  D_{J,0} \equiv \sqrt{2}~ T_{J,0},
  &&
  D_{J\Delta,+} \equiv -\p{T_{J\Delta} + T_{J\Delta}^\dag},
  &&
  D_{J\Delta,-} \equiv i\p{T_{J\Delta} - T_{J\Delta}^\dag},
  \label{eq:drive_ops}
\end{align}
for $\Delta\in\set{1,2,\cdots,J}$.  The drive operators satisfy the
same orthonormality relations as the GGM operators, provided in
\eqref{eq:GGM_inner}, and are similarly equal to the Pauli operators
in the case of $n=2$.  These operators have the convenient feature
that $D_{1,1,\pm}$ is (up to a gauge transformation) proportional to
the Hamiltonian induced on a spin-$n$ system by a classical driving
laser.  In terms of the drive operators,
\begin{align}
  H_{\t{int}} = -\f{u}{4} \sum_{p,q} \v D^{(p)}\c\v D^{(q)}
  = -u \v\D \c \v\D,
  \label{eq:H_int_drive}
\end{align}
where $\v D$ is a vector of all drive operators in
\eqref{eq:drive_ops} and $\v\D\equiv\f12\sum_p\v D^{(p)}$ is a
collective drive vector.

\bibliography{sun_notes.bib}

\end{document}
