\documentclass[aspectratio=43,usenames,dvipsnames,fleqn]{beamer}

%%% beamer theme
\usepackage{beamerthemelined}
\usetheme{default} % theme with outline sidebar on the right
\setbeamertemplate{headline}{} % clear header
\setbeamertemplate{navigation symbols}{} % clear navigation symbols
\setbeamertemplate{footline} % place frame number in footer
{
  \hbox{\begin{beamercolorbox}
      [wd=1\paperwidth,ht=2.25ex,dp=1ex,right]{framenumber}
      \usebeamercolor[fg]{subtitle}
      \insertframenumber{} / \inserttotalframenumber
      \hspace*{2ex}
    \end{beamercolorbox}}
}

\usepackage{changepage} % for adjustwidth environment

% remove the "Figure:" prefix from figure captions
\usepackage{caption}
\captionsetup[figure]{labelformat=empty}

\usepackage{graphicx} % for figures
\graphicspath{{../figures/}} % set path for all figures
\usepackage{grffile} % for figures with a dot in the file name

%%% symbols, notations, etc.
\usepackage{physics,braket,bm,amssymb} % physics and math
\usefonttheme[onlymath]{serif} % "regular" math font

\renewcommand{\t}{\text} % text in math mode
\newcommand{\f}[2]{\dfrac{#1}{#2}} % shorthand for fractions
\newcommand{\p}[1]{\left(#1\right)} % parenthesis
\renewcommand{\sp}[1]{\left[#1\right]} % square parenthesis
\renewcommand{\set}[1]{\left\{#1\right\}} % curly parenthesis
\newcommand{\bk}{\braket} % shorthand for braket notation

\renewcommand{\c}{\cdot}

\newcommand{\m}{\bm} % bold symbol
\renewcommand{\v}{\vec} % arrow vector

\usepackage{dsfont}
\newcommand{\1}{\mathds{1}}

\newcommand{\up}{\uparrow}
\newcommand{\dn}{\downarrow}

\renewcommand{\d}{\text{d}}
\newcommand{\x}{\text{x}}
\newcommand{\y}{\text{y}}
\newcommand{\z}{\text{z}}

\newcommand{\E}{\mathcal{E}}
\renewcommand{\O}{\mathcal{O}}
\newcommand{\Q}{\mathcal{Q}}
\renewcommand{\P}{\mathcal{P}}
\newcommand{\V}{\mathcal{V}}

\newcommand{\NN}{\mathbb{N}}
\newcommand{\ZZ}{\mathbb{Z}}

\let\var\relax
\DeclareMathOperator{\var}{var}
\newcommand{\ul}{\underline}
\newcommand{\EQPS}{=_{\text{PS}}}
\newcommand{\PS}{\text{PS}}

\renewcommand*{\thefootnote}{\alph{footnote}}

%%% figures
\usepackage{graphicx} % for figures
\graphicspath{{./figures/}} % set path for all figures
\usepackage[export]{adjustbox} % for vertical alignment in math
\newcommand{\diagram}[1]
{\,\includegraphics[valign=c,scale=0.75]{diagrams/#1.pdf}\,}

% custom colors
\usepackage{xcolor}
\definecolor{orange}{RGB}{255,127,14}
\definecolor{green}{RGB}{44,160,44}
\newcommand{\mblue}[1]{\textcolor{blue}{#1}}
\newcommand{\morange}[1]{\textcolor{orange}{#1}}
\newcommand{\mgreen}[1]{\textcolor{green}{#1}}
\newcommand{\mred}[1]{\textcolor{red}{#1}}

% for drawing arrows
\usepackage{tikz}
\usetikzlibrary{arrows,shapes}
\tikzstyle{every picture}+=[remember picture]
\newcommand{\tikzmark}[1]{
  \tikz[remember picture] \node[coordinate] (#1) {#1};
}

% for uncovering under/over braces
\newcommand<>{\uubrace}[2]{%
  \onslide#3 \underbrace{ \onslide<1->%
  #1%
  \onslide#3 }_{#2} \onslide<1->%
}
\newcommand<>{\uobrace}[2]{%
  \onslide#3 \overbrace{ \onslide<1->%
  #1%
  \onslide#3 }^{#2} \onslide<1->%
}

% for "not approximately equal" symbol
\usepackage{accsupp}
\providecommand*{\napprox}{%
  \BeginAccSupp{method=hex,unicode,ActualText=2249}%
  \not\approx
  \EndAccSupp{}%
}

%%%%%%%%%%%%%%%%%%%%%%%%%%%%%%%%%%%%%%%%%%%%%%%%%%%%%%%%%%%%%%%%%%%%%%

\title{SU($n$) ferromagnets near the Heisenberg point}%
\author{Michael A. Perlin}%
\date{17 March 2020}

\begin{document}

\begin{frame}[plain]
  \titlepage
\end{frame}
\addtocounter{framenumber}{-1}

\begin{frame}
  \frametitle{Origins and motivation}
  \begin{itemize}[<+->] \setlength{\itemsep}{1em}
  \item SU(2) spin squeezing: $\v s\c\v s$ interactions for a
    many-body gap
    \vspace{.5em}
    \begin{itemize}[<+->]
    \item Single-body $s_\z$ ``perturbation'' $\to$ one-axis twisting
      ($S_\z^2$)
    \end{itemize}
  \item Dipolar and trapped ion systems $\leftrightarrow$ XXZ
    model\vspace{.5em}
    \begin{itemize}[<+->]
    \item SU(2) interactions + two-body (ZZ) terms
    \end{itemize}
  \item Nuclear spin degrees of freedom $\to$ SU($n$) interactions
    \vspace{.5em}
    \begin{itemize}[<+->] \setlength{\itemsep}{.5em}
    \item Enhanced symmetry, large ground-state degeneracy
    \item Metrological / QIP resource (?)
    \item Simulating SY/SYK-like models (scrambling, holography, etc.)
    \item + more SU($n$) physics (e.g.~dynamical phases)
    \end{itemize}
  \item ``Perturbed'' SU($n$) ferromagnets
    \vspace{.5em}
    \begin{itemize}[<+->]
    \item Neighborhood of the Heisenberg critical point
    \end{itemize}
  \end{itemize}
\end{frame}

\begin{frame}
  \frametitle{SU($n$) Heisenberg ferromagnets}
  \begin{itemize} \setlength{\itemsep}{1.5em}
  \item SU($n$)-symmetric two-body interactions:
    \vspace{-.5em}
    \begin{align*}
      H_{\t{int}} = \sum_{\substack{\t{spin pairs}\\\p{p,q}}}
      {\color<2>{red} h_{pq}} \uobrace<4->{
        \sum_{\substack{\t{spin states}\\\mu,\nu}}
        \color<3>{red} s_{\mu\nu}^{(p)} s_{\nu\mu}^{(q)}}
      {\substack{\t{exchange}\\\t{interaction}}}
      \uncover<5->{
        = \sum_{\substack{\t{spin pairs}\\\p{p,q}}} h_{pq}
        \overbrace{\Pi_{pq}}^{\substack{\t{permutation}\\\t{operator}}}
      }
    \end{align*}
    \vspace{-1em}
  \item<2-> Power-law couplings:
    ${\color<2>{red} h_{pq}} \sim -\f1{\abs{p-q}^\alpha}$ \hfill
    ($\alpha\lesssim D$)
  \item<3-> Spin-flip operators
    ${\color<3>{red} s_{\mu\nu}} \equiv \op{\mu}{\nu}$ ~ \hfill
    ($\mu,\nu=1,2,\cdots,n$)
  \item<6-> Permutationally symmetric states $\leftrightarrow$
    occupation numbers
    \vspace{.5em}
    \begin{itemize} \setlength{\itemsep}{.5em}
    \item $\ket{m}$ with $m=\p{m_1,m_2,\cdots,m_n}$ and $\sum_j m_j=N$
      \hfill ($N$ spins)
    \item<7-> Stars and bars: degeneracy of ${ N+n-1 \choose N }$
    \end{itemize}
  \end{itemize}
\end{frame}

\begin{frame}
  \frametitle{Breaking SU($n$) symmetry}
  \begin{align*}
    H_{\t{int}} = \sum_{\substack{\t{spin pairs}\\\p{p,q}}} h_{pq} \Pi_{pq}
  \end{align*}

  \pause

  $M$-body ``perturbation:''
  \begin{align*}
    \V_M = \sum_{\substack{\t{choices of}\\M~\t{spins}~k}} w\p{k} O\p{k}
  \end{align*}
  \begin{itemize}[<+->] \setlength{\itemsep}{1em}
  \item symmetric $M$-spin operator $O$
    \vspace{.5em}
    \begin{itemize}
    \item e.g. Pauli $ZZ$ or $XY+YX$ \hfill ($n=2$, $M=2$)
    \end{itemize}
  \item<.-> $M$-index coupling tensor $w$
  \item External fields ($M=1$), two-body interactions ($M=2$)
  \item $M>2$: eigenstates of $H_{\t{int}}$
    \vspace{.5em}
    \begin{itemize}
    \item ``Multi-body spin-wave''-like states \hfill
      \uncover<+->{$\ket{mk}=\sum_je^{ikj}\sigma_+^{(j)}\ket{m}$}
    \end{itemize}
  \end{itemize}
\end{frame}

\begin{frame}
  \frametitle{Perturbation theory (1st order)}
  \begin{overlayarea}{\linewidth}{0.25\textheight}
    \begin{align*}
      H_{\t{int}}
      = \sum_{\substack{\t{spin pairs}\\\p{p,q}}} h_{pq} \Pi_{pq}
      &&
      \V_M = \sum_{\substack{\t{choices of}\\M~\t{spins}~k}} w\p{k} O\p{k}
    \end{align*}
  \end{overlayarea}
  \pause
  \begin{overlayarea}{\linewidth}{0.6\textheight}
    Effective Hamiltonian:
    \begin{align*}
      H_M^{(1)} = \P_0 \V_M \P_0 \uncover<3->{= \bar w\, \ul{O}}
    \end{align*}
    $\P_0=$ projector onto ground-state manifold

    \pause

    \begin{align*}
      \bar w \equiv \bk{w\p{k}}_{\t{choices}~k}
      &&
      \ul{O} \equiv \sum_{\t{choices}~k} O\p{k}
    \end{align*}

    \pause

    SU(2): $\ul{ZZ}\propto S_\z^2+\t{constant}$ \hfill (OAT)
  \end{overlayarea}
  \begin{overlayarea}{\linewidth}{0.1\textheight}
  \end{overlayarea}
\end{frame}

\begin{frame}
  \frametitle{Perturbation theory (2nd order)}
  \begin{overlayarea}{\linewidth}{0.25\textheight}
    \begin{align*}
      H_{\t{int}}
      = \sum_{\substack{\t{spin pairs}\\\p{p,q}}} h_{pq} \Pi_{pq}
      &&
      \V_M = \sum_{\substack{\t{choices of}\\M~\t{spins}~k}} w\p{k} O\p{k}
    \end{align*}
  \end{overlayarea}
  \pause
  \begin{overlayarea}{\linewidth}{0.6\textheight}
    Effective Hamiltonian:
    \begin{align*}
      H_M^{(2)}
      = -\sum_{\substack{\t{excitation}\\\t{energies}\\\Delta\ne0}}
      \f1\Delta
      {\color<.(5)>{red} \P_0 \V_M}
      {\color<.(4),.(6)>{red} \P_\Delta}
      {\color<.(3)>{red} \V_M \P_0}
    \end{align*}
    $\P_\Delta=$ projector onto manifold with excitation energy
    $\Delta$
    \pause
    \begin{center}
      \only<+-.(4)>{
        \begin{tikzpicture}
          \draw[ultra thick] (0,0) -- (10em,0);
          \draw (0,3.0em) -- (10em,3.0em);
          \draw (0,3.6em) -- (10em,3.6em);
          \draw (0,3.9em) -- (10em,3.9em);
          \draw (0,4.6em) -- (10em,4.6em);
          \only<.-.(1)>{
            \node (GS) at (5em,1em) {ground-state manifold};
          }
          \node (ES) at (5em,5.4em) {excited manifolds};
          \uncover<+->{
            \draw[->] (-.2em,0em) to [bend left] (-.2em,3.0em);
            \draw[->] (-.2em,0em) to [bend left] (-.2em,3.6em);
            \draw[->] (-.2em,0em) to [bend left] (-.2em,3.9em);
            \draw[->] (-.2em,0em) to [bend left] (-.2em,4.6em);
            \node (VL) at (-2em,2.3em) {\color<.>{red} $\V_M$};
          }
          \uncover<+->{
            \draw[<->] (1.5em,.2em) -- (1.5em,2.8em);
            \draw[<->] (3.5em,.2em) -- (3.5em,3.4em);
            \draw[<->] (5.5em,.2em) -- (5.5em,3.7em);
            \draw[<->] (7.5em,.2em) -- (7.5em,4.4em);
            \node (D1) at (2.5em,1.5em) {\color<.>{red} $\Delta_1$};
            \node (D1) at (4.5em,1.5em) {\color<.>{red} $\Delta_2$};
            \node (D1) at (6.5em,1.5em) {\color<.>{red} $\Delta_3$};
            \node (D1) at (8.5em,1.5em) {\color<.>{red} $\Delta_4$};
          }
          \uncover<+->{
            \draw[<-] (10.2em,0em) to [bend right] (10.2em,3.0em);
            \draw[<-] (10.2em,0em) to [bend right] (10.2em,3.6em);
            \draw[<-] (10.2em,0em) to [bend right] (10.2em,3.9em);
            \draw[<-] (10.2em,0em) to [bend right] (10.2em,4.6em);
            \node (VR) at (12em,2.3em) {\color<.>{red} $\V_M$};
          }
        \end{tikzpicture}
      }%
      \only<+->{
        \begin{tikzpicture}
          \draw[ultra thick] (0,0) -- (10em,0);
          \node (ES) at (5em,5.4em) {excited manifolds};
          \draw[->] (-.2em,0em) to [bend left] (-.2em,3.0em);
          \draw[->] (-.2em,0em) to [bend left] (-.2em,3.6em);
          \draw[->] (-.2em,0em) to [bend left] (-.2em,3.9em);
          \draw[->] (-.2em,0em) to [bend left] (-.2em,4.6em);
          \node (VL) at (-2em,2.3em) {$\V_M$};
          \draw[<-] (10.2em,0em) to [bend right] (10.2em,3.0em);
          \draw[<-] (10.2em,0em) to [bend right] (10.2em,3.6em);
          \draw[<-] (10.2em,0em) to [bend right] (10.2em,3.9em);
          \draw[<-] (10.2em,0em) to [bend right] (10.2em,4.6em);
          \node (VR) at (12em,2.3em) {$\V_M$};
          \draw[dashed] (0,2.8em) -- (10em,2.8em)
          -- (10em,4.8em) -- (0,4.8em) -- (0,2.8em);
          \node (VR) at (5em,3.8em) {\color{red} ???};
        \end{tikzpicture}
      }
    \end{center}
  \end{overlayarea}
  \begin{overlayarea}{\linewidth}{0.1\textheight}
  \end{overlayarea}
\end{frame}

\begin{frame}
  \frametitle{Constructing eigenstates of $H_{\t{int}}$}
  \begin{overlayarea}{\linewidth}{0.25\textheight}
    \begin{align*}
      H_{\t{int}}
      = \sum_{\substack{\t{spin pairs}\\\p{p,q}}} h_{pq} \Pi_{pq}
      &&
      \V_M = {\color<+(5)>{red}
        \sum_{\substack{\t{choices of}\\M~\t{spins}~k}} w\p{k} O\p{k}}
    \end{align*}
  \end{overlayarea}
  \begin{overlayarea}{\linewidth}{0.6\textheight}
    \uncover<+->{
      ``Diagnosing'' perturbations of ground states
      $\ket{\psi_\PS}$:
      \begin{align*}
        H_{\t{int}} \V_M \ket{\psi_\PS}
        \uncover<+->{
          = {\color<+>{red} E_0 \V_M} \ket{\psi_\PS}
          + {\color<+-.(2)>{red}
            \sum_{\substack{\t{choices of}\\M~\t{spins}~k}}
            \check{h}\sp{w}\p{k} O\p{k}}
          \ket{\psi_\PS}
        }
      \end{align*}
    }
    \uncover<+->{}
    \uncover<+->{
      Eigenvalue problem: find (tensor) eigenvectors $w_\Delta$ of
      $\check{h}$
      \begin{align*}
        \check{h}\sp{w_\Delta} = \Delta w_\Delta
        &&
        \uncover<+->{
          \V_M^\Delta = \sum_{\t{choices}~k} w_\Delta\p{k} O\p{k}
        }
      \end{align*}
    }
    \vspace{-2em}
    \uncover<+->{
      \begin{align*}
        H_{\t{int}} \V_M^\Delta \ket{\psi_\PS}
        = \p{E_0 + \Delta} \V_M^\Delta \ket{\psi_\PS}
      \end{align*}
    }
  \end{overlayarea}
  \begin{overlayarea}{\linewidth}{0.1\textheight}
    \begin{center}
      ~ \hfill
      \begin{tikzpicture}
        \draw[ultra thick] (0,0) -- (4em,0);
        \draw[dashed] (0,1em) -- (4em,1em)
        -- (4em,1.4em) -- (0,1.4em) -- (0,1em);
        \draw[<->] (-.2em,0em) to [out=135, in=-130]
        (-.2em,1.05em);
        \draw[<->] (-.2em,0em) to [out=135, in=-150]
        (-.2em,1.35em);
        \node (V) at (-1.3em,.6em) {$\V_M$};
      \end{tikzpicture}
      \hfill
      \uncover<.->{
        \begin{tikzpicture}
          \draw[ultra thick] (0,0) -- (4em,0);
          \draw (0em,1.2em) -- (4em,1.2em);
          \draw[<->] (-.2em,0em) to [out=135, in=-135] (-.2em,1.2em);
          \draw[<->] (2em,.2em) -- (2em,1em);
          \node (D) at (2.75em,.6em) {$\Delta$};
          \node (V) at (-1.3em,.6em) {$\V_M^\Delta$};
        \end{tikzpicture}
      }
      \hfill ~
    \end{center}
  \end{overlayarea}
\end{frame}

\begin{frame}
  \frametitle{Perturbation theory (2nd order, cont.)}
  \begin{overlayarea}{\linewidth}{0.25\textheight}
    \begin{align*}
      H_{\t{int}}
      = \sum_{\substack{\t{spin pairs}\\\p{p,q}}} h_{pq} \Pi_{pq}
      &&
      \V_M = \sum_{\substack{\t{choices of}\\M~\t{spins}~k}} w\p{k} O\p{k}
    \end{align*}
  \end{overlayarea}
  \uncover<+->{}
  \begin{overlayarea}{\linewidth}{0.6\textheight}
    Effective Hamiltonian:
    \begin{align*}
      H_M^{(2)}
      &= -\sum_{\Delta\ne0} \f1\Delta \P_0
      {\color<2>{red} \V_M \P_\Delta \V_M} \P_0
      \uncover<+->{
        = -\sum_{\Delta\ne0} \f1\Delta \P_0
        {\color<2>{red} \V_M^\Delta\V_M^\Delta} \P_0
      } \\
      \uncover<+->{
        &= -\sum_{\Delta\ne0} \f1\Delta
        \uubrace<.(2)->{\sum_{\t{choices}~k,\ell}
          w_\Delta\p{k} w_\Delta\p{\ell} \P_0 O\p{k} O\p{\ell} \P_0}
        {\t{$r$-body operators for $r\le 2M$ ($r=\abs{k\cup\ell}$)
            \uncover<.(3)>{({\color{red} !!!})}}}
      } \\[.5em]
      \uncover<+->{
        H_1^{(2)} &= -\sum_{\Delta\ne0}
        \f{\var\p{w_\Delta}}{\Delta\p{N-1}}
        \p{\ul{O}^2 - N\ul{O^2}}
        }
    \end{align*}
  \end{overlayarea}
  \begin{overlayarea}{\linewidth}{0.1\textheight}
    \begin{center}
      ~ \hfill
      \begin{tikzpicture}
        \draw[ultra thick] (0,0) -- (4em,0);
        \draw[dashed] (0,1em) -- (4em,1em)
        -- (4em,1.4em) -- (0,1.4em) -- (0,1em);
        \draw[<->] (-.2em,0em) to [out=135, in=-130]
        (-.2em,1.05em);
        \draw[<->] (-.2em,0em) to [out=135, in=-150]
        (-.2em,1.35em);
        \node (V) at (-1.3em,.6em) {$\V_M$};
      \end{tikzpicture}
      \hfill
      {
        \begin{tikzpicture}
          \draw[ultra thick] (0,0) -- (4em,0);
          \draw (0em,1.2em) -- (4em,1.2em);
          \draw[<->] (-.2em,0em) to [out=135, in=-135]
          (-.2em,1.2em);
          \draw[<->] (2em,.2em) -- (2em,1em);
          \node (D) at (2.75em,.6em) {$\Delta$};
          \node (V) at (-1.3em,.6em) {$\V_M^\Delta$};
        \end{tikzpicture}
      }
      \hfill ~
    \end{center}
  \end{overlayarea}
\end{frame}

\begin{frame}
  \frametitle{Beyond ground-state perturbation theory}
  \begin{itemize} \setlength{\itemsep}{1.5em}
  \item PT: effective dynamics in a low-energy subspace
  \item<2-> Restrict to the permutationally symmetric (Dicke) manifold
    \uncover<3->{
      \begin{align*}
        \P_0 = \sum_m \op{m}
        &&
        S_\z \to \P_0 S_\z \P_0 = \sum_m m \op{m}
      \end{align*}
    }
    \vspace{-1.5em}
  \item<4-> What if $\abs{\V_M}\sim\Delta$?
    \vspace{.5em}
    \begin{itemize}
    \item<6-> Permutation symmetry index $k$ (e.g.~spin-wave
      wavenumber)
    \end{itemize}
    \uncover<6->{
      \begin{align*}
        \tilde\P_0 = \sum_{m,k} \op{mk}
        &&
        S_\z \to \tilde\P_0 S_\z \tilde\P_0 = \sum_{m,k} m \op{mk}
      \end{align*}
    }
  \end{itemize}
  ~ \hfill
  \tikz[overlay,every node/.style={anchor=south}]{
    \only<1-4>{
      \begin{tikzpicture}
        \draw[ultra thick] (0,0) -- (5em,0);
        \draw[dashed] (0,1em) -- (5em,1em)
        -- (5em,1.4em) -- (0,1.4em) -- (0,1em);
        \draw[<->] (-.2em,0em) to [out=135, in=-130] (-.2em,1.05em);
        \draw[<->] (-.2em,0em) to [out=135, in=-150] (-.2em,1.35em);
        \node (VL) at (-1.4em,0em) {$\V_M$};
        \draw[<->] (6em,0) -- node[midway,right]
        {$\Delta$} (6em,1.2em);
        \draw (5.8em,0) -- (6.2em,0);
        \draw (5.8em,1.2em) -- (6.2em,1.2em);
        \vphantom{
          \draw[dashed] (0,2.4em) -- (5em,2.4em)
          -- (5em,2.8em) -- (0,2.8em) -- (0,2.4em);
          \draw[<->] (5.2em,1.05em) to [out=35, in=-35] (5.2em,2.45em);
          \draw[<->] (5.2em,1.35em) to [out=35, in=-35] (5.2em,2.75em);
          \node (VR) at (6.5em,1.1em) {$\V_M$};
        }
      \end{tikzpicture}
    }
    \only<5->{
      \begin{tikzpicture}
        \draw[ultra thick] (0,0) -- (5em,0);
        \draw (0,1.05em) -- (5em,1.05em);
        \draw (0,1.35em) -- (5em,1.35em);
        \draw[<->] (-.2em,0em) to [out=135, in=-130] (-.2em,1.05em);
        \draw[<->] (-.2em,0em) to [out=135, in=-150] (-.2em,1.35em);
        \node (VL) at (-1.4em,0em) {$\V_M$};
        \draw[<->] (6em,0) -- node[midway,right]
        {$\Delta$} (6em,1.2em);
        \draw (5.8em,0) -- (6.2em,0); \draw (5.8em,1.2em)
        -- (6.2em,1.2em);
        \draw[dashed] (0,2.4em) -- (5em,2.4em) -- (5em,2.8em)
        -- (0,2.8em) -- (0,2.4em);
        \draw[<->] (5.2em,1.05em) to [out=35, in=-35] (5.2em,2.45em);
        \draw[<->] (5.2em,1.35em) to [out=35, in=-35] (5.2em,2.75em);
        \node (VR) at (6.5em,1.1em) {$\V_M$};
      \end{tikzpicture}
    }
  }
  \hfill ~
\end{frame}

\begin{frame}
  \frametitle{Building operators in the multi-body excitation basis}
  \begin{itemize}[<+->] \setlength{\itemsep}{1.5em}
  \item Multi-body operator
    $\O_v=\displaystyle\sum_{\t{choices}~k} v\p{k} O\p{k}$
  \item Ground states $\ket{m}=\ket{m_1,m_2,\cdots,m_n}$
    \vspace{.5em}
    \begin{itemize}
    \item $m_j=$ occupation number of single-spin state $j$
    \end{itemize}
  \item Excited states
    $\ket{m,w} = \Q_w \ket{m} = \displaystyle\sum_{\t{choices}~k}
    w\p{k} Q\p{k} \ket{m}$
  \item Matrix elements of $\O_v$:
    $\bk{\ell,u|\O_v|m,w} \uncover<+->{ =\bk{\ell|\Q_u\O_v\Q_w|m} }$
  \item Projected operator products:
    $\P_0 \Q_u \O_v \Q_w \P_0$
    \vspace{.5em}
    \begin{itemize}[<+->]
    \item 2nd order perturbation theory: $\P_0 \V_M \V_M \P_0$
    \end{itemize}
  \end{itemize}
\end{frame}

\begin{frame}
  \frametitle{Projected operator products (example)}
  \vspace{1.5em}
  Two 2-body operators:
  \begin{multline*}
    \sum_{\substack{\t{spin pairs}\\\p{k,\ell},\p{p,q}}}
    v_{k\ell} w_{pq} O_{k\ell} Q_{pq}
    \pause \uncover<+->{
      = \sp{2\t{-body}} + \sp{3\t{-body}} + \sp{4\t{-body}}
    } \\
    \uncover<+->{
      \EQPS \diagram{two_body_2} O_{1,2} Q_{1,2}
      + \diagram{two_body_1} O_{1,2} Q_{1,3}
      + \diagram{two_body_0} O_{1,2} Q_{3,4}
    }
  \end{multline*}
  \uncover<.->{
    ($\EQPS$: equality within permutationally symmetric manifold)
  }
  \vspace{.5em}
  \begin{align*}
    \uncover<+->{
      \diagram{two_body_2}
      = \sum_{\substack{\t{spin pairs}\\\p{p,q}}} v_{pq} w_{pq}
    }
    &&
    \uncover<+->{
      \diagram{two_body_1}
      = \sum_{p,q,r~\t{distinct}} v_{pr} w_{qr}
    }
  \end{align*}
  \vspace{-1em}
  \uncover<+->{
    \begin{align*}
      \diagram{two_body_0}
      = \sum_{\substack{\t{spin pairs}~\p{k,\ell},\p{p,q}\\
          k,\ell,p,q~\t{distinct}}} v_{k\ell} w_{pq}
    \end{align*}
  }
\end{frame}

\begin{frame}
  \frametitle{Projected operator products (general)}
  \vspace{1em}
  \begin{itemize}[<+->] \setlength{\itemsep}{1em}
  \item Diagram $g:\t{powerset}\p{\ZZ_p}\to\NN_0$ \hfill ($p$
    operators)
    \vspace{.5em}
    \begin{itemize}
    \item $g_S=$ number of indices shared by tensors
      $j\in S\subset\ZZ_p$
    \end{itemize}
  \end{itemize}
  \vspace{-1.5em}
  \uncover<+->{
    \begin{adjustwidth}{-\leftmargin}{-\rightmargin}
      \begin{align*}
        \diagram{example_123}
        && \leftrightarrow &&
        \sum_{\substack{\t{spins}~a,b,c\\\t{spin pairs}~\p{c,d}\\
            a,b,c,d,e~\t{distinct}}}
        w_1\p{a,b,c} w_2\p{b,d,e} w_3\p{b,c,d,e}
      \end{align*}
    \end{adjustwidth}
  }
  \vspace{-1em}
  \uncover<+->{
    \begin{align*}
      g_{\set{1}} = 1
      &&
      g_{\set{1,2,3}} = 1
      &&
      g_{\set{1,3}} = 1
      &&
      g_{\set{2,3}} = 2
    \end{align*}
  }
  \vspace{-1em}
  \begin{itemize}[<+->] \setlength{\itemsep}{1em}
  \item Product of $p$ operators defined by $\p{w_j,O_j}$ for
    $j=1,2,\cdots,p$
  \end{itemize}
  \uncover<.->{
    \begin{adjustwidth}{-\leftmargin}{-\rightmargin}
      \begin{align*}
        \prod_{j\in\ZZ_p}
        \sp{\sum_{\substack{\t{choices of}\\M_j~\t{spins}~k_j}}
          w_j\p{k_j} O_j\p{k_j}}
        \EQPS \sum_{\t{valid diagrams}~g} w\p{g} O\p{g}
      \end{align*}
    \end{adjustwidth}
    }
\end{frame}

\begin{frame}
  \frametitle{Projected operator products (example 2)}
  Three $ZZ$-type operators: \hfill (matrix elements of one $ZZ$ op.)
  \begin{align*}
    \sum_{\substack{\t{spin pairs}\\\p{k,\ell},\p{p,q},\p{r,s}}}
    &u_{k\ell} v_{pq} w_{rs} Z_k Z_\ell Z_p Z_q Z_r Z_s \\
    \pause \uncover<+->{
      &\quad\EQPS A_0  + A_2 Z^{\otimes 2}
      + A_4 Z^{\otimes 4} + A_6 Z^{\otimes 6}
    }
  \end{align*}
  \uncover<+->{
    \vspace{-1.5em}
    \begin{align*}
      A_0 = \diagram{triple_0111}
      &&
      A_4 = \diagram{triple_01} + \diagram{triple_1}
      &&
      A_6 = \diagram{triple_0}
    \end{align*}
    \vspace{-1.5em}
    \begin{align*}
      A_2 = \diagram{triple_011} + \diagram{triple_02}
      + \diagram{triple_11} + \diagram{triple_2}
    \end{align*}
    (Unlabeled diagram $\to$ implicit sum over distinct labels
    $u,v,w$)}

  \vspace{1em}

  \uncover<+->{In practice: make a computer do it}
\end{frame}

\begin{frame}
  \frametitle{Summary}
  \begin{itemize}[<+->] \setlength{\itemsep}{1em}
  \item SU($n$) ferromagnet with $M$-body perturbation:
    \begin{align*}
      H_{\t{int}} = \sum_{\t{pairs}~\p{p,q}} h_{pq} \Pi_{pq}
      &&
      \V_M = \sum_{\t{choices}~k} w\p{k} O\p{k}
    \end{align*}
  \item Multi-body eigenvalue problem:
    \begin{align*}
      \check h\sp{w_\Delta} = \Delta w_\Delta
      &&
      \implies
      &&
      H_{\t{int}} \V_M^\Delta \ket{\psi_\PS}
      = \p{E_0+\Delta} \V_M^\Delta \ket{\psi_\PS}
    \end{align*}
  \item Perturbation theory, building eigenstates of $H_{\t{int}}$
  \item Multi-body operator products:
    \begin{align*}
      \prod_j \sp{\sum_{\t{choices}~k_j} w_j\p{k_j} O_j\p{k_j}}
      \EQPS \sum_{\t{diagrams}~g} w\p{g} O\p{g}
    \end{align*}
  \end{itemize}
\end{frame}

\begin{frame}
  \frametitle{Example: XXZ model with power-law couplings}
  \begin{overlayarea}{\linewidth}{0.25\textheight}
    \begin{adjustwidth}{-0.5\leftmargin}{-\rightmargin}
      \begin{align*}
        H_{\t{XXZ}} = - \sum_{j,k} \f1{\abs{r_j-r_k}^\alpha}
        \sp{J_\perp \p{s_\x^{(j)} s_\x^{(k)} + s_\y^{(j)} s_\y^{(k)}}
          + J_\z s_\y^{(j)} s_\z^{(k)}}
      \end{align*}
    \end{adjustwidth}
  \end{overlayarea}
  \pause
  \begin{overlayarea}{\linewidth}{0.75\textheight}
    \begin{itemize}[<+->] \setlength{\itemsep}{2em}
    \item Heisenberg point: $J_\z/J_\perp=1$
    \item First order perturbation theory at $J_\z\approx J_\perp$:
      one-axis twisting
      \begin{align*}
        H_{\t{eff}} = \chi S_\z^2
        &&
        \chi \sim
        \f{J_\z-J_\perp}{N\p{N-1}} \sum_{j,k} \f1{\abs{r_j-r_k}^\alpha}
      \end{align*}
    \item What if $J_\z \napprox J_\z$?
    \end{itemize}
  \end{overlayarea}
\end{frame}

\begin{frame}
  \frametitle{Example: XXZ model with power-law couplings (cont.)}
  \begin{overlayarea}{\linewidth}{0.25\textheight}
    \begin{adjustwidth}{-0.5\leftmargin}{-\rightmargin}
      \begin{align*}
        H_{\t{XXZ}} = - \sum_{j,k} \f1{\abs{r_j-r_k}^\alpha}
        \sp{J_\perp \p{s_\x^{(j)} s_\x^{(k)} + s_\y^{(j)} s_\y^{(k)}}
          + J_\z s_\y^{(j)} s_\z^{(k)}}
      \end{align*}
    \end{adjustwidth}
  \end{overlayarea}
  \begin{overlayarea}{\linewidth}{0.75\textheight}
    \begin{itemize} \setlength{\itemsep}{1em}
    \item $5\times 5$ (periodic) lattice with power law $\alpha=3$
      \hfill ~ \uncover<2-4>{
        $J_\z/J_\perp=$
        \makebox[2em]{%
          \only<2>{$0.5$}%
          \only<3>{$0$}%
          \only<4>{$-1$}%
        }
      }
    \item $\bk{\P_M}$: population in the $M$-body excitation manifold
    \end{itemize}
    \begin{center} \only<2>{
        \includegraphics[width=0.49\linewidth]
        {{../figures/shells/populations_L5_5_a3_z0.5}.pdf}
        \hfill
        \includegraphics[width=0.49\linewidth]
        {{../figures/shells/squeezing_L5_5_a3_z0.5}.pdf}
      } \only<3>{
        \includegraphics[width=0.49\linewidth]
        {{../figures/shells/populations_L5_5_a3_z0}.pdf}
        \hfill
        \includegraphics[width=0.49\linewidth]
        {{../figures/shells/squeezing_L5_5_a3_z0}.pdf}
      } \only<4>{
        \includegraphics[width=0.49\linewidth]
        {{../figures/shells/populations_L5_5_a3_z-1}.pdf}
        \hfill
        \includegraphics[width=0.49\linewidth]
        {{../figures/shells/squeezing_L5_5_a3_z-1}.pdf}
      } \only<5>{
        \includegraphics[width=0.49\linewidth]
        {{../figures/shells/populations_L5_5_a3}.pdf}
        \hfill
        \includegraphics[width=0.49\linewidth]
        {{../figures/shells/squeezing_L5_5_a3}.pdf}
      }
    \end{center}
  \end{overlayarea}
\end{frame}

\begin{frame}
  \frametitle{Addendum}
  \begin{itemize}[<+->] \setlength{\itemsep}{1.5em}
    \item Matrix elements of multi-local operators,
      e.g.~$\bk{\ell|Z_1Z_2|m}$
  \item Using symmetries to reduce computational costs
    \vspace{.5em}
    \begin{itemize}
    \item Translational invariance, isotropy
    \end{itemize}
  \item Multi-body eigenvalue problem
    \vspace{.5em}
    \begin{itemize}
    \item $4$-body, $7 \times 7$ lattice:
      $\sim 200,000 \stackrel{\t{TI}}{\to} \sim 4,000
      \stackrel{\t{iso}}{\to} 627$
    \end{itemize}
  \item ``Reducing'' diagrams, simplifying tensor contractions
  \item Outlook: spin squeezing, dynamical phases, SU($n$) physics
  \end{itemize}
\end{frame}

\begin{frame}[plain]
  ~ \vfill
  \centering
  \bf fin
  \vfill ~
\end{frame}
\addtocounter{framenumber}{-1}

\end{document}
