\documentclass[aspectratio=43,usenames,dvipsnames]{beamer}

%%% beamer theme
\usepackage{beamerthemelined}
\usetheme{default} % theme with outline sidebar on the right
\setbeamertemplate{headline}{} % clear header
\setbeamertemplate{navigation symbols}{} % clear navigation symbols
\setbeamertemplate{footline} % place frame number in footer
{
  \hbox{\begin{beamercolorbox}
      [wd=1\paperwidth,ht=2.25ex,dp=1ex,right]{framenumber}
      \usebeamercolor[fg]{subtitle}
      \insertframenumber{} / \inserttotalframenumber
      \hspace*{2ex}
    \end{beamercolorbox}}
}

% remove the "Figure:" prefix from figure captions
\usepackage{caption}
\captionsetup[figure]{labelformat=empty}

\usepackage{graphicx} % for figures
\graphicspath{{./figures/}} % set path for all figures

%%% symbols, notations, etc.
\usepackage{physics,braket,bm,amssymb} % physics and math
\usefonttheme[onlymath]{serif} % "regular" math font

\renewcommand{\t}{\text} % text in math mode
\newcommand{\f}[2]{\dfrac{#1}{#2}} % shorthand for fractions
\newcommand{\p}[1]{\left(#1\right)} % parenthesis
\renewcommand{\sp}[1]{\left[#1\right]} % square parenthesis
\renewcommand{\set}[1]{\left\{#1\right\}} % curly parenthesis
\newcommand{\bk}{\braket} % shorthand for braket notation

\renewcommand{\c}{\cdot}

\newcommand{\m}{\bm} % bold symbol
\renewcommand{\v}{\vec} % arrow vector

\usepackage{dsfont}
\newcommand{\1}{\mathds{1}}

\newcommand{\up}{\uparrow}
\newcommand{\dn}{\downarrow}

\renewcommand{\d}{\text{d}}
\newcommand{\x}{\text{x}}
\newcommand{\y}{\text{y}}
\newcommand{\z}{\text{z}}

\newcommand{\B}{\mathcal{B}}
\newcommand{\D}{\mathcal{D}}
\newcommand{\E}{\mathcal{E}}
\renewcommand{\H}{\mathcal{H}}
\newcommand{\I}{\mathcal{I}}
\newcommand{\M}{\mathcal{M}}
\newcommand{\N}{\mathcal{N}}
\renewcommand{\O}{\mathcal{O}}
\renewcommand{\P}{\mathcal{P}}
\newcommand{\Q}{\mathcal{Q}}
\newcommand{\R}{\mathcal{R}}
\newcommand{\T}{\mathcal{T}}
\renewcommand{\S}{\mathcal{S}}
\newcommand{\V}{\mathcal{V}}
\newcommand{\X}{\mathcal{X}}
\newcommand{\Z}{\mathcal{Z}}

\newcommand{\EE}{\mathbb{E}}
\newcommand{\RR}{\mathbb{R}}
\renewcommand{\SS}{\mathbb{S}}
\newcommand{\ZZ}{\mathbb{Z}}

\newcommand{\FS}{\text{FS}}

\DeclareMathOperator{\sign}{sign}
\let\var\relax
\DeclareMathOperator{\var}{var}

\renewcommand*{\thefootnote}{\alph{footnote}}


% custom colors
\usepackage{xcolor}
\definecolor{orange}{RGB}{255,127,14}
\definecolor{green}{RGB}{44,160,44}
\newcommand{\mblue}[1]{\textcolor{blue}{#1}}
\newcommand{\morange}[1]{\textcolor{orange}{#1}}
\newcommand{\mgreen}[1]{\textcolor{green}{#1}}
\newcommand{\mred}[1]{\textcolor{red}{#1}}

% for drawing arrows
\usepackage{tikz}
\usetikzlibrary{arrows,shapes}
\tikzstyle{every picture}+=[remember picture]
\newcommand{\tikzmark}[1]{
  \tikz[remember picture] \node[coordinate] (#1) {#1};
}

% for uncovering under/over braces
\newcommand<>{\uubrace}[2]{%
  \onslide#3 \underbrace{ \onslide<1->%
  #1%
  \onslide#3 }_{#2} \onslide<1->%
}
\newcommand<>{\uobrace}[2]{%
  \onslide#3 \overbrace{ \onslide<1->%
  #1%
  \onslide#3 }^{#2} \onslide<1->%
}

%%%%%%%%%%%%%%%%%%%%%%%%%%%%%%%%%%%%%%%%%%%%%%%%%%%%%%%%%%%%%%%%%%%%%%

\title{Multilevel spin models and SU($n$)-symmetric interactions}%
\author{Michael A. Perlin}%
\date{7 November 2019}

\begin{document}

\begin{frame}[plain]
  \titlepage
\end{frame}
\addtocounter{framenumber}{-1}

\begin{frame}
  \frametitle{Overview}
  \begin{itemize} \setlength{\itemsep}{1.5em}
  \item<+-> SU($n$)-symmetric interactions ($\sim\v S\c\v S$)
    \vspace{.5em}
    \begin{itemize} \setlength{\itemsep}{.5em}
    \item Cold atomic systems; SYK-like models, chaos/scrambling
    \item Simple form independent of $n$
    \end{itemize}
  \item<+-> Strong interactions
    \vspace{.5em}
    \begin{itemize} \setlength{\itemsep}{.5em}
    \item Gap protection, perturbations, and eigenstates
    \item Spin squeezing
    \end{itemize}
  \item<+-> Nuclear spins
    \vspace{.5em}
    \begin{itemize} \setlength{\itemsep}{.5em}
    \item Multilevel spin operators
    \item Application (Raman drive)
    \item Spin-orbit coupling, periodic drive (not covered)
    \end{itemize}
  \end{itemize}
\end{frame}

\begin{frame}
  \frametitle{SU($n$)-symmetric interactions}
  \pause
  \begin{itemize} \setlength{\itemsep}{1em}
  \item<+-> Familiar case of SU(2):
    \begin{align*}
      H_{\t{int}}
      = \sum_{p<q} \underbrace{h_{pq}}
      _{\substack{\t{coupling}\\\t{strengths}}} ~
      \underbrace{\v S_p \c\v S_q}_{\substack{\t{spin}\\\t{operators}}}
      \uncover<+->{
        \propto \sum_{p<q} h_{pq}
        \underbrace{\p{X_pX_q + Y_pY_q + Z_pZ_q}}
        _{\t{Pauli operators}}
      }
    \end{align*}
  \item<+-> Generalization to SU($n$)?
  \item<+-> Basis of operators on an $n$-level system:
    $\B_n\equiv\set{\O}$
    \vspace{.5em}
    \begin{itemize} \setlength{\itemsep}{.5em}
    \item SU(2): $\set{X, Y, Z, \1}$
    \item<+-> Linearly independent, normalized:
      $\tr\p{\O^\dag\Q}=\delta_{\O\Q}\times\t{const.}$
    \end{itemize}
    \uncover<+->{
      \begin{align*}
        H_{\t{int}} \sim \sum_{p,q} h_{pq} \sum_{\O\in\B_n} \O_p^\dag \O_q
      \end{align*}
    }
    \vspace{-2em}
  \item<+-> $H_{\t{int}}$ independent of basis choice (exercise for
    the reader)
    \vspace{.5em}
    \begin{itemize}
    \item<+-> Basis elements: $\op{\nu}{\mu}$ \hfill spin transition
      $\mu\to\nu$ \hfill ~
    \end{itemize}
  \end{itemize}
\end{frame}

\begin{frame}
  \frametitle{SU($n$)-symmetric interactions (cont.)}
  \uncover<+->{
    \begin{align*}
      H_{\t{int}} = \sum_{p<q} h_{pq}
      \sum_{\mu,\nu} \underbrace{\op{\mu}{\nu}^{(p)}}_{\O_p^\dag}
      \underbrace{\op{\nu}{\mu}^{(q)}}_{\O_q}
    \end{align*}
  }
  \uncover<+->{
    \begin{align*}
      \sum_{\mu,\nu} \op{\mu}{\nu}^{(p)} \op{\nu}{\mu}^{(q)}
      = \sum_{\mu,\nu} \op{\mu\nu}{\nu\mu}^{(p,q)}
      \uncover<+->{
        = \t{SWAP}_{pq} \equiv \Pi_{pq}
      }
    \end{align*}
  }
  \uncover<+->{
    \begin{align*}
      H_{\t{int}} = \sum_{p<q} h_{pq} \Pi_{pq}
    \end{align*}
  }
  \setlength{\leftmargini}{0em}
  \begin{itemize} \setlength{\itemsep}{1em}
  \item[]<+-> Independent of $n$!
  \item[]<+-> Special case: ferromagnetic interactions, $h_{pq}<0$
    \vspace{.5em}
    \begin{itemize} \setlength{\itemsep}{.5em}
    \item<+-> Ground states:
      $\Pi_{pq}\ket{\psi_{\t{ground}}}=\ket{\psi_{\t{ground}}}$
    \item<+-> ``Fully symmetric'' manifold $\leftrightarrow$ Dicke
      manifold in SU(2)
    \end{itemize}
  \end{itemize}
\end{frame}

\begin{frame}
  ~ \vfill
  \begin{center}
    \large \bf Assumption: fully symmetric manifold is gap-protected
  \end{center}
  \vspace{3em}
  Energy spectrum:
  \begin{center}
    \begin{tikzpicture}
      \draw[ultra thick] (0,0) -- (10em,0) node[right]
      {\bf fully symmetric states};
      \draw (0,3.0em) -- (10em,3.0em);
      \draw (0,3.1em) -- (10em,3.1em);
      \draw (0,3.2em) -- (10em,3.2em);
      \draw (0,3.3em) -- (10em,3.3em) node[right] {other states};
      \draw (0,3.4em) -- (10em,3.4em);
      \draw[<->] (5em,0.2em) -- (5em,2.8em)
      node[midway,right] {gap};
      \pause
      \uncover<+->{
        \draw (0,-3.0em) -- (10em,-3.0em);
        \draw (0,-3.1em) -- (10em,-3.1em) node[right]
        {other states};
        \draw (0,-3.2em) -- (10em,-3.2em);
        \draw (0,-3.3em) -- (10em,-3.3em);
        \draw (0,-3.4em) -- (10em,-3.4em);
        \draw[<->] (5em,-0.2em) -- (5em,-2.8em)
        node[midway,right] {gap};
      }
    \end{tikzpicture}
  \end{center}
  \vspace{1em}
  \uncover<.->{
    \begin{center}
      $\ket{\psi_{\t{initial}}}\in~\t{fully symmetric manifold}$
    \end{center}
  }
  \vfill ~
\end{frame}

\begin{frame}
  \frametitle{Perturbations and low-lying excitations}
  \begin{minipage}[t][0.2\textheight][t]{1.0\linewidth}
    \begin{align*}
      H_{\t{tot}} = H_{\t{int}} + \V\p{\vec v}
      &&
      H_{\t{int}} = \sum_{p<q} h_{pq} \Pi_{pq}
      &&
      \V\p{\vec v} = \sum_p v_p V_p
    \end{align*}
  \end{minipage}
  \pause
  \begin{minipage}[t][0.6\textheight][t]{1.0\linewidth}
    \vspace{1em}
    \begin{itemize} \setlength{\itemsep}{1.5em}
    \item<+-> $v_p$: scalar coefficient
    \item<.-> $V_p$: single-body operator on spin $p$ \vspace{.5em}
      \begin{itemize} \setlength{\itemsep}{.5em}
      \item Same operator for all spins
      \item Traceless (orthogonal to identity $\1_p$)
      \end{itemize}
    \item<+-> Example: inhomogenous magnetic field
      $\sum_p B_p s_\z^{(p)}$
    \end{itemize}
  \end{minipage}
\end{frame}

\begin{frame}
  \frametitle{Perturbations and low-lying excitations
    (1\textsuperscript{st} order)}
  \begin{minipage}[t][0.2\textheight][t]{1.0\linewidth}
    \begin{align*}
      H_{\t{tot}} = H_{\t{int}} + \V\p{\vec v}
      &&
      H_{\t{int}} = \sum_{p<q} h_{pq} \Pi_{pq}
      &&
      \V\p{\vec v} = \sum_p v_p V_p
    \end{align*}
  \end{minipage}
  \begin{minipage}[t][0.6\textheight][t]{1.0\linewidth}
    Projection onto fully symmetric manifold (FSM)
    \begin{align*}
      H_{\t{eff}}^{(1)}
      &= \P_0 \V \P_0
      = \sum_p v_p \P_0 V_p \P_0
      \pause
      \uncover<+->{
        = \sum_p v_p \P_0 V_1 \P_0
      } \\
      \uncover<+->{
        &= \sum_p v_p \P_0 \p{\f1N \sum_q V_q} \P_0
      } \\
      \uncover<+->{
        &\simeq \bar v V
        \hspace{6em} \t{(``$\simeq$'' = restriction to FSM)}
      }
    \end{align*}
    \vspace{-1em}
    \uncover<.->{
      \begin{align*}
        \bar v \equiv \f1N \sum_p v_p
        &&
        V \equiv \sum_q V_q
      \end{align*}
    }
  \end{minipage}
\end{frame}

\begin{frame}
  \frametitle{Perturbations and low-lying excitations
    (2\textsuperscript{nd} order)}
  \begin{minipage}[t][0.2\textheight][t]{1.0\linewidth}
    \begin{align*}
      H_{\t{tot}} = H_{\t{int}} + \V\p{\vec v}
      &&
      H_{\t{int}} = \sum_{p<q} h_{pq} \Pi_{pq}
      &&
      \V\p{\vec v} = \sum_p v_p V_p
    \end{align*}
  \end{minipage}
  \begin{minipage}[t][0.6\textheight][t]{1.0\linewidth}
    Coupling to excited states
    \vspace{.5em}
    \pause
    \uncover<+->{
      \begin{itemize}
      \item Projector $\P_\Delta$ onto eigenspace of $H_{\t{int}}$ w/
        excitation energy $\Delta$
      \end{itemize}
    }
    \uncover<+->{
      \begin{align*}
        H_{\t{eff}}^{(2)}
        = -\sum_{\Delta>0} \f1{\Delta} \P_0 \V \P_\Delta \V \P_0
      \end{align*}
    }
    \vspace{-1em}
    \uncover<+->{
      \vspace{-.5em}
      \begin{center}
        \begin{tikzpicture}
          \draw[ultra thick] (0,0) -- (10em,0);
          \only<1-7>{
            \draw (0,3.4em) -- (10em,3.4em);
            \draw (0,4.0em) -- (10em,4.0em);
            \draw (0,4.3em) -- (10em,4.3em);
            \draw (0,5.0em) -- (10em,5.0em);
          }
          \node (FS) at (5em,-1em) {fully symmetric manifold};
          \node (ES) at (5em,6em) {excited manifolds};
          \uncover<+->{
            \draw[->] (-.2em,0em) to [bend left] (-.2em,3.4em);
            \draw[->] (-.2em,0em) to [bend left] (-.2em,4.0em);
            \draw[->] (-.2em,0em) to [bend left] (-.2em,4.3em);
            \draw[->] (-.2em,0em) to [bend left] (-.2em,5.0em);
            \node (VL) at (-1.5em,2.5em) {$\V$};
          }
          \uncover<+->{
            \only<1-7>{
              \draw[<->] (1.5em,.2em) -- (1.5em,3.2em);
              \draw[<->] (3.5em,.2em) -- (3.5em,3.8em);
              \draw[<->] (5.5em,.2em) -- (5.5em,4.1em);
              \draw[<->] (7.5em,.2em) -- (7.5em,4.8em);
              \node (D1) at (2.5em,1.6em) {$\Delta_1$};
              \node (D1) at (4.5em,1.6em) {$\Delta_2$};
              \node (D1) at (6.5em,1.6em) {$\Delta_3$};
              \node (D1) at (8.5em,1.6em) {$\Delta_4$};
            }
          }
          \uncover<+->{
            \draw[<-] (10.2em,0em) to [bend right] (10.2em,3.4em);
            \draw[<-] (10.2em,0em) to [bend right] (10.2em,4.0em);
            \draw[<-] (10.2em,0em) to [bend right] (10.2em,4.3em);
            \draw[<-] (10.2em,0em) to [bend right] (10.2em,5.0em);
            \node (VR) at (11.5em,2.5em) {$\V$};
          }
          \uncover<+->{
            \draw[dashed] (0,3.2em) -- (10em,3.2em)
            -- (10em,5.2em) -- (0,5.2em) -- (0,3.2em);
            \node (VR) at (5em,4.2em) {???};
          }
        \end{tikzpicture}
      \end{center}
    }
  \end{minipage}
\end{frame}

\begin{frame}
  \frametitle{Perturbations and low-lying excitations
    (2\textsuperscript{nd} order, cont.)}
  \begin{minipage}[t][0.2\textheight][t]{1.0\linewidth}
    \begin{align*}
      H_{\t{tot}} = H_{\t{int}} + \V\p{\vec v}
      &&
      H_{\t{int}} = \sum_{p<q} h_{pq} \Pi_{pq}
      &&
      \V\p{\vec v} = \sum_p v_p V_p
    \end{align*}
  \end{minipage}
  \begin{minipage}[t][0.6\textheight][t]{1.0\linewidth}
    Diagnosing coupling to excited states
    \vspace{.5em}
    \begin{itemize} \setlength{\itemsep}{.5em}
    \item Fully symmetric (FS) state $\ket{\psi_\FS}$
      \begin{align*}
        \pause
        \uncover<+->{H_{\t{int}}} \V\p{\v v} \ket{\psi_\FS}
        \uncover<+->{
          = \p{E_0 + \Delta} \V\p{\v v} \ket{\psi_\FS}
          \hspace{.5em} \iff \hspace{.5em}
          \m f\p{\m h}\c\v v = \Delta \v v
        }
      \end{align*}
    \item<3-> FS interaction energy $E_0$, matrix $\m f\p{\m h}$
      determined by all $h_{pq}$
    \end{itemize}
    \vspace{.5em}~

    \begin{minipage}{0.3\linewidth}
      \begin{center}
        \begin{tikzpicture}
          \draw[ultra thick] (0,0) -- (4em,0);
          \only<1-4>{
            \draw[dashed] (0,1em) -- (4em,1em)
            -- (4em,1.4em) -- (0,1.4em) -- (0,1em);
            \draw[->] (-.2em,0em) to [out=135, in=-130]
            (-.2em,1.05em);
            \draw[->] (-.2em,0em) to [out=135, in=-150]
            (-.2em,1.35em);
            \node (V) at (-1.3em,.6em) {$\V$};
            \vphantom{
              \hphantom{
                \node (V) at (-2em,.6em) {$\V\p{\v v_\Delta}$};
                \node (EV) at (2em,-1.3em)
                {eigenvector $\v v_\Delta$};}
              }
            }
          \only<5->{
            \draw (0em,1.2em) -- (4em,1.2em);
            \draw[->] (-.2em,0em) to [out=135, in=-135]
            (-.2em,1.2em);
            \node (V) at (-2em,.6em) {$\V\p{\v v_\Delta}$};
            \draw[<->] (2em,.2em) -- (2em,1em);
            \node (D) at (2.75em,.6em) {$\Delta$};
            \node (EV) at (2em,-1.3em)
            {eigenvector $\v v_\Delta$ of $\m f\p{\m h}$};
            \vphantom{
              \draw[dashed] (0,1em) -- (4em,1em)
              -- (4em,1.4em) -- (0,1.4em) -- (0,1em);
            }
          }
        \end{tikzpicture}
      \end{center}
      \vspace{1em}
    \end{minipage}
    \begin{minipage}{0.7\linewidth}
      \uncover<+->{
        \begin{center}
          eigenvalue equation for $\v v$\,!
        \end{center}
      }
      \pause
      \vspace{-1em}
      \uncover<+->{
        \begin{align*}
          \v v &= \textstyle\sum_\Delta \v v_\Delta \\[.5em]
          \uncover<+->{
             \V\p{\v v} &= \textstyle\sum_\Delta \V\p{\v v_\Delta}
          }
        \end{align*}
      }
    \end{minipage}
  \end{minipage}
\end{frame}

\begin{frame}
  \frametitle{Perturbations and low-lying excitations
    (2\textsuperscript{nd} order, final)}
  \begin{minipage}[t][0.2\textheight][t]{1.0\linewidth}
    \begin{align*}
      H_{\t{tot}} = H_{\t{int}} + \V\p{\vec v}
      &&
      H_{\t{int}} = \sum_{p<q} h_{pq} \Pi_{pq}
      &&
      \V\p{\vec v} = \sum_p v_p V_p
    \end{align*}
  \end{minipage}
  \begin{minipage}[t][0.6\textheight][t]{1.0\linewidth}
    Coupling to excited states
    \begin{align*}
      H_{\t{eff}}^{(2)}
      &= -\sum_{\Delta>0} \f1{\Delta}
      \P_0 \V\p{\v v} \P_\Delta \V\p{\v v} \P_0 \\
      \pause
      \uncover<+->{
        &= -\sum_{\Delta>0} \f1{\Delta}
        \P_0 \V\p{\v v_\Delta} \V\p{\v v_\Delta} \P_0 \\
      }
      \uncover<+->{
        &\simeq \sp{\sum_{\Delta>0}
          \f{\var\p{\v v_\Delta}}{\Delta\p{N-1}}}
        \times \p{V^2 - N W}
      }
    \end{align*}
    \vspace{-1em}
    \uncover<.->{
      \begin{align*}
        V \equiv \sum_p V_p && W \equiv \sum_p V_p^2
      \end{align*}
    }
  \end{minipage}
\end{frame}

\begin{frame}
  \frametitle{Perturbations and low-lying excitations (summary)}
  \begin{minipage}[t][0.2\textheight][t]{1.0\linewidth}
    \begin{align*}
      H_{\t{tot}} = H_{\t{int}} + \V\p{\vec v}
      &&
      H_{\t{int}} = \sum_{p<q} h_{pq} \Pi_{pq}
      &&
      \V\p{\vec v} = \sum_p v_p V_p
    \end{align*}
  \end{minipage}
  \begin{minipage}[t][0.6\textheight][t]{1.0\linewidth}
    First order:
    \begin{align*}
      H_{\t{eff}}^{(1)} = \bar v V
    \end{align*}
    Second order:
    \begin{align*}
      H_{\t{eff}}^{(2)}
      = \sum_{\Delta>0} \f{\var\p{\v v_\Delta}}{\Delta\p{N-1}}
      \p{V^2 - N W}
    \end{align*}
    \pause

    \uncover<+->{
      SU(2): gap protection + inhomogeneous field = squeezing!
    }

    \vspace{1em}
    \uncover<+->{
      Eigenstates $\V\p{\v v_\Delta}\ket{\psi_\FS}$ of $H_{\t{int}}$
      depend {\it only} on choice of $\v v_\Delta$
    }
  \end{minipage}
\end{frame}

\begin{frame}
  \frametitle{Perturbations and low-lying excitations (two-body)}
  \begin{minipage}[t][0.2\textheight][t]{1.0\linewidth}
    \begin{align*}
      H_{\t{tot}} = H_{\t{int}} + \tilde\V\p{\m w}
      &&
      H_{\t{int}} = \sum_{p<q} h_{pq} \Pi_{pq}
    \end{align*}
  \end{minipage}
  \begin{minipage}[t][0.6\textheight][t]{1.0\linewidth}
    \begin{align*}
      \tilde\V\p{\m w} = \sum_{p<q} w_{pq} \times \f12\sp{X_p, Y_q}_+
    \end{align*}
    \vspace{-1em}
    \pause
    \uncover<+->{
      \begin{align*}
        \tilde H_{\t{eff}}^{(1)} = \f12 \bar w \times
        \p{\f12\sp{X,Y}_+ - \sum_p \f12\sp{X_p,Y_p}_+}
      \end{align*}
      \begin{align*}
        \bar w \equiv {N\choose2}^{-1} \sum_{p<q} w_{pq}
      \end{align*}
    }
    \uncover<+->{
      SU(2): squeezing at first order for $ZZ$-type interactions
    }
  \end{minipage}
\end{frame}

\begin{frame}
  ~ \vfill
  \begin{center}
    \large \bf Nuclear spins dynamics
  \end{center}
  \vfill ~
\end{frame}

\begin{frame}
  \frametitle{Nuclear spins and external fields}
  \vspace{1em}
  \begin{itemize} \setlength{\itemsep}{1.5em}
  \item<+-> Ultracold atoms with $n$ nuclear spins states
    \vspace{.5em}
    \begin{itemize}
    \item Yb ($n=6$), Cr ($n=7$), Sr ($n=10$), Er ($n=20$)
    \end{itemize}
  \item<+-> Collisional interactions independent of nuclear spin
    \vspace{.5em}
    \begin{itemize}
    \item SU($n$)-symmetric interactions
    \end{itemize}
  \item<+-> External controls: magnetic fields, polarized lasers,
    cavities
  \end{itemize}
  \vspace{.5em}
  \uncover<+->{
    \begin{align*}
      H_{\t{magnetic}} \propto \sum_\mu \mu \op{\mu}
    \end{align*}
  }
  \vspace{-1.5em}
  \uncover<+->{
    \begin{align*}
      H_{\t{laser}}
      \propto \sum_\mu \bk{I\mu;1,0|I,\mu+1} \sigma_+ + \t{h.c.}
    \end{align*}
    \vspace{-1.5em}
    \begin{align*}
      H_{\t{cavity}}
      \propto \sum_\mu \bk{I\mu;1,0|I,\mu+1} \sigma_+ a + \t{h.c.}
    \end{align*}
  }
\end{frame}

\begin{frame}
  \frametitle{Transition operators}
  \begin{itemize} \setlength{\itemsep}{1.5em}
  \item<+-> Nuclear spin $I$ with $n=2I+1$
    \begin{align*}
      T_{LM}
      \equiv \underbrace{\sqrt{\f{2L+1}{2I+1}}}
      _{\substack{\t{normalization}\\\t{factor}}}
      ~\sum_\mu~ \underbrace{\bk{I\mu;LM|I,\mu+M}}_{\t{CG coefficient}}
      \underbrace{\op{\mu+M}{\mu}}_{\t{spin change}}
    \end{align*}
  \item<+-> Absorption of a spin-$L$ particle with spin projection $M$
    \begin{align*}
      \uncover<+->{
        &H_{\t{magnetic}} \propto T_{1,0} \\
        &H_{\t{laser}} \propto T_{1,1} + \t{h.c.} \\
        &H_{\t{cavity}} \propto T_{1,1} a + \t{h.c.}
      }
    \end{align*}
  \item<+-> Orthonormal basis:
    \begin{align*}
      \tr\p{T_{LM}^\dag T_{L'M'}}=\delta_{LL'} \delta_{MM'}
    \end{align*}
  \end{itemize}
\end{frame}

\begin{frame}
  \frametitle{Drive operators}
  \vspace{1em}
  \begin{itemize} \setlength{\itemsep}{1.5em}
  \item<+-> Self-adjoint combinations of transition operators
    \begin{align*}
      \uncover<+->{
        D_{L,0} \equiv T_{L,0}
        &&
        D_{LM} \stackrel{M\ne0}{\equiv}
        \f{\eta_M}{\sqrt{2}} \p{T_{L\abs{M}}
          + \sign\p{M} T_{L\abs{M}}^\dag}
      }
    \end{align*}
    \begin{itemize}
    \item<.-> $\eta_M\in\set{\pm1}$ (choice of conventions)
    \end{itemize}
    \item<+-> Spin-operator-like
    \begin{align*}
      \p{D_{1,0}, D_{1,1}, D_{1,-1}} \sim \p{S_\z,S_\x,S_\y}
    \end{align*}
    \vspace{-2em}
    \begin{align*}
      \uncover<+->{
        H_{\t{magnetic}} \propto D_{1,0}
        &&
        H_{\t{laser}} \propto D_{1,1}
      }
    \end{align*}
  \item<+-> Coupling to a {\it classical} field of spin-$\p{L,M}$
    particles
  \end{itemize}
\end{frame}

\begin{frame}
  \frametitle{Application: Raman drive}
  Three lasers off-resonantly addressing an electronic excitation
  \vspace{.5em}
  \begin{itemize}
  \item<2-> Polarizations $m$, amplitudes $\Omega_m$, phases $\phi_m$,
    detuning $\Delta$
  \end{itemize}
  \vspace{1em}
  \uncover<2->{
    \begin{align*}
      H_{\t{Raman}}
      = \sum_{m\in\set{
          \only<1-3,5-6>{0}\only<4,7->{\mblue{0}},
          \only<1-4,6>{1}\only<5,7->{\morange{1}},
          \only<1-5>{-1}\only<6->{\mgreen{-1}}}}
      \Omega_m \p{e^{i\phi_m} T_{1,m} \otimes \sigma_+ + \t{h.c.}}
      + \only<1-2,4-6>{\Delta}\only<3,7->{\mred{\Delta}}
      \1 \otimes s_\z
    \end{align*}
  }
  \vspace{1em}
  \begin{center}
    \begin{tikzpicture}
      \draw (-5em,0em) -- (-7em,0em);
      \draw (-2em,0em) -- (-4em,0em);
      \draw (-1em,0em) -- (1em,0em);
      \draw (2em,0em) -- (4em,0em);
      \draw (5em,0em) -- (7em,0em);
      %
      \draw (-5em,3em) -- (-7em,3em);
      \draw (-2em,3em) -- (-4em,3em);
      \draw (-1em,3em) -- (1em,3em);
      \draw (2em,3em) -- (4em,3em);
      \draw (5em,3em) -- (7em,3em);
      %
      \only<1-2,4-6>{
        \draw[<->,thick] (8em,.2em) -- (8em,2.8em)
        node[midway, right] {$\Delta$};
      }
      \only<3,7->{
        \draw[<->,thick,red] (8em,.2em) -- (8em,2.8em)
        node[midway, right] {$\Delta$};
      }
      %
      \only<4,7->{
        \draw[<->,thick,blue] (-6em,.2em) -- (-6em,2.8em);
        \draw[<->,thick,blue] (-3em,.2em) -- (-3em,2.8em);
        \draw[<->,thick,blue] (0em,.2em) -- (0em,2.8em);
        \draw[<->,thick,blue] (3em,.2em) -- (3em,2.8em);
        \draw[<->,thick,blue] (6em,.2em) -- (6em,2.8em);
      }
      %
      \only<5,7->{
        \draw[<->,thick,orange] (-5.5em,.2em) -- (-3.5em,2.8em);
        \draw[<->,thick,orange] (-2.5em,.2em) -- (-.5em,2.8em);
        \draw[<->,thick,orange] (.5em,.2em) -- (2.5em,2.8em);
        \draw[<->,thick,orange] (3.5em,.2em) -- (5.5em,2.8em);
      }
      %
      \only<6->{
        \draw[<->,thick,green] (-5.5em,2.8em) -- (-3.5em,.2em);
        \draw[<->,thick,green] (-2.5em,2.8em) -- (-.5em,.2em);
        \draw[<->,thick,green] (.5em,2.8em) -- (2.5em,.2em);
        \draw[<->,thick,green] (3.5em,2.8em) -- (5.5em,.2em);
      }
    \end{tikzpicture}
  \end{center}
\end{frame}

\begin{frame}
  \frametitle{Application: Raman drive (cont.)}
  Effective Hamiltonian (2\textsuperscript{nd} order):

  \vspace{1em}
  \begin{align*}
    H_{\t{Raman,eff}} ~&\sim~
    \begin{tikzpicture}
      \draw (0em,0em) -- (1em,0em);
      \draw (0em,1em) -- (1em,1em);
      \draw (1.5em,0em) -- (2.5em,0em);
      \draw (1.5em,1em) -- (2.5em,1em);
      \draw[<->] (.5em,.1em) -- (.5em,.9em);
      \draw[<->] (2em,.1em) -- (2em,.9em);
    \end{tikzpicture}
    ~+~
    \begin{tikzpicture}
      \draw (0em,0em) -- (1em,0em);
      \draw (0em,1em) -- (1em,1em);
      \draw (1.5em,0em) -- (2.5em,0em);
      \draw (1.5em,1em) -- (2.5em,1em);
      \draw[<->] (.8em,.2em) -- (1.7em,.8em);
    \end{tikzpicture}
    ~+~
    \begin{tikzpicture}
      \draw (0em,0em) -- (1em,0em);
      \draw (0em,1em) -- (1em,1em);
      \draw (1.5em,0em) -- (2.5em,0em);
      \draw (1.5em,1em) -- (2.5em,1em);
      \draw[<->] (.8em,.8em) -- (1.7em,.2em);
    \end{tikzpicture}
    \\[1em]
     ~&\quad+~
    \begin{tikzpicture}
      \draw (0em,0em) -- (1em,0em);
      \draw (0em,1em) -- (1em,1em);
      \draw (1.5em,0em) -- (2.5em,0em);
      \draw (1.5em,1em) -- (2.5em,1em);
      \draw[->] (.5em,.1em) -- (.5em,.8em);
      \draw[->] (.8em,.8em) -- (1.7em,.2em);
    \end{tikzpicture}
    ~+~
    \begin{tikzpicture}
      \draw (0em,0em) -- (1em,0em);
      \draw (0em,1em) -- (1em,1em);
      \draw (1.5em,0em) -- (2.5em,0em);
      \draw (1.5em,1em) -- (2.5em,1em);
      \draw[->] (.8em,.2em) -- (1.7em,.8em);
      \draw[->] (2em,.8em) -- (2em,.1em);
    \end{tikzpicture}
    ~+~
    \begin{tikzpicture}
      \draw (0em,0em) -- (1em,0em);
      \draw (0em,1em) -- (1em,1em);
      \draw (1.5em,0em) -- (2.5em,0em);
      \draw (1.5em,1em) -- (2.5em,1em);
      \draw[<-] (.5em,.1em) -- (.5em,.8em);
      \draw[<-] (.8em,.8em) -- (1.7em,.2em);
    \end{tikzpicture}
    ~+~
    \begin{tikzpicture}
      \draw (0em,0em) -- (1em,0em);
      \draw (0em,1em) -- (1em,1em);
      \draw (1.5em,0em) -- (2.5em,0em);
      \draw (1.5em,1em) -- (2.5em,1em);
      \draw[<-] (.8em,.2em) -- (1.7em,.8em);
      \draw[<-] (2em,.8em) -- (2em,.1em);
    \end{tikzpicture}
    \\[1em]
    ~&\quad+~
    \begin{tikzpicture}
      \draw (0em,0em) -- (1em,0em);
      \draw (0em,1em) -- (1em,1em);
      \draw (1.5em,0em) -- (2.5em,0em);
      \draw (1.5em,1em) -- (2.5em,1em);
      \draw (3em,0em) -- (4em,0em);
      \draw (3em,1em) -- (4em,1em);
      \draw[->] (.8em,.2em) -- (1.7em,.8em);
      \draw[->] (2.2em,.8em) -- (3.3em,.2em);
    \end{tikzpicture}
    ~+~
    \begin{tikzpicture}
      \draw (0em,0em) -- (1em,0em);
      \draw (0em,1em) -- (1em,1em);
      \draw (1.5em,0em) -- (2.5em,0em);
      \draw (1.5em,1em) -- (2.5em,1em);
      \draw (3em,0em) -- (4em,0em);
      \draw (3em,1em) -- (4em,1em);
      \draw[<-] (.8em,.2em) -- (1.7em,.8em);
      \draw[<-] (2.2em,.8em) -- (3.3em,.2em);
    \end{tikzpicture}
  \end{align*}
  \pause
  \uncover<+->{
    \begin{align*}
      H_{\t{Raman,eff}}
      = \sum_{\substack{L\le 2\\\abs{M}\le L}}
      \Omega_{LM}^{(\phi)} D_{LM}
    \end{align*}
  }
\end{frame}

\begin{frame}
  \frametitle{Application: Raman drive (final)}
  \vspace{.5em}
  Phase-matched lasers: $\phi_m=m\phi$
  \vspace{.5em}
  \begin{itemize}
  \item<+-> Spin transition $\mu\to\mu+m$ imprints phase $m\phi$
    \begin{align*}
      H_{\t{Raman}}
      \simeq \sum_{\substack{L\le 2\\0\le M\le L}} \Omega_{LM} D_{LM}
    \end{align*}
  \item<+-> Amplitude matching:
    $\Omega_0 = \Omega_1 = \Omega_{-1}$
    \begin{align*}
      H_{\t{Raman}}
      \simeq \Omega_{1,1} D_{1,1} + \Omega_{2,2} D_{2,2}
    \end{align*}
  \item<+-> Amplitude matching:
    $\Omega_0 = \Omega_1 = -\Omega_{-1}$
    \begin{align*}
      H_{\t{Raman}}
      \simeq \Omega_{2,1} D_{2,1} + \Omega_{2,2} D_{2,2}
    \end{align*}
    \item<+-> Amplitude matching:
    $\Omega_0 = 0$ and $\abs{\Omega_1} = \abs{\Omega_{-1}}$
    \begin{align*}
      H_{\t{Raman}}
      \simeq \Omega_{2,0} D_{2,0} + \Omega_{2,2} D_{2,2}
    \end{align*}
  \end{itemize}
\end{frame}

\begin{frame}
  \frametitle{Addendum}
  \begin{itemize} \setlength{\itemsep}{1em}
  \item Synthetic spin-2 gauge fields
  \item Multilevel spin-orbit coupling
    \vspace{.5em}
    \begin{itemize}
    \item Weak-SOC limit
    \end{itemize}
  \item Periodic drives and effective interactions
  \item Collective dynamics
    \vspace{.5em}
    \begin{itemize}
    \item Collective driving fields (drive operators)
    \item Cavity QED (transition operators)
    \item Visualizing collective states?
    \end{itemize}
  \item Metrological applications?  Multilevel spin squeezing?
  \item SYK-like models, chaos, and scrambling?
  \end{itemize}
\end{frame}

\end{document}
