\documentclass[nofootinbib,notitlepage,11pt]{revtex4-2}

%%% linking references
\usepackage{hyperref}
\hypersetup{
  breaklinks=true,
  colorlinks=true,
  linkcolor=blue,
  filecolor=magenta,
  urlcolor=cyan,
}

%%% header / footer
\usepackage{fancyhdr} % easier header and footer management
\pagestyle{fancy} % page formatting style
\fancyhf{} % clear all header and footer text
\renewcommand{\headrulewidth}{0pt} % remove horizontal line in header
\usepackage{lastpage} % for referencing last page
\cfoot{\thepage~of \pageref{LastPage}} % "x of y" page labeling


%%% symbols, notations, etc.
\usepackage{physics,braket,bm,amssymb} % physics and math
\renewcommand{\t}{\text} % text in math mode
\newcommand{\f}[2]{\dfrac{#1}{#2}} % shorthand for fractions
\newcommand{\p}[1]{\left(#1\right)} % parenthesis
\renewcommand{\sp}[1]{\left[#1\right]} % square parenthesis
\renewcommand{\set}[1]{\left\{#1\right\}} % curly parenthesis
\newcommand{\bk}{\Braket} % shorthand for braket notation
\renewcommand{\v}{\bm} % bold vectors
\newcommand{\uv}[1]{\bm{\hat{#1}}} % unit vectors
\newcommand{\av}{\vec} % arrow vectors
\renewcommand{\d}{\text{d}} % for infinitesimals
\renewcommand{\c}{\cdot} % inner product

\usepackage{dsfont} % for identity operator
\newcommand{\1}{\mathds{1}}

\newcommand{\up}{\uparrow}
\newcommand{\dn}{\downarrow}

\newcommand{\x}{\text{x}}
\newcommand{\y}{\text{y}}
\newcommand{\z}{\text{z}}

\newcommand{\B}{\mathcal{B}}
\newcommand{\D}{\mathcal{D}}
\newcommand{\E}{\mathcal{E}}
\renewcommand{\H}{\mathcal{H}}
\newcommand{\I}{\mathcal{I}}
\newcommand{\M}{\mathcal{M}}
\newcommand{\N}{\mathcal{N}}
\renewcommand{\O}{\mathcal{O}}
\renewcommand{\P}{\mathcal{P}}
\newcommand{\Q}{\mathcal{Q}}
\newcommand{\R}{\mathcal{R}}
\newcommand{\T}{\mathcal{T}}
\renewcommand{\S}{\mathcal{S}}
\newcommand{\X}{\mathcal{X}}
\newcommand{\Z}{\mathcal{Z}}

\newcommand{\EE}{\mathbb{E}}
\renewcommand{\SS}{\mathbb{S}}
\newcommand{\ZZ}{\mathbb{Z}}

\newcommand{\FS}{\text{FS}}

\DeclareMathOperator{\sign}{sign}
\DeclareMathOperator{\cov}{cov}
\let\var\relax
\DeclareMathOperator{\var}{var}

\def\obra#1{\mathinner{({#1}|}}
\def\oket#1{\mathinner{|{#1})}}
\def\obk#1{\mathinner{({#1})}}
\def\oop#1#2{\oket{#1}\!\obra{#2}}

\usepackage[inline]{enumitem} % in-line lists and \setlist{} (below)
\setlist[enumerate,1]{label={(\roman*)}} % default in-line numbering
\setlist{nolistsep} % more compact spacing between environments

%%% text markup
\usepackage{color} % text color
\newcommand{\red}[1]{{\color{red} #1}}

%%%%%%%%%%%%%%%%%%%%%%%%%%%%%%%%%%%%%%%%%%%%%%%%%%%%%%%%%%%%%%%%%%%%%%
\begin{document}
\thispagestyle{fancy}

\title{Perturbation theory with collective states}%
\author{Michael A. Perlin}%
\date{\today}

\maketitle

We consider an array of $N$ multilevel spins with SU($n$)-symmetric
interactions that can be written in the form
\begin{align}
  H_{\t{int}} = -u \H,
  &&
  \H \equiv \sum_{p<q} \Pi_{pq},
  &&
  \Pi_{pq} \equiv \sum_{\mu,\nu} S_{\mu\nu}^{(p)} S_{\nu\mu}^{(q)},
\end{align}
where $S_{\mu\nu}^{(p)}\equiv\op{\mu}{\nu}_p$ flips the state of spin
$p$ to $\ket\mu$ from $\ket\nu$.  The ground-state manifold $\M_0$ of
the interaction Hamiltonian $H_{\t{int}}$ consists of fully symmetric
states with interaction energy $-{N\choose2}u$; this manifold is
gapped by an energy $Nu$ from the first excited manifold of spin-wave
states.

We wish to determine the effective dynamics induced on the
ground-state manifold $\M_0$ by weak perturbations of the form
\begin{align}
  \Z_1 \equiv \sum_p c_p Z_p,
  &&
  \Z_2 \equiv \f12 \sum_{p\ne q} c_{pq} Z_p Z_q,
\end{align}
where $Z_p$ is a single-body operator on spin $p$, and the
coefficients $c_{pq}$ satisfy
\begin{align}
  c_p = \sum_q c_{pq},
  &&
  c_{pq} = c_{qp},
  &&
  c_{pp} = 0.
\end{align}
The effective Hamiltonian $H_W$ induced on the ground-state manifold
$\M_0$ by a perturbation $\Z_W$ through second order in perturbation
theory is given by\cite{bravyi2011schrieffer, perlin2019effective}
\begin{align}
  H_W = H_W^{(1)} + H_W^{(2)},
  &&
  H_W^{(1)} = \P_0 \Z_W \P_0,
  &&
  H_W^{(2)} = - \P_0 \Z_W \E \Z_W \P_0,
  &&
  \E \equiv \sum_{k>0} \f{\P_k}{\Delta_k},
\end{align}
where $\P_k$ is a projector onto the $k$-th excited eigenspace of the
interaction Hamiltonian $H_{\t{int}}$, with interaction energy
$\Delta_k$ above that of fully symmetric manifold $\M_0$.  In order to
calculate $H_W$, we will make heavy use of
\begin{enumerate*}
\item the permutational symmetry of the ground-state manifold, and
\item the fact that sandwiching any operator $\O$ between the
  projectors $\P_0$ and $\P_k$ (i.e.~with $k>0$) annihilates any part
  of $\O$ that is not strictly off-diagonal in the subspaces preserved
  by $\P_0$ and $\P_k$.
\end{enumerate*}

%%%%%%%%%%%%%%%%%%%%%%%%%%%%%%%%%%%%%%%%%%%%%%%%%%%%%%%%%%%%%%%%%%%%%%
\section{Single-body perturbation}

We first compute the effective Hamiltonian induced by a sum over
single-body terms, $\Z_1$.  We therefore expand
\begin{align}
  H_1^{(1)} = \sum_p c_p \P_0 Z_p \P_0,
\end{align}
where the permutational symmetry of the ground-state manifold implies
that
\begin{align}
  H_1^{(1)} =  \sum_p c_p \P_0 Z_0 \P_0 = c \P_0 Z \P_0,
  &&
  c \equiv \f1N \sum_p c_p,
  &&
  Z = \sum_p Z_p.
\end{align}
In order to compute $\H_1^{(1)}$, we first expand, for an arbitrary
fully symmetric state $\ket\psi\in\M_0$
\begin{align}
  \H \Z_1 \ket\psi
  = \sum_{\substack{k\\p<q}} c_k \Pi_{pq} Z_k \ket\psi
  = \sum_{\substack{p<q\\k\notin\set{p,q}}} c_k \Pi_{pq} Z_k \ket\psi
  + \sum_{\substack{p<q\\k\in\set{p,q}}} c_k \Pi_{pq} Z_k \ket\psi.
\end{align}
If $k\notin\set{p,q}$, then the permutation operator $\Pi_{pq}$
commutes with $Z_k$ and annihilates on the fully symmetric state
$\ket\psi$, so
\begin{align}
  \sum_{\substack{p<q\\k\notin\set{p,q}}}
  c_k \Pi_{pq} Z_k \ket\psi
  = \sum_{\substack{k\\p<q\\p,q\ne k}}
  c_k Z_k \ket\psi
  = {N-1 \choose 2} \Z_1 \ket\psi.
\end{align}
In the case of $k\in\set{p,q}$, meanwhile,
\begin{align}
  \sum_{\substack{p<q\\k\in\set{p,q}}} c_k \Pi_{pq} Z_k \ket\psi
  = \sum_{k<q} c_k Z_q \ket\psi
  + \sum_{p<k} c_k Z_p \ket\psi
  = \sum_{k\ne q} c_k Z_q \ket\psi,
\end{align}
where
\begin{align}
  \sum_{k\ne q} c_k = \sum_k c_k - c_q = N c - c_q,
\end{align}
so
\begin{align}
  \sum_{\substack{p<q\\k\in\set{p,q}}} c_k \Pi_{pq} Z_k \ket\psi
  = - \sum_q c_q Z_q \ket\psi + N c \sum_q Z_q \ket\psi
  = - \Z_1 \ket\psi + c N Z \ket\psi.
\end{align}
Altogether, we have found that
\begin{align}
  \H \Z_1 \ket\psi
  =  E_1 \Z_1 \ket\psi + c N Z \ket\psi,
  &&
  E_1 \equiv {N-1 \choose 2} - 1
\end{align}
where $E_1$ is an eigenvalue of $\H$ corresponding to the manifold
$\M_1$ of spin-wave states.  We now make the ansatz
\begin{align}
  \Z_1 \ket\psi = \ket\phi = \phi_1 \ket{1} + \phi_0 \ket{0},
  &&
  \ket{0} = c N Z \ket\psi,
\end{align}
where $\ket{k}\in\M_k$ are eigenvectors of $\H$ with eigenvalues
$E_k$.  We then have that
\begin{align}
  \H \ket\phi
  = E_1 \ket\phi + \ket{0}
  = E_1 \phi_1 \ket{1} + \p{E_1 \phi_0 + 1} \ket{0}
  = E_1 \phi_1 \ket{1} + E_0 \phi_0 \ket{0},
\end{align}
which implies that $\phi_0=\p{E_0-E_1}^{-1}$, and in turn
\begin{align}
  \phi_1 \ket{1}
  = \ket\phi - \phi_0 \ket{0}
  = \Z_1 \ket\psi - \f{c N Z}{E_0-E_1} \ket\psi.
\end{align}
Using the fact that $E_0-E_1=N$, we thus discover that the vector
$\tilde\Z_1 \ket\psi$, with
\begin{align}
  \tilde\Z_1 \equiv \Z_1 - c Z = \sum_k \p{c_k-c} Z_k,
\end{align}
is an eigenvector of $\H$ with eigenvalue $E_1$.  Equivalently,
$\tilde\Z_1\ket\psi$ is an eigenvector of the interaction Hamiltonian
$H_{\t{int}}$ with energy $Nu$ above the ground-state energy, which
implies that
\begin{align}
  \E \Z_1 \P_0 = \E \tilde\Z_1 \P_0 = \f1{Nu} \tilde\Z_1 \P_0,
\end{align}
where we may freely take $\Z_1\to\tilde\Z_1$ above because the
difference $\Z_1-\tilde\Z_1$ is a permutationally symmetric operator
that is annihilated between the projectors in $\E$ and $\P_0$.  The
second-order effective Hamiltonian induced by a single-body
perturbation is thus
\begin{align}
  H_1^{(2)} = -\P_0 \Z_1 \E \Z_1 \P_0
  = -\P_0 \tilde\Z_1 \E \tilde\Z_1 \P_0
  = - \f1{Nu} \P_0 \tilde\Z_1 \tilde\Z_1 \P_0.
\end{align}
We now need only to simplify the product
\begin{align}
  \P_0 \tilde\Z_1 \tilde\Z_1 \P_0
  = \sum_{p,q} \p{c_p-c} \p{c_q-c} \P_0 Z_p Z_q \P_0.
\end{align}
This product is straightforward to simplify by using the permutational
symmetry of the ground-state manifold.  As an example, we consider
single-body operators $Z_p$ square to the identity, in which case the $p=q$ terms above are
\begin{align}
  \sum_p \p{c_p-c}^2 \P_0 Z_p^2 \P_0
  = N \var_p c_p \P_0,
  &&
  \var_p c_p \equiv \f1N \sum_p \p{c_p-c}^2.
\end{align}
Similarly, the $p\ne q$ terms are
\begin{align}
  \sum_{p\ne q} \p{c_p-c} \p{c_q-c} \P_0 Z_p Z_q \P_0
  \sum_p \p{c_p-c} \sum_{q\ne p} \p{c_q-c} \P_0 Z_0 Z_1 \P_0,
\end{align}
where
\begin{align}
  \sum_{q\ne p} \p{c_q-c} = \sum_q \p{c_q-c} - \p{c_p-c} = - \p{c_p-c},
\end{align}
so
\begin{align}
  \sum_p \p{c_p-c} \sum_{q\ne p} \p{c_q-c} = - N \var_p c_p,
\end{align}
and
\begin{align}
  \P_0 Z Z \P_0
  = \sum_p \P_0 Z_p^2 \P_0
  + \sum_{p\ne q} \P_0 Z_p Z_q \P_0
  = N \P_0 + N \p{N-1} \P_0 Z_0 Z_1 \P_0,
\end{align}
so
\begin{align}
  \sum_{p\ne q} \p{c_p-c} \p{c_q-c} \P_0 Z_p Z_q \P_0
  = - \f{\var_p c_p}{N-1}\p{\P_0 Z Z \P_0 - N \P_0}.
\end{align}
Altogether, if the single-body operators $Z_p$ square to the identity, then
\begin{align}
  \P_0 \tilde\Z_1 \tilde\Z_1 \P_0
  = - \f{\var_p c_p}{N-1} \p{\P_0 Z Z \P_0 - N^2 \P_0},
\end{align}
and in turn
\begin{align}
  H_1^{(2)} = \f{u}{N\p{N-1}} \var_p c_p \p{\P_0 Z Z \P_0 - N^2 \P_0}
  \simeq \f{u}{N\p{N-1}} \var_p c_p ZZ,
\end{align}
where $\simeq$ denotes equality up to
\begin{enumerate*}
\item a restrict to the ground-state manifold $\M_0$, and
\item overall constants that have no physical consequence.
\end{enumerate*}

%%%%%%%%%%%%%%%%%%%%%%%%%%%%%%%%%%%%%%%%%%%%%%%%%%%%%%%%%%%%%%%%%%%%%%
\section{Two-body perturbation}

The first-order effective Hamiltonian $H_2^{(1)}$ induced by the
two-body perturbation $\Z_2$ is straightforward to calculate by making
use of the permutational symmetry of the ground-state manifold.  We
therefore jump straight into the calculation of the second-order
effective Hamiltonian $H_2^{(2)}$.  As before, we pick an arbitrary
fully symmetric state $\ket\psi\in\M_0$ and expand
\begin{align}
  \H \Z_2 \ket\psi
  = \f12 \sum_{\substack{k\ne\ell\\p<q}}
  c_{k\ell} \Pi_{pq} Z_k Z_\ell \ket\psi,
\end{align}
where the permutation operator has no effect in terms with
$p,q\notin\set{k,\ell}$ or $\set{p,q}=\set{k,\ell}$, so
\begin{align}
  \H \Z_2 \ket\psi
  = \sp{{N-2 \choose 2} + 1} \Z_2 \ket\psi
  + \f12 \sum_{\substack{k,\ell\\k\ne q\ne\ell}}
  c_{k\ell} Z_q Z_\ell \ket\psi
  + \f12 \sum_{\substack{k,\ell\\\ell\ne q\ne k}}
  c_{k\ell} Z_k Z_q \ket\psi.
\end{align}
We can then simplify
\begin{align}
  \sum_{k\ne q} c_{k\ell}
  = \sum_k c_{k\ell} - c_{q\ell}
  = c_\ell - c_{q\ell},
\end{align}
which implies that
\begin{align}
  \sum_{\substack{k,\ell\\k=p\ne q\ne\ell}}
  c_{k\ell} Z_q Z_\ell \ket\psi
  = - \sum_{q\ne\ell} c_{q\ell} Z_q Z_\ell
  + \sum_{q\ne\ell} c_\ell Z_q Z_\ell
  = - \Z_2 + Z \Z_1 - \sum_\ell c_\ell Z_\ell^2,
\end{align}
and similarly
\begin{align}
  \sum_{\substack{k,\ell\\\ell\ne q\ne k}} c_{k\ell} Z_k Z_q \ket\psi
  = - \Z_2 + \Z_1 Z - \sum_k c_k Z_k^2.
\end{align}
We thus have that
\begin{align}
  \H \Z_2 \ket\psi = \sp{{ N-2\choose 2 } - 1} \Z_2 \ket\psi
  + \f12 \sp{Z,\Z_1}_+ \ket\psi - \sum_\ell c_\ell Z_\ell^2,
\end{align}
where $\sp{X,Y}_+ \equiv X Y + Y X$ is an anti-commutator.  Similarly
to the single-body case, we now make the ansatz
\begin{align}
  \Z_2 \ket\psi = \ket\phi
  = \phi_2 \ket{2} + \phi_1 \ket{1} + \phi_0 \ket{0},
\end{align}
where
\begin{align}
  \ket{1} = \f12 \sp{Z,\Z_1}_+ \ket\psi,
  &&
  \ket{0} = - \sum_\ell c_\ell Z_\ell^2 \ket\psi,
\end{align}
and $\ket{k}\in\M_k$ is an eigenvector of the $k$-th excited manifold
of the interaction Hamiltonian $H_{\t{int}}=-u\H$, with corresponding
eigenvalue $E_k$ with respect to $\H$.  We then have that
\begin{align}
  \H \ket\phi
  = E_2 \phi_2 \ket{2}
  + \p{E_2\phi_1 + 1} \ket{1}
  + \p{E_2\phi_0 + 1}\ket{0}
  = E_2 \phi_2 \ket{2} + E_1 \phi_1 \ket{2} + E_0 \phi_0 \ket{0},
\end{align}
which implies
\begin{align}
  \phi_1 = \f1{E_1-E_2} = \f1{N-2},
  &&
  \phi_0 = \f1{E_0-E_2} = \f1{2\p{N-1}},
\end{align}
and in turn
\begin{align}
  \phi_2 \ket{2}
  = \ket\phi - \phi_1 \ket{1} - \phi_0 \ket{0}.
\end{align}
We thus find that the vector $\tilde\Z_2\ket\psi$, with
\begin{align}
  \tilde\Z_2 \equiv \Z_2 - \f1{N-2} \times \f12 \sp{Z,\Z_1}_+
  + \f1{2\p{N-1}} \sum_\ell c_\ell Z_\ell^2,
\end{align}
is an eigenvector of $\H$ with eigenvalue $E_2={N-2 \choose 2}-1$.  If
the single-body operators square to the identity, then
\begin{align}
  \tilde\Z_2
  \equiv \Z_2 - \f1{N-2} \times \f12 \sp{Z,\Z_1}_+
  + \f{c}{2\p{N-1}},
  &&
  c = \f1{N^2} \sum_{p,q} c_{pq}.
\end{align}

\bibliography{multilevel_spin_notes.bib}

\end{document}
