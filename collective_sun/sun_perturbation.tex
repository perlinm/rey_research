\documentclass[nofootinbib,notitlepage,11pt]{revtex4-2}

%%% linking references
\usepackage{hyperref}
\hypersetup{
  breaklinks=true,
  colorlinks=true,
  linkcolor=blue,
  filecolor=magenta,
  urlcolor=cyan,
}

%%% header / footer
\usepackage{fancyhdr} % easier header and footer management
\pagestyle{fancy} % page formatting style
\fancyhf{} % clear all header and footer text
\renewcommand{\headrulewidth}{0pt} % remove horizontal line in header
\usepackage{lastpage} % for referencing last page
\cfoot{\thepage~of \pageref{LastPage}} % "x of y" page labeling


%%% symbols, notations, etc.
\usepackage{physics,braket,bm,amssymb} % physics and math
\renewcommand{\t}{\text} % text in math mode
\newcommand{\f}[2]{\dfrac{#1}{#2}} % shorthand for fractions
\newcommand{\p}[1]{\left(#1\right)} % parenthesis
\renewcommand{\sp}[1]{\left[#1\right]} % square parenthesis
\renewcommand{\set}[1]{\left\{#1\right\}} % curly parenthesis
\newcommand{\bk}{\Braket} % shorthand for braket notation

\renewcommand{\c}{\cdot} % inner product
\renewcommand{\oc}{\circ} % element-wise product

\newcommand{\m}{\bm} % bold symbol
\renewcommand{\v}{\vec} % arrow vector

\usepackage{dsfont} % for identity operator
\newcommand{\1}{\mathds{1}}

\newcommand{\up}{\uparrow}
\newcommand{\dn}{\downarrow}

\renewcommand{\d}{\text{d}}
\newcommand{\x}{\text{x}}
\newcommand{\y}{\text{y}}
\newcommand{\z}{\text{z}}

\newcommand{\e}{\varepsilon}

\newcommand{\B}{\mathcal{B}}
\newcommand{\D}{\mathcal{D}}
\newcommand{\E}{\mathcal{E}}
\renewcommand{\H}{\mathcal{H}}
\newcommand{\I}{\mathcal{I}}
\newcommand{\M}{\mathcal{M}}
\newcommand{\N}{\mathcal{N}}
\renewcommand{\O}{\mathcal{O}}
\renewcommand{\P}{\mathcal{P}}
\newcommand{\Q}{\mathcal{Q}}
\newcommand{\R}{\mathcal{R}}
\newcommand{\T}{\mathcal{T}}
\renewcommand{\S}{\mathcal{S}}
\newcommand{\V}{\mathcal{V}}
\newcommand{\X}{\mathcal{X}}
\newcommand{\Z}{\mathcal{Z}}

\newcommand{\EE}{\mathbb{E}}
\newcommand{\RR}{\mathbb{R}}
\renewcommand{\SS}{\mathbb{S}}
\newcommand{\ZZ}{\mathbb{Z}}

\newcommand{\FS}{\text{FS}}

\DeclareMathOperator{\sign}{sign}
\DeclareMathOperator{\cov}{cov}
\let\var\relax
\DeclareMathOperator{\var}{var}
\DeclareMathOperator{\diag}{diag}

\def\obra#1{\mathinner{({#1}|}}
\def\oket#1{\mathinner{|{#1})}}
\def\obk#1{\mathinner{({#1})}}
\def\oop#1#2{\oket{#1}\!\obra{#2}}

\usepackage[inline]{enumitem} % in-line lists and \setlist{} (below)
\setlist[enumerate,1]{label={(\roman*)}} % default in-line numbering
\setlist{nolistsep} % more compact spacing between environments

%%% text markup
\usepackage{color} % text color
\newcommand{\red}[1]{{\color{red} #1}}

%%%%%%%%%%%%%%%%%%%%%%%%%%%%%%%%%%%%%%%%%%%%%%%%%%%%%%%%%%%%%%%%%%%%%%
\begin{document}

\title{Perturbing SU($n$)-symmetric interactions}%
\author{Michael A. Perlin}%
\date{\today}

\maketitle

\tableofcontents

\section{Introduction}

We consider an array of $N$ multilevel spins with non-uniform
SU($n$)-symmetric interactions that can be written in the form
\begin{align}
  H_0 = \sum_{p<q} h_{pq} \Pi_{pq},
  &&
  \Pi_{pq} \equiv \sum_{\mu,\nu} S_{\mu\nu}^{(p)} S_{\nu\mu}^{(q)},
\end{align}
where $S_{\mu\nu}^{(p)}\equiv\op{\mu}{\nu}_p$ flips the state of spin
$p$ to $\ket\mu$ from $\ket\nu$; the operator $\Pi_{pq}$ permutes
spins $p$ and $q$; and the coefficients $h_{pq}$ satisfy
\begin{align}
  h_{pp} = 0
  &&
  h_{pq} \le 0,
  &&
  h_{pq} = h_{qp}.
\end{align}
The ground-state manifold $\M_0$ of the interaction Hamiltonian $H_0$
consists of fully symmetric states that are simultaneous $+1$
eigenstates of all permutation operators $\Pi_{pq}$.  We will assume
throughout these notes that the interaction Hamiltonian $H_0$ is
gapped, such that there is a finite energy difference between $\M_0$
and its orthogonal complement.  The energy of fully symmetric states
$\ket\psi\in\M_0$ is
\begin{align}
  E_0 \equiv \sum_{p<q} h_{pq} = \f12 \sum_{p,q} h_{pq}
  = \f12 \sum_p h_p = \f12 N h,
\end{align}
where we define
\begin{align}
  h_p \equiv \sum_q h_{pq},
  &&
  h \equiv \EE_p\sp{h_p} = \f1N \sum_{p,q} h_{pq},
  &&
  \EE_p\sp{X_p} \equiv \f1N \sum_p X_p.
\end{align}
We wish to determine the effective dynamics induced on the fully
symmetric manifold $\M_0$ by weak single- and two-body perturbations
of the form
\begin{align}
  \V_1 \equiv \sum_p v_p V_p,
  &&
  \V_2 \equiv \sum_{p<q} w_{pq} X_p Y_q
  = \f12 \sum_{p,q} w_{pq} X_p Y_q,
  \label{eq:perturbations}
\end{align}
where $V_p,X_p,Y_p$ are trace-zero operators on spin $p$, and we
enforce $w_{pq}=w_{qp}$ with $w_{pp}=0$.  The effective Hamiltonian
$H_M$ induced on the ground-state manifold $\M_0$ by an $M$-body
perturbation $\V_M$ through second order in perturbation theory is
given by\cite{bravyi2011schrieffer, perlin2019effective}
\begin{align}
  H_M = H_M^{(1)} + H_M^{(2)},
  &&
  H_M^{(1)} = \P_0 \V_M \P_0,
  &&
  H_M^{(2)} = - \P_0 \V_M \E \V_M \P_0,
  &&
  \E \equiv \sum_{\Delta>0} \f{\P_\Delta}{\Delta},
\end{align}
where $\P_\Delta$ is a projector onto the eigenspace of the
interaction Hamiltonian $H_0$, with interaction energy $\Delta$
above that of fully symmetric manifold $\M_0$.

%%%%%%%%%%%%%%%%%%%%%%%%%%%%%%%%%%%%%%%%%%%%%%%%%%%%%%%%%%%%%%%%%%%%%%
\section{Single-body perturbations}

To compute the effective Hamiltonian induced by the single-body
perturbation $\V_1$, we expand
\begin{align}
  H_1^{(1)} = \sum_p v_p \P_0 V_p \P_0,
\end{align}
and use the permutational symmetry of the ground-state manifold to
simplify
\begin{align}
  H_1^{(1)} =  \sum_p v_p \P_0 V_0 \P_0 \simeq v V,
  &&
  v \equiv \EE_p\sp{v_p},
  &&
  V = \sum_p V_p,
  \label{eq:H_1_1}
\end{align}
where $\simeq$ denotes equality up to a restriction to the
ground-state manifold $\M_0$.  In order to compute the second order
effective Hamiltonian $H_1^{(2)}$, we first choose an arbitrary fully
symmetric state $\ket\psi\in\M_0$ and expand (see Appendix
\ref{sec:H_V1_psi}):
\begin{align}
  H_0 \V_1 \ket\psi
  = E_0 \V_1 \ket\psi
  + \sum_{k,q} \p{h_{kq} v_q - h_k v_k} V_k \ket\psi.
  \label{eq:H_V1_psi}
\end{align}
Denoting a vector of the coefficients $v_q$ by $\v v$, a matrix of all
coupling constants $h_{pq}$ by $\m h$, and a vector of all
$h_q\equiv\sum_k h_{kq}$ by $\v h$, we thus find that the vector
$\V_1\ket\psi$ is an eigenvector of the interaction Hamiltonian $H_0$
if the coefficients $\v v$ satisfy
\begin{align}
  \m h \c \v v - \v h\oc\v v
  = \p{\m h-\diag\v h}\c\v v
  = \Delta\p{\v v} \v v,
  \label{eq:cond_1}
\end{align}
where $\m h\c\v v$ is an ordinary matrix-vector product of $\m h$ and
$\v v$, with components $\sum_q h_{kq} v_q$; $\v h\oc\v v$ is an
element-wise (Hadamard) product of $\v h$ and $\v v$, with components
$h_k v_k$; $\diag\v h$ is a matrix with $\v h$ on the diagonal and
zeroes everywhere else; and $\Delta\p{\v v}$ is some constant that may
depend on $\v v$.  If the interaction Hamiltonian $H_0$ is
translationally invariant, then all $h_q=h$ and $\v h\oc\v v=h\v v$,
so the condition in \eqref{eq:cond_1} simplifies to
\begin{align}
  \m h \c\v v = \sp{\Delta\p{\v v}+h} \v v.
\end{align}
If we {\it construct} a perturbation $\V_1\p{\v v_\Delta}$ with
coefficients $\v v_\Delta$ that satisfy \eqref{eq:cond_1} with
eigenvalue $\Delta$, then
\begin{align}
  H_0 \V_1\p{\v v_\Delta} \ket\psi
  = \sp{E_0 + \Delta} \V_1\p{\v v_\Delta} \ket\psi.
\end{align}
Finding operators $\V_1\p{\v v}$ that generate eigenvectors of the
interaction Hamiltonian $H_0$ when they act on fully symmetric states
$\ket\psi$ thus reduces to finding eigenvectors $\v v$ of
$\m h-\diag\v h$.  Interestingly, the energy of the (unnormalized)
state $\V_1\p{\v v}\ket\psi$ depends only on the coefficients $\v v$,
and is entirely independent of the state $\ket\psi$ or choice of
single-spin trace-zero operator used to build $\V_1$ in
\eqref{eq:perturbations}.

If the interaction Hamiltonian $H_0$ is gapped, then a vector
$\V_1\p{\v v_\Delta}\ket\psi$ with $\Delta=0$ must lie with in the
fully symmetric manifold $\M_0$, which implies that the operator
$\V_1\p{\v v_\Delta}$ preserves the permutational symmetry of
$\ket\psi$.  The only way for $\V_1\p{\v v_\Delta}$ to obey
permutational symmetry is for $\v v_\Delta$ to be a constant vector,
i.e.~$\v v_\Delta=v_\Delta\v\1$ with $\1$ a constant vector of ones.
Indeed, a constant vector $\v\1$ satisfies the condition in
\eqref{eq:cond_1} with eigenvalue $0$.  All other vectors
$\v v_\Delta$ satisfying \eqref{eq:cond_1} with $\Delta\ne0$ must be
orthogonal to $\v\1$, and therefore mean-zero.

We now return to the task of computing the second-order effective
Hamiltonian $H_1^{(2)}$.  Any coefficient vector $\v v$ can be
expanded into its projections $\v v_\Delta$ onto the eigenspace of
vectors satisfying \eqref{eq:cond_1} with eigenvalue $\Delta$,
i.e.~$\v v = \sum_\Delta \v v_\Delta$, which also allows us to expand
\begin{align}
  \V_1\p{\v v} = \sum_\Delta \V_1\p{\v v_\Delta},
\end{align}
where each operator $\V_1\p{\v v_\Delta}$ generates a state with
interaction energy $\Delta$ above that of the fully symmetric
manifold.  We can therefore simplify
\begin{align}
  H_1^{(2)}
  = -\P_0 \V_1\p{\v v} \E \V_1\p{\v v} \P_0
  = - \sum_{\Delta>0} \f1{\Delta}
  \P_0 \V_1\p{\v v} \P_\Delta \V_1\p{\v v} \P_0
  = - \sum_{\Delta>0} \f1{\Delta} \P_0 \V_1\p{\v v_\Delta}^2 \P_0,
\end{align}
where the the product $\P_0 \V_1\p{\v v_\Delta}^2 \P_0$ is worked out
in Appendix \ref{sec:PVVP}, giving us
\begin{align}
  H_1^{(2)} \simeq \f1{N-1} \sum_{\Delta>0}
  \f{\var\p{\v v_\Delta}}{\Delta} \p{V^2 - N W},
  \label{eq:H_1_2}
\end{align}
where $\simeq$ denotes equality up to a restriction to the fully
symmetric manifold $\M_0$, and
\begin{align}
  \var\p{\v x} \equiv \EE_p\sp{\p{x_p-\EE_q\sp{x_q}}^2},
  &&
  W \equiv \sum_p V_p^2.
\end{align}

%%%%%%%%%%%%%%%%%%%%%%%%%%%%%%%%%%%%%%%%%%%%%%%%%%%%%%%%%%%%%%%%%%%%%%
\section{Two-body perturbations}

The first-order effective Hamiltonian $H_2^{(1)}$ induced on the fully
symmetric manifold $\M_0$ by the two-body perturbation $\V_2$ in
\eqref{eq:perturbations} can be computed using the permutational
symmetry of $\M_0$ (see Appendix \ref{sec:PXYP}):
\begin{align}
  H_2^{(1)}
  = \P_0 \V_2 \P_0
  = \sum_{p<q} w_{pq} \P_0 X_p Y_q \P_0
  \simeq \f{w}{N-1} \p{\f12 \sp{X,Y}_+ - Z},
\end{align}
where
\begin{align}
  w \equiv \f1N \sum_{p<q} w_{pq},
  &&
  X \equiv \sum_p X_p,
  &&
  Y \equiv \sum_p Y_p,
  &&
  Z \equiv \f12 \sum_p \sp{X_p, Y_p}_+,
\end{align}
and $\sp{X,Y}_+\equiv XY+YX$ is an anti-commutator.  In order to
compute the second-order effective Hamiltonian $H_2^{(2)}$, as before
we pick an arbitrary fully symmetric state $\ket\psi\in\M_0$ and
expand (see Appendix \ref{sec:H_V2_psi})
\begin{align}
  H_0 \V_2 \ket\psi
  = E_0 \V_2 \ket\psi
  + \sum_{k,\ell} \p{\sp{h_{k\ell}-\p{\f{h_k+h_\ell}{2}}} w_{k\ell}
    + \sum_q h_{kq} w_{q\ell}} X_k Y_\ell \ket\psi
  - \sum_{k,\ell} h_{k\ell} w_{k\ell} Z_k \ket\psi,
  \label{eq:H_V2_psi}
\end{align}
where $Z_k\equiv\p{1/2}\sp{X_k,Y_k}_+$.  We thus find that the vector
$\V_2\ket\psi$ is an eigenvector of the interaction Hamiltonian $H_0$
if the matrix $\m w$ of all $w_{k\ell}$ satisfies
\begin{align}
  \bar{\m h}\oc\m w + \m h\c\m w = \e\p{\m w} \m w,
  &&
  \bar{\m h}
  \equiv \m h - \f12\p{\v h\otimes\v\1 + \v\1\otimes\v h},
  \label{eq:cond_2}
\end{align}
where $\bar{\m h}\oc\m w$ denotes an element-wise (Hadamard) product
of $\bar{\m h}$ and $\m w$, with components
$\bar h_{k\ell} w_{k\ell}$; $\m h\c\m w$ denotes an ordinary matrix
product of $\m h$ and $\m w$, with components
$\sum_q h_{kq} w_{q\ell}$; $\e\p{\m w}$ is some constant that may
depend on $\m w$; $\v\1$ is a constant vector of ones; and
$\v h\otimes\v\1$ is a tensor product of $\v h$ and $\v\1$.  Note that
the diagonal components of the condition in \eqref{eq:cond_2}
automatically enforce that the last sum in \eqref{eq:H_V2_psi}
vanishes.

We can re-cast the condition in \eqref{eq:cond_2} as an eigenvalue
problem by writing the coefficient matrix $\m w$ as a vector in
$\RR^{N^2}$,
\begin{align}
  \m w = \sum_{k,\ell} w_{k\ell} \op{k}{\ell}
  \to \v{\m w} \equiv \sum_{k,\ell} w_{k\ell} \ket{k\ell},
\end{align}
and defining the matrix
\begin{align}
  \check{\m h}
  \equiv \sum_{k,\ell} \bar h_{k\ell} \op{k\ell}
  + \sum_{k,q,\ell} h_{kq} \op{k\ell}{q\ell},
  \label{eq:h_super_mat}
\end{align}
in terms of which the condition in \eqref{eq:cond_2} becomes the
eigenvalue equation
\begin{align}
  \check{\m h} \c \v{\m w} = \e\p{\m w} \v{\m w}.
  \label{eq:cond_2_eig}
\end{align}
The matrix $\check{\m h}$ in \eqref{eq:h_super_mat} can be written in
the block-diagonal form
\begin{align}
  \check{\m h} = \sum_\ell \m h_\ell \otimes \op{\ell},
  &&
  \m h_\ell
  \equiv \sum_{k,q}\p{\delta_{kq} \bar h_{k\ell} + h_{kq}} \op{k}{q},
\end{align}
which reduces the eigenvalue equation in \eqref{eq:cond_2_eig} to the
$N$ smaller, uncoupled eigenvalue equations
\begin{align}
  \m h_\ell \c \v{\m w}_\ell = \e\p{\m w} \v{\m w}_\ell,
  &&
  \v{\m w}_\ell \equiv \sum_k w_{k\ell} \ket{k}.
  \label{eq:cond_2_eig_block}
\end{align}
We can thus (numerically) diagonalize the blocks $\m h_\ell$
independently, and classify their eigenvectors $\v{\m w}_\ell$ by the
corresponding eigenvalues $\e\p{\m w}$.  ``Total'' eigenvectors
$\v{\m w}$ of $\check{\m h}$ are then built from eigenvectors
$\v{\m w}_\ell$ within each block $\m h_\ell$ that share the same
eigenvalue $\e\p{\m w}$.  The task of constructing eigenvectors
$\V_2\p{\m w}\ket\psi$ of $H_0$ and determining their energy thus
reduces to finding the eigenvectors and eigenvalues of $N$ matrices
$\m h_\ell$ of size $N\times N$.

If the interaction Hamiltonian $H_0$ is translationally invariant,
then $\bar h_{k\ell}=h_{k\ell}-h$, and $h_{k\ell}$ depends only on the
displacement between the spins indexed by $k$ and $\ell$.  The blocks
$\m h_\ell$ then take the form
\begin{align}
  \m h_{\ell} = \m\Sigma_\ell \c \m h_0 \c \m\Sigma_\ell^\dag,
  &&
  \m\Sigma_\ell \equiv \sum_k \op{k+\ell}{k},
  \label{eq:block_shift}
\end{align}
which implies that we need only find the eigenvectors and eigenvalues
of a single block $\m h_0$, after which we can determine the
eigenvectors of any block simply by appropriate applications of the
shift matrix $\m\Sigma_\ell$.  Note that in dimensions $D>1$, the
relation in \eqref{eq:block_shift} holds for vectors $k,\ell$ that
index lattice sites, i.e.~$k=\p{k_1,k_2,\cdots,k_D}$ and likewise with
$\ell$.

As with the case of single-body perturbations, we can decompose any
matrix $\m w$ into its projections $\m w_\e$ onto the eigenspace of
matrices satisfying \eqref{eq:cond_2} with eigenvalue $\e$,
i.e.~$\m w=\sum_\e\m w_\e$, which allows us to expand
\begin{align}
  \V_2\p{\m w} = \sum_\e \V_2\p{\m w_\e}.
\end{align}
We can then simplify
\begin{align}
  H_2^{(2)} = - \P_0 \V_2\p{\m w} \E \V_2\p{\m w} \P_0
  = -\sum_{\e>0} \f1\e \P_0 \V_2\p{\m w_\e} \P_\e \V_2\p{\m w_\e} \P_0
  = -\sum_{\e>0} \f1\e \P_0 \V_2\p{\m w_\e}^2 \P_0,
\end{align}
where the product $\P_0 \V_2\p{\m w_\e}^2 \P_0$ is worked out in
Appendix \ref{sec:PVVP_2}, giving us \red{[to do: finish calculation]}
\begin{align}
  H_2^{(2)} =
\end{align}

\vspace{3cm}

\appendix

%%%%%%%%%%%%%%%%%%%%%%%%%%%%%%%%%%%%%%%%%%%%%%%%%%%%%%%%%%%%%%%%%%%%%%
\section{Diagnosing perturbed states with the interaction Hamiltonian}

Here we simplify vectors of the form $H_0\V_M\ket\psi$ to arrive at
the expansions in \eqref{eq:H_V1_psi} and \eqref{eq:H_V2_psi}.

%%%%%%%%%%%%%%%%%%%%%%%%%%%%%%%%%%%%%%%%%%%%%%%%%%
\subsection{Single-body perturbation}
\label{sec:H_V1_psi}

We wish to simplify
\begin{align}
  H_0 \V_1 \ket\psi
  = \sum_{\substack{k\\p<q}} h_{pq} v_k \Pi_{pq} V_k \ket\psi,
  \label{eq:H_V1_psi_start}
\end{align}
which has terms with $k\in\set{p,q}$, and terms with
$k\notin\set{p,q}$.  In the case of $k\notin\set{p,q}$, the
permutation operator $\Pi_{pq}$ commutes with $V_k$ and annihilates on
$\ket\psi$, leaving us the sum
\begin{align}
  \sum_{\substack{p<q\\k\notin\set{p,q}}} h_{pq}
  = \f12 \sum_{\substack{p,q\\k\notin\set{p,q}}} h_{pq}
  = \f12 \sum_{p,q} h_{pq}
  - \f12 \sum_{\substack{p,q\\k\in\set{p,q}}} h_{pq}
  = \f12 N h - \sum_q h_{kq}
  = E_0 - h_k,
\end{align}
where we have used the facts that $h_{pp}=0$ and $h_{pq}=h_{qp}$.  The
terms with $k\notin\set{p,q}$ in \eqref{eq:H_V1_psi_start} are then
\begin{align}
  \sum_{\substack{p<q\\k\notin\set{p,q}}}
  h_{pq} v_k \Pi_{pq} V_k \ket\psi
  = \sum_k \p{E_0 - h_k} v_k V_k \ket\psi
  = E_0 \V_1 \ket\psi - \sum_k h_k v_k V_k \ket\psi.
\end{align}
In the case of $k\in\set{p,q}$, we have
\begin{align}
  \sum_{\substack{p<q\\k\in\set{p,q}}}
  h_{pq} v_k \Pi_{pq} V_k \ket\psi
  = \sum_{k<q} h_{kq} v_k V_q \ket\psi
  + \sum_{p<k} h_{pk} v_k V_p \ket\psi
  = \sum_{k,q} h_{kq} v_k V_q \ket\psi,
\end{align}
which with some re-indexing gives us
\begin{align}
  H_0 \V_1 \ket\psi
  = E_0 \V_1 \ket\psi
  + \sum_{k,q} \p{h_{kq} v_q - h_k v_k} V_k \ket\psi.
\end{align}

%%%%%%%%%%%%%%%%%%%%%%%%%%%%%%%%%%%%%%%%%%%%%%%%%%
\subsection{Two-body perturbation}
\label{sec:H_V2_psi}

We wish to simplify
\begin{align}
  H_0 \V_2 \ket\psi
  = \f12 \sum_{\substack{k,\ell\\p<q}} w_{k\ell} h_{pq}
  \Pi_{pq} X_k Y_\ell \ket\psi,
  \label{eq:H_V2_psi_start}
\end{align}
which has terms with $p,q\notin\set{k,\ell}$, one term with
$\set{p,q}=\set{k,\ell}$, and terms with $p,q\in\set{k,\ell}$.  The
permutation operator $\Pi_{pq}$ acts trivially on
$X_k Y_\ell \ket\psi$ when $p,q\notin\set{k,\ell}$ or
$\set{p,q}=\set{k,\ell}$, leaving a sum over $p,q$ of the form
\begin{align}
  \sum_{\substack{p<q\\p,q\notin\set{k,\ell}}} h_{pq} + h_{k\ell}
  = \sum_{p<q} h_{pq}
  - \sum_{\substack{p\in\set{k,\ell}\\q\notin\set{k,\ell}}} h_{pq}
  = E_0 + 2 h_{k\ell} - h_k - h_\ell.
\end{align}
The corresponding terms in \eqref{eq:H_V2_psi_start} with
$p,q\notin\set{k,\ell}$ and $\set{p,q}=\set{k,\ell}$ are then
\begin{align}
  E_0 \V_2 \ket\psi + \f12 \sum_{k,\ell}
  \p{2 h_{k\ell} - h_k - h_\ell}
  w_{k\ell} X_k Y_\ell \ket\psi,
\end{align}
while the terms in \eqref{eq:H_V2_psi_start} with only one of
$p,q\in\set{k,\ell}$ simplify to
\begin{align}
  \f12 \sum_{\substack{k,\ell\\q\notin\set{k,\ell}}} w_{k\ell}
  \p{h_{kq} X_q Y_\ell + h_{\ell q} X_k Y_q} \ket\psi
  = \sum_{k,\ell,q} w_{k\ell} h_{kq} X_q Y_\ell \ket\psi
  - \sum_{k,\ell} w_{k\ell} h_{k\ell} Z_k \ket\psi,
\end{align}
where
\begin{align}
  Z_k \equiv \f12 \sp{X_k, Y_k}_+.
\end{align}
In total,
\begin{align}
  H_0 \V_2 \ket\psi
  = E_0 \V_2 \ket\psi
  + \sum_{k,\ell} \p{\sp{h_{k\ell}-\p{\f{h_k+h_\ell}{2}}} w_{k\ell}
    + \sum_q h_{kq} w_{q\ell}} X_k Y_\ell \ket\psi
  - \sum_{k,\ell} h_{k\ell} w_{k\ell} Z_k \ket\psi.
\end{align}


%%%%%%%%%%%%%%%%%%%%%%%%%%%%%%%%%%%%%%%%%%%%%%%%%%%%%%%%%%%%%%%%%%%%%%
\section{Simplifying operator products restricted to the
  fully-symmetric manifold}

%%%%%%%%%%%%%%%%%%%%%%%%%%%%%%%%%%%%%%%%%%%%%%%%%%
\subsection{Single-body product}
\label{sec:PVVP}

Computing the second-order effective Hamiltonian $H_1^{(2)}$ induced
on the fully symmetric manifold $\M_0$ by the perturbation $\V_1$ in
\eqref{eq:perturbations} requires us to simplify the product
\begin{align}
  \P_0 \V_1\p{\v v}^2 \P_0
  = \sum_{p,q} v_p v_q \P_0 V_p V_q \P_0
  = \sum_p v_p^2 \P_0 V_p^2 \P_0
  + \sum_{p\ne q} v_p v_q \P_0 V_p V_q \P_0,
  \label{eq:PVVP_start}
\end{align}
where the coefficients $\v v$ satisfy \eqref{eq:cond_1} with
eigenvalue $\Delta\ne0$, which implies that $\v v$ is mean-zero.  The
first sum on the right of \eqref{eq:PVVP_start} is then
\begin{align}
  \sum_p v_p^2 \P_0 V_p^2 \P_0
  = \sum_p v_p^2 \P_0 V_0^2 \P_0
  = \var\p{\v v} \P_0 W \P_0,
  \label{eq:PVVP_eq}
\end{align}
where
\begin{align}
  \var\p{\v v} \equiv \EE_p\sp{\p{v_p-\EE_q\sp{v_q}}^2}
  = \EE_p\sp{v_p^2} - \EE_q\sp{v_q}^2,
  &&
  W \equiv \sum_p V_p^2.
\end{align}
The second sum in \eqref{eq:PVVP_start} is
\begin{align}
  \sum_{p\ne q} v_p v_q \P_0 V_p V_q \P_0
  = \sum_{p\ne q} v_p v_q \P_0 V_0 V_1 \P_0,
  \label{eq:PVVP_neq_start}
\end{align}
where the sum over coefficients in \eqref{eq:PVVP_neq_start} can be
simplified to
\begin{align}
  \sum_{p\ne q} v_p v_q
  = \sum_{p,q} v_p v_q - \sum_{p=q} v_p v_q
  = - \sum_p v_p^2 = - N \var\p{\v v}.
\end{align}
In order to simplify the product of operators in
\eqref{eq:PVVP_neq_start}, we expand
\begin{align}
  \P_0 V^2 \P_0
  = \sum_p \P_0 V_p^2 \P_0
  + \sum_{p\ne q} \P_0 V_p V_q \P_0
  = \P_0 W \P_0 + N \p{N-1} \P_0 V_0 V_1 \P_0,
  \label{eq:PVVP_neq_ops}
\end{align}
which implies that
\begin{align}
  \sum_{p\ne q} v_p v_q \P_0 V_p V_q \P_0
  = - \var\p{\v v} N \P_0 V_0 V_1 \P_0
  = -\f{\var\p{\v v}}{N-1} \P_0 \p{V^2 - W} \P_0
  \label{eq:PVVP_neq}
\end{align}
Altogether, we have that
\begin{align}
  \P_0 \V_1\p{\v v}^2 \P_0
  = -\f{\var\p{\v v}}{N-1} \P_0 \p{V^2 - N W} \P_0,
\end{align}

%%%%%%%%%%%%%%%%%%%%%%%%%%%%%%%%%%%%%%%%%%%%%%%%%%
\subsection{Two-body product at first order}
\label{sec:PXYP}

Computing the first-order effective Hamiltonian $H_2^{(1)}$ induced on
the fully symmetric manifold $\M_0$ by the perturbation $\V_2$ in
\eqref{eq:perturbations} requires us to simplify the product
\begin{align}
  \P_0 \V_2 \P_0
  = \sum_{p<q} w_{pq} \P_0 X_p Y_q \P_0
  = w \times N \P_0 X_0 Y_1 \P_0,
  \label{eq:PXYP}
\end{align}
where
\begin{align}
  w \equiv \f1N \sum_{p<q}
\end{align}
is the mean value of the pair-wise coefficients in $\V_2$.  To
simplify the operator content of \eqref{eq:PXYP}, we expand
\begin{align}
  \f12 \P_0 \sp{X, Y}_+ \P_0
  = \P_0 Z \P_0 + N\p{N-1} \P_0 X_0 Y_1 \P_0,
\end{align}
where $\sp{A,B}_+\equiv AB+BA$ is an anti-commutator and
\begin{align}
  Z \equiv \f12 \sum_p \sp{X_p,Y_p}_+.
\end{align}
We thus find that
\begin{align}
  \P_0 \V_2 \P_0
  = \f{w}{N-1} \P_0 \p{\f12 \sp{X,Y}_+ -  Z } \P_0.
\end{align}

%%%%%%%%%%%%%%%%%%%%%%%%%%%%%%%%%%%%%%%%%%%%%%%%%%
\subsection{Two-body product at second order}
\label{sec:PVVP_2}

Computing the second-order effective Hamiltonian $H_2^{(2)}$ induced
on the fully symmetric manifold $\M_0$ by the perturbation $\V_2$ in
\eqref{eq:perturbations} requires us to simplify the product
\begin{align}
  \P_0 \V_2\p{\m w}^2 \P_0
  = \f14 \sum_{k,\ell,p,q} w_{k\ell} w_{pq} \P_0 X_k Y_\ell X_p Y_q \P_0.
\end{align}
The terms with $p,q\notin\set{k,\ell}$ are
\begin{align}
  \f14 \sum_{\substack{k,\ell\\p,q\notin\set{k,\ell}}} w_{k\ell} w_{pq}
  \P_0 X_0 Y_1 X_2 Y_3 \P_0.
\end{align}


\red{[to do: finish calculation]}

\bibliography{multilevel_spin_notes.bib}

\end{document}
