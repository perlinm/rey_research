\documentclass[nofootinbib,notitlepage,11pt]{revtex4-2}

%%% linking references
\usepackage{hyperref}
\hypersetup{
  breaklinks=true,
  colorlinks=true,
  linkcolor=blue,
  filecolor=magenta,
  urlcolor=cyan,
}

%%% header / footer
\usepackage{fancyhdr} % easier header and footer management
\pagestyle{fancy} % page formatting style
\fancyhf{} % clear all header and footer text
\renewcommand{\headrulewidth}{0pt} % remove horizontal line in header
\usepackage{lastpage} % for referencing last page
\cfoot{\thepage~of \pageref{LastPage}} % "x of y" page labeling


%%% symbols, notations, etc.
\usepackage{physics,braket,bm,amssymb} % physics and math
\renewcommand{\t}{\text} % text in math mode
\newcommand{\f}[2]{\dfrac{#1}{#2}} % shorthand for fractions
\newcommand{\p}[1]{\left(#1\right)} % parenthesis
\renewcommand{\sp}[1]{\left[#1\right]} % square parenthesis
\renewcommand{\set}[1]{\left\{#1\right\}} % curly parenthesis
\newcommand{\bk}{\Braket} % shorthand for braket notation

\renewcommand{\c}{\cdot} % inner product
\renewcommand{\oc}{\circ} % element-wise product

\newcommand{\m}{\bm} % bold symbol
\renewcommand{\v}{\vec} % arrow vector

\usepackage{dsfont} % for identity operator
\newcommand{\1}{\mathds{1}}

\newcommand{\up}{\uparrow}
\newcommand{\dn}{\downarrow}

\renewcommand{\d}{\text{d}}
\newcommand{\x}{\text{x}}
\newcommand{\y}{\text{y}}
\newcommand{\z}{\text{z}}

\newcommand{\e}{\varepsilon}

\newcommand{\B}{\mathcal{B}}
\newcommand{\D}{\mathcal{D}}
\newcommand{\E}{\mathcal{E}}
\renewcommand{\H}{\mathcal{H}}
\newcommand{\I}{\mathcal{I}}
\newcommand{\M}{\mathcal{M}}
\newcommand{\N}{\mathcal{N}}
\renewcommand{\O}{\mathcal{O}}
\renewcommand{\P}{\mathcal{P}}
\newcommand{\Q}{\mathcal{Q}}
\newcommand{\R}{\mathcal{R}}
\newcommand{\T}{\mathcal{T}}
\renewcommand{\S}{\mathcal{S}}
\newcommand{\V}{\mathcal{V}}
\newcommand{\X}{\mathcal{X}}
\newcommand{\Z}{\mathcal{Z}}

\newcommand{\EE}{\mathbb{E}}
\renewcommand{\SS}{\mathbb{S}}
\newcommand{\ZZ}{\mathbb{Z}}

\newcommand{\FS}{\text{FS}}

\DeclareMathOperator{\sign}{sign}
\DeclareMathOperator{\cov}{cov}
\let\var\relax
\DeclareMathOperator{\var}{var}
\DeclareMathOperator{\diag}{diag}

\def\obra#1{\mathinner{({#1}|}}
\def\oket#1{\mathinner{|{#1})}}
\def\obk#1{\mathinner{({#1})}}
\def\oop#1#2{\oket{#1}\!\obra{#2}}

\usepackage[inline]{enumitem} % in-line lists and \setlist{} (below)
\setlist[enumerate,1]{label={(\roman*)}} % default in-line numbering
\setlist{nolistsep} % more compact spacing between environments

%%% text markup
\usepackage{color} % text color
\newcommand{\red}[1]{{\color{red} #1}}

%%%%%%%%%%%%%%%%%%%%%%%%%%%%%%%%%%%%%%%%%%%%%%%%%%%%%%%%%%%%%%%%%%%%%%
\begin{document}
\thispagestyle{fancy}

\title{Perturbing SU($n$)-symmetric interactions}%
\author{Michael A. Perlin}%
\date{\today}

\maketitle

We consider an array of $N$ multilevel spins with non-uniform
SU($n$)-symmetric interactions that can be written in the form
\begin{align}
  H_0 = \sum_{p<q} h_{pq} \Pi_{pq},
  &&
  \Pi_{pq} \equiv \sum_{\mu,\nu} S_{\mu\nu}^{(p)} S_{\nu\mu}^{(q)},
\end{align}
where $S_{\mu\nu}^{(p)}\equiv\op{\mu}{\nu}_p$ flips the state of spin
$p$ to $\ket\mu$ from $\ket\nu$; the operator $\Pi_{pq}$ permutes
spins $p$ and $q$, and the coefficients $h_{pq}$ satisfy
\begin{align}
  h_{pp} = 0
  &&
  h_{pq} \le 0,
  &&
  h_{pq} = h_{qp}.
\end{align}
The ground-state manifold of the interaction Hamiltonian $H_0$
consists of fully symmetric states that are simultaneous $+1$
eigenstates of all permutation operators $\Pi_{pq}$.  The energy of
these states is
\begin{align}
  E_0 \equiv \sum_{p<q} h_{pq} = \f12 \sum_{p,q} h_{pq}
  = \f12 \sum_p h_p = \f12 N h,
\end{align}
where we define
\begin{align}
  h_p \equiv \sum_q h_{pq},
  &&
  h \equiv \EE_p\sp{h_p} = \f1N \sum_{p,q} h_{pq},
  &&
  \EE_p\sp{X_p} \equiv \f1N \sum_p X_p.
\end{align}
We wish to determine the effective dynamics induced on the manifold
$\M_0$ of fully symmetric states by weak perturbations of the form
\begin{align}
  \V_1 \equiv \sum_p v_p V_p,
  &&
  \V_2 \equiv \sum_{p<q} v_{pq} V_p V_q,
  \label{eq:perturbations}
\end{align}
where $V_p$ is a single-body operator on spin $p$, and
$v_{pq}=v_{qp}$.  The effective Hamiltonian $H_W$ induced on the
ground-state manifold $\M_0$ by a perturbation $\V_W$ through second
order in perturbation theory is given by\cite{bravyi2011schrieffer,
  perlin2019effective}
\begin{align}
  H_W = H_W^{(1)} + H_W^{(2)},
  &&
  H_W^{(1)} = \P_0 \V_W \P_0,
  &&
  H_W^{(2)} = - \P_0 \V_W \E \V_W \P_0,
  &&
  \E \equiv \sum_{k>0} \f{\P_k}{\Delta_k},
\end{align}
where $\P_k$ is a projector onto the $k$-th excited eigenspace of the
interaction Hamiltonian $H_0$, with interaction energy $\Delta_k$
above that of fully symmetric manifold $\M_0$.

As an immediate disclaimer, we note that these notes do {\bf not}
solve the general problem of calculating effective Hamiltonians
$H_1,H_2$ for arbitrary perturbations $\V_1,\V_2$.  Instead, we
\begin{enumerate*}
\item calculate the first-order effective Hamiltonians
  $H_1^{(1)},H_2^{(1)}$,
\item determine conditions under which the perturbations $\V_1,\V_2$
  generate eigenvectors of the interaction Hamiltonian $H_0$, and
\item calculate $H_1$ in the case that these conditions are satisfied
  by $\V_1$.
\end{enumerate*}
When the conditions for $\V_1,\V_2$ to generate eigenvectors of $H_0$
are not satisfied, we hope that the techniques in these notes can be
used to construct eigenvectors and eigenvalues of $H_0$ that will
enable the use of ordinary perturbation theory to determine the
effective Hamiltonians $H_1,H_2$ induced on the fully symmetric
manifold $\M_0$ by the perturbations $\V_1,\V_2$.

%%%%%%%%%%%%%%%%%%%%%%%%%%%%%%%%%%%%%%%%%%%%%%%%%%%%%%%%%%%%%%%%%%%%%%
\section{Single-body perturbations}

We first compute the effective Hamiltonian induced by the single-body
perturbation $\V_1$.  We therefore expand
\begin{align}
  H_1^{(1)} = \sum_p v_p \P_0 V_p \P_0,
\end{align}
and use the permutational symmetry of the ground-state manifold to
simplify
\begin{align}
  H_1^{(1)} =  \sum_p v_p \P_0 V_0 \P_0 \simeq v V,
  &&
  v \equiv \EE_p\sp{v_p},
  &&
  V = \sum_p V_p,
  \label{eq:H_1_1}
\end{align}
where $\simeq$ denotes equality up to a restriction to the
ground-state manifold $\M_0$.  In order to compute the second order
effective Hamiltonian $\H_1^{(2)}$, we first choose an arbitrary fully
symmetric state $\ket\psi\in\M_0$ and expand
\begin{align}
  H_0 \V_1 \ket\psi
  = \sum_{\substack{k\\p<q}} h_{pq} v_k \Pi_{pq} V_k \ket\psi
  = \sum_{\substack{p<q\\k\notin\set{p,q}}}
  h_{pq} v_k \Pi_{pq} V_k \ket\psi
  + \sum_{\substack{p<q\\k\in\set{p,q}}}
  h_{pq} v_k \Pi_{pq} V_k \ket\psi.
  \label{eq:H_V1_psi_start}
\end{align}
In the case of $k\notin\set{p,q}$, the permutation operator $\Pi_{pq}$
commutes with $V_k$ and annihilates on $\ket\psi$, leaving us the sum
\begin{align}
  \sum_{\substack{p<q\\k\notin\set{p,q}}} h_{pq}
  = \f12 \sum_{\substack{p,q\\k\notin\set{p,q}}} h_{pq}
  = \f12 \sum_{p,q} h_{pq}
  - \f12 \sum_{\substack{p,q\\k\in\set{p,q}}} h_{pq}
  = \f12 N h - \sum_q h_{kq}
  = E_0 - h_k,
\end{align}
where we have used the facts that $h_{pp}=0$ and $h_{pq}=h_{qp}$.  The
terms with $k\notin\set{p,q}$ in \eqref{eq:H_V1_psi_start} are then
\begin{align}
  \sum_{\substack{p<q\\k\notin\set{p,q}}}
  h_{pq} v_k \Pi_{pq} V_k \ket\psi
  = \sum_k \p{E_0 - h_k} v_k V_k \ket\psi
  = E_0 \V_1 \ket\psi - \sum_k h_k v_k V_k \ket\psi.
\end{align}
In the case of $k\in\set{p,q}$, we have
\begin{align}
  \sum_{\substack{p<q\\k\in\set{p,q}}}
  h_{pq} v_k \Pi_{pq} V_k \ket\psi
  = \sum_{k<q} h_{kq} v_k V_q \ket\psi
  + \sum_{p<k} h_{pk} v_k V_p \ket\psi
  = \sum_{k,q} h_{kq} v_k V_q \ket\psi,
\end{align}
so in total,
\begin{align}
  H_0 \V_1 \ket\psi
  = E_0 \V_1 \ket\psi + \sum_{q,k} h_{kq} \p{v_k - v_q} V_q \ket\psi.
  \label{eq:H_V1_psi}
\end{align}
Denoting a vector of the coefficients $v_q$ by $\v v$, a matrix of all
coupling constants $h_{pq}$ by $\m h$, and a vector of all
$h_q\equiv\sum_k h_{kq}$ by $\v h$, we thus find that the vector
$\V_1\ket\psi$ is an eigenvector of the interaction Hamiltonian $H_0$
if the coefficients $\v v$ satisfy
\begin{align}
  \m h \c \v v - \v h_\d\oc\v v = \Delta\p{\v v} \v v,
  \label{eq:v_cond_1}
\end{align}
where $\m h\c\v v$ is an ordinary matrix-vector product of $\m h$ and
$\v v$, with components $\sum_q h_{kq} v_q$; $\v h_\d\oc\v v$ is an
element-wise (Hadamard) product of $\v h_\d$ and $\v v$, with
components $h_k v_k$; and $\Delta\p{\v v}$ is some constant that may
depend on $\v v$.  If the interaction Hamiltonian $H_0$ is
translationally invariant, then all $h_q=h$ and
$\v h_\d\oc\v v=h\v v$, so the condition in \eqref{eq:v_cond_1}
simplifies to
\begin{align}
  \m h \c\v v = \sp{\Delta\p{\v v}+h} \v v.
\end{align}
If we construct a perturbation $\V_1\p{\v v}$ with coefficients $\v v$
that satisfy \eqref{eq:v_cond_1}, then
\begin{align}
  H_0 \V_1\p{\v v} \ket\psi
  = \sp{E_0 + \Delta\p{\v v}} \V_1\p{\v v} \ket\psi,
\end{align}
for any fully symmetric state $\ket\psi\in\M_0$, which for
$\Delta\p{\v v}>0$ implies that
\begin{align}
  \E \V_1\p{\v v} \P_0
  = \sum_{k>0} \f1{\Delta_k} \P_k \V_1\p{\v v} \P_0
  = \f1{\Delta\p{\v v}} \V_1\p{\v v} \P_0,
\end{align}
and in turn
\begin{align}
  H_1^{(2)}
  = -\P_0 \V_1\p{\v v} \E \V_1\p{\v v} \P_0
  = -\f1{\Delta\p{\v v}} \P_0 \V_1\p{\v v}^2 \P_0.
\end{align}
Computing the second-order effective Hamiltonian $H_1^{(2)}$ thus
requires us to simplify the product
\begin{align}
  \P_0 \V_1\p{\v v}^2 \P_0
  = \sum_{p,q} v_p v_q \P_0 V_p V_q \P_0
  = \sum_p v_p^2 \P_0 V_p^2 \P_0
  + \sum_{p\ne q} v_p v_q \P_0 V_p V_q \P_0.
  \label{eq:PVVP_start}
\end{align}
We note that vectors $\v v$ satisfying \eqref{eq:v_cond_1} and
$\Delta\p{\v v}\ne 0$ must be mean-zero, as evident by summing over
the components of \eqref{eq:v_cond_1}:
\begin{align}
  \Delta\p{\v v} \sum_k v_k
  = \sum_{k,q} h_{kq} v_q - \sum_k h_k v_k
  = \sum_q h_q v_q - \sum_k h_k v_k
  = 0.
  \label{eq:D_v_0}
\end{align}
It follows that the first sum on the right of \eqref{eq:PVVP_start} is
\begin{align}
  \sum_p v_p^2 \P_0 V_p^2 \P_0
  = \sum_p v_p^2 \P_0 V_0^2 \P_0
  = \var\p{\v v} \P_0 W \P_0,
  &&
  W \equiv \sum_p V_p^2,
\end{align}
where
\begin{align}
  \var\p{\v x} \equiv \EE_p\sp{\p{x_p-\EE_q\sp{x_q}}^2}
  = \EE_p\sp{x_p^2} - \EE_q\sp{x_q}^2
\end{align}
is the variance of the elements in the vector $\v x$.  Similarly, the
second sum on the right of \eqref{eq:PVVP_start} is
\begin{align}
  \sum_{p\ne q} v_p v_q \P_0 V_p V_q \P_0
  = \sum_{p\ne q} v_p v_q \P_0 V_0 V_1 \P_0,
  \label{eq:PVVP_neq_start}
\end{align}
where if the coefficients in $\v v$ are mean-zero, then
\begin{align}
  \sum_{p\ne q} v_p v_q
  = \sum_p v_p \sum_{q\ne p} v_q
  = \sum_p v_p \p{\sum_q v_q - v_p}
  = - \sum_p v_p^2
  = - N \var\p{\v v}.
\end{align}
To evaluate the operator content of \eqref{eq:PVVP_neq_start}, we
expand
\begin{align}
  \P_0 V^2 \P_0
  = \sum_p \P_0 V_p^2 \P_0
  + \sum_{p\ne q} \P_0 V_p V_q \P_0
  = \P_0 W \P_0 + N \p{N-1} \P_0 V_0 V_1 \P_0,
  \label{eq:PVVP_neq_ops}
\end{align}
which implies that
\begin{align}
  \sum_{p\ne q} v_p v_q \P_0 V_p V_q \P_0
  = -\f{\var\p{\v v}}{N-1}\P_0\p{V^2 - W} \P_0.
  \label{eq:PVVP_neq}
\end{align}
Altogether, we have that
\begin{align}
  \P_0 \V_1\p{\v v}^2 \P_0
  = -\f{\var\p{\v v}}{N-1} \P_0 \p{V^2 - N W} \P_0,
\end{align}
and so
\begin{align}
  H_1^{(2)} = \f{\var\p{\v v}}{\p{N-1}\Delta\p{\v v}} \p{V^2 - N W},
\end{align}
where we implicitly restrict the effective Hamiltonian $H_1^{(2)}$ to
the ground-state manifold $\M_0$, and note that the result above holds
only for perturbations $\V_1\p{\v v}$ constructed from vectors $\v v$
that satisfy the eigenvalue equation in \eqref{eq:v_cond_1}.

%%%%%%%%%%%%%%%%%%%%%%%%%%%%%%%%%%%%%%%%%%%%%%%%%%%%%%%%%%%%%%%%%%%%%%
\section{Two-body perturbations}

The first-order effective Hamiltonian $H_2^{(1)}$ induced on the fully
symmetric manifold $\M_0$ by the two-body perturbation in $\V_2$ in
\eqref{eq:perturbations} is
\begin{align}
  H_2^{(1)}
  = \P_0 \V_2 \P_0
  = \sum_{p<q} v_{pq} \P_0 V_p V_q \P_0
  \simeq \f{v}{N-1} \p{V^2 - W},
  &&
  v \equiv \f1N \sum_{p<q} v_{pq},
\end{align}
where we made use of the previous result in \eqref{eq:PVVP_neq_ops}.
In order to compute the second-order effective Hamiltonian
$H_2^{(2)}$, as before we pick an arbitrary fully symmetric state
$\ket\psi\in\M_0$ and expand
\begin{align}
  H \V_2 \ket\psi
  = \sum_{\substack{k<\ell\\p<q}} v_{k\ell} h_{pq}
  \Pi_{pq} V_k V_\ell \ket\psi
  = \f12 \sum_{\substack{k,\ell\\p<q}} v_{k\ell} h_{pq}
  \Pi_{pq} V_k V_\ell \ket\psi.
  \label{eq:H_V2_psi_start}
\end{align}
The permutation operator $\Pi_{pq}$ acts trivially on
$V_k V_\ell \ket\psi$ when $p,q\notin\set{k,\ell}$ or
$\set{p,q}=\set{k,\ell}$, leaving a sum over $p,q$ of the form
\begin{align}
  \sum_{\substack{p<q\\p,q\notin\set{k,\ell}}} h_{pq} + h_{k\ell}
  = \sum_{p<q} h_{pq}
  - \sum_{\substack{p<q\\p\in\set{k,\ell}\\q\notin\set{k,\ell}}} h_{pq}
  - \sum_{\substack{p<q\\p\notin\set{k,\ell}\\q\in\set{k,\ell}}} h_{pq}
  = E_0 - \sum_{\substack{p\in\set{k,\ell}\\q\notin\set{k,\ell}}} h_{pq}
  = E_0 + 2 h_{k\ell} - h_k - h_\ell.
\end{align}
The corresponding terms in \eqref{eq:H_V2_psi_start} with
$p,q\notin\set{k,\ell}$ and $\set{p,q}=\set{k,\ell}$ are then
\begin{align}
  E_0 \V_2 \ket\psi + \f12 \sum_{k,\ell}
  \p{2 h_{k\ell} - h_k - h_\ell} v_{k\ell} V_k V_\ell \ket\psi,
\end{align}
while the terms in \eqref{eq:H_V2_psi_start} with only one of
$p,q\in\set{k,\ell}$ simplify to
\begin{align}
  \f12 \sum_{\substack{k,\ell\\q\notin\set{k,\ell}}}
  v_{k\ell} \p{h_{kq} V_q V_\ell + h_{\ell q} V_k V_q} \ket\psi
  = \sum_{k,\ell,q} v_{k\ell} h_{kq} V_q V_\ell \ket\psi
  - \sum_{k,\ell} v_{k\ell} h_{k\ell} V_k^2 \ket\psi.
\end{align}
In total,
\begin{align}
  H_0 \V_2 \ket\psi
  = E_0 \V_2 \ket\psi
  + \sum_{k,\ell} \p{\sp{h_{k\ell}-\p{\f{h_k+h_\ell}{2}}} v_{k\ell}
    + \sum_q h_{kq} v_{q\ell}} V_k V_\ell \ket\psi
  - \sum_{k,\ell} h_{k\ell} v_{k\ell} V_k^2 \ket\psi.
  \label{eq:H_V2_psi_mid}
\end{align}
We thus find that the vector $\V_2\ket\psi$ is an eigenvector of the
interaction Hamiltonian $H_0$ if the matrix $\m v$ of all $v_{k\ell}$
satisfies, for some constant $\e\p{\m v}$ that may depend on $\m v$,
\begin{align}
  \bar{\m h}\oc\m v + \m h\c\m v = \e\p{\m v} \m v,
  &&
  \bar{\m h}
  \equiv \m h - \f12\p{\v h_\d\otimes\v\1 + \v\1\otimes\v h_\d},
  \label{eq:v_cond_2}
\end{align}
where $\bar{\m h}\oc\m v$ denotes an element-wise (Hadamard) product
of $\bar{\m h}$ and $\m v$, with components
$\tilde h_{k\ell} v_{k\ell}$; $\m h\c\m v$ denotes an ordinary matrix
product of $\m h$ and $\m v$, with components
$\sum_q h_{kq} v_{q\ell}$; $\v\1$ is a constant vector of ones; and
$\v h_\d\otimes\v\1$ is a tensor product of $\v h_\d$ and $\v\1$.
Note that the diagonal components of the condition in
\eqref{eq:v_cond_2} automatically enforce that the last sum in
\eqref{eq:H_V2_psi_mid} vanishes.  If the interaction Hamiltonian
$H_0$ is translationally invariant, then all $h_q=h$ and
$\v h_\d\otimes\v\1\oc\m v=\v\1\otimes\v h_\d\oc\m v=h\m v$, so the
condition in \eqref{eq:v_cond_2} becomes
\begin{align}
  \m h\oc\m v + \m h\c\m v = \sp{\e\p{\m v}+h} \m v,
\end{align}
If we construct a perturbation $\V_2\p{\m v}$ with coefficients $\m v$
that satisfy \eqref{eq:v_cond_2}, then
\begin{align}
  H_0 \V_2\p{\m v} \ket\psi
  = \sp{E_0 + 2\e\p{\m v}} \V_2\p{\m v} \ket\psi.
\end{align}
As a curiosity, we note that if $\V_1\p{\v v}\ket\psi$ is an
eigenvector of $H_0$, then the coefficient vector $\v v$ must satisfy
\eqref{eq:D_v_0}, which we used to determine that $\v v$ was mean-zero
if $\Delta\p{\v v}\ne0$; the analogous condition on matrices $\m v$ in
the two-body case is
\begin{align}
  N \e\p{\m v} v = \f12 \sum_{k,\ell} h_{k\ell} v_{k\ell},
\end{align}
where $v\equiv\sum_{p<q}v_{pq}/N$ is the mean value of the
coefficients in $\V_2\p{\m v}$.

%%%%%%%%%%%%%%%%%%%%%%%%%%%%%%%%%%%%%%%%%%%%%%%%%%%%%%%%%%%%%%%%%%%%%%
\section{Finding eigenstates of the interaction Hamiltonian}

Here we discuss the construction of eigenstates of the interaction
Hamiltonian $H_0$.  If a vector $\v v$ and matrix $\m v$ satisfy the
conditions in \eqref{eq:v_cond_1} and \eqref{eq:v_cond_2}, then the
single- and two-body operators $\V_1\p{\v v}$ and $\V_2\p{\m v}$
generate eigenvectors of the interaction Hamiltonian $H_0$ when they
act on any fully symmetric state $\ket\psi\in\M_0$.  We can use this
fact to construct eigenstates of $H_0$ by finding solutions to
\eqref{eq:v_cond_1} and \eqref{eq:v_cond_2}.

The single-body case is straightforward: finding vectors $\v v$ that
satisfy \eqref{eq:v_cond_1} is equivalent to finding eigenvectors of
the matrix $\m h-\m h_\d$, where $\m h_\d\equiv\diag\v h_\d$ is a
diagonal matrix.  The constant $\Delta\p{\v v}$ is then the eigenvalue
of $\m h-\m h_\d$ that corresponds to the eigenvector $\v v$.  The
task of constructing eigenvectors $\V_1\p{\v v}\ket\psi$ of $H_0$ and
determining their energy thus reduces to finding the eigenvectors and
eigenvalues of the $N\times N$ matrix $\m h-\m h_\d$.

The two-body case requires a bit of work to re-cast as an
eigenvector/eigenvalue problem.  We first write the matrix $\m v$ as a
vector,
\begin{align}
  \m v = \sum_{p,q} v_{pq} \op{k}{q}
  \to \v{\m v} \equiv \sum_{p,q} v_{pq} \ket{pq},
\end{align}
and define the matrix
\begin{align}
  \check{\m h}
  \equiv \sum_{k,\ell} \bar h_{k\ell} \op{k\ell}
  + \sum_{k,q,\ell} h_{kq} \op{k\ell}{q\ell},
  \label{eq:h_super_mat}
\end{align}
in terms of which the condition in \eqref{eq:v_cond_2} becomes
\begin{align}
  \check{\m h} \c \v{\m v} = \e\p{\m v} \v{\m v}.
\end{align}
Note that the matrix $\check{\m h}$ in \eqref{eq:h_super_mat} can be
written in the block-diagonal form
\begin{align}
  \check{\m h} = \sum_\ell \m h_\ell \otimes \op{\ell},
  &&
  \m h_\ell
  \equiv \sum_{k,q}\p{\delta_{kq} \bar h_{k\ell} + h_{kq}} \op{k}{q},
\end{align}
where each block $\m h_\ell$ is an $N\times N$ matrix.  We can
diagonalize these blocks independently, and classify their
eigenvectors by corresponding eigenvalues.  ``Total'' eigenvectors
$\v{\m v}$ of $\check{\m h}$ are then essentially concatenations of
eigenvectors from within each block $\m h_\ell$ that all share the
same eigenvalue.  The task of constructing eigenvectors
$\V_2\p{\m v}\ket\psi$ of $H_0$ and determining their energy thus
reduces to finding the eigenvectors and eigenvalues of $N$ matrices
$\m h_\ell$ of size $N\times N$.  If the interaction Hamiltonian $H_0$
is translationally invariant, then $\bar h_{k\ell}=h_{k\ell}-h$, and
$h_{k\ell}$ depends only on the displacement between the spins indexed
by $k$ and $\ell$.  The blocks $\m h_\ell$ then take the form
\begin{align}
  \m h_{\ell} = \m\omega_\ell \c \m h_0 \c \m\omega_\ell^\dag,
  &&
  \m\omega_\ell \equiv \sum_k \op{k+\ell}{k},
  \label{eq:block_shift}
\end{align}
which implies that we need only find the eigenvectors and eigenvalues
of a single block $\m h_0$, after which we can determine the
eigenvectors of any block simply by appropriate applications of the
shift matrix $\m\omega_\ell$.  Note that in dimensions $D>1$, the
relation in \eqref{eq:block_shift} holds for vectors $k,\ell$ that
index lattice sites, i.e.~$k=\p{k_1,k_2,\cdots,k_D}$ and likewise with
$\ell$.

\bibliography{multilevel_spin_notes.bib}

\end{document}
