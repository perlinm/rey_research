\documentclass[nofootinbib,notitlepage,11pt]{revtex4-2}

%%% linking references
\usepackage{hyperref}
\hypersetup{
  breaklinks=true,
  colorlinks=true,
  linkcolor=blue,
  filecolor=magenta,
  urlcolor=cyan,
}

%%% header / footer
\usepackage{fancyhdr} % easier header and footer management
\pagestyle{fancy} % page formatting style
\fancyhf{} % clear all header and footer text
\renewcommand{\headrulewidth}{0pt} % remove horizontal line in header
\usepackage{lastpage} % for referencing last page
\cfoot{\thepage~of \pageref{LastPage}} % "x of y" page labeling


%%% symbols, notations, etc.
\usepackage{physics,braket,bm,amssymb} % physics and math
\renewcommand{\t}{\text} % text in math mode
\newcommand{\f}[2]{\dfrac{#1}{#2}} % shorthand for fractions
\newcommand{\p}[1]{\left(#1\right)} % parenthesis
\renewcommand{\sp}[1]{\left[#1\right]} % square parenthesis
\renewcommand{\set}[1]{\left\{#1\right\}} % curly parenthesis
\newcommand{\bk}{\Braket} % shorthand for braket notation
\renewcommand{\v}{\bm} % bold vectors
\newcommand{\uv}[1]{\bm{\hat{#1}}} % unit vectors
\newcommand{\av}{\vec} % arrow vectors
\renewcommand{\d}{\text{d}} % for infinitesimals
\renewcommand{\c}{\cdot} % inner product

\usepackage{dsfont} % for identity operator
\newcommand{\1}{\mathds{1}}

\newcommand{\up}{\uparrow}
\newcommand{\dn}{\downarrow}

\newcommand{\x}{\text{x}}
\newcommand{\y}{\text{y}}
\newcommand{\z}{\text{z}}

\newcommand{\B}{\mathcal{B}}
\newcommand{\D}{\mathcal{D}}
\newcommand{\E}{\mathcal{E}}
\renewcommand{\H}{\mathcal{H}}
\newcommand{\I}{\mathcal{I}}
\newcommand{\M}{\mathcal{M}}
\newcommand{\N}{\mathcal{N}}
\renewcommand{\O}{\mathcal{O}}
\renewcommand{\P}{\mathcal{P}}
\newcommand{\Q}{\mathcal{Q}}
\newcommand{\R}{\mathcal{R}}
\newcommand{\T}{\mathcal{T}}
\renewcommand{\S}{\mathcal{S}}
\newcommand{\V}{\mathcal{V}}
\newcommand{\X}{\mathcal{X}}
\newcommand{\Z}{\mathcal{Z}}

\newcommand{\EE}{\mathbb{E}}
\renewcommand{\SS}{\mathbb{S}}
\newcommand{\ZZ}{\mathbb{Z}}

\newcommand{\FS}{\text{FS}}

\DeclareMathOperator{\sign}{sign}
\DeclareMathOperator{\cov}{cov}
\let\var\relax
\DeclareMathOperator{\var}{var}

\def\obra#1{\mathinner{({#1}|}}
\def\oket#1{\mathinner{|{#1})}}
\def\obk#1{\mathinner{({#1})}}
\def\oop#1#2{\oket{#1}\!\obra{#2}}

\usepackage[inline]{enumitem} % in-line lists and \setlist{} (below)
\setlist[enumerate,1]{label={(\roman*)}} % default in-line numbering
\setlist{nolistsep} % more compact spacing between environments

%%% text markup
\usepackage{color} % text color
\newcommand{\red}[1]{{\color{red} #1}}

%%%%%%%%%%%%%%%%%%%%%%%%%%%%%%%%%%%%%%%%%%%%%%%%%%%%%%%%%%%%%%%%%%%%%%
\begin{document}
\thispagestyle{fancy}

\title{Perturbing SU($n$)-symmetric interactions}%
\author{Michael A. Perlin}%
\date{\today}

\maketitle

We consider an array of $N$ multilevel spins with non-uniform
SU($n$)-symmetric interactions that can be written in the form
\begin{align}
  H_0 = \sum_{p<q} h_{pq} \Pi_{pq},
  &&
  \Pi_{pq} \equiv \sum_{\mu,\nu} S_{\mu\nu}^{(p)} S_{\nu\mu}^{(q)},
\end{align}
where $S_{\mu\nu}^{(p)}\equiv\op{\mu}{\nu}_p$ flips the state of spin
$p$ to $\ket\mu$ from $\ket\nu$; the operator $\Pi_{pq}$ permutes
spins $p$ and $q$, and the coefficients $h_{pq}$ satisfy
\begin{align}
  h_{pp} = 0
  &&
  h_{pq} \le 0,
  &&
  h_{pq} = h_{qp}.
\end{align}
The ground-state manifold of the interaction Hamiltonian $H_0$
consists of fully symmetric states that are simultaneous $+1$
eigenstates of all permutation operators $\Pi_{pq}$.  The energy of
these states is
\begin{align}
  E_0 \equiv \sum_{p<q} h_{pq} = \f12 \sum_{p,q} h_{pq}
  = \f12 \sum_p h_p = \f12 N h,
\end{align}
where we define
\begin{align}
  h_p \equiv \sum_q h_{pq},
  &&
  h \equiv \EE_p\sp{h_p} = \f1N \sum_{p,q} h_{pq},
  &&
  \EE_p\sp{X_p} \equiv \f1N \sum_p X_p.
\end{align}
We wish to determine the effective dynamics induced on the manifold
$\M_0$ of fully symmetric states by weak perturbations of the form
\begin{align}
  \V_1 \equiv \sum_p v_p V_p,
  &&
  \V_2 \equiv \sum_{p<q} v_{pq} V_p V_q,
  \label{eq:perturbations}
\end{align}
where $V_p$ is a single-body operator on spin $p$, and
$v_{pq}=v_{qp}$.  The effective Hamiltonian $H_W$ induced on the
ground-state manifold $\M_0$ by a perturbation $\V_W$ through second
order in perturbation theory is given by\cite{bravyi2011schrieffer,
  perlin2019effective}
\begin{align}
  H_W = H_W^{(1)} + H_W^{(2)},
  &&
  H_W^{(1)} = \P_0 \V_W \P_0,
  &&
  H_W^{(2)} = - \P_0 \V_W \E \V_W \P_0,
  &&
  \E \equiv \sum_{k>0} \f{\P_k}{\Delta_k},
\end{align}
where $\P_k$ is a projector onto the $k$-th excited eigenspace of the
interaction Hamiltonian $H_0$, with interaction energy $\Delta_k$
above that of fully symmetric manifold $\M_0$.

As an immediate disclaimer, we note that these notes do {\bf not}
solve the general problem of calculating $H_1,H_2$ for arbitrary
perturbations $\V_1,\V_2$.  Instead, we determine conditions under
which the perturbations $\V_1,\V_2$ generate eigenvectors of the
interaction Hamiltonian $H_0$, and calculate $H_1$ in the case that
these conditions are satisfied.

%%%%%%%%%%%%%%%%%%%%%%%%%%%%%%%%%%%%%%%%%%%%%%%%%%%%%%%%%%%%%%%%%%%%%%
\section{Single-body perturbations}

We first compute the effective Hamiltonian induced by the single-body
perturbation $\V_1$.  We therefore expand
\begin{align}
  H_1^{(1)} = \sum_p v_p \P_0 V_p \P_0,
\end{align}
and use the permutational symmetry of the ground-state manifold to
simplify
\begin{align}
  H_1^{(1)} =  \sum_p v_p \P_0 V_0 \P_0 \simeq v V,
  &&
  v \equiv \EE_p\sp{v_p},
  &&
  V = \sum_p V_p,
  \label{eq:H_1_1}
\end{align}
where $\simeq$ denotes equality up to a restriction to the
ground-state manifold $\M_0$.  In order to compute the second order
effective Hamiltonian $\H_1^{(2)}$, we first choose an arbitrary fully
symmetric state $\ket\psi\in\M_0$ and expand
\begin{align}
  H_0 \V_1 \ket\psi
  = \sum_{\substack{k\\p<q}} h_{pq} v_k \Pi_{pq} V_k \ket\psi
  = \sum_{\substack{p<q\\k\notin\set{p,q}}}
  h_{pq} v_k \Pi_{pq} V_k \ket\psi
  + \sum_{\substack{p<q\\k\in\set{p,q}}}
  h_{pq} v_k \Pi_{pq} V_k \ket\psi.
  \label{eq:H_V1_psi_start}
\end{align}
In the case of $k\notin\set{p,q}$, the permutation operator $\Pi_{pq}$
commutes with $V_k$ and annihilates on $\ket\psi$, leaving us the sum
\begin{align}
  \sum_{\substack{p<q\\k\notin\set{p,q}}} h_{pq}
  = \f12 \sum_{\substack{p,q\\k\notin\set{p,q}}} h_{pq}
  = \f12 \sum_{p,q} h_{pq}
  - \f12 \sum_{\substack{p,q\\k\in\set{p,q}}} h_{pq}
  = \f12 N h - \sum_q h_{kq}
  = E_0 - h_k,
\end{align}
where we have used the facts that $h_{pp}=0$ and $h_{pq}=h_{qp}$.  The
terms with $k\notin\set{p,q}$ in \eqref{eq:H_V1_psi_start} are then
\begin{align}
  \sum_{\substack{p<q\\k\notin\set{p,q}}}
  h_{pq} v_k \Pi_{pq} V_k \ket\psi
  = \sum_k \p{E_0 - h_k} v_k V_k \ket\psi
  = E_0 \V_1 \ket\psi - \sum_k h_k v_k V_k \ket\psi.
\end{align}
In the case of $k\in\set{p,q}$, we have
\begin{align}
  \sum_{\substack{p<q\\k\in\set{p,q}}}
  h_{pq} v_k \Pi_{pq} V_k \ket\psi
  = \sum_{k<q} h_{kq} v_k V_q \ket\psi
  + \sum_{p<k} h_{pk} v_k V_p \ket\psi
  = \sum_{k,q} h_{kq} v_k V_q \ket\psi,
\end{align}
so in total,
\begin{align}
  H_0 \V_1 \ket\psi
  = E_0 \V_1 \ket\psi + \sum_{q,k} h_{kq} \p{v_k - v_q} V_q \ket\psi.
  \label{eq:H_V1_psi}
\end{align}
Denoting a vector of the coefficients $v_q$ by $\v v$, a matrix of all
coupling constants $h_{pq}$ by $\v h$, and a diagonal matrix with
$h_q\equiv\sum_k h_{kq}$ on the diagonals by $\tilde{\v h}$, we thus
find that the vector $\V_1\ket\psi$ is an eigenvector of the
interaction Hamiltonian $H_0$ if the coefficients $\v v$ satisfy the
eigenvalue equation
\begin{align}
  \p{\v h - \tilde{\v h}}\c\v v = \Delta\p{\v v} \v v
  \label{eq:v_eig_1}
\end{align}
where $\v x\c\v y$ denotes a standard matrix product of $\v x$ and
$\v y$, and $\Delta\p{\v v}$ is an eigenvalue of
$\v h - \tilde{\v h}$.  If the interaction Hamiltonian $H_0$ is
translationally invariant, then all $h_q=h$, and $\tilde{\v h}=h\,\1$
is proportional to the identity matrix $\1$, so the condition in
\eqref{eq:v_eig_1} becomes
\begin{align}
  \v h \c\v v = \sp{\Delta\p{\v v}-h} \v v.
\end{align}
If we construct a perturbation $\V_1\p{\v v}$ with coefficients $\v v$
that satisfy \eqref{eq:v_eig_1}, then
\begin{align}
  H_0 \V_1\p{\v v} \ket\psi
  = \sp{E_0 + \Delta\p{\v v}} \V_1\p{\v v} \ket\psi,
\end{align}
for any fully symmetric state $\ket\psi\in\M_0$, which for
$\Delta\p{\v v}>0$ implies that
\begin{align}
  \E \V_1\p{\v v} \P_0
  = \sum_{k>0} \f1{\Delta_k} \P_k \V_1\p{\v v} \P_0
  = \f1{\Delta\p{\v v}} \V_1\p{\v v} \P_0,
\end{align}
and in turn
\begin{align}
  H_1^{(2)}
  = -\P_0 \V_1\p{\v v} \E \V_1\p{\v v} \P_0
  = -\f1{\Delta\p{\v v}} \P_0 \V_1\p{\v v}^2 \P_0.
\end{align}
Computing the second-order effective Hamiltonian $H_1^{(2)}$ thus
requires us to simplify the product
\begin{align}
  \P_0 \V_1\p{\v v}^2 \P_0
  = \sum_{p,q} v_p v_q \P_0 V_p V_q \P_0
  = \sum_p v_p^2 \P_0 V_p^2 \P_0
  + \sum_{p\ne q} v_p v_q \P_0 V_p V_q \P_0.
  \label{eq:PVVP_start}
\end{align}
We note that vectors $\v v$ satisfying \eqref{eq:v_eig_1} with
$\Delta\p{\v v}\ne 0$ must be mean-zero, as
\begin{align}
  v = \f1N \sum_q v_q
  = \f{\sum_{q,k} h_{qk} v_k}{N\sp{\Delta\p{\v v}+h}}
  = \f{\sum_k h_k v_k}{N\sp{\Delta\p{\v v}+h}}
  = \f{h v}{\Delta\p{\v v}+h}
  = \f{v}{\Delta\p{\v v}/h+1},
\end{align}
which is only possible if $\Delta\p{\v v}=0$.  It follows that the
first sum on the right of \eqref{eq:PVVP_start} is
\begin{align}
  \sum_p v_p^2 \P_0 V_p^2 \P_0
  = \sum_p v_p^2 \P_0 V_0^2 \P_0
  = \var\p{\v v} \P_0 W \P_0,
  &&
  W \equiv \sum_p V_p^2,
\end{align}
where
\begin{align}
  \var\p{\v x} \equiv \EE_p\sp{\p{x_p-\EE_q\sp{x_q}}^2}
  = \EE_p\sp{x_p^2} - \EE_q\sp{x_q}^2
\end{align}
is the variance of the elements in the vector $\v x$.  Similarly, the
second sum on the right of \eqref{eq:PVVP_start} is
\begin{align}
  \sum_{p\ne q} v_p v_q \P_0 V_p V_q \P_0
  = \sum_{p\ne q} v_p v_q \P_0 V_0 V_1 \P_0,
  \label{eq:PVVP_neq_start}
\end{align}
where if the coefficients in $\v v$ are mean-zero, then
\begin{align}
  \sum_{p\ne q} v_p v_q
  = \sum_p v_p \sum_{q\ne p} v_q
  = \sum_p v_p \p{\sum_q v_q - v_p}
  = - \sum_P v_p^2
  = - N \var\p{\v v}.
\end{align}
To evaluate the operator content of \eqref{eq:PVVP_neq}, we expand
\begin{align}
  \P_0 V^2 \P_0
  = \sum_p \P_0 V_p^2 \P_0
  + \sum_{p\ne q} \P_0 V_p V_q \P_0
  = \P_0 W \P_0 + N \p{N-1} \P_0 V_0 V_1 \P_0,
\end{align}
which implies that
\begin{align}
  \sum_{p\ne q} v_p v_q \P_0 V_p V_q \P_0
  = -\f{\var\p{\v v}}{N-1}\P_0\p{V^2 - W} \P_0.
  \label{eq:PVVP_neq}
\end{align}
Altogether, we have that
\begin{align}
  \P_0 \V_1\p{\v v}^2 \P_0
  = -\f{\var\p{\v v}}{N-1} \P_0 \p{V^2 - N W} \P_0,
\end{align}
and so
\begin{align}
  H_1^{(2)} = \f{\var\p{\v v}}{\p{N-1}\Delta\p{\v v}} \p{V^2 - N W},
\end{align}
where we implicitly restrict the effective Hamiltonian $H_1^{(2)}$ to
the ground-state manifold $\M_0$, and note that the result above holds
only for
\begin{enumerate*}
\item translationally invariant Hamiltonians, and
\item perturbations $\V_1\p{\v v}$ constructed from vectors $\v v$
  that satisfy the eigenvalue equation in \eqref{eq:v_eig_1}.
\end{enumerate*}

%%%%%%%%%%%%%%%%%%%%%%%%%%%%%%%%%%%%%%%%%%%%%%%%%%%%%%%%%%%%%%%%%%%%%%
\section{Two-body perturbations}

The first-order effective Hamiltonian $H_2^{(1)}$ induced on the fully
symmetric manifold $\M_0$ by the two-body perturbation in $\V_2$ in
\eqref{eq:perturbations} is
\begin{align}
  H_2^{(1)}
  = \P_0 \V_2 \P_0
  = \sum_{p<q} v_{pq} \P_0 V_p V_q \P_0
  = \f12 \sum_{p\ne q} v_{pq} \P_0 V_p V_q \P_0
  \simeq -\f12 \f{\var\p{\v v}}{N-1} \p{V^2 - W},
\end{align}
where we made use of the result in \eqref{eq:PVVP_neq}.  In order to
compute the second-order effective Hamiltonian $H_2^{(2)}$, as before
we pick an arbitrary fully symmetric state $\ket\psi\in\M_0$ and
expand
\begin{align}
  H \V_2 \ket\psi
  = \sum_{\substack{k<\ell\\p<q}} v_{k\ell} h_{pq}
  \Pi_{pq} V_k V_\ell \ket\psi.
  \label{eq:H_V2_psi_start}
\end{align}
The permutation operator $\Pi_{pq}$ acts trivially on
$V_k V_\ell \ket\psi$ when $p,q\notin\set{k,\ell}$ or
$\set{p,q}=\set{k,\ell}$, leaving a sum of the form
\begin{align}
  \sum_{\substack{p<q\\p,q\notin\set{k,\ell}}} h_{pq} + h_{k\ell}
  = \sum_{p<q} h_{pq}
  - \sum_{\substack{p<q\\p\in\set{k,\ell}\\q\notin\set{k,\ell}}} h_{pq}
  - \sum_{\substack{p<q\\p\notin\set{k,\ell}\\q\in\set{k,\ell}}} h_{pq}
  = E_0 - \sum_{\substack{p\in\set{k,\ell}\\q\notin\set{k,\ell}}} h_{pq}
  = E_0 - 2 h + 2 h_{k\ell},
\end{align}
where we assumed translational invariance with all $h_p=h$.  The
corresponding terms in \eqref{eq:H_V2_psi_start} are then
\begin{align}
  \sum_{k<\ell}
  \sp{\sum_{\substack{p<q\\p,q\notin\set{k,\ell}}} h_{pq} + h_{k\ell}}
  v_{k\ell} V_k V_\ell \ket\psi
  = \p{E_0-2h} \V_2 \ket\psi
  + \sum_{k,\ell} h_{k\ell} v_{k\ell} V_k V_\ell \ket\psi.
\end{align}
The terms in \eqref{eq:H_V2_psi_start} with only one of
$p,q\in\set{k,\ell}$, meanwhile, reduce to
\begin{align}
  \f12 \sum_{\substack{k,\ell\\q\notin\set{k,\ell}}}
  v_{k\ell} \p{h_{kq} V_q V_\ell + h_{\ell q} V_k V_q} \ket\psi
  = \sum_{k,\ell,q} v_{k\ell} h_{kq} V_q V_\ell \ket\psi
  - \sum_{k,\ell} v_{k\ell} h_{k\ell} V_k^2 \ket\psi.
\end{align}
In total,
\begin{align}
  H_0 \V_2 \ket\psi
  = \p{E_0 - 2h} \V_2 \ket\psi
  + \sum_{k,\ell} \p{h_{k\ell} v_{k\ell} + \sum_q h_{kq} v_{q\ell}}
  V_k V_\ell \ket\psi
  - \sum_{k,\ell} h_{k\ell} v_{k\ell} V_k^2 \ket\psi.
  \label{eq:H_V2_psi_mid}
\end{align}
We thus find that the vector $\V_2\ket\psi$ is an eigenvector of the
interaction Hamiltonian $H_0$ if the matrix $\v v$ of all $v_{k\ell}$
satisfies, for some constant $\varepsilon\p{\v v}$,
\begin{align}
  \v h\circ \v v + \v h\c\v v
  = \varepsilon\p{\v v} \v v,
  \label{eq:v_eig_2}
\end{align}
where $\v h\circ\v v$ denotes an element-wise (Hadamard) product of
the matrices $\v h$ and $\v v$, i.e.~with components
$h_{k\ell} v_{k\ell}$; and $\v h\c\v v$ denotes an ordinary matrix
product, with components $\sum_q h_{kq} v_{q\ell}$.  Note that the
diagonal components of the condition in \eqref{eq:v_eig_2} enforce
that the last sum in \eqref{eq:H_V2_psi_mid} vanishes.  If we
construct a perturbation $\V_2\p{\v v}$ with coefficients $\v v$ that
satisfy \eqref{eq:v_eig_2}, then
\begin{align}
  H_0 \V_2\p{\v v} \ket\psi
  = \sp{E_0 - 2h + 2\varepsilon\p{\v v}} \V_2\p{\v v} \ket\psi.
\end{align}

\bibliography{multilevel_spin_notes.bib}

\end{document}
