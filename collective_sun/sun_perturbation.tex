\documentclass[nofootinbib,notitlepage,11pt]{revtex4-2}

%%% linking references
\usepackage{hyperref}
\hypersetup{
  breaklinks=true,
  colorlinks=true,
  linkcolor=blue,
  filecolor=magenta,
  urlcolor=cyan,
}

%%% header / footer
\usepackage{fancyhdr} % easier header and footer management
\pagestyle{fancy} % page formatting style
\fancyhf{} % clear all header and footer text
\renewcommand{\headrulewidth}{0pt} % remove horizontal line in header
\usepackage{lastpage} % for referencing last page
\cfoot{\thepage~of \pageref{LastPage}} % "x of y" page labeling

%%% symbols, notations, etc.
\usepackage{physics,braket,bm,amssymb} % physics and math
\renewcommand{\t}{\text} % text in math mode
\newcommand{\f}[2]{\dfrac{#1}{#2}} % shorthand for fractions
\newcommand{\p}[1]{\left(#1\right)} % parenthesis
\renewcommand{\sp}[1]{\left[#1\right]} % square parenthesis
\renewcommand{\set}[1]{\left\{#1\right\}} % curly parenthesis
\newcommand{\bk}{\Braket} % shorthand for braket notation

\renewcommand{\c}{\cdot} % inner product

\newcommand{\m}{\bm} % bold symbol
\renewcommand{\v}{\vec} % arrow vector

\usepackage{dsfont} % for identity operator
\newcommand{\1}{\mathds{1}}

\newcommand{\up}{\uparrow}
\newcommand{\dn}{\downarrow}

\renewcommand{\d}{\text{d}}
\newcommand{\x}{\text{x}}
\newcommand{\y}{\text{y}}
\newcommand{\z}{\text{z}}

\newcommand{\A}{\mathcal{A}}
\newcommand{\B}{\mathcal{B}}
\newcommand{\C}{\mathcal{C}}
\newcommand{\D}{\mathcal{D}}
\newcommand{\E}{\mathcal{E}}
\newcommand{\G}{\mathcal{G}}
\newcommand{\I}{\mathcal{I}}
\renewcommand{\L}{\mathcal{L}}
\newcommand{\M}{\mathcal{M}}
\newcommand{\N}{\mathcal{N}}
\renewcommand{\O}{\mathcal{O}}
\renewcommand{\P}{\mathcal{P}}
\renewcommand{\S}{\mathcal{S}}
\newcommand{\T}{\mathcal{T}}

\newcommand{\PP}{\mathbb{P}}
\renewcommand{\SS}{\mathbb{S}}
\newcommand{\TT}{\mathbb{T}}
\newcommand{\ZZ}{\mathbb{Z}}

\newcommand{\PS}{\text{PS}}
\newcommand{\EQPS}{=_{\text{PS}}}
\newcommand{\col}{\underline}

\let\var\relax
\DeclareMathOperator{\var}{var}
\DeclareMathOperator{\cov}{cov}
\DeclareMathOperator{\diag}{diag}
\newcommand{\ul}{\underline}

\usepackage{accents} % for the undertilde
\newcommand{\ut}{\undertilde}

\newcommand{\floor}[1]{\lfloor{#1}\rfloor}
\newcommand{\ceil}[1]{\lceil{#1}\rceil}

\def\obra#1{\mathinner{({#1}|}}
\def\oket#1{\mathinner{|{#1})}}
\def\obk#1{\mathinner{({#1})}}
\def\oop#1#2{\oket{#1}\!\obra{#2}}

\usepackage{mathtools} % for coloneqq

\usepackage[inline]{enumitem} % in-line lists and \setlist{} (below)
\setlist[enumerate,1]{label={(\roman*)}} % default in-line numbering
\setlist{nolistsep} % more compact spacing between environments

%%% figures
\usepackage{graphicx} % for figures
\graphicspath{{./figures/}} % set path for all figures
\usepackage[export]{adjustbox} % for vertical alignment in math
\newcommand{\diagram}[1]
{\,\includegraphics[valign=c]{diagrams/#1.pdf}\,}

% alphanumeric footnotes
\renewcommand*{\thefootnote}{\alph{footnote}}

%%% text markup
\usepackage{color} % text color
\newcommand{\red}[1]{{\color{red} #1}}

%%%%%%%%%%%%%%%%%%%%%%%%%%%%%%%%%%%%%%%%%%%%%%%%%%%%%%%%%%%%%%%%%%%%%%
\begin{document}

\title{SU($n$) spin models near the Heisenberg critical point}%
\author{Michael A. Perlin}%
\date{\today}

\maketitle

\tableofcontents

\section{Introduction}

\red{TODO: write an introduction}

%%%%%%%%%%%%%%%%%%%%%%%%%%%%%%%%%%%%%%%%%%%%%%%%%%%%%%%%%%%%%%%%%%%%%%
\section{The SU($n$) Heisenberg model}

We consider an array of $N$ multilevel spins with SU($n$)-symmetric
interactions that can be written in the form
\begin{align}
  H_0 = \sum_{\substack{p,q\in\ZZ_N\\p<q}} h_{pq} \Pi_{pq},
  &&
  \Pi_{pq} \equiv \sum_{\mu,\nu\in\ZZ_n} s_{\mu\nu}^{(p)} s_{\nu\mu}^{(q)},
  \label{eq:H_0}
\end{align}
where $p,q$ index individual spins; $\ZZ_k$ is the set of integers
modulo $k$; $h_{pq}$ are scalar coupling constants; $\mu,\nu$ index
states in an orthonormal basis for the $n$-dimensional Hilbert space
of a single spin; the operator $s_{\mu\nu}^{(p)}\equiv\op{\mu}{\nu}_p$
flips the state of spin $p$ to $\ket\nu$ from $\ket\mu$; and the
operator $\Pi_{pq}$ permutes the states of spins $p$ and $q$.  The
permutation operator $\Pi_{pq}$ is essentially a multilevel
generalization of the SU(2)-symmetric spin interaction
$\v s_p\c\v s_q$, with $\v s\equiv\p{\v s_\x,s_\y,s_\z}$,
$s_\alpha\equiv\sigma_\alpha/2$, and $\sigma_\alpha$ a single-spin
Pauli operator.

If the coefficients $h_{pq}$ of the interaction Hamiltonian $H_0$ are
all negative, then the ground-state manifold $\M_0$ of the interaction
Hamiltonian $H_0$ consists of permutationally symmetric (PS) states
that are simultaneous $+1$ eigenstates of all permutation operators
$\Pi_{pq}$.  We will assume that all $h_{pq}<0$ throughout these
notes, but comment that this restriction can be relaxed to the
assumption that the initial state of all spins is permutationally
symmetric (e.g.~a spin-polarized state).  We will also assume that the
PS manifold $\M_0$ is gapped away from all orthogonal states by an
interaction energy $\Delta_{\t{gap}}>0$.  Power-law couplings of the
form $h_{pq}\sim-1/\abs{r_p-r_q}^\alpha$ with positions $r_p,r_q$ on a
$D$-dimensional lattice, for example, always yield a non-vanishing
spectral gap $\Delta_{\t{gap}}$ for long-range interactions with
$\alpha\le D$ (see Appendix \ref{sec:spectral_gap}).

The PS manifold $\M_0$ is spanned by a basis of states
$\ket{m}=\ket{m_1,m_2,\cdots,m_n}$ with a definite occupation number
$m_\mu$ of the single-spin state $\mu$, and $\sum_\mu m_\mu=N$.  In
the case of $n=2$, for example, the PS manifold $\M_0$ is precisely
the Dicke manifold\cite{dicke1954coherence} of $N$ qubits, spanned by
Dicke states $\set{\ket{m_0,m_1}}$ for which $m_0+m_1=N$ and the
integer $m_0$ ($m_1$) indicates the total occupation number of qubit
state $\ket{0}$ ($\ket{1}$).  The dimension of the PS manifold $\M_0$
is determined by the number of ways to assign $N$ spins to $n$ states,
which is ${N+n-1 \choose n-1}\sim N^{n-1}$.  The energy of any PS
state $\ket\psi\in\M_0$ with respect to the interaction Hamiltonian
$H_0$ in \eqref{eq:H_0} is
\begin{align}
  E_0 = \sum_{\substack{p,q\in\ZZ_N\\p<q}} h_{pq}.
\end{align}

%%%%%%%%%%%%%%%%%%%%%%%%%%%%%%%%%%%%%%%%%%%%%%%%%%%%%%%%%%%%%%%%%%%%%%
\section{Breaking SU($n$) symmetry (perturbatively)}
\label{eq:pert_theory}

In order to induce non-trivial dynamics for initially PS states
$\ket\psi\in\M_0$, which are otherwise stationary eigenstates of the
interaction Hamiltonian $H_0$ in \eqref{eq:H_0}, we will add to the
Hamiltonian an operator $\hat\O$ of the general form
\begin{align}
  \hat\O \equiv \sum_{\p{X,\m w}\in\O} X_{\m w},
  &&
  X_{\m w} \f1M! \equiv \sum_{k\in\D_N\p{M}} w_k X_k,
  \label{eq:pert}
\end{align}
where $\hat\O$ is constructed from the set $\O\equiv\set{\p{X,\m w}}$
of $M$-spin operators $X$ together with dimension-$M$ (i.e.~$M$-index)
tensors $\m w$; $k=\p{k_1,k_2,\cdots,k_M}$ is a list of of $M$ spins
that $X_k$ acts on; $w_k$ is a scalar entry of $\m w$; and
\begin{align}
  \D_N\p{M} \equiv
  \set{ k \in \ZZ_N^M: \t{all entries of}~k~\t{are distinct} },
  \label{eq:off_diags}
\end{align}
is the strictly off-diagonal part of $\ZZ_N^M$.  The factor of $1/M!$
is included in \eqref{eq:pert} by convention to average over the $M!$
terms addressing any given set of spins.  Due to the permutational
symmetry enforced by the interaction Hamiltonian $H_0$, we will find
it convenient to decompose $\D_N\p{M}=\pi\p{M}\times\C_N\p{M}$, where
$\pi\p{M}$ is the permutation group of $\ZZ_M$, and
\begin{align}
  \C_N\p{M} \equiv \set{ k \subset \ZZ_N: \abs{k} = M },
  \label{eq:choices}
\end{align}
is the set of subsets (``choices'') of $M$ elements from $\ZZ_N$.  We
can then expand
\begin{align}
  X_{\m w} = \f1{M!} \sum_{\sigma\in\pi\p{M}} \sum_{k\in\C_N\p{M}}
  w_{\sigma\p{k}} X_{\sigma\p{k}}.
\end{align}
For simplicity, we will assume throughout this work that $\hat\O$ is
homogeneous in the sense that $M$ is the same for all
$\p{X,\m w}\in\O$, with the understanding that a generalization to
inhomogeneous operators $\hat\O$ is straightforward.

Single-body ($M=1$) operators correspond to classical external fields,
e.g.~$s_\z$ for a magnetic field, and two-body ($M=2$) operators
correspond to two-body interactions that deviate from $H_0$, such as
$s_\z\otimes s_\z$ interactions in an XXZ model\red{[TODO:cite]} near
the Heisenberg critical point.  While we would be hard-pressed to find
a physical implementation of multi-body spin interactions with $M>2$,
considering this case will be necessary to construct low-lying
eigenstates of $H_0$, which will in turn enable simulating the
dynamics induced by non-perturbative additions $\hat\O$ to the SU($n$)
Heisenberg model, with operator norms
$\norm*{\hat\O}\sim\Delta_{\t{gap}}$.

When the operator norm of $\hat\O$ is sufficiently small, namely
$\norm*{\hat\O}<\Delta_{\t{gap}}/2$, we can treat its effect on the PS
manifold $\M_0$ perturbatively.  The effective Hamiltonian
$H_{\t{eff}}$ induced by $\hat\O$ on the ground-state manifold $\M_0$
through second order in perturbation theory
is\cite{bravyi2011schrieffer, perlin2019effective}
\begin{align}
  H_{\t{eff}} = H_{\t{eff}}^{(1)} + H_{\t{eff}}^{(2)},
  &&
  H_{\t{eff}}^{(1)} = \P_0 \hat\O \P_0,
  &&
  H_{\t{eff}}^{(2)} = - \sum_{\Delta\ne0}
  \f1\Delta \P_0 \hat\O \P_\Delta \hat\O \P_0,
  \label{eq:H_eff}
\end{align}
where $\P_\Delta$ is a projector onto the eigenspace $\E_\Delta$ of
the interaction Hamiltonian $H_0$ with interaction energy $\Delta$
above that of PS manifold $\M_0$.  In particular, $\P_0$ is a
projector onto $\M_0$ itself, and $\E_0=\M_0$.  The first order
effective Hamiltonian $H_{\t{eff}}^{(1)}$ is straightforward to
simplify using the permutational symmetry of $\M_0$:
\begin{align}
  H_{\t{eff}}^{(1)} \EQPS \sum_{\p{X,\m w}\in\O} \bar w\,\col{X},
  \label{eq:H_eff_1}
\end{align}
where $\EQPS$ denotes equality under a restriction to $\M_0$; the
coefficient $\bar w$ is the average of all $w_k$; and the collective
operator $\col{X}$ is a permutation-averaged sum of all $X_k$, i.e.
\begin{align}
  \bar w \equiv \f1{\abs{\D_N\p{M}}}
  \sum_{k\in\D_N\p{M}} w_{\sigma\p{k}},
  &&
  \col{X} \equiv \f1{\abs{\pi\p{M}}} \sum_{\sigma\in\pi\p{M}}
  \sum_{k\in\C_N\p{M}} X_{\sigma\p{k}},
\end{align}
with $\abs{\D_N\p{M}}=M!\times{N\choose M}=\prod_{j=0}^{M-1}\p{N-j}$
and $\abs{\pi\p{M}}=M!$.  An immediate consequence of
\eqref{eq:H_eff_1} is the fact that perturbing SU(2)-symmetric
interactions by two-body $s_\z\otimes s_\z$ interactions yields the
one-axis twisting (OAT) Hamiltonian\cite{kitagawa1993squeezed,
  ma2011quantum} $H_{\t{OAT}}=\chi S_\z^2$ at first order in
perturbation theory, with the OAT strength $\chi$ equal to the mean
coefficient of the two-body $s_\z\otimes s_\z$ terms.

\red{[TODO: fix remainder of this section to account for perturbations
  that do not obey permutational symmetry]}

The second order effective Hamiltonian $H_{\t{eff}}^{(2)}$ is not as
straightforward to simplify due to the presence of a projector
$\P_\Delta$ onto the manifold $\E_\Delta$ of states with excitation
energy $\Delta$ (with respect to $H_0$).  For each term $X_{\m w}$ of
$\hat\O$, this projector essentially picks off the part of $X_{\m w}$
that is strictly off-diagonal with respect to the ground- and
excited-state manifolds $\E_0$ and $\E_\Delta$.  To decompose
$X_{\m w}$ into components that generate states of definite excitation
energy when acting on PS states $\M_0$, we pick an arbitrary state
$\ket\psi\in\M_0$ and expand (see Appendix \ref{sec:eigenstates})
\begin{align}
  H_0 X_{\m w} \ket\psi
  = E_0 X_{\m w} \ket\psi + X_{\hat{\m h}\p{\m w}} \ket\psi,
  \label{eq:diagnosis}
\end{align}
where $\hat{\m h}$ is a linear map on the space of dimension-$M$
tensors, determined entirely by the couplings $h_{pq}$ of the
interaction Hamiltonian $H_0$.  The image $\hat{\m h}\p{\m w}$ of
$\m w$ under $\hat{\m h}$ is then itself a dimension-$M$ tensor that
defines the $M$-body operator $X_{\hat{\m h}\p{\m w}}$.  To write out
$\hat{\m h}\p{\m w}$ explicitly, we first define a matrix
$\m h\equiv\sum_{p,q\in\ZZ_N} h_{pq} \op{p}{q}$ of all couplings in
$H_0$ by enforcing that $h_{pq}=h_{qp}$ and $h_{pp}=0$.  We denote a
contraction of $\m h$ with the $a$-th index of $\m w$ by
$\m h \circ_a\m w$, such that the $k$-th entry of $\m h \circ_a\m w$
is
\begin{align}
  \p{\m h \circ_a \m w}_k
  = \sum_{p\in\ZZ_N\setminus k} h_{k_a p} w_{k_{a:p}},
\end{align}
where $k\equiv\p{k_1,k_2,\cdots,k_M}$ is a list of indices;
$k_{a:p}\equiv\p{k_1,k_2,\cdots,k_{a-1},p,k_{a+1},\cdots,k_M}$ is the
same list $k$, but with the $a$-th index $k_a$ replaced by $p$; and we
explicitly exclude terms with $p\in k$ that vanish due to the fact
that $\m h$ and $\m w$ are strictly off-diagonal.  We then define a
dimension-$M$ tensor $\tilde{\m h}$ with entries
\begin{align}
  \tilde h_k \equiv \sum_{\substack{p\in k\\q\in\ZZ_N\setminus k}} h_{pq}
\end{align}
for all $k\in\ZZ_N^M$, and we denote an element-wise (Kronecker)
product of $\tilde{\m h}$ and $\m w$ by $\tilde{\m h} * \m w$, with
entries $\tilde h_k w_k$.  These definitions allow us to expand
\begin{align}
  \hat{\m h}\p{\m w}
  \equiv \sum_{a\in\ZZ_M} \m h \circ_a \m w - \tilde{\m h} * \m w.
  \label{eq:multi_body_eig_op}
\end{align}
Viewed an operator on the vector space of dimension-$M$ tensors, the
map $\hat{\m h}$ is real and symmetric under transposition, and is
therefore diagonalizable.  Diagonalizing this operator yields
eigenvalues $\Delta$ and eigenvectors $\m m_\Delta$ for which
$\hat{\m h}\p{\m m_\Delta}=\Delta\m m_\Delta$.  Finding eigenvalues
and eigenvectors of $\hat{\m h}$ nominally requires diagonalizing a
matrix of dimensions
${N\choose M}\times{N\choose M}\sim N^M\times N^M$, but spatial
symmetries of $H_0$ such as translational invariance or isotropy can
drastically reduce the complexity of this eigenvalue problem.  We
discuss this reduction in Appendix \ref{sec:symmetries}.

Defining $\m w_\Delta$ to be the projection of the tensor $\m w$ onto
the $\Delta$-eigenspace of $\hat{\m h}$, we can expand
\begin{align}
  \m w = \sum_\Delta \m w_\Delta,
  &&
  \hat{\m h}\p{\m w} = \sum_\Delta \Delta \m w_\Delta,
\end{align}
and in turn
\begin{align}
  X_{\m w} = \sum_\Delta X_{\m w_\Delta},
  &&
  X_{\hat{\m h}\p{\m w}} = \sum_\Delta \Delta X_{\m w_\Delta}.
\end{align}
Substituting these expansions into \eqref{eq:diagnosis}, we find
\begin{align}
  H_0 X_{\m w} \ket\psi
  = \sum_\Delta\p{E_0 + \Delta} X_{\m w_\Delta} \ket\psi,
\end{align}
which implies that the action of $X_{\m w_\Delta}$ on a PS state
$\ket\psi\in\M_0$ generates a state of definite interaction energy
$E_0+\Delta$.  It follows that
$\P_\Delta X_{\m w} \P_0 = \P_\Delta X_{\m w_\Delta} \P_0 =
X_{\m{w}_\Delta} \P_0$, so
\begin{align}
  H_{\t{eff}}^{(2)} = -\sum_{\Delta\ne0} \f1\Delta
  \sum_{\p{X,\m v},\p{Y,\m w}\in\O}
  \P_0 X_{\m v_\Delta} Y_{\m w_\Delta} \P_0.
\end{align}
The product $\P_0 X_{\m v_\Delta} Y_{\m w_\Delta} \P_0$ generally
includes $r$-body operators for all positive integers $r\le2M$.  We
provide a general prescription for simplifying such products in
Appendix \ref{sec:operator_product}, and explicitly work out the
single-body ($M=1$) and two-body ($M=2$) cases in Appendices
\ref{sec:single_pair_prod} and \ref{sec:two_pair_prod}.  If $X$ is a
single-body operator ($M=1$), then
\begin{align}
  H_{\t{eff}}^{(2)}
  = \sum_{\Delta\ne0} \sum_{\p{X,\m v},\p{Y,\m w}\in\O}
  \f{\cov\p{\m v_\Delta,\m w_\Delta}}{\p{N-1}\Delta}
  \p{\col{X}\,\col{Y} - N \col{XY}},
  \label{eq:H_eff_2_1}
\end{align}
where
\begin{align}
  \cov\p{\m v,\m w} \equiv
  \f1N \sum_{p\in\ZZ_N} \p{v_p-\bar v}\p{w_p-\bar w}
\end{align}
is the covariance between $\m v$ and $\m w$.  In the SU(2) case,
\eqref{eq:H_eff_2_1} implies that an inhomogeneous magnetic field
($X\propto s_\z$) yields an OAT Hamiltonian at second order
perturbation theory.

%%%%%%%%%%%%%%%%%%%%%%%%%%%%%%%%%%%%%%%%%%%%%%%%%%%%%%%%%%%%%%%%%%%%%%
\section{Beyond ground-state perturbation theory}
\label{eq:excited_states}

In order to find the effective Hamiltonian $H_{\t{eff}}^{(2)}$ induced
by the multi-body operator $\hat\O$ on the PS manifold $\M_0$ at
second order in perturbation theory, we constructed states with
definite excitation energy $\Delta$ with respect to $H_0$.  These
states are generated by the action of an $M$-body operator
$X_{\m w_\Delta}$ on a PS state $\ket\psi\in\M_0$.  We denote the
manifold of all states that can be generated by the action of some
$M$-body operator, but not any $r$-body operator with $r<M$, by
$\M_M$; we will refer to $\M_M$ as the $M$-body excitation manifold.
We then denote the a direct sum of all $\M_r$ for $r\le M$ by
$\ul{\M}_M$, i.e.~$\ul{\M}_M\equiv\bigoplus_{r\le M}\M_r$.  The
manifold $\ul{\M}_M$ consists of all states accessible by the action
of $r$-body operators with $r\le M$ on PS states in $\M_0$.

By construction, the $M$-body operator $X_{\m w}$ generally couples PS
states in $\M_0$ to asymmetric states in $\ul{\M}_M\setminus\M_0$.  If
the coupling between $\M_0$ and its complement is weak compared to the
spectral gap $\Delta_{\t{gap}}$ of $H_0$, then we can treat the effect
of $X_{\m w}$ perturbatively and derive an effective Hamiltonian
$H_{\t{eff}}$ on $\M_0$.  This ground-state perturbation theory
formally breaks down when the operator norm
$\norm{X_{\m w}}\ge\Delta_{\t{gap}}/2$, in which case the operator
$X_{\m w}$ first couples $\M_0$ to $\ul{\M}_M$, then $\ul{\M}_M$ to
$\ul{\M}_{2M}$, and so on.  Even if
$\norm{X_{\m w}}>\Delta_{\t{gap}}/2$, however, the energetic penalty
incurred by $H_0$ from breaking the permutational symmetry of an
initial state $\ket\psi\in\M_0$ suppresses the leakage of population
into manifolds $\M_r$ of increasing $r$.  One might therefore capture
the dynamics of an initial PS state $\ket\psi\in\M_0$ by projecting
$X_{\m w}$ onto $\ul{\M}_{r_{\t{max}}}$ for some $r_{\t{max}}\ge M$,
resulting in a theory analogous to first-order perturbation theory on
$\ul{\M}_{r_{\t{max}}}$.  In the perturbative regime
$\norm{X_{\m w}}<\Delta_{\t{gap}}/2$, such a theory captures all
virtual processes through order $2\floor{r_{\t{max}}/M}+1$ in a formal
perturbation theory on $\M_0$.  In the non-perturbative regime
$\norm{X_{\m w}}\ge\Delta_{\t{gap}}/2$, the error incurred by
truncating the Hilbert space of a spin system past
$\ul{\M}_{r_{\t{max}}}$ is not clear a priori.  Nonetheless,
simulations restricted to $\ul{\M}_{r_{\t{max}}}$ can be benchmarked
by examining the maximum populations within all $\M_r$ for
$r\le r_{\t{max}}$ throughout simulation.  As long as
\begin{enumerate*}
\item these
populations fall off with increasing $r$,
\item and the maximal population within large-$r$ manifolds $\M_r$ is
  small,
\end{enumerate*}
one should expect such simulations to faithfully capture (at least
qualitatively) the dynamical behavior of the spin system under
consideration.

In order to project an operator $X_{\m w}$ onto
$\ul{\M}_{r_{\t{max}}}$, we first need to construct a basis for
$\ul{\M}_{r_{\t{max}}}$, and then compute all matrix elements of
$X_{\m w}$ in the constructed basis.  Our strategy will be to first
identify a basis for $\M_0$, and then find, for each
$r=1,\cdots,r_{\t{max}}$, a suitable set of $r$-body operators
$\set{O}$ and dimension-$r$ tensors $\set{\m u}$ for which the states
$\ket{O,\m u,m}\propto O_{\m u}\ket{m}$ form an orthonormal basis for
$\M_r$.  Matrix elements of $X_{\m w}$ with respect to this basis for
$\ul{\M}_{r_{\t{max}}}$ then take the form
\begin{align}
  \bk{O,\m u,\ell|X_{\m w}|Q,\m v,m}
  = \f{\bk{\ell|O_{\m u}^\dag X_{\m w} Q_{\m v}|m}}
  {\sqrt{\bk{\ell|O_{\m u}^\dag O_{\m u}|\ell}
      \bk{m|Q_{\m v}^\dag Q_{\m v}|m}}}.
  \label{eq:matrix_element}
\end{align}
In order to compute these matrix elements, we simplify products of the
form $\P_0 O_{\m u}^\dag X_{\m w} Q_{\m v} \P_0$ and
$\P_0 O_{\m u}^\dag O_{\m u} \P_0$, with $\P_0$ a projector onto
$\M_0$, in Appendix \ref{sec:operator_product}.

A suitable basis for $\M_0$ is the set of PS states
$\ket{m}=\ket{m_1,m_2,\cdots,m_n}$ with a definite occupation number
$m_\mu$ of the single-spin state $\mu$.




\vspace{3cm}

% For consistency with \eqref{eq:matrix_element}, for each state
% $\ket{m}$ in this basis for $\M_0$ we define
% $\ket{\1,1,m}\equiv\ket{m}$, with $\1$ the identity operator, and we
% define $\1_1\equiv\1$.


\red{[TODO: finish section]}



\newpage
\appendix

%%%%%%%%%%%%%%%%%%%%%%%%%%%%%%%%%%%%%%%%%%%%%%%%%%%%%%%%%%%%%%%%%%%%%%
\section{Existence of a spectral gap with long-range power-law
  interactions}
\label{sec:spectral_gap}

Here we show that power-law SU($n$)-symmetric interactions of the form
in \eqref{eq:H_0} with $h_{pq}=-h/\abs{p-q}^\alpha$ yield a
non-vanishing many-body interaction energy gap when $\alpha\le D$,
where $D$ is the dimension of the lattice.  On a periodic lattice of
$N=L^D$ spins, the spectral gap of the interaction Hamiltonian is (see
Appendix \ref{sec:single_body_eigenstates})
\begin{align}
  \Delta
  = \sum_{d\in\ZZ_L^D} h_{0,d} \sp{\cos\p{d \c k_{\t{SE}}}-1}
  = h \sum_{\substack{d\in\ZZ_L^D\\\abs{d}\ge1}}
  \f{1-\cos\p{d \c k_{\t{SE}}}}{\abs{d}^\alpha},
\end{align}
where the domain of $d$ is defined in terms of integers modulo $L$,
i.e.~$\ZZ_L\simeq\set{0,1,\cdots,L-1}$ with the relation $\simeq$
denoting an association that ignores the cyclic structure of $\ZZ_L$;
$k_{\t{SE}}\in\ZZ_L^D\times2\pi/L$ is a wavenumber for a
singly-excited spin-wave state; and the norm $\abs{d}$ for a vector
$d$ on a periodic lattice is implicitly understood to mean the
smallest Euclidean distance of $d$ from a fixed origin.  In order to
yield the smallest excitation energy $\Delta$, the wavenumber
$k_{\t{SE}}$ should maximize the contribution of the cosine term
above, which is achieved by a wavenumber that minimizes the
oscillations of this term when integrated over the entire lattice.  A
suitable candidate for a minimal wavenumber is
$k_{\t{SE}}=\p{2\pi/L,0,0,\cdots}$, which corresponds to an excitation
energy
\begin{align}
  \Delta = h \sum_{\substack{d\in\ZZ_L^D\\\abs{d}\ge1}}
  \f{1-\cos\p{d_12\pi/L}}{\abs{d}^\alpha}.
\end{align}
Defining $\epsilon\equiv2/L$ and a rescaled domain symmetrized about
$0$, $\SS_\epsilon\simeq\ZZ_L/\epsilon$ with
$\SS_\epsilon\subset\sp{-1,1}$, we substitute $x\simeq\epsilon d$ to
get
\begin{align}
  \Delta
  = h \sum_{\substack{x\in\SS_\epsilon^D\\\abs{x}\ge\epsilon}}
  \f{1-\cos\p{\pi x_1}}{\abs{x/\epsilon}^\alpha}
  = h \epsilon^{\alpha-D} \sum_{\substack{x\in\SS_L^D\\\abs{x}\ge\epsilon}}
  \epsilon^D \f{1-\cos\p{\pi x_1}}{\abs{x}^\alpha}.
\end{align}
As $\epsilon\to0$ the discrete sum over $x$ is well approximated by an
integral that avoids an infinitesimal region about the origin, i.e.
\begin{align}
  \Delta = h \epsilon^{\alpha-D} \I_{D\epsilon},
  &&
  \I_{D\epsilon}
  \equiv \int_{\TT_1^D\setminus\TT_\epsilon^D} \d^Dx\,
  \f{1-\cos\p{\pi x_1}}{\abs{x}^\alpha},
\end{align}
where the integral $\I_{D\epsilon}$ is defined using the interval
$\TT_a\equiv\p{-a,a}$.  The integrand of $\I_{D\epsilon}$ is strictly
positive and well-behaved on the entirety of its domain except for the
origin, where depending on the value of $\alpha$ the integrand may
vanish or diverge as $\abs{x}\to0$.  Together, these facts mean that
\begin{align}
  \I_{D\epsilon} \stackrel{\epsilon\to0}{\sim} \epsilon^{-\gamma},
  &&
  \Delta \stackrel{\epsilon\to0}{\sim} h \epsilon^{\alpha-D-\gamma},
\end{align}
for some $\gamma\ge0$, which implies that gap $\Delta$ is
non-vanishing when $\alpha\le D\le D+\gamma$.

For the sake of completion, we add that in fact the gap $\Delta$
always vanishes in the thermodynamic limit when $\alpha>D$.  To see
this behavior, we note that the asymptotic dependence of the integral
$I_{D\epsilon}$ on $\epsilon$ is determined by the behavior of its
integrand when $\abs{x}\sim\epsilon$, in which case
$1-\cos\p{\pi x_1}\sim x_1^2$, so
\begin{align}
  \I_{D\epsilon}
  \sim \int_{\TT_1^D\setminus\TT_\epsilon^D} \d^Dx\,
  \f{x_1^2}{\abs{x}^\alpha}.
\end{align}
We can then use the fact that $x_1^2\le\abs{x}^2$ and change to
spherical coordinates to find that
\begin{align}
  \I_{D\epsilon} \lesssim
  \int_{\TT_1^D\setminus\TT_\epsilon^D} \d^Dx\,
  \f{\abs{x}^2}{\abs{x}^\alpha}
  \sim \int_\epsilon^1 \d x\, x^{D+1-\alpha}
  \sim
  \begin{cases}
    \epsilon^0 & \alpha < D+2 \\
    \log\p{1/\epsilon} & \alpha = D+2 \\
    \epsilon^{D+2-\alpha} & \alpha > D+2
  \end{cases}.
\end{align}
It follows that the spectral gap
\begin{align}
  \Delta \stackrel{\epsilon\to0}{\lesssim}
  \begin{cases}
    \epsilon^{\alpha-D} & \alpha < D+2 \\
    \epsilon^2 \log\p{1/\epsilon} & \alpha = D + 2 \\
    \epsilon^2 & \alpha > D+2
  \end{cases},
\end{align}
which vanishes as $\epsilon\to0$ for all $\alpha>D$.

%%%%%%%%%%%%%%%%%%%%%%%%%%%%%%%%%%%%%%%%%%%%%%%%%%%%%%%%%%%%%%%%%%%%%%
\section{Constructing excitation energy eigenstates}
\label{sec:eigenstates}

\red{[TODO: fix section to consider the case without permutational
  symmetry of $X_{\m w}$]}

In this section, we simplify the products of the form
$H_0X_{\m w}\ket\psi$ to arrive at the expansion in
\eqref{eq:diagnosis}, which defines conditions under which a
multi-body perturbation $X_{\m w_\Delta}$ generates eigenstates of the
SU($n$)-symmetric interaction Hamiltonian $H_0$ in \eqref{eq:H_0} when
acting on a permutationally symmetric state $\ket\psi\in\M_0$.  Here
$X_{\m w}$ is an $M$-body perturbation defined by
\begin{align}
  X_{\m w}
  \equiv \sum_{k\in\D_N\p{M}} w_k X_k
  = \sum_{k\in\C_N\p{M}} \sum_{\sigma\in\pi\p{M}}
  w_{\sigma\p{k}} X_{\sigma\p{k}},
\end{align}
where $X$ is an $M$-spin operator; $k\equiv\p{k_1,k_2,\cdots,k_M}$ is
a list of $M$ spins that $X_k$ acts on; $\m w$ is a dimension-$M$
(i.e.~$M$-index) tensor with scalar entries $w_k$; $\D_N\p{M}$ is the
strictly off-diagonal part of $\ZZ_N^M$, defined in
\eqref{eq:off_diags}; $\C_N\p{M}$ is the set of all choices of $M$
elements from $\ZZ_N$, defined in \eqref{eq:choices}; and $\pi\p{M}$
is the permutation group of $\ZZ_M$.

%%%%%%%%%%%%%%%%%%%%%%%%%%%%%%%%%%%%%%%%%%%%%%%%%%
\subsection{Single-body excitations}
\label{sec:single_body_eigenstates}

In order to build familiarity with the problem at hand, we first
consider the simple case of a single-body perturbation and simplify
\begin{align}
  H_0 X_{\m w} \ket\psi
  = \sum_{\substack{\p{p,q}\in\C_N\p{2}\\k\in\ZZ_N}}
  h_{pq} w_k \Pi_{pq} X_k \ket\psi.
  \label{eq:single_body_diagnosis_start}
\end{align}
The sum in \eqref{eq:single_body_diagnosis_start} has terms with
$k\in\set{p,q}$, and terms with $k\notin\set{p,q}$.  In the case of
$k\notin\set{p,q}$, the permutation operator $\Pi_{pq}$ commutes with
$X_k$ and annihilates on $\ket\psi$, leaving terms at each fixed $k$
of the form
\begin{align}
  \sum_{\substack{\p{p,q}\in\C_N\p{2}\\k\notin\set{p,q}}} h_{pq}
  = \sum_{\p{p,q}\in\C_N\p{2}} h_{pq}
  - \sum_{\substack{\p{p,q}\in\C_N\p{2}\\k\in\set{p,q}}} h_{pq}
  = E_0 - \f12 \sum_{\substack{p,q\in\ZZ_N\\k\in\set{p,q}}} h_{pq}
  = E_0 - h_k,
\end{align}
where we define $h_{pq}$ for arbitrary $p,q$ by enforcing
$h_{pq}=h_{qp}$ and $h_{pp}=0$; and $h_k \equiv \sum_\ell h_{k\ell}$.
The terms in \eqref{eq:single_body_diagnosis_start} with
$k\notin\set{p,q}$ are then
\begin{align}
  \sum_{\substack{\p{p,q}\in\C_N\p{2}\\k\notin\set{p,q}}}
  h_{pq} w_k \Pi_{pq} X_k \ket\psi
  = \sum_{k\in\ZZ_N} \p{E_0 - h_k} w_k X_k \ket\psi
  = E_0 X_{\m w} \ket\psi - \sum_{k\in\ZZ_N} h_k w_k X_k \ket\psi
\end{align}
while the terms in \eqref{eq:single_body_diagnosis_start} with
$k\in\set{p,q}$ take the form
\begin{align}
  \sum_{\substack{\p{p,q}\in\C_N\p{2}\\k\in\set{p,q}}}
  h_{pq} v_k \Pi_{pq} X_k \ket\psi
  = \sum_{\p{k,q}\in\C_N\p{2}} h_{kq} w_k X_q \ket\psi
  + \sum_{\p{p,k}\in\C_N\p{2}} h_{pk} w_k X_p \ket\psi
  = \sum_{k,q\in\ZZ_N} h_{kq} w_k X_q \ket\psi
\end{align}
which implies
\begin{align}
  H_0 X_{\m w} \ket\psi
  = E_0 X_{\m w} \ket\psi
  + \sum_{p\in\ZZ_N} \p{\sum_{q\in\ZZ_N} h_{pq} w_q - h_p w_p}
  X_p \ket\psi.
  \label{eq:single_body_diagnosis_end}
\end{align}
The action of the single-body perturbation $X_{\m w}$ on a
permutationally symmetric state therefore generates an eigenstate of
$H_0$ with energy $E_0+\Delta$ if the vector
$\v{\m w}\equiv\sum_{k\in\ZZ_N} w_k \ket{k}$ satisfies the eigenvalue
equation
\begin{align}
  \p{\m h - \diag\v h} \c \v{\m w} = \Delta \v{\m w},
  \label{eq:single_body_eig}
\end{align}
where $\m h\equiv\sum_{p,q\in\ZZ_N} h_{pq} \op{p}{q}$ is a matrix of
all couplings $h_{pq}$; the vector
$\v h\equiv\sum_{p,q\in\ZZ_N} h_{pq} \ket{p}$ is the sum of all
columns of $\m h$; and $\diag\v h$ is a matrix with $\v h$ on the
diagonal and zeroes everywhere else.

When the interaction Hamiltonian $H_0$ is translationally invariant,
the single-body eigenvalue problem in \eqref{eq:single_body_eig} is
solvable analytically.  In this case, the couplings $h_{pq}$ depend
only on the separation $\abs{p-q}$, so the eigenvectors of $\m h$ are
plane waves of the form
\begin{align}
  \v{\m w}_k \equiv \sum_{p\in\ZZ_L^D} e^{ip\c k} \ket{p},
\end{align}
where on a $D$-dimensional periodic lattice of $N=L^D$ spins, lattice
sites are indexed by vectors $p\in\ZZ_L^D$, and wavenumbers take on
values $k\in\ZZ_L^D\times2\pi/L$.  The corresponding eigenvalues of
$\m h$ can be determined by expanding
\begin{align}
  \m h\c{\m w}_k
  = \sum_{p,q\in\ZZ_L^D} h_{pq} e^{iq\c k} \ket{p}
  = \sum_{p,d\in\ZZ_L^D} h_{p,p+d} e^{i\p{p+d}\c k} \ket{p}
  = \sum_{d\in\ZZ_L^D} h_{0,d} e^{id\c k} \v{\m w}_k
  = \sum_{d\in\ZZ_L^D} h_{0,d} \cos\p{d\c k} \v{\m w}_k,
\end{align}
where the imaginary terms vanish because $h_{0,d}=h_{0,-d}$.  The
remainder of \eqref{eq:single_body_eig} that we need to sort out is
$\diag\v h$, where all
$h_p=\sum_{q\in\ZZ_L^D}h_{pq}=\sum_{d\in\ZZ_L^D}h_{0,d}$ are equal,
which implies that $\diag\v h=\sum_{d\in\ZZ_L^D}h_{0,d}$ is a scalar.
We thus find that
\begin{align}
  \p{\m h - \diag\v h}\c\v v_k = \Delta_k \v{\m w}_k,
  &&
  \Delta_k \equiv \sum_{d\in\ZZ_L^D} h_{0,d} \sp{\cos\p{d\c k}-1}.
\end{align}

%%%%%%%%%%%%%%%%%%%%%%%%%%%%%%%%%%%%%%%%%%%%%%%%%%
\subsection{Multi-body excitations}
\label{sec:multi_body_eigenstates}

\red{[TODO: fix section to consider the case without permutational
  symmetry of $X_{\m w}$]}

We now consider the full $M$-body case
\begin{align}
  H_0 X_{\m w} \ket\psi
  = \sum_{\substack{\p{p,q}\in\C_N\p{2}\\k\in\C_N\p{M}}}
  h_{pq} w_k \Pi_{pq} X_k \ket\psi.
  \label{eq:diagnosis_start}
\end{align}
The sum in \eqref{eq:diagnosis_start} has terms with $p,q\in k$, terms
with $p,q\notin k$, and terms with only one of $p$ or $q\in k$.  The
permutation operator $\Pi_{pq}$ acts trivially on $X_k\ket\psi$ when
$p,q\in k$ or $p,q\notin k$, so
\begin{align}
  \sum_{\substack{k\in\C_N\p{M}\\\p{p,q}\in\C_N\p{2}\\
      p,q\in k~\t{or}~p,q\notin k}}
  h_{pq} w_k \Pi_{pq} X_k \ket\psi
  = \sum_{k\in\C_N\p{M}}
  \sum_{\substack{\p{p,q}\in\C_N\p{2}\\p,q\in k~\t{or}~p,q\notin k}}
  h_{pq} w_k X_k \ket\psi.
\end{align}
For any fixed $k\in\C_N\p{M}$, we can simplify
\begin{align}
  \sum_{\substack{\p{p,q}\in\C_N\p{2}\\p,q\in k~\t{or}~p,q\notin k}} h_{pq}
  = \sum_{\p{p,q}\in\C_N\p{2}} h_{pq}
  - \sum_{\substack{p\in k\\q\in\ZZ_N\setminus k}} h_{pq}
  = E_0 - \tilde h_k,
  &&
  \tilde h_k \equiv \sum_{\substack{p\in k\\q\in\ZZ_N\setminus k}} h_{pq},
\end{align}
where we split a sum over $\p{p,q}\in\C_N\p{2}$ with the restriction
that $p,q\in k$ or $p,q\notin k$ into an unrestricted sum minus a
remainder with the complementary restriction; and $E_0$ is the
interaction energy of PS states $\ket\psi\in\M_0$.  The terms in
\eqref{eq:diagnosis_start} with $p,q\in k$ or $p,q\notin k$ are then
\begin{align}
  \sum_{k\in\C_N\p{M}} \p{E_0 - \tilde h_k} w_k X_k \ket\psi
  = E_0 X_{\m w} \ket\psi
  - \sum_{k\in\C_N\p{M}} \p{\tilde{\m h} * \m w}_k X_k \ket\psi,
\end{align}
where $\tilde{\m h}$ is a tensor with entries $\tilde h_k$; and
$\tilde{\m h} * \m w$ denotes an element-wise (Kronecker) product of
$\tilde{\m h}$ and $\m w$, with entries $\tilde h_k w_k$.  The
remaining terms in \eqref{eq:diagnosis_start} with only one of
$p\in k$ or $q\in k$ are
\begin{align}
  \sum_{k\in\C_N\p{M}} \sum_{\substack{p\in\ZZ_N\\p\notin k}} \sum_{q\in k}
  h_{pq} w_k \Pi_{pq} X_k \ket\psi
  = \sum_{k\in\C_N\p{M}} \sum_{\substack{p\in\ZZ_N\\p\notin k}}
  \sum_{a\in\ZZ_M} h_{p k_a} w_k \Pi_{p k_a} X_k \ket\psi.
\end{align}
where $k=\p{k_1,k_2,\cdots,k_M}$, with $k_a$ the $a$-th entry in $k$.
Defining $k_{a:p}$ to be a list that is equal to $k$ but with the
$a$-th number $k_a$ replaced by $p$,
$k_{a:p}\equiv\p{k_1,k_2,\cdots,k_{a-1},p,k_{a+1},\cdots,k_M}$, we can
simplify
\begin{align}
  \Pi_{p k_a} X_k \ket\psi
  = \Pi_{p k_a} X_k \Pi_{p k_a} \ket\psi
  = X_{k_{a:p}}\ket\psi,
\end{align}
and switch the labels for $p$ and $k_a$ to find that
\begin{align}
  \sum_{k\in\C_N\p{M}} \sum_{\substack{p\in\ZZ_N\\p\notin k}}
  \sum_{q\in k} h_{pq} w_k  \Pi_{pq} X_k \ket\psi
  = \sum_{k\in\C_N\p{M}} \sum_{\substack{p\in\ZZ_N\\p\notin k}}
  \sum_{a\in\ZZ_M} h_{pk_a} w_{k_{a:p}} X_k \ket\psi.
\end{align}
For each fixed $a$, the sum over $p$ above is essentially a
contraction of the matrix $\m h$ with the $a$-th index of the tensor
$\m w$.  Denoting such a contraction by $\m h\c_a\m w$, we can
therefore write this result as
\begin{align}
  \sum_{k\in\C_N\p{M}} \sum_{a\in\ZZ_M}
  \p{\m h \circ_a \m w}_k X_k \ket\psi,
\end{align}
where, for clarity, the matrix elements of $\m h\circ\m w$ are
\begin{align}
  \p{\m h \circ_a \m w}_k
  \equiv \sum_{\substack{p\in\ZZ_N\\p\notin k}} h_{pk_a} w_{k_{a:p}}
  = \sum_{\substack{p\in\ZZ_N\\p\notin k}}
  h_{pk_a} w_{k_1 k_2 \cdots k_{a-1} p k_{a+1} \cdots k_M}.
\end{align}
Altogether, we thus find that
\begin{align}
  H_0 X_{\m w} \ket\psi
  = E_0 X_{\m w} \ket\psi + \sum_{k\in\C_N\p{M}}
  \hat{\m h}\p{\m w}_k X_k \ket\psi,
  &&
  \hat{\m h}\p{\m w}
  \equiv \sum_{a\in\ZZ_M} \m h \circ_a \m w - \tilde{\m h} * \m w.
  \label{eq:diagnosis_end}
\end{align}
The action of the $M$-body perturbation $X_{\m w}$ on a
permutationally symmetric state therefore generates an eigenstate of
$H_0$ with energy $E_0+\Delta$ if the coefficient tensor $\m w$
satisfies the generalized eigenvalue equation
$\hat{\m h}\p{\m w} = \Delta \m w$, which can be written in the form
\begin{align}
  \hat{\m h} \c \v{\m w} = \Delta \v{\m w},
  \label{eq:multi_body_eig_mat}
\end{align}
where
\begin{align}
  \hat{\m h} \equiv \sum_{k\in\C_N\p{M}}
  \sum_{\substack{p\in\ZZ_N\\p\notin k}}
  \sum_{a\in\ZZ_M} h_{pk_a} \op{k}{k_{a:p}}
  - \sum_{k\in\C_N\p{M}} \tilde h_k \op{k},
  &&
  \v{\m w} \equiv \sum_{k\in\C_N\p{M}} w_k \ket{k}.
\end{align}
If $M=1$, then
$\p{\m h\circ\m w}_k=\sum_{\ell\in\ZZ_N} h_{k\ell} w_\ell$ and
$\tilde h_k = h_k \equiv \sum_{\ell\in\ZZ_N} h_{k\ell}$, so
\begin{align}
  H_0 X_{\m w} \ket\psi
  = E_0 X_{\m w} \ket\psi + \sum_{k\in\ZZ_N}
  \p{\sum_{\ell\in\ZZ_N} h_{k\ell} w_\ell - h_k w_k} X_k \ket\psi,
\end{align}
which is precisely what we found in
\eqref{eq:single_body_diagnosis_end}.  If $M=2$, then
$\p{\m h\circ\m w}_{pq} = \sum_{k\in\ZZ_N}
\p{h_{pk}w_{kq}+h_{qk}w_{pk}}$, so
\begin{align}
  H_0 X_{\m w} \ket\psi
  = E_0 X_{\m w} \ket\psi + \sum_{\p{p,q}\in\C_N\p{2}}
  \sp{\sum_{k\in\ZZ_N}\p{h_{pk} w_{kq} + h_{qk} w_{pk}}
    - \tilde h_{pq} w_{pq}}
  X_k \ket\psi,
\end{align}
where $\tilde h_{pq}=h_p+h_q-2h_{pq}$.

%%%%%%%%%%%%%%%%%%%%%%%%%%%%%%%%%%%%%%%%%%%%%%%%%%
\subsection{Exploiting spatial symmetries}
\label{sec:symmetries}

Solving the eigenvalue problem in \eqref{eq:multi_body_eig_mat} to
find $M$-body perturbations $X_{\m w}$ that generate states of
definite interaction energy nominally requires diagonalizing the
matrix $\hat{\m h}$ with dimensions
${N\choose M}\times{N\choose M}\sim N^M\times N^M$.  If the
interaction Hamiltonian $H_0$ has any spatial symmetries, however,
then we can use these symmetries to reduce the complexity of this
eigenvalue problem.  Translational invariance, for example, can reduce
this problem to that of diagonalizing a matrix with dimensions
$\le{N-1\choose M-1}\times{N-1\choose M-1}\sim N^{M-1}\times N^{M-1}$,
and isotropy can further reduce the problem size by factors that
depend on the lattice geometry.  The complexity of diagonalizing a
matrix with dimensions $d\times d$ is $O\p{d^3}$, so even a reduction
of $d$ by a factor of 2 yields practically significant gains (a factor
of $2^3=8$) in the runtime of diagonalization.  Here, we provide a
general prescription for exploiting the spatial symmetries of $H_0$ to
simplify the eigenvalue problem in \eqref{eq:multi_body_eig_mat}.

Our task is essentially to express the multi-body eigenvalue problem
in \eqref{eq:multi_body_eig_mat} in a basis that respects the spatial
symmetries of $H_0$.  These symmetries essentially break the set of
choices $\C_N\p{M}$ into equivalence classes $\S_M$\footnote{For
  brevity, we will suppress the explicit dependence of $\S_M$ on $N$,
  as well as the dependence of $\S_M$ on any other factors such as
  lattice geometry.}, where elements $s\in\S_M$ are subsets
$s\subset\C_N\p{M}$ that are invariant under the appropriate symmetry
group action.  In the case of a translationally invariant and
isotropic system with two-body perturbations, for example, each
equivalence class $s\in\S_2$ would consist of all pairs
$\p{p,q}\in\C_N\p{2}$ that are a fixed distance $\abs{p-q}=d_s$ apart,
i.e.~$s=\set{\p{p,q}\in\C_N\p{2}:\abs{p-q}=d_s}$.

For simplicity, we first assume that the $M$-body operator $X_{\m w}$
obeys the same spatial symmetries as $H_0$.  For any equivalence class
$s\in\S_M$, we then define a uniform superposition $\ket{s}$ over all
choices $k\in s$, as well as the projector $\P_\S$ onto the space of
uniform superpositions:
\begin{align}
  \ket{s} \equiv \f1{\sqrt{\abs{s}}} \sum_{k\in s} \ket{k},
  &&
  \P_{\S_M} \equiv \sum_{s\in\S_M} \op{s},
\end{align}
where $\abs{s}$ is the number of elements in $s$.  We then expand the
coefficient vector $\v{\m w}$ in this symmetrized basis:
\begin{align}
  \v{\m w} = \sum_{k\in\C_N\p{M}} w_k \ket{k}
  = \sum_{s\in\S_M} w_s \sqrt{\abs{s}} \ket{s},
\end{align}
where $w_s\equiv w_k$ for any $k\in s$.  Finally, we define the
projection of $\hat{\m h}$ onto the symmetric subspace:
\begin{align}
  \hat{\m h}_{\S_M}
  \equiv \P_{\S_M} \hat{\m h} \P_{\S_M}
  = \sum_{s\in\S_M} \ket{s} \sp{
    \sum_{k\in s} \sum_{\substack{p\in\ZZ_N\\p\notin k}} \sum_{a\in\ZZ_M}
    \f{h_{p k_a}}{\sqrt{\abs{s}\abs{\sp{k_{a:p}}}}} \bra{\sp{k_{a:p}}}
    - \tilde h_s \bra{s}},
\end{align}
where $\tilde h_s\equiv \tilde h_k$ for any $k\in s$; $k_a$ is the
$a$-th element of $k$;
$k_{a:p}\equiv\p{k_1,k_2,\cdots,k_{a-1},p,k_{a+1},\cdots,k_M}$ is $k$,
but with the $a$-th number $k_a$ replaced by $p$; and $\sp{k_{a:p}}$
is the equivalence class of $k_{a:p}$.  The reduced multi-body
eigenvalue problem is then
\begin{align}
  \hat{\m h}_{\S_M} \c \v{\m w} = \Delta \v{\m w}.
\end{align}
More generally, even if the couplings $\v{\m w}$ of $X_{\m w}$ do not
obey the spatial symmetries as $H_0$, these symmetries generally endow
$\hat{\m h}$ with a block-diagonal structure. \red{[TODO: finish
  discussion/derivation]}

%%%%%%%%%%%%%%%%%%%%%%%%%%%%%%%%%%%%%%%%%%%%%%%%%%%%%%%%%%%%%%%%%%%%%%
\section{Permutation-symmetrized products of multi-body operators}
\label{sec:operator_product}

\red{[TODO: write introductory text for this section]}

%%%%%%%%%%%%%%%%%%%%%%%%%%%%%%%%%%%%%%%%%%%%%%%%%%
\subsection{The general case: a diagrammatic expansion}

Given a system of $N$ spins, we wish to project a product of
multi-body operators onto the permutationally symmetric (PS) manifold
$\M_0$.  For $p$ multi-body operators, such a product takes the form
\begin{align}
  \m X_{\m w}
  \equiv \prod_{\alpha\in\ZZ_p} \f1{M_\alpha!}
  \sum_{k_\alpha\in\D_N\p{M_\alpha}}
  w_\alpha\p{k_\alpha} X_\alpha\p{k_\alpha}
  = \f1{\m M!} \sum_{k\in\D_N\p{\m M}} \prod_{\alpha\in\ZZ_p}
  w_\alpha\p{k_\alpha} X_\alpha\p{k_\alpha},
  \label{eq:sym_prod_start}
\end{align}
where $\m X\equiv\p{X_1,X_2,\cdots,X_p}$ is a vector of
$M_\alpha$-spin operators $X_\alpha$;
$\m w\equiv\p{w_1,w_2,\cdots,w_p}$ is a vector of dimension-$M_\alpha$
tensors $w_\alpha$; $\m M\equiv\p{M_1,M_2,\cdots,M_p}$ is a vector of
the dimensions $M_\alpha$;
$k_\alpha=\p{k_{\alpha,1},k_{\alpha,2},\cdots,k_{\alpha,M_\alpha}}$ is
a choice of $M_\alpha$ spins that $X_\alpha\p{k_\alpha}$ acts on;
$w_\alpha\p{k_\alpha}$ is a scalar entry of $w_\alpha$;
$\D_N\p{M_\alpha}$ is the strictly off-diagonal part of $\ZZ_N^M$,
defined in \eqref{eq:off_diags};
$\D_N\p{\m M}\equiv\bigotimes_{M_\alpha\in\m M}\D_N\p{M_\alpha}$, such
that each $k\in\D_N\p{\m M}$ has the decomposition
$k=\p{k_1,k_2,\cdots,k_p}$ with $k_\alpha\in\D_N\p{M_\alpha}$; and
$\m M!\equiv\prod_{M_\alpha\in\m M}M_\alpha!$ for shorthand.  For
later convenience, we define
$w_\alpha\p{k_\alpha}=X_\alpha\p{k_\alpha}=0$ for any $k_\alpha$ with
repeated entries.

We now classify terms in \eqref{eq:sym_prod_start} by the numbers
$g_S$ of indices shared by all tensors $w_\alpha$ with
$\alpha\in S\subset\ZZ_p$.  For example, a term of the form
$w_1\p{a,b,c} w_2\p{b,d,e} w_3\p{b,c,d,e}$ with distinct indices
$a,b,c,d,e$ would have
\begin{align}
  g_{\set{1}} = 1,
  &&
  g_{\set{1,2,3}} = 1,
  &&
  g_{\set{1,3}} = 1,
  &&
  g_{\set{2,3}} = 2,
\end{align}
and $g_S=0$ for all other subsets $S\subset\ZZ_3$.  This index
assignment can be associated with the Venn diagram
\begin{align}
  \diagram{example_123},
  \label{eq:venn_diagram}
\end{align}
where $g_S$ is determined the number of dots at the intersection of
circles $\alpha\in S$.  We denote the set of all nonempty subsets of
$\ZZ_p$ by $\PP^*\p{\ZZ_p}$, and denote a choice of $g_S$ for all
$S\in\PP^*\p{\ZZ_p}$ by $g$.  With representations such as
\eqref{eq:venn_diagram} in mind, we will loosely refer to $g$ as
simply a ``diagram''.  We define
$\abs{g}\equiv\sum_{S\in\PP^*\p{\ZZ_p}}g_S$ to be the number of
indices considered in $g$ (or equivalently the number of dots in the
diagram $g$), and we denote the set of all diagrams for
$p=\dim\p{\m M}$ tensors with dimensions $\m M$ by $\G\p{\m M}$.  For
consistency, all diagrams $g\in\G\p{\m M}$ must associate exactly
$M_\alpha$ indices with the dimension-$M_\alpha$ tensor $w_\alpha$,
i.e.
\begin{align}
  \sum_{S\in\PP^*\p{\ZZ_p}\,:\,\alpha\in S} g_S
  = \sum_{R\in\PP^*\p{\ZZ_p\setminus\set{\alpha}}} g_{\set{\alpha}\cup R}
  = M_\alpha
\end{align}
for all $\alpha\in\ZZ_p$.  A classification of the terms in
\eqref{eq:sym_prod_start} by diagrams $g\in\G\p{\m M}$ allows us to
expand
\begin{align}
  \m X_{\m w} \EQPS \sum_{g\in\G\p{\m M}} \m X_{\m w}\p{g},
  \label{eq:sym_prod_diagrams}
\end{align}
where $\EQPS$ denotes equality under a restriction to $\M_0$, and
$\m X_{\m w}\p{g}$ is the sum of all terms in
\eqref{eq:sym_prod_start} that are associated with the diagram $g$.

To find an explicit form of $\m X_{\m w}\p{g}$, we first consider all
distinct ways to assign $\abs{g}$ indices to specific tensor factors
of $\m w$ and $\m X$ in a manner consistent with $g$.  For example, if
$\m X=\p{X_1,X_2}$ with dimensions $\p{M_1,M_2}=\p{2,1}$ and
$\abs{g}=2$ with $g_{\set{1,2}}=g_{\set{2}}=1$, then
$X_1\p{a,b} X_2\p{b}$ and $X_1\p{b,a} X_2\p{b}$ are distinct
assignments of $\abs{g}$ indices to $\m X$ in a manner consistent with
$g$, provided that $a\ne b$.  After assigning $\abs{g}$ indices to
$\m w$ and $\m X$ in a manner consistent with $g$, we need to sum over
all values that these indices can take, which amounts to a sum over
all $k\in\D_N\p{\abs{g}}$.

A specific assignment of indices can be thought of as a choice, for
each of the $M_\alpha$ tensor factors of $w_\alpha$ and $X_\alpha$, of
which dot in the diagram $g$ that tensor factor is associated with
(i.e.~out of the $M_\alpha$ dots within circle $\alpha$).  There are
nominally $M_\alpha!$ such choices for a given $\alpha$, and therefore
$\prod_{\alpha\in\ZZ_p}M_\alpha!=\m M!$ choices total.  The only catch
is that these choices treat all dots in $g$ as distinct, whereas the
$g_S$ dots within region $S$ of $g$ are actually indistinguishable.
The assignment $X_1\p{a,b,c} X_2\p{a,b}$, for example, is the same as
$X_1\p{b,a,c} X_2\p{b,a}$, because $a$ and $b$ are dummy indices.
Treating $g_S$ dummy indices as as distinguishable results in
over-counting distinct index assignments by a factor of $g_S!$.
Denoting the set of all distinct index assignments according to $g$ by
$\L\p{g}$, the number of such assignments is therefore
\begin{align}
  \abs{\L\p{g}} = \m M! \times \S\p{g},
  &&
  \S\p{g} \equiv \sp{\prod_{S\in\PP^*\p{\ZZ_p}} g_S!}^{-1}.
  \label{eq:assignment_num}
\end{align}
Altogether, we have
\begin{align}
  \m X_{\m w}\p{g} \equiv \f{\S\p{g}}{\abs{\L\p{g}}}
  \sum_{\ell\in\L\p{g}} \sum_{k\in\D_N\p{\abs{g}}}
  \m w\p{k,\ell} \m X\p{k,\ell},
\end{align}
where $\m w\p{k,\ell}$ and $\m X\p{k,\ell}$ denote the scalar and
operator acquired by assigning the indices
$k=\p{k_1,k_2,\cdots,k_{\abs{g}}}$ to $\m w$ and $\m X$ according to
$\ell$.  The specific choice of spins $k$ in $\m X\p{k,\ell}$ does not
matter after projection onto the PS manifold $\M_0$, so the dependence
of $\m X\p{k,\ell}$ on $k$ is superfluous, which allows us to write
\begin{align}
  \m X_{\m w}\p{g}
  \EQPS \f{\S\p{g}}{\abs{\L\p{g}}} \sum_{\ell\in\L\p{g}}
  \m w\p{\ell} \m X\p{\ell},
  &&
  \m w\p{\ell} \equiv \sum_{k\in\D_N\p{\abs{g}}} \m w\p{k,\ell}.
  \label{eq:diagram_terms}
\end{align}
If, furthermore, all tensors in $\m w$ (or all operators in $\m X$)
are {\it permutationally symmetric}, or invariant under arbitrary
permutations of their tensor factors, then
\begin{align}
  \m X_{\m w}\p{g} \stackrel{\t{sym}}{=}_{\t{PS}}
  \S\p{g} \m w\p{g} \m X\p{g},
\end{align}
where $\stackrel{\t{sym}}{=}_{\t{PS}}$ denotes equality up to
\begin{enumerate*}
\item permutational symmetry of all $\m w$ or $\m X$, and
\item restriction to the PS manifold $\M_0$; and
\end{enumerate*}
\begin{align}
  \m w\p{g} \equiv \f1{\abs{\L\p{g}}}
  \sum_{\ell\in\L\p{g}} \m w\p{\ell}
  &&
  \m X\p{g} \equiv \f1{\abs{\L\p{g}}}
  \sum_{\ell\in\L\p{g}} \m X\p{\ell}.
  \label{eq:diagram_factors}
\end{align}
In summary, the product of operators $\m X_{\m w}$ in
\eqref{eq:sym_prod_start} admits the diagrammatic expansion in
\eqref{eq:sym_prod_diagrams}, as a sum of $\m X_{\m w}\p{g}$ over all
diagrams $g\in\G\p{\m M}$.  Each term $\m X_{\m w}\p{g}$ is, up to a
symmetry factor $\S\p{g}$, an average of $\m w\p{\ell} \m X\p{\ell}$
over all distinct index assignments $\ell\in\L\p{g}$, as in
\eqref{eq:diagram_terms}.  The scalar $\m w\p{\ell}$ is obtained by
assigning indices $k=\p{k_1,k_2,\cdots,k_{\abs{g}}}$ to the tensors
$\m w$ as prescribed by the $\ell$, and summing over all
$k\in\D_N\p{\abs{g}}$, while the $\abs{g}$-spin operator
$\m X\p{\ell}$ is obtained by contracting the operators in $\m X$ in
accordance with the index assignment $\ell$.  If the tensors $\m w$ or
operators $\m X$ are all permutationally symmetric, then
$\m X_{\m w}\p{g}$ factorizes into $\S\p{g} \m w\p{g} \m X\p{g}$,
where $\m w\p{g}$ and $\m X\p{g}$ are respectively averages of
$\m w\p{\ell}$ and $\m X\p{\ell}$ over all index assignments
$\ell\in\L\p{g}$.

%%%%%%%%%%%%%%%%%%%%%%%%%%%%%%%%%%%%%%%%%%%%%%%%%%
\subsection{Formalizing diagrams and minimizing computational costs}
\label{sec:diagrams}

Diagrams such as \eqref{eq:venn_diagram} play a central role in
simplifying products of multi-body operators, through the expansion in
\eqref{eq:sym_prod_diagrams}.  The purpose of diagrams is to
succinctly represent the scalar content of each term in this
expansion.  Here, we define these diagrams as formal objects that
enable us to systematically perform otherwise intractable
calculations.  We then derive rules to ``reduce'' these diagrams in
such a way as to minimize the computational cost of their numerical
evaluation.

All diagrams consist of $p$ circles, with each circle $\alpha\in\ZZ_p$
labeled by some tensor $w_\alpha$, as in \eqref{eq:venn_diagram} (or
below).  The simplest diagrams have $g_S$ filled dots (``$\bullet$'')
at the intersection of the circles $\alpha\in S\subset\ZZ_p$, and are
simply equal to the scalar $\S\p{g} \m w\p{g}$ defined by
\eqref{eq:assignment_num}, \eqref{eq:diagram_terms}, and
\eqref{eq:diagram_factors}.  If all tensors $w_\alpha$ are
permutationally symmetric, for example, then
\begin{align}
  \diagram{example_123}
  = \f1{2!} \sum_{\p{a,b,c,d,e}\in\D_N\p{5}}
  w_1\p{a,b,c} w_2\p{b,d,e} w_3\p{b,c,d,e}.
  \label{eq:example_123}
\end{align}
When the tensors $\m w$ are not all permutationally symmetric, one can
additionally equip a diagram with a specific index assignment
$\ell\in\L\p{g}$, in which case the diagram is equal to
$\S\p{g} \m w\p{\ell}$.  Diagrams that are not labeled with a specific
index assignment $\ell$ are then an average over all $\ell\in\L\p{g}$.
To keep mathematical expressions simple, we will use symmetric tensors
in all examples in this section, but note that our discussions apply
just as well to the asymmetric case.

%%%%%%%%%%%%%%%%%%%%%%%%%%%%%%%%%%%%%%%%%%%%%%%%%%
\subsubsection{Reducing diagrams}

A diagram such as \eqref{eq:example_123} with $\abs{g}$ dots nominally
takes $O\p{N^{\abs{g}}}$ time to compute due to the interdependence of
all indices.  We wish to ``reduce'' such diagrams and minimize their
computational cost as much as possible.  Our general strategy will be
to replace constrained sums that make all indices interdependent by
unconstrained sums that can be carried out independently, in sequence.
In order to represent different constraints, we need to define
diagrams in which filled dots (``$\bullet$'') may be replaced by empty
dots (``$\circ$'') or crosses (``$\bm\times$'').

Filled dots in a diagram correspond to indices that are constrained to
have different values.  The presence of $F$ filled dots thus
corresponds to a sum over $\D_N\p{F}$.  Every empty dot, meanwhile,
corresponds to an index that is summed over $\ZZ_N$ without additional
constraints, e.g.
\begin{align}
  \diagram{example_o}
  \propto \sum_{\p{a,b,e}\in\D_N\p{3}} \sum_{c,d,e\in\ZZ_N}
  v\p{a,b,c,d,e} w\p{e,f},
\end{align}
where the proportionality is up to an appropriate symmetry factor (in
this case, $1/2!\times1/2!$; clarified in Section
\ref{sec:symmetry_factors}).  Crosses, meanwhile, correspond to
indices that are {\it only} summed over the values of ``filled dot
indices'', e.g.
\begin{align}
  \diagram{example_x}
  = \sum_{\p{a,e}\in\D_N\p{2}} \sum_{b,c\in\set{a,e}}
  \sum_{d\in\ZZ_N} u\p{a,b,c} v\p{b,d,e} w\p{b,c,e}.
\end{align}
Empty dots and crosses allow us to decompose diagrams with a high
computational cost into different diagrams with a lower computational
cost.  We will essentially use two tricks to reduce diagrams:
eliminating filled dots (which will add empty dots and crosses), and
eliminating crosses (which will move around filled dots).  These two
tricks will be applied in sequence until only empty dots remain.

Given a diagram containing only dots (filled or empty), we can
decompose any filled dot into an empty dot and a cross.  For example,
\begin{align}
  \diagram{example_elim}
  = \diagram{example_elim_o}
  - \diagram{example_elim_x}
  \label{eq:example_elim}
\end{align}
where
\begin{align}
  \diagram{example_elim}
  = \sum_{\p{a,b,c}\in\D_N\p{3}} v\p{a,b} w\p{b,c},
\end{align}
\begin{align}
  \diagram{example_elim_o}
  = \sum_{\p{b,c}\in\D_N\p{2}} v\p{\circ,b} w\p{b,c},
  &&
  v\p{\circ,b} \equiv \sum_{a\in\ZZ_N} v\p{a,b},
  \label{eq:example_elim_o}
\end{align}
\begin{align}
  \diagram{example_elim_x}
  = \sum_{\p{b,c}\in\D_N\p{2}} \sum_{a\in\set{b,c}}
  v\p{a,b} w\p{b,c}.
  \label{eq:example_elim_x}
\end{align}
The decomposition in \eqref{eq:example_elim} essentially breaks up a
sum over $a\in\ZZ_N\setminus\set{b,c}$ into a difference of sums over
$a\in\ZZ_N$ and $a\in\set{b,c}$.  The sum over $a\in\ZZ_N$ to compute
$v\p{\circ,b}$ has $O\p{N^2}$ cost, and can be performed prior to
evaluating the diagram in \eqref{eq:example_elim_o}.  The
decomposition in \eqref{eq:example_elim} therefore splits one
$O\p{N^3}$ diagram into two $O\p{N^2}$ diagrams.

After decomposing a filled dot into an empty dot and a cross, the next
step is to immediately eliminate the cross.  To eliminate a cross from
a diagram, we first note that the corresponding index can ignore other
indices on tensors that contain the index, which is to say that we can
replace the sum over $a\in\set{b,c}$ in \eqref{eq:example_elim_x} by a
sum over $a\in\set{c}$.  For each remaining index addressed by a
cross, meanwhile, the cross simply forces the corresponding indices to
be equal, which is equivalent to moving a filled dot into a different
region:
\begin{align}
  \diagram{example_elim_x}
  = \sum_{\p{b,c}\in\D_N\p{2}} \sum_{a\in\set{c}} v\p{a,b} w\p{b,c}
  = \sum_{\p{b,c}\in\D_N\p{2}} v\p{c,b} w\p{b,c}
  = 2 \diagram{example_elim_x_full}.
  \label{eq:example_elim_final}
\end{align}
The added factor of 2 accounts for the fact that the two diagrams in
\eqref{eq:example_elim_final} have different symmetry factors, whereas
re-writing index constraints (i.e.~by eliminating crosses) does not
affect any numerical prefactors on the tensor contraction represented
by a diagram.  We can generally repeat the above process of
sequentially eliminating filled dots and crosses until only empty dots
remain.  The only complication in carrying out this process is that,
in the case of permutationally symmetric tensors, we have to manually
keep track of symmetry factors such as the factor of 2 above.

%%%%%%%%%%%%%%%%%%%%%%%%%%%%%%%%%%%%%%%%%%%%%%%%%%
\subsubsection{Symmetry factors}
\label{sec:symmetry_factors}

As seen in \eqref{eq:example_elim_final}, the process of simplifying
diagrams with permutationally symmetric tensors generally involves
symmetry factors that need to be kept track of by hand.  Here we
discuss the rules governing symmetry factors that appear when
decomposing such diagrams.  In general, eliminating a filled dot from
a region with $F$ filled dots makes a diagram pick up a factor of
$1/F$, e.g.
\begin{align}
  \diagram{example_sym}
  = \f14 \diagram{example_sym_o}
  - \f14 \diagram{example_sym_x},
\end{align}
Similarly, adding an empty dot to a region with $E$ empty dots makes
the diagram pick up a factor of $E+1$, so
\begin{align}
  \diagram{example_sym_o}
  = \f23 \diagram{example_sym_oo}
  - \f13 \diagram{example_sym_ox}.
\end{align}
Finally, eliminating a cross by moving a filled dot from an ``old''
region with $F_{\t{old}}$ filled dots into a ``new'' region with
$F_{\t{new}}$ filled dots picks up an overall factor of
$F_{\t{new}}+1$, so
\begin{align}
  \diagram{example_sym_x} = 2 \diagram{example_sym_x_elim},
\end{align}
where the factor of $1/F_{\t{old}}$ that is acquired due to removing a
dot from the old region is, loosely speaking, canceled out by the
$F_{\t{old}}$ choices of which dot to move.  More precisely,
eliminating a cross and moving a dot corresponds to forcing the index
associated with the cross to be equal to the index associated with the
filled dot (in the ``old'' region), and there are $F_{\t{old}}$ such
indices to choose from.

%%%%%%%%%%%%%%%%%%%%%%%%%%%%%%%%%%%%%%%%%%%%%%%%%%
\subsection{Matrix elements of multi-spin operators}

The process of constructing and simplifying diagrams to compute
coefficients of the diagrammatic operator product expansion in
\eqref{eq:sym_prod_diagrams} is systematic enough to be converted into
an algorithm for execution on a computer.  The last step necessary to
automate the evaluation of \eqref{eq:sym_prod_diagrams} is the
expansion of the multi-spin operators $\m X\p{\ell}$ in a basis for
the PS manifold $\M_0$.  Without loss of generality, we consider a
single $M$-spin operator $X$; our task is essentially to find
coefficients of the expansion
\begin{align}
  \P_0 X \P_0
  = \sum_{a,b\in\B_0} \bk{a|X|b} \op{a}{b},
\end{align}
where $\set{\ket{a}:a\in\B_0}$ is complete bases for $\M_0$, and the
choice of $M$ spins on which $X$ acts is arbitrary.  In a system of
$N$ spins with $n$ states each, a suitable basis for $\M_0$ can be
labeled by the occupation number $a_\mu$ of each single-spin state,
namely $a=\p{a_1,a_2,\cdots,a_n}$ with $\sum_\mu a_\mu=N$.  Written
out explicitly,
\begin{align}
  \ket{a} = \f1{\sqrt{\C\p{a}}}
  \sum_{\substack{\t{distinct}\\\t{permutations}\\\Pi~\t{of}~\tilde a}}
  \Pi \ket{\tilde a},
  &&
  \ket{\tilde a} \equiv \bigotimes_\mu \ket{\mu}^{\otimes a_\mu},
  &&
  \C\p{a} \equiv \f{\p{\sum_\mu a_\mu}!}{\prod_\mu a_\mu!},
\end{align}
where the multinomial coefficient $\C\p{a}$ counts the number of
distinct ways to permute the tensor factors of $\ket{\tilde a}$, and
enforces $\bk{a|a}=1$.  We denote the set of all valid assignments of
$N$ (identical) spins to $n$ (distinguishable) states by $\A_n\p{N}$,
such that the basis $\set{\ket{a}:a\in\A_n\p{N}}$ spans
permutationally symmetric manifold of $N$ spins.  Identifying this
basis allows us to expand
\begin{align}
  \bk{a|X|b}
  = \sum_{\substack{\alpha,\beta\in\A_n\p{M}\\\alpha\le a,\,\beta\le b}}
  \delta_{a-\alpha,b-\beta}
  \sqrt{\f{\C\p{\alpha}\C\p{a-\alpha}\C\p{\beta}\C\p{b-\beta}}
    {\C\p{a}\C\p{b}}}
  \bk{\alpha|X|\beta},
  \label{eq:multi_body_eval}
\end{align}
where the restriction $\alpha\le a$ and difference $a-\alpha$ are
evaluated element-wise, i.e.~$\alpha<a\implies \alpha_\mu\le a_\mu$
and $\p{a-\alpha}_\mu=a_\mu-\alpha_\mu$ for all $\mu$; and
$\delta_{cd}=1$ if $c=d$ and zero otherwise.  We sum over both
$\alpha$ and $\beta$ above merely to keep the expression symmetric
with respect to transposition; in practice, one can simply sum over
$\alpha\in\A_n\p{M}$ and set $\beta=b-a+\alpha$.  Note that, by slight
abuse of notation, the operator $X$ on the left of
\eqref{eq:multi_body_eval} acts on an arbitrary choice of $M$ spins
(out of $N$), whereas the operator $X$ on the right of
\eqref{eq:multi_body_eval} is simply an $M$-spin operator, with matrix
elements $\bk{\alpha|X|\beta}$ evaluated with respect to the $M$-spin
states $\ket\alpha,\ket\beta$.

%%%%%%%%%%%%%%%%%%%%%%%%%%%%%%%%%%%%%%%%%%%%%%%%%%
\subsection{Two single-body operators}
\label{sec:single_pair_prod}

Here we simplify the single-body product
\begin{align}
  X_{\m v} Y_{\m w}
  = \sp{\sum_{p,q\in\ZZ_N} v_p w_q X_p Y_q}
  \EQPS \diagram{single_body_0} X_1 Y_2
  + \diagram{single_body_1} X_1 Y_1,
\end{align}
where $\P_0$ is a projector onto the PS manifold $\M_0$; $p,q$ index
individual spins; $v_p,w_p$ are scalars; and $X,Y$ are single-spin
operators.  This product appears, for example, in the calculation of
the second-order effective Hamiltonian $H_1^{(2)}$ induced on the
permutationally symmetric manifold $\M_0$ by a single-body
perturbation.  Defining
\begin{align}
  \v{\m m} \equiv \sum_{p\in\ZZ_N} m_p \ket{p},
  &&
  \col{m} \equiv \sum_{p\in\ZZ_N} m_p,
\end{align}
for both $\m m\in\set{\m v,\m w}$, we can simplify
\begin{align}
  \diagram{single_body_1} = \v{\m v}\c\v{\m w},
\end{align}
and
\begin{align}
  \diagram{single_body_0}
  = \diagram{single_body_0_o} - \diagram{single_body_0_x}
  = \diagram{single_body_0_oo} - \diagram{single_body_1}
  = \col{v}\,\col{w} - \v{\m v} \c\v{\m w},
\end{align}
so
\begin{align}
  \sum_{p,q\in\ZZ_N} v_p w_q X_p Y_q
  \EQPS \col{v}\,\col{w} X_1 Y_2
  - \v{\m v}\c\v{\m w} \p{X_1 Y_2 - X_1 Y_1},
\end{align}
In order to write this result in terms of collective operators
$\col{Z} \equiv \sum_{p\in\ZZ_N} Z_p$, we expand
\begin{align}
  \col{X Y} \EQPS N X_1 Y_1,
  &&
  \col{X}\,\col{Y} \EQPS \col{XY} + N\p{N-1} X_1 Y_2,
\end{align}
which implies that
\begin{align}
  \sum_{p,q\in\ZZ_N} v_p w_q X_p Y_q
  \EQPS \f{\col{v}\,\col{w}}{N\p{N-1}}
  \p{\col{X}\,\col{Y} - \col{XY}}
  - \f{\v{\m v}\c\v{\m w}}{N\p{N-1}}
  \p{\col{X}\,\col{Y} - N\col{XY}}.
\end{align}

%%%%%%%%%%%%%%%%%%%%%%%%%%%%%%%%%%%%%%%%%%%%%%%%%%
\subsection{Two two-body operators}
\label{sec:two_pair_prod}

Here we simplify the two-body product
\begin{multline}
  X_{\m v} Y_{\m w}
  = \sp{\sum_{\p{k,\ell},\p{p,q}\in\D_N\p{2}}
    v_{k\ell} w_{pq} X_{k\ell} Y_{pq}} \\
  \EQPS \diagram{two_body_0} X_{1,2} Y_{3,4} + \diagram{two_body_1}
  X_{1,2} Y_{2,3} + \diagram{two_body_2} X_{1,2} Y_{1,2},
\end{multline}
where $\P_0$ is a projector onto the PS manifold $\M_0$; $k,\ell,p,q$
index individual spins that the two-body operators $X_{k\ell},Y_{pq}$
act on; and $v_{k\ell},w_{rs}$ are scalar coefficients with
$m_{pq}=m_{qp}$ and $m_{pp}=0$ for each of $\m m\in\set{\m v,\m w}$.
This product appears, for example, in the calculation of the
second-order effective Hamiltonian $H_2^{(2)}$ induced on the PS
manifold $\M_0$ by a permutationally symmetric two-body perturbation.
Defining
\begin{align}
  \v{\m m} \equiv \sum_{\p{p,q}\in\C_N\p{2}} m_{pq} \ket{p,q},
  &&
  \v m \equiv \sum_{p,q\in\ZZ_N} m_{pq} \ket{p},
  &&
  \col{m} \equiv \sum_{\p{p,q}\in\C_N\p{2}} m_{pq},
\end{align}
for both $\m m\in\set{\m v,\m w}$, we can write
\begin{align}
  \diagram{two_body_2} = \v{\m v} \c\v{\m w},
\end{align}
and
\begin{align}
  \diagram{two_body_1}
  &= \sp{\diagram{two_body_1_o}} - \sp{\diagram{two_body_1_x}} \\
  &= \sp{\diagram{two_body_1_oo} - \diagram{two_body_1_ox}}
  - \sp{2\diagram{two_body_2}} \label{eq:mid_zero} \\
  &= \v v \c \v w - 2 \v{\m v} \c \v{\m w},
\end{align}
where the middle diagram in \eqref{eq:mid_zero} contains an empty sum
for the cross, and is therefore equal to zero.  Finally, with a bit of
more work (or a computer) we can work out that
\begin{align}
  \diagram{two_body_0}
  = \diagram{two_body_0_oooo} - \diagram{two_body_1_ooo}
  + \diagram{two_body_2_oo}
  = \col{v}\,\col{w} - \v v\c\v w + \v{\m v}\c\v{\m w}.
\end{align}
Altogether,
\begin{align}
  X_{\m v} Y_{\m w}
  \EQPS \p{\col{v}\,\col{w} - \v v\c\v w + \v{\m v}\c\v{\m w}}
  X_{1,2} Y_{3,4}
  + \p{\v v\c\v w - 2 \v{\m v} \c \v{\m w}} X_{1,2} Y_{2,3}
  + \v{\m v} \c \v{\m w}\, X_{1,2} Y_{1,2}.
\end{align}
In the special case that $X=Y=Z\otimes Z$ with a single-spin operator
$Z$ for which $Z^2=1$, we can further simplify
\begin{align}
  \sum_{\p{k,\ell},\p{p,q}\in\D_N\p{2}} v_{k\ell} w_{pq}
  Z_k Z_\ell Z_p Z_q
  \EQPS \p{\col{v}\,\col{w} - \v v\c\v w + \v{\m v}\c\v{\m w}}
  Z^{\otimes 4}
  + \p{\v v\c\v w - \v{\m v} \c \v{\m w}} Z^{\otimes 2}
  + \v{\m v}\c\v{\m w},
\end{align}
which we can write in terms of collective operators
$\col{Z}\equiv\sum_{p\in\ZZ_N}Z_p$ as
\begin{multline}
  \sum_{\p{k,\ell},\p{p,q}\in\C_N\p{2}} v_{k\ell} w_{pq}
  Z_k Z_\ell Z_p Z_q \\
  \EQPS \f{\col{v}\,\col{w} - \v v\c\v w + \v{\m v}\c\v{\m w}}{N!_4}
  \sp{\col{Z}^4 - 2\p{3N-4} \col{Z}^2}
  + \f{\v v\c\v w - 2 \v{\m v} \c \v{\m w}}{N!_2}\, \col{Z}^2 \\
  + \f{3\,\col{v}\,\col{w} - N \v v\c\v w + N\p{N-2}\v{\m v}\c\v{\m
      w}}{\p{N-1}\p{N-3}},
\end{multline}
where
\begin{align}
  N!_k \equiv \prod_{j=0}^{k-1} \p{N-j} = \f{N!}{\p{N-k}!}
\end{align}
denotes a falling factorial.

%%%%%%%%%%%%%%%%%%%%%%%%%%%%%%%%%%%%%%%%%%%%%%%%%%
\subsection{Three two-body Ising-like operators}

Here we simplify the product
\begin{align}
  Z_{\m u\m v\m w}
  \equiv \sum_{\p{k,\ell},\p{p,q},\p{r,s}\in\C_N\p{2}}
    u_{k\ell} v_{pq} w_{rs} Z_k Z_\ell Z_p Z_q Z_r Z_s,
  &&
  Z^2 = 1,
\end{align}
projected onto the PS manifold $\M_0$, where $k,\ell,p,q,r,s$ index
individual spins; $Z$ is a single-spin operator; and
$u_{k\ell},v_{pq},w_{rs}$ are scalar coefficients with $m_{pq}=m_{qp}$
and $m_{pp}=0$ for each of $\m m\in\set{\m u,\m v,\m w}$.  This
product appears, for example, in the projection of Ising-type
($s_\z\otimes s_\z$) interactions onto the two-body excitation
manifold $\M_2$.  Collecting terms according to the number of $Z$
operators that remain after accounting for $Z^2=1$, we find that
\begin{align}
  Z_{\m u\m v\m w} \EQPS A_6 Z^{\otimes 6} + A_4 Z^{\otimes 4}
  + A_2 Z^{\otimes 2} + A_0,
  \label{eq:triple_multi}
\end{align}
with
\begin{align}
  A_6 \equiv \diagram{triple_0},
  &&
  A_4 \equiv \diagram{triple_01} + \diagram{triple_1},
  &&
  A_0 \equiv \diagram{triple_0111},
  \label{eq:triple_64}
\end{align}
\begin{align}
  A_2 \equiv \diagram{triple_011} + \diagram{triple_02}
  + \diagram{triple_11} + \diagram{triple_2},
  \label{eq:triple_2}
\end{align}
where a diagram with unlabeled circles denotes a sum over all distinct
label assignments, e.g.
\begin{align}
  \diagram{triple_1} \equiv \diagram{triple_1_uvw},
  \label{eq:triple_uvw}
\end{align}
\begin{align}
  \diagram{triple_011}
  \equiv \diagram{triple_011_uvw}
  + \diagram{triple_011_vwu} + \diagram{triple_011_wuv}.
  \label{eq:triple_uvw_3}
\end{align}
The last diagram in each of $A_6,A_4,A_2,A_0$ as written in
\eqref{eq:triple_64}, \eqref{eq:triple_2} thus has only one assignment
of labels, as in \eqref{eq:triple_uvw}, while all other unlabeled
diagrams have three assignments, as in \eqref{eq:triple_uvw_3}.

Having identified diagrammatic representations of the coefficients
$A_6,A_4,A_2,A_0$, we now need to simplify all diagrams to reduce
their computational complexity, as discussed in Appendix
\ref{sec:diagrams}.  The process of eliminating filled dots and
crosses from a diagram is algorithmic enough to execute on a computer,
doing which we find
\begin{align}
  A_6 = D_6 - D_4 + D_2 - D_0,
  &&
  A_4 = D_4 - 2 D_2 + 3 D_0,
  &&
  A_2 = D_2 - 3 D_0,
  &&
  A_0 \equiv D_0,
\end{align}
where
\begin{align}
  D_6 \equiv \diagram{triple_0_o},
  &&
  D_4 \equiv \diagram{triple_01_o} - 2 \diagram{triple_1_o},
  &&
  D_0 \equiv \diagram{triple_0111_o},
\end{align}
\begin{align}
  D_2 \equiv \diagram{triple_011_o}
  + \diagram{triple_02_o}
  - 2 \diagram{triple_11_o}
  + 4 \diagram{triple_2_o}.
\end{align}
Defining, for each of $\m m\in\set{\m u,\m v,\m w}$,
\begin{align}
  \v{\m m} \equiv \sum_{\p{p,q}\in\C_N\p{2}} m_{pq} \ket{\set{p,q}},
  &&
  \v m \equiv \sum_{p\in\ZZ_N} m_p \ket{p},
  &&
  m_p \equiv \sum_{q\in\ZZ_N} m_{pq},
  &&
  \col{m} \equiv \sum_{\p{p,q}\in\C_N\p{2}} m_{pq},
\end{align}
we can expand
\begin{align}
  D_6 = \col{u}\,\col{v}\,\col{w},
  &&
  D_4 = \col{u}\,\v v\c\v w + \col{v}\,\v w\c\v u
  + \col{w}\,\v u\c\v v - 2 \sum_{p\in\ZZ_N} u_p v_p w_p,
  &&
  D_0 = \tr\p{\m u \c \m v \c \m w},
\end{align}
\begin{multline}
  D_2 = \v u \c\m v\c\v w + \v v \c\m w\c\v u + \v w \c\m u\c\v v
  + \col{u}\,\v{\m v}\c\v{\m w} + \col{v}\,\v{\m w}\c\v{\m u}
  + \col{w}\,\v{\m u}\c\v{\m v} \\
  - 2 \sum_{p,q\in\ZZ_N} \p{u_p v_{pq} w_{pq}
    + v_p w_{pq} u_{pq} + w_p u_{pq} v_{pq}}
  + 4 \sum_{\p{p,q}\in\C_N\p{2}} u_{pq} v_{pq} w_{pq}.
\end{multline}
Finally, we can also write the product $Z_{\m u\m v\m w}$ in
\eqref{eq:triple_multi} in terms of the collective operator
$\col{Z} \equiv \sum_{p\in\ZZ_N} Z_p$:
\begin{align}
  Z_{\m u\m v\m w} \EQPS
  \tilde A_6 \col{Z}^6 + \tilde A_4 \col{Z}^4
  + \tilde A_2 \col{Z}^2 + \tilde A_0,
  \label{eq:triple_col}
\end{align}
where
\begin{align}
  \tilde A_6 \equiv \f{A_6}{N!_6},
  &&
  \tilde A_4 \equiv \f{A_4}{N!_4} - 5\p{3N-8} \f{A_6}{N!_6},
\end{align}
\begin{align}
  \tilde A_2 \equiv \f{A_2}{N!_2} - 2\p{3N-4} \f{A_4}{N!_4}
  + \p{45N^2-210N+184} \f{A_6}{N!_6},
\end{align}
\begin{align}
  \tilde A_0 \equiv A_0 - \f{A_2}{N-1}
  + \f{3A_4}{\p{N-1}\p{N-3}}
  - \f{15A_6}{\p{N-1}\p{N-3}\p{N-5}}.
\end{align}

\bibliography{\jobname.bib}

\end{document}
