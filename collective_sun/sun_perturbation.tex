\documentclass[nofootinbib,notitlepage,11pt]{revtex4-2}

%%% linking references
\usepackage{hyperref}
\hypersetup{
  breaklinks=true,
  colorlinks=true,
  linkcolor=blue,
  filecolor=magenta,
  urlcolor=cyan,
}

%%% header / footer
\usepackage{fancyhdr} % easier header and footer management
\pagestyle{fancy} % page formatting style
\fancyhf{} % clear all header and footer text
\renewcommand{\headrulewidth}{0pt} % remove horizontal line in header
\usepackage{lastpage} % for referencing last page
\cfoot{\thepage~of \pageref{LastPage}} % "x of y" page labeling

%%% symbols, notations, etc.
\usepackage{physics,braket,bm,amssymb} % physics and math
\renewcommand{\t}{\text} % text in math mode
\newcommand{\f}[2]{\dfrac{#1}{#2}} % shorthand for fractions
\newcommand{\p}[1]{\left(#1\right)} % parenthesis
\renewcommand{\sp}[1]{\left[#1\right]} % square parenthesis
\renewcommand{\set}[1]{\left\{#1\right\}} % curly parenthesis
\newcommand{\bk}{\Braket} % shorthand for braket notation

\renewcommand{\c}{\cdot} % inner product

\newcommand{\m}{\bm} % bold symbol
\renewcommand{\v}{\vec} % arrow vector

\usepackage{dsfont} % for identity operator
\newcommand{\1}{\mathds{1}}

\newcommand{\up}{\uparrow}
\newcommand{\dn}{\downarrow}

\renewcommand{\d}{\text{d}}
\newcommand{\x}{\text{x}}
\newcommand{\y}{\text{y}}
\newcommand{\z}{\text{z}}

\newcommand{\C}{\mathcal{C}}
\newcommand{\E}{\mathcal{E}}
\newcommand{\G}{\mathcal{G}}
\renewcommand{\H}{\mathcal{H}}
\newcommand{\I}{\mathcal{I}}
\renewcommand{\L}{\mathcal{L}}
\newcommand{\M}{\mathcal{M}}
\renewcommand{\O}{\mathcal{O}}
\renewcommand{\P}{\mathcal{P}}
\renewcommand{\S}{\mathcal{S}}
\newcommand{\V}{\mathcal{V}}
\newcommand{\X}{\mathcal{X}}

\newcommand{\PP}{\mathbb{P}}
\renewcommand{\SS}{\mathbb{S}}
\newcommand{\TT}{\mathbb{T}}
\newcommand{\ZZ}{\mathbb{Z}}

\newcommand{\PS}{\text{PS}}
\newcommand{\EQPS}{=_{\text{PS}}}
\newcommand{\col}{\underline}

\DeclareMathOperator{\diag}{diag}
\newcommand{\ul}{\underline}

\usepackage{accents}
\newcommand{\ut}{\undertilde}

\newcommand{\floor}[1]{\lfloor{#1}\rfloor}
\newcommand{\ceil}[1]{\lceil{#1}\rceil}

\def\obra#1{\mathinner{({#1}|}}
\def\oket#1{\mathinner{|{#1})}}
\def\obk#1{\mathinner{({#1})}}
\def\oop#1#2{\oket{#1}\!\obra{#2}}

\usepackage[inline]{enumitem} % in-line lists and \setlist{} (below)
\setlist[enumerate,1]{label={(\roman*)}} % default in-line numbering
\setlist{nolistsep} % more compact spacing between environments

%%% figures
\usepackage{graphicx} % for figures
\graphicspath{{./figures/}} % set path for all figures
\usepackage[export]{adjustbox} % for vertical alignment in math
\newcommand{\diagram}[1]
{\,\includegraphics[valign=c]{diagrams/#1.pdf}\,}

%%% text markup
\usepackage{color} % text color
\newcommand{\red}[1]{{\color{red} #1}}

%%%%%%%%%%%%%%%%%%%%%%%%%%%%%%%%%%%%%%%%%%%%%%%%%%%%%%%%%%%%%%%%%%%%%%
\begin{document}

\title{Perturbing SU($n$)-symmetric interactions}%
\author{Michael A. Perlin}%
\date{\today}

\maketitle

\tableofcontents

\section{Introduction}

We consider an array of $N$ multilevel spins with SU($n$)-symmetric
interactions that can be written in the form
\begin{align}
  H_0 = \sum_{p<q} h_{pq} \Pi_{pq},
  &&
  \Pi_{pq} \equiv \sum_{\mu,\nu} S_{\mu\nu}^{(p)} S_{\nu\mu}^{(q)},
  \label{eq:ints}
\end{align}
where $p,q$ index individual spins; $\mu,\nu$ index states in an
orthonormal basis for the $n$-dimensional Hilbert space $\H_n$ of a
single spin; $h_{pq}$ are scalar coefficients; the operator
$S_{\mu\nu}^{(p)}\equiv\op{\mu}{\nu}_p$ flips the state of spin $p$ to
$\ket\mu$ from $\ket\nu$; and the operator $\Pi_{pq}$ permutes the
states of spins $p$ and $q$.  The permutation operator $\Pi_{pq}$ is a
multilevel generalization of the SU(2)-symmetric spin interaction
$\v S_p\c\v S_q$ for $\v S=\p{s_\x,s_\y,s_\z}$, with $s_\alpha$ an
individual spin operator proportional to the Pauli matrix
$\sigma_\alpha$.

We wish to determine the effective dynamics induced on the
ground-state manifold $\M_0$ of $H_0$ by weak single- and two-body
perturbations of the form
\begin{align}
  \V_1 \equiv \sum_X \sum_p v_{Xp} X_p = \sum_X \v v_X\c\v X,
  &&
  \V_2 \equiv \sum_O \sum_{p<q} w_{Opq} O_{pq}
  = \sum_O \v{\m w}_O\c \v{\m O}
  \label{eq:perturbations}
\end{align}
where $p,q$ index individual spins; $v_{Xp}$ and $w_{Opq}$ are real
numbers; $X$ is a trace-zero single-spin operator on $\H_n$; $O$ is a
trace-zero two-spin operator on $\H_n\otimes\H_n$ that obeys pair-wise
permutational symmetry, i.e.~$O_{pq}=O_{qp}$; $\v v_X$ and $\v X$ are
$N$-component vectors defined by
\begin{align}
  \v v_X \equiv \sum_p v_{Xp} \ket{p},
  &&
  \v X \equiv \sum_p X_p \ket{p};
\end{align}
and finally $\v{\m w_O}$ and $\v{\m O}$ are ${N \choose 2}$-component
vectors defined by
\begin{align}
  \v{\m w}_O \equiv \sum_{p<q} w_{Opq} \ket{\set{p,q}},
  &&
  \v{\m O} \equiv \sum_{p<q} O_{pq} \ket{\set{p,q}},
\end{align}
where the basis vectors $\ket{\set{p,q}}$ are labeled by a choice of
two distinct spins $p,q$.

If the coefficients $h_{pq}$ of the interaction Hamiltonian $H_0$ are
all negative, then the ground-state manifold $\M_0$ of the interaction
Hamiltonian $H_0$ consists of permutationally symmetric states that
are simultaneous $+1$ eigenstates of all permutation operators
$\Pi_{pq}$.  We will adopt the restriction that all $h_{pq}<0$
throughout these notes, but note that this restriction can be relaxed
to the assumption that the initial state of the spins is
permutationally symmetric (e.g.~a spin-polarized state).  We will also
assume throughout these notes that the permutationally symmetric
manifold $\M_0$ is gapped away from all other states by an interaction
energy difference that is large compared to any coupling between
$\M_0$ and its orthogonal complement.  Power-law couplings of the form
$h_{pq}=-h/\abs{p-q}^\alpha$ on a $D$-dimensional lattice, for
example, yield a non-vanishing spectral gap in the thermodynamic limit
when $\alpha\le D$ (see Appendix \ref{sec:gap}).

The effective Hamiltonian $H_M$ induced on the ground-state manifold
$\M_0$ by an $M$-body perturbation $\V_M$ through second order in
perturbation theory is given by\cite{bravyi2011schrieffer,
  perlin2019effective}
\begin{align}
  H_M = H_M^{(1)} + H_M^{(2)},
  &&
  H_M^{(1)} = \P_0 \V_M \P_0,
  &&
  H_M^{(2)} = - \P_0 \V_M \E \V_M \P_0,
  &&
  \E \equiv \sum_{\Delta\ne0} \f{\P_\Delta}{\Delta},
\end{align}
where $\P_\Delta$ is a projector onto the eigenspace of the
interaction Hamiltonian $H_0$, with interaction energy $\Delta$ above
that of permutationally symmetric manifold $\M_0$; that is, the
interaction energy of states in the image of $\P_\Delta$ is
$E_0+\Delta$, where
\begin{align}
  E_0 = \sum_{p<q} h_{pq}
\end{align}
is the interaction energy of permutationally symmetric states
$\ket\psi\in\M_0$.

%%%%%%%%%%%%%%%%%%%%%%%%%%%%%%%%%%%%%%%%%%%%%%%%%%%%%%%%%%%%%%%%%%%%%%
\section{Single-body perturbations}
\label{sec:single_body_pert}

To compute the effective Hamiltonian induced by the single-body
perturbation $\V_1$, we use the permutational symmetry of the
ground-state manifold to simplify
\begin{align}
  H_1^{(1)} = \sum_{X,p} v_{Xp} \P_0 X_p \P_0
  \EQPS \sum_{X,p} v_{Xp} X_1
  \EQPS \sum_X \bar v_X \col{X},
  \label{eq:H_1_1}
\end{align}
where $\EQPS$ denotes equality under a restriction to the fully
symmetric manifold $\M_0$, and
\begin{align}
  \bar v_X \equiv \f1N \sum_p v_{Xp},
  &&
  \col{X} \equiv \sum_p X_p,
\end{align}
are respectively the mean value in $\v v_X$ and the collective version
of $X$.  In order to compute the second order effective Hamiltonian
$H_1^{(2)}$, we first choose an arbitrary permutationally symmetric
state $\ket\psi\in\M_0$ and expand (see Appendix \ref{sec:H_V1_psi}):
\begin{align}
  H_0 \V_1 \ket\psi
  = E_0 \V_1 \ket\psi
  + \sum_X \sum_p \p{\sum_q h_{pq} v_{Xq} - h_p v_{Xp}} X_p
  \ket\psi,
  &&
  h_p \equiv \sum_q h_{pq},
  \label{eq:H_V1_psi}
\end{align}
where we define $h_{pq}$ for all $p,q$ (as opposed to only $p<q$) by
enforcing $h_{pq}=h_{qp}$ and $h_{pp}=0$.  We thus find that a
perturbation $\V_1^\Delta\ket\psi$ (indexed by $\Delta$) is an
eigenvector of the interaction Hamiltonian $H_0$ if its coefficient
vectors $\v v_X^\Delta$ all satisfy the eigenvalue equation
\begin{align}
  \p{\m h - \diag\v h}\c\v v = \Delta \v v,
  &&
  \m h \equiv \sum_{p,q} h_{pq} \op{p}{q},
  &&
  \v h \equiv \sum_q h_q \ket{q},
  \label{eq:cond_1}
\end{align}
where $\m h$ is a matrix of all $h_{pq}$, and
$\diag\v h\equiv\sum_p h_p \op{p}$ is a matrix with $\v h$ on the
diagonal and zeros everywhere else.  Solving the general eigenvalue
problem in \eqref{eq:cond_1} requires diagonalizing the $N\times N$
matrix $\m h-\diag\v h$.  When the interaction Hamiltonian $H_0$ is
translationally invariant, however, this eigenvalue problem can be
solved analytically; we provide the corresponding solution in Appendix
\ref{sec:trans_inv}.

If we {\it construct} a perturbation $\V_1^\Delta$ with coefficients
$\v v_X^\Delta$ that satisfy \eqref{eq:cond_1} with eigenvalue
$\Delta$, then
\begin{align}
  H_0 \V_1^\Delta \ket\psi = \p{E_0 + \Delta} \V_1^\Delta \ket\psi.
\end{align}
Interestingly, the energy of the state $\V_1^\Delta\ket\psi$ depends
only on the coefficients $\v v_X^\Delta$, and is entirely independent
of the state $\ket\psi$ or the choice of a trace-zero single-spin
operators $X$ used to build $\V_1^\Delta$.  Finding operators
$\V_1^\Delta$ that generate eigenvectors of the interaction
Hamiltonian $H_0$ when they act on permutationally symmetric states
$\ket\psi\in\M_0$ thus reduces to finding eigenvectors of the
interaction matrix $\m h-\diag\v h$.

If the permutationally symmetric manifold $\M_0$ is gapped away from
all other states, then then a vector $\V_1^\Delta\ket\psi$ with
$\Delta=0$ must lie within the permutationally symmetric manifold
$\M_0$, which implies that the operator $\V_1^\Delta$ preserves the
permutational symmetry of $\M_0$.  Indeed, a constant vector
$\v\1\equiv\p{1,1,1,\cdots}$ of ones satisfies the condition in
\eqref{eq:cond_1} with eigenvalue $0$, which implies that all vectors
$\v v$ satisfying \eqref{eq:cond_1} with $\Delta\ne0$ must be
orthogonal to $\v\1$, and therefore mean-zero.

We now return to the task of computing the second-order effective
Hamiltonian $H_1^{(2)}$.  Any coefficient vector $\v v_X$ can be
expanded into its projections $\v v_X^\Delta$ onto the eigenspace of
vectors satisfying \eqref{eq:cond_1} with eigenvalue $\Delta$,
i.e.~$\v v_X = \sum_\Delta \v v_X^\Delta$, which also allows us to
expand
\begin{align}
  \V_1 = \sum_\Delta \V_1^\Delta,
  &&
  \V_1^\Delta \equiv \sum_X \v v_X^\Delta \c \v X,
\end{align}
where each operator $\V_1^\Delta$ generates a state with definite
interaction energy $\Delta$ above that of the permutationally
symmetric manifold $\M_0$.  We can therefore simplify
\begin{align}
  H_1^{(2)}
  = - \P_0 \V_1 \E \V_1 \P_0
  = - \sum_{\Delta\ne0} \f1{\Delta}
  \P_0 \V_1^\Delta \P_\Delta \V_1^\Delta \P_0
  = - \sum_{\Delta\ne0} \f1{\Delta} \P_0 \p{\V_1^\Delta}^2 \P_0,
\end{align}
where the product $\P_0 \p{\V_1^\Delta}^2 \P_0$ is worked out in
Appendix \ref{sec:PXYP}.  In total, we find
\begin{align}
  H_1^{(2)}
  \EQPS \f1{N\p{N-1}} \sum_{X,Y} \sum_{\Delta\ne0}
  \f{\v v_X^\Delta\c\v v_Y^\Delta}{\Delta}
  \p{\col{X}\,\col{Y} - N \col{XY}},
  \label{eq:H_1_2}
\end{align}
where $\EQPS$ denotes equality under a restriction to the fully
symmetric manifold $\M_0$; $\v v_X^\Delta$ is the projection of
$\v v_X$ onto the $\Delta$-eigenspace of the matrix $\m h-\diag\v h$;
and for any single-spin operator $Z$, we define the collective
operator
\begin{align}
  \col{Z} \equiv \sum_p Z_p.
\end{align}

%%%%%%%%%%%%%%%%%%%%%%%%%%%%%%%%%%%%%%%%%%%%%%%%%%%%%%%%%%%%%%%%%%%%%%
\section{Two-body perturbations}
\label{sec:two_body_pert}

The first-order effective Hamiltonian $H_2^{(1)}$ induced on the fully
symmetric manifold $\M_0$ by the two-body perturbation $\V_2$ in
\eqref{eq:perturbations} is simply the restriction of $\V_2$ onto
$\M_0$:
\begin{align}
  H_2^{(1)} = \sum_O \sum_{p<q} w_{Opq} \P_0 O_{pq} \P_0
  \EQPS \sum_O \sum_{p<q} w_{Opq} O_{1,2}
  \EQPS \sum_O \bar w_O \col{O},
\end{align}
where
\begin{align}
  \bar w_O \equiv {N \choose 2}^{-1} \sum_{p<q} w_{Opq},
  &&
  \col{O} \equiv \sum_{p<q} O_{pq},
\end{align}
are an average of $w_{Opq}$ and a collective version of the two-spin
operator $O_{pq}$.  If the two-spin operator $O=X\otimes Y$, then
\begin{align}
  H_2^{(1)} \EQPS \f12 \sum_{\p{X,Y}} \bar w_{\p{X,Y}}
  \p{\col{X}\,\col{Y} - \col{XY}}.
\end{align}
In order to compute the second-order effective Hamiltonian
$H_2^{(2)}$, as before we pick an arbitrary permutationally symmetric
state $\ket\psi\in\M_0$ and expand (see Appendix \ref{sec:H_VM_psi})
\begin{align}
  H_0 \V_2 \ket\psi
  = E_0 \V_2 \ket\psi
  + \sum_O \sum_{p<q}
  \sp{\sum_k \p{h_{kp} w_{Okq} w_{Okq} + h_{kq} w_{Opk}}
  - \bar h_{pq} w_{Opq}}
  O_{pq} \ket\psi,
  \label{eq:H_V2_psi}
\end{align}
where to keep notation compact, we define
\begin{align}
  \bar h_{pq} \equiv h_p + h_q - 2 h_{pq},
  &&
  h_p \equiv \sum_q h_{pq},
\end{align}
and we define $w_{Opq}$ for all $p,q$ (as opposed to only $p<q$) by
enforcing $w_{Opq}=w_{Oqp}$ and $w_{Opp}=0$.  We thus find that the
vector $\V_2^\Delta\ket\psi$ (indexed by $\Delta$) is an eigenvector
of the interaction Hamiltonian $H_0$ if its coefficients
$\v{\m w}_O^\Delta$ all satisfy the eigenvalue equation
\begin{align}
  \check{\m h} \c \v{\m w} = \Delta \v{\m w},
  \label{eq:cond_2}
\end{align}
where
\begin{align}
  \check{\m h}
  \equiv \sum_{p<q} \ket{\set{p,q}} \sp{\sum_{k\notin\set{p,q}}
    \p{h_{kp} \bra{\set{k,q}} + h_{kq} \bra{\set{p,k}}}
    - \bar h_{pq} \bra{\set{p,q}}}.
  \label{eq:h_super_mat}
\end{align}
Solving the general eigenvalue problem in \eqref{eq:cond_2} requires
diagonalizing the ${N \choose 2}\times{N \choose 2}\sim N^2\times N^2$
matrix $\check{\m h}$.  When both he interaction Hamiltonian $H_0$
translationally invariant, however, this eigenvalue problem can be
reduced to that of diagonalizing an $\sim N\times N$ matrix; we
discuss this reduction in Appendix \ref{sec:trans_inv}.

If we {\it construct} a perturbation $\V_2^\Delta$ with coefficients
$\v{\m w}_O^\Delta$ that satisfy \eqref{eq:cond_2} with eigenvalue
$\Delta$, then
\begin{align}
  H_0 \V_2^\Delta \ket\psi = \p{E_0 + \Delta} \V_2^\Delta \ket\psi,
\end{align}
which implies that $\V_2^\Delta\ket\psi$ is a state of definite energy
that depends only on the coefficients $\m w_O^\Delta$.  Similarly to
the case of single-body perturbations, we can decompose any vector
$\v{\m w}_O$ into its projections $\v{\m w}_O^\Delta$ onto the
eigenspace of vectors satisfying \eqref{eq:cond_2} with eigenvalue
$\Delta$, i.e.~$\v{\m w}_O=\sum_\Delta\v{\m w}_O^\Delta$, which allows
us to expand
\begin{align}
  \V_2 = \sum_\Delta \V_2^\Delta,
  &&
  \V_2^\Delta \equiv \sum_O \v{\m w}_O^\Delta \c \v{\m O},
\end{align}
and in turn write
\begin{align}
  H_2^{(2)} = - \P_0 \V_2 \E \V_2 \P_0
  = -\sum_{\Delta\ne0} \f1\Delta \P_0 \p{\V_2^\Delta}^2 \P_0,
\end{align}
where the product $\P_0 \p{\V_2^\Delta}^2 \P_0$ is simplified in
Appendix \ref{sec:POQP}.  In the special case that
\begin{align}
  \V_2 = \sum_{p<q} w_{pq} s_\z^{(p)} s_\z^{(q)},
\end{align}
where $s_\z\equiv\p{\op\up-\op\dn}/2$ is an SU(2) spin-$z$ operator,
the second-order effective Hamiltonian $H_2^{(2)}$ takes the form
\begin{align}
  H_2^{(2)} \simeq_\PS W_4 S_\z^4 - W_2 S_\z^2,
\end{align}
where $\simeq_\PS$ denotes equality up to scalar terms and a
restriction to the permutationally symmetric manifold;
$S_\z\equiv\sum_p s_\z^{(p)}$ is a collective SU(2) spin operator; and
$W_k$ are scalar coefficients.  In order to expand the coefficients
$W_k$, for any ${N \choose 2}$-component vector
$\v{\m m}=\sum_{p<q}m_{pq}\ket{\set{p,q}}$ we define the $N$-component
vector
\begin{align}
  \v m \equiv \sum_{p,q} m_{pq} \ket{p},
\end{align}
where we define $m_{pq}$ for all $p,q$ (as opposed to only $p<q$) by
enforcing $m_{pq}=m_{qp}$ and $m_{pp}=0$.  We then denote the
projection of $\v{\m m}$ onto the $\Delta$-eigenspace of the matrix
$\check{\m h}$ defined in \eqref{eq:h_super_mat} by $\v{\m m}_\Delta$,
which allows us to concisely write
\begin{align}
  W_4
  \equiv \sum_{\Delta\ne0} \f1\Delta \f{\v w_\Delta\c\v w_\Delta
    - \v{\m w}_\Delta\c\v{\m w}_\Delta}{N!_4},
\end{align}
and
\begin{align}
  W_2
  \equiv \sum_{\Delta\ne0} \f1{\Delta} \sp{\f{\v w_\Delta\c\v w_\Delta
      - 2\v{\m w}_\Delta\c\v{\m w}_\Delta}{4N!_2}
    + \f{\p{3N-4}\p{\v w_\Delta\c\v w_\Delta
        - \v{\m w}_\Delta\c\v{\m w}_\Delta}}{2N!_4}},
\end{align}
where
\begin{align}
  N!_k \equiv \prod_{j=0}^{k-1} \p{N-j} = \f{N!}{\p{N-k}!}
\end{align}
denotes a falling factorial.

\newpage
\appendix

%%%%%%%%%%%%%%%%%%%%%%%%%%%%%%%%%%%%%%%%%%%%%%%%%%%%%%%%%%%%%%%%%%%%%%
\section{Existence of a spectral gap with power-law interactions}
\label{sec:gap}

Here we show that power-law SU($n$)-symmetric interactions of the form
in \eqref{eq:ints} with $h_{pq}=-h/\abs{p-q}^\alpha$ yield a
non-vanishing many-body interaction energy gap when $\alpha\le D$,
where $D$ is the dimension of the lattice.  On a periodic lattice of
$N=L^D$ spins, the spectral gap of the interaction Hamiltonian is (see
Appendix \ref{sec:trans_inv_single})
\begin{align}
  \Delta
  = \sum_{d\in\ZZ_L^D} h_{0,d} \sp{\cos\p{d \c k_{\t{SE}}}-1}
  = h \sum_{\substack{d\in\ZZ_L^D\\\abs{d}\ge1}}
  \f{1-\cos\p{d \c k_{\t{SE}}}}{\abs{d}^\alpha},
\end{align}
where the domain of $d$ is defined in terms of integers modulo $L$,
i.e.~$\ZZ_L\simeq\set{0,1,\cdots,L-1}$ with the relation $\simeq$
denoting an association that ignores the cyclic structure of $\ZZ_L$;
$k_{\t{SE}}\in\ZZ_L\times2\pi/L$ is a wavenumber for a singly-excited
spin-wave state; and the norm $\abs{d}$ for a vector $d$ on a periodic
lattice is implicitly understood to mean the smallest Euclidean
distance of $d$ from a fixed origin.  In order to yield the smallest
excitation energy $\Delta$, the wavenumber $k_{\t{SE}}$ should
maximize the contribution of the cosine term above, which is achieved
by a wavenumber that minimizes the oscillations of this term when
integrated over the entire lattice.  A suitable candidate for a
minimal wavenumber is $k_{\t{SE}}=\p{2\pi/L,0,0,\cdots}$, which
corresponds to an excitation energy
\begin{align}
  \Delta = h \sum_{\substack{d\in\ZZ_L^D\\\abs{d}\ge1}}
  \f{1-\cos\p{d_12\pi/L}}{\abs{d}^\alpha}.
\end{align}
Defining $\epsilon\equiv2/L$ and a rescaled domain symmetrized about
$0$, $\SS_\epsilon\simeq\ZZ_L/\epsilon$ with
$\SS_\epsilon\subset\sp{-1,1}$, we substitute $x\simeq\epsilon d$ to
get
\begin{align}
  \Delta
  = h \sum_{\substack{x\in\SS_\epsilon^D\\\abs{x}\ge\epsilon}}
  \f{1-\cos\p{\pi x_1}}{\abs{x/\epsilon}^\alpha}
  = h \epsilon^{\alpha-D} \sum_{\substack{x\in\SS_L^D\\\abs{x}\ge\epsilon}}
  \epsilon^D \f{1-\cos\p{\pi x_1}}{\abs{x}^\alpha}.
\end{align}
As $\epsilon\to0$ the discrete sum over $x$ becomes an integral that
avoids an infinitesimal region about the origin, i.e.
\begin{align}
  \Delta = h \epsilon^{\alpha-D} \I_{D\epsilon},
  &&
  \I_{D\epsilon}
  \equiv \int_{\TT_1^D\setminus\TT_\epsilon^D} \d^Dx\,
  \f{1-\cos\p{\pi x_1}}{\abs{x}^\alpha},
\end{align}
where the integral $\I_{D\epsilon}$ is defined using the interval
$\TT_a\equiv\p{-a,a}$.  The integrand of $\I_{D\epsilon}$ is strictly
positive and well-behaved on the entirety of its domain except for the
origin, where depending on the value of $\alpha$ the integrand may
vanish or diverge as $\abs{x}\to0$.  Together, these facts mean that
\begin{align}
  \I_{D\epsilon} \stackrel{\epsilon\to0}{\sim} \epsilon^{-\gamma},
  &&
  \Delta \stackrel{\epsilon\to0}{\sim} h \epsilon^{\alpha-D-\gamma},
\end{align}
for some $\gamma\ge0$, which implies that gap $\Delta$ is
non-vanishing when $\alpha\le D\le D+\gamma$.

For the sake of completion, we add that in fact the gap $\Delta$
always vanishes in the thermodynamic limit when $\alpha>D$.  To see
this behavior, we note that the asymptotic dependence of the integral
$I_{D\epsilon}$ on $\epsilon$ is determined by the behavior of its
integrand when $\abs{x}\sim\epsilon$, in which case
$1-\cos\p{\pi x_1}\sim x_1^2$, so
\begin{align}
  \I_{D\epsilon}
  \sim \int_{\TT_1^D\setminus\TT_\epsilon^D} \d^Dx\,
  \f{x_1^2}{\abs{x}^\alpha}.
\end{align}
We can then use the fact that $x_1^2\le\abs{x}^2$ and change to
spherical coordinates to find that
\begin{align}
  \I_{D\epsilon} \lesssim
  \int_{\TT_1^D\setminus\TT_\epsilon^D} \d^Dx\,
  \f{\abs{x}^2}{\abs{x}^\alpha}
  \sim \int_\epsilon^1 \d x\, x^{D+1-\alpha}
  \sim
  \begin{cases}
    \epsilon^0 & \alpha < D+2 \\
    \log\p{1/\epsilon} & \alpha = D+2 \\
    \epsilon^{D+2-\alpha} & \alpha > D+2
  \end{cases}.
\end{align}
It follows that the spectral gap
\begin{align}
  \Delta \stackrel{\epsilon\to0}{\lesssim}
  \begin{cases}
    \epsilon^{\alpha-D} & \alpha < D+2 \\
    \epsilon^2 \log\p{1/\epsilon} & \alpha = D + 2 \\
    \epsilon^2 & \alpha > D+2
  \end{cases},
\end{align}
which vanishes as $\epsilon\to0$ for all $\alpha>D$.

%%%%%%%%%%%%%%%%%%%%%%%%%%%%%%%%%%%%%%%%%%%%%%%%%%%%%%%%%%%%%%%%%%%%%%
\section{Constructing eigenstates of multi-body perturbations}
\label{sec:H_VM_psi}

In this section, we simplify the products of the form
$H_0\V_M\ket\psi$ to arrive at expansions such as \eqref{eq:H_V1_psi}
and \eqref{eq:H_V2_psi}, which define conditions under which a
multi-body perturbation $\V_M$ generates eigenstates of the
SU($n$)-symmetric interaction Hamiltonian $H_0$ in \eqref{eq:ints}
when acting on a permutationally symmetric state $\ket\psi\in\M_0$.
Here $\V_M$ is an $M$-body perturbation of the form
\begin{align}
  \V_M \equiv \sum_{k\in\C_N\p{M}} w_k O_k.
\end{align}
where $O$ is an $M$-spin operator that is invariant under arbitrary
permutation of its tensor factors; $k\equiv\p{k_1,k_2,\cdots,k_M}$ is
a choice of $M$ distinct spins that $O$ acts on; and $\C_N\p{n}$ is a
choice of $n$ elements from a set of $N$.

%%%%%%%%%%%%%%%%%%%%%%%%%%%%%%%%%%%%%%%%%%%%%%%%%%
\subsection{Single-body perturbation}
\label{sec:H_V1_psi}

We first simplify
\begin{align}
  H_0 \V_1 \ket\psi
  = \sum_k \sum_{p<q} h_{pq} v_k \Pi_{pq} X_k \ket\psi,
  \label{eq:H_V1_psi_start}
\end{align}
where without loss of generality we consider a single-body
perturbation $\V_1$ built out of only one single-spin operator $X$.
The sum in \eqref{eq:H_V1_psi_start} has terms with $k\in\set{p,q}$,
and terms with $k\notin\set{p,q}$.  In the case of $k\notin\set{p,q}$,
the permutation operator $\Pi_{pq}$ commutes with $X_k$ and
annihilates on $\ket\psi$, leaving terms at each fixed $k$ of the form
\begin{align}
  \sum_{\substack{p<q\\k\notin\set{p,q}}} h_{pq}
  = \sum_{p<q} h_{pq} - \sum_{\substack{p<q\\k\in\set{p,q}}} h_{pq}
  = E_0 - \f12 \sum_{\substack{p,q\\k\in\set{p,q}}} h_{pq}
  = E_0 - h_k,
\end{align}
where we define $h_{pq}$ for all $p,q$ (as opposed to only $p<q$) by
enforcing $h_{pq}=h_{qp}$ and $h_{pp}=0$; and
$h_k \equiv \sum_\ell h_{k\ell}$.  The terms in
\eqref{eq:H_V1_psi_start} with $k\notin\set{p,q}$ are then
\begin{align}
  \sum_{\substack{p<q\\k\notin\set{p,q}}}
  h_{pq} v_k \Pi_{pq} X_k \ket\psi
  = \sum_k \p{E_0 - h_k} v_k X_k \ket\psi
  = E_0 \V_1 \ket\psi - \sum_k h_k v_k X_k \ket\psi
\end{align}
while the terms in \eqref{eq:H_V1_psi_start} with $k\in\set{p,q}$ take
the form
\begin{align}
  \sum_{\substack{p<q\\k\in\set{p,q}}}
  h_{pq} v_k \Pi_{pq} X_k \ket\psi
  = \sum_{k<q} h_{kq} v_k X_q \ket\psi
  + \sum_{p<k} h_{pk} v_k X_p \ket\psi
  = \sum_{k,q} h_{kq} v_k X_q \ket\psi
\end{align}
which implies
\begin{align}
  H_0 \V_1 \ket\psi
  = E_0 \V_1 \ket\psi
  + \sum_p \p{\sum_{q} h_{pq} v_q - h_p v_p}
  X_p \ket\psi.
  \label{eq:H_V1_psi_end}
\end{align}

%%%%%%%%%%%%%%%%%%%%%%%%%%%%%%%%%%%%%%%%%%%%%%%%%%
\subsection{Multi-body perturbation}
\label{sec:H_VM_psi}

We wish to simplify
\begin{align}
  H_0 \V_M \ket\psi
  = \sum_{\substack{\p{p,q}\in\C_N\p{2}\\k\in\C_N\p{M}}} h_{pq} w_k
  \Pi_{pq} O_k \ket\psi,
  \label{eq:H_VM_psi_start}
\end{align}
where without loss of generality we consider an $M$-body perturbation
$\V_M$ built out of only one $M$-spin operator $O$.  The sum in
\eqref{eq:H_VM_psi_start} has terms with $p,q\in k$, terms with
$p,q\notin k$, and terms with only one of $p$ or $q\in k$.  The
permutation operator $\Pi_{pq}$ acts trivially on $O_k\ket\psi$ when
$p,q\in k$ or $p,q\notin k$, leaving terms at each fixed $k$ of the
form
\begin{align}
  \sum_{\substack{\p{p,q}\in\C_N\p{2}\\p,q\in k~\t{or}~p,q\notin k}} h_{pq}
  = \sum_{\p{p,q}\in\C_N\p{2}} h_{pq}
  - \sum_{p\in k} \sum_{\substack{q\in\ZZ_N\\q\notin k}} h_{pq}
  = E_0 - \sum_{p\in k}
  \p{\sum_{q\in\ZZ_N} h_{pq} - \sum_{q\in k} h_{pq}},
\end{align}
where we first split a sum over $\p{p,q}\in\C_N\p{2}$ with the
restriction that $p,q\in k~\t{or}~p,q\notin k$ into an unrestricted
sum minus a remainder, and then similarly split the sum over all spins
$q$ with the restriction that $q\notin k$ into a sum over all $q$
minus the remainder $q\in k$.  We thus find that
\begin{align}
  \sum_{\substack{\p{p,q}\in\C_N\p{2}\\p,q\in k~\t{or}~p,q\notin k}} h_{pq}
  = E_0 - \bar h_k,
  &&
  \bar h_k \equiv \sum_{p\in k}\p{h_p - \sum_{q\in k}h_{pq}},
  &&
  h_p \equiv \sum_q h_{pq}.
\end{align}
The terms in \eqref{eq:H_VM_psi_start} with only one of $p\in k$ or
$q\in k$, meanwhile, are
\begin{align}
  \sum_{k\in\C_N\p{M}} \sum_{\substack{p\in\ZZ_N\\p\notin k}} \sum_{q\in k}
  h_{pq} w_k \Pi_{pq} O_k \ket\psi
  = \sum_{k\in\C_N\p{M}} \sum_{\substack{p\in\ZZ_N\\p\notin k}}
  \sum_{j=1}^M h_{p k_j} w_k \Pi_{p k_j} O_k \ket\psi.
\end{align}
Defining
$k_{jp}\equiv\p{k_1,k_2,\cdots,k_{j-1},p,k_{j+1},\cdots,k_M}$, i.e.~an
element of $\C_N\p{M}$ in which the $j$-th number $k_j$ has been
replaced by $p$, we can simplify
\begin{align}
  \Pi_{p k_j} O_k \ket\psi = O_{k_{jp}}\ket\psi,
\end{align}
and switch the labels for $p$ and $k_j$ to find that
\begin{align}
  \sum_{k\in\C_N\p{M}} \sum_{\substack{p\in\ZZ_N\\p\notin k}} \sum_{q\in k}
  h_{pq} w_k  \Pi_{pq} O_k \ket\psi
  = \sum_{k\in\C_N\p{M}} \sum_{\substack{p\in\ZZ_N\\p\notin k}}
  \sum_{j=1}^M h_{pk_j} w_{k_{jp}} O_k \ket\psi.
\end{align}
Defining
\begin{align}
  \p{h\circ w}_k
  \equiv \sum_{\substack{p\in\ZZ_N\\p\notin k}} \sum_{j=1}^M
  h_{pk_j} w_{k_{jp}}
  = \sum_{\substack{p\in\ZZ_N\\p\notin k}}
  \p{h_{k_1 p} w_{pk_2k_3\cdots k_M}
    + h_{k_2 p} w_{k_1pk_3\cdots k_M}
    + \cdots + h_{k_M p} w_{k_1k_2\cdots k_{M-1} p}},
\end{align}
we thus find that
\begin{align}
  H_0 \V_M \ket\psi
  = E_0 \V_M \ket\psi + \sum_{k\in\C_N\p{M}}
  \sp{\p{h\circ w}_k - \bar h_k w_k} O_k \ket\psi.
\end{align}
If $M=1$, then $\p{h\circ w}_k=\sum_\ell h_{k\ell}w_\ell$ and
$\bar h_k=h_k$, so
\begin{align}
  H_0 \V_1 \ket\psi
  = E_0 \V_1 \ket\psi + \sum_{k\in\ZZ_N}
  \p{\sum_{\ell\in\ZZ_N} h_{k\ell} w_\ell - h_k w_k} O_k \ket\psi,
\end{align}
which is precisely what we found in \eqref{eq:H_V1_psi_end}.  If
$M=2$, then $\p{h\circ w}_{pq}=\sum_k\p{h_{kp} w_{kq}+h_{kq}w_{pk}}$,
so
\begin{align}
  H_0 \V_2 \ket\psi
  = E_0 \V_2 \ket\psi + \sum_{\p{p,q}\in\C_N\p{2}}
  \sp{\sum_{k\in\ZZ_N}\p{h_{kp} w_{kq} + h_{kq} w_{pk}}
    - \bar h_{pq} w_{pq}}
  O_k \ket\psi,
\end{align}
where $\bar h_{pq}=h_p+h_q-2h_{pq}$ with $h_p\equiv\sum_qh_{pq}$.

%%%%%%%%%%%%%%%%%%%%%%%%%%%%%%%%%%%%%%%%%%%%%%%%%%%%%%%%%%%%%%%%%%%%%%
\section{Constructing eigenstates of translationally invariant
  systems}
\label{sec:trans_inv}

Here we simplify the eigenvalue problems that appear at second order
in the single- and two-body theory in Sections
\ref{sec:single_body_pert} and \ref{sec:two_body_pert} in the case of
a translationally invariant system with periodic boundary conditions.
We solve the single-body eigenvalue problem analytically, and reduce
the two-body eigenvalue problem to that of diagonalizing
$\p{N-1}\times\p{N-1}$ matrices.  Finally, for we discuss the further
reduction of the two-body problem to that of diagonalizing matrices
with dimensions $\sim\p{N/2^D}\times\p{N/2^D}$ in the case of an
isotropic system on a $D$-dimensional periodic lattice, but leave this
reduction to future work.

%%%%%%%%%%%%%%%%%%%%%%%%%%%%%%%%%%%%%%%%%%%%%%%%%%
\subsection{Single-body eigenvalue problem}
\label{sec:trans_inv_single}

For reference, the single-body eigenvalue problem is
\begin{align}
  \p{\m h - \diag\v h}\c\v v = \Delta \v v,
  \label{eq:cond_1_ref}
\end{align}
where $\m h$ is a matrix of all couplings $h_{pq}$ that appear in the
interaction Hamiltonian $H_0$ in \eqref{eq:ints}; $\v h$ is a vector
of all $h_p\equiv\sum_q h_{pq}$; $\diag\v h$ is a matrix with $\v h$
on the diagonal and zeroes everywhere else; and $\v v$ is a vector of
coefficients used to construct the single-body perturbation
$\V_1\p{\v v}$ defined in \eqref{eq:perturbations}.  When the
interaction Hamiltonian $H_0$ is translationally invariant, the
couplings $h_{pq}$ depend only on the separation $\pm\p{p-q}$, so the
eigenvectors of $\m h$ are plane waves of the form
\begin{align}
  \v v_k \equiv \sum_p e^{ip\c k} \ket{p},
\end{align}
where on a $D$-dimensional periodic lattice of $N=L^D$ spins, lattice
sites are indexed by vectors
$p\in\ZZ_L^D\simeq\set{0,1,\cdots,L-1}^{\otimes D}$, and wavenumbers
take on values $k\in\ZZ_L^D\times2\pi/L$.  The corresponding
eigenvalues of $\m h$ can be determined by expanding
\begin{align}
  \m h\c\v v_k = \sum_{p,q} h_{pq} e^{iq\c k} \ket{p}
  = \sum_{p,d} h_{p,p+d} e^{i\p{p+d}\c k} \ket{p}
  = \sum_d h_{0,d} e^{id\c k} \v v_k
  = \sum_d h_{0,d} \cos\p{d\c k} \v v_k,
\end{align}
where the imaginary terms vanish because $h_{0,d}=h_{0,-d}$.  The
remainder of \eqref{eq:cond_1_ref} that we need to sort out is
$\diag\v h$, where all $h_p=\sum_qh_{pq}=\sum_dh_{0,d}$ are equal,
which implies that $\diag\v h=\sum_dh_{0,d}$ is a scalar.  We thus
find that
\begin{align}
  \p{\m h - \diag\v h}\c\v v_k = \Delta_k \v v_k,
  &&
  \Delta_k \equiv \sum_d h_{0,d} \sp{\cos\p{d\c k}-1}.
\end{align}

%%%%%%%%%%%%%%%%%%%%%%%%%%%%%%%%%%%%%%%%%%%%%%%%%%
\subsection{Two-body eigenvalue problem}
\label{sec:trans_inv_two}

The two-body eigenvalue equation in \eqref{eq:cond_2} is designed to
find solutions $\m w$ to the equation
\begin{align}
  \sum_k \p{h_{pk} w_{kq} + w_{pk} h_{kq}}
  - \bar h_{pq} w_{pq}
  = \Delta w_{pq},
  &&
  \bar h_{pq} \equiv h_p + h_q - 2h_{pq},
  &&
  h_p \equiv \sum_q h_{pq},
  \label{eq:cond_2_ref_start}
\end{align}
which can be written in matrix form as
\begin{align}
  \P_{\t{off-diag}}\p{\m h\c\m w + \m w\c\m h} - \bar{\m h} * \m w
  = \Delta \m w,
  \label{eq:cond_2_ref}
\end{align}
where $\m h$, $\bar{\m h}$, and $\m w$ are matrices of all $h_{pq}$,
$\bar h_{pq}$ and $w_{pq}$;
$\P_{\t{off-diag}}\p{\m m}\equiv \m m-\sum_p\op{p}\m m\op{p}$ denotes
the projection of $\m m$ onto the space of strictly off-diagonal
matrices; and $\bar{\m h}*\m w$ denotes an element-wise (Kronecker)
product of $\bar{\m h}$ and $\m w$, i.e.~with matrix elements
$\bar h_{pq}w_{pq}$.  In order to find solutions to
\eqref{eq:cond_2_ref} when the interaction Hamiltonian $H_0$ is
translationally invariant, in which case $h_{pq}$ and $\bar h_{pq}$
depend only on the separation $\pm\p{p-q}$, we expand all matrices in
a basis that the respects translational symmetry.  To this end, for
all lattice displacements $d\in\ZZ_L^D$ and wavenumbers
$k\in\ZZ_L^D\times2\pi/L$ we define the {\it cycle matrices}
\begin{align}
  \m m_{dk} \equiv \sum_p e^{ip\c k} \op{p+d}{p},
  \label{eq:cycle}
\end{align}
where the element-wise sums in $p+d$ are implicitly taken modulo $L$
(i.e.~the length of the periodic lattice along each axis).  In terms
of these cycle matrices, we define
\begin{align}
  \tilde w_{dk} \equiv \f1N \tr\p{\m m_{dk}^\dag \m w},
\end{align}
and expand
\begin{align}
  \m w = \sum_{d,k} \tilde w_{dk} \m m_{dk},
  &&
  \m h = \sum_{q,d} h_{q+d,q} \op{q+d}{q}
  = \sum_d h_{d,0} \m m_{d,0},
  &&
  \bar{\m h} = \sum_d \bar h_{d,0} \m m_{d,0}.
\end{align}
We can then write the condition in \eqref{eq:cond_2_ref} as a new set
of conditions
\begin{align}
  \sum_{\substack{c,d\\c+d\ne0}}
  h_{c,0} \tilde w_{dk} \p{1 + e^{-ic\c k}} \m m_{c+d,k}
  - \sum_{d\ne0} \bar h_{d,0} \tilde w_{dk} \m m_{dk}
  = \Delta \sum_{d\ne0} \tilde w_{dk} \m m_{dk},
\end{align}
that must hold for all wavenumbers $k$.  With some re-indexing, we can
write
\begin{align}
  \sum_{c,d\ne0} h_{d-c,0} \tilde w_{ck}
  \p{1 + e^{-i\p{d-c}\c k}} \m m_{dk}
  - \sum_{d\ne0} \bar h_{d,0} \tilde w_{dk} \m m_{dk}
  = \Delta \sum_{d\ne0} \tilde w_{dk} \m m_{dk},
  \label{eq:cond_2_cycle}
\end{align}
and define
\begin{align}
  \ut{\check{\m h}}_k
  \equiv \sum_{c,d\ne0} h_{d-c,0} \p{1+e^{-i\p{d-c}\c k}} \op{d}{c}
  - \sum_{d\ne0} \bar h_{d,0} \op{d},
  &&
  \ut{\v{\m w}}_k \equiv \sum_{d\ne0} \tilde w_{dk} \ket{d},
  \label{eq:cycle_vecs}
\end{align}
in order to express \eqref{eq:cond_2_cycle} as the set of eigenvalue
equations
\begin{align}
  \ut{\check{\m h}}_k \c \ut{\v{\m w}}_k = \Delta \ut{\v{\m w}}_k.
  \label{eq:cond_2_square}
\end{align}
The appearance of a wavenumber index $k$ in \eqref{eq:cond_2_square}
implies that all fixed-$k$ components of a general coefficient matrix
$\m w$ must simultaneously satisfy \eqref{eq:cond_2_square} in order
to generate states of fixed excitation energy $\Delta$ when the
corresponding perturbation $\V_2^\Delta$ acts on a permutationally
symmetric state $\ket\psi\in\M_0$.  The decomposition of an arbitrary
coefficient vector $\v{\m w}=\sum_\Delta\v{\m w}_\Delta$ into
coefficient vectors $\v{\m w}_\Delta$ that generate a
fixed-excitation-energy perturbation $\V_2^\Delta$ is then performed
by projecting each $k$-component of $\v{\m w}$ onto the corresponding
$\Delta$-eigenspace of $\ut{\check{\m h}}_k$.  If the perturbation
$\V_2$ is translationally invariant, then the $k$-components of
$\v{\m w}$ for all $k\ne 0$ all vanish, so we only need to consider
the case of $k=0$.

We can reduce the eigenvalue problem in \eqref{eq:cond_2_square} even
further in the case of an isotropic system, in which the couplings
$h_{pq}$ depend only on the distance $\abs{p-q}$.  On a $D$
dimensional lattice, isotropy implies one reflection symmetry for each
lattice axis, which reduces the degrees of freedom in of
$\ut{\check{\m h}}_k$ from $\sim N\times N$ to
$\sim\p{N/2^D}\times\p{N/2^D}$.  We leave this reduction to future
work.

%%%%%%%%%%%%%%%%%%%%%%%%%%%%%%%%%%%%%%%%%%%%%%%%%%
\subsection{The fully collective model}

Here we solve the single- and two-body eigenvalue problems in the case
of a fully collective spin model with uniform couplings:
$h_{pq}=-1+\delta_{pq}$.  In this case, the single-body excitation
energies $\Delta_k$ are
\begin{align}
  \Delta_k
  = - \sum_{d\ne 0} \sp{\cos\p{d\c k}-1}
  = N \p{1-\delta_{k,0}}.
\end{align}
In order to solve the two-body eigenvalue problem, we expand
\begin{align}
  \ut{\check{\m h}}_k
  &= -\sum_{\substack{c,d\ne0\\c\ne d}}
  \p{1+e^{-i\p{d-c}\c k}} \op{d}{c}
  + \sum_{d\ne0} \sp{2\p{N-1}-2} \op{d} \\
  &= -\sum_{c,d\ne0} \p{1+e^{-i\p{d-c}\c k}} \op{d}{c}
  + 2\p{N-1} \1_{N-1},
\end{align}
where $\1_{N-1}\equiv\sum_{d\ne0}\op{d}$ is the $\p{N-1}\times\p{N-1}$
identity matrix, and define the vectors
\begin{align}
  \ket{\ut{p}}
  \equiv \f1{\sqrt{N-1}} \sum_{d\ne 0} e^{i p\c d}\ket{d},
\end{align}
in terms of which
\begin{align}
  \ut{\check{\m h}}_k
  = 2 \p{N-1} \1_{N-1} - \p{N-1} \p{\op{\ut{0}} + \op{\ut{k}}}.
\end{align}
The vectors $\ket{\ut{p}}$ are normalized, but not orthogonal:
$\braket{\ut{p}|\ut{q}}=\delta_{pq}-\p{1-\delta_{pq}}/\p{N-1}$.  We
therefore have
\begin{align}
  \ut{\check{\m h}}_k
  = 2 \p{N-1} \1_{N-1}
  - \sum_{s\in\set{\pm}} \varepsilon_s \op{\ut{k}^s}
\end{align}
where
\begin{align}
  \varepsilon_\pm \equiv N - 1 \mp 1,
  &&
  \ket{\ut{k}^\pm}
  \equiv \p{1\mp\f1{N-1}}^{-1/2}
  \f{\ket{\ut{0}}\pm\ket{\ut{k}}}{\sqrt{2}}.
\end{align}
The vectors $\ket{\ut{k}^\pm}$ are orthonormal when $k\ne0$.  When
$k=0$, meanwhile, we have
\begin{align}
  \ut{\check{\m h}}_0 = 2\p{N-1} \p{\1_{N-1} - \op{\ut{0}}}.
\end{align}
Remembering from \eqref{eq:cycle_vecs} that vectors $\ket{\ut{k}^\pm}$
are associated with the cycle matrices $\m m_{dk}$ defined in
\eqref{eq:cycle} via
\begin{align}
  \ket{\ut{k}^\pm}
  \propto \ket{\ut{0}} \pm \ket{\ut{k}}
  \propto \sum_{d\ne0} \p{1 \pm e^{ik\c d}} \ket{d}
  \to \sum_{d\ne0} \p{1 \pm e^{ik\c d}} \m m_{dk},
\end{align}
one can work out that the vectors $\ket{\ut{k}^\pm}$ correspond to
matrices that are strictly symmetric ($+$) or anti-symmetric ($-$)
under transposition.  The two-body perturbations considered in this
work, however, are defined using coefficient vectors
$\set{\v{\m w}_O}$ that correspond to symmetric matrices
$\set{\m w_O}$, with $w_{Opq}=w_{Oqp}$.  We can therefore neglect the
eigenvectors $\ket{\ut{k}^-}$ and corresponding eigenvalues
$\varepsilon_-$ entirely for the purposes of this work.

Altogether, the excitation energies corresponding to non-trivial
eigenstates of $\ut{\check{\m h}}_k$ are
\begin{align}
  \Delta_0 \equiv 0,
  &&
  \Delta_+ \equiv 2\p{N-1} - \varepsilon_+ = N,
\end{align}
while the energy of all orthogonal two-body excitations is
\begin{align}
  \Delta_\perp \equiv 2\p{N-1}.
\end{align}

%%%%%%%%%%%%%%%%%%%%%%%%%%%%%%%%%%%%%%%%%%%%%%%%%%%%%%%%%%%%%%%%%%%%%%
\section{Operator products restricted to the permutationally symmetric
  manifold}
\label{sec:sym_prod}

\subsection{The general case}

Given a system of $N$ spins, we wish to project a product of
multi-body operators onto the permutationally symmetric manifold.  For
$p$ multi-body operators, the projection of such a product takes the
form
\begin{align}
  \P_0 \sp{\prod_{j\in\ZZ_p}
    \sum_{k\in\C_N\p{M_j}} w_j\p{k} O_j\p{k}} \P_0,
  \label{eq:sym_prod_proj}
\end{align}
where $\P_0$ is a projector onto the permutationally symmetric
manifold; $k\in\C_N\p{M_j}$ is a choice of $M_j$ distinct spins;
$O_j\p{k}$ denotes the action of an $M_j$-spin operator on spins $k$
that is invariant under permutations of $k$; and $w_j\p{k}$ is a
scalar coefficient for $O_j\p{k}$.

To simplify the product in \eqref{eq:sym_prod_proj}, we first write
\begin{align}
  \prod_{j\in\ZZ_p} \sum_{k\in\C_N\p{M_j}} w_j\p{k} O_j\p{k}
  = \prod_{j\in\ZZ_p} \f1{M_j!} \sum_{k\in\ZZ_N^{M_j}}
  w_j\p{k} O_j\p{k},
  \label{eq:sym_prod_vec}
\end{align}
where we extend the definitions of $w_j\p{k}$ and $O_j\p{k}$ to all
vectors $k\in\ZZ_N^{M_j}$ by defining symmetric dimension-$M_j$
tensors $w_j$ and $O_j$ (i.e.~tensors with $M_j$ indices, with $O_j$
an {\it operator-valued} tensor) for which $w_j\p{k}=O_j\p{k}=0$ if
$k$ contains any repeated symbols.  We then collect terms whose
operator content is equivalent up to a permutation of spins.  To this
end, we classify terms in \eqref{eq:sym_prod_vec} by the numbers $g_S$
of indices shared by all tensors $w_j,O_j$ with $j\in S\subset\ZZ_p$.
For example, a term of the form
$w_1\p{a,b,c} w_2\p{b,d,e} w_3\p{b,c,d,e}$ with distinct indices
$\p{a,b,c,d,e}\in\C_N\p{5}$ would have
\begin{align}
  g_{\set{1}} = 1,
  &&
  g_{\set{1,2,3}} = 1,
  &&
  g_{\set{1,3}} = 1,
  &&
  g_{\set{2,3}} = 2,
\end{align}
and $g_S=0$ for all other subsets $S\subset\ZZ_3$.  This index
assignment can be represented by the Venn diagram
\begin{align}
  \diagram{example_123},
  \label{eq:venn_diagram}
\end{align}
where $g_S$ is determined the number of dots at the intersection of
circles $j\in S$.  We denote an assignment of distinct indices
according to choices of $g_S$ for all $S\in\PP\p{\ZZ_p}$ by $g$, where
$\PP\p{\ZZ_p}$ is the power set (i.e.~set of all subsets) of $\ZZ_p$.
Given a vector $\m M$ of the dimensions $M_j$, we then denote the set
of all valid index assignments $g$ by $\G\p{\m M}$.  For consistency,
all valid index assignments $g\in\G\p{\m M}$ must assign exactly $M_j$
indices to the dimension-$M_j$ tensor $w_j$, i.e.~
\begin{align}
  \sum_{S\in\PP\p{\ZZ_p}\,:\,j\in S} g_S
  = \sum_{R\in\PP\p{\ZZ_p\setminus\set{j}}} g_{\set{j}\cup R}
  = M_j
\end{align}
for all $j\in\ZZ_p$.  In order to make the index assignments
determined by choices of $g_S$ for all $S\in\PP\p{\ZZ_p}$ unique, we
also enforce that $g_{\set{}}=0$ for all $g\in\G\p{\m M}$, which is
equivalent to saying that every dot in the corresponding Venn diagram
must lie within some circle.  The set of valid index assignments
$\G\p{\m M}$ is then essentially the set of all Venn diagrams of the
form in \eqref{eq:venn_diagram} with $p=\dim\p{\m M}$ circles and a
total of $M_j$ dots within circle $j$.

A classification of terms in \eqref{eq:sym_prod_vec} by index
assignments $g\in\G\p{\m M}$ allows us to expand
\begin{align}
  \prod_{j\in\ZZ_p} \sum_{k\in\ZZ_N^{M_j}} w_j\p{k} O_j\p{k}
  \EQPS \sum_{g\in\G\p{\m M}} \S\p{g} w\p{g} O\p{g},
  \label{eq:sym_prod_group_start}
\end{align}
where $\EQPS$ denotes equality up to a restriction to the fully
symmetric manifold, $O\p{g}$ is an operator acquired by assigning
indices to the tensors $O_j$ in a manner consistent with $g$; $w\p{g}$
is a scalar acquired by summing over all indices assigned to the
tensors $w_j$ according to $g$; and $\S\p{g}$ is a symmetry factor
accounting for the number of equivalent ways to assign indices
according to $g$.

In order to write out the factors in \eqref{eq:sym_prod_group_start}
explicitly, we identify the set of values that the distinct indices
assigned according to $g$ can take:
\begin{align}
  \C_N\p{g}
  \equiv \set{ k \in \bigotimes_{S\in\PP\p{\ZZ_p}} \C_N\p{g_S}
    : \t{all elements of $k$ are distinct} },
  \label{eq:index_values}
\end{align}
where each $k\in\C_N\p{g}$ has the natural decomposition
$k=\p{k_S:S\in\PP\p{\ZZ_p}}$ with $k_S\in\C_N\p{g_S}$.  In words,
$\C_N\p{g}$ consists of all the ways to choose, for each subset $S$ of
the tensors $\set{w_j}$, a set $k_S$ of $g_S$ distinct spins, with the
restriction that all $k_S$ are orthogonal: if $z\in k_S$, then
$z\notin k_R$ for $R\ne S$.  Having defined the set $\C_N\p{g}$ of
index values, we define the restriction of $k\in\C_N\p{g}$ to values
associated with $w_j$ and $O_j$:
\begin{align}
  k_j \equiv \bigcup_{S\in\PP\p{\ZZ_p} : j\in S} k_S
  = \bigcup_{R\in\PP\p{\ZZ_p\setminus\set{j}}} k_{\set{j}\cup R},
\end{align}
where the union of lists $k_S$ denotes a concatenation of those lists.
These definitions allow us to expand
\begin{align}
  w\p{g} \equiv \sum_{k\in\C_N\p{g}} \prod_{j\in\ZZ_p} w_j\p{k_j},
  &&
  O\p{g} \equiv \prod_{j\in\ZZ_p}
  O_j\p{\ell_j}~\t{for some}~\ell\in\C_N\p{g},
  \label{eq:diagram_factors}
\end{align}
where the choice of $\ell\in\C_N\p{g}$ does not matter after
projection onto permutationally symmetric manifold, and
\begin{align}
  \S\p{g} \equiv \sp{\prod_{j\in\ZZ_p} { M_j \choose g_S : j\in S }}
  \sp{\prod_{S\in\PP\p{\ZZ_p}} \p{g_S!}^\abs{S}}
  = \prod_{j\in\ZZ_p} M_j!,
  \label{eq:sym_prod_symmetry}
\end{align}
where the first product in \eqref{eq:sym_prod_symmetry} contains
multinomial coefficients
\begin{align}
  { M \choose a_1, a_2, \cdots, a_m }
  \equiv \f{M!}{a_1!a_2!\cdots a_m!}
\end{align}
that account for the number of ways to partition the indices of each
$w_j,O_j$ into sets of shared indices as specified by the values of
$g_S$ for all $j\in S$, and the second product in
\eqref{eq:sym_prod_symmetry} accounts for the number of ways to
permute each set of shared indices on every tensor.  Altogether, the
symmetry factors $\S\p{g}$ cancel out with the $\sim M_j!$ prefactors
in \eqref{eq:sym_prod_vec}, so
\begin{align}
  \prod_{j\in\ZZ_p} \sum_{k\in\C_N\p{M_j}} w_j\p{k} O_j\p{k}
  \EQPS \sum_{g\in\G\p{\m M}} w\p{g} O\p{g}.
  \label{eq:sym_prod_group}
\end{align}

%%%%%%%%%%%%%%%%%%%%%%%%%%%%%%%%%%%%%%%%%%%%%%%%%%
\subsection{Formalizing diagrams and reducing computational costs}
\label{sec:diagrams}

Diagrams such as \eqref{eq:venn_diagram} will play a major role in the
calculations in this section.  Here, we define these diagrams as
formal objects that we can use in mathematical expressions, enabling
us to systematically perform calculations that are otherwise
intractable.  We then derive rules to ``simplify'' these diagrams in
such a way as to reduce the computational cost of their numerical
evaluation.

The simplest diagrams have $g_S$ dots (``$\bullet$'') at the
intersection of the circles $j\in S\in\PP\p{\ZZ_p}$, and are simply
equal to the associated scalar $w\p{g}$ defined in
\eqref{eq:diagram_factors}, e.g.
\begin{align}
  \diagram{example_123}
  \equiv \sum_{\substack{a,b,c,\in\ZZ_N\\\p{d,e}\in\C_N\p{2}\\
      \abs{\set{a,b,c,d,e}}=5}}
  w_1\p{a,b,c} w_2\p{b,d,e} w_3\p{b,c,d,e}.
\end{align}
Such a diagram with $\abs{g}$ dots nominally takes $\O\p{N^{\abs{g}}}$
time to compute.  In practice, this computational cost can be
prohibitive for performing numerical simulations of systems with any
appreciable size.  We therefore wish to simplify diagrams in order to
reduce their computational cost as much as possible.  Our general
strategy will be to replace constrained sums that make all indices
interdependent (e.g.~$\abs{\set{a,b,c,d,e}}=5$ above) by unconstrained
sums that can be carried out in sequence.  In order to represent
different constraints, we need to define diagrams in which filled dots
(``$\bullet$'') may be replaced by empty dots (``$\circ$'') or crosses
(``$\bm\times$'').

If a region of a diagram contains $F$ filled dots, those dots
correspond to indices that sum over $\C_N\p{F}$ with the constraint
that they are not equal to any other filled dots in a diagram.  $E$
empty dots in a region, meanwhile, correspond to indices that are
summed over $\C_N\p{E}$ without any additional constraints, e.g.
\begin{align}
  \diagram{example_o}
  \equiv \sum_{\substack{\p{a,b}\in\C_N\p{2}\\e\in\ZZ_N\\
      \abs{\set{a,b,e}}=3}}
  \sum_{\substack{\p{c,d}\in\C_N\p{2}\\f\in\ZZ_N}}
  v\p{a,b,c,d,e} w\p{e,f}.
\end{align}
Empty dots are therefore entirely independent of filled dots.
Crosses, meanwhile, are {\it entirely dependent} on filled dots:
``cross indices'' are summed {\it only} over the values of ``filled
dot indices'', e.g.
\begin{align}
  \diagram{example_x}
  \equiv \sum_{\substack{a,e\in\ZZ_N\\\abs{\set{a,e}}=2}}
  \sum_{b,c\in\set{a,e}} \sum_{d\in\ZZ_N}
  u\p{a,b,c} v\p{b,d,e} w\p{b,c,e}.
\end{align}
Empty dots and crosses allow us to decompose diagrams with a high
computational cost into different diagrams with a lower computational
cost.  We will essentially use two tricks to decompose diagrams:
eliminating filled dots (which will add empty dots and crosses), and
eliminating crosses (which will move around filled dots).

%%%%%%%%%%%%%%%%%%%%%%%%%%%%%%%%%%%%%%%%%%%%%%%%%%
\subsubsection{Decomposing diagrams}

Given a diagram containing only dots (filled or empty), we can
decompose any filled dot into an empty dot and a cross.  For example,
\begin{align}
  \diagram{example_elim}
  = \diagram{example_elim_o}
  - \diagram{example_elim_x}
  \label{eq:example_elim}
\end{align}
where
\begin{align}
  \diagram{example_elim}
  \equiv \sum_{\substack{a,b,c\in\ZZ_N\\\abs{\set{a,b,c}}=2}}
  v\p{a,b} w\p{b,c},
\end{align}
\begin{align}
  \diagram{example_elim_o}
  \equiv \sum_{\substack{b,c\in\ZZ_N\\\abs{\set{b,c}}=2}}
  v\p{\circ,b} w\p{b,c},
  &&
  v\p{\circ,b} \equiv \sum_{a\in\ZZ_N} v\p{a,b},
  \label{eq:example_elim_o}
\end{align}
\begin{align}
  \diagram{example_elim_x}
  \equiv \sum_{\substack{b,c\in\ZZ_N\\\abs{\set{b,c}}=2}}
  \sum_{a\in\set{b,c}} v\p{a,b} w\p{b,c}.
\end{align}
The decomposition in \eqref{eq:example_elim} essentially breaks up a
sum over $a\in\ZZ_N\setminus\set{b,c}$ into a difference of sums over
$a\in\ZZ_N$ and $a\in\set{b,c}$.  The sum over $a\in\ZZ_N$ to compute
$v\p{\circ,b}$ has $\O\p{N^2}$ cost, and can be performed prior to
evaluating the diagram in \eqref{eq:example_elim_o}.  The
decomposition in \eqref{eq:example_elim} therefore splits one
$\O\p{N^3}$ diagram into two $\O\p{N^2}$ diagrams.

After decomposing a filled dot into an empty dot and a cross, the next
step is to immediately eliminate the cross.  To eliminate a cross from
a diagram, we first note that the corresponding index can ignore other
indices on tensors that contain the index; that is (pay attention to
the sum over $a$),
\begin{align}
  \diagram{example_elim_x}
  \equiv \sum_{\substack{b,c\in\ZZ_N\\\abs{\set{b,c}}=2}}
  \sum_{a\in\set{b,c}} v\p{a,b} w\p{b,c}
  = \sum_{\substack{b,c\in\ZZ_N\\\abs{\set{b,c}}=2}}
  \sum_{a\in\set{c}} v\p{a,b} w\p{b,c}.
\end{align}
For each remaining index addressed by a cross, meanwhile, the cross
simply forces the corresponding indices to be equal, which is
equivalent to moving a filled dot into a different region:
\begin{align}
  \diagram{example_elim_x}
  = 2 \diagram{example_elim_x_full}
  = 2 \sum_{\p{b,c}\in\C_N\p{2}} v\p{c,b} w\p{b,c}.
  \label{eq:example_elim_final}
\end{align}
The factor of 2 accounts for the fact that the left diagram sums over
$b,c\in\ZZ_N$, while the right diagram sums over
$\p{b,c}\in\C_N\p{2}$.  We can generally repeat the above process of
eliminating filled dots and crosses until only empty dots remain in a
diagram.  The only complication in carrying out this process is that
we have to manually keep track of symmetry factors at each step, such
as the factor of 2 in \eqref{eq:example_elim_final}.

%%%%%%%%%%%%%%%%%%%%%%%%%%%%%%%%%%%%%%%%%%%%%%%%%%
\subsubsection{Symmetry factors}

Here we discuss the rules governing symmetry factors that appear when
decomposing diagrams.  In general, eliminating a filled dot from a
region with $F$ filled dots makes a diagram pick up a factor of $1/F$,
e.g.
\begin{align}
  \diagram{example_sym}
  = \f14 \diagram{example_sym_o}
  - \f14 \diagram{example_sym_x}.
\end{align}
This factor can be derived by replacing the sum over $\C_N\p{F}$ for
the filled dots within the region by a sum over $\ZZ_N^{F}$, which
picks up a factor of $1/F!$ to account for the number of ways to
permute $F$ values.  Setting aside one filled dot for replacement by
an empty dot and a cross, the sum over $\ZZ_N^{F-1}$ for the remaining
dots can be changed back to a sum over $\C_N\p{F-1}$, picking up a
factor of $\p{F-1}!$.  The overall factor picked up throughout this
procedure is therefore $\p{F-1}!/F!=1/F$.  Similarly, adding an empty
dot to a region with $E$ empty dots makes the diagram pick up a factor
of $\p{E+1}!/E!=E+1$, so
\begin{align}
  \diagram{example_sym_o}
  = \f23 \diagram{example_sym_oo}
  - \f13 \diagram{example_sym_ox}.
\end{align}
Finally, eliminating a cross by moving a filled dot from an ``old''
region with $F_{\t{old}}$ filled dots into a ``new'' region with
$F_{\t{new}}$ filled dots picks up an overall factor of
$F_{\t{new}}+1$, so
\begin{align}
  \diagram{example_sym_x}
  = 2 \diagram{example_sym_x_elim},
\end{align}
where the factor of $1/F_{\t{old}}$ that is acquired from removing a
dot from the old region is canceled out by a factor of $F_{\t{old}}$
that accounts for the number of ways to move a dot from the old region
to the new one.

%%%%%%%%%%%%%%%%%%%%%%%%%%%%%%%%%%%%%%%%%%%%%%%%%%
\subsection{Two single-body operators}
\label{sec:PXYP}

Calculating the second-order effective Hamiltonian $H_1^{(2)}$ induced
on the permutationally symmetric manifold $\M_0$ by the perturbation
$\V_1$ in \eqref{eq:perturbations} requires us to simplify a product
of the form
\begin{align}
  \P_0 \sp{\sum_{p,q} v_p w_q X_p Y_q} \P_0
  \EQPS \diagram{single_body_0} X_1 Y_2
  + \diagram{single_body_1} X_1 Y_1,
  \label{eq:PXYP_start}
\end{align}
where $p,q$ index individual spins; $X,Y$ are single-spin operators;
and the coefficients $v_p,w_q$ satisfy \eqref{eq:cond_1} with
eigenvalue $\Delta\ne0$, which implies that they are mean-zero.
Defining $\v m \equiv \sum_p m_p \ket{p}$ for $m\in\set{v,w}$, we can
simplify
\begin{align}
  \diagram{single_body_1}
  \equiv \sum_p v_p w_p
  = \v v\c\v w,
\end{align}
and
\begin{align}
  \diagram{single_body_0}
  = \diagram{single_body_0_o} - \diagram{single_body_0_x}
  = \diagram{single_body_0_o} - \diagram{single_body_1}
  = - \v v \c\v w,
\end{align}
so
\begin{align}
  \sum_{p,q} v_p w_q X_p Y_q
  \EQPS - \v v\c\v w \p{X_1 Y_2 - X_1 Y_1},
\end{align}
In order to write this result in terms of collective operators
$\col{Z} \equiv \sum_p Z_p$, we expand
\begin{align}
  \col{X Y} \EQPS N X_1 Y_1,
  &&
  \col{X}\,\col{Y}
  = \sum_{p=q} X_p Y_q + \sum_{p\ne q} X_p Y_q
  \EQPS \col{XY} + N\p{N-1} X_1 Y_2,
\end{align}
which implies that
\begin{align}
  \sum_{p,q} v_p w_q X_p Y_q
  \EQPS - \f{\v v \c\v w}{N\p{N-1}}
  \p{\col{X}\,\col{Y} - N \col{XY}}.
\end{align}

%%%%%%%%%%%%%%%%%%%%%%%%%%%%%%%%%%%%%%%%%%%%%%%%%%
\subsection{Two two-body operators}
\label{sec:POQP}

Calculating the second-order effective Hamiltonian $H_2^{(2)}$ induced
on the permutationally symmetric manifold $\M_0$ by the perturbation
$\V_2$ in \eqref{eq:perturbations} requires us to simplify a product
of the form
\begin{multline}
  \P_0 \sp{\sum_{k<\ell} \sum_{p<q}
    v_{k\ell} w_{pq} O_{k\ell} Q_{pq}} \P_0 \\
  \EQPS \diagram{two_body_0} O_{1,2} Q_{3,4}
  + \diagram{two_body_1} O_{1,2} Q_{1,3}
  + \diagram{two_body_2} O_{1,2} Q_{1,2},
  \label{eq:POQP_start}
\end{multline}
where $k,\ell,p,q$ index individual spins; $O$ and $Q$ are
permutationally-symmetric two-spin operators; and the coefficients
$v_{k\ell},w_{pq}$ satisfy \eqref{eq:cond_2} with eigenvalue
$\Delta\ne0$, which implies that they are mean-zero.  Defining
\begin{align}
  \v{\m m} \equiv \sum_{p<q} m_{pq} \ket{\set{p,q}},
  &&
  \v m \equiv \sum_p m_p \ket{p},
  &&
  m_p \equiv \sum_q m_{pq},
\end{align}
for $m\in\set{v,w}$, we can write
\begin{align}
  \diagram{two_body_2} \equiv \sum_{\p{p,q}\in\C_N\p{2}} v_{pq} w_{pq}
  = \v{\m v} \c\v{\m w},
\end{align}
and simplify
\begin{align}
  \diagram{two_body_1}
  &= \sp{\diagram{two_body_1_o}} - \sp{\diagram{two_body_1_x}} \\
  &= \sp{\diagram{two_body_1_oo} - \diagram{two_body_1_ox}}
  - \sp{2\diagram{two_body_2}} \label{eq:mid_zero} \\
  &= \v v \c \v w - 2 \v{\m v} \c \v{\m w},
\end{align}
where the middle diagram in \eqref{eq:mid_zero} contains an empty sum
for the cross, and is therefore equal to zero.  Similarly neglecting
diagrams with empty sums and additionally using the fact that
$v_{pq},w_{pq}$ are mean-zero, we can work out
\begin{align}
  \diagram{two_body_0}
  &= \f12 \sp{\diagram{two_body_0_o}}
  - \f12 \sp{\diagram{two_body_0_x}} \\
  &= \f12 \sp{\diagram{two_body_0_oo} - \diagram{two_body_0_ox}}
  - \f12 \sp{\diagram{two_body_1}} \\
  &= \f12 \sp{-\diagram{two_body_1_o}}
  - \f12 \p{\v v\c\v w - 2\v{\m v}\c\v{\m w}} \\
  &= -\v v\c\v w + \v{\m v}\c\v{\m w}.
\end{align}
Altogether, we thus have that
\begin{multline}
  \sum_{k<\ell} \sum_{p<q} v_{k\ell} w_{pq} O_{k\ell} Q_{pq} \\
  \EQPS - \p{\v v \c \v w - \v{\m v} \c \v{\m w}} O_{1,2} Q_{3,4}
  + \p{\v v\c\v w - 2 \v{\m v} \c \v{\m w}} O_{1,2} Q_{1,3}
  + \v{\m v} \c \v{\m w}\, O_{1,2} Q_{1,2}.
  \label{eq:POQP}
\end{multline}

%%%%%%%%%%%%%%%%%%%%%%%%%%%%%%%%%%%%%%%%%%%%%%%%%%
\subsection{Three two-body Ising-like operators}

Here we simplify a product of three two-body operators of a particular
form projected onto the permutationally symmetric manifold:
\begin{align}
  \X \equiv \P_0 \sp{\sum_{k<\ell} \sum_{p<q} \sum_{r<s}
    u_{k\ell} v_{pq} w_{rs} X_k X_\ell X_p X_q X_r X_s} \P_0,
  &&
  X^2 = 1,
\end{align}
where $k,\ell,p,q,r,s$ index individual spins; $X$ is a single-spin
operator; and $u_{k\ell},v_{pq},w_{rs}$ are arbitrary scalar
coefficients with $m_{pq}=m_{qp}$ and $m_{pp}=0$ for each of
$m\in\set{u,v,w}$.  Collecting terms according to the number of $X$
operators that remain after accounting for $X^2=1$, we find that
\begin{align}
  \X \EQPS A_6 X^{\otimes 6} + A_4 X^{\otimes 4}
  + A_2 X^{\otimes 2} + A_0,
  \label{eq:triple_multi}
\end{align}
with
\begin{align}
  A_6 \equiv \diagram{triple_0},
  &&
  A_4 \equiv \diagram{triple_01} + \diagram{triple_1},
  &&
  A_0 \equiv \diagram{triple_0111},
  \label{eq:triple_64}
\end{align}
\begin{align}
  A_2 \equiv \diagram{triple_011} + \diagram{triple_02}
  + \diagram{triple_11} + \diagram{triple_2},
  \label{eq:triple_2}
\end{align}
where a diagram with unlabeled circles denotes a sum over all distinct
label assignments, e.g.
\begin{align}
  \diagram{triple_1} \equiv \diagram{triple_1_uvw},
  \label{eq:triple_uvw}
\end{align}
\begin{align}
  \diagram{triple_011}
  \equiv \diagram{triple_011_uvw}
  + \diagram{triple_011_vwu} + \diagram{triple_011_wuv}.
  \label{eq:triple_uvw_3}
\end{align}
The first diagram in each of $A_6,A_4,A_2$ as written in
\eqref{eq:triple_64}, \eqref{eq:triple_2} thus has only one assignment
of labels, as in \eqref{eq:triple_uvw}, while all other unlabeled
diagrams have three assignments, as in \eqref{eq:triple_uvw_3}.

Having identified diagrammatic representations of the coefficients
$A_6,A_4,A_2$, we now need to simplify all diagrams to reduce their
computational complexity, as discussed in Section \ref{sec:diagrams}.
Simplifying all these diagrams by hand is tedious, but the process of
eliminating filled dots and crosses from a diagram is algorithmic
enough to execute on a computer, doing which we find
\begin{align}
  A_6 = D_0 - D_1 + D_2 - D_3,
  &&
  A_4 = D_1 - 2 D_2 + 3 D_3,
  &&
  A_2 = D_2 - 3 D_3,
  &&
  A_0 \equiv D_3,
\end{align}
where
\begin{align}
  D_0 \equiv \diagram{triple_0_o},
  &&
  D_1 \equiv \diagram{triple_01_o} - 2 \diagram{triple_1_o},
  &&
  D_3 \equiv \diagram{triple_0111_o},
\end{align}
\begin{align}
  D_2 \equiv \diagram{triple_011_o}
  + \diagram{triple_02_o}
  - 2 \diagram{triple_11_o}
  + 4 \diagram{triple_2_o}.
\end{align}
Defining, for each of $m\in\set{u,v,w}$,
\begin{align}
  \m m \equiv \sum_{p,q} m_{pq} \op{p}{q},
  &&
  \v{\m m} \equiv \sum_{p<q} m_{pq} \ket{\set{p,q}},
  &&
  \v m \equiv \sum_p m_p \ket{p},
  &&
  m_p \equiv \sum_q m_{pq},
  &&
  \col{m} \equiv \sum_{p<q} m_{pq},
\end{align}
we can expand
\begin{align}
  D_0 = \col{u}\,\col{v}\,\col{w},
  &&
  D_1 = \col{u}\,\v v\c\v w + \col{v}\,\v w\c\v u
  + \col{w}\,\v u\c\v v - 2 \sum_p u_p v_p w_p,
  &&
  D_3 = \tr\p{\m u \c \m v \c \m w},
\end{align}
\begin{multline}
  D_2 = \v u \c\m v\c\v w + \v v \c\m w\c\v u + \v w \c\m u\c\v v
  + \col{u}\,\v{\m v}\c\v{\m w} + \col{v}\,\v{\m w}\c\v{\m u}
  + \col{w}\,\v{\m u}\c\v{\m v} \\
  - 2 \sum_{p,q} \p{u_p v_{pq} w_{pq}
    + v_p w_{pq} u_{pq} + w_p u_{pq} v_{pq}}
  + 4 \sum_{p<q} u_{pq} v_{pq} w_{pq}.
\end{multline}
Finally, we can also write the product $\X$ in \eqref{eq:triple_multi}
in terms of the collective operator $\col{X} \equiv \sum_p X_p$:
\begin{align}
  \X \simeq_\PS
  \tilde A_6 \col{X}^6 + \tilde A_4 \col{X}^4 + \tilde A_2 \col{X}^2,
  \label{eq:triple_col}
\end{align}
where $\simeq_\PS$ denotes equality up to scalar terms and a
restriction to the permutationally symmetric manifold, and
\begin{align}
  \tilde A_6 \equiv \f{A_6}{N!_6},
  &&
  \tilde A_4 \equiv \f{A_4}{N!_4} - 5\p{3N-8} \f{A_6}{N!_6},
\end{align}
\begin{align}
  \tilde A_2 \equiv \f{A_2}{N!_2} - 2\p{3N-4} \f{A_4}{N!_4}
  + \p{45N^2-210N+184} \f{A_6}{N!_6}.
\end{align}

% %%%%%%%%%%%%%%%%%%%%%%%%%%%%%%%%%%%%%%%%%%%%%%%%%%%%%%%%%%%%%%%%%%%%%%
% \section{Collective and multi-local operators in the permutationally
%   symmetric manifold}
% \label{sec:multi_to_collective}

% Here we provide a decomposition multi-local operators into sums of
% collective operators (and vice versa) within the permutationally
% symmetric manifold of $N$ spins.  Multi-local operators are operators
% of the form
% \begin{align}
%   O_S \equiv \bigotimes_{O\in S} O,
%   \label{eq:multi_op_set}
% \end{align}
% where $S\equiv\set{O}$ is a set of single-spin operators $O$.  The
% order of tensor factors in \eqref{eq:multi_op_set} does not matter
% after projecting operators onto the permutationally symmetric
% manifold, as we always will in this section.  While the decomposition
% of a collective operator into a sum of multi-local operators is
% unique, the opposite is not true: there is generally no unique
% decomposition of a multi-local operator into a sum of collective
% operators.  We can, however, enforce a canonical decomposition by
% choosing a set of collective operators that obey the same symmetries
% as $O_S$, namely invariance under arbitrary permutations of the
% single-spin operators $O\in S$.

% Even though $S$ contains $\abs{S}$ spins, a collective operator built
% from the single-spin operators $O\in S$ may have multiple
% (single-spin) operators acting on the same spin, resulting in a
% $k$-local operator with $k<\abs{S}$.  We therefore {\it
%   partition}\footnote{A partition $P\equiv\set{p}$ of $S$ is a set of
%   mutually disjoint, non-empty subsets $p\subset S$ with
%   $\bigcup_{p\in P}p=S$.} $S$ into subsets of operators that act on
% the same spin.  For any partition $P\equiv\set{p}$ of $S$ (here
% $p\subset S$ is a subset of operators $O\in S$ that address the same
% spin), we then define the $\abs{P}$-local operator
% \begin{align}
%   O_P \equiv \bigotimes_{p\in P} \bigodot_{O\in p} O,
% \end{align}
% where $\bigodot_{x\in\X} x$ denotes a symmetrized product of all
% elements in $\X$, i.e.
% \begin{align}
%   \bigodot_{x\in\X} x
%   \equiv \f1{\abs{\X}!}
%   \sum_{\substack{\t{permutations}\\\sigma~\t{of}~\X}}
%   ~ \prod_{k=1}^{\abs{\X}} x_{\sigma_k},
% \end{align}
% with $\sigma_k$ the $k$-th element of the permutation $\sigma$ when
% expressed in one-line notation; for example,
% \begin{align}
%   \bigodot_{j=1}^2 x_j &= x_1 \odot x_2 = \f12 \p{x_1x_2 + x_2x_1}, \\
%   \bigodot_{j=1}^3 x_j
%   &= \f1{3!} \p{x_1x_2x_3 + x_1x_3x_2 + x_2x_1x_3
%     + x_2x_3x_1 + x_3x_1x_2 + x_3x_2x_1}.
% \end{align}
% The set of multi-local operators $\set{O_P:\t{partitions}~P~\t{of}~S}$
% consists of all operators that
% \begin{enumerate*}
% \item can be built from ordinary products and tensor products of all
%   single-spin operators $O\in S$, and
% \item are invariant under arbitrary permutations of $S$.
% \end{enumerate*}
% For any set $S$ of single-spin operators and any partition $P$ of $S$,
% we can similarly define the collective operators
% \begin{align}
%   \O_S \equiv \bigodot_{O\in S} \col{O},
%   &&
%   \O_P \equiv \bigodot_{p\in P} \col{\textstyle\bigodot_{O\in p} O},
% \end{align}
% where we underline a single-spin operator to denote its collective
% version, i.e.~$\col{Q}\equiv\sum_j Q^{(j)}$, where $Q^{(j)}$ is an
% operator that acts with the single-spin operator $Q$ on spin $j$ and
% trivially on all other spins.  By construction, the set of collective
% operators $\set{\O_P:\t{partitions}~P~\t{of}~S}$ is one-to-one to the
% set of multi-local operators $\set{O_P:\t{partitions}~P~\t{of}~S}$,
% and it is straightforward to work out that for any set $S$ of
% single-spin operators,
% \begin{align}
%   \O_S
%   \EQPS \sum_{\substack{\t{partitions}\\P~\t{of}~S}}
%   N!_{\abs{P}} O_P,
%   &&
%   N!_k \equiv \prod_{j=0}^{k-1}\p{N-j}! = \f{N!}{\p{N-k}!},
%   \label{eq:collective_to_multi_full}
% \end{align}
% where $\EQPS$ denotes equality under a restriction to the fully
% symmetric manifold.  For any multi-local operator $O_S$ indexed by a
% set $S$ of single-spin operators, we can collect expansions of the
% form \eqref{eq:collective_to_multi_full} for all collective operators
% built from non-empty subsets of $S$ into a system of linear equations
% $\L$, and in turn solve $\L$ for $O_S$.  When
% $\abs{S}\in\set{1,2,3,4}$ (and, by conjecture, for all $S$), by use of
% a computer algebra system (namely, Mathematica) we find that
% \begin{align}
%   N!_{\abs{S}} O_S
%   &\EQPS \sum_{\substack{\t{partitions}\\P~\t{of}~S}}
%   \p{-1}^{\abs{S}-\abs{P}} \O_P \times \prod_{p\in P}\p{\abs{p}-1}!
%   \label{eq:multi_to_collective_full} \\
%   &\EQPS \f1{\abs{S}!} \sum_{\substack{\t{permutations}\\\Pi~\t{of}~S}}
%   \sum_{\substack{\t{ordered}\\\t{partitions}\\\tilde P~\t{of}~\Pi}}
%   \p{-1}^{\abs{S}-\abs{\tilde P}} \O_{\tilde P}
%   \times \prod_{\tilde p\in\tilde P} \p{\abs{\tilde p}-1}!
%   \label{eq:multi_to_collective_full_simp}
% \end{align}
% where we identify permutations $\Pi$ of $S$ with an ordered list
% (tuple) $\p{O_1,O_2,\cdots,O_{\abs{S}}}$ of all single-spin operators
% in $S$; and we enforce that partitions $\tilde P$ of an ordered list
% $\Pi$, as well as parts $\tilde p\in\tilde P$, are lexicographically
% ordered in accordance with the ordering of operators in $\Pi$.  If
% $S=\set{X,Y,Z}$, for example, then the ordered partitions of the
% trivial permutation $\Pi=\p{X,Y,Z}$ are
% \begin{align}
%   \p{\p{X,Y,Z}}, &&
%   \p{\p{X,Y},\p{Z}}, &&
%   \p{\p{X,Z},\p{Y}}, &&
%   \p{\p{X},\p{Y,Z}}, &&
%   \p{\p{X},\p{Y},\p{Z}}.
% \end{align}
% For ordered lists $\tilde S$ and ordered partitions $\tilde P$, we
% define
% \begin{align}
%   O_{\tilde S} \equiv \bigotimes_{O\in\tilde S} O,
%   &&
%   O_{\tilde P}
%   \equiv \bigotimes_{\tilde p\in\tilde P} \prod_{O\in\tilde p} O,
%   &&
%   \O_{\tilde S} \equiv \prod_{O\in\tilde S} \col{O},
%   &&
%   \O_{\tilde P}
%   \equiv \prod_{\tilde p\in\tilde P}
%   \col{\textstyle\prod_{O\in\tilde p} O},
% \end{align}
% where products over elements of $\tilde S,\tilde P,\tilde p$ are taken
% in order according to the elements of these ordered lists, e.g.~if
% $\tilde S=\p{O_1,O_2,\cdots,O_{\abs{S}}}$, then
% $\prod_{O\in\tilde S}\equiv O_1O_2\cdots O_{\abs{S}}$.  These
% definitions allow us to expand
% \begin{align}
%   \O_{\tilde S} \EQPS
%   \sum_{\substack{\t{ordered}\\\t{partitions}\\\tilde P~\t{of}~\tilde S}}
%   N!_{\abs{\tilde P}} O_{\tilde P},
% \end{align}
% in analogy to \eqref{eq:collective_to_multi_full}.

% While the expansions in \eqref{eq:multi_to_collective_full} and
% \eqref{eq:multi_to_collective_full_simp} closely resemble each other,
% in practice the latter involves less computational overhead due to the
% lack of symmetrization in the construction of $\O_{\tilde P}$.  There
% are, in principle, even more economical decompositions of a
% multi-local operators into collective opreators, for example
% \begin{align}
%   N!_2 X\otimes Y
%   &\EQPS \col{X}\,\col{Y} - \col{XY}, \\
%   N!_3 X\otimes Y \otimes Z
%   &\EQPS \col{X}\,\col{Y}\,\col{Z}
%   - \col{X}\,\col{YZ} - \col{Y}\,\col{XZ} - \col{Z}\,\col{XY}
%   + \col{ZXY} + \col{YXZ},
% \end{align}
% but these decompositions generally do not preserve the manifest
% symmetry of a multi-local operator $O_S$ under arbitrary permutations
% of $S$, and we do not know of any general prescription to arrive at
% such decompositions aside from solving systems of equations relating
% collective operators to multi-local operators.

% On a final note, we consider the special case that all operators
% $O\in S$ are identical, for which we provide the expansions
% \begin{align}
%   N!_2 X\otimes X \EQPS \col{X}^2 - \col{X^2},
%   &&
%   N!_3 X^{\otimes 3} \EQPS \col{X}^3
%   - 3 \col{X} \odot \col{X^2} + 2\col{X^3},
% \end{align}
% \begin{align}
%   N!_4 X^{\otimes 4}
%   \EQPS \col{X}^4 - 4 \col{X}^2 \odot \col{X^2}
%   - 2 \col{X} \, \col{X}^2 \, \col{X}
%   + 3 \p{\col{X^2}}^2 + 8 \col{X} \odot \col{X^3}
%   - 6 \col{X^4}.
% \end{align}
% If furthermore $X^2=1$, as with e.g.~the Pauli operators of SU(2),
% then
% \begin{align}
%   N!_2 X\otimes X \EQPS \col{X}^2 - N,
%   &&
%   N!_3 X^{\otimes 3} \EQPS \col{X}^3 - \p{3N-2} \col{X},
% \end{align}
% \begin{align}
%   N!_4 X^{\otimes 4} \EQPS \col{X}^4
%   - 2\p{3N-4} \col{X}^2 + 3 N \p{N-2}.
% \end{align}

\bibliography{multilevel_spin_notes.bib}

\end{document}
