\documentclass[nofootinbib,notitlepage,11pt]{revtex4-2}

%%% linking references
\usepackage{hyperref}
\hypersetup{
  breaklinks=true,
  colorlinks=true,
  linkcolor=blue,
  filecolor=magenta,
  urlcolor=cyan,
}

%%% header / footer
\usepackage{fancyhdr} % easier header and footer management
\pagestyle{fancy} % page formatting style
\fancyhf{} % clear all header and footer text
\renewcommand{\headrulewidth}{0pt} % remove horizontal line in header
\usepackage{lastpage} % for referencing last page
\cfoot{\thepage~of \pageref{LastPage}} % "x of y" page labeling


%%% symbols, notations, etc.
\usepackage{physics,braket,bm,amssymb} % physics and math
\renewcommand{\t}{\text} % text in math mode
\newcommand{\f}[2]{\dfrac{#1}{#2}} % shorthand for fractions
\newcommand{\p}[1]{\left(#1\right)} % parenthesis
\renewcommand{\sp}[1]{\left[#1\right]} % square parenthesis
\renewcommand{\set}[1]{\left\{#1\right\}} % curly parenthesis
\newcommand{\bk}{\Braket} % shorthand for braket notation

\renewcommand{\c}{\cdot} % inner product
\renewcommand{\oc}{\circ} % element-wise product

\newcommand{\m}{\bm} % bold symbol
\renewcommand{\v}{\vec} % arrow vector

\usepackage{dsfont} % for identity operator
\newcommand{\1}{\mathds{1}}

\newcommand{\up}{\uparrow}
\newcommand{\dn}{\downarrow}

\renewcommand{\d}{\text{d}}
\newcommand{\x}{\text{x}}
\newcommand{\y}{\text{y}}
\newcommand{\z}{\text{z}}

\newcommand{\e}{\varepsilon}

\newcommand{\B}{\mathcal{B}}
\newcommand{\D}{\mathcal{D}}
\newcommand{\E}{\mathcal{E}}
\renewcommand{\H}{\mathcal{H}}
\newcommand{\I}{\mathcal{I}}
\newcommand{\M}{\mathcal{M}}
\newcommand{\N}{\mathcal{N}}
\renewcommand{\O}{\mathcal{O}}
\renewcommand{\P}{\mathcal{P}}
\newcommand{\Q}{\mathcal{Q}}
\newcommand{\R}{\mathcal{R}}
\newcommand{\T}{\mathcal{T}}
\renewcommand{\S}{\mathcal{S}}
\newcommand{\V}{\mathcal{V}}
\newcommand{\X}{\mathcal{X}}
\newcommand{\Z}{\mathcal{Z}}

\newcommand{\EE}{\mathbb{E}}
\newcommand{\RR}{\mathbb{R}}
\renewcommand{\SS}{\mathbb{S}}
\newcommand{\ZZ}{\mathbb{Z}}

\newcommand{\FS}{\text{FS}}

\DeclareMathOperator{\sign}{sign}
\DeclareMathOperator{\cov}{cov}
\let\var\relax
\DeclareMathOperator{\var}{var}
\DeclareMathOperator{\diag}{diag}

\def\obra#1{\mathinner{({#1}|}}
\def\oket#1{\mathinner{|{#1})}}
\def\obk#1{\mathinner{({#1})}}
\def\oop#1#2{\oket{#1}\!\obra{#2}}

\usepackage[inline]{enumitem} % in-line lists and \setlist{} (below)
\setlist[enumerate,1]{label={(\roman*)}} % default in-line numbering
\setlist{nolistsep} % more compact spacing between environments

%%% text markup
\usepackage{color} % text color
\newcommand{\red}[1]{{\color{red} #1}}

%%%%%%%%%%%%%%%%%%%%%%%%%%%%%%%%%%%%%%%%%%%%%%%%%%%%%%%%%%%%%%%%%%%%%%
\begin{document}

\title{Perturbing SU($n$)-symmetric interactions}%
\author{Michael A. Perlin}%
\date{\today}

\maketitle

\tableofcontents

\section{Introduction}

We consider an array of $N$ multilevel spins with non-uniform
SU($n$)-symmetric interactions that can be written in the form
\begin{align}
  H_0 = \sum_{p<q} h_{pq} \Pi_{pq},
  &&
  \Pi_{pq} \equiv \sum_{\mu,\nu} S_{\mu\nu}^{(p)} S_{\nu\mu}^{(q)},
  \label{eq:ints}
\end{align}
where $S_{\mu\nu}^{(p)}\equiv\op{\mu}{\nu}_p$ flips the state of spin
$p$ to $\ket\mu$ from $\ket\nu$; the operator $\Pi_{pq}$ permutes
spins $p$ and $q$; and the $h_{pq}$ in the Hamiltonian are all
negative: $h_{pq}<0$.

The ground-state manifold $\M_0$ of the interaction Hamiltonian $H_0$
consists of fully symmetric states that are simultaneous $+1$
eigenstates of all permutation operators $\Pi_{pq}$.  We will assume
throughout these notes that the interaction Hamiltonian $H_0$ is
gapped, i.e.~that there is a finite energy difference between $\M_0$
and its orthogonal complement.  The energy of fully symmetric states
$\ket\psi\in\M_0$ is
\begin{align}
  E_0 \equiv \sum_{p<q} h_{pq}
  = \f12 \sum_{p,q} h_{pq}
  = \f12 \sum_p h_p
  = \f12 N h,
\end{align}
where for convenience define coefficients $h_{pq}$ for all $p,q$ (as
opposed to only $p<q$) by enforcing $h_{pq}=h_{qp}$ and $h_{pp}=0$;
and
\begin{align}
  h_p \equiv \sum_q h_{pq},
  &&
  h \equiv \EE_p\sp{h_p} = \f1N \sum_p h_p.
\end{align}
We wish to determine the effective dynamics induced on the fully
symmetric manifold $\M_0$ by weak single- and two-body perturbations
of the form
\begin{align}
  \V_1 \equiv \sum_p v_p V_p = \v v\c\v V,
  &&
  \V_2 \equiv \sum_{p<q} w_{pq} X_p Y_q
  = \f12 \sum_{p,q} w_{pq} X_p Y_q
  = \f12 \v X \c\m w \c\v Y,
  \label{eq:perturbations}
\end{align}
where $V_p,X_p,Y_p$ are trace-zero operators on spin $p$;
$\v V,\v X,\v Y$ are respectively vectors of all $V_p,X_p,Y_p$; $\v v$
and $\m w$ are respectively a vector and matrix of all $v_p,w_{pq}$;
we enforce $w_{pq}=w_{qp}$ and $w_{pp}=0$; and we use a dot ($\c$) to
denote an inner product between matrices or vectors as appropriate.

The effective Hamiltonian $H_M$ induced on the ground-state manifold
$\M_0$ by an $M$-body perturbation $\V_M$ through second order in
perturbation theory is given by\cite{bravyi2011schrieffer,
  perlin2019effective}
\begin{align}
  H_M = H_M^{(1)} + H_M^{(2)},
  &&
  H_M^{(1)} = \P_0 \V_M \P_0,
  &&
  H_M^{(2)} = - \P_0 \V_M \E \V_M \P_0,
  &&
  \E \equiv \sum_{\Delta>0} \f{\P_\Delta}{\Delta},
\end{align}
where $\P_\Delta$ is a projector onto the eigenspace of the
interaction Hamiltonian $H_0$, with interaction energy $\Delta$
above that of fully symmetric manifold $\M_0$.

%%%%%%%%%%%%%%%%%%%%%%%%%%%%%%%%%%%%%%%%%%%%%%%%%%%%%%%%%%%%%%%%%%%%%%
\section{Single-body perturbations}

To compute the effective Hamiltonian induced by the single-body
perturbation $\V_1$, we expand
\begin{align}
  H_1^{(1)} = \v v \c \P_0 \v V \P_0,
\end{align}
and use the permutational symmetry of the ground-state manifold to
simplify
\begin{align}
  H_1^{(1)} =  \sum_p v_p \P_0 V_0 \P_0 \simeq v V,
  &&
  v \equiv \EE_p\sp{v_p},
  &&
  V \equiv \sum_p V_p,
  \label{eq:H_1_1}
\end{align}
where $\simeq$ denotes equality up to a restriction to the
ground-state manifold $\M_0$.  In order to compute the second order
effective Hamiltonian $H_1^{(2)}$, we first choose an arbitrary fully
symmetric state $\ket\psi\in\M_0$ and expand (see Appendix
\ref{sec:H_V1_psi}):
\begin{align}
  H_0 \V_1 \ket\psi
  = E_0 \V_1 \ket\psi
  + \p{\v v\c\m h - \v v\oc\v h}\c \v V \ket\psi,
  \label{eq:H_V1_psi_start}
\end{align}
where $\m h$ is a matrix of all coefficients $h_{pq}$ in the
interaction Hamiltonian $H_0$ in \eqref{eq:ints}, and $\v v\oc\v h$ is
the element-wise (Hadamard) product of $\v v$ and $\v h$, i.e.~a
vector with components $v_kh_k$.  We thus find that the vector
$\V_1\ket\psi$ is an eigenvector of the interaction Hamiltonian $H_0$
if the coefficients $\v v$ satisfy
\begin{align}
  \m h \c \v v - \v h\oc\v v
  = \p{\m h - \diag\v h}\c\v v
  = \Delta\p{\v v} \v v,
  \label{eq:cond_1}
\end{align}
where $\diag\v h$ is a matrix with $\v h$ on the diagonal and zeroes
everywhere else; and $\Delta\p{\v v}$ is some constant that may depend
on $\v v$.  If the interaction Hamiltonian $H_0$ is translationally
invariant, then all $h_q=h$ and $\v h\oc\v v=h\v v$, so the condition
in \eqref{eq:cond_1} simplifies to
\begin{align}
  \m h \c\v v = \sp{\Delta\p{\v v}+h} \v v.
\end{align}
If we {\it construct} a perturbation $\V_1\p{\v v_\Delta}$ with
coefficients $\v v_\Delta$ that satisfy \eqref{eq:cond_1} with
eigenvalue $\Delta$, then
\begin{align}
  H_0 \V_1\p{\v v_\Delta} \ket\psi
  = \sp{E_0 + \Delta} \V_1\p{\v v_\Delta} \ket\psi.
  \label{eq:H_V1_psi}
\end{align}
Interestingly, the energy of the (unnormalized) state
$\V_1\p{\v v}\ket\psi$ depends only on the coefficients $\v v$, and is
entirely independent of the state $\ket\psi$ or choice of single-spin
trace-zero operator used to build $\V_1$ in \eqref{eq:perturbations}.
Finding operators $\V_1\p{\v v}$ that generate eigenvectors of the
interaction Hamiltonian $H_0$ when they act on fully symmetric states
$\ket\psi$ thus reduces to finding eigenvectors $\v v$ of the
interaction matrix $\m h-\diag\v h$.

If the interaction Hamiltonian $H_0$ is gapped, then a vector
$\V_1\p{\v v_\Delta}\ket\psi$ with $\Delta=0$ must lie within the
fully symmetric manifold $\M_0$, which implies that the operator
$\V_1\p{\v v_\Delta}$ preserves the permutational symmetry of $\M_0$.
The only way for $\V_1\p{\v v_\Delta}$ to obey permutational symmetry
is for $\v v_\Delta$ to be a constant vector,
i.e.~$\v v_\Delta=v_\Delta\v\1$ with $\1$ a constant vector of ones.
Indeed, a constant vector $\v\1$ satisfies the condition in
\eqref{eq:cond_1} with eigenvalue $0$.  All other vectors
$\v v_\Delta$ satisfying \eqref{eq:cond_1} with $\Delta\ne0$ must be
orthogonal to $\v\1$, and therefore mean-zero.

We now return to the task of computing the second-order effective
Hamiltonian $H_1^{(2)}$.  Any coefficient vector $\v v$ can be
expanded into its projections $\v v_\Delta$ onto the eigenspace of
vectors satisfying \eqref{eq:cond_1} with eigenvalue $\Delta$,
i.e.~$\v v = \sum_\Delta \v v_\Delta$, which also allows us to expand
\begin{align}
  \V_1\p{\v v} = \sum_\Delta \V_1\p{\v v_\Delta},
\end{align}
where each operator $\V_1\p{\v v_\Delta}$ generates a state with
interaction energy $\Delta$ above that of the fully symmetric
manifold.  We can therefore simplify
\begin{align}
  H_1^{(2)}
  = -\P_0 \V_1\p{\v v} \E \V_1\p{\v v} \P_0
  = - \sum_{\Delta>0} \f1{\Delta}
  \P_0 \V_1\p{\v v} \P_\Delta \V_1\p{\v v} \P_0
  = - \sum_{\Delta>0} \f1{\Delta} \P_0 \V_1\p{\v v_\Delta}^2 \P_0,
\end{align}
where the the product $\P_0 \V_1\p{\v v_\Delta}^2 \P_0$ is worked out
in Appendix \ref{sec:PVVP}, giving us
\begin{align}
  H_1^{(2)} \simeq \f1{N-1} \sum_{\Delta>0}
  \f{\var\p{\v v_\Delta}}{\Delta} \p{V^2 - N W}.
  \label{eq:H_1_2}
\end{align}
Here $\simeq$ denotes equality up to a restriction to the fully
symmetric manifold $\M_0$, and
\begin{align}
  \var\p{\v x} \equiv \EE_p\sp{\p{x_p-\EE_q\sp{x_q}}^2}
  = \EE_p\sp{x_p^2} - \EE_q\sp{x_q}^2,
  &&
  W \equiv \sum_p V_p^2,
\end{align}
with $\EE_p\sp{x_p}\equiv\sum_px_p/N$ the mean of $\v x$.

%%%%%%%%%%%%%%%%%%%%%%%%%%%%%%%%%%%%%%%%%%%%%%%%%%%%%%%%%%%%%%%%%%%%%%
\section{Two-body perturbations}

The first-order effective Hamiltonian $H_2^{(1)}$ induced on the fully
symmetric manifold $\M_0$ by the two-body perturbation $\V_2$ in
\eqref{eq:perturbations} can be computed using the permutational
symmetry of $\M_0$ (see Appendix \ref{sec:PXYP}):
\begin{align}
  H_2^{(1)}
  = \f12 \P_0 \v X\c\m w\c\v Y\P_0
  \simeq \f12 w \p{\tilde Z - Z},
\end{align}
where in terms of the anti-commutator $\sp{A,B}_+\equiv AB+BA$ and
collective operators $X\equiv\sum_p X_p$ and $Y\equiv\sum_p Y_p$, we
define
\begin{align}
  w \equiv {N\choose 2}^{-1} \sum_{p<q} w_{pq},
  &&
  \tilde Z \equiv \f12 \sp{X,Y}_+
  &&
  Z \equiv \sum_p \f12 \sp{X_p, Y_p}_+.
\end{align}
In order to compute the second-order effective Hamiltonian
$H_2^{(2)}$, as before we pick an arbitrary fully symmetric state
$\ket\psi\in\M_0$ and expand (see Appendix \ref{sec:H_V2_psi})
\begin{align}
  H_0 \V_2 \ket\psi
  = E_0 \V_2 \ket\psi
  + \v X \c \p{\bar{\m h}\oc\m w + \m h\c\m w} \c \v Y \ket\psi
  - \v\1 \c \p{\m h\oc\m w} \c \v Z \ket\psi,
  \label{eq:H_V2_psi}
\end{align}
where $\v\1$ is an $N$-component vector of ones,
\begin{align}
  \bar{\m h}
  \equiv \m h - \f12\p{\v h\otimes\v\1 + \v\1\otimes\v h},
\end{align}
is a modified interaction matrix with entries $h_{pq}-\p{h_p+h_q}/2$;
and $\m x\oc\m y$ is an element-wise (Hadamard) product of $\m x$ and
$\m y$, with entries $x_{pq}y_{pq}$.  We thus find that the vector
$\V_2\ket\psi$ is an eigenvector of the interaction Hamiltonian $H_0$
if the matrix $\m w$ of all $w_{pq}$ satisfies
\begin{align}
  \bar{\m h}\oc\m w + \m h\c\m w = \f12 \Delta\p{\m w} \m w,
  \label{eq:cond_2}
\end{align}
for some constant $\Delta\p{\m w}$ that may depend on $\m w$; $\v\1$
is a constant vector of ones.  Remembering that both $\m h$ and $\m w$
are both symmetric (under transposition) with zeros on the diagonal,
the diagonal components of the condition in \eqref{eq:cond_2}
automatically enforce that the last term in \eqref{eq:H_V2_psi}
vanishes.

We can re-cast the condition in \eqref{eq:cond_2} as an eigenvalue
problem by writing the coefficient matrix $\m w$ as a vector in
$\RR^{N^2}$,
\begin{align}
  \m w = \sum_{p,q} w_{pq} \op{p}{q}
  \to \v{\m w} \equiv \sum_{p,q} w_{pq} \ket{pq},
\end{align}
and defining the matrix
\begin{align}
  \check{\m h}
  \equiv \sum_{p,q} \bar h_{pq} \op{pq}
  + \sum_{p,q,\ell} h_{pq} \op{p\ell}{q\ell},
  \label{eq:h_super_mat}
\end{align}
in terms of which the condition in \eqref{eq:cond_2} becomes the
eigenvalue equation
\begin{align}
  \check{\m h} \c \v{\m w} = \f12 \Delta\p{\m w} \v{\m w}.
  \label{eq:cond_2_eig}
\end{align}
The matrix $\check{\m h}$ in \eqref{eq:h_super_mat} can be written in
the block-diagonal form
\begin{align}
  \check{\m h} = \sum_\ell \m h_\ell \otimes \op{\ell},
  &&
  \m h_\ell
  \equiv \sum_{p,q}\p{\delta_{pq} \bar h_{p\ell} + h_{pq}} \op{p}{q},
\end{align}
which reduces the eigenvalue equation in \eqref{eq:cond_2_eig} to the
$N$ smaller, uncoupled eigenvalue equations
\begin{align}
  \m h_\ell \c \v{\m w}_\ell = \f12 \Delta\p{\m w} \v w_\ell,
  &&
  \v w_\ell \equiv \sum_p w_{p\ell} \ket{p}.
  \label{eq:cond_2_eig_block}
\end{align}
We can thus (numerically) diagonalize the blocks $\m h_\ell$
independently, and classify their eigenvectors $\v w_\ell$ by the
corresponding eigenvalues $\Delta\p{\m w}/2$.  ``Total'' eigenvectors
$\v{\m w}$ of $\check{\m h}$ are then built from eigenvectors
$\v{\m w}_\ell$ within each block $\m h_\ell$ that share the same
eigenvalue $\Delta\p{\m w}/2$.  The task of constructing eigenvectors
$\V_2\p{\m w}\ket\psi$ of $H_0$ and determining their energy thus
reduces to finding the eigenvectors and eigenvalues of $N$ matrices
$\m h_\ell$ of size $N\times N$.

If the interaction Hamiltonian $H_0$ is translationally invariant,
then $\bar h_{k\ell}=h_{k\ell}-h$, and $h_{k\ell}$ depends only on the
displacement between the spins indexed by $k$ and $\ell$.  The blocks
$\m h_\ell$ then take the form
\begin{align}
  \m h_{\ell} = \m\Sigma_\ell \c \m h_0 \c \m\Sigma_\ell^\dag,
  &&
  \m\Sigma_\ell \equiv \sum_p \op{p+\ell}{p},
  \label{eq:block_shift}
\end{align}
which implies that we need only find the eigenvectors and eigenvalues
of a single block $\m h_0$, after which we can determine the
eigenvectors of any block simply by appropriate applications of the
shift matrix $\m\Sigma_\ell$.  Note that in dimensions $D>1$, the
relation in \eqref{eq:block_shift} holds for vectors $p,\ell$ that
index lattice sites, i.e.~$p=\p{p_1,p_2,\cdots,p_D}$ and likewise with
$\ell$.

If we {\it construct} a perturbation $\V_2\p{\m w_\Delta}$ with
coefficients $\m w_\Delta$ that satisfy \eqref{eq:cond_1} with
eigenvalue $\Delta/2$, then
\begin{align}
  H_0 \V_2\p{\m w_\Delta} \ket\psi
  = \sp{E_0 + \Delta} \V_2\p{\m w_\Delta} \ket\psi.
  \label{eq:H_V1_psi}
\end{align}
As with the case of single-body perturbations, we can decompose any
matrix $\m w$ into its projections $\m w_\Delta$ onto the eigenspace
of matrices satisfying \eqref{eq:cond_2} with eigenvalue $\Delta/2$,
i.e.~$\m w=\sum_\Delta\m w_\Delta$, which allows us to expand
\begin{align}
  \V_2\p{\m w} = \sum_\Delta \V_2\p{\m w_\Delta}.
\end{align}
We can then simplify
\begin{align}
  H_2^{(2)} = - \P_0 \V_2\p{\m w} \E \V_2\p{\m w} \P_0
  = -\sum_{\Delta>0} \f1\Delta \P_0 \V_2\p{\m w_\Delta}^2 \P_0,
\end{align}
where the product $\P_0 \V_2\p{\m w_\Delta}^2 \P_0$ can be written in
terms collective operators using techniques similar to those in
Appendix \ref{sec:sym_prod}.

\appendix

%%%%%%%%%%%%%%%%%%%%%%%%%%%%%%%%%%%%%%%%%%%%%%%%%%%%%%%%%%%%%%%%%%%%%%
\section{Diagnosing perturbed states with the interaction Hamiltonian}

Here we simplify vectors of the form $H_0\V_M\ket\psi$ to arrive at
the expansions in \eqref{eq:H_V1_psi_start} and \eqref{eq:H_V2_psi}.

%%%%%%%%%%%%%%%%%%%%%%%%%%%%%%%%%%%%%%%%%%%%%%%%%%
\subsection{Single-body perturbation}
\label{sec:H_V1_psi}

We wish to simplify
\begin{align}
  H_0 \V_1 \ket\psi
  = \sum_{\substack{k\\p<q}} h_{pq} v_k \Pi_{pq} V_k \ket\psi,
  \label{eq:H_V1_psi_start}
\end{align}
which has terms with $k\in\set{p,q}$, and terms with
$k\notin\set{p,q}$.  In the case of $k\notin\set{p,q}$, the
permutation operator $\Pi_{pq}$ commutes with $V_k$ and annihilates on
$\ket\psi$, leaving us the sum
\begin{align}
  \sum_{\substack{p<q\\k\notin\set{p,q}}} h_{pq}
  = \f12 \sum_{\substack{p,q\\k\notin\set{p,q}}} h_{pq}
  = \f12 \sum_{p,q} h_{pq}
  - \f12 \sum_{\substack{p,q\\k\in\set{p,q}}} h_{pq}
  = E_0 - h_k,
\end{align}
where we have used the facts that $h_{pq}=h_{qp}$ and $h_{pp}=0$.  The
terms with $k\notin\set{p,q}$ in \eqref{eq:H_V1_psi_start} are then
\begin{align}
  \sum_{\substack{p<q\\k\notin\set{p,q}}}
  h_{pq} v_k \Pi_{pq} V_k \ket\psi
  = \sum_k \p{E_0 - h_k} v_k V_k \ket\psi
  = E_0 \V_1 \ket\psi - \v h\oc\v v \c\v V \ket\psi,
\end{align}
where $\v h\oc\v v$ is an element-wise (Hadamard) product of $\v h$
and $\v v$.  In the case of $k\in\set{p,q}$, we have
\begin{align}
  \sum_{\substack{p<q\\k\in\set{p,q}}}
  h_{pq} v_k \Pi_{pq} V_k \ket\psi
  = \sum_{k<q} h_{kq} v_k V_q \ket\psi
  + \sum_{p<k} h_{pk} v_k V_p \ket\psi
  = \sum_{k,q} h_{kq} v_k V_q \ket\psi
  = \v v\c\m h\c\v V \ket\psi,
\end{align}
which implies
\begin{align}
  H_0 \V_1 \ket\psi
  = E_0 \V_1 \ket\psi
  + \p{\v v\c\m h - \v v\oc\v h}\c \v V \ket\psi.
\end{align}

%%%%%%%%%%%%%%%%%%%%%%%%%%%%%%%%%%%%%%%%%%%%%%%%%%
\subsection{Two-body perturbation}
\label{sec:H_V2_psi}

We wish to simplify
\begin{align}
  H_0 \V_2 \ket\psi
  = \f12 \sum_{\substack{k,\ell\\p<q}} w_{k\ell} h_{pq}
  \Pi_{pq} X_k Y_\ell \ket\psi,
  \label{eq:H_V2_psi_start}
\end{align}
which has terms with $p,q\notin\set{k,\ell}$, one term with
$\set{p,q}=\set{k,\ell}$, and terms with $p,q\in\set{k,\ell}$.  The
permutation operator $\Pi_{pq}$ acts trivially on
$X_k Y_\ell \ket\psi$ when $p,q\notin\set{k,\ell}$ or
$\set{p,q}=\set{k,\ell}$, leaving a sum over $p,q$ of the form
\begin{align}
  \sum_{\substack{p<q\\p,q\notin\set{k,\ell}}} h_{pq} + h_{k\ell}
  = \sum_{p<q} h_{pq}
  - \sum_{\substack{p\in\set{k,\ell}\\q\notin\set{k,\ell}}} h_{pq}
  = E_0 + 2 h_{k\ell} - h_k - h_\ell.
\end{align}
The corresponding terms in \eqref{eq:H_V2_psi_start} with
$p,q\notin\set{k,\ell}$ and $\set{p,q}=\set{k,\ell}$ are then
\begin{align}
  E_0 \V_2 \ket\psi + \sum_{k,\ell}
  \sp{h_{k\ell} - \f12\p{h_k+h_\ell}} w_{k\ell} X_k Y_\ell \ket\psi
  = E_0 \V_2 \ket\psi + \v X \c \p{\bar{\m h}\oc\m w} \c \v Y \ket\psi,
\end{align}
where $\bar{\m h}\oc\m w$ is an element-wise (Hadamard) product
between $\bar{\m h}$ and $\m w$,
\begin{align}
  \bar{\m h} \equiv \m h - \f12\p{\v h\otimes\v\1 + \v\1\otimes\v h},
\end{align}
and $\v\1$ is a vector of ones.  The terms in
\eqref{eq:H_V2_psi_start} with only one of $p,q\in\set{k,\ell}$
simplify to
\begin{align}
  \f12 \sum_{\substack{k,\ell\\q\notin\set{k,\ell}}} w_{k\ell}
  \p{h_{kq} X_q Y_\ell + h_{\ell q} X_k Y_q} \ket\psi
  &= \sum_{k,\ell,q} w_{k\ell} h_{kq} X_q Y_\ell \ket\psi
  - \sum_{k,\ell} w_{k\ell} h_{k\ell} Z_k \ket\psi \\
  &= \v X \c\m h \c\m w\c\v Y \ket\psi
  - \v\1 \c \p{\m h\oc\m w}\c \v Z \ket\psi,
\end{align}
where
\begin{align}
  Z_k \equiv \f12 \sp{X_k, Y_k}_+,
\end{align}
and $\v Z$ is a vector of all $Z_p$.  In total,
\begin{align}
  H_0 \V_2 \ket\psi
  = E_0 \V_2 \ket\psi
  + \v X \c \p{\bar{\m h}\oc\m w + \m h\c\m w} \c \v Y \ket\psi
  - \v\1 \c \p{\m h\oc\m w}\c \v Z \ket\psi.
\end{align}

%%%%%%%%%%%%%%%%%%%%%%%%%%%%%%%%%%%%%%%%%%%%%%%%%%%%%%%%%%%%%%%%%%%%%%
\section{Simplifying operator products restricted to the fully
  symmetric manifold}
\label{sec:sym_prod}

%%%%%%%%%%%%%%%%%%%%%%%%%%%%%%%%%%%%%%%%%%%%%%%%%%
\subsection{Single-body product}
\label{sec:PVVP}

Calculating the second-order effective Hamiltonian $H_1^{(2)}$ induced
on the fully symmetric manifold $\M_0$ by the perturbation $\V_1$ in
\eqref{eq:perturbations} requires us to simplify the product
\begin{align}
  \P_0 \V_1\p{\v v}^2 \P_0
  = \sum_{p,q} v_p v_q \P_0 V_p V_q \P_0
  = \sum_p v_p^2 \P_0 V_p^2 \P_0
  + \sum_{p\ne q} v_p v_q \P_0 V_p V_q \P_0,
  \label{eq:PVVP_start}
\end{align}
where the coefficients $\v v$ satisfy \eqref{eq:cond_1} with
eigenvalue $\Delta\ne0$, which implies that $\v v$ is mean-zero.  The
first sum on the right of \eqref{eq:PVVP_start} is then
\begin{align}
  \sum_p v_p^2 \P_0 V_p^2 \P_0
  = \sum_p v_p^2 \P_0 V_0^2 \P_0
  = \var\p{\v v} \P_0 W \P_0,
  \label{eq:PVVP_eq}
\end{align}
where
\begin{align}
  \var\p{\v v} \equiv \EE_p\sp{\p{v_p-\EE_q\sp{v_q}}^2}
  = \EE_p\sp{v_p^2} - \EE_q\sp{v_q}^2,
  &&
  W \equiv \sum_p V_p^2.
\end{align}
The second sum in \eqref{eq:PVVP_start} is
\begin{align}
  \sum_{p\ne q} v_p v_q \P_0 V_p V_q \P_0
  = \sum_{p\ne q} v_p v_q \P_0 V_0 V_1 \P_0,
  \label{eq:PVVP_neq_start}
\end{align}
where the sum over coefficients in \eqref{eq:PVVP_neq_start} can be
simplified to
\begin{align}
  \sum_{p\ne q} v_p v_q
  = \sum_{p,q} v_p v_q - \sum_{p=q} v_p v_q
  = - \sum_p v_p^2 = - N \var\p{\v v}.
\end{align}
In order to simplify the product of operators in
\eqref{eq:PVVP_neq_start}, we expand
\begin{align}
  \P_0 V^2 \P_0
  = \sum_p \P_0 V_p^2 \P_0
  + \sum_{p\ne q} \P_0 V_p V_q \P_0
  = \P_0 W \P_0 + N \p{N-1} \P_0 V_0 V_1 \P_0,
  \label{eq:PVVP_neq_ops}
\end{align}
which implies that
\begin{align}
  \sum_{p\ne q} v_p v_q \P_0 V_p V_q \P_0
  = - \var\p{\v v} N \P_0 V_0 V_1 \P_0
  = -\f{\var\p{\v v}}{N-1} \P_0 \p{V^2 - W} \P_0
  \label{eq:PVVP_neq}
\end{align}
Altogether, we have that
\begin{align}
  \P_0 \V_1\p{\v v}^2 \P_0
  = -\f{\var\p{\v v}}{N-1} \P_0 \p{V^2 - N W} \P_0,
\end{align}

%%%%%%%%%%%%%%%%%%%%%%%%%%%%%%%%%%%%%%%%%%%%%%%%%%
\subsection{Two-body product at first order}
\label{sec:PXYP}

Calculating the first-order effective Hamiltonian $H_2^{(1)}$ induced
on the fully symmetric manifold $\M_0$ by the perturbation $\V_2$ in
\eqref{eq:perturbations} requires us to simplify the product
\begin{align}
  \P_0 \V_2 \P_0
  = \f12 \P_0 \v X\c\m w\c\v Y\P_0
  = \f12 \sum_{p\ne q} w_{pq} \P_0 X_p Y_q \P_0
  = {N\choose 2} w \P_0 X_0 Y_1 \P_0,
  \label{eq:PXYP_start}
\end{align}
where ${N\choose2}=N\p{N-1}/2$ and
\begin{align}
  w \equiv {N\choose 2}^{-1} \sum_{p<q} w_{pq}
\end{align}
is the mean value of the pair-wise coefficients in $\V_2$.  To
simplify the operator content of \eqref{eq:PXYP_start}, we define
\begin{align}
  Z_p \equiv \f12\sp{X_p, Y_p}_+,
  &&
  Z \equiv \sum_p Z_p,
  &&
  \tilde Z \equiv \f12 \sp{X, Y}_+,
\end{align}
and expand
\begin{align}
  \P_0 \tilde Z \P_0 = \P_0 Z \P_0 + N\p{N-1} \P_0 X_0 Y_1 \P_0.
\end{align}
Substituting the product $\P_0 X_0 Y_1 \P_0$ back into
\eqref{eq:PXYP_start}, we thus find that
\begin{align}
  \P_0 \V_2 \P_0
  = \f12 w \P_0 \p{\tilde Z -  Z } \P_0.
\end{align}

\bibliography{multilevel_spin_notes.bib}

\end{document}
