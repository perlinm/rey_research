\documentclass[nofootinbib,notitlepage,11pt]{revtex4-2}

%%% linking references
\usepackage{hyperref}
\hypersetup{
  breaklinks=true,
  colorlinks=true,
  linkcolor=blue,
  filecolor=magenta,
  urlcolor=cyan,
}

%%% header / footer
\usepackage{fancyhdr} % easier header and footer management
\pagestyle{fancy} % page formatting style
\fancyhf{} % clear all header and footer text
\renewcommand{\headrulewidth}{0pt} % remove horizontal line in header
\usepackage{lastpage} % for referencing last page
\cfoot{\thepage~of \pageref{LastPage}} % "x of y" page labeling


%%% symbols, notations, etc.
\usepackage{physics,braket,bm,amssymb} % physics and math
\renewcommand{\t}{\text} % text in math mode
\newcommand{\f}[2]{\dfrac{#1}{#2}} % shorthand for fractions
\newcommand{\p}[1]{\left(#1\right)} % parenthesis
\renewcommand{\sp}[1]{\left[#1\right]} % square parenthesis
\renewcommand{\set}[1]{\left\{#1\right\}} % curly parenthesis
\newcommand{\bk}{\Braket} % shorthand for braket notation

\renewcommand{\c}{\cdot} % inner product
\renewcommand{\oc}{\circ} % element-wise product

\newcommand{\m}{\bm} % bold symbol
\renewcommand{\v}{\vec} % arrow vector

\usepackage{dsfont} % for identity operator
\newcommand{\1}{\mathds{1}}

\newcommand{\up}{\uparrow}
\newcommand{\dn}{\downarrow}

\renewcommand{\d}{\text{d}}
\newcommand{\x}{\text{x}}
\newcommand{\y}{\text{y}}
\newcommand{\z}{\text{z}}

\newcommand{\B}{\mathcal{B}}
\newcommand{\D}{\mathcal{D}}
\newcommand{\E}{\mathcal{E}}
\renewcommand{\H}{\mathcal{H}}
\newcommand{\I}{\mathcal{I}}
\renewcommand{\L}{\mathcal{L}}
\newcommand{\M}{\mathcal{M}}
\newcommand{\N}{\mathcal{N}}
\renewcommand{\O}{\mathcal{O}}
\renewcommand{\P}{\mathcal{P}}
\newcommand{\Q}{\mathcal{Q}}
\newcommand{\R}{\mathcal{R}}
\newcommand{\T}{\mathcal{T}}
\renewcommand{\S}{\mathcal{S}}
\newcommand{\V}{\mathcal{V}}
\newcommand{\X}{\mathcal{X}}
\newcommand{\Y}{\mathcal{Y}}
\newcommand{\Z}{\mathcal{Z}}

\newcommand{\EE}{\mathbb{E}}
\newcommand{\RR}{\mathbb{R}}
\renewcommand{\SS}{\mathbb{S}}
\newcommand{\ZZ}{\mathbb{Z}}

\newcommand{\PP}{\mathbb{Z}}

\newcommand{\FS}{\text{FS}}
\newcommand{\EQFS}{=_{\text{FS}}}
\newcommand{\col}{\underline}

\DeclareMathOperator{\sign}{sign}
\DeclareMathOperator{\cov}{cov}
\let\var\relax
\DeclareMathOperator{\var}{var}
\DeclareMathOperator{\diag}{diag}
\let\flat\relax
\DeclareMathOperator{\flat}{flat}
\DeclareMathOperator{\counts}{counts}

\def\obra#1{\mathinner{({#1}|}}
\def\oket#1{\mathinner{|{#1})}}
\def\obk#1{\mathinner{({#1})}}
\def\oop#1#2{\oket{#1}\!\obra{#2}}

\usepackage[inline]{enumitem} % in-line lists and \setlist{} (below)
\setlist[enumerate,1]{label={(\roman*)}} % default in-line numbering
\setlist{nolistsep} % more compact spacing between environments

%%% text markup
\usepackage{color} % text color
\newcommand{\red}[1]{{\color{red} #1}}

%%%%%%%%%%%%%%%%%%%%%%%%%%%%%%%%%%%%%%%%%%%%%%%%%%%%%%%%%%%%%%%%%%%%%%
\begin{document}

\title{Perturbing SU($n$)-symmetric interactions}%
\author{Michael A. Perlin}%
\date{\today}

\maketitle

\tableofcontents

\section{Introduction}

We consider an array of $N$ multilevel spins with non-uniform
SU($n$)-symmetric interactions that can be written in the form
\begin{align}
  H_0 = \sum_{p<q} h_{pq} \Pi_{pq}
  = \f12 \sum_{p,q} h_{pq} \Pi_{pq},
  &&
  \Pi_{pq} \equiv \sum_{\mu,\nu} S_{\mu\nu}^{(p)} S_{\nu\mu}^{(q)},
  \label{eq:ints}
\end{align}
where the coefficients $h_{pq}$ in the Hamiltonian satisfy
$h_{pq}=h_{qp}$ and $h_{pp}=0$;
$S_{\mu\nu}^{(p)}\equiv\op{\mu}{\nu}_p$ flips the state of spin $p$ to
$\ket\mu$ from $\ket\nu$; and the operator $\Pi_{pq}$ permutes spins
$p$ and $q$.  The permutation operator $\Pi_{pq}$ is a multilevel
generalization of the SU(2)-symmetric spin interaction
$\m S_p\c\m S_q$.

We wish to determine the effective dynamics induced on the
ground-state manifold $\M_0$ of $H_0$ by weak single- and two-body
perturbations of the form
\begin{align}
  \V_1 \equiv \sum_p v_p X_p = \v v\c\v X,
  &&
  \V_2 \equiv \sum_{p<q} w_{pq} \times \f12 \p{X_p Y_q + X_q Y_p}
  = \f12 \v X \c\m w \c\v Y,
  \label{eq:perturbations}
\end{align}
where $X,Y$ are self-adjoint\footnote{In future versions of these
  notes, we should relax the restriction that $X,Y$ are self-adjoint.}
trace-zero operators on an $n$-dimensional Hilbert space; $X_p,Y_p$
denote the action of $X,Y$ on spin $p$; $\v X,$ $\v Y$, $\v v$, and
$\m w$ are respectively vectors and a matrix of all
$X_p,Y_p,v_p,w_{pq}$; we use a dot ($\c$) to denote an inner product
between vectors and matrices as appropriate; and we enforce
$w_{pq}=w_{qp}$ and $w_{pp}=0$.

If the coefficients $h_{pq}$ of the interaction Hamiltonian $H_0$ are
all negative, then the ground-state manifold $\M_0$ of the interaction
Hamiltonian $H_0$ consists of fully symmetric states that are
simultaneous $+1$ eigenstates of all permutation operators $\Pi_{pq}$.
We will adopt the restriction that all $h_{pq}<0$ throughout these
notes, but note that this restriction can be relaxed to the assumption
that the initial state of the spins is fully symmetric (e.g.~a
spin-polarized state).  We will also assume throughout these notes
that the fully symmetric manifold $\M_0$ is gapped away from all other
states by an interaction energy difference that is large compared to
any coupling between $\M_0$ and its orthogonal complement.

The effective Hamiltonian $H_M$ induced on the ground-state manifold
$\M_0$ by an $M$-body perturbation $\V_M$ through second order in
perturbation theory is given by\cite{bravyi2011schrieffer,
  perlin2019effective}
\begin{align}
  H_M = H_M^{(1)} + H_M^{(2)},
  &&
  H_M^{(1)} = \P_0 \V_M \P_0,
  &&
  H_M^{(2)} = - \P_0 \V_M \E \V_M \P_0,
  &&
  \E \equiv \sum_{\Delta>0} \f{\P_\Delta}{\Delta},
\end{align}
where $\P_\Delta$ is a projector onto the eigenspace of the
interaction Hamiltonian $H_0$, with interaction energy $\Delta$ above
that of fully symmetric manifold $\M_0$; that is, the interaction
energy of states in the image of $\P_\Delta$ is $E_0+\Delta$, where
\begin{align}
  E_0 = \sum_{p<q} h_{pq}
\end{align}
is the interaction energy of fully symmetric states $\ket\psi\in\M_0$.

%%%%%%%%%%%%%%%%%%%%%%%%%%%%%%%%%%%%%%%%%%%%%%%%%%%%%%%%%%%%%%%%%%%%%%
\section{Single-body perturbations}

To compute the effective Hamiltonian induced by the single-body
perturbation $\V_1$, we use the permutational symmetry of the
ground-state manifold to simplify
\begin{align}
  H_1^{(1)} = \sum_p v_p \P_0 X_p \P_0
  = \sum_p v_p \P_0 X_0 \P_0 \EQFS \bar v \col{X},
  \label{eq:H_1_1}
\end{align}
where $\EQFS$ denotes equality under a restriction to the fully
symmetric manifold $\M_0$, and
\begin{align}
  \bar v \equiv \f1N \sum_p v_p,
  &&
  \col{X} \equiv \sum_p X_p.
\end{align}
In order to compute the second order effective Hamiltonian
$H_1^{(2)}$, we first choose an arbitrary fully symmetric state
$\ket\psi\in\M_0$ and expand (see Appendix \ref{sec:H_V1_psi}):
\begin{align}
  H_0 \V_1 \ket\psi
  = E_0 \V_1 \ket\psi
  + \p{\sum_{p,q} h_{pq} v_p X_q - \sum_q h_q v_q X_q} \ket\psi,
  \label{eq:H_V1_psi}
\end{align}
where $h_q\equiv\sum_p h_{pq}$ is the net interaction energy
associated with spin $p$.  We thus find that the vector $\V_1\ket\psi$
is an eigenvector of the interaction Hamiltonian $H_0$ if the
coefficients $\v v$ satisfy
\begin{align}
  \p{\m h - \diag\v h}\c\v v = \Delta\p{\v v} \v v,
  \label{eq:cond_1}
\end{align}
where $\v h,\m h$ are respectively a vector and matrix of all
$h_p,h_{pq}$; $\diag\v h$ is a matrix with $\v h$ on the diagonal and
zeros everywhere else; and $\Delta\p{\v v}$ is some constant that may
depend on $\v v$.  If we {\it construct} a perturbation
$\V_1\p{\v v_\Delta}$ with coefficients $\v v_\Delta$ that satisfy
\eqref{eq:cond_1} with eigenvalue $\Delta$, then
\begin{align}
  H_0 \V_1\p{\v v_\Delta} \ket\psi
  = \p{E_0 + \Delta} \V_1\p{\v v_\Delta} \ket\psi.
\end{align}
Interestingly, the energy of the (unnormalized) state
$\V_1\p{\v v_\Delta}\ket\psi$ depends only on the coefficients
$\v v_\Delta$, and is entirely independent of the state $\ket\psi$ or
choice of single-spin trace-zero operator $X$ used to build $\V_1$ in
\eqref{eq:perturbations}.  Finding operators $\V_1\p{\v v_\Delta}$
that generate eigenvectors of the interaction Hamiltonian $H_0$ when
they act on fully symmetric states $\ket\psi$ thus reduces to finding
eigenvectors $\v v_\Delta$ of the interaction matrix $\m h-\diag\v h$.

If the fully symmetric manifold $\M_0$ is gapped away from all other
states, then then a vector $\V_1\p{\v v_\Delta}\ket\psi$ with
$\Delta=0$ must lie within the fully symmetric manifold $\M_0$, which
implies that the operator $\V_1\p{\v v_\Delta}$ preserves the
permutational symmetry of $\M_0$.  Indeed, a constant vector
$\v\1\equiv\p{1,1,1,\cdots}$ of ones satisfies the condition in
\eqref{eq:cond_1} with eigenvalue $0$.  All other vectors
$\v v_\Delta$ satisfying \eqref{eq:cond_1} with $\Delta\ne0$ must be
orthogonal to $\v\1$, and therefore mean-zero.

We now return to the task of computing the second-order effective
Hamiltonian $H_1^{(2)}$.  Any coefficient vector $\v v$ can be
expanded into its projections $\v v_\Delta$ onto the eigenspace of
vectors satisfying \eqref{eq:cond_1} with eigenvalue $\Delta$,
i.e.~$\v v = \sum_\Delta \v v_\Delta$, which also allows us to expand
\begin{align}
  \V_1\p{\v v} = \sum_\Delta \V_1\p{\v v_\Delta},
\end{align}
where each operator $\V_1\p{\v v_\Delta}$ generates a state with
definite interaction energy $\Delta$ above that of the fully symmetric
manifold $\M_0$.  We can therefore simplify
\begin{align}
  H_1^{(2)}
  = - \P_0 \V_1\p{\v v} \E \V_1\p{\v v} \P_0
  = - \sum_{\Delta>0} \f1{\Delta}
  \P_0 \V_1\p{\v v} \P_\Delta \V_1\p{\v v} \P_0
  = - \sum_{\Delta>0} \f1{\Delta} \P_0 \V_1\p{\v v_\Delta}^2 \P_0,
\end{align}
where the product $\P_0 \V_1\p{\v v_\Delta}^2 \P_0$ is worked out in
Appendix \ref{sec:PXXP}.  In total, we find
\begin{align}
  H_1^{(2)}
  \EQFS \f1{N\p{N-1}}
  \sum_{\Delta>0} \f{\v v_\Delta\c\v v_\Delta}{\Delta}
  \sp{\p{\col{X}}^2 - N \p{\col{X^2}}},
  \label{eq:H_1_2}
\end{align}
where $\EQFS$ denotes equality under a restriction to the fully
symmetric manifold $\M_0$, $\v v_\Delta$ is the projection of $\v v$
onto the $\Delta$-eigenspace of the matrix $\m h-\diag\v h$, and
\begin{align}
  \col{X} \equiv \sum_p X_p,
  &&
  \col{X^2} \equiv \sum_p X_p^2.
\end{align}

%%%%%%%%%%%%%%%%%%%%%%%%%%%%%%%%%%%%%%%%%%%%%%%%%%%%%%%%%%%%%%%%%%%%%%
\section{Two-body perturbations}

The first-order effective Hamiltonian $H_2^{(1)}$ induced on the fully
symmetric manifold $\M_0$ by the two-body perturbation $\V_2$ in
\eqref{eq:perturbations} is simply the restriction of $\V_2$ onto
$\M_0$ (see Appendix \ref{sec:PXYP}):
\begin{align}
  H_2^{(1)}
  = \P_0 \V_2 \P_0
  \EQFS \f12 \bar w \p{\col{X}\odot\col{Y} - \col{X\odot Y}},
\end{align}
where $A\odot B\equiv\p{AB+BA}/2$ is a symmetrized product of $A$ and
$B$; and
\begin{align}
  \bar w \equiv {N\choose 2}^{-1} \sum_{p<q} w_{pq},
  &&
  \col{X} \equiv \sum_p X_p,
  &&
  \col{Y} \equiv \sum_p Y_p,
  &&
  \col{X\odot Y} \equiv \sum_p X_p\odot Y_p.
\end{align}
In order to compute the second-order effective Hamiltonian
$H_2^{(2)}$, as before we pick an arbitrary fully symmetric state
$\ket\psi\in\M_0$ and expand (see Appendix \ref{sec:H_V2_psi})
\begin{align}
  H_0 \V_2 \ket\psi
  = E_0 \V_2 \ket\psi
  + \sum_{k,\ell} \bar h_{k\ell} w_{k\ell} X_k Y_\ell \ket\psi
  + \f12 \sum_{k\ne\ell} \sum_q \p{h_{kq} w_{q\ell} + w_{kq} h_{q\ell}}
  X_k Y_\ell \ket\psi,
  \label{eq:H_V2_psi}
\end{align}
where to keep notation compact, we define
\begin{align}
  \bar h_{k\ell} \equiv h_{k\ell} - \f12\p{h_k + h_\ell},
  &&
  h_k \equiv \sum_\ell h_{k\ell}.
\end{align}
We thus find that the vector $\V_2\ket\psi$ is an eigenvector of the
interaction Hamiltonian $H_0$ if the coefficients $\m w$ satisfy
\begin{align}
  2\bar{\m h} \circ \m w + \m h\c\m w + \m w\c\m h
  = \Delta\p{\m w} \m w,
  \label{eq:cond_2}
\end{align}
where $\m w,\m h,\bar{\m h}$ are matrices of all
$w_{k\ell},h_{k\ell},\bar h_{k\ell}$; $\bar{\m h} \circ \m w$ is an
element-wise (Hadamard) product of $\bar{\m h}$ and $\m w$, i.e.~with
matrix elements $\bar h_{k\ell} w_{k\ell}$; and $\Delta\p{\m w}$ is
some constant that may depend on $\m w$.

We can recast the condition in \eqref{eq:cond_2} as an eigenvalue
problem by writing the coefficient matrix $\m w$ as a vector
$\v{\m w}$ in $\RR^N\times\RR^N$,
\begin{align}
  \m w = \sum_{p,q} w_{pq} \op{p}{q}
  \to \v{\m w} \equiv \flat\m w \equiv \sum_{p,q} w_{pq} \ket{pq},
\end{align}
where $\flat\m m$ is a flattening of the matrix $\m m$ into a vector
$\vec{\m m}$, and defining the matrix
\begin{align}
  \check{\m h}
  \equiv 2\diag\p{\flat\bar{\m h}}
  + \m h\otimes\1_N + \1_N\otimes\m h
  = 2 \sum_{p,q} \bar h_{pq} \op{pq}
  + \sum_{p,q,\ell} h_{pq} \p{\op{p\ell}{q\ell} + \op{\ell p}{\ell q}},
  \label{eq:h_super_mat}
\end{align}
where $\1_D$ is a $D\times D$ identity matrix.  The condition in
\eqref{eq:cond_2} is then equivalently the eigenvalue equation
\begin{align}
  \p{\check{\m d} \c \check{\m h} \c \check{\m d}} \c \v{\m w}
  = \Delta\p{\m w} \v{\m w},
  &&
  \check{\m d} \equiv \1_{N^2} - \sum_\ell \op{\ell\ell},
  \label{eq:cond_2_eig}
\end{align}
where $\check{\m d}$ annihilates the diagonal part of a (flattened)
matrix it acts on, while keeping the off-diagonal components the same.
We conjugate $\check{\m h}$ by $\check{\m d}$ to enforce that its
eigenvectors $\v{\m w}$ correspond to matrices $\m w$ with zeros on
the diagonal, i.e.~$w_{pp}=0$.

Both numerical experiments and physical arguments suggest that the
$N^2\times N^2$ matrix $\check{\m d} \c \check{\m h} \c \check{\m d}$
is highly degenerate, which means that we might be able to simplify
\eqref{eq:cond_2_eig} even further.  We defer this simplification to
future work, and for now bear the (large) computational cost of
diagonalizing a $N^2\times N^2$ matrix.

If we {\it construct} a perturbation $\V_2\p{\m w_\Delta}$ with
coefficients $\m w_\Delta$ that satisfy \eqref{eq:cond_2_eig} with
eigenvalue $\Delta$, then
\begin{align}
  H_0 \V_2\p{\m w_\Delta} \ket\psi
  = \p{E_0 + \Delta} \V_2\p{\m w_\Delta} \ket\psi,
\end{align}
which implies that $\V_2\p{\m w_\Delta}\ket\psi$ is a state of
definite energy that depends only on the coefficients $\m w_\Delta$.
As with the case of single-body perturbations, we can decompose any
matrix $\m w$ into its projections $\m w_\Delta$ onto the eigenspace
of matrices satisfying \eqref{eq:cond_2_eig} with eigenvalue $\Delta$,
i.e.~$\m w=\sum_\Delta\m w_\Delta$, which allows us to expand
\begin{align}
  \V_2\p{\m w} = \sum_\Delta \V_2\p{\m w_\Delta},
\end{align}
and in turn write
\begin{align}
  H_2^{(2)} = - \P_0 \V_2\p{\m w} \E \V_2\p{\m w} \P_0
  = -\sum_{\Delta>0} \f1\Delta \P_0 \V_2\p{\m w_\Delta}^2 \P_0,
\end{align}
where the product $\P_0 \V_2\p{\m w_\Delta}^2 \P_0$ is simplified in
Appendices \ref{sec:PXYXYP} and \ref{sec:multi_to_collective}.  If
$X=Y=s_\z$, where $s_\z\equiv\p{\op\up-\op\dn}/2$ is an SU(2) spin
operator, then we can expand
\begin{align}
  H_2^{(2)} \EQFS W_4\p{\m h,\m w} S_\z^4 - W_2\p{\m h,\m w} S_\z^2,
\end{align}
where $S_\z\equiv\sum_p s_\z^{(p)}$ is a collective SU(2) spin
operator, $W_k\p{\m h,\m w}$ are scalar coefficients, and we have
neglected scalar terms in $H_2^{(2)}$ without physical consequence.
In order to expand the coefficients $W_k\p{\m h,\m w}$, for any matrix
$\m m$ we define the vectors
\begin{align}
  \v{\m m} \equiv \sum_{p,q} m_{pq} \ket{pq},
  &&
  \v m \equiv \sum_{p,q} m_{pq} \ket{p},
\end{align}
using which we have
\begin{align}
  W_4\p{\m h,\m w}
  \equiv \sum_\Delta \f1\Delta \f{\v w_\Delta\c\v w_\Delta
    - \v{\m w}_\Delta\c\v{\m w}_\Delta/2}{N!_4},
\end{align}
\begin{align}
  W_2\p{\m h,\m w}
  \equiv \sum_\Delta \f1{\Delta} \sp{\f{\v w_\Delta\c\v w_\Delta
      - \v{\m w}_\Delta\c\v{\m w}_\Delta}{4N!_2}
    + \f{\p{3N-4}\p{\v w_\Delta\c\v w_\Delta
        - \v{\m w}_\Delta\c\v{\m w}_\Delta/2}}{2N!_4}},
\end{align}
where
\begin{align}
  N!_k \equiv \prod_{j=0}^{k-1} \p{N-j} = \f{N!}{\p{N-k}!}.
\end{align}
Note that the dependence of the coefficients of $W_k\p{\m h,\m w}$ on
the interaction matrix $\m h$ is implicit in the decomposition of the
coefficient matrix $\m w$ into components $\m w_\Delta$ as
$\m w=\sum_\Delta\m w_\Delta$, where $\m w_\Delta$ is the projection
of $\m w$ onto the $\Delta$-eigenspace of the operator
$\check{\m d}\c\check{\m h}\c\check{\m d}$ defined by
\eqref{eq:h_super_mat} and \eqref{eq:cond_2_eig}.

\appendix

%%%%%%%%%%%%%%%%%%%%%%%%%%%%%%%%%%%%%%%%%%%%%%%%%%%%%%%%%%%%%%%%%%%%%%
\section{Diagnosing perturbed states with the interaction Hamiltonian}

Here we simplify vectors of the form $H_0\V_M\ket\psi$ to arrive at
the expansions in \eqref{eq:H_V1_psi} and \eqref{eq:H_V2_psi}.

%%%%%%%%%%%%%%%%%%%%%%%%%%%%%%%%%%%%%%%%%%%%%%%%%%
\subsection{Single-body perturbation}
\label{sec:H_V1_psi}

We wish to simplify
\begin{align}
  H_0 \V_1 \ket\psi
  = \sum_{\substack{k\\p<q}} h_{pq} v_k \Pi_{pq} X_k \ket\psi,
  \label{eq:H_V1_psi_start}
\end{align}
which has terms with $k\in\set{p,q}$, and terms with
$k\notin\set{p,q}$.  In the case of $k\notin\set{p,q}$, the
permutation operator $\Pi_{pq}$ commutes with $X_k$ and annihilates on
$\ket\psi$, leaving us the sum
\begin{align}
  \sum_{\substack{p<q\\k\notin\set{p,q}}} h_{pq}
  = \f12 \sum_{\substack{p,q\\k\notin\set{p,q}}} h_{pq}
  = \f12 \sum_{p,q} h_{pq}
  - \f12 \sum_{\substack{p,q\\k\in\set{p,q}}} h_{pq}
  = E_0 - h_k,
\end{align}
where we have used the facts that $h_{pq}=h_{qp}$ and $h_{pp}=0$.  The
terms in \eqref{eq:H_V1_psi_start} with $k\notin\set{p,q}$ are then
\begin{align}
  \sum_{\substack{p<q\\k\notin\set{p,q}}}
  h_{pq} v_k \Pi_{pq} X_k \ket\psi
  = \sum_k \p{E_0 - h_k} v_k X_k \ket\psi
  = E_0 \V_1 \ket\psi - \sum_k h_k v_k X_k \ket\psi
\end{align}
while the terms with $k\in\set{p,q}$ are
\begin{align}
  \sum_{\substack{p<q\\k\in\set{p,q}}}
  h_{pq} v_k \Pi_{pq} X_k \ket\psi
  = \sum_{k<q} h_{kq} v_k X_q \ket\psi
  + \sum_{p<k} h_{pk} v_k X_p \ket\psi
  = \sum_{k,q} h_{kq} v_k X_q \ket\psi
\end{align}
which implies
\begin{align}
  H_0 \V_1 \ket\psi
  = E_0 \V_1 \ket\psi
  + \p{\sum_{p,q} h_{pq} v_p X_q - \sum_q h_q v_q X_q} \ket\psi.
\end{align}

%%%%%%%%%%%%%%%%%%%%%%%%%%%%%%%%%%%%%%%%%%%%%%%%%%
\subsection{Two-body perturbation}
\label{sec:H_V2_psi}

We wish to simplify
\begin{align}
  H_0 \V_2 \ket\psi
  = \f12 \sum_{\substack{k,\ell\\p<q}} w_{k\ell} h_{pq}
  \Pi_{pq} X_k Y_\ell \ket\psi,
  \label{eq:H_V2_psi_start}
\end{align}
which has one term with $\set{p,q}=\set{k,\ell}$, terms with
$p,q\notin\set{k,\ell}$, and terms with only one of
$p,q\in\set{k,\ell}$.  The permutation operator $\Pi_{pq}$ acts
trivially on $X_k Y_\ell \ket\psi$ when $\set{p,q}=\set{k,\ell}$ or
$p,q\notin\set{k,\ell}$, leaving a sum over $p,q$ of the form
\begin{align}
  h_{k\ell} + \sum_{\substack{p<q\\p,q\notin\set{k,\ell}}} h_{pq}
  = \sum_{p<q} h_{pq}
  - \sum_{\substack{p\in\set{k,\ell}\\q\notin\set{k,\ell}}} h_{pq}
  = E_0 + 2 h_{k\ell} - h_k - h_\ell
  = E_0 + 2 \bar h_{k\ell},
\end{align}
where
\begin{align}
  \bar h_{k\ell} \equiv h_{k\ell} - \f12\p{h_k + h_\ell}.
\end{align}
The corresponding terms in \eqref{eq:H_V2_psi_start} with
$p,q\notin\set{k,\ell}$ and $\set{p,q}=\set{k,\ell}$ are then
\begin{align}
  E_0 \V_2 \ket\psi
  + \sum_{k,\ell} \bar h_{k\ell} w_{k\ell} X_k Y_\ell \ket\psi.
\end{align}
Using similar techniques, we can simplify the terms in
\eqref{eq:H_V2_psi_start} with only one of $p,q\in\set{k,\ell}$ to
\begin{align}
  \f12 \sum_{k\ne\ell} \sum_q \p{h_{kq} w_{q\ell} + w_{kq} h_{q\ell}}
  X_k Y_\ell \ket\psi.
\end{align}
In total,
\begin{align}
  H_0 \V_2 \ket\psi
  = E_0 \V_2 \ket\psi
  + \sum_{k,\ell} \bar h_{k\ell} w_{k\ell} X_k Y_\ell \ket\psi
  + \f12 \sum_{k\ne\ell} \sum_q \p{h_{kq} w_{q\ell} + w_{kq} h_{q\ell}}
  X_k Y_\ell \ket\psi.
\end{align}

%%%%%%%%%%%%%%%%%%%%%%%%%%%%%%%%%%%%%%%%%%%%%%%%%%%%%%%%%%%%%%%%%%%%%%
\section{Simplifying operator products restricted to the fully
  symmetric manifold}
\label{sec:sym_prod}

%%%%%%%%%%%%%%%%%%%%%%%%%%%%%%%%%%%%%%%%%%%%%%%%%%
\subsection{Single-body product}
\label{sec:PXXP}

Calculating the second-order effective Hamiltonian $H_1^{(2)}$ induced
on the fully symmetric manifold $\M_0$ by the perturbation $\V_1$ in
\eqref{eq:perturbations} requires us to simplify
\begin{align}
  \P_0 \V_1\p{\v v}^2 \P_0
  \EQFS \sum_{p,q} v_p v_q X_p X_q
  = \sum_p v_p^2 X_p^2
  + \sum_{p\ne q} v_p v_q X_p X_q,
  \label{eq:PXXP_start}
\end{align}
where $\EQFS$ denotes equality under a restriction to the fully
symmetric manifold $\M_0$; and the coefficients $\v v$ satisfy
\eqref{eq:cond_1} with eigenvalue $\Delta\ne0$, which implies that
$\v v$ is mean-zero.  The first sum on the right of
\eqref{eq:PXXP_start} is then
\begin{align}
  \sum_p v_p^2 X_p^2
  \EQFS \sum_p v_p^2 X_0^2
  \EQFS \f1N \v v\c\v v\, \col{X^2},
  &&
  \col{X^2} \equiv \sum_p X_p^2.
  \label{eq:PXXP_eq}
\end{align}
The second sum on the right of \eqref{eq:PXXP_start} is
\begin{align}
  \sum_{p\ne q} v_p v_q X_p X_q
  \EQFS \sum_{p\ne q} v_p v_q X_0 X_1,
  \label{eq:PXXP_neq_start}
\end{align}
where the sum over coefficients in \eqref{eq:PXXP_neq_start} can be
simplified to
\begin{align}
  \sum_{p\ne q} v_p v_q
  = \sum_{p,q} v_p v_q - \sum_{p=q} v_p v_q
  = - \v v\c\v v
\end{align}
In order to simplify the product of operators in
\eqref{eq:PXXP_neq_start}, we expand
\begin{align}
  \p{\col{X}}^2
  = \sum_p X_p^2 + \sum_{p\ne q} X_p X_q
  \EQFS \col{X^2} + N \p{N-1} X_0 X_1,
  \label{eq:PXXP_neq_ops}
\end{align}
which implies that
\begin{align}
  \sum_{p\ne q} v_p v_q X_p X_q
  \EQFS - \v v\c\v v X_0 X_1
  \EQFS -\f{\v v\c\v v}{N\p{N-1}}
  \sp{\p{\col{X}}^2 - N \p{\col{X^2}}}
  \label{eq:PXXP_neq}
\end{align}
Altogether, we have that
\begin{align}
  \P_0 \V_1\p{\v v}^2 \P_0
  \EQFS -\f{\v v\c\v v}{N\p{N-1}}
  \sp{\p{\col{X}}^2 - N \p{\col{X^2}}}.
\end{align}

%%%%%%%%%%%%%%%%%%%%%%%%%%%%%%%%%%%%%%%%%%%%%%%%%%
\subsection{Two-body product, first order}
\label{sec:PXYP}

Calculating the first-order effective Hamiltonian $H_2^{(1)}$ induced
on the fully symmetric manifold $\M_0$ by the perturbation $\V_2$ in
\eqref{eq:perturbations} requires us to simplify
\begin{align}
  \P_0 \V_2 \P_0
  \EQFS \f12 \sum_{p\ne q} w_{pq} X_p Y_q
  = {N\choose 2} \bar w X_0 Y_1,
  \label{eq:PXYP_start}
\end{align}
where $\EQFS$ denotes equality under a restriction to the fully
symmetric manifold $\M_0$, ${N\choose2}=N\p{N-1}/2$ is a binomial
coefficient, and
\begin{align}
  \bar w \equiv {N\choose 2}^{-1} \sum_{p<q} w_{pq}
\end{align}
is the mean value of the pair-wise couplings in $\V_2$.  To simplify
the operator content of \eqref{eq:PXYP_start}, we define
\begin{align}
  \col{X} \equiv \sum_p X_p,
  &&
  \col{Y} \equiv \sum_p Y_p,
  &&
  A \odot B \equiv \f12\p{AB+BA},
  &&
  \col{X\odot Y} \equiv \sum_p X_p \odot Y_p,
\end{align}
and expand
\begin{align}
  \col{X}\odot\col{Y} \EQFS \col{X\odot Y} + N\p{N-1} X_0 Y_1.
\end{align}
Substituting the product $X_0 Y_1$ back into \eqref{eq:PXYP_start}
gives us
\begin{align}
  \P_0 \V_2 \P_0
  \EQFS \f12 \bar w \p{\col{X}\odot\col{Y} - \col{X\odot Y}}.
\end{align}

%%%%%%%%%%%%%%%%%%%%%%%%%%%%%%%%%%%%%%%%%%%%%%%%%%
\subsection{Two-body product, second order}
\label{sec:PXYXYP}

Calculating the second-order effective Hamiltonian $H_2^{(2)}$ induced
on the fully symmetric manifold $\M_0$ by the perturbation $\V_2$ in
\eqref{eq:perturbations} requires us to simplify
\begin{align}
  \P_0 \V_2^2 \P_0
  \EQFS \f14 \sum_{k,\ell,p,q}
  w_{k\ell} w_{pq} X_k Y_\ell X_p Y_q,
  \label{eq:PXYXYP_start}
\end{align}
where $\EQFS$ denotes equality under a restriction to the fully
symmetric manifold $\M_0$, and the coefficients $\m w$ satisfy
\eqref{eq:cond_2} with eigenvalue $\Delta\p{\m w}\ne0$, which implies
that $\m w$ is mean-zero.  To simplify the sum in
\eqref{eq:PXYXYP_start}, we first consider the terms with
$p,q\in\set{k,\ell}$:
\begin{align}
  \sum_{\substack{k,\ell\\p,q\in\set{k,\ell}}}
  w_{k\ell} w_{pq} X_k Y_\ell X_p Y_q
  \EQFS \v{\m w}\c\v{\m w} \p{X_0^2 Y_1^2 + X_0 Y_0 Y_1 X_1},
  &&
  \v{\m w} \equiv \sum_{k,\ell} w_{k\ell} \ket{k\ell}.
\end{align}
The terms of the sum in \eqref{eq:PXYXYP_start} with
$p\in\set{k,\ell}$ and $q\notin\set{k,\ell}$ or vice versa are
\begin{align}
  \sum_{\substack{k,\ell\\p\in\set{k,\ell}\\q\notin\set{k,\ell}}}
  w_{k\ell} w_{pq} X_k Y_\ell X_p Y_q
  &\EQFS \sum_{\substack{k,\ell\\q\notin\set{k,\ell}}}
  w_{k\ell} w_{k q} \p{X_0^2 Y_1 Y_2 + Y_0 X_0 X_1 Y_2}, \\
  \sum_{\substack{k,\ell\\p\notin\set{k,\ell}\\q\in\set{k,\ell}}}
  w_{k\ell} w_{pq} X_k Y_\ell X_p Y_q
  &\EQFS \sum_{\substack{k,\ell\\p\notin\set{k,\ell}}}
  w_{k\ell} w_{pk} \p{X_0 Y_0 X_1 Y_2 + Y_0^2 X_1 X_2},
\end{align}
where
\begin{align}
  \sum_{\substack{k,\ell\\p\notin\set{k,\ell}}} w_{k\ell} w_{pk}
  = \sum_{k,\ell,p} w_{k\ell} w_{kp} - \sum_{k,\ell} w_{k\ell}^2
  = \v w\c\v w - \v{\m w}\c\v{\m w},
  &&
  \v w \equiv \sum_{k,\ell} w_{k\ell} \ket{k}.
\end{align}
Finally, the terms of the sum in \eqref{eq:PXYXYP_start} with
$p,q\notin\set{k,\ell}$ are
\begin{align}
  \sum_{\substack{k,\ell\\p,q\notin\set{k,\ell}}}
  w_{k\ell} w_{pq} X_k Y_\ell X_p Y_q
  \EQFS \sum_{\substack{k,\ell\\p,q\notin\set{k,\ell}}}
  w_{k\ell} w_{pq} X_0 X_1 Y_2 Y_3,
\end{align}
where, using the fact that $\m w$ is mean-zero,
\begin{align}
  \sum_{\substack{k,\ell\\p,q\notin\set{k,\ell}}} w_{k\ell} w_{pq}
  = - \sum_{\substack{k,\ell\\p,q\in\set{k,\ell}}} w_{k\ell} w_{pq}
  - \sum_{\substack{k,\ell\\p\in\set{k,\ell}\\q\notin\set{k,\ell}}}
  w_{k\ell} w_{pq}
  - \sum_{\substack{k,\ell\\p\notin\set{k,\ell}\\q\in\set{k,\ell}}}
  w_{k\ell} w_{pq}
  = 2 \v{\m w}\c\v{\m w} - 4 \v w\c\v w.
\end{align}
Altogether, we thus have that
\begin{multline}
  \P_0 \V_2^2 \P_0
  \EQFS -\p{\v w\c\v w - \f12 \v{\m w}\c\v{\m w}} X_0 X_1 Y_2 Y_3
  + \f14 \v{\m w}\c\v{\m w} \p{X_0^2 Y_1^2 + X_0 Y_0 Y_1 X_1} \\
  + \f14 \p{\v w\c\v w - \v{\m w}\c\v{\m w}}
  \p{X_0^2 Y_1 Y_2 + X_0 Y_0 X_1 Y_2
    + Y_0 X_0 X_1 Y_2 + Y_0^2 X_1 X_2}.
  \label{eq:PXYXYP}
\end{multline}
In principle, our next task is to write this result in terms of
collective operators within the fully symmetric manifold; this task
turns out to be rather non-trivial in general.  We therefore discuss
the decomposition of multi-local operators (e.g.~$X_0Y_0X_1Y_2$, a
3-local operator) into a sum of collective operators within the fully
symmetric manifold in Appendix \ref{sec:multi_to_collective}.  At the
end of Appendix \ref{sec:multi_to_collective}, we provide analytical
expressions for the special cases of
\begin{enumerate*}
\item multi-local operators that are tensor powers of a single
  operator, i.e.~of the form $X^{\otimes k}$; and
\item multi-local operators of the form $X^{\otimes k}$ with $X^2=1$.
\end{enumerate*}
These cases are relevant for e.g.~the SU(2) case with Ising-type
interactions of the form $Z\otimes Z$, with $Z$ a single-spin Pauli
operator.

%%%%%%%%%%%%%%%%%%%%%%%%%%%%%%%%%%%%%%%%%%%%%%%%%%%%%%%%%%%%%%%%%%%%%%
\section{Collective and multi-local operators in the fully symmetric
  manifold}
\label{sec:multi_to_collective}

Here we provide a decomposition multi-local operators into sums of
collective operators (and vice versa) within the fully symmetric
manifold of $N$ spins.  Multi-local operators are operators of the
form
\begin{align}
  O_S \equiv \bigotimes_{O\in S} O,
  \label{eq:multi_op_set}
\end{align}
where $S\equiv\set{O}$ is a set of single-spin operators $O$.  The
order of tensor factors in \eqref{eq:multi_op_set} does not matter
after projecting operators onto the fully symmetric manifold, as we
always will in this section.  While the decomposition of a collective
operator into a sum of multi-local operators is unique, the opposite
is not true: there is generally no unique decomposition of a
multi-local operator into a sum of collective operators.  We can,
however, enforce a canonical decomposition by choosing a set of
collective operators that obey the same symmetries as $O_S$, namely
invariance under arbitrary permutations of the single-spin operators
$O\in S$.  To this end, for any partition\footnote{A partition
  $P\equiv\set{p}$ of $S$ is a set of mutually disjoint, non-empty
  subsets $p\subset S$ with $\bigcup_{p\in P}p=S$.}  $P\equiv\set{p}$
of $S$ we define the $\abs{P}$-local operator
\begin{align}
  O_P \equiv \bigotimes_{p\in P} \bigodot_{O\in p} O,
\end{align}
where $\bigodot_{x\in\X} x$ denotes a symmetrized product of all
elements in $\X$, i.e.
\begin{align}
  \bigodot_{x\in\X} x
  \equiv \f1{\abs{\X}!}
  \sum_{\substack{\t{permutations}\\\sigma~\t{of}~\X}}
  ~ \prod_{k=1}^{\abs{\X}} x_{\sigma_k},
\end{align}
with $\sigma_k$ the $k$-th element of the permutation $\sigma$ when
expressed in one-line notation; for example,
\begin{align}
  \bigodot_{j=1}^2 x_j &= x_1 \odot x_2 = \f12 \p{x_1x_2 + x_2x_1}, \\
  \bigodot_{j=1}^3 x_j
  &= \f1{3!} \p{x_1x_2x_3 + x_1x_3x_2 + x_2x_1x_3
    + x_2x_3x_1 + x_3x_1x_2 + x_3x_2x_1}.
\end{align}
In words, a partition $P$ of $S$ consists of an assignment of
operators $O\in S$ to individual parts $p\subset S$; each part $p$ is
associated with a tensor factor of $O_P$, and the operators $O\in p$
act on that tensor factor.  The set of multi-local operators
$\set{O_P:\t{partitions}~P~\t{of}~S}$ consists of all operators that
\begin{enumerate*}
\item can be built from ordinary and tensor products of all
  single-spin operators $O\in S$, and
\item are invariant under arbitrary permutations of $S$ (after
  projection onto the fully symmetric manifold).
\end{enumerate*}
For any set $S$ of single-spin operators and any partition $P$ of $S$,
we can similarly define the collective operators
\begin{align}
  \O_S \equiv \bigodot_{O\in S} \col{O},
  &&
  \O_P \equiv \bigodot_{p\in P} \col{\textstyle\bigodot_{O\in p} O},
\end{align}
where we underline a single-spin operator to denote its collective
version, i.e.~$\col{Q}\equiv\sum_j Q^{(j)}$, where $Q^{(j)}$ is an
operator that acts with the single-spin operator $Q$ on spin $j$ and
trivially on all other spins.  By construction, the set of collective
operators $\set{\O_P:\t{partitions}~P~\t{of}~S}$ is one-to-one to the
set of multi-local operators $\set{O_P:\t{partitions}~P~\t{of}~S}$,
and it is straightforward to work out that for any set $S$ of
single-spin operators,
\begin{align}
  \O_S
  \EQFS \sum_{\substack{\t{partitions}\\P~\t{of}~S}}
  N!_{\abs{P}}\, O_P,
  &&
  N!_k \equiv \prod_{j=0}^{k-1}\p{N-j}! = \f{N!}{\p{N-k}!},
  \label{eq:collective_to_multi_full}
\end{align}
where $\EQFS$ denotes equality under a restriction to the fully
symmetric manifold.  For any multi-local operator $O_S$ indexed by a
set $S$ of single-spin operators, we can collect expansions of the
form \eqref{eq:collective_to_multi_full} for all collective operators
built from non-empty subsets of $S$ into a system of linear equations
$\L$, and in turn solve $\L$ for $O_S$.  When
$\abs{S}\in\set{1,2,3,4}$ (and, by conjecture, for all $S$), by use of
a computer algebra system (namely, Mathematica) we find that
\begin{align}
  N!_{\abs{S}}\, O_S
  &\EQFS \sum_{\substack{\t{partitions}\\P~\t{of}~S}}
  \p{-1}^{\abs{S}-\abs{P}} \O_P \times \prod_{p\in P}\p{\abs{p}-1}!
  \label{eq:multi_to_collective_full} \\
  &\EQFS \f1{\abs{S}!} \sum_{\substack{\t{permutations}\\\Pi~\t{of}~S}}
  \sum_{\substack{\t{ordered}\\\t{partitions}\\\tilde P~\t{of}~\Pi}}
  \p{-1}^{\abs{S}-\abs{\tilde P}} \O_{\tilde P}
  \times \prod_{\tilde p\in\tilde P} \p{\abs{\tilde p}-1}!
  \label{eq:multi_to_collective_full_simp}
\end{align}
where we identify permutations $\Pi$ of $S$ with an ordered list
(tuple) $\p{O_1,O_2,\cdots,O_{\abs{S}}}$ of all single-spin operators
in $S$; and we enforce that partitions $\tilde P$ of an ordered list
$\Pi$, as well as parts $\tilde p\in\tilde P$, are lexicographically
ordered in accordance with the ordering of operators in $\Pi$.  If
$S=\set{X,Y,Z}$, for example, then the ordered partitions of the
trivial permutation $\Pi=\p{X,Y,Z}$ are
\begin{align}
  \p{\p{X,Y,Z}}, &&
  \p{\p{X,Y},\p{Z}}, &&
  \p{\p{X,Z},\p{Y}}, &&
  \p{\p{X},\p{Y,Z}}, &&
  \p{\p{X},\p{Y},\p{Z}}.
\end{align}
For ordered lists $\tilde S$ and ordered partitions $\tilde P$, we
define
\begin{align}
  O_{\tilde S} \equiv \bigotimes_{O\in\tilde S} O,
  &&
  O_{\tilde P}
  \equiv \bigotimes_{\tilde p\in\tilde P} \prod_{O\in\tilde p} O,
  &&
  \O_{\tilde S} \equiv \prod_{O\in\tilde S} \col{O},
  &&
  \O_{\tilde P}
  \equiv \prod_{\tilde p\in\tilde P}
  \col{\textstyle\prod_{O\in\tilde p} O},
\end{align}
where products over elements of $\tilde S,\tilde P,\tilde p$ are taken
in order according to the elements of these ordered lists, e.g.~if
$\tilde S=\p{O_1,O_2,\cdots,O_{\abs{S}}}$, then
$\prod_{O\in\tilde S}\equiv O_1O_2\cdots O_{\abs{S}}$.  These
definitions allow us to expand
\begin{align}
  \O_{\tilde S} \EQFS
  \sum_{\substack{\t{ordered}\\\t{partitions}\\\tilde P~\t{of}~\tilde S}}
  N!_{\abs{\tilde P}} O_{\tilde P},
\end{align}
in analogy to \eqref{eq:collective_to_multi_full}.

While the expansions in \eqref{eq:multi_to_collective_full} and
\eqref{eq:multi_to_collective_full_simp} closely resemble each other,
in practice the latter involves less computational overhead due to the
lack of symmetrization in the construction of $\O_{\tilde P}$.  There
are, in principle, even more economical decompositions of a
multi-local operators into collective opreators, for example
\begin{align}
  N!_2\, X\otimes Y
  &\EQFS \col{X}\,\col{Y} - \col{XY}, \\
  N!_3\, X\otimes Y \otimes Z
  &\EQFS \col{X}\,\col{Y}\,\col{Z}
  - \col{X}\,\col{YZ} - \col{Y}\,\col{XZ} - \col{Z}\,\col{XY}
  + \col{ZXY} + \col{YXZ},
\end{align}
but these decompositions generally do not preserve the manifest
symmetry of a multi-local operator $O_S$ under arbitrary permutations
of $S$, and we do not know of any general prescription to arrive at
such decompositions aside from solving systems of equations relating
collective operators to multi-local operators.

On a final note, we consider the special case that all operators
$O\in S$ are identical, for which we provide the expansions
\begin{align}
  N!_2\, X\otimes X \EQFS \p{\col{X}}^2 - \p{\col{X^2}},
  &&
  N!_3\, X^{\otimes 3} \EQFS \p{\col{X}}^3
  - 3 \col{X} \odot \col{X^2} + 2\p{\col{X^3}},
\end{align}
\begin{align}
  N!_4\, X^{\otimes 4}
  \EQFS \p{\col{X}}^4 - 4 \p{\col{X}}^2\odot\p{\col{X^2}}
  - 2 \col{X} \p{\col{X}}^2 \col{X}
  + 3 \p{\col{X^2}}^2 + 8 \col{X} \odot \p{\col{X^3}}
  - 6 \p{\col{X^4}}.
\end{align}
If furthermore $X^2=1$, as with e.g.~the Pauli operators of SU(2),
then
\begin{align}
  N!_2\, X\otimes X \EQFS \p{\col{X}}^2 - N,
  &&
  N!_3\, X^{\otimes 3} \EQFS \p{\col{X}}^3 - \p{3N-2} \col{X},
\end{align}
\begin{align}
  N!_4\, X^{\otimes 4} \EQFS \p{\col{X}}^4
  - 2\p{3N-4} \p{\col{X}}^2 + 3 N \p{N-2}.
\end{align}

\bibliography{multilevel_spin_notes.bib}

\end{document}