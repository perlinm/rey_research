\documentclass[nofootinbib,notitlepage,11pt]{revtex4-2}

%%% linking references
\usepackage{hyperref}
\hypersetup{
  breaklinks=true,
  colorlinks=true,
  linkcolor=blue,
  filecolor=magenta,
  urlcolor=cyan,
}

%%% header / footer
\usepackage{fancyhdr} % easier header and footer management
\pagestyle{fancy} % page formatting style
\fancyhf{} % clear all header and footer text
\renewcommand{\headrulewidth}{0pt} % remove horizontal line in header
\usepackage{lastpage} % for referencing last page
\cfoot{\thepage~of \pageref{LastPage}} % "x of y" page labeling


%%% symbols, notations, etc.
\usepackage{physics,braket,bm,amssymb} % physics and math
\renewcommand{\t}{\text} % text in math mode
\newcommand{\f}[2]{\dfrac{#1}{#2}} % shorthand for fractions
\newcommand{\p}[1]{\left(#1\right)} % parenthesis
\renewcommand{\sp}[1]{\left[#1\right]} % square parenthesis
\renewcommand{\set}[1]{\left\{#1\right\}} % curly parenthesis
\newcommand{\bk}{\Braket} % shorthand for braket notation
\renewcommand{\v}{\bm} % bold vectors
\newcommand{\uv}[1]{\bm{\hat{#1}}} % unit vectors
\newcommand{\av}{\vec} % arrow vectors
\renewcommand{\d}{\text{d}} % for infinitesimals
\renewcommand{\c}{\cdot} % inner product

\usepackage{dsfont} % for identity operator
\newcommand{\1}{\mathds{1}}

\newcommand{\up}{\uparrow}
\newcommand{\dn}{\downarrow}

\newcommand{\x}{\text{x}}
\newcommand{\y}{\text{y}}
\newcommand{\z}{\text{z}}

\newcommand{\B}{\mathcal{B}}
\newcommand{\D}{\mathcal{D}}
\newcommand{\E}{\mathcal{E}}
\renewcommand{\H}{\mathcal{H}}
\newcommand{\I}{\mathcal{I}}
\newcommand{\J}{\mathcal{J}}
\newcommand{\M}{\mathcal{M}}
\newcommand{\N}{\mathcal{N}}
\renewcommand{\O}{\mathcal{O}}
\renewcommand{\P}{\mathcal{P}}
\newcommand{\Q}{\mathcal{Q}}
\newcommand{\R}{\mathcal{R}}
\newcommand{\T}{\mathcal{T}}
\renewcommand{\S}{\mathcal{S}}
\newcommand{\V}{\mathcal{V}}
\newcommand{\X}{\mathcal{X}}
\newcommand{\Z}{\mathcal{Z}}

\newcommand{\EE}{\mathbb{E}}
\renewcommand{\SS}{\mathbb{S}}
\newcommand{\ZZ}{\mathbb{Z}}

\newcommand{\FS}{\text{FS}}

\DeclareMathOperator{\sign}{sign}
\DeclareMathOperator{\cov}{cov}
\let\var\relax
\DeclareMathOperator{\var}{var}

\def\obra#1{\mathinner{({#1}|}}
\def\oket#1{\mathinner{|{#1})}}
\def\obk#1{\mathinner{({#1})}}
\def\oop#1#2{\oket{#1}\!\obra{#2}}

\usepackage[inline]{enumitem} % in-line lists and \setlist{} (below)
\setlist[enumerate,1]{label={(\roman*)}} % default in-line numbering
\setlist{nolistsep} % more compact spacing between environments

%%% text markup
\usepackage{color} % text color
\newcommand{\red}[1]{{\color{red} #1}}

%%%%%%%%%%%%%%%%%%%%%%%%%%%%%%%%%%%%%%%%%%%%%%%%%%%%%%%%%%%%%%%%%%%%%%
\begin{document}
\thispagestyle{fancy}

\title{Multilevel generalizations of collective spin dynamics and
  squeezing}%
\author{Michael A. Perlin}%
\date{\today}

\maketitle

\tableofcontents

%%%%%%%%%%%%%%%%%%%%%%%%%%%%%%%%%%%%%%%%%%%%%%%%%%%%%%%%%%%%%%%%%%%%%%
\section{Collective SU($n$)-symmetric interactions}
\label{sec:int}

We consider a collection of $n$-level fermions on a 1-D lattice, with
a single-particle Hamiltonian
\begin{align}
  H_{\t{lat}} = -t \sum_{q,\mu} \cos\p{qa} c_{q\mu}^\dag c_{q\mu},
  \label{eq:H_lat}
\end{align}
where $t$ is the nearest-neighbor tunneling rate, $a$ is the spacing
between neighboring lattice sites, and $c_{q\mu}$ is the fermionic
annihilation operator for a particle in quasi-momentum mode $q$ with
internal state (e.g.~nuclear spin) $\mu$.  Under the frozen-mode
approximation, collective two-body SU($n$)-symmetric interactions
between these fermions take form
\begin{align}
  H_{\t{int}} = \f{u}{2} \sum_{p,q,\mu,\nu}
  \p{c_{p\mu}^\dag c_{q\nu}^\dag c_{q\nu} c_{p\mu}
    + c_{q\mu}^\dag c_{p\nu}^\dag c_{q\nu} c_{p\mu}},
\end{align}
where $u\equiv fU/N$ with $U$ the two-body on-site interaction energy,
$f$ the filling fraction of lattice sites, and $N$ the total number of
fermions.  Defining the spin operators
$S_{\mu\nu}^{(p)}\equiv c_{p\mu}^\dag c_{p\nu}$ for each
quasi-momentum $p$, we can alternately write
\begin{align}
  H_{\t{int}} = \f{u}{2} \sum_{p,q,\mu,\nu}
  \p{S_{\mu\mu}^{(p)} S_{\nu\nu}^{(q)} - S_{\mu\nu}^{(p)} S_{\nu\mu}^{(q)}}
  = \f{u}{2}\p{N^2 - \sum_{p,q,\mu,\nu}
    S_{\mu\nu}^{(p)} S_{\nu\mu}^{(q)}},
\end{align}
where $N$ is the total number of particles.  In the absence of
coherence between sectors of different particle number, we can neglect
the $\sim N^2$ term and simply write
\begin{align}
  H_{\t{int}}
  = - \f{u}{2} \sum_{p,q,\mu,\nu} S_{\mu\nu}^{(p)} S_{\nu\mu}^{(q)}.
\end{align}
In the case of SU(2), at this point we would expand the spin operators
$S_{\mu\nu}^{(p)}$ in terms of Pauli operators in order to convert
$H_{\t{int}}$ into an SU(2) spin Hamiltonian\cite{he2019engineering}.
To generalize this procedure to the case of SU($n$), we first define a
vector $\v S^{(p)}$ of all spin operators $S_{\mu\nu}^{(p)}$, and
write
\begin{align}
  H_{\t{int}}
  = - \f{u}{2} \sum_{p,q} {\v S^{(p)}}^\dag \c \v S^{(q)}
  = - \f{u}{2} \v\S^\dag \c \v\S,
  &&
  \v\S\equiv \sum_p \v S^{(p)}.
  \label{eq:H_int_spin}
\end{align}
In the bulk of this section, we identify various bases for the space
of linear operators on the Hilbert space $\H_n$ of an $n$-level
system.  For brevity, we denote this space of operators by
$\B\p{\H_n}$, and note that $\B\p{\H_n}$ is itself an
$n^2$-dimensional Hilbert space equipped with a trace
(Hilbert-Schmidt) inner product
\begin{align}
  \obk{\O|\Q} \equiv \tr\p{\O^\dag Q},
\end{align}
where $\oket{\Q}$ and $\obra{\O}$ respectively denote vectors in
$\B\p{\H_n}$ and its dual space $\B\p{\H_n}^*$.  Similarly to the set
of all spin operators $S_{\mu\nu}\equiv\op{\mu}{\nu}$ on the
single-particle Hilbert space at each quasi-momentum, all of the bases
we consider will be orthonormal with respect to the trace inner
product.  As we show in Appendix \ref{sec:changing_bases}, the
collective SU($n$)-symmetric interaction Hamiltonian in
\eqref{eq:H_int_spin} takes an identical form in any orthonormal basis
for $\B\p{\H_n}$.

%%%%%%%%%%%%%%%%%%%%%%%%%%%%%%%%%%%%%%%%%%%%%%%%%%
\subsection{Generalized Gell-Mann (GGM) operators}
\label{sec:ggm_ops}

The most obvious generalization of Pauli operators to the case of
SU($n$) are the {\it generalized Gell-Mann (GGM)
  operators}\cite{hioe1981level, bertlmann2008bloch}, defined for
$j,k,\ell\in\set{0,1,\cdots,n-1}$ with $j<k$ and $\ell>0$ by
\begin{align}
  \lambda_0 \equiv \sqrt{\f{2}{n}}~ \1,
  &&
  \lambda_\ell \equiv \sqrt{\f{2}{\ell\p{\ell+1}}}
  \p{\sum_{n=0}^{\ell-1}\op{n} - \ell \op{\ell}},
  \label{eq:GGM_diag}
\end{align}
\begin{align}
  \lambda_{jk,\x} \equiv \op{j}{k} + \op{k}{j},
  &&
  \lambda_{jk,\y} \equiv i\p{\op{j}{k} - \op{k}{j}},
  \label{eq:GGM_off_diag}
\end{align}
where $\1$ is the identity operator.  The self-adjoint GGM operators
in \eqref{eq:GGM_diag} and \eqref{eq:GGM_off_diag} provide a complete
basis for the space of operators on the Hilbert space $\H_n$ of an
$n$-level system, coincide exactly with the Pauli operators in the
case of $n=2$, and satisfy the orthonormality condition
\begin{align}
  \obk{\lambda_a|\lambda_b} = 2\delta_{ab},
  \label{eq:GGM_inner}
\end{align}
where $a,b$ index any GGM operator in \eqref{eq:GGM_diag} or
\eqref{eq:GGM_off_diag}.  Denoting a vector of all GGM operators by
$\v\lambda$, we can therefore write (see Appendix
\ref{sec:changing_bases})
\begin{align}
  H_{\t{int}}
  = -\f{u}{4} \sum_{p,q} \v\lambda^{(p)} \c \v\lambda^{(q)}
  = -\f{u}{4} \v\Lambda \c \v\Lambda,
  &&
  \v\Lambda \equiv \sum_p \v\lambda^{(p)}.
  \label{eq:H_int_GGM}
\end{align}

%%%%%%%%%%%%%%%%%%%%%%%%%%%%%%%%%%%%%%%%%%%%%%%%%%
\subsection{Transition operators}
\label{sec:trans_ops}

The GGM operators in \eqref{eq:GGM_diag} and \eqref{eq:GGM_off_diag}
provide a convenient operator basis to describe dynamics obeying
SU($n$) symmetry.  An external driving field addressing $n$-level
nuclear spins, however, will generally violate SU($n$) symmetry, and
instead obeys the symmetries of the {\it polarization} or {\it
  transition operators} defined by\cite{kryszewski2006alternative,
  bertlmann2008bloch}
\begin{align}
  T_{LM}
  \equiv \sqrt{\f{2L+1}{2I+1}} \sum_{\mu,\nu}
  \bk{I\mu;LM|I\nu} \op{\nu}{\mu},
  \label{eq:trans_ops}
\end{align}
where $L\in\set{0,1,\cdots,n-1}$ and $M\in\set{-L,-L+1,\cdots,L}$
index a total spin and its projection onto a quantization axis;
$I\equiv\p{n-1}/2$ is the maximal angular momentum of an $n$-level
spin; $\mu,\nu\in\set{-I,-I+1,\cdots,I}$ index projections of an
$n$-level nuclear spin onto a quantization axis; and
$\bk{I\mu;LM|I\nu}$ is a Clebsch-Gordan coefficient.  The transition
operator $T_{LM}$ is closely related to the associated Legendre
polynomial $P_{LM}\p{x}$ for $x\in\sp{-1,1}$.  Operationally, $T_{LM}$
is proportional to the transition induced on a nuclear spin by the
absorption of a spin-$L$ boson with spin projection $M$ onto a
quantization axis.

Similarly to the GGM operators in \eqref{eq:GGM_diag} and
\eqref{eq:GGM_off_diag}, the transition operators in
\eqref{eq:trans_ops} provide a complete basis for the space of
operators on the Hilbert space $\H_n$ of an $n$-level system, and
satisfy the orthonormality condition
\begin{align}
  \obk{T_{LM}|T_{L'M'}} = \delta_{LL'} \delta_{MM'},
\end{align}
which implies
\begin{align}
  H_{\t{int}} = -\f{u}{2} \sum_{p,q} {\v T^{(p)}}^\dag \c \v T^{(q)}
  = -\f{u}{2} \v\T^\dag \c \v\T,
  &&
  \v\T \equiv \sum_p\v T^{(p)},
  \label{eq:H_int_trans}
\end{align}
where $\v T$ is a vector of all transition operators in
\eqref{eq:trans_ops}.

%%%%%%%%%%%%%%%%%%%%%%%%%%%%%%%%%%%%%%%%%%%%%%%%%%
\subsection{Drive operators}
\label{sec:drive_ops}

Unlike the GGM operators in \eqref{eq:GGM_diag} and
\eqref{eq:GGM_off_diag}, the transition operators in
\eqref{eq:trans_ops} are not self-adjoint.  In order to express
Hamiltonians in terms of self-adjoint operators, we define the {\it
  drive operators}
\begin{align}
  D_{LM} \equiv \f{\eta_M}{\sqrt{2}}
  \sp{T_{L\abs{M}} + \sign\p{M} T_{L\abs{M}}^\dag},
  &&
  \eta_M \equiv
  \begin{cases}
    \sqrt{2} & M = 0 \\
    \p{-1}^M \sign\p{M}^{3/2} & M \ne 0
  \end{cases},
  \label{eq:drive_ops}
\end{align}
for integer $M$ with $\abs{M}\le L$.  The drive operator $D_{LM}$ is
proportional to the Hamiltonian induced on a nuclear spin by an
external classical field of spin-$L$ particles with spin projection
$M$ onto a quantization axis.  Similarly to the GGM operators in
\eqref{eq:GGM_diag} and \eqref{eq:GGM_off_diag}, the drive operators
are proportional to the Pauli operators in the case of $n=2$, and
satisfy the orthonormality condition
\begin{align}
  \obk{D_{LM}|D_{L'M'}} = \delta_{LL'}\delta_{MM'},
\end{align}
which implies
\begin{align}
  H_{\t{int}} = -\f{u}{2} \sum_{p,q} \v D^{(p)}\c\v D^{(q)}
  = -\f{u}{2} \v\D \c \v\D,
  &&
  \v\D \equiv \sum_p \v D^{(p)},
  \label{eq:H_int_drive}
\end{align}
where $\v D$ is a vector of all drive operators in
\eqref{eq:drive_ops}.

The drive operators $D_{1,0}$ and $D_{1,\pm1}$ are proportional to the
Hamiltonians induced on an $n$-level spin by an external magnetic
field and a classical spin-polarized driving laser; loosely speaking,
one can associate
$\p{D_{1,0},D_{1,1},D_{1,-1}}\sim\p{S_\z,S_\x,S_\y}$, where $S_\alpha$
is a collective SU(2) spin-$\alpha$ operator for $\p{n-1}$ 2-level
spins\cite{perlin2020shorttime}.  More concretely, we can embed the
Hilbert space of a single $n$-level spin into the
permutationally-symmetric (Dicke) manifold of states for $\p{n-1}$
2-level spins via
\begin{align}
  \ket{\mu}_{\t{single}} \to \ket{I+\mu}_{\t{collective}}
  \propto S_+^{I+\mu} \ket\dn^{\otimes\p{n-1}},
  \label{eq:collective_embedding}
\end{align}
where $I\equiv\p{n-1}/2$ is a total spin,
$\mu\in\set{-I,-I+1,\cdots,I}$ indexes a state of the single $n$-level
spin, and $S_+$ is a collective spin-raising operator for $\p{n-1}$
2-level spins.  This embedding takes
\begin{align}
  \xi_1 \times \p{D_{1,0}, D_{1,1}, D_{1,-1}}
  \to \p{S_\z, S_\x, S_\y},
  \label{eq:spin_ops}
\end{align}
where
\begin{align}
  \xi_L
  \equiv \sqrt{2L+1}\, \f{L!}{\p{2L+1}!}
  \sp{\prod_{\ell=-L}^L\p{n+\ell}}^{1/2},
  \label{eq:scale_fac}
\end{align}
is a scale factor that will appear frequently in the theory of drive
operators.

%%%%%%%%%%%%%%%%%%%%%%%%%%%%%%%%%%%%%%%%%%%%%%%%%%
\subsection{Weyl operators}

Weyl operators\cite{bertlmann2008bloch} and self-adjoint linear
combinations thereof\cite{asadian2016heisenbergweyl} also provide an
interesting basis of operators for finite-dimensional Hilbert spaces.
These operators are related to the notions of position, momentum, and
phase-space displacement in a discrete space.  As our present work is
motivated by multilevel systems that are realized with nuclear spin
(angular momentum) degrees of freedom, however, the transition and
drive operators are more natural to use than the Weyl operators, so we
will not discuss Weyl operators any further.

%%%%%%%%%%%%%%%%%%%%%%%%%%%%%%%%%%%%%%%%%%%%%%%%%%
\subsection{Subsystem permutations and the fully symmetric manifold}
\label{sec:perm_ops}

For our final re-expression of collective SU($n$)-symmetric
interactions, we abandon the use of single-spin operators entirely to
express the interaction Hamiltonian $H_{\t{int}}$ in terms of
operators that permute tensor factors (subsystems) of a Hilbert space
with an $N$-fold tensor product structure (e.g.~an array of nuclear
spins).  Specifically, we denote the Hilbert space of an $n$-level
system by $\H_n$, denote its $N$-fold tensor product by
$\H_n^{\otimes N}$, and identify the set $\Pi\equiv\set{\Pi_\sigma}$
of operators that permute tensor factors of $\H_n^{\otimes N}$
according to elements $\sigma\in\SS_N$ of the symmetric group of order
$N$.  We then expand
\begin{align}
  H_{\t{int}}
  = -\f{u}{2} \sum_{p,q,\mu,\nu} S_{\mu\nu}^{(p)} S_{\nu\mu}^{(q)}
  = -u\sum_{\substack{p<q\\\mu,\nu}} S_{\mu\nu}^{(p)} S_{\nu\mu}^{(q)}
  - \f{u}{2} \sum_{p,\mu,\nu} S_{\mu\nu}^{(p)} S_{\nu\mu}^{(p)},
  \label{eq:H_int_perm_start}
\end{align}
where the first sum,
\begin{align}
  \sum_{\substack{p<q\\\mu,\nu}} S_{\mu\nu}^{(p)} S_{\nu\mu}^{(q)}
  = \sum_{\substack{p<q\\\mu,\nu}} \op{\mu}{\nu}^{(p)} \op{\nu}{\mu}^{(q)}
  = \sum_{\substack{p<q\\\mu,\nu}} \op{\mu\nu}{\nu\mu}^{(p,q)}
  = \sum_{p<q} \Pi_{pq}
  \label{eq:spin_perm}
\end{align}
is simply a sum over all permutations $\Pi_{pq}$ of subsystems $p$ and
$q$; and the second sum,
\begin{align}
  \sum_{p,\mu,\nu} S_{\mu\nu}^{(p)} S_{\nu\mu}^{(p)}
  = \sum_{p,\mu,\nu} \op{\mu}{\nu}^{(p)} \op{\nu}{\mu}^{(p)}
  = n \sum_{p,\mu} \op{\mu}{\mu}^{(p)}
  = n \sum_p \1^{(p)}
  = n N,
  \label{eq:spin_const}
\end{align}
is merely a constant.  Substituting \eqref{eq:spin_perm} and
\eqref{eq:spin_const} into \eqref{eq:H_int_perm_start} yields
\begin{align}
  H_{\t{int}} = -u \sum_{p<q} \Pi_{pq} - \f12 N n u
  \simeq -u\sum_{p<q} \Pi_{pq},
  \label{eq:H_int_perm}
\end{align}
where $\simeq$ denotes equality up to an overall constant with no
physical consequence.  The form of the interaction Hamiltonian
$H_{\t{int}}$ makes it evident that its ground-state manifold
$\M_{\FS}$ is the set of {\it fully symmetric} (FS) states that have
eigenvalue 1 with respect to all pairwise permutation operators
$\Pi_{pq}$.  The fully symmetric manifold $\M_\FS$ is spanned by a
basis $\set{\ket{\v m}}$ of fully symmetric states that can be
uniquely identified by a list of integers
$\v m\equiv\p{m_0,m_1,\cdots,m_{n-1}}\in\ZZ_{N+1}^n$, where $m_s$
indicates the total occupation number of subsystem state $s$, and
$\sum_sm_s=N$.  In the case of $n=2$, the fully symmetric manifold
$\M_\FS$ is precisely the Dicke manifold\cite{dicke1954coherence} of
$N$ qubits, spanned by Dicke states $\set{\ket{m_0,m_1}}$ for which
$m_0+m_1=N$ and the integer $m_0$ ($m_1$) indicates the total
occupation number of qubit state $\ket{0}$ ($\ket{1}$).  The dimension
of fully symmetric manifold $\M_\FS$ is determined by the number of
ways to assign $N$ subsystems to $n$ states, which is
${N+n-1 \choose n-1}$.  The energy of any fully symmetric state
$\ket{\psi_\FS}\in\M_\FS$ with respect to the interaction Hamiltonian
$H_{\t{int}}$ in \eqref{eq:H_int_perm} is
$-u{N \choose 2}=-\p{1/2}N\p{N-1}u$.

In addition to the fact that the fully symmetric manifold $\M_\FS$ is
ground-state manifold of the interaction Hamltonian $H_{\t{int}}$, an
important feature of $\M_\FS$ is its closure under the action of
collective operators of the form $\Q=\sum_pQ^{(p)}$, as well as linear
combinations and products thereof.  Closure of $\M_\FS$ under
collective the action of collective operators, and in particular under
collective dynamics, is a straightforward consequence of the
permutational symmetry obeyed by these operators.

%%%%%%%%%%%%%%%%%%%%%%%%%%%%%%%%%%%%%%%%%%%%%%%%%%%%%%%%%%%%%%%%%%%%%%
\section{External drive and spin-orbit coupling}
\label{sec:drive_SOC}

A driving laser addressing spin-1/2 fermions on a 1-D lattice will
induce the Hamiltonian
\begin{align}
  \left. H_{\t{drive}}^{(\phi)} \right|_{n=2}
  = \f12 \Omega \sum_j
  \p{e^{i\phi j} c_{j,\up}^\dag c_{j,\dn}
    + e^{-i\phi j} c_{j,\dn}^\dag c_{j,\up}},
  \label{eq:H_drive_2}
\end{align}
where $\Omega$ is a (real) driving amplitude, $j\in\mathbb{Z}$ indexes
a lattice site, and $\phi$ is the relative phase of the drive between
adjacent lattice sites.  The multilevel analogue of
\eqref{eq:H_drive_2} is
\begin{align}
  H_{\t{drive}}^{(\phi)}
  = -\f12 \tilde\Omega \sum_j \p{e^{i\phi j} T_{1,1}^{(j)}
    + e^{-i\phi j} T_{1,1}^{(j)\dag}}
  = \Omega \D_{1,1}^{(\phi)},
  &&
  \Omega \equiv \tilde \Omega \xi_1,
  \label{eq:H_drive}
\end{align}
where we define the re-scaled driving amplitude $\tilde\Omega$ for
convenience, and
\begin{align}
  \D_{LM}^{(\phi)}
  \equiv \sum_j \sp{\cos\p{M\phi j} D_{LM}^{(j)}
    - \sin\p{M\phi j} D_{L,-M}^{(j)}},
  \label{eq:drive_rot}
\end{align}
is an inhomogeneously rotated collective drive operator.  We can make
this drive spatially homogenous via the gauge transformation
$c_{j\mu}^\dag \to e^{-i\mu\phi j} c_{j\mu}^\dag$, which takes
 \begin{align}
   H_{\t{drive}}^{(\phi)} \to H_{\t{drive}} = \Omega \D_{1,1},
   &&
   \D_{LM} \equiv \D_{LM}^{(0)} = \sum_j D_{LM}^{(j)}.
\end{align}
The SU($n$) symmetry of collective interactions considered in this
work implies that the interaction Hamiltonian $H_{\t{int}}$ in
\eqref{eq:H_int_perm} is unaffected by this gauge transformation.  The
single-particle Hamiltonian $H_{\t{lat}}$ in \eqref{eq:H_lat},
however, transforms as (see Appendix \ref{sec:lat_drive})
\begin{align}
  H_{\t{lat}}
  \to H_{\t{lat}}^{(\phi)}
  = -t \sum_{q,\mu} \cos\p{qa+\mu\phi} c_{q\mu}^\dag c_{q\mu}
  = \sum_{q,L} B_L^{(\phi,q)} D_{L,0}^{(q)},
  \label{eq:H_lat_SOC_B}
\end{align}
with effective inhomogeneous driving field amplitudes
\begin{align}
  B_L^{(\phi,q)} = -t w_L\p{qa} A_L^{(\phi)},
  \label{eq:B_L_phi}
\end{align}
where
\begin{align}
  w_L\p{\theta} \equiv
  \begin{cases}
    \cos\theta & L~\t{even} \\
    \sin\theta & L~\t{odd}
  \end{cases},
  &&
  A_L^{(\phi)} \equiv \p{-1}^L \sqrt{\f{2L+1}{2I+1}}
  \sum_\mu \bk{I\mu;L,0|I\mu} w_L\p{\mu\phi}.
  \label{eq:A_L_phi}
\end{align}
The multilevel spin-orbit coupling (SOC) exhibited by the
singe-particle Hamiltonian $H_{\t{lat}}^{(\phi)}$ thus allows for the
simulation of synthetic gauge fields with large spin ($L>1$).  As we
will see in Section \ref{sec:drive_raman}, multi-laser drives can also
simulate synthetic large-spin fields with non-zero spin projection
($M\ne0$) onto a quantization axis.

%%%%%%%%%%%%%%%%%%%%%%%%%%%%%%%%%%%%%%%%%%%%%%%%%%%%%%%%%%%%%%%%%%%%%%
\section{Perturbative treatment of spin-orbit coupling}
\label{sec:pert_SOC}

The single-particle Hamiltonian $H_{\t{lat}}^{(\phi)}$ breaks
permutational symmetry, thereby coupling states within the fully
symmetric manifold $\M_\FS$ to states outsides it.  If SOC is
sufficiently weak, however, transitions outside the fully symmetric
manifold $\M_\FS$ are energetically suppressed by interactions, and we
can account for the effect of SOC perturbatively.  In the regime of
weak SOC with $\abs{\phi}\ll1$, by considering
\begin{enumerate*}
\item all $L\le12$ for arbitrary $n$, and
\item all $L<n$ for all $n\le25$
\end{enumerate*}
(and conjecturing the same functional form for all $L$ and $n$), we
find that
% expand Clebsch-Gordan coefficients in a power series according to
% \url{functions.wolfram.com/07.38.06.0008.01}
\begin{align}
  A_L^{(\phi)}
  = \p{-1}^{L+\lfloor L/2\rfloor} \xi_L \phi^L + O\p{\phi^{L+2}},
  \label{eq:A_L_phi_small}
\end{align}
where $\xi_L$ is a scale factor defined in \eqref{eq:scale_fac}.  A
perturbative treatment of SOC through second order in the SOC angle
$\phi$ yields the following effective Hamiltonian in the fully
symmetric manifold $\M_\FS$ (see Appendix \ref{sec:SOC_pert}):
\begin{align}
  H_{\t{eff}} = H_{\t{eff}}^{(1)} + H_{\t{eff}}^{(2)},
\end{align}
\begin{align}
  H_{\t{eff}}^{(1)}
  &= \phi t \, \EE_p\sp{\sin\p{pa}} \xi_1 \D_{1,0}
  + \phi^2 t \, \EE_q\sp{\cos\p{qa}} \xi_2 \D_{2,0}
  + O\p{\phi^3},
  \label{eq:H_eff_1_phi} \\
  H_{\t{eff}}^{(2)}
  &= \f{\phi^2 t^2}{\p{N-1}fU} \var_q\sp{\sin\p{qa}}
  \p{\xi_1^2 \D_{1,0}^2 - 2 N \xi_2 \D_{2,0}}
  + O\p{\phi^3},
  \label{eq:H_eff_2_phi}
\end{align}
where
\begin{align}
  \EE_q\sp{X_q} \equiv \f1N \sum_q X_q,
  &&
  \var_q\sp{X_q} \equiv \EE_q\sp{\p{X_q-\EE_k\sp{X_k}}}
  = \EE_q\sp{X_q^2}-\EE_k\sp{X_k}^2,
\end{align}
are the mean and variance of $X_q$ over all occupied quasi-momenta
$q$.  In the case of SU(2), the effective Hamiltonians in
\eqref{eq:H_eff_1_phi} and \eqref{eq:H_eff_2_phi} reduce to the
spin-squeezing Hamiltonians derived in Ref.~\cite{he2019engineering}.
To facilitate comparison, we note that in the case of SU(2) the
re-scaled drive operator $\xi_1\D_{1,0}$ is precisely the collective
spin-$z$ operator $S_\z$ that appears in
Ref.~\cite{he2019engineering}.  Furthermore, drive operators $D_{LM}$
vanish when $L\ge n$, so in the case of SU(2) the operator $\D_{2,0}$
has no contribution to the effective Hamiltonians in
\eqref{eq:H_eff_1_phi} and \eqref{eq:H_eff_2_phi}.


%%%%%%%%%%%%%%%%%%%%%%%%%%%%%%%%%%%%%%%%%%%%%%%%%%%%%%%%%%%%%%%%%%%%%%
\section{Raman drive}
\label{sec:drive_raman}

Rather than coupling the states of a multilevel system directly with a
single driving laser, as in Section \ref{sec:drive_SOC}, we now
consider coupling through an off-resonant Raman transition to a highly
excited state.  Specifically, we consider a three-laser drive of the
form
\begin{align}
  H_{\t{Raman,bare}}^{(\phi)}
  = \sum_j \sum_{m\in\set{1,0,-1}} \Omega_m \p{e^{i\phi_m j}
    T_{1,m}^{(j)} \otimes \op{\up}{\dn} + \t{h.c.}}
  + \f{\Delta}{2} \1 \otimes \p{\op{\up} - \op{\dn}},
  \label{eq:raman_bare}
\end{align}
where $j$ indexes a lattice site, $\Omega_m$ is a (real) driving
amplitude; $\phi_m$ is the relative phase of drive $m$ between
adjacent lattice sites; $\op{r}{s}$ for $r,s\in\set{\up,\dn}$ is a
pseudo-spin operator for the auxiliary degree of freedom that is
off-resonantly addressed by the lasers; $\Delta$ is a detuning of the
lasers from the energy of an excited state; and $\1$ is the identity
operator.  If the detuning is large, $\abs{\Delta}\gg\abs{\Omega_m}$,
then at second order in perturbation theory the effective action of
$H_{\t{Raman,bare}}^{(\phi)}$ within the $\ket{\dn}$-state manifold
takes the form
\begin{align}
  H_{\t{Raman}}^{(\phi)}
  = \sum_{\tau\in\set{0,1,2}} H_{\t{Raman},\tau}^{(\phi)}
\end{align}
where $H_{\t{Raman},\tau}^{(\phi)}$ generates nuclear spin transitions
$\mu\to\mu\pm\tau$:
\begin{align}
  H_{\t{Raman},0}^{(\phi)}
  &= -\sum_j\sum_m \f{\Omega_m^2}{\Delta}
  T_{1,m}^{(j)\dag} T_{1,m}^{(j)},
  \label{eq:raman_0} \\
  H_{\t{Raman},1}^{(\phi)}
  &= -\sum_j\sum_{s\in\set{\pm1}}
  \f{\Omega_0\Omega_s}{\Delta} e^{i\p{\phi_s-\phi_0} j}
  T_{1,0}^{(j)\dag} T_{1,s}^{(j)} + \t{h.c.},
  \\
  H_{\t{Raman},2}^{(\phi)}
  &= -\sum_j\sum_{s\in\set{\pm1}}
  \f{\Omega_+\Omega_-}{\Delta} e^{i\p{\phi_s-\phi_{-s}}j}
  T_{1,-s}^{(j)\dag} T_{1,s}^{(j)}.
  \label{eq:raman_2}
\end{align}
In terms of the drive operators, we can expand
\begin{align}
  H_{\t{Raman}}^{(\phi)}
  = \sum_j \sum_{\substack{L\le2\\\abs{M}\le L}}
  \Omega_{LM}^{(\phi,j)} D_{LM}^{(j)},
  \label{eq:raman_drive}
\end{align}
where the coefficients $\Omega_{LM}^{(\phi,j)}$ are provided in
Appendix \ref{sec:drive_raman_coeff}.  If the driving lasers are
phase-matched with $\phi_m=m\phi$, such that a nuclear spin transition
$\mu\to\mu+m$ on site $j$ imprints the phase $m\phi j$, then the drive
$H_{\t{Raman}}^{(\phi)}$ takes the form
\begin{align}
  \left. H_{\t{Raman}}^{(\phi)} \right|_{\phi_m=m\phi}
  = \sum_{\substack{L\le2\\0\le M\le L}}
  \Omega_{LM} \D_{LM}^{(\phi)},
  \label{eq:drive_raman_matched}
\end{align}
with rotated collective drive operators $\D_{LM}^{(\phi)}$ defined in
\eqref{eq:drive_rot} and uniform driving amplitudes $\Omega_{LM}$
provided in Appendix \ref{sec:drive_raman_coeff}.  Similarly to the
case of a direct drive considered in Section \ref{sec:drive_SOC}, the
Raman drive $H_{\t{Raman}}^{(\phi)}$ in \eqref{eq:drive_raman_matched}
can be made homogenous through the gauge transformation
$c_{j\mu}^\dag \to e^{i\mu\phi j} c_{j\mu}^\dag$, which takes
\begin{align}
  \left. H_{\t{Raman}}^{(\phi)} \right|_{\phi_m=m\phi}
  \to H_{\t{Raman}}
  = \sum_{\substack{L\le2\\0\le M\le L}} \Omega_{LM} \D_{LM}.
\end{align}
The effective driving amplitudes $\Omega_{LM}$ can be tuned through
the bare driving amplitudes $\Omega_m$, and satisfy
\begin{align}
  \Omega_{0,0} &\propto \Omega_+^2 + \Omega_-^2 + \Omega_0^2,
  &
  \Omega_{1,0} &\propto \Omega_+^2 - \Omega_-^2,
  &
  \Omega_{2,0} &\propto \Omega_+^2 + \Omega_-^2 - 2\Omega_0^2,
  \\
  \Omega_{1,1} &\propto \Omega_0 \p{\Omega_+ + \Omega_-},
  &
  \Omega_{2,1} &\propto \Omega_0 \p{\Omega_+ - \Omega_-},
  &
  \Omega_{2,2} &\propto \Omega_+ \Omega_-.
\end{align}
In particular, the amplitudes $\Omega_{1,0}$, $\Omega_{2,0}$, and
$\Omega_{2,1}$ can be simultaneously tuned to zero by setting
$\Omega_0=\Omega_+=\Omega_-$, at which point the only remaining
effective drives in $H_{\t{Raman}}$ are $\sim\D_{1,1}$ and
$\sim\D_{2,2}$ (as $\D_{0,0}\propto\1$ is just constant shift in
energy).  If instead $\Omega_0=\pm\Omega_\pm$, then the amplitudes
$\Omega_{1,0}$, $\Omega_{2,0}$ and $\Omega_{1,1}$ vanish, leaving only
the drives $\D_{2,1}$ and $\D_{2,2}$.

%%%%%%%%%%%%%%%%%%%%%%%%%%%%%%%%%%%%%%%%%%%%%%%%%%%%%%%%%%%%%%%%%%%%%%
\section{Periodic drive}

We now consider the effect of applying periodic drive of the form
\begin{align}
  H_{\t{pd}}\p{t} = \sum_p \chi_p \omega \cos\p{\omega t} D_\z^{(p)},
\end{align}
where $\omega$ is a driving frequency, $\chi_p$ is a dimensionless
driving amplitude for spin $p$, and
$D_\z^{(p)} \equiv \xi_1 D_{1,0}^{(p)}$ with $\xi_1$ defined in
\eqref{eq:scale_fac}.  The periodic drive $H_{\t{pd}}\p{t}$ can be
implemented directly using external magnetic fields or driving lasers.
Transition operators $T_{LM}$ transform under rotations generated by
the drive operator $D_\z$ as
\begin{align}
  e^{i\phi D_\z} T_{LM} e^{-i\phi D_\z} = e^{iM\phi} T_{LM}.
\end{align}
If we move into the rotating frame of the drive, then the interaction
Hamiltonian transforms as
\begin{align}
  H_{\t{int}} \to \tilde H_{\t{int}}
  \equiv U_{\t{pd}}\p{t}^\dag H_{\t{int}} U_{\t{pd}}\p{t}
\end{align}
where
\begin{align}
  U_{\t{pd}}\p{t} \equiv \exp\sp{-i\int_0^td\tau\,H_{\t{pd}}\p{\tau}}
  = \prod_p \exp\sp{-i \chi_p \sin\p{\omega t} D_\z^{(p)}}.
\end{align}
Expanding the interaction Hamiltonian $H_{\t{int}}$ in terms of
transition operators $T_{LM}$, we find that
\begin{align}
  \tilde H_{\t{int}}
  = -\f{u}{2} \sum_{p,q,L,M} e^{iM\p{\chi_q-\chi_p}\sin\p{\omega t}}
  T_{LM}^{(p)\dag} T_{LM}^{(q)}
  = -\f{u}{2} \sum_{p,q,L,M,n}
  \J_n\p{M\sp{\chi_q-\chi_p}} e^{in\omega t}
  T_{LM}^{(p)\dag} T_{LM}^{(q)},
\end{align}
where $\J_n$ is the $n$-th order Bessel function of the first kind.
If the drive frequency $\omega$ is much larger than the interaction
strength $u$, i.e.~$\omega\gg u$, then we can neglect oscillating
terms with $n\ne0$, arriving at
\begin{align}
  \tilde H_{\t{int}} \approx -\f{u}{2} \sum_{p,q,L,M}
  \J_0\p{M\sp{\chi_q-\chi_p}}
  T_{LM}^{(p)\dag} T_{LM}^{(q)}.
\end{align}

%%%%%%%%%%%%%%%%%%%%%%%%%%%%%%%%%%%%%%%%%%%%%%%%%%%%%%%%%%%%%%%%%%%%%%
\section{Possible future directions}

\begin{itemize}
\item Visualization of multilevel states
  \begin{itemize}
  \item Single-particle eigenstates of drive operators
  \item Single-particle dynamics induced by drive operators
  \item Collective states and dynamics
  \end{itemize}
\item Implications of SOC
  \begin{itemize}
  \item Special angles $\phi$?
  \end{itemize}
\item Different (more realistic / general) drives
\item Generalizations of squeezing beyond SU(2)?
\item Consider different interactions
  \begin{itemize}
  \item Include electronic states
  \item Super-exchange regime
  \item Non-uniform (local / disordered) interactions
  \end{itemize}
\end{itemize}

\appendix

%%%%%%%%%%%%%%%%%%%%%%%%%%%%%%%%%%%%%%%%%%%%%%%%%%%%%%%%%%%%%%%%%%%%%%
\section{Changing operator bases for SU($n$)-symmetric interactions}
\label{sec:changing_bases}

Here we show that the collective interaction Hamiltonian in
\eqref{eq:H_int_spin} takes an identical form in any orthonormal basis
for $\B\p{\H_n}$, i.e.~the space of linear operators on the Hilbert
space $\H_n$ of an $n$-level system, equipped with the trace
(Hilbert-Schmidt) inner product
\begin{align}
  \obk{\O|\Q} \equiv \tr\p{\O^\dag \Q}.
\end{align}
To maintain generality, we choose an arbitrary basis
$X\equiv\set{X_j}$ for $\B\p{\H_n}$ that satisfies the orthonormality
condition
\begin{align}
  \obk{X_j|X_k} = \N_X \delta_{jk},
\end{align}
for $j\in\set{0,1,\cdots,n^2-1}$.  This orthonormality condition
implies that we can resolve the identity (super-)operator $\I$ with
$\I\oket{\O}=\oket{\O}$ for any $\oket\O\in\B\p{\H_n}$ as
\begin{align}
  \I = \f1{\N_X} \sum_j \oop{X_j}{X_j}
  = \f1{\N_X} \sum_j \oop{X_j^\dag}{X_j^\dag},
\end{align}
where we used the fact that $X^\dag\equiv\set{X_j^\dag}$ is also an
orthonormal basis for $\B\p{\H_n}$, satisfying the same orthonormality
condition as $X$; these two bases are transformed into each other by
unitary $\sum_j\oop{X_j^\dag}{X_j}/\N_X$.

The collective interaction Hamiltonian in \eqref{eq:H_int_spin} takes
the form
\begin{align}
  H = \sum_{p,q,j} X_j^{(p)} X_j^{(q)\dag},
\end{align}
which essentially consists of single-body terms ($p=q$) and two-body
interactions ($p\ne q$) of the form
\begin{align}
  \sum_j X_j X_j^\dag,
  &&
  \sum_j X_j \otimes X_j^\dag.
\end{align}
Choosing an arbitrary orthonormal basis $Y\equiv\set{Y_j}$ for
$\B\p{\H_n}$ with corresponding norm $\N_Y$, we can expand
\begin{align}
  X_j = \f1{\N_Y} \sum_k Y_k \obk{Y_k|X_j},
  &&
  X_j^\dag = \f1{\N_Y} \sum_k Y_k^\dag \obk{Y_k^\dag|X_j^\dag}
  = \f1{\N_Y} \sum_k Y_k^\dag \obk{X_j|Y_k},
\end{align}
where we used the fact that
\begin{align}
  \obk{Y_k^\dag|X_j^\dag}
  = \tr\p{Y_k X_j^\dag}
  = \tr\p{X_j^\dag Y_k}
  = \obk{X_j|Y_k}.
\end{align}
Denoting an arbitrary bilinear operation by $\odot$, by resolution of
the identity we therefore find
\begin{align}
  \sum_j X_j \odot X_j^\dag
  = \f1{\N_Y^2} \sum_{j,k,\ell} Y_k \obk{Y_k|X_j}
  \odot Y_\ell^\dag \obk{X_j|Y_\ell}
  = \f{\N_X}{\N_Y^2} \sum_{k,\ell} Y_k \odot Y_\ell^\dag
  \obk{Y_k|\I|Y_\ell}
  = \f{\N_X}{\N_Y} \sum_k Y_k \odot Y_k^\dag,
\end{align}
which in particular implies that
\begin{align}
  \sum_j X_j X_j^\dag
  = \f{\N_X}{\N_Y} \sum_k Y_k Y_k^\dag,
  &&
  \sum_j X_j \otimes X_j^\dag
  = \f{\N_X}{\N_Y} \sum_k Y_k \otimes Y_k^\dag,
\end{align}
and in turn
\begin{align}
  H = \f{\N_X}{\N_Y} \sum_{p,q,j} Y_j^{(p)} {Y_j^{(q)}}^\dag.
\end{align}
The collective interaction Hamiltonian in \eqref{eq:H_int_spin} thus
takes an identical form in any orthonormal basis for the space
$\B\p{\H_n}$ of linear operators on the Hilbert space $\H_n$ of an
$n$-level system, up to appropriate rescaling factors $\N_X/\N_Y$.

%%%%%%%%%%%%%%%%%%%%%%%%%%%%%%%%%%%%%%%%%%%%%%%%%%%%%%%%%%%%%%%%%%%%%%
\section{Spin-orbit coupling with drive operators}
\label{sec:lat_drive}

Here we derive the drive-operator ($D_{LM}$) expansion of the
spin-orbit coupled Hamiltonian $H_{\t{lat}}^{(\phi)}$ in
\eqref{eq:H_lat_SOC_B}.  In the basis of spin operators
$S_{\mu\mu}^{(q)} \equiv c_{q\mu}^\dag c_{q\mu}$, this Hamiltonian
takes the form
\begin{align}
  H_{\t{lat}}^{(\phi)}
  = -t \sum_{q,\mu} \cos\p{qa+\mu\phi} S_{\mu\mu}^{(q)}
  \label{eq:H_lat_spin}
\end{align}
Defining the diagonal operators
\begin{align}
  \tilde d_{\phi+} \equiv \sum_\mu \cos\p{\mu\phi} S_{\mu\mu},
  &&
  \tilde d_{\phi-} \equiv \sum_\mu \sin\p{\mu\phi} S_{\mu\mu},
\end{align}
which are respectively even and odd under spin inversion
($\mu\to-\mu$), we can expand
\begin{align}
  \sum_\mu \cos\p{qa+\mu\phi} S_{\mu\mu}
  = \cos\p{qa} \tilde d_{\phi+} - \sin\p{qa} \tilde d_{\phi-},
\end{align}
and resolve the identity (super-)operator $\I$ with
$\I \tilde d_{\phi\pm} = \tilde d_{\phi\pm}$ in the basis of the drive operators
$D_{LM}$ (see Appendix \ref{sec:changing_bases}), finding
\begin{align}
  \sum_\mu \cos\p{qa+\mu\phi} S_{\mu\mu}
  = \cos\p{qa} \f12 \sum_{L,M} \obk{\tilde d_{\phi+}|D_{LM}} D_{LM}
  - \sin\p{qa} \f12 \sum_{L,M} \obk{\tilde d_{\phi-}|D_{LM}} D_{LM}.
\end{align}
We now note that the drive operators $D_{LM}$ are strictly
off-diagonal for $M\ne0$, and furthermore that $D_{L,0}$ with even
(odd) $L$ is even (odd) under spin inversion, which implies
\begin{align}
  \sum_{L,M} \obk{\tilde d_{\phi+}|D_{LM}} D_{LM}
  = \sum_{L\,\t{even}} \obk{\tilde d_{\phi+}|D_{L,0}} D_{L,0},
\end{align}
and likewise with $\tilde d_{\phi-}$ and odd $L$.  In total, we can
write
\begin{align}
  \sum_\mu \cos\p{qa+\mu\phi} S_{\mu\mu}
  = \cos\p{qa} \f12 \sum_{L\,\t{even}}
  \obk{\tilde d_{\phi+}|D_{L,0}} D_{L,0}
  - \sin\p{qa} \f12 \sum_{L\,\t{odd}}
  \obk{\tilde d_{\phi-}|D_{L,0}} D_{L,0},
\end{align}
or, more compactly,
\begin{align}
  \sum_\mu \cos\p{qa+\mu\phi} S_{\mu\mu}
  = \sum_L w_L\p{qa} A_L^{(\phi)} D_{L,0},
  &&
  w_L\p{\theta} \equiv
  \begin{cases}
    \cos\theta & L~\t{even} \\
    \sin\theta & L~\t{odd}
  \end{cases},
  \label{eq:compact_SOC_drive}
\end{align}
with effective driving field amplitudes
\begin{align}
  A_L^{(\phi)}
  \equiv \p{-1}^L \f12 \obk{\tilde d_{\phi,\p{-1}^L}|D_{L,0}}
  = \p{-1}^L \sqrt{\f{2L+1}{2I+1}}
  \sum_\mu \bk{I\mu;L,0|I\mu} w_L\p{\mu\phi}.
\end{align}
Substituting the result in \eqref{eq:compact_SOC_drive} into the
single-particle Hamiltonian in \eqref{eq:H_lat_spin} yields
\begin{align}
  H_{\t{lat}}^{(\phi)} = \sum_{q,L} B_L^{(\phi,q)} D_{L,0}^{(q)},
  &&
  B_L^{(\phi,q)} \equiv -J
  w_L\p{qa} A_L^{(\phi)}.
\end{align}

%%%%%%%%%%%%%%%%%%%%%%%%%%%%%%%%%%%%%%%%%%%%%%%%%%%%%%%%%%%%%%%%%%%%%%
\section{Transition operator product expansion}
\label{sec:trans_prod}

Here we derive multiplication rules for transition operators on an $n$-dimensional Hilbert space, which allows us to expand
\begin{align}
  T_{\ell_1 m_1} T_{\ell_2 m_2}
  = \sum_{L,M} g_{\ell_1 m_1;\ell_2 m_2}^{LM} T_{LM},
\end{align}
with structure constants
\begin{align}
  g_{\ell_1 m_1;\ell_2 m_2}^{LM}
  \equiv \obk{T_{LM} | T_{\ell_1 m_1} T_{\ell_2 m_2}}
  = \tr\p{T_{LM}^\dag T_{\ell_1 m_1} T_{\ell_2 m_2}}.
  \label{eq:trans_struct}
\end{align}
For reference, the transition operators $T_{LM}$ are defined for
integers $L,M$ with $0\le L<n$ and $\abs{M}\le L$ in terms of
Clebsch-Gordan coefficients $\bk{\ell_1 m_1;\ell_2 m_2|\ell_3 m_3}$ by
\begin{align}
  T_{LM} \equiv \sqrt{\f{2L+1}{2I+1}}
  \sum_{\mu,\nu} \bk{I\mu;LM|I\nu} \op{\nu}{\mu},
  &&
  I \equiv \f{n-1}{2}.
\end{align}
Using the symmetry properties of Clebsch-Gordan coefficients, namely
\begin{align}
  \bk{\ell_1 m_1; \ell_2 m_2| \ell_3 m_3}
  &= \p{-1}^{\ell_2+m_2} \sqrt{\f{2\ell_3+1}{2\ell_1+1}}
  \bk{\ell_3,-m_3; \ell_2 m_2| \ell_1,-m_1} \\
  \bk{\ell_1 m_1; \ell_2 m_2| \ell_3 m_3}
  &= \p{-1}^{\ell_1+\ell_2-\ell_3}
  \bk{\ell_1,-m_1; \ell_2,-m_2| \ell_3,-m_3},
\end{align}
we can find that
\begin{align}
  T_{LM}^\dag
  = \sqrt{\f{2L+1}{2I+1}}
  \sum_{\mu,\nu} \p{-1}^M \bk{I\nu;L,-M|I\mu} \op{\mu}{\nu}
  = \p{-1}^M T_{L,-M}.
\end{align}
Substituting this result into \eqref{eq:trans_struct} and expanding
Clebsch-Gordan coefficients in terms of Wigner 3-$j$ symbols as
\begin{align}
  \bk{\ell_1 m_1; \ell_2 m_2| \ell_3 m_3}
  = \p{-1}^{-\ell_1+\ell_2-m_3} \sqrt{\p{2\ell_3+1}}
  \begin{pmatrix}
    \ell_1 & \ell_2 & \ell_3 \\
    m_1 & m_2 & -m_3
  \end{pmatrix},
\end{align}
we find that
\begin{multline}
  g_{\ell_1 m_1;\ell_2 m_2}^{LM}
  = \delta_{M,m_1+m_2}
  \p{-1}^M \sqrt{\p{2L+1}\p{2\ell_1+1}\p{2\ell_2+1}} \\
  \times \sum_{\mu,\nu,\rho} \p{-1}^{-3I+L+\ell_1+\ell_2-\mu-\nu-\rho}
  \begin{pmatrix}
    I & L & I \\
    \mu & -M & -\nu
  \end{pmatrix}
  \begin{pmatrix}
    I & \ell_1 & I \\
    \nu & m_1 & -\rho
  \end{pmatrix}
  \begin{pmatrix}
    I & \ell_2 & I \\
    \rho & m_2 & -\mu
  \end{pmatrix},
\end{multline}
where we can evaluate the sum and introduce Wigner 6-$j$ symbols to
get
\begin{align}
  g_{\ell_1 m_1;\ell_2 m_2}^{LM}
  &= \delta_{M,m_1+m_2} \p{-1}^{2I+M}
  \sqrt{\p{2L+1}\p{2\ell_1+1}\p{2\ell_2+1}}
  \begin{pmatrix}
    L & \ell_1 & \ell_2 \\
    M & -m_1 & -m_2
  \end{pmatrix}
  \begin{Bmatrix}
    L & \ell_1 & \ell_2 \\
    I & I & I
  \end{Bmatrix} \\
  &= \delta_{M,m_1+m_2} \p{-1}^{2I+L}
  \sqrt{\p{2\ell_1+1}\p{2\ell_2+1}}
  \bk{\ell_1 m_1; \ell_2 m_2| LM}
  \begin{Bmatrix}
    \ell_1 & \ell_2 & L \\
    I & I & I
  \end{Bmatrix}.
  \label{eq:trans_prod}
\end{align}
The Wigner 6-$j$ symbol is symmetric under any permutation of its
columns, while Clebsch-Gordan coefficients satisfy
\begin{align}
  \bk{\ell_1 m_1; \ell_2 m_2| \ell_3 m_3}
  = \p{-1}^{\ell_1+\ell_2-\ell_3}
  \bk{\ell_2 m_2; \ell_1 m_1| \ell_3 m_3},
\end{align}
which implies that
\begin{align}
  g_{\ell_1 m_1;\ell_2 m_2}^{LM}
  = \p{-1}^{\ell_1+\ell_2-L} g_{\ell_2 m_2;\ell_1 m_1}^{LM}.
\end{align}

%%%%%%%%%%%%%%%%%%%%%%%%%%%%%%%%%%%%%%%%%%%%%%%%%%%%%%%%%%%%%%%%%%%%%%
\section{Drive operator product expansions}
\label{sec:drive_prod}

The drive operators are defined by
\begin{align}
  D_{LM} \equiv \f{\eta_M}{\sqrt{2}}
  \sp{T_{L\abs{M}} + \sign\p{M} T_{L\abs{M}}^\dag},
  &&
  \eta_M \equiv
  \begin{cases}
    \sqrt{2} & M = 0 \\
    \p{-1}^M \sign\p{M}^{3/2} & M \ne 0
  \end{cases},
\end{align}
or, equivalently,
\begin{align}
  D_{LM} \equiv \f{\tau_M}{\sqrt{2}}
  \sp{T_{LM} + \sign\p{M} T_{LM}^\dag},
  &&
  \tau_M \equiv
  \begin{cases}
    \sqrt{2} & M = 0 \\
    \p{-1}^M \sign\p{M}^{M+1/2} & M \ne 0
  \end{cases},
\end{align}
which implies that
\begin{align}
  T_{LM} = \f{\tilde\tau_M}{\sqrt{2}} \p{D_{LM} + i D_{L,-M}},
  &&
  \tilde\tau_M \equiv
  \begin{cases}
    \sqrt{2}/\p{1+i} & M = 0 \\
    \tau_M^* & M \ne 0
  \end{cases},
\end{align}
where $\tau_M^*$ is the complex conjugate of $\tau_M$.  In order to
evaluate the product of two drive operators, we can thus expand them
in terms of transition operators, evaluate the corresponding products
of transition operators (see Appendix \ref{sec:trans_prod}), and
finally expand the result in terms of drive operators.  The general
product of two drive operators takes the form
\begin{align}
  D_{\ell_1 m_1} D_{\ell_2 m_2}
  = \f{\tau_{m_1}\tau_{m_2}}{\sqrt{2}}
  \sum_L \p{g_{\ell_1 m_1;\ell_2 m_2}^{L,m_1+m_2}
    E_{\ell_1 m_1;\ell_2 m_2}^{(L,+)}
    + g_{\ell_1 m_1;\ell_2,-m_2}^{L,m_1-m_2}
    E_{\ell_1 m_1;\ell_2 m_2}^{(L,-)}},
\end{align}
where in terms of
\begin{align}
  \epsilon_M \equiv \p{-1}^M \sign\p{M},
\end{align}
we have
\begin{align}
  E_{\ell_1 m_1;\ell_2 m_2}^{(L,+)}
  &\equiv \sum_{s\in\set{\pm1}}
  \f12 \sp{\tilde\tau_{m_1+m_2}
    + i s \p{-1}^{L+\ell_1+\ell_2}
    \epsilon_{m_1} \epsilon_{m_2} \tilde\tau_{-\p{m_1+m_2}}}
  \sqrt{s} D_{L,s\p{m_1+m_2}},
  \\
  E_{\ell_1 m_1;\ell_2 m_2}^{(L,-)}
  &\equiv \sum_{s\in\set{\pm1}}
  \f12 \sp{\epsilon_{m_2} \tilde\tau_{m_1-m_2}
    + i s \p{-1}^{L+\ell_1+\ell_2}
    \epsilon_{m_1} \tilde\tau_{-\p{m_1-m_2}}}
  \sqrt{s} D_{L,s\p{m_1-m_2}}.
\end{align}
If $m_1\pm m_2=0$, these operators simplify to
\begin{align}
  E_{\ell_1 m;\ell_2,-m}^{(L,+)}
  &= \f1{\sqrt{2}} \sp{1 - \p{-1}^{L+\ell_1+\ell_2}
    \p{1-\delta_{m,0}}} D_{L,0}, \\
  E_{\ell_1 m;\ell_2 m}^{(L,-)}
  &= \f{\epsilon_m}{\sqrt{2}}
  \sp{1 + \p{-1}^{L+\ell_1+\ell_2}} D_{L,0},
\end{align}
and are otherwise
\begin{align}
  E_{\ell_1 m_1;\ell_2 m_2}^{(L,+)}
  &\stackrel{m_1+m_2\ne0}{=}
  \tau_{m_1+m_2}^* \sum_{s\in\set{\pm1}}
  \f12 \sp{1 + s \p{-1}^{L+\ell_1+\ell_2}
    \epsilon_{m_1} \epsilon_{m_2} \epsilon_{m_1+m_2}}
  \sqrt{s} D_{L,s\p{m_1+m_2}}, \\
  E_{\ell_1 m_1;\ell_2 m_2}^{(L,-)}
  &\stackrel{m_1-m_2\ne0}{=}
  \tau_{m_1-m_2}^* \sum_{s\in\set{\pm1}}
  \f12 \sp{\epsilon_{m_2} + s \p{-1}^{L+\ell_1+\ell_2}
    \epsilon_{m_1} \epsilon_{m_1-m_2}}
  \sqrt{s} D_{L,s\p{m_1-m_2}}.
\end{align}
If $\p{m_1,m_2}=\p{0,m}$, then we can use the fact that
$i\tau_m\tau_{-m}^*=\epsilon_m$ for $m\ne0$ to simplify
\begin{align}
  D_{\ell_1,0} D_{\ell_2 m}
  = \sum_L g_{\ell_1,0,\ell_2,m}^{Lm}
  \sum_{s\in\set{\pm1}} \f12 \sp{1 + s \p{-1}^{L+\ell_1+\ell_2}}
  \sqrt{s} D_{L,sm}.
\end{align}

%%%%%%%%%%%%%%%%%%%%%%%%%%%%%%%%%%%%%%%%%%%%%%%%%%%%%%%%%%%%%%%%%%%%%%
\section{Weak SOC in the fully symmetric manifold}
\label{sec:SOC_pert}

Here we derive the first- and second-order effective Hamiltonians
induced on the fully symmetric manifold $\M_\FS$ by the
single-particle weak-SOC Hamiltonian
\begin{align}
  H_{\t{lat}}^{(\phi)}
  = \sum_{q,L} B_L^{(\phi,q)} D_{L,0}^{(q)},
\end{align}
where $D_{L,0}^{(q)}$ is a drive operator that acts on spin $q$, and
$B_L^{(\phi,q)}$ is an effective inhomogeneous driving field amplitude
defined in \eqref{eq:B_L_phi} and \eqref{eq:A_L_phi}.  A perturbative
treatment of $H_{\t{lat}}^{(\phi)}$ yields the effective
Hamiltonians[\red{TODO: cite perturbation theory notes}]
\begin{align}
  H_{\t{eff}}^{(1)}
  &\equiv \sum_L \EE_q\sp{B_L^{(\phi,q)}} \D_{L,0},
  \\
  H_{\t{eff}}^{(2)}
  &\equiv \sum_{J,K}
  \f{\cov_q\sp{B_J^{(\phi,q)}, B_K^{(\phi,q)}}}{u\p{N-1}}
  \p{\D_{J,0} \D_{K,0} - N \sum_L g_{JKL} \D_{L,0}},
\end{align}
where $\EE_q\sp{X_q} \equiv \sum_q X_q / N$ is the mean of $X_q$ over
all occupied quasi-momenta $q$;
\begin{align}
  \cov_q\sp{X_q,Y_q}
  \equiv \EE_q\sp{\p{X_q-\EE_k\sp{X_k}}\p{Y_q-\EE_\ell\sp{Y_\ell}}}
  = \EE_q\sp{X_q Y_q} - \EE_k\sp{X_k} \EE_\ell\sp{Y_\ell},
  \label{eq:cov}
\end{align}
is the covariance of $X_q$ and $Y_q$ over all occupied quasi-momenta
$q$; and $g_{JKL} \equiv g_{J,0;K,0}^{L,0}$ is a structure constant of
the transition operator algebra, defined in \eqref{eq:trans_prod}.
Substituting the form of $B_L^{(\phi,q)}$ in \eqref{eq:B_L_phi}, we
get
\begin{align}
  H_{\t{eff}}^{(1)}
  &= -t \sum_L \EE_q\sp{w_L\p{qa}} A_L^{(\phi)} \D_{L,0},
  \label{eq:H_eff_1}
  \\
  H_{\t{eff}}^{(2)}
  &\equiv \f{t^2}{fU\p{N-1}} \sum_{J,K}
  \cov_q\sp{w_J\p{qa}, w_K\p{qa}} A_J^{(\phi)} A_K^{(\phi)}
  \p{\D_{J,0} \D_{K,0} - N \sum_L g_{JKL} \D_{L,0}}.
  \label{eq:H_eff_2}
\end{align}
where $t$ is the nearest-neighbor tunneling rate, $a$ is the spacing
between neighboring lattice sites, $f$ is the filling fraction of
spatial modes, $U$ is the two-body on-site interaction energy, and
$w_L$ is cosine (sine) for even (odd) $L$.  Note that as the spatial
filling fraction $f\to1$, the sums over occupied quasi-momenta $q$
become sums over nearly all angles $q\in\ZZ_N\times2\pi/N$.  In this
case, when both $J,K$ are even or odd (i.e.~$J=K~\t{mod}~2$) the
covariance in \eqref{eq:H_eff_2} becomes
$\cov_p\sp{\sin\p{p},\sin\p{p}}\to1/2$ or
$\cov_p\sp{\cos\p{p},\cos\p{p}}\to1/2$, whereas when one of $J,K$ is
even and the other is odd (i.e.~$J\ne K~\t{mod}~2$), this covariance
becomes $\cov_p\sp{\cos\p{p},\sin\p{p}}\to0$.  In the case of SU(2),
the effective Hamiltonian $H_{\t{eff}}^{(2)}$ in \eqref{eq:H_eff_2}
reduces to the one-axis twisting Hamiltonian derived in
Ref.~\cite{he2019engineering}.

The effective Hamiltonians in \eqref{eq:H_eff_1} and
\eqref{eq:H_eff_2} are respectively first and second order in the SOC
Hamiltonian $H_{\t{lat}}^{(\phi)}$.  We can extract the leading and
next-to-leading order dependence of these effective Hamiltonians on
the SOC angle $\phi$ by using the expansions of $A_L^{(\phi)}$ in
\eqref{eq:A_L_phi_small}, which gives us
\begin{align}
  H_{\t{eff}}^{(1)}
  &= \phi t \, \EE_p\sp{\sin\p{pa}} \xi_1 \D_{1,0}
  + \phi^2 t \, \EE_q\sp{\cos\p{qa}} \xi_2 \D_{2,0}
  + O\p{\phi^3},
  \label{eq:H_eff_1_phi_apndx} \\
  H_{\t{eff}}^{(2)}
  &= \f{\phi^2 t^2}{\p{N-1}fU} \var_q\sp{\sin\p{qa}}
  \p{\xi_1^2 \D_{1,0}^2 - 2 N \xi_2 \D_{2,0}}
  + O\p{\phi^3},
  \label{eq:H_eff_2_phi_apndx}
\end{align}
where the factors $\xi_L$ are defined in \eqref{eq:scale_fac},
\begin{align}
  \var_q\sp{X_q} \equiv \cov_p\sp{X_p,X_p}
  = \EE_p\sp{\p{X_p-\EE_q\sp{X_q}}^2}
  = \EE_p\sp{X_p^2} - \EE_q\sp{X_q}^2
\end{align}
is the variance of $X_q$ over all occupied quasi-momenta $q$, and we
have neglected scalar terms $\sim\D_{0,0}\propto\1$ that have no
physical consequence.  Note that in the case of SU(2), the re-scaled
drive operator $\xi_1\D_{1,0}$ is precisely the collective spin-$z$
operator $S_\z$ that appears in Ref.~\cite{he2019engineering}.
Furthermore, drive operators $D_{LM}$ vanish when $L\ge n$, so in the
case of SU(2) the operator $\D_{2,0}$ has no contribution to the
effective Hamiltonians in \eqref{eq:H_eff_2_phi_apndx} and
\eqref{eq:H_eff_2_phi_apndx}.

%%%%%%%%%%%%%%%%%%%%%%%%%%%%%%%%%%%%%%%%%%%%%%%%%%%%%%%%%%%%%%%%%%%%%%
\section{Raman drive coefficients}
\label{sec:drive_raman_coeff}

Here we provide the coefficients $\Omega_{LM}^{(\phi,j)}$ of the
effective Raman driving Hamiltonian
\begin{align}
  H_{\t{Raman}}^{(\phi)}
  = \sum_j \sum_{\substack{L\in\set{0,1,2}\\\abs{M}\le L}}
  \Omega_{LM}^{(\phi,j)} D_{LM}^{(j)}
  \label{eq:raman_drive_apndx}
\end{align}
considered in Section \ref{sec:drive_raman}.  These coefficients are
computed using the structure constants of the transition operator
algebra provided in Appendix \ref{sec:trans_prod}.  For clarity, if
any coefficient $\Omega_{LM}^{(\phi,j)}$ is independent of the SOC
angle $\phi$ or lattice site $j$, we suppress the explicit dependence
on $\p{\phi,j}$, i.e.~taking $\Omega_{LM}^{(\phi,j)}\to\Omega_{LM}$.
Defining the re-scaled driving amplitudes
\begin{align}
  \tilde\Omega_{LM}^{(\phi,j)} \equiv \xi_L^{-1} \Omega_{LM}^{(\phi,j)},
  &&
  \tilde\Omega_m \equiv \xi_1^{-1} \Omega_m,
\end{align}
where where the scale factors $\xi_L$ are defined in
\eqref{eq:scale_fac} and $\Omega_m$ are the (real) driving amplitudes
in the ``bare'' Raman drive in \eqref{eq:raman_bare}, the coefficients
of the effective Raman drive in \eqref{eq:raman_drive_apndx} are
determined by
\begin{align}
  \Delta \tilde\Omega_{0,0}
  \equiv -\f{\xi_1^2}{\xi_0^2} \sum_m \tilde\Omega_m^2,
  &&
  \Delta \tilde\Omega_{1,0}
  \equiv \f12 \p{\tilde\Omega_+^2 - \tilde\Omega_-^2},
  &&
  \Delta \tilde\Omega_{2,0}
  \equiv \tilde\Omega_+^2 + \tilde\Omega_-^2 - 2\tilde\Omega_0^2,
  \label{eq:O_X0}
\end{align}
\begin{align}
  \tilde\Omega_{L,1}^{(\phi,j)}
  &\equiv \sum_{s\in\set{\pm1}}
  \omega_{Ls} \cos\p{\sp{\phi_s-\phi_0}j},
  &
  \tilde\Omega_{L,-1}^{(\phi,j)}
  &\equiv -\sum_{s\in\set{\pm1}}
  s \omega_{Ls} \sin\p{\sp{\phi_s-\phi_0}j}, \\
  \tilde\Omega_{2,2}^{(\phi,j)}
  &\equiv \omega_{2,2} \cos\p{\sp{\phi_+-\phi_-}j},
  &
  \tilde\Omega_{2,-2}^{(\phi,j)}
  &\equiv -\omega_{2,2} \sin\p{\sp{\phi_+-\phi_-}j},
\end{align}
\begin{align}
  \Delta \omega_{1\pm}
  \equiv \f1{\sqrt{2}} \, \tilde\Omega_0 \tilde\Omega_\pm,
  &&
  \Delta \omega_{2\pm}
  \equiv \pm \sqrt{6} \, \tilde\Omega_0\tilde\Omega_\pm,
  &&
  \Delta \omega_{2,2}
  \equiv 2\sqrt{3} \, \tilde\Omega_+\tilde\Omega_-.
\end{align}
Here $\phi_m$ and $\Delta$ are respectively the phases and detuning
that appear in the ``bare'' Raman drive in \eqref{eq:raman_bare}.  If
the driving lasers are phase-matched with $\phi_m=m\phi$, such that a
nuclear spin transition $\mu\to\mu+m$ on site $j$ imprints the phase
$m\phi j$, then the drive $H_{\t{Raman}}^{(\phi)}$ takes the form
\begin{align}
  \left. H_{\t{Raman}}^{(\phi)} \right|_{\phi_m=m\phi}
  = \sum_{\substack{L\le2\\0\le M\le L}} \Omega_{LM} \D_{LM}^{(\phi)},
\end{align}
with inhomogeneously rotated collective drive operators
\begin{align}
  \D_{LM}^{(\phi)}
  \equiv \sum_j \sp{\cos\p{M\phi j} D_{LM}^{(j)}
    - \sin\p{M\phi j} D_{L,-M}^{(j)}},
\end{align}
and uniform coefficients determined by \eqref{eq:O_X0} and
\begin{align}
  \tilde\Omega_{1,-1} = \tilde\Omega_{2,-1}
  = \tilde\Omega_{2,-2} \equiv 0,
  &&
  \tilde\Omega_{L,1} \equiv \sum_{s\in\set{\pm1}} \omega_{Ls},
  &&
  \tilde\Omega_{2,2} \equiv \omega_{2,2}.
\end{align}


\bibliography{multilevel_spin_notes.bib}

\end{document}

% Level statistics of Hamiltonian
% Dynamics of single-particle observables
% Dynamics of two-point correlators
% - Different sites
% - Two-time correlators
% TWA dynamics