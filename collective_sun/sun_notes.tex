\documentclass[nofootinbib,notitlepage,11pt]{revtex4-2}

%%% linking references
\usepackage{hyperref}
\hypersetup{
  breaklinks=true,
  colorlinks=true,
  linkcolor=blue,
  filecolor=magenta,
  urlcolor=cyan,
}

%%% header / footer
\usepackage{fancyhdr} % easier header and footer management
\pagestyle{fancy} % page formatting style
\fancyhf{} % clear all header and footer text
\renewcommand{\headrulewidth}{0pt} % remove horizontal line in header
\usepackage{lastpage} % for referencing last page
\cfoot{\thepage~of \pageref{LastPage}} % "x of y" page labeling


%%% symbols, notations, etc.
\usepackage{physics,braket,bm,amssymb} % physics and math
\renewcommand{\t}{\text} % text in math mode
\newcommand{\f}[2]{\dfrac{#1}{#2}} % shorthand for fractions
\newcommand{\p}[1]{\left(#1\right)} % parenthesis
\renewcommand{\sp}[1]{\left[#1\right]} % square parenthesis
\renewcommand{\set}[1]{\left\{#1\right\}} % curly parenthesis
\newcommand{\bk}{\Braket} % shorthand for braket notation
\renewcommand{\v}{\bm} % bold vectors
\newcommand{\uv}[1]{\bm{\hat{#1}}} % unit vectors
\newcommand{\av}{\vec} % arrow vectors
\renewcommand{\d}{\text{d}} % for infinitesimals
\renewcommand{\c}{\cdot} % inner product

\usepackage{dsfont} % for identity operator
\newcommand{\1}{\mathds{1}}

\newcommand{\up}{\uparrow}
\newcommand{\dn}{\downarrow}

\newcommand{\x}{\text{x}}
\newcommand{\y}{\text{y}}
\newcommand{\z}{\text{z}}

\newcommand{\B}{\mathcal{B}}
\newcommand{\D}{\mathcal{D}}
\newcommand{\I}{\mathcal{I}}
\newcommand{\N}{\mathcal{N}}
\renewcommand{\O}{\mathcal{O}}
\newcommand{\Q}{\mathcal{Q}}
\newcommand{\T}{\mathcal{T}}
\renewcommand{\S}{\mathcal{S}}

\def\obra#1{\mathinner{({#1}|}}
\def\oket#1{\mathinner{|{#1})}}
\def\obk#1{\mathinner{({#1})}}
\def\oop#1#2{\oket{#1}\!\obra{#2}}

%%%%%%%%%%%%%%%%%%%%%%%%%%%%%%%%%%%%%%%%%%%%%%%%%%%%%%%%%%%%%%%%%%%%%%
\begin{document}
\thispagestyle{fancy}

\title{Multi-level generalizations of collective spin dynamics and
  squeezing}%
\author{Michael A. Perlin}%
\date{\today}

\maketitle

\tableofcontents

%%%%%%%%%%%%%%%%%%%%%%%%%%%%%%%%%%%%%%%%%%%%%%%%%%%%%%%%%%%%%%%%%%%%%%
\section{Collective SU($n$)-symmetric interactions}

We consider a collection of $n$-level fermions on a 1-D lattice, with
a single-particle Hamiltonian
\begin{align}
  H_{\t{lat}} = -J \sum_{q,\mu} \cos\p{qa} c_{q\mu}^\dag c_{q\mu},
  \label{eq:H_lat}
\end{align}
where $J$ is the nearest-neighbor tunneling rate, $a$ is the spacing
between neighboring lattice sites, and $c_{q\mu}$ is the fermionic
annihilation operator for a particle in quasi-momentum mode $q$ with
internal state (e.g.~nuclear spin) $\mu$.  Under the frozen-mode
approximation, two-body collective SU($n$)-symmetric interactions
between these fermions take form\cite{perlin2019effective}
\begin{align}
  H_{\t{int}} = \f12 u \sum_{p,q,\mu,\nu}
  \p{c_{p\mu}^\dag c_{q\nu}^\dag c_{q\nu} c_{p\mu}
    + c_{q\mu}^\dag c_{p\nu}^\dag c_{q\nu} c_{p\mu}},
\end{align}
where $u\equiv U/L$ is the two-body on-site interaction energy $U$
divided by the number of lattice sites $L$.  Defining the spin
operators $s_{\mu\nu}^{(p)}\equiv c_{p\mu}^\dag c_{p\nu}$ for each
quasi-momentum $p$, we can alternately write
\begin{align}
  H_{\t{int}} = \f12 u \sum_{p,q,\mu,\nu}
  \p{s_{\mu\mu}^{(p)} s_{\nu\nu}^{(q)} - s_{\mu\nu}^{(p)} s_{\nu\mu}^{(q)}}
  = \f12 u\p{N^2 - \sum_{p,q,\mu,\nu} s_{\mu\nu}^{(p)} s_{\nu\mu}^{(q)}},
\end{align}
where $N$ is the total number of particles.  In the absence of
coherence between sectors of different particle number, we can neglect
the $\sim N^2$ term and simply write
\begin{align}
  H_{\t{int}}
  = - \f12 u \sum_{p,q,\mu,\nu} s_{\mu\nu}^{(p)} s_{\nu\mu}^{(q)}.
\end{align}
In the case of SU(2), at this point we would expand the spin operators
$s_{\mu\nu}^{(p)}$ in terms of Pauli operators in order to convert
$H_{\t{int}}$ into an SU(2) spin Hamiltonian.  To generalize this
procedure to the case of SU($n$), we first define a vector
$\v s^{(p)}$ of all spin operators $s_{\mu\nu}^{(p)}$, and write
\begin{align}
  H_{\t{int}}
  = - \f12 u \sum_{p,q} {\v s^{(p)}}^\dag \v s^{(q)}
  = - \f12 u \v\S^\dag \c \v\S,
  &&
  \v\S\equiv \sum_p \v s^{(p)}.
  \label{eq:H_int_spin}
\end{align}
In the following sections, we identify various bases for the space of
operators on the Hilbert space of an $n$-level system.  For brevity,
we denote this space of operators by $\B_n$, and note that $\B_n$ is
itself an $n^2$-dimensional Hilbert space equipped with a trace
(Hilbert-Schmidt) inner product
\begin{align}
  \obk{\O|\Q} \equiv \tr\p{\O^\dag Q},
\end{align}
where $\oket{\Q}$ and $\obra{\O}$ respectively denote vectors in
$\B_n$ and its dual space $\B_n^*$.  Similarly to the set of all spin
operators $s_{\mu\nu}\equiv\op{\mu}{\nu}$ on the single-particle
Hilbert space at each quasi-momentum, all of the bases we consider
will be orthonormal with respect to the trace inner product.  As we
show in Appendix \ref{sec:changing_bases}, the collective
SU($n$)-symmetric interaction Hamiltonian in \eqref{eq:H_int_spin}
takes an identical form in any orthonormal basis for $\B_n$.

%%%%%%%%%%%%%%%%%%%%%%%%%%%%%%%%%%%%%%%%%%%%%%%%%%
\subsection{Generalized Gell-Mann (GGM) operators}

The most obvious generalization of Pauli operators to the case of
SU($n$) are the {\it generalized Gell-Mann (GGM)
  operators}\cite{hioe1981level, bertlmann2008bloch}, defined for
$j,k,\ell\in\set{0,1,\cdots,n-1}$ with $k<j$ and $\ell>0$ by
\begin{align}
  \lambda_0 \equiv \sqrt{\f{2}{n}}~ \1,
  &&
  \lambda_\ell \equiv \sqrt{\f{2}{\ell\p{\ell+1}}}
  \p{\sum_{n=0}^{\ell-1}\op{n} - \ell \op{\ell}},
  \label{eq:GGM_diag}
\end{align}
\begin{align}
  \lambda_{jk,+} \equiv \op{j}{k} + \op{k}{j},
  &&
  \lambda_{jk,-} \equiv i\p{\op{j}{k} - \op{k}{j}},
  \label{eq:GGM_off_diag}
\end{align}
where $\1$ is the identity operator.  The self-adjoint GGM operators
in \eqref{eq:GGM_diag} and \eqref{eq:GGM_off_diag} provide a complete
basis for the space of operators on the Hilbert space of an $n$-level
system, coincide exactly with the Pauli operators in the case of
$n=2$, and satisfy the orthonormality condition
\begin{align}
  \obk{\lambda_a|\lambda_b} = 2\delta_{ab},
  \label{eq:GGM_inner}
\end{align}
where $a,b$ index any GGM operator in \eqref{eq:GGM_diag} or
\eqref{eq:GGM_off_diag}.  Denoting a vector of all GGM operators by
$\v\lambda$, we can therefore write (see Appendix
\ref{sec:changing_bases})
\begin{align}
  H_{\t{int}}
  = -\f14 u \sum_{p,q} \v\lambda^{(p)} \c \v\lambda^{(q)}
  = -u \v\Lambda \c \v\Lambda,
  &&
  \v\Lambda \equiv \f12 \sum_p \v\lambda^{(p)}.
  \label{eq:H_int_GGM}
\end{align}

%%%%%%%%%%%%%%%%%%%%%%%%%%%%%%%%%%%%%%%%%%%%%%%%%%
\subsection{Transition operators}

The GGM operators in \eqref{eq:GGM_diag} and \eqref{eq:GGM_off_diag}
provide a convenient operator basis to describe dynamics obeying
SU($n$) symmetry.  An external driving field addressing $n$-level
nuclear spins, however, will generally violate SU($n$) symmetry, and
instead obeys the symmetries of the {\it polarization} or {\it
  transition operators} defined by\cite{kryszewski2006alternative,
  bertlmann2008bloch}
\begin{align}
  T_{LM} \equiv \sqrt{\f{2L+1}{2I+1}} \sum_{\mu,\nu}
  \bk{I\mu;I\nu,LM} \op{\mu}{\nu},
  \label{eq:trans_ops}
\end{align}
where $L\in\set{0,1,\cdots,n-1}$ and $M\in\set{-L,-L+1,L}$ index a
total spin and its projection onto a quantization axis;
$I\equiv\p{n-1}/2$ is the maximal angular momentum of an $n$-level
spin; $\mu,\nu\in\set{-I,-I+1,\cdots,I}$ index projections of an
$n$-level nuclear spin onto a quantization axis; and
$\bk{I\mu;I\nu,LM}$ is a Clebsch-Gordan coefficient.  Similarly to the
GGM operators in \eqref{eq:GGM_diag} and \eqref{eq:GGM_off_diag}, the
transition operators in \eqref{eq:trans_ops} provide a complete basis
for the space of operators on the Hilbert space of an $n$-level
system, and satisfy the orthonormality condition
\begin{align}
  \obk{T_{LM}|T_{L'M'}} = \delta_{LL'} \delta_{MM'},
\end{align}
which implies
\begin{align}
  H_{\t{int}} = -\f12 u \sum_{p,q} {\v T^{(p)}}^\dag \c \v T^{(q)}
  = -\f12 u \v\T^\dag \c \v\T,
  &&
  \v\T \equiv \v\T\equiv\sum_p\v T^{(p)},
  \label{eq:H_int_trans}
\end{align}
where $\v T$ is a vector of all transition operators in
\eqref{eq:trans_ops}.

%%%%%%%%%%%%%%%%%%%%%%%%%%%%%%%%%%%%%%%%%%%%%%%%%%
\subsection{Drive operators}

Unlike the GGM operators in \eqref{eq:GGM_diag} and
\eqref{eq:GGM_off_diag}, the transition operators in
\eqref{eq:trans_ops} are not self-adjoint.  In order to express
Hamiltonians in terms of self-adjoint operators, we define the {\it
  drive operators}
\begin{align}
  D_{L,0} \equiv \sqrt{2}~ T_{L,0},
  &&
  D_{LM} \stackrel{M>0}{\equiv} -\p{T_{L\Delta} + T_{L\Delta}^\dag},
  &&
  D_{LM} \stackrel{M<0}{\equiv} i\p{T_{L\Delta} - T_{L\Delta}^\dag},
  \label{eq:drive_ops}
\end{align}
for integer $M$ with $\abs{M}\le L$.  Similarly to the GGM operators
in \eqref{eq:GGM_diag} and \eqref{eq:GGM_off_diag}, the drive
operators are equal to the Pauli operators in the case of $n=2$, and
satisfy the orthonormality condition
\begin{align}
  \obk{D_{LM}|D_{L'M'}} = 2\delta_{LL'}\delta_{MM'},
\end{align}
where $a,b$ index any drive operator in \eqref{eq:drive_ops}.  These
operators have the convenient feature that $D_{1,1}$ is (up to a gauge
transformation) proportional to the Hamiltonian induced on an
$n$-level spin by a classical driving laser.  In terms of the drive
operators,
\begin{align}
  H_{\t{int}} = -\f14 u \sum_{p,q} \v D^{(p)}\c\v D^{(q)}
  = -u \v\D \c \v\D,
  &&
  \v\D \equiv \f12 \sum_p \v D^{(p)},
  \label{eq:H_int_drive}
\end{align}
where $\v D$ is a vector of all drive operators in
\eqref{eq:drive_ops}.

We note that Weyl operators\cite{bertlmann2008bloch} and self-adjoint
linear combinations thereof\cite{asadian2016heisenbergweyl} also
provide an interesting basis of operators, generalizing continuous
phase-space displacement operators to the case of a finite-dimensional
Hilbert space.  As our present work is motivated by multi-level
systems that are realized with nuclear spin (angular momentum) degrees
of freedom, however, the transition and drive operators are more
natural to consider than the Weyl operators, so we will not discuss
the latter any further.

%%%%%%%%%%%%%%%%%%%%%%%%%%%%%%%%%%%%%%%%%%%%%%%%%%%%%%%%%%%%%%%%%%%%%%
\section{External drive and spin-orbit coupling}

An external driving laser addressing a spin-$1/2$ particle on a
lattice will induce the Hamiltonian
\begin{align}
  \left. H_{\t{drive}}^{(\phi)} \right|_{n=2}
  = \f12 \Omega \sum_j e^{i\phi_j} c_{j,\up}^\dag c_{j,\dn} + \t{h.c.},
  \label{eq:H_drive_2}
\end{align}
where $\Omega$ is a drive strength, $j$ indexes a lattice site, and
$\phi_j$ is a site-dependent phase.  The multi-level analogue of
\eqref{eq:H_drive_2} is
\begin{align}
  H_{\t{drive}}^{(\phi)}
  = -\f12 \Omega \sum_j e^{i\phi_j} T_{1,1}^{(j)} + \t{h.c.}.
  \label{eq:H_drive}
\end{align}
We can make this drive spatially homogenous via the gauge
transformation $c_{j\mu} \to e^{i\mu\phi_j} c_{j\mu}$, after which the
drive becomes
\begin{align}
  H_{\t{drive}}
  = -\f12 \Omega \sum_j T_{1,1}^{(j)} + \t{h.c.}
  = \f12 \Omega \sum_j D_{1,1}
  = \Omega \D_{1,1},
\end{align}
where $\D_{LM}\equiv\sum_jD_{LM}^{(j)}/2$ is a collective drive
operator, with e.g.~$\D_{1,1}=S_\x$ in the case of $n=2$.  The SU($n$)
symmetry of collective interactions implies that the interaction
Hamiltonian $H_{\t{int}}$ in \eqref{eq:H_int_drive} is invariant under
this gauge transformation, but the single-particle Hamiltonian in
\eqref{eq:H_lat} transforms as
\begin{align}
  H_{\t{lat}}
  \to H_{\t{lat}}^{(\phi)}
  = -J \sum_{q,\mu} \cos\p{qa+\mu\phi} c_{q\mu}^\dag c_{q\mu},
  \label{eq:H_lat_SOC}
\end{align}
where $\phi$ is the relative phase of the driving laser between
adjacent lattice sites.  In the basis of drive operators, this
Hamiltonian takes the form (see Appendix \ref{sec:lat_drive})
\begin{align}
  H_{\t{lat}}^{(\phi)} = -J \sum_{q,L} B_L\p{q,\phi} D_{L,0}^{(q)},
  \label{eq:H_lat_drive}
\end{align}
with effective driving field strengths
\begin{align}
  B_L\p{q,\phi} = \p{-1}^L f_L\p{qa} \sqrt{\f{L+1/2}{n}}
  \sum_\mu \bk{I\mu;I\mu,L,0} f_L\p{\mu\phi},
  &&
  f_L\p{\theta} \equiv
  \begin{cases}
    \cos\theta & L~\t{even} \\
    \sin\theta & L~\t{odd}
  \end{cases}.
\end{align}


%%%%%%%%%%%%%%%%%%%%%%%%%%%%%%%%%%%%%%%%%%%%%%%%%%%%%%%%%%%%%%%%%%%%%%
\section{Classifying eigenstates}

[not yet written]


\appendix

%%%%%%%%%%%%%%%%%%%%%%%%%%%%%%%%%%%%%%%%%%%%%%%%%%%%%%%%%%%%%%%%%%%%%%
\section{Changing operator bases for collective SU($n$)-symmetric
  interactions}
\label{sec:changing_bases}

Here we show that the collective interaction Hamiltonian in
\eqref{eq:H_int_spin} takes an identical form in any orthonormal basis
for $\B_n$, i.e.~the space of operators on the Hilbert space of an
$n$-level system, equipped with the trace (Hilbert-Schmidt) inner
product
\begin{align}
  \obk{\O|\Q} \equiv \tr\p{\O^\dag \Q}.
\end{align}
To maintain generality, we choose an arbitrary basis $\set{X_j}$
satisfying the orthonormality condition
\begin{align}
  \obk{X_j|X_k} = \N_X \delta_{jk},
\end{align}
for $j\in\set{0,1,\cdots,n^2-1}$.  This orthonormality condition
implies that we can resolve the identity (super-)operator $\I$ with
$\I\oket{\O}=\oket{\O}$ for any $\oket\O\in\B_n$ as
\begin{align}
  \I = \f1{\N_X} \sum_j \oop{X_j}{X_j}
  = \f1{\N_X} \sum_j \oop{X_j^\dag}{X_j^\dag},
\end{align}
where we used the fact that $\set{X_j^\dag}$ is also an orthonormal
basis for $\B_n$, satisfying the same orthonormality condition as
$\set{X_j}$; these two bases are transformed into each other by
unitary $\sum_j\oop{X_j^\dag}{X_j}/\sqrt{\N_X}$.

The collective interaction Hamiltonian in \eqref{eq:H_int_spin} takes
the form
\begin{align}
  H = \sum_{p,q,j} {X_j^{(p)}}^\dag X_j^{(q)},
\end{align}
which essentially consists of two-body interactions ($p\ne q$) and
self-interactions ($p=q$), respectively of the form
\begin{align}
  \sum_j X_j \otimes X_j^\dag,
  &&
  \sum_j X_j X_j^\dag.
\end{align}
Choosing a different basis $\set{Y_j}$ for $\B_n$ with corresponding
norm $\N_Y$, we can expand
\begin{align}
  X_j = \f1{\N_Y} \sum_k Y_k \obk{Y_k|X_j},
  &&
  X_j^\dag = \f1{\N_Y} \sum_k Y_k^\dag \obk{Y_k^\dag|X_j^\dag}
  = \f1{\N_Y} \sum_k Y_k^\dag \obk{X_j|Y_k},
\end{align}
where we used the fact that
\begin{align}
  \obk{Y_k^\dag|X_j^\dag}
  = \tr\p{Y_k X_j^\dag}
  = \tr\p{X_j^\dag Y_k}
  = \obk{X_j|Y_k}.
\end{align}
By resolution of the identity, we can therefore find
\begin{align}
  \sum_j X_j \otimes X_j^\dag
  = \f1{\N_Y^2} \sum_{j,k,\ell} Y_k \obk{Y_k|X_j}
  \otimes Y_\ell^\dag \obk{X_j|Y_\ell}
  = \f{\N_X}{\N_Y^2} \sum_{k,\ell} Y_k \otimes Y_\ell^\dag
  \obk{Y_k|\I|Y_\ell}
  = \f{\N_X}{\N_Y} \sum_k Y_k \otimes Y_k^\dag,
\end{align}
and similarly
\begin{align}
  \sum_j X_j X_j^\dag
  = \f1{\N_Y^2} \sum_{j,k,\ell} Y_k Y_\ell^\dag
  \obk{Y_k|X_j} \obk{X_j|Y_\ell}
  = \f{\N_X}{\N_Y} \sum_k Y_k Y_k^\dag.
\end{align}
The collective interaction Hamiltonian in \eqref{eq:H_int_spin} thus
takes an identical form in any orthonormal basis for the space $\B_n$
of operators on the Hilbert space of an $n$-level system, up to
appropriate re-scaling factors $\N_X/\N_Y$.

%%%%%%%%%%%%%%%%%%%%%%%%%%%%%%%%%%%%%%%%%%%%%%%%%%%%%%%%%%%%%%%%%%%%%%
\section{Spin-orbit coupling with drive operators}
\label{sec:lat_drive}

Here we take the single-particle spin-orbit coupled Hamiltonian
$H_{\t{lat}}^{(\phi)}$ in \eqref{eq:H_lat_SOC}, and derive its
expression in \eqref{eq:H_lat_drive} in terms of the drive operators
$D_{LM}$.  In the basis of ordinary spin operators $s_{\mu\mu}$, this
Hamiltonian takes the form
\begin{align}
  H_{\t{lat}}^{(\phi)}
  = -J \sum_{q,\mu} \cos\p{qa+\mu\phi} s_{\mu\mu}^{(q)}
  \label{eq:H_lat_spin}
\end{align}
Defining the diagonal ``spin-wave'' operators
\begin{align}
  \omega_{\phi,+} \equiv \sum_\mu \cos\p{\mu\phi} s_{\mu\mu},
  &&
  \omega_{\phi,-} \equiv \sum_\mu \sin\p{\mu\phi} s_{\mu\mu},
\end{align}
which are respectively even and odd under spin inversion
($\mu\to-\mu$), we can simplify
\begin{align}
  \sum_\mu \cos\p{qa+\mu\phi} s_{\mu\mu}
  = \cos\p{qa} \omega_{\phi,+} - \sin\p{qa} \omega_{\phi,-},
\end{align}
and resolve the identity (super-)operator $\I$ with
$\I \omega_{\phi,\pm} = \omega_{\phi,\pm}$ in the basis of the drive
operators, finding
\begin{align}
  \sum_\mu \cos\p{qa+\mu\phi} s_{\mu\mu}
  = \cos\p{qa} \f12 \sum_{L,M} \obk{\omega_{\phi,+}|D_{LM}} D_{LM}
  - \sin\p{qa} \f12 \sum_{L,M} \obk{\omega_{\phi,-}|D_{LM}} D_{LM}.
\end{align}
We now note that the drive operators $D_{LM}$ are strictly
off-diagonal for $M\ne0$, and furthermore that $D_{L,0}$ with even
(odd) $L$ is even (odd) under spin inversion, which implies
\begin{align}
  \sum_{L,M} \obk{\omega_{\phi,+}|D_{LM}} D_{LM}
  = \sum_{L\,\t{even}} \obk{\omega_{\phi,+}|D_{L,0}} D_{L,0},
\end{align}
and likewise with $\omega_{\phi,-}$ and odd $L$.  In total, we can
write
\begin{align}
  \sum_\mu \cos\p{qa+\mu\phi} s_{\mu\mu}
  = \cos\p{qa} \f12 \sum_{L\,\t{even}}
  \obk{\omega_{\phi,+}|D_{L,0}} D_{L,0}
  - \sin\p{qa} \f12 \sum_{L\,\t{odd}}
  \obk{\omega_{\phi,-}|D_{L,0}} D_{L,0},
\end{align}
or, more compactly,
\begin{align}
  \sum_\mu \cos\p{qa+\mu\phi} s_{\mu\mu}
  = \sum_L B_L\p{q,\phi} D_{L,0},
  \label{eq:compact_SOC_drive}
\end{align}
with effective driving field strengths
\begin{align}
  B_L\p{q,\phi}
  &\equiv \p{-1}^L f_L\p{qa} \f12
  \sum_\mu f_L\p{\mu\phi} \obk{s_{\mu\mu}|D_{L,0}} \\
  &= \p{-1}^L f_L\p{qa} \sqrt{\f{L+1/2}{n}}
  \sum_\mu \bk{I\mu;I\mu,L,0} f_L\p{\mu\phi},
\end{align}
where
\begin{align}
  f_L\p{\theta} \equiv
  \begin{cases}
    \cos\theta & L~\t{even} \\
    \sin\theta & L~\t{odd}
  \end{cases}.
\end{align}
Substituting the result in \eqref{eq:compact_SOC_drive} into the
single-particle Hamiltonian in \eqref{eq:H_lat_spin} yields
\begin{align}
  H_{\t{lat}}^{(\phi)} = -J \sum_{q,L} B_L\p{q,\phi} D_{L,0}^{(q)}.
\end{align}


\bibliography{sun_notes.bib}

\end{document}
