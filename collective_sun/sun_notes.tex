\documentclass[nofootinbib,notitlepage,11pt]{revtex4-2}

%%% linking references
\usepackage{hyperref}
\hypersetup{
  breaklinks=true,
  colorlinks=true,
  linkcolor=blue,
  filecolor=magenta,
  urlcolor=cyan,
}

%%% header / footer
\usepackage{fancyhdr} % easier header and footer management
\pagestyle{fancy} % page formatting style
\fancyhf{} % clear all header and footer text
\renewcommand{\headrulewidth}{0pt} % remove horizontal line in header
\usepackage{lastpage} % for referencing last page
\cfoot{\thepage~of \pageref{LastPage}} % "x of y" page labeling


%%% symbols, notations, etc.
\usepackage{physics,braket,bm,amssymb} % physics and math
\renewcommand{\t}{\text} % text in math mode
\newcommand{\f}[2]{\dfrac{#1}{#2}} % shorthand for fractions
\newcommand{\p}[1]{\left(#1\right)} % parenthesis
\renewcommand{\sp}[1]{\left[#1\right]} % square parenthesis
\renewcommand{\set}[1]{\left\{#1\right\}} % curly parenthesis
\newcommand{\bk}{\Braket} % shorthand for braket notation
\renewcommand{\v}{\bm} % bold vectors
\newcommand{\uv}[1]{\bm{\hat{#1}}} % unit vectors
\newcommand{\av}{\vec} % arrow vectors
\renewcommand{\d}{\text{d}} % for infinitesimals
\renewcommand{\c}{\cdot} % inner product

\usepackage{dsfont} % for identity operator
\newcommand{\1}{\mathds{1}}

\newcommand{\up}{\uparrow}
\newcommand{\dn}{\downarrow}

\newcommand{\x}{\text{x}}
\newcommand{\y}{\text{y}}
\newcommand{\z}{\text{z}}

\newcommand{\B}{\mathcal{B}}
\newcommand{\D}{\mathcal{D}}
\newcommand{\I}{\mathcal{I}}
\newcommand{\N}{\mathcal{N}}
\renewcommand{\O}{\mathcal{O}}
\newcommand{\Q}{\mathcal{Q}}
\newcommand{\T}{\mathcal{T}}
\renewcommand{\S}{\mathcal{S}}

\def\obra#1{\mathinner{({#1}|}}
\def\oket#1{\mathinner{|{#1})}}
\def\obk#1{\mathinner{({#1})}}
\def\oop#1#2{\oket{#1}\!\obra{#2}}

\usepackage[inline]{enumitem} % in-line lists and \setlist{} (below)
\setlist{nolistsep} % more compact spacing between environments

%%%%%%%%%%%%%%%%%%%%%%%%%%%%%%%%%%%%%%%%%%%%%%%%%%%%%%%%%%%%%%%%%%%%%%
\begin{document}
\thispagestyle{fancy}

\title{Multi-level generalizations of collective spin dynamics and
  squeezing}%
\author{Michael A. Perlin}%
\date{\today}

\maketitle

\tableofcontents

%%%%%%%%%%%%%%%%%%%%%%%%%%%%%%%%%%%%%%%%%%%%%%%%%%%%%%%%%%%%%%%%%%%%%%
\section{Collective SU($n$)-symmetric interactions}
\label{sec:int}

We consider a collection of $n$-level fermions on a 1-D lattice, with
a single-particle Hamiltonian
\begin{align}
  H_{\t{lat}} = -J \sum_{q,\mu} \cos\p{qa} c_{q\mu}^\dag c_{q\mu},
  \label{eq:H_lat}
\end{align}
where $J$ is the nearest-neighbor tunneling rate, $a$ is the spacing
between neighboring lattice sites, and $c_{q\mu}$ is the fermionic
annihilation operator for a particle in quasi-momentum mode $q$ with
internal state (e.g.~nuclear spin) $\mu$.  Under the frozen-mode
approximation, two-body collective SU($n$)-symmetric interactions
between these fermions take form\cite{perlin2019effective}
\begin{align}
  H_{\t{int}} = \f{u}{2} \sum_{p,q,\mu,\nu}
  \p{c_{p\mu}^\dag c_{q\nu}^\dag c_{q\nu} c_{p\mu}
    + c_{q\mu}^\dag c_{p\nu}^\dag c_{q\nu} c_{p\mu}},
\end{align}
where $u\equiv U/L$ is the two-body on-site interaction energy $U$
divided by the number of lattice sites $L$.  Defining the spin
operators $s_{\mu\nu}^{(p)}\equiv c_{p\mu}^\dag c_{p\nu}$ for each
quasi-momentum $p$, we can alternately write
\begin{align}
  H_{\t{int}} = \f{u}{2} \sum_{p,q,\mu,\nu}
  \p{s_{\mu\mu}^{(p)} s_{\nu\nu}^{(q)} - s_{\mu\nu}^{(p)} s_{\nu\mu}^{(q)}}
  = \f{u}{2}\p{N^2 - \sum_{p,q,\mu,\nu} s_{\mu\nu}^{(p)} s_{\nu\mu}^{(q)}},
\end{align}
where $N$ is the total number of particles.  In the absence of
coherence between sectors of different particle number, we can neglect
the $\sim N^2$ term and simply write
\begin{align}
  H_{\t{int}}
  = - \f{u}{2} \sum_{p,q,\mu,\nu} s_{\mu\nu}^{(p)} s_{\nu\mu}^{(q)}.
\end{align}
In the case of SU(2), at this point we would expand the spin operators
$s_{\mu\nu}^{(p)}$ in terms of Pauli operators in order to convert
$H_{\t{int}}$ into an SU(2) spin Hamiltonian.  To generalize this
procedure to the case of SU($n$), we first define a vector
$\v s^{(p)}$ of all spin operators $s_{\mu\nu}^{(p)}$, and write
\begin{align}
  H_{\t{int}}
  = - \f{u}{2} \sum_{p,q} {\v s^{(p)}}^\dag \v s^{(q)}
  = - \f{u}{2} \v\S^\dag \c \v\S,
  &&
  \v\S\equiv \sum_p \v s^{(p)}.
  \label{eq:H_int_spin}
\end{align}
In the following sections, we identify various bases for the space of
operators on the Hilbert space of an $n$-level system.  For brevity,
we denote this space of operators by $\B_n$, and note that $\B_n$ is
itself an $n^2$-dimensional Hilbert space equipped with a trace
(Hilbert-Schmidt) inner product
\begin{align}
  \obk{\O|\Q} \equiv \tr\p{\O^\dag Q},
\end{align}
where $\oket{\Q}$ and $\obra{\O}$ respectively denote vectors in
$\B_n$ and its dual space $\B_n^*$.  Similarly to the set of all spin
operators $s_{\mu\nu}\equiv\op{\mu}{\nu}$ on the single-particle
Hilbert space at each quasi-momentum, all of the bases we consider
will be orthonormal with respect to the trace inner product.  As we
show in Appendix \ref{sec:changing_bases}, the collective
SU($n$)-symmetric interaction Hamiltonian in \eqref{eq:H_int_spin}
takes an identical form in any orthonormal basis for $\B_n$.

%%%%%%%%%%%%%%%%%%%%%%%%%%%%%%%%%%%%%%%%%%%%%%%%%%
\subsection{Generalized Gell-Mann (GGM) operators}
\label{sec:ggm_ops}

The most obvious generalization of Pauli operators to the case of
SU($n$) are the {\it generalized Gell-Mann (GGM)
  operators}\cite{hioe1981level, bertlmann2008bloch}, defined for
$j,k,\ell\in\set{0,1,\cdots,n-1}$ with $k<j$ and $\ell>0$ by
\begin{align}
  \lambda_0 \equiv \sqrt{\f{2}{n}}~ \1,
  &&
  \lambda_\ell \equiv \sqrt{\f{2}{\ell\p{\ell+1}}}
  \p{\sum_{n=0}^{\ell-1}\op{n} - \ell \op{\ell}},
  \label{eq:GGM_diag}
\end{align}
\begin{align}
  \lambda_{jk,+} \equiv \op{j}{k} + \op{k}{j},
  &&
  \lambda_{jk,-} \equiv i\p{\op{j}{k} - \op{k}{j}},
  \label{eq:GGM_off_diag}
\end{align}
where $\1$ is the identity operator.  The self-adjoint GGM operators
in \eqref{eq:GGM_diag} and \eqref{eq:GGM_off_diag} provide a complete
basis for the space of operators on the Hilbert space of an $n$-level
system, coincide exactly with the Pauli operators in the case of
$n=2$, and satisfy the orthonormality condition
\begin{align}
  \obk{\lambda_a|\lambda_b} = 2\delta_{ab},
  \label{eq:GGM_inner}
\end{align}
where $a,b$ index any GGM operator in \eqref{eq:GGM_diag} or
\eqref{eq:GGM_off_diag}.  Denoting a vector of all GGM operators by
$\v\lambda$, we can therefore write (see Appendix
\ref{sec:changing_bases})
\begin{align}
  H_{\t{int}}
  = -\f{u}{4} \sum_{p,q} \v\lambda^{(p)} \c \v\lambda^{(q)}
  = -\f{u}{4} \v\Lambda \c \v\Lambda,
  &&
  \v\Lambda \equiv \sum_p \v\lambda^{(p)}.
  \label{eq:H_int_GGM}
\end{align}

%%%%%%%%%%%%%%%%%%%%%%%%%%%%%%%%%%%%%%%%%%%%%%%%%%
\subsection{Transition operators}
\label{sec:trans_ops}

The GGM operators in \eqref{eq:GGM_diag} and \eqref{eq:GGM_off_diag}
provide a convenient operator basis to describe dynamics obeying
SU($n$) symmetry.  An external driving field addressing $n$-level
nuclear spins, however, will generally violate SU($n$) symmetry, and
instead obeys the symmetries of the {\it polarization} or {\it
  transition operators} defined by\cite{kryszewski2006alternative,
  bertlmann2008bloch}
\begin{align}
  T_{LM}
  \equiv \sqrt{\f{2L+1}{2I+1}} \sum_{\mu,\nu}
  \bk{I\mu,LM;I\nu} \op{\nu}{\mu},
  \label{eq:trans_ops}
\end{align}
where $L\in\set{0,1,\cdots,n-1}$ and $M\in\set{-L,-L+1,L}$ index a
total spin and its projection onto a quantization axis;
$I\equiv\p{n-1}/2$ is the maximal angular momentum of an $n$-level
spin; $\mu,\nu\in\set{-I,-I+1,\cdots,I}$ index projections of an
$n$-level nuclear spin onto a quantization axis; and
$\bk{I\mu;I\nu,LM}$ is a Clebsch-Gordan coefficient.  The transition
operator $T_{LM}$ is proportional to the transition induced on a
nuclear spin by the absorption of a spin-$L$ particle with spin
projection $M$ onto a quantization axis.  Similarly to the GGM
operators in \eqref{eq:GGM_diag} and \eqref{eq:GGM_off_diag}, the
transition operators in \eqref{eq:trans_ops} provide a complete basis
for the space of operators on the Hilbert space of an $n$-level
system, and satisfy the orthonormality condition
\begin{align}
  \obk{T_{LM}|T_{L'M'}} = \delta_{LL'} \delta_{MM'},
\end{align}
which implies
\begin{align}
  H_{\t{int}} = -\f{u}{2} \sum_{p,q} {\v T^{(p)}}^\dag \c \v T^{(q)}
  = -\f{u}{2} \v\T^\dag \c \v\T,
  &&
  \v\T \equiv \sum_p\v T^{(p)},
  \label{eq:H_int_trans}
\end{align}
where $\v T$ is a vector of all transition operators in
\eqref{eq:trans_ops}.

%%%%%%%%%%%%%%%%%%%%%%%%%%%%%%%%%%%%%%%%%%%%%%%%%%
\subsection{Drive operators}
\label{sec:drive_ops}

Unlike the GGM operators in \eqref{eq:GGM_diag} and
\eqref{eq:GGM_off_diag}, the transition operators in
\eqref{eq:trans_ops} are not self-adjoint.  In order to express
Hamiltonians in terms of self-adjoint operators, we define the {\it
  drive operators}
\begin{align}
  D_{L,0} \equiv \sqrt{2}~ T_{L,0},
  &&
  D_{LM} \stackrel{M>0}{\equiv} -\p{T_{L\abs{M}} + T_{L\abs{M}}^\dag},
  &&
  D_{LM} \stackrel{M<0}{\equiv} i\p{T_{L\abs{M}} - T_{L\abs{M}}^\dag},
  \label{eq:drive_ops}
\end{align}
for integer $M$ with $\abs{M}\le L$.  The drive operator $D_{LM}$ is
proportional to the Hamiltonian induced on a nuclear spin by an
external classical field of spin-$L$ particles with spin projection
$M$ onto a quantization axis.  Similarly to the GGM operators in
\eqref{eq:GGM_diag} and \eqref{eq:GGM_off_diag}, the drive operators
are equal to the Pauli operators in the case of $n=2$, and satisfy the
orthonormality condition
\begin{align}
  \obk{D_{LM}|D_{L'M'}} = 2\delta_{LL'}\delta_{MM'},
\end{align}
which implies
\begin{align}
  H_{\t{int}} = -\f{u}{4} \sum_{p,q} \v D^{(p)}\c\v D^{(q)}
  = -\f{u}{4} \v\D \c \v\D,
  &&
  \v\D \equiv \sum_p \v D^{(p)},
  \label{eq:H_int_drive}
\end{align}
where $\v D$ is a vector of all drive operators in
\eqref{eq:drive_ops}.

The drive operators $D_{1,0}$ and $D_{1,\pm1}$ are proportional to the
Hamiltonians induced on an $n$-level spin by an external magnetic
field and a classical spin-polarized driving laser; loosely speaking,
one can identify $\p{D_{1,0},D_{1,1},D_{1,-1}}\sim\p{S_\z,S_\x,S_\y}$,
where $S_\alpha$ is a collective SU(2) spin-$\alpha$
operator\cite{perlin2019shorttime}.  More concretely, we can embed the
Hilbert space of a single $n$-level spin into the
permutationally-symmetric (Dicke) manifold of states for $\p{n-1}$
2-level spins via
\begin{align}
  \ket{\mu}_{\t{single}} \to \ket{I+\mu}_{\t{collective}}
  \propto S_+^{I+\mu} \ket\dn^{\otimes\p{n-1}},
\end{align}
where $I\equiv\p{n-1}/2$ is a total spin,
$\mu\in\set{-I,-I+1,\cdots,I}$ indexes a state of the single $n$-level
spin, and $S_+$ is a collective spin-raising operator for $\p{n-1}$
2-level spins.  This embedding takes
\begin{align}
  \p{D_{1,0},D_{1,1},D_{1,-1}}
  \to \p{S_\z,S_\x,S_\y}
  \times \sqrt{\f{3}{I \p{I+1/2} \p{I+1}}}.
\end{align}

We note that Weyl operators\cite{bertlmann2008bloch} and self-adjoint
linear combinations thereof\cite{asadian2016heisenbergweyl} also
provide an interesting basis of operators, generalizing continuous
phase-space displacement operators to the case of a finite-dimensional
Hilbert space.  As our present work is motivated by multi-level
systems that are realized with nuclear spin (angular momentum) degrees
of freedom, however, the transition and drive operators are more
natural to consider than the Weyl operators, so we will not discuss
the latter any further.

%%%%%%%%%%%%%%%%%%%%%%%%%%%%%%%%%%%%%%%%%%%%%%%%%%%%%%%%%%%%%%%%%%%%%%
\section{External drive and spin-orbit coupling}
\label{sec:drive_SOC}

An driving laser addressing spin-$1/2$ fermions on a 1-D lattice will
induce the Hamiltonian
\begin{align}
  \left. H_{\t{drive}}^{(\phi)} \right|_{n=2}
  = \Omega \sum_j
  \p{e^{-i\phi_j} c_{j,\up}^\dag c_{j,\dn}
    + e^{i\phi_j} c_{j,\dn}^\dag c_{j,\up}},
  \label{eq:H_drive_2}
\end{align}
where $\Omega$ is a (real) driving amplitude, $j$ indexes a lattice
site, and $\phi_j$ is a site-dependent phase.  The multi-level
analogue of \eqref{eq:H_drive_2} is
\begin{align}
  H_{\t{drive}}^{(\phi)}
  = -\Omega \sum_j \p{e^{-i\phi_j} T_{1,1}^{(j)}
    + e^{i\phi_j} {T_{1,1}^{(j)}}^\dag}
  = \Omega \D_{1,1}^{(\phi)},
  \label{eq:H_drive}
\end{align}
where
\begin{align}
  \D_{LM}^{(\phi)}
  \equiv \sum_j \sp{\cos\p{M\phi_j} D_{LM}^{(j)}
    + \sin\p{M\phi_j} D_{L,-M}^{(j)}},
  \label{eq:drive_rot}
\end{align}
is an inhomogeneously rotated collective drive operator.  We can make
this drive spatially homogenous via the gauge transformation
$c_{j\mu}^\dag \to e^{i\mu\phi_j} c_{j\mu}^\dag$, after which the
drive becomes simply
\begin{align}
  H_{\t{drive}} = \Omega \D_{1,1}.
\end{align}
The SU($n$) symmetry of collective interactions implies that the
interaction Hamiltonian $H_{\t{int}}$ in \eqref{eq:H_int_drive} is
invariant under this gauge transformation, but the single-particle
Hamiltonian in \eqref{eq:H_lat} transforms as
\begin{align}
  H_{\t{lat}}
  \to H_{\t{lat}}^{(\phi)}
  = -J \sum_{q,\mu} \cos\p{qa-\mu\phi} c_{q\mu}^\dag c_{q\mu},
  \label{eq:H_lat_SOC}
\end{align}
where $\phi$ is the relative phase of the driving laser between
adjacent lattice sites.  In the basis of drive operators, this
Hamiltonian takes the form (see Appendix \ref{sec:lat_drive})
\begin{align}
  H_{\t{lat}}^{(\phi)} = -J \sum_{q,L} B_L\p{q,\phi} D_{L,0}^{(q)},
  \label{eq:H_lat_drive}
\end{align}
with effective driving field strengths
\begin{align}
  B_L\p{q,\phi} = f_L\p{qa} \sqrt{\f{L+1/2}{n}}
  \sum_\mu \bk{I\mu,L,0;I\mu} f_L\p{\mu\phi},
  &&
  f_L\p{\theta} \equiv
  \begin{cases}
    \cos\theta & L~\t{even} \\
    \sin\theta & L~\t{odd}
  \end{cases}.
\end{align}

%%%%%%%%%%%%%%%%%%%%%%%%%%%%%%%%%%%%%%%%%%%%%%%%%%%%%%%%%%%%%%%%%%%%%%
\section{Raman drive}
\label{sec:drive_raman}

Rather than coupling the states of a multi-level system directly with
a single driving laser, as in Section \ref{sec:drive_SOC}, we now
consider coupling through an off-resonant Raman transition to a highly
excited state.  Specifically, we consider a three-laser drive of the
form
\begin{align}
  H_{\t{Raman,bare}}^{(\phi)}
  = \sum_j \sum_{m\in\set{1,0,-1}} \p{\Omega_m e^{-i\phi_{jm}}
    T_{1,m}^{(j)} \otimes \op{\up}{\dn} + \t{h.c.}}
  + \f{\Delta}{2} \1 \otimes \p{\op{\up} - \op{\dn}},
  \label{eq:raman_bare}
\end{align}
where $j$ indexes a lattice site, $\Omega_m$ is a (real) driving
amplitude; $\phi_{jm}$ is a site- and drive-dependent phase;
$\op{r}{s}$ for $r,s\in\set{\up,\dn}$ is a pseudo-spin operator for
the auxiliary degree of freedom off-resonantly addressed by the
lasers; $\Delta$ is a detuning of the lasers from the energy of an
excited state; and $\1$ is the identity operator.  If the detuning is
large, $\Delta\gg\Omega_m$, then at second order in perturbation
theory we can find that the effective action of
$H_{\t{Raman,bare}}^{(\phi)}$ within the $\ket{\dn}$-state manifold is
\begin{align}
  H_{\t{Raman}}^{(\phi)}
  = \sum_{\tau\in\set{0,1,2}} H_{\t{Raman},\tau}^{(\phi)}
\end{align}
with
\begin{align}
  H_{\t{Raman},0}^{(\phi)}
  &\equiv -\sum_j\sum_m \f{\Omega_m^2}{\Delta}
  {T_{1,m}^{(j)}}^\dag T_{1,m}^{(j)}, & \label{eq:raman_0} \\
  H_{\t{Raman},1}^{(\phi)}
  &\equiv -\sum_j\sum_{s\in\set{\pm1}}
  \f{\Omega_0\Omega_s}{\Delta} e^{-i\phi_{js}}
  {T_{1,0}^{(j)}}^\dag T_{1,s}^{(j)} + \t{h.c.},
  \\
  H_{\t{Raman},2}^{(\phi)}
  &\equiv -\f{\Omega_+\Omega_-}{\Delta}
  \sum_j\sum_{s\in\set{\pm1}} e^{-is\p{\phi_{j+}-\phi_{j-}}}
  {T_{1,-s}^{(j)}}^\dag T_{1,s}^{(j)},
  \label{eq:raman_2}
\end{align}
where by choice of gauge we set $\phi_{j,0}=0$, or equivalently
re-define $\phi_{jm}-\phi_{j,0}\to\phi_{jm}$, and we identify the
subscripts $\pm\leftrightarrow\pm1$.  In terms of the drive operators,
we can expand
\begin{align}
  H_{\t{Raman}}^{(\phi)}
  = \sum_j \sum_{\substack{L\le2\\\abs{M}\le L}}
  \Omega_{LM}^{(\phi,j)} D_{LM}^{(j)},
  \label{eq:raman_drive}
\end{align}
where the coefficients $\Omega_{LM}^{(\phi,j)}$ are provided in
Appendix \ref{sec:drive_raman_coeff}.  If the driving lasers are
phase-matched with $\phi_{js}\equiv s\phi_j$, then the drive further
simplifies to
\begin{align}
  H_{\t{Raman}}^{(\phi)}
  = \sum_{\substack{L\le2\\0\le M\le L}} \Omega_{LM} \D_{LM}^{(\phi)}
  \label{eq:drive_raman_matched}
\end{align}
with rotated collective drive operators $\D_{LM}^{(\phi)}$ defined in
\eqref{eq:drive_rot}, and uniform coefficients $\Omega_{LM}$ that have
the scaling
\begin{align}
  \Omega_{0,0} &\propto \Omega_+^2 + \Omega_-^2 + \Omega_0^2,
  &
  \Omega_{1,0} &\propto \Omega_+^2 - \Omega_-^2,
  &
  \Omega_{1,1} &\propto \Omega_0 \p{\Omega_+ + \Omega_-}, \\
  \Omega_{2,0} &\propto \Omega_+^2 + \Omega_-^2 - 2\Omega_0^2,
  &
  \Omega_{2,1} &\propto \Omega_0 \p{\Omega_+ - \Omega_-},
  &
  \Omega_{2,2} &\propto \Omega_+ \Omega_-.
\end{align}
Similarly to the case of a direct drive considered in Section
\ref{sec:drive_SOC}, the Raman drive $H_{\t{Raman}}^{(\phi)}$ in
\eqref{eq:drive_raman_matched} can be made homogenous through the
gauge transformation $c_{j\mu}^\dag \to e^{i\mu\phi_j} c_{j\mu}^\dag$,
under which
\begin{align}
  H_{\t{Raman}}^{(\phi)} \to H_{\t{Raman}}
  \equiv \sum_{\substack{L\le2\\0\le M\le L}} \Omega_{LM} \D_{LM}.
  \label{eq:drive_raman_homo}
\end{align}


%%%%%%%%%%%%%%%%%%%%%%%%%%%%%%%%%%%%%%%%%%%%%%%%%%%%%%%%%%%%%%%%%%%%%%
\section{Future directions}

\begin{itemize}
\item Implications of SOC
  \begin{itemize}
  \item Perturbative regime: $\phi\ll1$
  \item Special angles $\phi$
  \end{itemize}
\item Squeezing
  \begin{itemize}
  \item Collection of $n$-level spins
  \item Generalizations of squeezing beyond SU(2)
  \end{itemize}
\item Consider a different type of drive
\item Consider different interactions
  \begin{itemize}
  \item Include electronic states
  \item Super-exchange regime
  \item Non-uniform (``disordered'') interactions
  \end{itemize}
\end{itemize}



\appendix

%%%%%%%%%%%%%%%%%%%%%%%%%%%%%%%%%%%%%%%%%%%%%%%%%%%%%%%%%%%%%%%%%%%%%%
\section{Changing operator bases for collective SU($n$)-symmetric
  interactions}
\label{sec:changing_bases}

Here we show that the collective interaction Hamiltonian in
\eqref{eq:H_int_spin} takes an identical form in any orthonormal basis
for $\B_n$, i.e.~the space of operators on the Hilbert space of an
$n$-level system, equipped with the trace (Hilbert-Schmidt) inner
product
\begin{align}
  \obk{\O|\Q} \equiv \tr\p{\O^\dag \Q}.
\end{align}
To maintain generality, we choose an arbitrary basis
$X\equiv\set{X_j}$ satisfying the orthonormality condition
\begin{align}
  \obk{X_j|X_k} = \N_X \delta_{jk},
\end{align}
for $j\in\set{0,1,\cdots,n^2-1}$.  This orthonormality condition
implies that we can resolve the identity (super-)operator $\I$ with
$\I\oket{\O}=\oket{\O}$ for any $\oket\O\in\B_n$ as
\begin{align}
  \I = \f1{\N_X} \sum_j \oop{X_j}{X_j}
  = \f1{\N_X} \sum_j \oop{X_j^\dag}{X_j^\dag},
\end{align}
where we used the fact that $\set{X_j^\dag}$ is also an orthonormal
basis for $\B_n$, satisfying the same orthonormality condition as
$\set{X_j}$; these two bases are transformed into each other by
unitary $\sum_j\oop{X_j^\dag}{X_j}/\N_X$.

The collective interaction Hamiltonian in \eqref{eq:H_int_spin} takes
the form
\begin{align}
  H = \sum_{p,q,j} X_j^{(p)} {X_j^{(q)}}^\dag,
\end{align}
which essentially consists of two-body interactions ($p\ne q$) and
self-interactions ($p=q$), respectively of the form
\begin{align}
  \sum_j X_j \otimes X_j^\dag,
  &&
  \sum_j X_j X_j^\dag.
\end{align}
Choosing a different basis $Y\equiv\set{Y_j}$ for $\B_n$ with
corresponding norm $\N_Y$, we can expand
\begin{align}
  X_j = \f1{\N_Y} \sum_k Y_k \obk{Y_k|X_j},
  &&
  X_j^\dag = \f1{\N_Y} \sum_k Y_k^\dag \obk{Y_k^\dag|X_j^\dag}
  = \f1{\N_Y} \sum_k Y_k^\dag \obk{X_j|Y_k},
\end{align}
where we used the fact that
\begin{align}
  \obk{Y_k^\dag|X_j^\dag}
  = \tr\p{Y_k X_j^\dag}
  = \tr\p{X_j^\dag Y_k}
  = \obk{X_j|Y_k}.
\end{align}
By resolution of the identity, we can therefore find
\begin{align}
  \sum_j X_j \otimes X_j^\dag
  = \f1{\N_Y^2} \sum_{j,k,\ell} Y_k \obk{Y_k|X_j}
  \otimes Y_\ell^\dag \obk{X_j|Y_\ell}
  = \f{\N_X}{\N_Y^2} \sum_{k,\ell} Y_k \otimes Y_\ell^\dag
  \obk{Y_k|\I|Y_\ell}
  = \f{\N_X}{\N_Y} \sum_k Y_k \otimes Y_k^\dag,
\end{align}
and similarly
\begin{align}
  \sum_j X_j X_j^\dag
  = \f1{\N_Y^2} \sum_{j,k,\ell} Y_k Y_\ell^\dag
  \obk{Y_k|X_j} \obk{X_j|Y_\ell}
  = \f{\N_X}{\N_Y} \sum_k Y_k Y_k^\dag.
\end{align}
The collective interaction Hamiltonian in \eqref{eq:H_int_spin} thus
takes an identical form in any orthonormal basis for the space $\B_n$
of operators on the Hilbert space of an $n$-level system, up to
appropriate re-scaling factors $\N_X/\N_Y$.

%%%%%%%%%%%%%%%%%%%%%%%%%%%%%%%%%%%%%%%%%%%%%%%%%%%%%%%%%%%%%%%%%%%%%%
\section{Spin-orbit coupling with drive operators}
\label{sec:lat_drive}

Here we take the single-particle spin-orbit coupled Hamiltonian
$H_{\t{lat}}^{(\phi)}$ in \eqref{eq:H_lat_SOC}, and derive its
expression in \eqref{eq:H_lat_drive} in terms of the drive operators
$D_{LM}$.  In the basis of ordinary spin operators $s_{\mu\mu}$, this
Hamiltonian takes the form
\begin{align}
  H_{\t{lat}}^{(\phi)}
  = -J \sum_{q,\mu} \cos\p{qa-\mu\phi} s_{\mu\mu}^{(q)}
  \label{eq:H_lat_spin}
\end{align}
Defining the diagonal ``spin-wave'' operators
\begin{align}
  w_{\phi,+} \equiv \sum_\mu \cos\p{\mu\phi} s_{\mu\mu},
  &&
  w_{\phi,-} \equiv \sum_\mu \sin\p{\mu\phi} s_{\mu\mu},
\end{align}
which are respectively even and odd under spin inversion
($\mu\to-\mu$), we can simplify
\begin{align}
  \sum_\mu \cos\p{qa+\mu\phi} s_{\mu\mu}
  = \cos\p{qa} w_{\phi,+} + \sin\p{qa} w_{\phi,-},
\end{align}
and resolve the identity (super-)operator $\I$ with
$\I w_{\phi,\pm} = w_{\phi,\pm}$ in the basis of the drive
operators $D_{LM}$, finding (see Appendix \ref{sec:changing_bases})
\begin{align}
  \sum_\mu \cos\p{qa+\mu\phi} s_{\mu\mu}
  = \cos\p{qa} \f12 \sum_{L,M} \obk{w_{\phi,+}|D_{LM}} D_{LM}
  + \sin\p{qa} \f12 \sum_{L,M} \obk{w_{\phi,-}|D_{LM}} D_{LM}.
\end{align}
We now note that the drive operators $D_{LM}$ are strictly
off-diagonal for $M\ne0$, and furthermore that $D_{L,0}$ with even
(odd) $L$ is even (odd) under spin inversion, which implies
\begin{align}
  \sum_{L,M} \obk{w_{\phi,+}|D_{LM}} D_{LM}
  = \sum_{L\,\t{even}} \obk{w_{\phi,+}|D_{L,0}} D_{L,0},
\end{align}
and likewise with $w_{\phi,-}$ and odd $L$.  In total, we can
write
\begin{align}
  \sum_\mu \cos\p{qa+\mu\phi} s_{\mu\mu}
  = \cos\p{qa} \f12 \sum_{L\,\t{even}}
  \obk{w_{\phi,+}|D_{L,0}} D_{L,0}
  + \sin\p{qa} \f12 \sum_{L\,\t{odd}}
  \obk{w_{\phi,-}|D_{L,0}} D_{L,0},
\end{align}
or, more compactly,
\begin{align}
  \sum_\mu \cos\p{qa+\mu\phi} s_{\mu\mu}
  = \sum_L B_L\p{q,\phi} D_{L,0},
  \label{eq:compact_SOC_drive}
\end{align}
with effective driving field strengths
\begin{align}
  B_L\p{q,\phi}
  \equiv f_L\p{qa} \f12
  \sum_\mu f_L\p{\mu\phi} \obk{s_{\mu\mu}|D_{L,0}}
  = f_L\p{qa} \sqrt{\f{L+1/2}{n}}
  \sum_\mu \bk{I\mu,L,0;I\mu} f_L\p{\mu\phi},
\end{align}
where
\begin{align}
  f_L\p{\theta} \equiv
  \begin{cases}
    \cos\theta & L~\t{even} \\
    \sin\theta & L~\t{odd}
  \end{cases}.
\end{align}
Substituting the result in \eqref{eq:compact_SOC_drive} into the
single-particle Hamiltonian in \eqref{eq:H_lat_spin} yields
\begin{align}
  H_{\t{lat}}^{(\phi)} = -J \sum_{q,L} B_L\p{q,\phi} D_{L,0}^{(q)}.
\end{align}

%%%%%%%%%%%%%%%%%%%%%%%%%%%%%%%%%%%%%%%%%%%%%%%%%%%%%%%%%%%%%%%%%%%%%%
\section{Transition operator product expansion}
\label{sec:trans_prod}

Here we compute, for an $n$-level spin with total spin
$I\equiv\p{n-1}/2$, the structure constants of the transition operator
algebra, expanding
\begin{align}
  T_{\ell_2 m_2} T_{\ell_1 m_1}
  = \sum_{L,M} T_{LM} \obk{T_{LM} | T_{\ell_2 m_2} T_{\ell_1 m_1}},
\end{align}
where the transition operators are defined by
\begin{align}
  T_{LM} \equiv \sqrt{\f{2L+1}{2I+1}}
  \sum_{\mu,\nu} \bk{I\mu,LM;I\nu} \op{\nu}{\mu},
\end{align}
with Clebsch-Gordan coefficients $\bk{I\mu,LM;I\nu}$, and the inner
product is
\begin{align}
  \obk{T_{LM} | T_{\ell_2 m_2} T_{\ell_1 m_1}}
  = \tr\p{T_{LM}^\dag T_{\ell_2 m_2} T_{\ell_1 m_1}}
  = \p{-1}^M \tr\p{T_{L,-M} T_{\ell_2 m_2} T_{\ell_1 m_1}}.
\end{align}
The Clebsch-Gordan coefficients can be written in terms of Wigner
3-$j$ symbols as
\begin{align}
  \bk{\ell_1 m_1, \ell_2 m_2;\ell_3 m_3}
  = \p{-1}^{-\ell_1+\ell_2-m_3} \sqrt{\p{2\ell_3+1}}
  \begin{pmatrix}
    \ell_1 & \ell_2 & \ell_3 \\
    m_1 & m_2 & -m_3
  \end{pmatrix}.
\end{align}
Substituting transition operators and replacing Clebsch-Gordan
coefficients by Wigner 3-$j$ symbols, we find that
\begin{multline}
  \obk{T_{LM} | T_{\ell_2 m_2} T_{\ell_1 m_1}}
  = \delta_{m_1+m_2,M}
  \p{-1}^M \sqrt{\p{2\ell_1+1}\p{2\ell_2+1}\p{2L+1}} \\
  \times \sum_{\mu,\nu,\rho} \p{-1}^{-3I+\ell_1+\ell_2+L-\mu-\nu-\rho}
  \begin{pmatrix}
    I & \ell_1 & I \\
    \mu & m_1 & -\nu
  \end{pmatrix}
  \begin{pmatrix}
    I & \ell_2 & I \\
    \nu & m_2 & -\rho
  \end{pmatrix}
  \begin{pmatrix}
    I & L & I \\
    \rho & -M & -\mu
  \end{pmatrix},
\end{multline}
where we can evaluate the sum and introduce Wigner 6-$j$ symbols to
get
\begin{multline}
  \obk{T_{LM} | T_{\ell_2 m_2} T_{\ell_1 m_1}} \\
  = \delta_{m_1+m_2,M} \p{-1}^{2I+M}
  \sqrt{\p{2\ell_1+1}\p{2\ell_2+1}\p{2L+1}}
  \begin{pmatrix}
    \ell_1 & \ell_2 & L \\
    m_1 & m_2 & -M
  \end{pmatrix}
  \begin{Bmatrix}
    \ell_1 & \ell_2 & L \\
    J & J & J
  \end{Bmatrix} \\
  = \delta_{m_1+m_2,M} \p{-1}^{2I+\ell_1-\ell_2}
  \sqrt{\p{2\ell_1+1}\p{2\ell_2+1}}
  \bk{\ell_1 m_2, \ell_2 m_2; LM}
  \begin{Bmatrix}
    \ell_1 & \ell_2 & L \\
    J & J & J
  \end{Bmatrix}.
\end{multline}

%%%%%%%%%%%%%%%%%%%%%%%%%%%%%%%%%%%%%%%%%%%%%%%%%%%%%%%%%%%%%%%%%%%%%%
\section{Raman drive coefficients}
\label{sec:drive_raman_coeff}

Here we provide the coefficients $\Omega_{LM}^{(\phi,j)}$ of the
effective Raman driving Hamiltonian
\begin{align}
  H_{\t{Raman}}^{(\phi)}
  = \sum_j \sum_{\substack{L\in\set{0,1,2}\\\abs{M}\le L}}
  \Omega_{LM}^{(\phi,j)} D_{LM}^{(j)}
\end{align}
considered in Section \ref{sec:drive_raman}.  These coefficients are
computed by evaluating products of transition operators in
\eqref{eq:raman_0}--\eqref{eq:raman_2}, using both the structure
constants of the transition operator algebra (derived in Appendix
\ref{sec:trans_prod}) and the fact that
$T_{LM}^\dag = \p{-1}^M T_{L,-M}$.  For clarity, if any coefficient
$\Omega_{LM}^{(\phi,j)}$ is independent of $\phi$, we drop the
explicit dependence on $\phi$, i.e.~taking
$\Omega_{LM}^{(\phi,j)}\to\Omega_{LM}$.
\begin{align}
  \Omega_{0,0} \equiv -\f1{\sqrt{2n}} \sum_m\f{\Omega_m^2}{\Delta},
  &&
  \Omega_{1,0} \equiv \sqrt{\f{3}{2n\p{n^2-1}}}~
  \f{\Omega_+^2 - \Omega_-}{\Delta},
  \label{eq:O_01_0}
\end{align}
\begin{align}
  \Omega_{2,0} \equiv \sqrt{\f{n^2-4}{10n\p{n^2-1}}}~
  \f{\Omega_+ + \Omega_- - 2\Omega_0}{\Delta},
  \label{eq:O_2_0}
\end{align}
\begin{align}
  \Omega_{L,1}^{(\phi,j)}
  \equiv \sum_{s\in\set{\pm1}} \omega_{Ls} \cos\phi_{js},
  &&
  \Omega_{L,-1}^{(\phi,j)}
  \equiv \sum_{s\in\set{\pm1}} s \omega_{Ls} \sin\phi_{js},
\end{align}
\begin{align}
  \Omega_{2,2}^{(\phi,j)}
  \equiv \omega_{2,2} \cos\p{\phi_{j+}-\phi_{j-}},
  &&
  \Omega_{2,-2}^{(\phi,j)}
  \equiv \omega_{2,2} \sin\p{\phi_{j+}-\phi_{j-}},
\end{align}
\begin{align}
  \omega_{1\pm} \equiv \f{\sqrt3}{\sqrt{n\p{n^2-1}}}~
  \f{\Omega_0 \Omega_\pm}{\Delta},
  &&
  \omega_{2\pm} \equiv \pm \f{\sqrt{3/5}}{n\p{n^2-1}}
  \sqrt{\f{\p{n+2}!}{\p{n-3}!}}~ \f{\Omega_0\Omega_\pm}{\Delta},
\end{align}
\begin{align}
  \omega_{2,2} \equiv -\f{\sqrt{6/5}}{n\p{n^2-1}}
  \sqrt{\f{\p{n+2}!}{\p{n-3}!}}~ \f{\Omega_+\Omega_-}{\Delta}.
\end{align}
Here $\Omega_m$, $\phi_{jm}$, and $\Delta$ are respectively the (real)
driving amplitudes, phases, and detuning that appear in the ``bare''
Raman drive in \eqref{eq:raman_bare}, and by choice of gauge we set
$\phi_{j,0}=0$, or equivalently re-define
$\phi_{jm}-\phi_{j,0}\to\phi_{jm}$.  If the driving lasers are
phase-matched with $\phi_{js}\equiv s\phi_j$, then the drive further
simplifies to
\begin{align}
  H_{\t{Raman}}^{(\phi)}
  = \sum_{\substack{L\le2\\0\le M\le L}} \Omega_{LM} \D_{LM}^{(\phi)},
\end{align}
with inhomogeneously rotated collective drive operators
\begin{align}
  \D_{LM}^{(\phi)}
  \equiv \sum_j \sp{\cos\p{M\phi_j} D_{LM}^{(j)}
    + \sin\p{M\phi_j} D_{L,-M}^{(j)}},
\end{align}
and uniform coefficients given by \eqref{eq:O_01_0}, \eqref{eq:O_2_0},
and
\begin{align}
  \Omega_{1,-1} = \Omega_{2,-1} = \Omega_{2,-2} \equiv 0,
  &&
  \Omega_{L,1} \equiv \sum_{s\in\set{\pm1}} \omega_{Ls},
  &&
  \Omega_{2,2} \equiv \omega_{2,2}.
\end{align}



\bibliography{sun_notes.bib}

\end{document}
