\documentclass[pra,reprint,longbibliography]{revtex4-1}

% linking references
\usepackage{hyperref}
\hypersetup{
  breaklinks=true,
  colorlinks=true,
  linkcolor=blue,
  filecolor=magenta,
  urlcolor=cyan,
}

%%% symbols, notations, etc.
\usepackage{physics,braket,bm,commath,amssymb}
\renewcommand{\t}{\text} % text in math mode
\newcommand{\f}[2]{\dfrac{#1}{#2}} % shorthand for fractions
\newcommand{\p}[1]{\left(#1\right)} % parenthesis
\renewcommand{\sp}[1]{\left[#1\right]} % square parenthesis
\renewcommand{\set}[1]{\left\{#1\right\}} % curly parenthesis
\renewcommand{\v}{\bm} % bold vectors
\newcommand{\uv}[1]{\v{\hat{#1}}} % unit vectors
\renewcommand{\c}{\cdot} % inner product
\newcommand{\bk}{\Braket} % shorthand for braket notation

\newcommand{\C}{\mathcal{C}}
\newcommand{\D}{\mathcal{D}}
\newcommand{\F}{\mathcal{F}}
\newcommand{\I}{\mathcal{I}}
\newcommand{\J}{\mathcal{J}}
\renewcommand{\O}{\mathcal{O}}
\renewcommand{\S}{\mathcal{S}}

\newcommand{\N}{\mathbb{N}}

\newcommand{\z}{\text{z}}
\newcommand{\x}{\text{x}}
\newcommand{\y}{\text{y}}
\newcommand{\Z}{\text{Z}}
\newcommand{\X}{\text{X}}
\newcommand{\Y}{\text{Y}}
\newcommand{\bmu}{{\bar\mu}}
\newcommand{\bnu}{{\bar\nu}}

\usepackage{dsfont} % for identity operator
\newcommand{\1}{\mathds{1}}

\newcommand{\up}{\uparrow}
\newcommand{\dn}{\downarrow}

\usepackage[inline]{enumitem} % for inline enumeration

%%% figures
\usepackage{graphicx} % for figures
\usepackage[caption=false]{subfig} % subfigures (via \subfloat[]{})
\newcommand{\sref}[1]{\protect\subref{#1}} % for referencing subfigures
\graphicspath{{./figures/}} % set path for all figures

% for strikeout text
% normalem included to prevent underlining titles in the bibliography
\usepackage[normalem]{ulem}

% for leaving notes in the text
\newcommand{\note}[1]{\textcolor{red}{#1}}


\begin{document}

\title{Short-time simulations of open collective quantum spin systems}

\author{Michael A. Perlin}
\email{mika.perlin@gmail.com}
\author{Ana Maria Rey}
\affiliation{JILA, National Institute of Standards and Technology and
  University of Colorado, 440 UCB, Boulder, Colorado 80309, USA}
\affiliation{Center for Theory of Quantum Matter, 440 UCB, Boulder,
  Colorado 80309, USA}
\affiliation{Department of Physics, University of Colorado, 390 UCB,
  Boulder, Colorado 80309, USA}

\begin{abstract}
  Abstract goes here.
\end{abstract}

\maketitle


\section{Introduction}

Collective spin systems are a versatile resource in quantum science
for a range of applications including quantum-enhanced metrology and
quantum simulation.  Theoretical interest in such systems dates back
to the mid-twentieth century with the introduction of the Dicke
model\cite{dicke1954coherence} that describes atoms cooperatively
interacting with a single mode of a radiation field, and the
Lipkin-Meshkov-Glick (LMG) model, a toy model for testing many-body
approximation methods in contemporary nuclear
physics\cite{lipkin1965validity, meshkov1965validity,
  glick1965validity}.  On the experimental side, the development of
advanced trapping, cooling, and control techniques in atomic,
molecular, and optical (AMO) systems have enabled the realization of
collective spin models in a broad range of platforms, including atomic
ensembles\cite{takano2009spin, appel2009mesoscopic, chen2010heralded,
  schleier-smith2010states, chen2011conditional}, Bose-Einstein
condensates\cite{riedel2010atomchipbased, gross2010nonlinear,
  klinder2015dynamical}, ultracold Fermi
gasses\cite{martin2013quantum, smale2018observation}, trapped
ions\cite{bohnet2016quantum}, optical
cavities\cite{leroux2010implementation, bohnet2014reduced,
  cox2016deterministic, hosten2016measurement, hosten2016quantum,
  zhiqiang2017nonequilibrium, lewis-swan2018robust,
  norcia2018cavitymediated}, and molecular
sensors\cite{jones2009magnetic}.  These implementations have triggered
studies of a variety of rich subjects, including quantum
criticality\cite{latorre2005entanglement, alcalde2007functional,
  morrison2008dissipationdriven, morrison2008collective,
  sarandy2009classical, wang2012quantum, kessler2012dissipative,
  majd2014lmg}, non-equilibrium
phenomena\cite{walls1978nonequilibrium, morrison2008dynamical,
  bhattacherjee2014nonequilibrium, klinder2015dynamical,
  maghrebi2016nonequilibrium, zhiqiang2017nonequilibrium,
  lang2018concurrence, smale2018observation}, and precision
metrology\cite{wineland1992spin, itano1993quantum,
  kitagawa1993squeezed, ma2011quantum, takano2009spin,
  appel2009mesoscopic, chen2010heralded, schleier-smith2010states,
  chen2011conditional, riedel2010atomchipbased, gross2010nonlinear,
  zhong2010simplified, agarwal1997atomic, lau2014proposal,
  huang2015quantum, wineland1992spin, kitagawa1993squeezed,
  schleier-smith2010squeezing, leroux2010implementation,
  bohnet2014reduced, cox2016deterministic, hosten2016measurement,
  hosten2016quantum, lewis-swan2018robust, norcia2018cavitymediated,
  jones2009magnetic, he2019engineering}.

One of the primary motivations for studying collective spin systems is
their application to quantum-enhanced metrology.  Quantum projection
noise limits the error $\Delta\phi$ in the measurement of a phase
angle $\phi$ with $N$ independent spins to
$\Delta\phi\sim1/\sqrt{N}$\cite{wineland1992spin, itano1993quantum,
  ma2011quantum}.  Collective spin systems enable a means to break
through this limit via the preparation of many-body entangled states
such as NOON\cite{jones2009magnetic, chen2010heralded,
  zhong2010simplified}, spin-cat\cite{agarwal1997atomic,
  lau2014proposal, huang2015quantum}, and spin-squeezed
states\cite{wineland1992spin, kitagawa1993squeezed, takano2009spin,
  appel2009mesoscopic, schleier-smith2010states,
  schleier-smith2010squeezing, leroux2010implementation,
  riedel2010atomchipbased, gross2010nonlinear, chen2011conditional,
  bohnet2014reduced, cox2016deterministic, hosten2016measurement,
  hosten2016quantum, lewis-swan2018robust, norcia2018cavitymediated,
  ma2011quantum, he2019engineering} that allow for measurement errors
$\Delta\phi\sim1/N^\varepsilon$ with $1/2<\varepsilon\le1$, where
$\varepsilon=1$ saturates the Heisenberg
limit\cite{zwierz2010general}.  Generating such states typically
requires nonlinear dynamics that are realized using collisional,
photon-mediated, or phonon-mediated interactions.  Although a truly
collective spin model requires uniform, all-to-all interactions, as
long as measurements do not distinguish between constituent particles,
even non-uniform systems can be effectively described by a uniform
model with renormalized parameters\cite{hu2015entangled}.

In the absence of decoherence, permutation symmetry and total spin
conservation divide the total Hilbert space of a collective spin
system into superseletion sectors that grow only linearly with system
size $N$, thereby admitting efficient classical simulation of its
dynamics.  Decoherence will generally violate total spin conservation
and require the use of density operators, increasing the dimension of
accessible state space to $\sim N^3$\cite{hartmann2016generalized,
  xu2013simulating} and allowing for exact simulations of
$N\lesssim100$ particles.  If decoherence is sufficiently weak,
dynamics can be numerically solvable for $N\lesssim10^5$ particles via
``quantum trajectory'' Monte Carlo methods\cite{plenio1998quantumjump,
  zhang2018montecarlo} (also known as ``quantum jump'' or ``Monte
Carlo wavefunction'' methods) that can reproduce all expectation
values of interest.  When decoherence is strong, however, or when the
number of jump operators grows extensively with system size (e.g.~for
single-spin decay), these Monte Carlo methods can take a prohibitively
long time to converge, as simulations become dominated by incoherent
jumps that generate large numbers of distinct quantum trajectories
that need to be averaged over to compute expectation values.

In this work, we present a new method to simulate short-time dynamics
of collective spin systems with permutationally-symmetric sources of
decoherence.  This method is based on a short-time expansion of exact
solutions to the equations of motion for Heisenberg operators.
Evaluating this expansion requires using the structure constants of a
collective spin operator algebra; the calculation of these structure
constants (in Appendices \ref{sec:identities}--\ref{sec:prod_general})
is one of the main technical results of this work, which we hope will
empower both analytical and numerical studies of collective spin
systems in the future.  We benchmark our method against both exact and
quantum trajectory Monte Carlo computations of spin squeezing in
accessible parameter regimes, highlighting both advantages and
limitations of our short-time expansion.  Finally, we showcase an
application of our method with computations that are inaccessible to
other numerical methods.

[More text about the content of this paper.]


\section{Theory}

In this section we provide the basic theory for our method to compute
expectation values of collective spin operators, deferring lengthy
derivations to the appendices.  We consider systems composed of $N$
distinct spin-1/2 particles.  Defining individual spin-1/2 operators
$s_{n=\x,\y,\z}\equiv\sigma_n/2$ and
$s_\pm\equiv s_\x\pm is_\y=\sigma_\pm$ with Pauli operators
$\sigma_n$, we denote an operator that acts with $s_n$ on the spin
indexed by $j$ and trivially (i.e.~with the identity $\1$) on all
other spins by $s_n^{(j)}$.  We then define the collective spin
operators $S_n\equiv\sum_{j=1}^Ns_n^{(j)}$ for
$n\in\set{\x,\y,\z,+,-}$.  Identifying the set $\set{\S_{\v m}}$ as a
basis for all collective spin operators, with
$\v m\equiv\p{m_+,m_\z,m_-}\in\N_0^3$ and
$\S_{\v m}\equiv S_+^{m_+} S_\z^{m_\z} S_-^{m_-}$, we can expand all
collective spin operators in the form
\begin{align}
  \O = \sum_{\v m} \O_{\v m} \S_{\v m}.
\end{align}
The time evolution of a Heisenberg operator $\O\p{t}$ is then
determined by
\begin{align}
  \f{d}{dt} \O\p{t}
  = T \O\p{t}
  = \sum_{\v m, \v n} \S_{\v m} T_{\v m\v n} \O_{\v n}\p{t}
  \label{eq:EOM}
\end{align}
for a time derivative (super-)operator $T$ with matrix elements
$T_{\v m\v n}$ defined by
\begin{align}
  T \S_{\v n} \equiv i \sp{H, \S_{\v n}}_-
  + \sum_\J \gamma_\J \D\p{\J} \S_{\v n}
  = \sum_{\v m} \S_{\v m} T_{\v m\v n},
  \label{eq:time_deriv}
\end{align}
where $\sp{X,Y}_\pm\equiv XY\pm YX$; $H$ is the collective-spin
Hamiltonian; $\J$ is a set of jump operators with a corresponding
decoherence rate $\gamma_\J$; and is $\D\p{\J}$ is a
Heisenberg-picture dissipator, or Lindblad superoperator, defined by
\begin{align}
  \D\p{\J} \O
  \equiv \sum_{J\in\J} \p{J^\dag \O J - \f12\sp{J^\dag J,\O}_+}.
\end{align}
Decoherence via uncorrelated decay of individual spins at a rate
$\gamma_-$, for example, would be described by the set of jump
operators $\J_-=\set{s_-^{(j)}:j=1,2,\cdots,N}$ with
$\gamma_{\J_-}=\gamma_-$.  The commutator in Eq.~\eqref{eq:time_deriv}
can be computed by expanding the product
$\S_{\v\ell}\S_{\v m}=\sum_{\v n}f_{\v\ell\v m\v n}\S_{\v n}$ with
structure constants $f_{\v\ell\v m\v n}$ that are worked out in
Appendices \ref{sec:identities}--\ref{sec:prod_general}, and the
effects of decoherence from jump operators (i.e.~elements of $\J$) of
the form $\gamma^{(j)}=\sum_n\gamma_ns_n^{(j)}$ and
$\Gamma=\sum_n\Gamma_nS_n$ are worked out in Appendices
\ref{sec:sandwich_single}--\ref{sec:decoherence_collective}.  We
consider these calculations to be some of the main technical
contributions of this work, with potential applications beyond the
short-time simulation method presented here.  These ingredients are
sufficient to compute matrix elements $T_{\v m\v n}$ of the time
derivative operator $T$ in Eq.~\eqref{eq:time_deriv} in most cases of
practical interest.

The time derivative operator $T$ will generally couple spin operators
$\S_{\v n}$ to spin operators $\S_{\v m}$ with higher ``weight'',
i.e.~with $\abs{\v m}>\abs{\v n}$, where
$\abs{\v\ell}\equiv\sum_\alpha\ell_\alpha$.  The growth of operator
weight signifies the growth of many-body correlations.  In practice,
keeping track of this growth will eventually require more
computational resources than are available, meaning we must somehow
truncate our equations of motion.  The simplest truncation strategy
would be to take
\begin{align}
  \f{d}{dt} \O\p{t}
  \to \sum_{w\p{\v m}<W} \S_{\v m}
  \sum_{\v n} T_{\v m\v n} \O_{\v n}\p{t}
  \label{eq:weight_truncation}
\end{align}
for some weight measure $w$, e.g.~$w\p{\v m}=\abs{\v m}$, and a
high-weight cutoff $W$.  The truncation in
Eq.~\eqref{eq:weight_truncation} closes the system of differential
equations defined by Eq.~\eqref{eq:EOM}, and allows us to solve it
using standard numerical methods.  Some initial conditions for this
system of differential equations, namely expectation values of
collective spin operators with respect to spin-polarized (Gaussian)
states that are generally simple to prepare experimentally, are
provided in Appendix \ref{sec:initial_conditions}.

The truncation strategy in Eq.~\eqref{eq:weight_truncation} has a few
limitations:
\begin{enumerate*}[label=(\roman*)]
\item simulating a system of differential equations for a large number
  of operators can be time-consuming,
\item the weight measure $w$ may need to be chosen carefully, as the
  optimal measure is generally system-dependent, and
\item simulation results can only be trusted up to the time at which
  the initial value of operators $\S_{\v m}$ with weight
  $w\p{\v m}\ge W$ have a non-negligible contribution to expectation
  values of interest.\label{pt:limitation}
\end{enumerate*}
The last limitation in particular unavoidably applies in some form to
any method tracking only a subset of all relevant operators.  We
therefore devise an alternate truncation strategy built around
limitation \ref{pt:limitation}.

We can expand Heisenberg operators $\O\p{t}$ in a Taylor series about
the time $t=0$ to write
\begin{align}
  \O\p{t}
  = e^{t T} \O\p{0}
  = \sum_{k\ge0} \f{t^k}{k!}
  \sum_{\v m, \v n} \S_{\v m} T^k_{\v m\v n} \O_{\v n}\p{0},
  \label{eq:time_series}
\end{align}
where the matrix elements $T^k_{\v m\v n}$ of the $k$-th time
derivative operator $T^k$ are
\begin{align}
  T^0_{\v m\v n} &\equiv \delta_{\v m\v n}, \\
  T^1_{\v m\v n} &\equiv T_{\v m\v n}, \\
  T^{k>1}_{\v m\v n}
  &\equiv \sum_{\v p_1,\v p_2,\cdots,\v p_{k-1}}
  T_{\v m\v p_{k-1}} \cdots T_{\v p_3\v p_2}
  T_{\v p_2\v p_1} T_{\v p_1\v n},
\end{align}
with $\delta_{\v m\v n}=1$ if $\v m=\v n$ and 0 otherwise.  For
sufficiently short times, we can truncate the series in
Eq.~\eqref{eq:time_series} by taking
\begin{align}
  \O\p{t}
  \to \sum_{k=0}^M \f{t^k}{k!}
  \sum_{\v m, \v n} \S_{\v m} T^k_{\v m\v n} \O_{\v n}\p{0}.
  \label{eq:order_truncation}
\end{align}
We refer Eq.~\eqref{eq:order_truncation} as the truncated short-time
(TST) expansion of Heisenberg operators.  Unlike the truncation in
Eq.~\eqref{eq:weight_truncation}, the nonzero matrix elements
$T^k_{\v m\v n}$ for $k=0,1,\cdots,M$ in
Eq.~\eqref{eq:order_truncation} tell us which operators are relevant
for computing the expectation value $\bk{\O\p{t}}$ to a fixed order
$M$.  Note that using the relation $\S_{\v m}^\dag=\S_{\v m^*}$ for
$\v m^*\equiv\p{m_-,m_\z,m_+}$, which by Hermitian conjugation of
Eq.~\eqref{eq:EOM} also implies that
$T_{\v m^*\v n^*}=T_{\v m\v n}^*$, can cut both the number of
initial-time expectation values $\bk{\S_{\v m}}_{t=0}$ and the number
qof matrix elements $T_{\v m\v n}$ that we may need to explicitly
compute roughly in half.

In principle, the factorial suppression of terms at higher orders of
the expansion in Eq.~\eqref{eq:time_series} implies for any given time
$t$, there exists a truncation order $M_t$ for which the truncation
error in Eq.~\eqref{eq:order_truncation} is negligibly small.  In
practice, only a maximal truncation order $M_{\t{max}}$ is accessible
with limited computational resources, such that the TST expansion in
Eq.~\eqref{eq:order_truncation} only allows computation of the
expectation value $\bk{\O\p{t}}$ to a maximal time $t_{M_{\t{max}}}$.
As we will see in the following section, $M_{\t{max}}=35$ will suffice
for the computation of collective spin correlators up to times that
are relevant e.g.~for spin squeezing protocols.


\section{Spin squeezing, benchmarking, and breakdown}
\label{sec:squeezing}

To benchmark our method for computing collective spin correlators, we
consider three collective spin models known to generate spin-squeezed
states: the one-axis twisting (OAT), two-axis twisting (TAT), and
twist-and-turn (TNT) models described by the collective spin
Hamiltonians\cite{kitagawa1993squeezed, micheli2003manyparticle}
\begin{align}
  H_{\t{OAT}} &= \chi S_\z^2, \label{eq:OAT} \\
  H_{\t{TAT}} &= \f{\chi}{3} \p{S_\z^2 - S_\y^2}, \label{eq:TAT} \\
  H_{\t{TNT}} &= \chi S_\z^2 + \Omega S_\x, \label{eq:TNT}
\end{align}
where we include a factor of $1/3$ in the TAT Hamiltonian because it
naturally appears in realistic proposals to experimentally implement
TAT\cite{liu2011spin, huang2015twoaxis}.  For simplicity, we further
fix $\Omega=\chi S$ (with $S\equiv N/2$ throughout this work) to the
critical value known to maximize the entanglement generation rate of
TNT in the large-$N$ limit\cite{micheli2003manyparticle,
  sorelli2015fast}.

Note that the OAT model is a special case of the zero-field Ising
model, whose quantum dynamics admits an exact analytic solution even
in the presence of decoherence\cite{foss-feig2013nonequilibrium}.  We
will benchmark our calculations using these analytical OAT results
wherever applicable (see Appendix \ref{sec:OAT}, as well as the
Supplementary Material of Ref.~[\citenum{bohnet2016quantum}]).

The Hamiltonians in Eqs.~\eqref{eq:OAT}--\eqref{eq:TNT} squeeze the
initial product state $\ket{\X}\propto\p{\ket\up+\ket\dn}^{\otimes N}$
with $S_\x\ket{\X}=S\ket{\X}$.  Our measure of spin squeezing is the
directionally-unbiased Ramsey squeezing parameter determined by the
maximal gain in resolution $\Delta\phi$ of a phase angle $\phi$ over
that achieved by any spin-polarized product state
(e.g.~$\ket{\X}$)\cite{wineland1992spin, ma2011quantum},
\begin{align}
  \xi^2
  \equiv \f{\p{\Delta\phi_{\t{min}}}^2}{\p{\Delta\phi_{\t{polarized}}}^2}
  = \f{N}{\abs{\bk{\v S}}^2}
  \min_{\uv n\perp\bk{\v S}} \bk{\p{\v S\c\uv n}^2},
  \label{eq:squeezing}
\end{align}
where $\v S\equiv\p{S_\x,S_\y,S_\z}$ is a collective spin vector, and
the minimization is performed over all unit vectors $\uv n$ in the
plane orthogonal to the mean spin vector $\bk{\v S}$.  This squeezing
parameter is entirely determined by one- and two-spin correlators of
the form $\bk{S_m}$ and $\bk{S_m S_n}$.  In the case of the coherent
dynamics discussed here, these correlators are obtainable via exact
simulations of quantum dynamics in the $\p{N+1}$-dimensional Dicke
manifold of states $\set{\ket{S,m}}$ with net spin $S$ and spin
projection $m$ onto the $z$ axis, i.e.~with
$\bk{S,m|\v S^2|S,m}=S\p{S+1}$ and $\bk{S,m|S_\z|S,m}=m$ for
$m\in\set{-S,-S+1,\cdots,S}$.  In the presence of
permutationally-symmetric decoherence, these correlators are
obtainable with the collective-spin quantum trajectory Monte Carlo
method developed in ref.~\cite{zhang2018montecarlo}.  In this work,
these exact and quantum trajectory simulations will be used to
benchmark the TST expansion in Eq.~\eqref{eq:order_truncation}.

\begin{figure}
  \centering
  \subfloat[Squeezing under coherent dynamics
  \label{fig:squeezing_coherent}]
  {\includegraphics{coherent.pdf}} \\
  \subfloat[Squeezing with decoherence:
  $\gamma_-=\gamma_+=\gamma_\z=\chi$
  \label{fig:squeezing_incoherent}]
  {\includegraphics{decoherence_weak.pdf}}
  \caption{Spin squeezing of $N=10^4$ spins initially in $\ket\X$
    under \sref{fig:squeezing_coherent} coherent and
    \sref{fig:squeezing_incoherent} incoherent evolution, computed
    using exact methods (solid lines), quantum trajectory simulations
    (dots), and the TST expansion in Eq.~\eqref{eq:order_truncation}
    with $M=35$ (darker dashed lines).  Solid circles mark the times
    at which the TST expansion gives an unphysical result with
    $\xi^2<0$.}
  \label{fig:benchmarking}
\end{figure}

Figure \ref{fig:benchmarking} compares the squeezing parameter $\xi^2$
for $N=10^4$ spins initially in the state $\ket{\X}$ evolved under the
Hamiltonians in Eqs.~\eqref{eq:OAT}--\eqref{eq:TNT}, as computed via
both benchmarking simulations and the TST expansion in
Eq.~\eqref{eq:order_truncation} with $M=35$.  Squeezing is shown for
both coherent dynamics (Figure \ref{fig:squeezing_coherent}), as well
as dynamics in the presence of spontaneous decay, excitation, and
dephasing of individual spins at rates $\chi$ (Figure
\ref{fig:squeezing_incoherent}), respectively described by the sets of
jump operators $\J_n=\set{s_n^{(j)}}$ with corresponding decoherence
rates $\gamma_n=\chi$ for $n\in\set{-,+,\z}$.  The results shown in
Figure \ref{fig:benchmarking} were computed in a rotated basis with
$\p{s_\z,s_\x}\to\p{s_\x,-s_\z}$ and
$\ket\X\to\ket{-\Z}\equiv\ket\dn^{\otimes N}$ (together with
appropriate transformations of the Hamiltonian and jump operators);
the only effect of this rotation on the results presented in Figure
\ref{fig:benchmarking} is to prolong the time for which the TST
expansion agrees with the benchmarking simulations.  The reason for
different results in rotated basis has to do with the breakdown of the
TST expansion, which we discuss further below.

The main lesson from Figure \ref{fig:benchmarking} is that the TST
expansion gives essentially exact results right up until a sudden and
drastic departure that is simple to diagnose by inspection.  This
drastic departure occurs because the TST expansion neglects
high-weight (i.e.~large-$\abs{\v m}$) operators $\S_{\v m}$ whose
contributions to a Heisenberg operator $\O\p{t}$ of interest
eventually become non-negligible.  For sufficiently large times $t$,
these individual contributions may generally be large compared to the
value of $\bk{\O\p{t}}$, in which case their truncation will yield
non-physical results.  This breakdown mechanism is also the reason for
prolonged agreement between the TST expansion and benchmarking methods
when squeezing the initial state $\ket{-\Z}$ rather than $\ket{\X}$:
for initial states $\ket{-\Z}$, all initial-time correlators
$\bk{\S_{\v m}}_{t=0}$ vanish unless $m_+=m_-=0$ (see Appendix
\ref{sec:initial_conditions}), so there is a substantially smaller
number of neglected terms with non-zero contribution to the correlator
$\bk{\O\p{t}}$.

Although the TST expansion breaks down at short times, it has two key
advantages over other methods to compute collective spin correlators.
First, computing spin correlators with the TST expansion is generally
much faster and requires much fewer computing resources than the
alternatives.  The quantum trajectory Monte Carlo simulations
performed for Figure \ref{fig:squeezing_incoherent}, for example, take
$\sim10^4$ CPU hours to compute on standard modern computing hardware;
the bulk of this time is spent performing sparse matrix
multiplication, leaving little room to further optimize run time.
Parallelization can reduce the actual run time of these quantum
trajectory simulations to $\sim10$ hours, but at the cost of greatly
increasing computing resource requirements.  The TST expansion results
in Figure \ref{fig:squeezing_incoherent}, meanwhile, take $\sim10$
seconds to compute with a single CPU on similar computing hardware.

Second, the TST expansion enables computing spin correlators in
strong-decoherence regimes entirely inaccessible to other methods.  As
an example, Figure \ref{fig:decoherence_strong} shows squeezing of
$N=10^4$ spins initially in $\ket{\X}$, undergoing spontaneous decay,
excitation, and dephasing of individual spins at rates
$\gamma_-=\gamma_+=\gamma_\z=100\chi$.  These results show that TNT is
much more robust to this kind of decoherence than OAT or TAT, a
finding that could not be deduced from the weak-decoherence
simulations presented in Figure \ref{fig:benchmarking}.
Strong-decoherence computations of this sort were used to put lower
bounds on theoretically achievable spin squeezing via TAT with
decoherence in Ref.~[\citenum{he2019engineering}], exemplifying a
concrete application of the TST expansion and the collective-spin
structure constants calculated in this work.

\begin{figure}
  \centering
  \includegraphics{decoherence_strong.pdf}
  \caption{Spin squeezing of $N=10^4$ spins initially in $\ket\X$
    under incoherent evolution spontaneous with decay, excitation, and
    dephasing of individual spins at rates
    $\gamma_-=\gamma_+=\gamma_\z=100\chi$, computed using the TST
    expansion in Eq.~\eqref{eq:order_truncation} with $M=35$.  Solid
    circles mark the times at which the TST expansion gives an
    unphysical result with $\xi^2<0$.}
  \label{fig:decoherence_strong}
\end{figure}


\section{Two-time correlation functions}
\label{sec:two_time}

As a final example of collective-spin physics that is numerically
accessible via the TST expansion of Heisenberg operators, we consider
the calculation of collective-spin two-time correlation functions.  In
particular, we consider the effect of decoherence on the short-time
behavior of the two-time connected correlator
\begin{align}
  C\p{t}
  \equiv \f1S\sp{\bk{S_+\p{t}S_-\p{0}}-\bk{S_+\p{t}}\bk{S_-\p{0}}}
  \label{eq:two_time}
\end{align}
in the context of the squeezing models in Section \ref{sec:squeezing}.
In an equilibrium setting, correlation functions of this sort contain
information about the linear response of Heisenberg operators to
perturbations of a system[CITE]\cite{CITE}; in a non-equilibrium
setting, they contribute to short-time linear response (see Appendix
\ref{sec:linear_response}).  Similar correlators have made appearances
as order parameters that diagnose time-crystalline phases of
matter\cite{tucker2018shattered}, and related multi-time correlators,
known as out-of-time-ordered correlators, are commonly examined for
signatures of quantum chaos and information
scrambling[CITE]\cite{CITE}.

\begin{figure*}
  \centering
  \includegraphics{Sp_Sm.pdf}
  \caption{Real (solid lines) and imaginary (dashed lines) parts of
    the two-time connected correlator $C\p{t}$ defined in
    Eq.~\eqref{eq:two_time} for $N=10^4$ spins initially in the
    polarized state $\ket{\X}\propto\p{\ket\up+\ket\dn}^{\otimes N}$
    evolving under the squeezing Hamiltonians in
    Eqs.~\eqref{eq:OAT}--\eqref{eq:TNT}.  Results are shown for time
    evolution without decoherence (leftmost panel), as well as with
    decoherence at rates $\gamma_n$ or $\Gamma_n$, respectively
    corresponding to single-spin or collective jump operators
    $s_n^{(j)}$ or $S_n$.  All axes in the sub-figures are shared.  At
    short times, the correlations captured by $C\p{t}$ are suppressed
    by single-spin decay ($\gamma_-$), collective decay ($\Gamma_-$),
    and single-spin dephasing ($\gamma_\z$); these correlations are
    relatively insensitive to collective dephasing ($\Gamma_\z$), and
    can change sign or grow from spontaneous single-spin ($\gamma_+$)
    or collective ($\Gamma_+$) excitation.}
  \label{fig:two_time}
\end{figure*}

Figure \ref{fig:two_time} shows the behavior of $C\p{t}$ for $N=10^4$
spins, initially in the state
$\ket\X\propto\p{\ket\up+\ket\dn}^{\otimes N}$, evolving under the
squeezing Hamiltonians in Eqs.~\eqref{eq:OAT}--\eqref{eq:TNT} both
with and without various sources of decoherence.  The two-time
correlator $C\p{t}$ degrades at short times under certain types of
decoherence, but seems relatively insensitive to collective dephasing
(with jump operator $S_\z$), and can qualitatively change behavior or
grow in the presence of single-spin or collective excitation (with
jump operators $s_+^{(j)}$ or $S_+$).  Figure \ref{fig:two_time}
serves as an example for the type of quantity and behavior that is
accessible at short times with the TST expansion.  This example is
straightforward to extend to multi-time and out-of-time-ordered
correlators (OTOCs), as well as equilibrium settings.


\section{Conclusions}

We have presented a new method for computing correlators in collective
spin systems at short times.  This method is based on truncating a
short-time expansion of Heisenberg operators, and can access
correlators on time scales that are relevant to metrological
applications such as spin squeezing.  In order to evaluate the
truncated short-time (TST) expansion of Heisenberg operators, we have
computed the structure constants of a collective spin operator
algebra, which we hope will empower future analytical and numerical
studies of collective spin systems.  Even though we considered only
non-equilibrium spin-squeezing processes in this work, our method can
be applied directly in an equilibrium setting, and is straightforward
to generalize to systems such as trapped ions and optical cavities
with collective spin-boson interactions.  In such contexts, our method
may be used to benchmark the short-time effects of decoherence, or
compute quantities such as out-of-time-ordered correlators (OTOCs)
that diagnose the onset of quantum chaos and information scrambling.


\section{Acknowledgments}

We acknowledge helpful discussions with Robert Lewis-Swan and Kris
Tucker.  This work is supported by the Air Force Office of Scientific
Research (AFOSR) grant FA9550-18-1-0319; the AFOSR Multidisciplinary
University Research Initiative (MURI) grant; the Defense Advanced
Research Projects Agency (DARPA) and Army Research Office (ARO) grant
W911NF-16-1-0576; the National Science Foundation (NSF) grant
PHY-1820885; JILA-NSF grant PFC-173400; and the National Institute of
Standards and Technology (NIST).


\newpage
\onecolumngrid
\appendix

\section{Analytical results for the one-axis twisting model}
\label{sec:OAT}

The one-axis twisting (OAT) Hamiltonian for $N$ spin-1/2 particles
takes the form
\begin{align}
  H_{\t{OAT}}
  = \chi S_\z^2
  = \f12 \chi \sum_{j<k} \sigma_\z^{(j)} \sigma_\z^{(k)} + \f14 N \chi,
\end{align}
where $\sigma_\z^{(j)}$ represents a Pauli-$z$ operator acting on spin
$j$.  This model is a special case of the zero-field Ising Hamiltonian
previously solved in Ref.~[\citenum{foss-feig2013nonequilibrium}] via
exact, analytical treatment of the quantum trajectory Monte Carlo
method for computing expectation values.  The solution therein
accounts for coherent evolution in addition to decoherence via
uncorrelated single-spin decay, excitation, and dephasing respectively
at rates $\gamma_-$, $\gamma_+$, and $\gamma_\z$ (denoted by
$\Gamma_{\t{ud}}$, $\Gamma_{\t{du}}$, and $\Gamma_{\t{el}}$ in
Ref.~[\citenum{foss-feig2013nonequilibrium}]).  Letting $S\equiv N/2$
and $\mu,\nu\in\set{+1,-1}$, we adapt expectation values computed in
Ref.~[\citenum{foss-feig2013nonequilibrium}] for the initial state
$\ket\X\propto\p{\ket\up+\ket\dn}^{\otimes N}$ with
$S_\x\ket\X=S\ket\X$ evolving under $H_{\t{OAT}}$, finding
\begin{align}
  \bk{S_+}
  &= S e^{-\Gamma t} \Phi\p{\chi,t}^{N-1}, \label{eq:S+_OAT} \\
  \bk{S_\mu S_\z}
  &= -\f{\mu}{2}\bk{S_\mu} + S \p{S-\f12} e^{-\Gamma t}
  \Psi\p{\mu\chi,t} \Phi\p{\chi,t}^{N-2}, \\
  \bk{S_\mu S_\nu}
  &= \delta_{\mu,-\nu} \p{S + \mu\bk{S_\z}}
  + S \p{S-\f12} e^{-2\Gamma t}
  \Phi\p{\sp{\mu+\nu}\chi,t}^{N-2}, \label{eq:SS+-_OAT}
\end{align}
where
\begin{align}
  \Phi\p{X,t}
  &\equiv e^{-\lambda t} \sp{\cos\p{t\sqrt{q_X^2-r}}
    + \lambda t~\t{sinc}\p{t\sqrt{q_X^2-r}}},
  \\
  \Psi\p{X,t}
  &\equiv e^{-\lambda t} \p{iq_X-\gamma}t~
  \t{sinc}\p{t\sqrt{q_X^2-r}},
\end{align}
for
\begin{align}
  \gamma \equiv -\f12 \p{\gamma_+ - \gamma_-},
  &&
  \lambda \equiv \f12 \p{\gamma_+ + \gamma_-},
  &&
  \Gamma \equiv \lambda + \f12\gamma_\z,
  &&
  r \equiv \gamma_+ \gamma_-,
  &&
  q_X \equiv X + i\gamma,
\end{align}
In order to compute spin squeezing as measured by the Ramsey squeezing
parameter $\xi^2$ defined in \eqref{eq:squeezing}, we additionally
need analytical expressions for $\bk{S_\z}$ and $\bk{S_\z^2}$.  As
these operators commute with both the OAT Hamiltonian and the
single-spin operators $\sigma_\z^{(j)}$, their evolution is governed
entirely by decay-type decoherence (see Appendix
\ref{sec:decay_single}), which means
\begin{align}
  \f{d}{dt} S_\z
  &= S\p{\gamma_+-\gamma_-} - \p{\gamma_++\gamma_-} S_\z,
  \\
  \f{d}{dt}\p{S_\z^2}
  &= S\p{\gamma_++\gamma_-} + 2\p{\gamma_+-\gamma_-}\p{S-\f12} S_\z
  - 2 \p{\gamma_++\gamma_-} S_\z^2.
\end{align}
The initial conditions $\bk{S_\z}_{t=0}=0$ and $\bk{S_\z^2}_{t=0}=S/2$
then imply
\begin{align}
  \bk{S_\z}
  &= S\p{\f{\gamma_+-\gamma_-}{\gamma_++\gamma_-}}
  \p{1-e^{-\p{\gamma_+ + \gamma_-} t}}, \\
  \bk{S_\z^2} &= S \sp{\f12 + \p{S-\f12} \f{\bk{S_\z}^2}{S^2}}.
  \label{eq:Sz_OAT}
\end{align}
The expectation values in \eqref{eq:S+_OAT}--\eqref{eq:SS+-_OAT} and
\eqref{eq:Sz_OAT} are sufficient to compute the spin squeezing
parameter $\xi^2$ in \eqref{eq:squeezing} at any time throughout
evolution of the initial state $\ket\X$ under $H_{\t{OAT}}$, under
appropriate assumptions about the relevant sources of decoherence.


\section{Basic spin operator identities}
\label{sec:identities}

The appendices in this work make ubiquitous use of various spin
operator identities; we collect and derive some basic identities here
for reference.  Note that despite the working definition of collective
spin operators from $S_n=\sum_js_n^{(j)}$, the identities we will
derive involving only collective spin operators apply just as well to
large-spin operators that cannot be expressed as the sum of individual
spin-1/2 operators.  The elementary commutation relations between spin
operators are, with $\bmu\equiv-\mu\in\set{+1,-1}$ for brevity,
\begin{align}
  \sp{s_\z^{(j)},s_\mu^{(k)}}_-
  &= \delta_{jk} \mu s_\mu^{(j)},
  &
  \sp{S_\z,s_\mu^{(j)}}_-
  &= \sp{s_\z^{(j)},S_\mu}_- = \mu s_\mu^{(j)},
  &
  \sp{S_\z,S_\mu}_-
  &= \mu S_\mu,
  \label{eq:comm_z_base} \\
  \sp{s_\mu^{(j)},s_\bmu^{(k)}}_-
  &= \delta_{jk} 2 \mu s_\z^{(j)},
  &
  \sp{S_\mu,s_\bmu^{(j)}}_-
  &= \sp{s_\mu^{(j)},S_\bmu}_- = 2 \mu s_\z^{(j)},
  &
  \sp{S_\mu,S_\bmu}_-
  &= 2 \mu S_\z.
  \label{eq:comm_mu_base}
\end{align}
These relations can be used to inductively compute identities
involving powers of collective spin operators.  By pushing through one
spin operator at a time, we can find
\begin{align}
  \p{\mu S_\z}^m s_\mu^{(j)}
  = \p{\mu S_\z}^{m-1} s_\mu^{(j)} \p{1 + \mu S_\z}
  = \p{\mu S_\z}^{m-2} s_\mu^{(j)} \p{1 + \mu S_\z}^2
  = \cdots
  = s_\mu^{(j)} \p{1 + \mu S_\z}^m,
  \label{eq:push_z_mu_Ss}
\end{align}
and
\begin{align}
  \mu s_\z^{(j)} S_\mu^m
  = S_\mu \mu s_\z^{(j)} S_\mu^{m-1} + s_\mu^{(j)} S_\mu^{m-1}
  = \cdots
  = S_\mu^m \mu s_\z^{(j)} + ms_\mu^{(j)} S_\mu^{m-1},
  \label{eq:push_z_mu_sS}
\end{align}
where we will generally find it nicer to express results in terms of
$\mu s_\z^{(j)}$ and $\mu S_\z$ rather than $s_\z^{(j)}$ and $S_\z$.
Summing over the single-spin index $j$ in both of the cases above
gives us the purely collective-spin versions of these identities:
\begin{align}
  \p{\mu S_\z}^m S_\mu = S_\mu \p{1 + \mu S_\z}^m,
  &&
  \mu S_\z S_\mu^m = S_\mu^m \p{m + \mu S_\z},
  \label{eq:push_z_mu_single}
\end{align}
where we can repeat the process of pushing through individual $S_\z$
operators $\ell$ times to get
\begin{align}
  \p{\mu S_\z}^\ell S_\mu^m
  = \p{\mu S_\z}^{\ell-1} S_\mu^m \p{m + \mu S_\z}
  = \p{\mu S_\z}^{\ell-2} S_\mu^m \p{m + \mu S_\z}^2
  = \cdots
  = S_\mu^m \p{m + \mu S_\z}^\ell.
  \label{eq:push_z_mu}
\end{align}
Multiplying \eqref{eq:push_z_mu} through by $\p{\mu\nu}^\ell$ (for
$\nu\in\set{+1,-1}$) and taking its Hermitian conjugate, we can say
that more generally
\begin{align}
  \p{\nu S_\z}^\ell S_\mu^m
  = S_\mu^m \p{\mu\nu m+\nu S_\z}^\ell,
  &&
  S_\mu^m \p{\nu S_\z}^\ell
  = \p{-\mu\nu m+\nu S_\z}^\ell S_\mu^m.
\end{align}
Finding commutation relations between powers of transverse spin
operators, i.e.~$S_\mu$ and $S_\bmu$, turns out to be considerably
more difficult than the cases we have worked out thus far.  We
therefore save this work for Appendix \ref{sec:comm_transverse}.


\section{Commutation relations between powers of transverse spin
  operators}
\label{sec:comm_transverse}

To find commutation relations between powers of transverse collective
spin operators, we first compute
\begin{align}
  S_\mu^m s_\bmu^{(j)}
  &= S_\mu^{m-1}s_\bmu^{(j)} S_\mu
  + S_\mu^{m-1} 2\mu s_\z^{(j)} \\
  &= S_\mu^{m-2} s_\bmu^{(j)} S_\mu^2
  + S_\mu^{m-2} 2\mu s_\z^{(j)} S_\mu
  + S_\mu^{m-1} 2\mu s_\z^{(j)} \\
  &= s_\bmu^{(j)} S_\mu^m
  + \sum_{k=0}^{m-1} S_\mu^k 2\mu s_\z^{(j)} S_\mu^{m-k-1}
  \label{eq:push_mu_Ss_start}.
\end{align}
While \eqref{eq:push_mu_Ss_start} gives us the commutator
$\sp{S_\mu^m,s_\bmu^{(j)}}_-$, we would like to enforce an ordering on
products of spin operators, which will ensure that we only keep track
of operators that are linearly independent.  We choose (for now) to
impose an ordering with all $s_\bmu^{(j)}$ operators on the left, and
all $s_\z^{(j)}$ operators on the right, as such an ordering will
prove convenient for the calculations in this section\footnote{In
  retrospect, it would likely have been nicer to have all
  $s_\mu^{(j)}$ operators on the right throughout these calculations,
  due to the enhanced symmetry expressions would have with respect to
  Hermitian conjugation.  In any case, we provide the final result of
  this section in both ordering conventions, and therefore feel no
  need to reproduce these calculations with a different ordering of
  spin operators.}.  This choice of ordering compels us to expand
\begin{align}
  \sum_{k=0}^{m-1} S_\mu^k 2\mu s_\z^{(j)} S_\mu^{m-k-1}
  &= \sum_{k=0}^{m-1} S_\mu^k
  \sp{2\p{m-k-1} s_\mu^{(j)} S_\mu^{m-k-2}
    + S_\mu^{m-k-1} 2\mu s_\z^{(j)}} \\
  &= m \p{m-1} s_\mu^{(j)} S_\mu^{m-2}
  + m S_\mu^{m-1} 2\mu s_\z^{(j)},
\end{align}
which implies
\begin{align}
  S_\mu^m s_\bmu^{(j)}
  = s_\bmu^{(j)} S_\mu^m + m \p{m-1} s_\mu^{(j)} S_\mu^{m-2}
  + m S_\mu^{m-1} 2\mu s_\z^{(j)},
  \label{eq:push_mu_Ss}
\end{align}
and in turn
\begin{align}
  S_\mu^m S_\bmu = S_\bmu S_\mu^m
  + m S_\mu^{m-1} \p{m - 1 + 2\mu S_\z}.
  \label{eq:push_mu_single}
\end{align}
As the next logical step, we take on the task of computing
\begin{align}
  S_\mu^m S_\bmu^n
  = S_\mu^{m-1} S_\bmu^n S_\mu
  + n \sp{S_\mu^{m-1} S_\bmu^{n-1} \p{1 - n + 2\mu S_\z}}
  = S_\bmu^n S_\mu^m
  + n \sum_{k=0}^{m-1} S_\mu^{m-k-1} S_\bmu^{n-1}
  \p{1 - n + 2\mu S_\z} S_\mu^k,
\end{align}
which implies
\begin{align}
  \sp{S_\mu^m, S_\bmu^n}_-
  = C_{mn;\mu}
  \equiv n \sum_{k=0}^{m-1} S_\mu^{m-k-1} S_\bmu^{n-1}
  \p{1 - n + 2\mu S_\z} S_\mu^k.
\end{align}
We now need rearrange the operators in $C_{mn;\mu}$ into a standard
order, which means pushing all $S_\z$ operators to the right and, for
the purposes of this calculation, all $S_\bmu$ operators to the left.
We begin by pushing $S_\mu^k$ to the left of $S_\z$, which takes
$2\mu S_\z\to 2\mu S_\z+2k$, and then push $S_\mu^{m-k-1}$ to the
right of $S_\bmu^{n-1}$, giving us
\begin{align}
  C_{mn;\mu}
  &= n \sum_{k=0}^{m-1}
  \p{S_\bmu^{n-1} S_\mu^{m-k-1} + C_{m-k-1,n-1;\mu}} S_\mu^k
  \p{2k + 1 - n + 2\mu S_\z} \\
  &= D_{mn;\mu}
  + n \sum_{k=0}^{m-2} C_{m-k-1,n-1;\mu}
  S_\mu^k \p{2k + 1 - n + 2\mu S_\z},
  \label{eq:C_mn}
\end{align}
where we have dropped the last ($k=m-1$) term in the remaining sum
because $C_{m-k-1,n-1;\mu}=0$ if $k=m-1$, and
\begin{align}
  D_{mn;\mu}
  \equiv mn S_\bmu^{n-1} S_\mu^{m-1} \p{m - n + 2\mu S_\z}.
  \label{eq:D_mn}
\end{align}
To our despair, we have arrived in \eqref{eq:C_mn} at a {\it
  recursive} formula for $C_{mn;\mu}$.  Furthermore, we have not even
managed to order all spin operators, as $C_{m-k-1,n-1;\mu}$ contains
$S_\z$ operators that are to the left of $S_\mu^k$.  To sort all spin
operators once and for all, we define
\begin{align}
  C_{mn;\mu}^{(k)} \equiv C_{m-k,n;\mu} S_\mu^k,
  &&
  D_{mn;\mu}^{(k)} \equiv D_{m-k,n;\mu} S_\mu^k,
\end{align}
which we can expand as
\begin{align}
  D_{mn;\mu}^{(k)}
  &= \p{m-k}n S_\bmu^{n-1} S_\mu^{m-k-1}
  \p{m-k-n+2\mu S_\z} S_\mu^k \\
  &= \p{m-k}n S_\bmu^{n-1} S_\mu^{m-1} \p{k+m-n+2\mu S_\z},
  \label{eq:D_mn_k}
\end{align}
and
\begin{align}
  C_{mn;\mu}^{(k)}
  &= D_{m-k,n;\mu} S_\mu^k + n \sum_{j=0}^{m-k-2}
  C_{m-k-j-1,n-1;\mu} S_\mu^j \p{2j+1-n+2\mu S_\z} S_\mu^k \\
  &= D_{mn;\mu}^{(k)} + n \sum_{j=0}^{m-k-2}
  C_{m-k-j-1,n-1;\mu} S_\mu^{j+k} \p{2j+2k+1-n+2\mu S_\z} \\
  &= D_{mn;\mu}^{(k)} + n \sum_{j=0}^{m-k-2}
  C_{m-1,n-1;\mu}^{(k+j)} \p{2\sp{j+k}+1-n+2\mu S_\z} \\
  &= D_{mn;\mu}^{(k)} + n \sum_{j=k}^{m-2}
  C_{m-1,n-1;\mu}^{(j)} \p{2j+1-n+2\mu S_\z}.
  \label{eq:C_mn_k}
\end{align}
While the resulting expression in \eqref{eq:C_mn_k} strongly resembles
that in \eqref{eq:C_mn}, there is one crucial difference: all spin
operators in \eqref{eq:C_mn_k} have been sorted into a standard order.
We can now repeatedly substitute $C_{mn;\mu}^{(k)}$ into itself, each
time decreasing $m$ and $n$ by 1, until one of $m$ or $n$ reaches
zero.  Such repeated substitution yields the expansion
\begin{align}
  C_{mn;\mu}
  = C_{mn;\mu}^{(0)}
  = D_{mn;\mu}
  + \sum_{p=1}^{\min\set{m,n}-1} E_{mn;\mu}^{(p)},
  \label{eq:C_mn_E}
\end{align}
where the first two terms in the sum over $p$ are
\begin{align}
  E_{mn;\mu}^{(1)}
  &= n \sum_{k=0}^{m-2} D_{m-1,n-1;\mu}^{(k)} \p{2k+1-n+2\mu S_\z}, \\
  E_{mn;\mu}^{(2)}
  &= n \sum_{k_1=0}^{m-2} \p{n-1} \sum_{k_2=k_1}^{m-3}
  D_{m-2,n-2;\mu}^{(k_2)} \p{2k_2+2-n+2\mu S_\z} \p{2k_1+1-n+2\mu S_\z},
\end{align}
and more generally for $p>1$,
\begin{align}
  E_{mn;\mu}^{(p)}
  = \f{n!}{\p{n-p}!}
  \sum_{k_1=0}^{m-2} \sum_{k_2=k_1}^{m-3} \cdots\sum_{k_p=k_{p-1}}^{m-p-1}
  D_{m-p,n-p;\mu}^{(k_p)} \prod_{j=1}^p \p{2k_j+j-n+2\mu S_\z}.
  \label{eq:E_mn_p}
\end{align}
In principle, the expressions in \eqref{eq:D_mn}, \eqref{eq:D_mn_k},
\eqref{eq:C_mn_E}, and \eqref{eq:E_mn_p} suffice to evaluate the
commutator $\sp{S_\mu^m,S_\bmu^n}_- = C_{mn;\mu}$, but this result is
-- put lightly -- quite a mess: the expression for $E_{mn;\mu}^{(p)}$
in \eqref{eq:E_mn_p} involves a sum over $p$ mutually dependent
intermediate variables, each term of which additionally contains a
product of $p$ factors.  We therefore devote the rest of this section
to simplifying our result for the commutator
$\sp{S_\mu^m,S_\bmu^n}_-$.

Observing that in \eqref{eq:E_mn_p} we always have
$0\le k_1\le k_2\le\cdots\le k_p\le m-p-1$, we can rearrange the order
of the sums and relabel $k_p\to\ell$ to get
\begin{align}
  E_{mn;\mu}^{(p)}
  = \f{n!}{\p{n-p}!}
  \sum_{\ell=0}^{m-p-1} D_{m-p,n-p;\mu}^{(\ell)} \p{2\ell+F_{np;\mu}}
  \sum_{\p{\v k,p-1,\ell}} \prod_{j=1}^{p-1} \p{2k_{p-j}-j+F_{np;\mu}},
  \label{eq:E_mn_p_sum}
\end{align}
where for shorthand we define
\begin{align}
  F_{np;\mu} \equiv p - n + 2\mu S_\z,
  &&
  \sum_{\p{\v k,q,\ell}} X \equiv
  \sum_{k_1=0}^\ell \sum_{k_2=k_1}^\ell
  \cdots \sum_{k_q=k_{q-1}}^\ell X.
\end{align}
We now further define
\begin{align}
  f_{np\ell;\mu}\p{k,q} \equiv \p{\ell-k+q} \p{\ell+k-q+F_{np;\mu}},
\end{align}
and evaluate sums successively over $k_{p-1},k_{p-2},\cdots,k_1$,
finding
\begin{align}
  \sum_{\p{\v k,p-1,\ell}} \prod_{j=1}^{p-1} \p{2k_{p-j}-j+F_{np;\mu}}
  &= \sum_{\p{\v k,p-2,\ell}}
  \prod_{j=2}^{p-1} \p{2k_{p-j}-j+F_{np;\mu}}
  f_{np\ell;\mu}\p{k_{p-2},1} \\
  &= \f1{\p{r-1}!} \sum_{\p{\v k,p-r,\ell}}
  \prod_{j=r}^{p-1} \p{2k_{p-j}-j+F_{np;\mu}}
  \prod_{q=1}^{r-1} f_{np\ell;\mu}\p{k_{p-r},q} \\
  &= \f1{\p{p-1}!} \prod_{q=1}^{p-1} f_{np\ell;\mu}\p{0,q} \\
  &= { \ell + p - 1 \choose p - 1 }
  \prod_{q=1}^{p-1} \p{\ell-q+F_{np;\mu}}.
\end{align}
Substitution of this result together with $D_{m-p,n-p;\mu}^{(\ell)}$
using \eqref{eq:D_mn_k} into \eqref{eq:E_mn_p_sum} then gives us
\begin{align}
  E_{mn;\mu}^{(p)}
  = \f{n!}{\p{n-p-1}!} S_\bmu^{n-p-1} S_\mu^{m-p-1} G_{mnp;\mu}
\end{align}
with
\begin{align}
  G_{mnp;\mu}
  &\equiv \sum_{\ell=0}^{m-p-1} { \ell + p - 1 \choose p - 1 }
  \p{m-p-\ell} \p{\ell+m-p+F_{np;\mu}}
  \p{2\ell + F_{np;\mu}}
  \prod_{q=1}^{p-1} \p{\ell-q+F_{np;\mu}} \\
  &= { m \choose p + 1 } \prod_{q=0}^p \p{m-p-q+F_{np;\mu}}.
\end{align}
We can further simplify
\begin{align}
  \prod_{q=0}^p \p{m-p-q+F_{np;\mu}}
  = \prod_{q=0}^p \p{m-n-q+2\mu S_\z}
  = \sum_{q=0}^{p+1} \p{-1}^{p+1-q}
  { p+1 \brack q } \p{m-n+2\mu S_\z}^q,
\end{align}
where ${ p \brack q }$ is an unsigned Stirling number of the first
kind, and finally
\begin{align}
  \sum_{q=0}^p \p{-1}^{p-q} { p \brack q } \p{m-n+2\mu S_\z}^q
  &= \sum_{q=0}^p \p{-1}^{p-q} { p \brack q } \sum_{\ell=0}^q
  { q \choose \ell } \p{m-n}^{q-\ell} \p{2\mu S_\z}^\ell \\
  &= \sum_{\ell=0}^p 2^\ell \sum_{q=\ell}^p \p{-1}^{p-q}
  { p \brack q } { q \choose \ell } \p{m-n}^{q-\ell} \p{\mu S_\z}^\ell.
\end{align}
Putting everything together, we finally have
\begin{align}
  E_{mn;\mu}^{(p-1)}
  = p! { m \choose p } { n \choose p }
  S_\bmu^{n-p} S_\mu^{m-p}
  \sum_{\ell=0}^p \epsilon_{mn}^{p\ell} \p{\mu S_\z}^\ell,
\end{align}
with
\begin{align}
  \epsilon_{mn}^{p\ell}
  \equiv 2^\ell \sum_{q=\ell}^p \p{-1}^{p-q}
  { p \brack q } { q \choose \ell } \p{m-n}^{q-\ell},
\end{align}
where in this final form $E_{mn;\mu}^{(0)} = D_{mn;\mu}$, which
together with the expansion for $C_{mn;\mu}$ in \eqref{eq:C_mn_E}
implies that
\begin{align}
  \sp{S_\mu^m, S_\bmu^n}_-
  = \sum_{p=1}^{\min\set{m,n}}
  p! { m \choose p } { n \choose p } S_\bmu^{n-p} S_\mu^{m-p}
  \sum_{\ell=0}^p \epsilon_{mn}^{p\ell} \p{\mu S_\z}^\ell,
  \label{eq:comm_mu}
\end{align}
and
\begin{align}
  S_\mu^m S_\bmu^n
  = \sum_{p=0}^{\min\set{m,n}}
  p! { m \choose p } { n \choose p } S_\bmu^{n-p} S_\mu^{m-p}
  \sum_{\ell=0}^p \epsilon_{mn}^{p\ell} \p{\mu S_\z}^\ell.
  \label{eq:push_mu_bmu}
\end{align}
If we wish to order products of collective spin operators with $S_\z$
in between $S_\bmu$ and $S_\mu$, then
\begin{align}
  S_\mu^m S_\bmu^n
  = \sum_{p=0}^{\min\set{m,n}} p! { m \choose p } { n \choose p }
  S_\bmu^{n-p} Z_{mn;\bmu}^{(p)} S_\mu^{m-p},
\end{align}
where
\begin{align}
  Z_{mn;\bmu}^{(p)}
  \equiv \sum_{\ell=0}^p \epsilon_{mn}^{p\ell}
  \p{-\sp{m-p} + \mu S_\z}^\ell
  = \sum_{q=0}^p \zeta_{mn}^{pq} \p{\bmu S_\z}^q,
  \label{eq:Z_mnp}
\end{align}
with
\begin{align}
  \zeta_{mn}^{pq}
  \equiv \sum_{\ell=q}^p \epsilon_{mn}^{p\ell}
  { \ell \choose q } \p{-1}^\ell \p{m-p}^{\ell-q}
  = \p{-1}^p 2^q \sum_{s=q}^p
  { p \brack s } { s \choose q } \p{m+n-2p}^{s-q}.
  \label{eq:zeta_mnpq}
\end{align}
Here ${ p \brack s }$ is an unsigned Stirling number of the first
kind.


\section{Product of arbitrary ordered collective spin operators}
\label{sec:prod_general}

The most general product of collective spin operators that we need to
compute is
\begin{align}
  \S^{pqr}_{\ell mn;\mu}
  = S_\mu^p \p{\mu S_\z}^q S_\bmu^r
  S_\mu^\ell \p{\mu S_\z}^m S_\bmu^n
  = \sum_{k=0}^{\min\set{r,\ell}} k! { r \choose k } { \ell \choose k }
  S_\mu^{p+\ell-k} \tilde Z_{qr\ell m;\mu}^{(k)} S_\bmu^{r+n-k},
  \label{eq:general_product}
\end{align}
where
\begin{align}
  \tilde Z_{qr\ell m;\mu}^{(k)}
  &\equiv \p{\ell-k+\mu S_\z}^q
  Z_{r\ell;\mu}^{(k)} \p{r-k+\mu S_\z}^m \\
  &= \sum_{a=0}^k \zeta_{r\ell}^{ka}
  \sum_{b=0}^q \p{\ell-k}^{q-b} { q \choose b }
  \sum_{c=0}^m \p{r-k}^{m-c} { m \choose c }
  \p{\mu S_\z}^{a+b+c},
\end{align}
is defined in terms of $Z_{r\ell;\mu}^{(k)}$ and $\zeta_{r\ell}^{ka}$
as respectively given in \eqref{eq:Z_mnp} and \eqref{eq:zeta_mnpq}.
The (anti-)commutator of two ordered products of collective spin
operators is then
\begin{align}
  \sp{S_\mu^p \p{\mu S_\z}^q S_\bmu^r,
    S_\mu^\ell \p{\mu S_\z}^m S_\bmu^n}_\pm
  = \S^{pqr}_{\ell mn;\mu} \pm \S^{\ell mn}_{pqr;\mu}.
\end{align}


\section{Sandwich identities for single-spin decoherence calculations}
\label{sec:sandwich_single}

In this section we derive several identities that will be necessary
for computing the effects of single-spin decoherence on ordered
products of collective spin operators, i.e.~on operators of the form
$S_\mu^\ell \p{\mu S_\z}^m S_\bmu^n$.  These identities all involve
sandwiching a collective spin operator between operators that act on
individual spins only, and summing over all individual spin indices.
Our general strategy will be to use commutation relations to push
single-spin operators together, and then evaluate the sum to arrive at
an expression in terms of only collective spin operators.

We first compute sums of single-spin operators sandwiching
$\p{\mu S_\z}^m$, when necessary making use of the identity in
\eqref{eq:push_z_mu_Ss}.  Up to Hermitian conjugation, the unique
cases are, for $S\equiv N/2$ and $\mu,\nu\in\set{+1,-1}$,
\begin{align}
  \sum_j s_\z^{(j)} \p{\mu S_\z}^m s_\z^{(j)}
  &= \sum_j s_\z^{(j)} s_\z^{(j)} \p{\mu S_\z}^m
  = \f14 \sum_j \1_j \p{\mu S_\z}^m
  = \f12 S \p{\mu S_\z}^m, \\
  \sum_j s_\z^{(j)} \p{\mu S_\z}^m s_\nu^{(j)}
  &= \p{\mu S_\z}^m \sum_j s_\z^{(j)} s_\nu^{(j)}
  = \f12 \p{\mu S_\z}^m \nu S_\nu
  = \f12 \nu S_\nu \p{\mu\nu+\mu S_\z}^m, \\
  \sum_j s_\nu^{(j)} \p{\mu S_\z}^m s_\nu^{(j)}
  &= \sum_j s_\nu^{(j)} s_\nu^{(j)} \p{\mu\nu+\mu S_\z}^m
  = 0, \\
  \sum_j s_\bnu^{(j)} \p{\mu S_\z}^m s_\nu^{(j)}
  &= \sum_j s_\bnu^{(j)} s_\nu^{(j)} \p{\mu\nu+\mu S_\z}^m
  = \p{S-\nu S_\z} \p{\mu\nu+\mu S_\z}^m.
\end{align}
We are now equipped to derive similar identities for more general
collective spin operators.  Making heavy use of identities
\eqref{eq:push_z_mu_sS} and \eqref{eq:push_mu_Ss} to push single-spin
operators through transverse collective-spin operators, we again work
through all combinations that are unique up to Hermitian conjugation,
finding
\begin{align}
  \sum_j s_\z^{(j)} S_\mu^\ell \p{\mu S_\z}^m S_\bmu^n s_\z^{(j)}
  &= \f12 \p{S-\ell-n} S_\mu^\ell \p{\mu S_\z}^m S_\bmu^n
  + \ell n S_\mu^{\ell-1} \p{S+\mu S_\z}
  \p{-1+\mu S_\z}^m S_\bmu^{n-1},
  \label{eq:san_z_z} \allowdisplaybreaks \\
  \sum_j s_\z^{(j)} S_\mu^\ell \p{\mu S_\z}^m S_\bmu^n s_\mu^{(j)}
  &= \f12 \mu S_\mu^{\ell+1} \p{1+\mu S_\z}^m S_\bmu^n
  - \mu n \p{S-\ell-\f12\sp{n-1}} S_\mu^\ell
  \p{\mu S_\z}^m S_\bmu^{n-1} \notag \\
  &\qquad - \mu\ell n\p{n-1} S_\mu^{\ell-1}
  \p{S+\mu S_\z} \p{-1+\mu S_\z}^m S_\bmu^{n-2},
  \label{eq:san_z_mu} \allowdisplaybreaks \\
  \sum_j s_\z^{(j)} S_\mu^\ell \p{\mu S_\z}^m S_\bmu^n s_\bmu^{(j)}
  &= -\f12 \mu S_\mu^\ell \p{\mu S_\z}^m S_\bmu^{n+1}
  + \mu \ell S_\mu^{\ell-1} \p{S+\mu S_\z} \p{-1+\mu S_\z}^m S_\bmu^n,
  \label{eq:san_z_bmu} \allowdisplaybreaks \\
  \sum_j s_\mu^{(j)} S_\mu^\ell \p{\mu S_\z}^m S_\bmu^n s_\mu^{(j)}
  &= n S_\mu^{\ell+1} \p{\mu S_\z}^m S_\bmu^{n-1}
  - n\p{n-1} S_\mu^\ell \p{S+\mu S_\z} \p{-1+\mu S_\z}^m S_\bmu^{n-2},
  \label{eq:san_mu_mu} \allowdisplaybreaks \\
  \sum_j s_\mu^{(j)} S_\mu^\ell \p{\mu S_\z}^m S_\bmu^n s_\bmu^{(j)}
  &= S_\mu^\ell \p{S+\mu S_\z}\p{-1+\mu S_\z}^m S_\bmu^n,
  \label{eq:san_mu_bmu} \allowdisplaybreaks \\
  \sum_j s_\bmu^{(j)} S_\mu^\ell \p{\mu S_\z}^m S_\bmu^n s_\mu^{(j)}
  &= S_\mu^\ell \p{S - \ell - n - \mu S_\z}
  \p{1+\mu S_\z}^m S_\bmu^n
  + \ell n \p{2S - \ell - n + 2}
  S_\mu^{\ell-1} \p{\mu S_\z}^m S_\bmu^{n-1} \notag \\
  &\qquad + \ell n \p{\ell-1} \p{n-1} S_\mu^{\ell-2} \p{S+\mu S_\z}
  \p{-1+\mu S_\z}^m S_\bmu^{n-2}.
  \label{eq:san_bmu_mu}
\end{align}


\section{Uncorrelated, permutationally-symmetric single-spin
  decoherence}
\label{sec:decoherence_single}

In this section we work out the effects of permutationally-symmetric
decoherence of individual spins on collective spin operators of the
form $S_\mu^\ell \p{\mu S_\z}^m S_\bmu^n$.  For compactness, we define
\begin{align}
  \D\p{\gamma} \O
  \equiv \D\p{\set{\gamma^{(j)}:j=1,2,\cdots,N}} \O
  = \sum_j\p{{\gamma^{(j)}}^\dag \O \gamma^{(j)}
    - \f12\sp{{\gamma^{(j)}}^\dag \gamma^{(j)}, \O}_+},
\end{align}
where $\gamma$ is an operator that acts on a single spin,
$\gamma^{(j)}$ is an operator that acts with $\gamma$ on spin $j$ and
trivially on all other spins, and $N$ is the total number of spins.


\subsection{Decay-type decoherence}
\label{sec:decay_single}

The effect of decoherence via uncorrelated decay ($\mu=-1$) or
excitation ($\mu=1$) of individual spins is described by
\begin{align}
  \D\p{s_\mu} \O
  = \sum_j\p{s_\bmu^{(j)} \O s_\mu^{(j)}
    - \f12\sp{s_\bmu^{(j)} s_\mu^{(j)},\O}_+}
  = \sum_j s_\bmu^{(j)} \O s_\mu^{(j)}
  - S \O + \f{\mu}{2} \sp{S_\z, \O}_+.
\end{align}
In order to determine the effect of this decoherence on general
collective spin operators, we expand the anti-commutator
\begin{align}
  \sp{S_\z, S_\mu^\ell \p{\mu S_\z}^m S_\bmu^n}_+
  = S_\z S_\mu^\ell \p{\mu S_\z}^m S_\bmu^n
  + S_\mu^\ell \p{\mu S_\z}^m S_\bmu^n S_\z
  = \mu S_\mu^\ell\p{\ell+n+2\mu S_\z} \p{\mu S_\z}^m S_\bmu^n,
\end{align}
which implies, using \eqref{eq:san_mu_bmu},
\begin{align}
  \D\p{s_\bmu} \p{S_\mu^\ell \p{\mu S_\z}^m S_\bmu^n}
  = S_\mu^\ell \p{S+\mu S_\z}\p{-1+\mu S_\z}^m S_\bmu^n
  - S_\mu^\ell\sp{S + \f12\p{\ell+n} + \mu S_\z}
  \p{\mu S_\z}^m S_\bmu^n,
  \label{eq:decay_diff}
\end{align}
and, using \eqref{eq:san_bmu_mu},
\begin{align}
  \D\p{s_\mu} \p{S_\mu^\ell \p{\mu S_\z}^m S_\bmu^n}
  &= S_\mu^\ell \p{S - \ell - n - \mu S_\z} \p{1+\mu S_\z}^m S_\bmu^n
  - S_\mu^\ell\sp{S - \f12\p{\ell+n} - \mu S_\z}
  \p{\mu S_\z}^m S_\bmu^n \notag \\
  &\qquad + \ell n \p{2S - \ell - n + 2}
  S_\mu^{\ell-1} \p{\mu S_\z}^m S_\bmu^{n-1} \notag \\
  &\qquad + \ell n \p{\ell-1} \p{n-1} S_\mu^{\ell-2} \p{S + \mu S_\z}
  \p{-1+\mu S_\z}^m S_\bmu^{n-2}.
  \label{eq:decay_same}
\end{align}
Decoherence via jump operators $s_\bmu^{(j)}$ only couples operators
$S_\mu^\ell \p{\mu S_\z}^m S_\bmu^n$ to operators
$S_\mu^\ell \p{\mu S_\z}^{m'} S_\bmu^n$ with $m'\le m$.  Decoherence
via jump operators $s_\mu^{(j)}$, meanwhile, makes operators
$S_\mu^\ell \p{\mu S_\z}^m S_\bmu^n$ ``grow'' in $m$ through the last
term in \eqref{eq:decay_same}, although the sum $\ell+m+n$ does not
grow.


\subsection{Dephasing}
\label{sec:dephasing_single}

The effect of decoherence via single-spin dephasing is described by
\begin{align}
  \D\p{s_\z} \O
  = \sum_j\p{s_\z^{(j)} \O s_\z^{(j)}
    - \f12\sp{s_\z^{(j)} s_\z^{(j)},\O}_+}
  = \sum_j s_\z^{(j)} \O s_\z^{(j)} - \f12 S \O.
\end{align}
From \eqref{eq:san_z_z}, we then have
\begin{align}
  \D\p{s_\z} \p{S_\mu^\ell \p{\mu S_\z}^m S_\bmu^n}
  = -\f12\p{\ell+n} S_\mu^\ell \p{\mu S_\z}^m S_\bmu^n
  + \ell n S_\mu^{\ell-1} \p{S + \mu S_\z}
  \p{-1 + \mu S_\z}^m S_\bmu^{n-1}.
\end{align}
Decoherence via single-spin dephasing makes operators
$S_\mu^\ell \p{\mu S_\z}^m S_\bmu^n$ ``grow'' in $m$, although the sum
$\ell+m+n$ does not grow.


\subsection{The general case}
\label{sec:general_single}

The most general type of single-spin decoherence is described by
\begin{align}
  \D\p{\gamma} \O
  = \sum_j\p{{\gamma^{(j)}}^\dag \O \gamma^{(j)}
    - \f12\sp{{\gamma^{(j)}}^\dag \gamma^{(j)}, \O}_+},
  &&
  \gamma \equiv \gamma_\z s_\z + \gamma_+ s_+ + \gamma_- s_-.
  \label{eq:D_general_single}
\end{align}
To simplify \eqref{eq:D_general_single}, we expand
\begin{align}
  \gamma^\dag \O \gamma
  = \abs{\gamma_\z}^2 s_\z \O s_\z
  + \sum_\mu \p{\abs{\gamma_\mu}^2 s_\bmu \O s_\mu
    + \gamma_\bmu^* \gamma_\mu s_\mu \O s_\mu
    + \gamma_\z^* \gamma_\mu s_\z \O s_\mu
    + \gamma_\bmu^* \gamma_\z s_\mu \O s_\z},
\end{align}
and
\begin{align}
  \gamma^\dag \gamma
  = \f14 \abs{\gamma_\z}^2
  + \f12 \sum_\mu \sp{\abs{\gamma_\mu}^2 \p{1-2\mu s_\z}
    + \mu \p{\gamma_\z^*\gamma_\mu - \gamma_\bmu^*\gamma_\z} s_\mu},
\end{align}
which implies
\begin{align}
  \D\p{\gamma} \O
  &= \sum_{X\in\set{\z,+,-}} \abs{\gamma_X}^2 \D\p{s_X} \O
  + \sum_{\mu,j}
  \p{\gamma_\bmu^* \gamma_\mu s_\mu^{(j)} \O s_\mu^{(j)}
    + \gamma_\z^* \gamma_\mu s_\z^{(j)} \O s_\mu^{(j)}
    + \gamma_\bmu^* \gamma_\z s_\mu^{(j)} \O s_\z^{(j)}}
  \notag \\
  &\qquad -\f14 \sum_\mu \mu
  \p{\gamma_\z^*\gamma_\mu - \gamma_\bmu^*\gamma_\z} \sp{S_\mu, \O}_+.
\end{align}
In order to compute the effect of this decoherence on general
collective spin operators, we expand the anti-commutator
\begin{align}
  \sp{S_\mu, S_\mu^\ell \p{\mu S_\z}^m S_\bmu^n}_+
  = S_\mu^{\ell+1} \sp{\p{\mu S_\z}^m+\p{1+\mu S_\z}^m} S_\bmu^n
  - n S_\mu^\ell \p{n-1+2\mu S_\z} \p{\mu S_\z}^m S_\bmu^{n-1}.
  \label{eq:S_mu_acomm}
\end{align}
Recognizing a resemblance between terms in \eqref{eq:S_mu_acomm} and
\eqref{eq:san_z_mu}, we collect terms to simplify
\begin{align}
  \sum_j s_\z^{(j)} S_\mu^\ell \p{\mu S_\z}^m S_\bmu^n s_\mu^{(j)}
  - \f14 \mu \sp{S_\mu, S_\mu^\ell \p{\mu S_\z}^m S_\bmu^n}_+
  = K_{\ell mn;\mu} + L_{\ell mn;\mu}
  \label{eq:dec_z_mu}
\end{align}
and likewise
\begin{align}
  \sum_j s_\mu^{(j)} S_\mu^\ell \p{\mu S_\z}^m S_\bmu^n s_\z^{(j)}
  + \f14 \mu \sp{S_\mu, S_\mu^\ell \p{\mu S_\z}^m S_\bmu^n}_+
  = K_{\ell mn;\mu} + M_{\ell mn;\mu}
  \label{eq:dec_mu_z}
\end{align}
with
\begin{align}
  K_{\ell mn;\mu}
  &\equiv \f14 \mu S_\mu^{\ell+1}
  \sp{\p{1+\mu S_\z}^m-\p{\mu S_\z}^m} S_\bmu^n, \\
  L_{\ell mn;\mu}
  &\equiv -\mu n S_\mu^\ell \sp{S-\ell-\f34\p{n-1}-\f12\mu S_\z}
  \p{\mu S_\z}^m S_\bmu^{n-1}
  - \mu\ell n\p{n-1} S_\mu^{\ell-1}
  \p{S+\mu S_\z} \p{-1+\mu S_\z}^m S_\bmu^{n-2}, \\
  M_{\ell mn;\mu}
  &\equiv \mu n S_\mu^\ell \sp{\p{S+\mu S_\z}\p{-1+\mu S_\z}^m
    - \f12\p{\f12\sp{n-1}+\mu S_\z}\p{\mu S_\z}^m} S_\bmu^{n-1}.
\end{align}
Defining for completion
\begin{align}
  P_{\ell mn;\mu}
  &\equiv \sum_j s_\mu^{(j)} S_\mu^\ell
  \p{\mu S_\z}^m S_\bmu^n s_\mu^{(j)}
  = n S_\mu^{\ell+1} \p{\mu S_\z}^m S_\bmu^{n-1}
  - n\p{n-1} S_\mu^\ell \p{S+\mu S_\z} \p{-1+\mu S_\z}^m S_\bmu^{n-2},
\end{align}
and
\begin{align}
  Q_{\ell mn;\mu}^{(\gamma)}
  \equiv \gamma_\bmu^* \gamma_\mu P_{\ell mn;\mu}
  + \p{\gamma_\z^* \gamma_\mu + \gamma_\bmu^* \gamma_\z}
  K_{\ell mn;\mu}
  + \gamma_\z^* \gamma_\mu L_{\ell mn;\mu}
  + \gamma_\bmu^* \gamma_\z M_{\ell mn;\mu},
  \label{eq:Q_single}
\end{align}
we finally have
\begin{align}
  \D\p{\gamma} \p{S_\mu^\ell \p{\mu S_\z}^m S_\bmu^n}
  = \sum_{X\in\set{\z,+,-}} \abs{\gamma_X}^2
  \D\p{s_X} \p{S_\mu^\ell \p{\mu S_\z}^m S_\bmu^n}
  + Q_{\ell mn;\mu}^{(\gamma)} + \sp{Q_{nm\ell;\mu}^{(\gamma)}}^\dag.
\end{align}
Note that the sum $\ell+m+n$ for operators
$S_\mu^\ell \p{\mu S_\z}^m S_\bmu^n$ does not grow under this type of
decoherence.


\section{Sandwich identities for collective-spin decoherence
  calculations}
\label{sec:sandwich_collective}

In analogy with the work in Appendix \ref{sec:sandwich_single}, in
this section we work out sandwich identities necessary for
collective-spin decoherence calculations.  The simplest cases are
\begin{align}
  S_\mu S_\mu^\ell \p{\mu S_\z}^m S_\bmu^n S_\bmu
  &= S_\mu^{\ell+1} \p{\mu S_\z}^m S_\bmu^{n+1},
  \allowdisplaybreaks \\
  S_\mu S_\mu^\ell \p{\mu S_\z}^m S_\bmu^n S_\z
  &= \mu S_\mu^{\ell+1} \p{n+\mu S_\z} \p{\mu S_\z}^m S_\bmu^n,
  \allowdisplaybreaks \\
  S_\z S_\mu^\ell \p{\mu S_\z}^m S_\bmu^n S_\z
  &= S_\mu^\ell \sp{\ell n + \p{\ell+n} \mu S_\z + \p{\mu S_\z}^2}
  \p{\mu S_\z}^m S_\bmu^n.
\end{align}
With a bit more work, we can also find
\begin{align}
  S_\mu^\ell \p{\mu S_\z}^m S_\bmu^n S_\mu
  &= S_\mu^{\ell+1} \p{1+\mu S_\z}^m S_\bmu^n
  - n S_\mu^\ell \p{n-1+2\mu S_\z} \p{\mu S_\z}^m S_\bmu^{n-1},
\end{align}
which implies
\begin{align}
  S_\mu S_\mu^\ell \p{\mu S_\z}^m S_\bmu^n S_\mu
  &= S_\mu^{\ell+2} \p{1+\mu S_\z}^m S_\bmu^n
  - n S_\mu^{\ell+1} \p{n-1+2\mu S_\z} \p{\mu S_\z}^m S_\bmu^{n-1},
  \allowdisplaybreaks \\
  S_\z S_\mu^\ell \p{\mu S_\z}^m S_\bmu^n S_\mu
  &= \mu S_\mu^{\ell+1} \p{\ell+1+\mu S_\z} \p{1+\mu S_\z}^m S_\bmu^n
  \notag \\
  &\qquad - \mu n S_\mu^\ell
  \sp{\ell\p{n-1} + \p{2\ell+n-1}\mu S_\z + 2\p{\mu S_\z}^2}
  \p{\mu S_\z}^m S_\bmu^{n-1}.
\end{align}
Finally, we compute
\begin{align}
  S_\bmu S_\mu^\ell \p{\mu S_\z}^m S_\bmu^n S_\mu
  &= \sp{S_\mu^\ell S_\bmu - \ell S_\mu^{\ell-1} \p{\ell-1+2\mu S_\z}}
  \p{\mu S_\z}^m
  \sp{S_\mu S_\bmu^n - n \p{n-1+2\mu S_\z} S_\bmu^{n-1}} \notag \\
  &= S_\mu^\ell S_\bmu \p{\mu S_\z}^m S_\mu S_\bmu^n \notag \\
  &\qquad - S_\mu^\ell
  \sp{\ell\p{\ell+1} + n\p{n+1}+2\p{\ell+n}\mu S_\z}
  \p{1+\mu S_\z}^m S_\bmu^n \notag \\
  &\qquad + \ell n S_\mu^{\ell-1}
  \sp{\p{\ell-1}\p{n-1}+2\p{\ell+n-2}\mu S_\z + 4\p{\mu S_\z}^2}
  \p{\mu S_\z}^m S_\bmu^{n-1},
\end{align}
where
\begin{multline}
  S_\bmu \p{\mu S_\z}^m S_\mu
  = S_\bmu S_\mu \p{1+\mu S_\z}^m
  = \p{S_\mu S_\bmu - 2\mu S_\z} \p{1+\mu S_\z}^m
  = S_\mu \p{2+\mu S_\z}^m S_\bmu - 2\mu S_\z \p{1+\mu S_\z}^m,
\end{multline}
so
\begin{align}
  S_\bmu S_\mu^\ell \p{\mu S_\z}^m S_\bmu^n S_\mu
  &= S_\mu^{\ell+1} \p{2+\mu S_\z}^m S_\bmu^{n+1} \notag \\
  &\qquad - S_\mu^\ell
  \sp{\ell\p{\ell+1} + n\p{n+1}+2\p{\ell+n+1}\mu S_\z}
  \p{1+\mu S_\z}^m S_\bmu^n \notag \\
  &\qquad + \ell n S_\mu^{\ell-1}
  \sp{\p{\ell-1}\p{n-1}+2\p{\ell+n-2}\mu S_\z + 4\p{\mu S_\z}^2}
  \p{\mu S_\z}^m S_\bmu^{n-1}.
\end{align}


\section{Collective spin decoherence}
\label{sec:decoherence_collective}

In this section we work out the effects of collective decoherence on
general collective spin operators.  For shorthand, we define
\begin{align}
  \D\p{\Gamma} \O
  \equiv \D\p{\set{\Gamma}} \O
  = \Gamma^\dag \O \Gamma - \f12\sp{\Gamma^\dag \Gamma, \O}_+,
\end{align}
where $\Gamma$ is a collective spin jump operator.

\subsection{Decay-type decoherence and dephasing}
\label{sec:decay_dephasing_collective}

Making use of the results in Appendix \ref{sec:sandwich_collective},
we find that the effects of collective decay-type decoherence on
general collective spin operators are given by
\begin{align}
  \D\p{S_\bmu} \p{S_\mu^\ell \p{\mu S_\z}^m S_\bmu^n}
  &= -S_\mu^{\ell+1} \sp{\p{1+\mu S_\z}^m - \p{\mu S_\z}^m} S_\bmu^{n+1}
  \notag \\
  &\qquad + \f12 S_\mu^\ell \sp{\ell\p{\ell-1} + n\p{n-1}
    + 2\p{\ell+n}\mu S_\z} \p{\mu S_\z}^m S_\bmu^n,
\end{align}
and
\begin{align}
  \D\p{S_\mu} \p{S_\mu^\ell \p{\mu S_\z}^m S_\bmu^n}
  &= S_\mu^{\ell+1} \sp{\p{2+\mu S_\z}^m-\p{1+\mu S_\z}^m} S_\bmu^{n+1}
  \notag \\
  &\qquad - S_\mu^\ell
  \sp{\ell\p{\ell+1} + n\p{n+1}+2\p{\ell+n+1}\mu S_\z}
  \p{1+\mu S_\z}^m S_\bmu^n \notag \\
  &\qquad + \f12 S_\mu^\ell
  \sp{\ell\p{\ell+1} + n\p{n+1}+2\p{\ell+n+2}\mu S_\z}
  \p{\mu S_\z}^m S_\bmu^n \notag \\
  &\qquad + \ell n S_\mu^{\ell-1}
  \sp{\p{\ell-1}\p{n-1}+2\p{\ell+n-2}\mu S_\z + 4\p{\mu S_\z}^2}
  \p{\mu S_\z}^m S_\bmu^{n-1}.
\end{align}
Similarly, the effect of collective dephasing is given by
\begin{align}
  \D\p{S_\z} \p{S_\mu^\ell \p{\mu S_\z}^m S_\bmu^n}
  = -\f12 \p{\ell-n}^2 S_\mu^\ell \p{\mu S_\z}^m S_\bmu^n.
\end{align}


\subsection{The general case}
\label{sec:general_collective}

More generally, we consider jump operators of the form
\begin{align}
  \Gamma \equiv \Gamma_\z S_\z + \Gamma_+ S_+ + \Gamma_- S_-,
\end{align}
whose decoherence effects are determined by
\begin{align}
  \Gamma^\dag \O \Gamma
  = \abs{\Gamma_\z}^2 S_\z \O S_\z
  + \sum_\mu \p{\abs{\Gamma_\mu}^2 S_\bmu \O S_\mu
    + \Gamma_\bmu^* \Gamma_\mu S_\mu \O S_\mu
    + \Gamma_\z^* \Gamma_\mu S_\z \O S_\mu
    + \Gamma_\bmu^* \Gamma_\z S_\mu \O S_\z},
\end{align}
and
\begin{align}
  \Gamma^\dag \Gamma
  = \abs{\Gamma_\z}^2 S_\z^2
  + \sum_\mu \p{\abs{\Gamma_\mu}^2 S_\bmu S_\mu
    + \Gamma_\z^*\Gamma_\mu S_\z S_\mu
    + \Gamma_\bmu^* \Gamma_\z S_\mu S_\z
    + \Gamma_\bmu^* \Gamma_\mu S_\mu^2},
\end{align}
which implies
\begin{align}
  \D\p{\Gamma} \O
  &= \sum_{X\in\set{\z,+,-}} \abs{\Gamma_X}^2 \D\p{S_X} \O
  + \sum_\mu \p{\Gamma_\bmu^* \Gamma_\mu S_\mu \O S_\mu
    + \Gamma_\z^* \Gamma_\mu S_\z \O S_\mu
    + \Gamma_\bmu^* \Gamma_\z S_\mu \O S_\z}
  \notag \\
  &\qquad -\f12 \sum_\mu\p{\Gamma_\bmu^* \Gamma_\mu \sp{S_\mu^2, \O}_+
    + \Gamma_\z^*\Gamma_\mu \sp{S_\z S_\mu, \O}_+
    + \Gamma_\bmu^* \Gamma_\z \sp{S_\mu S_\z, \O}_+}.
\end{align}
In order to compute the effect of this decoherence on general
collective spin operators, we expand the anti-commutators
\begin{align}
  \sp{S_\mu^2, S_\mu^\ell \p{\mu S_\z}^m S_\bmu^n}_+
  &= S_\mu^{\ell+2} \sp{\p{2+\mu S_\z}^m+\p{\mu S_\z}^m} S_\bmu^n
  - 2n S_\mu^{\ell+1} \p{n+2\mu S_\z} \p{1+\mu S_\z}^m S_\bmu^{n-1}
  \notag \\
  &\qquad + n\p{n-1} S_\mu^\ell \sp{\p{n-1}\p{n-2}
    + 2\p{2n-3}\mu S_\z + 4\p{\mu S_\z}^2} \p{\mu S_\z}^m S_\bmu^{n-2},
  \allowdisplaybreaks \\
  \sp{S_\z S_\mu, S_\mu^\ell \p{\mu S_\z}^m S_\bmu^n}_+
  &= \mu S_\mu^{\ell+1} \sp{\p{\ell+1+\mu S_\z}\p{\mu S_\z}^m
    + \p{n+1+\mu S_\z} \p{1+\mu S_\z}^m } S_\bmu^n \notag \\
  &\qquad - \mu n S_\mu^\ell \sp{n \p{n-1}
    + \p{3n-1}\mu S_\z + 2\p{\mu S_\z}^2} \p{\mu S_\z}^m S_\bmu^{n-1},
  \allowdisplaybreaks \\
  \sp{S_\mu S_\z, S_\mu^\ell \p{\mu S_\z}^m S_\bmu^n}_+
  &= \mu S_\mu^{\ell+1} \sp{\p{\ell+\mu S_\z}\p{\mu S_\z}^m
    + \p{n+\mu S_\z} \p{1+\mu S_\z}^m} S_\bmu^n \notag \\
  &\qquad - \mu n S_\mu^\ell \sp{\p{n-1}^2
    + 3\p{n-1}\mu S_\z + 2\p{\mu S_\z}^2} \p{\mu S_\z}^m S_\bmu^{n-1}.
\end{align}
Collecting terms and defining
\begin{align}
  \Gamma_{\z,\mu}^{(\pm)}
  &\equiv \f12\p{\Gamma_\z^* \Gamma_\mu \pm \Gamma_\bmu^* \Gamma_\z},
  \allowdisplaybreaks \\
  \tilde L_{\ell mn;\mu}^{(\Gamma)}
  &\equiv \mu \sp{\p{\ell-n+\f12} \Gamma_{\z,\mu}^{(+)}
    + \p{\ell+\f12} \Gamma_{\z,\mu}^{(-)}}
  S_\mu^{\ell+1} \p{1+\mu S_\z}^m S_\bmu^n \notag \\
  &\qquad -\mu \sp{\p{\ell-n+\f12} \Gamma_{\z,\mu}^{(+)}
    + \p{n+\f12} \Gamma_{\z,\mu}^{(-)}}
  S_\mu^{\ell+1} \p{\mu S_\z}^m S_\bmu^n \notag \\
  &\qquad + \mu \Gamma_{\z,\mu}^{(-)}
  S_\mu^{\ell+1} \mu S_\z
  \sp{\p{1+\mu S_\z}^m - \p{\mu S_\z}^m} S_\bmu^n,
  \allowdisplaybreaks \\
  \tilde M_{\ell mn;\mu}^{(\Gamma)}
  &= -\mu n\p{n-1} \sp{\p{\ell-n+\f12} \Gamma_{\z,\mu}^{(+)}
    + \p{\ell-\f12} \Gamma_{\z,\mu}^{(-)}}
  S_\mu^\ell \p{\mu S_\z}^m S_\bmu^{n-1} \notag \\
  &\qquad - 2\mu n \sp{\p{\ell-n+\f12} \Gamma_{\z,\mu}^{(+)}
    + \p{\ell+\f12n-1} \Gamma_{\z,\mu}^{(-)}}
  S_\mu^\ell \p{\mu S_\z}^{m+1} S_\bmu^{n-1} \notag \\
  &\qquad - 2\mu n \Gamma_{\z,\mu}^{(-)}
  S_\mu^\ell \p{\mu S_\z}^{m+2} S_\bmu^{n-1},
  \allowdisplaybreaks \\
  \tilde P_{\ell mn;\mu}
  &\equiv -\f12 S_\mu^{\ell+2}
  \sp{\p{2+\mu S_\z}^m - 2\p{1+\mu S_\z}^m + \p{\mu S_\z}^m}
  S_\bmu^n \notag \\
  &\qquad + n S_\mu^{\ell+1} \sp{\p{n+2\mu S_\z} \p{1+\mu S_\z}^m
    - \p{n-1+2\mu S_\z} \p{\mu S_\z}^m}
  S_\bmu^{n-1} \notag \\
  &\qquad -n\p{n-1} S_\mu^\ell
  \sp{\f12\p{n-1}\p{n-2} + \p{2n-3}\mu S_\z + 2\p{\mu S_\z}^2}
  \p{\mu S_\z}^m S_\bmu^{n-2},
  \allowdisplaybreaks \\
  \tilde Q_{\ell mn;\mu}^{(\Gamma)}
  &\equiv \Gamma_\bmu^* \Gamma_\mu \tilde P_{\ell mn;\mu}
  + \tilde L_{\ell mn;\mu}^{(\Gamma)}
  + \tilde M_{\ell mn;\mu}^{(\Gamma)},
\end{align}
we then have
\begin{align}
  \D\p{\Gamma} \p{S_\mu^\ell \p{\mu S_\z}^m S_\bmu^n}
  = \sum_{X\in\set{\z,+,-}} \abs{\Gamma_X}^2
  \D\p{S_X} \p{S_\mu^\ell \p{\mu S_\z}^m S_\bmu^n}
  + \tilde Q_{\ell mn;\mu}^{(\Gamma)}
  + \sp{\tilde Q_{nm\ell;\mu}^{(\Gamma)}}^\dag.
\end{align}
Note that the sum $\ell+m+n$ for operators
$S_\mu^\ell \p{\mu S_\z}^m S_\bmu^n$ grows by one if $\Gamma_\mu\ne0$
or $\Gamma_\bmu\ne0$, and does not grow otherwise.


\section{Initial conditions}
\label{sec:initial_conditions}

Here we compute the expectation values of collective spin operators
with respect to spin-polarized (Gaussian) states.  These states are
parameterized by polar and azimuthal angles $\theta\in[0,\pi)$,
$\phi\in[0,2\pi)$, and lie within the Dicke manifold spanned by states
$\ket{k}\propto S_+^{S+k}\ket{\dn}^{\otimes N}$ with $S\equiv N/2$ and
$S_\z\ket{k}=k\ket{k}$:
\begin{align}
  \ket{\theta,\phi}
  \equiv \sp{\cos\p{\theta/2} e^{-i\phi/2} \ket\up
    + \sin\p{\theta/2} e^{i\phi/2} \ket\dn}^{\otimes N}
  = \sum_{k=-S}^S { N \choose S+k }^{1/2}
  \cos\p{\theta/2}^{S+k} \sin\p{\theta/2}^{S-k} e^{-ik\phi} \ket{k}.
\end{align}
We can likewise expand, within the Dicke manifold,
\begin{align}
  S_\z = \sum_{k=-S}^S k \op{k},
  &&
  S_\mu = \sum_{k=-S+\delta_{\mu,-1}}^{S-\delta_{\mu,1}}
  g_\mu\p{k} \op{k+\mu}{k}
  = \sum_{k=-S+\delta_{\bmu,-1}}^{S-\delta_{\bmu,1}}
  g_\bmu\p{k} \op{k}{k+\bmu},
\end{align}
where $\bmu\equiv-\mu\in\set{+1,-1}$ and
\begin{align}
  g_\mu\p{k} \equiv \sqrt{\p{S-\mu k}\p{S+\mu k+1}},
\end{align}
which implies
\begin{align}
  S_\mu^\ell \p{\mu S_\z^m} S_\bmu^n
  &= \sum_{k=-S+\delta_{\mu,-1}\max\set{\ell,n}}
  ^{S-\delta_{\mu,1}\max\set{\ell,n}} \p{\mu k}^m
  \sp{\prod_{p=0}^{\ell-1} g_\mu\p{k+\mu p}}
  \sp{\prod_{q=0}^{n-1} g_\mu\p{k+\mu q}}
  \op{k+\mu\ell}{k+\mu n} \\
  &= \sum_{\mu k=-\mu S-\delta_{\mu,-1}\max\set{\ell,n}}
  ^{\mu S-\delta_{\mu,1}\max\set{\ell,n}} \p{\mu k}^m
  \f{\p{S-\mu k}!}{\p{S+\mu k}!}
  \sp{\f{\p{S+\mu k+\ell}!}{\p{S-\mu k-\ell}!}
    \f{\p{S+\mu k+n}!}{\p{S-\mu k-n}!}}^{1/2}
  \op{k+\mu\ell}{k+\mu n} \\
  &= \sum_{k=-S}^{S-\max\set{\ell,n}} k^m
  \f{\p{S-k}!}{\p{S+k}!}
  \sp{\f{\p{S+k+\ell}!}{\p{S-k-\ell}!}
    \f{\p{S+k+n}!}{\p{S-k-n}!}}^{1/2}
  \op{\mu\p{k+\ell}}{\mu\p{k+n}}.
\end{align}
This expansion allows us to compute the expectation value
\begin{align}
  \bk{\theta,\phi|S_\mu^\ell \p{\mu S_\z^m} S_\bmu^n|\theta,\phi}
  &= e^{i\phi \mu\p{\ell-n}} N! \sum_{k=-S}^{S-\max\set{\ell,n}}
  \f{k^m \p{S-k}! f_{\mu\ell n}\p{k,\theta}}
  {\p{S+k}! \p{S-k-\ell}! \p{S-k-n}!} \\
  &= e^{i\phi \mu\p{\ell-n}} \p{-1}^m N! \sum_{k=0}^{N-\max\set{\ell,n}}
  \f{\p{S-k}^m \p{N-k}! \tilde f_{\mu\ell n}\p{k,\theta}}
  {k! \p{N-k-\ell}! \p{N-k-n}!}
\end{align}
where
\begin{align}
  f_{\mu\ell n}\p{k,\theta}
  \equiv \cos\p{\theta/2}^{N+\mu\p{2k+\ell+n}}
  \sin\p{\theta/2}^{N-\mu\p{2k+\ell+n}},
\end{align}
\begin{align}
  \tilde f_{\mu\ell n}\p{k,\theta}
  \equiv f_{\mu\ell n}\p{k-S,\theta}
  = \cos\p{\theta/2}^{2N\delta_{\mu,-1}+\mu\p{2k+\ell+n}}
  \sin\p{\theta/2}^{2N\delta_{\mu,1}-\mu\p{2k+\ell+n}}.
\end{align}
Defining the states
\begin{align}
  \ket{+\Z} \equiv \ket{0,0} = \ket\up^{\otimes N}, &&
  \ket{-\Z} \equiv \ket{\pi,0} = \ket\dn^{\otimes N}, &&
  \ket\X \equiv \ket{\pi/2,0}
  = \p{\f{\ket\up+\ket\dn}{\sqrt2}}^{\otimes N},
\end{align}
some particular expectation values of interest are
\begin{align}
  \bk{\nu\Z|S_\mu^\ell \p{\mu S_\z}^m S_\bmu^n|\nu\Z}
  = \delta_{\ell n} \times
  \begin{cases}
    \p{S-n}^m \f{N! n!}{\p{N-n}!} & \mu = \nu, \\
    \delta_{n,0} \p{-S}^m & \mu \ne \nu,
  \end{cases},
\end{align}
and
\begin{align}
  \bk{\X|S_\mu^\ell \p{\mu S_\z}^m S_\bmu^n|\X}
  = \p{-1}^m \f{N!}{2^N} \sum_{k=0}^{N-\max\set{\ell,n}}
  \f{\p{S-k}^m \p{N-k}!}{k!\p{N-k-\ell}!\p{N-k-n}!}.
\end{align}


\section{Short-time linear response and two-time correlators}
\label{sec:linear_response}

Here, we discuss the appearance of two-time correlation functions in
the short-time linear response of Heisenberg operators to
perturbations of a Hamiltonian.  Consider an initial Hamiltonian $H$
perturbed by an operator $V$ with $\norm{V}\ll\norm{H}$, where
$\norm{\O}$ denotes the operator norm of $\O$, such that the net
Hamiltonian is $\widetilde H=H+V$.  We denote the generator of
Heisenberg time evolution under the perturbed (unperturbed)
Hamiltonian by $\widetilde T$ ($T$).  These generators are related by
\begin{align}
  \widetilde T = T + i\hat V
\end{align}
where $\hat V$ is a superoperator whose action on operators $\O$ is
defined by
\begin{align}
  \hat V \O \equiv \sp{V,\O}_-.
\end{align}
Through quadratic order in the time $t$ and linear order in the
perturbation $\hat V$, we can say that
\begin{align}
  e^{t\widetilde T}
  \approx \f12\sp{e^{tT}, e^{it\hat V}}_+
  \approx e^{tT} + \f12 it \sp{e^{t T}, \hat V}_+.
\end{align}
Defining perturbed and unperturbed Heisenberg operators
$\widetilde\O\p{t}\equiv e^{t\widetilde T}\O$ and
$\O\p{t}\equiv e^{tT}\O$, we thus find that for sufficiently small
times $t$ and weak perturbations $V$,
\begin{align}
  \widetilde\O\p{t} - \O\p{t}
  = \p{e^{t\widetilde T} - e^{tT}} \O
  \approx \f12 i t \p{\sp{V,\O}_-\p{t} + \sp{V,\O\p{t}}_-}.
  \label{eq:response}
\end{align}
Two-time operators $V\O\p{t}$ and $\O\p{t} V$, in addition to the
Heisenberg operators $\p{V\O}\p{t}$ and $\p{\O V}\p{t}$, thus
determine the short-time linear response of operators $\O\p{t}$ to
perturbations $V$ of a Hamiltonian.


% \section{Changing operator bases, and a general product for algebras
%   of permutationally-symmetric operators}
% \label{sec:new_basis}

% For a collective spin system composed of many small spins, the
% operators $\S_{\v m}\equiv S_+^{m_+} S_\z^{m_\z} S_-^{m_-}$ contain
% $\abs{\v m}$-body operators (for $\abs{\v m}\equiv\sum_\mu m_\mu$),
% but also carry lots of ``baggage'' in the form of $k$-body operators
% with $k<\abs{\v m}$.  To cut down on this overhead, we try to change
% our basis for collective spin operators from $\S_{\v m}$ to the
% purely-$\abs{\v m}$-body operators
% \begin{align}
%   \tilde\S_{\v m} \equiv \sum_{\p{\v j;\v m}} P_{\v j},
%   &&
%   P_{\v j} \equiv \prod_{\mu,a} s_\mu^{\p{j^\mu_a}},
%   \label{eq:S_P}
% \end{align}
% where $\v m \equiv \p{m_+,m_\z,m_-}\in\mathbb{N}_0^3$ with $m_\mu$ the
% number of $s_\mu$ operators in each term of $\S_{\v m}$; $\v j$ is an
% ordered list of indices for the spins addressed by $P_{\v j}$; and the
% sum over $\p{\v j;\v m}$ denotes a sum over all possible $\v j$ for
% which all indices in $\v j$ are distinct and
% $\abs{\set{j^\mu_a}} = m_\mu$, such that $j^\mu_a$ indexes the $a$-th
% spin addressed by an $s_\mu$ operator in $P_{\v j}$.

% In order to make use of the operators $\tilde S_{\v m}$, we will need
% to compute products of the form $\tilde S_{\v m} \tilde S_{\v n}$.  We
% therefore expand such a product into terms that have exactly $s$ spins
% addressed by both operators:
% \begin{align}
%   \tilde\S_{\v m} \tilde\S_{\v n}
%   = \sum_{s\ge0} \sum_{\p{\v j,\v k;\v m,\v n,s}} P_{\v j} P_{\v k},
%   &&
%   \sum_{\p{\v j,\v k;\v m,\v n,s}} X \equiv
%   \sum_{\substack{\p{\v j;\v m},\p{\v k;\v n} \\
%       \abs{\set{j_\alpha^\mu}\cap\set{k_\beta^\nu}}=s}} X.
%   \label{eq:SS_PP}
% \end{align}
% Collecting terms in which $r_{\mu\nu}$ of the $s_\mu$ operators in
% $P_{\v j}$ address the same spin as an $s_\nu$ operator in $P_{\v k}$,
% we have
% \begin{align}
%   \tilde\S_{\v m} \tilde\S_{\v n}
%   = \sum_{s\ge0} \sum_{\p{\v r;\v m,\v n,s}} f_{\v m\v n\v r}
%   \sum_{\p{\v J,\v K;\v m,\v n,\v r}} P_{\v J} Q_{\v K},
%   \label{eq:SS_PQ}
% \end{align}
% where the sum over $\p{\v r;\v m,\v n,s}$ denotes a sum over all
% values of $\v r$ with the restrictions
% \begin{align}
%   r_{\mu\nu} \ge 0 ~\forall~ \mu,\nu,
%   &&
%   \sum_\nu r_{\mu\nu} \le m_\mu,
%   &&
%   \sum_\mu r_{\mu\nu} \le n_\nu,
%   &&
%   \sum_{\mu,\nu} r_{\mu\nu} = s;
%   \label{eq:rest_r}
% \end{align}
% the factor $f_{\v m\v n\v r}$ counts the number of ways to pair
% operators in $P_{\v j}$ and $P_{\v k}$ according to $\v r$, or
% \begin{align}
%   f_{\v m\v n\v r}
%   &\equiv \sp{\prod_{\mu,\nu}
%     { m_\mu - \sum_{\rho<\nu} r_{\mu\rho} \choose r_{\mu\nu} }
%     { n_\mu - \sum_{\rho<\nu} r_{\rho\mu} \choose r_{\nu\mu} }}
%   \sp{\prod_{\mu,\nu} r_{\mu\nu}!} \\
%   &= \sp{\prod_\mu \f{m_\mu!}{\p{m_\mu-\sum_\rho r_{\mu\rho}}!}
%     \f{n_\mu!}{\p{n_\mu-\sum_\rho r_{\rho\mu}}!}}
%   \sp{\prod_{\mu,\nu} r_{\mu\nu}!}^{-1};
% \end{align}
% $\v J$ and $\v K$ are respectively ordered lists of indices for spins
% addressed by one and two single-spin operators in a term of the
% product $P_{\v j} P_{\v k}$ with fixed $\v r$; the sum over
% $\p{\v J,\v K;\v m,\v n,\v r}$ denotes a sum over all values of
% $\v J,\v K$ with the restrictions
% \begin{align}
%   \abs{\set{J_a^\mu}}
%   = m_\mu + n_\mu - \sum_\rho\p{r_{\mu\rho}+r_{\rho\mu}},
%   &&
%   \abs{\set{K^{\mu\nu}_a}} = r_{\mu\nu},
%   &&
%   \set{J_a^\lambda} \cap \set{K_b^{\mu\nu}} = \varnothing,
% \end{align}
% with $\varnothing$ denoting the empty set; and finally, similarly to
% $P_{\v j}$ in \eqref{eq:S_P} we define
% \begin{align}
%   Q_{\v K} \equiv \prod_{\mu,\nu,a}
%   s_\mu^{(K^{\mu\nu}_a)} s_\nu^{(K^{\mu\nu}_a)}.
% \end{align}
% As the sum in \eqref{eq:SS_PQ} is invariant under permutation of the
% indices in $\v K$, we can safely neglect keeping track of these
% indices and simply write
% \begin{align}
%   Q_{\v K}
%   = \prod_{\mu,\nu} \prod_{a=1}^{r_{\mu\nu}}
%   s_\mu^{(K^{\mu\nu}_a)} s_\nu^{(K^{\mu\nu}_a)}
%   \simeq \bigotimes_{\mu,\nu} \bigotimes_{a=1}^{r_{\mu\nu}} s_\mu s_\nu
%   = \bigotimes_{\mu,\nu} \bigotimes_{a=1}^{r_{\mu\nu}}
%   \sum_\rho \eta_{\mu\nu\rho} s_\rho,
%   \label{eq:Q_K_eta}
% \end{align}
% where $\simeq$ denotes equality up to a re-indexing of spins and
% additional tensor factors of the single-spin identity operator
% (i.e.~$\1$); we have introduced explicit dependence on
% $r_{\mu\nu}=\abs{\set{K^{\mu\nu}_a}}$ for brevity; and
% $\eta_{\mu\nu\rho}$ is a structure constant defined by
% $s_\mu s_\nu=\sum_\rho\eta_{\mu\nu\rho}s_\rho$.  Unlike the sums over
% $\mu,\nu$ in most of the work above, the sum over $\rho$ here includes
% an index for the identity operator $s_0\equiv\1$.  Distributing the
% product of sums into a sum of products gives us
% \begin{align}
%   Q_{\v K}
%   \simeq \sum_{\v\rho} \bigotimes_{\mu,\nu,a}
%   \eta_{\mu\nu\rho^{\mu\nu}_a} s_{\rho^{\mu\nu}_a}
%   = \sum_{\v\rho} \sp{\prod_{\mu,\nu,a}\eta_{\mu\nu\rho^{\mu\nu}_a}}
%   \bigotimes_{\mu,\nu,a} s_{\rho^{\mu\nu}_a},
%   \label{eq:Q_K_rho}
% \end{align}
% where implicitly $\abs{\set{\rho^{\mu\nu}_a}}=r_{\mu\nu}$, as
% $\p{\mu,\nu,a}$ essentially index a factor in \eqref{eq:Q_K_eta} and
% $\rho^{\mu\nu}_a$ indexes one of the terms in that factor.  Letting
% $\tilde\rho^{\mu\nu}_\kappa$ denote the number of elements in the
% index list
% $\p{\rho^{\mu\nu}_1,\rho^{\mu\nu}_2,\cdots,\rho^{\mu\nu}_{r_{\mu\nu}}}$
% that are equal to $\kappa$, we observe that two terms in
% \eqref{eq:Q_K_rho} with, say, $\v\rho=\v\rho_1$ and $\v\rho=\v\rho_2$
% are equal up to a permutation of indices if
% $\tilde{\v\rho}_1=\tilde{\v\rho}_2$.  The degeneracy (under
% permutation) of terms in \eqref{eq:Q_K_rho}, i.e.~the number of
% $\v\rho$ that are consistent with $\tilde{\v\rho}$, is
% \begin{align}
%   g_{\tilde{\v\rho}}
%   \equiv \prod_{\mu,\nu,\kappa}
%   { r_{\mu\nu} - \sum_{\lambda<\kappa} \tilde\rho^{\mu\nu}_\lambda
%     \choose \tilde\rho^{\mu\nu}_\kappa }
%   = \sp{\prod_{\mu,\nu} r_{\mu\nu}!}
%   \sp{\prod_{\mu,\nu,\kappa} \tilde\rho^{\mu\nu}_\kappa!}^{-1}.
% \end{align}
% Defining additionally
% \begin{align}
%   \v\eta^{\tilde{\v\rho}}
%   \equiv \prod_{\mu,\nu,\kappa}
%   \p{\eta_{\mu\nu\kappa}}^{\tilde\rho^{\mu\nu}_\kappa},
%   \label{eq:eta_rho}
% \end{align}
% we can collect like factors and group together equivalent terms in
% \eqref{eq:Q_K_rho} to write
% \begin{align}
%   Q_{\v K}
%   \simeq \sum_{\tilde{\v\rho}} g_{\tilde{\v\rho}} \v\eta^{\tilde{\v\rho}}
%   \bigotimes_{\kappa}
%   s_\kappa^{\otimes\sum_{\mu,\nu}\tilde\rho^{\mu\nu}_\kappa}
%   \simeq \sum_{\tilde{\v\rho}} g_{\tilde{\v\rho}} \v\eta^{\tilde{\v\rho}}
%   \1^{\otimes\sum_{\mu,\nu}\tilde\rho^{\mu\nu}_0}
%   \otimes \bigotimes_{\kappa\ne0}
%   s_\kappa^{\otimes\sum_{\mu,\nu}\tilde\rho^{\mu\nu}_\kappa},
%   \label{eq:Q_K_rho_simp}
% \end{align}
% where again implicitly
% $\sum_\kappa\tilde\rho^{\mu\nu}_\kappa=r_{\mu\nu}$ for consistency
% with \eqref{eq:Q_K_eta} and \eqref{eq:Q_K_rho}, and we have explicitly
% factored out the identity operators in each term of $Q_{\v K}$.

% We have now simplified $Q_{\v K}$ sufficiently to substitute it back
% into \eqref{eq:SS_PQ}, which (removing the tilde from $\tilde{\v\rho}$
% to simplify notation) gives us
% \begin{align}
%   \tilde\S_{\v m} \tilde\S_{\v n}
%   = \sum_{s\ge0} \sum_{\p{\v r;\v m,\v n,s}} f_{\v m\v n\v r}
%   \sum_{\p{\v\rho;\v r}} g_{\v\rho} \v\eta^{\v\rho}
%   c_{\v\ell_{\v m\v n\v r\v\rho}\v\rho}
%   \tilde\S_{\v\ell_{\v m\v n\v r\v\rho}},
%   \label{eq:SS_simp}
% \end{align}
% where the sum over $\p{\v\rho;\v r}$ denotes a sum over all values of
% $\v\rho$ and with the restrictions
% \begin{align}
%   \abs{\set{\rho^{\mu\nu}_\kappa}} = \abs{\set{s_\kappa}},
%   &&
%   \sum_{\kappa} \rho^{\mu\nu}_\kappa = r_{\mu\nu};
% \end{align}
% the combinatorial factor
% \begin{align}
%   c_{\v\ell\v\rho}
%   \equiv \f{\p{N - \abs{\v\ell}}!}
%   {\p{N - \abs{\v\ell} - \sum_{\mu,\nu}\rho^{\mu\nu}_0}!}
% \end{align}
% accounts for the number of terms that are equivalent up to a
% permutation of indices involving the identity operators in
% \eqref{eq:Q_K_rho_simp}; and the number of $s_\mu$ operators in each
% term of $\tilde\S_{\v\ell_{\v m\v n\v r\v\rho}}$ is
% \begin{align}
%   \ell_{\v m\v n\v r\v\rho;\mu}
%   = m_\mu + n_\mu - \sum_\kappa \p{r_{\mu\kappa}+r_{\kappa\mu}}
%   + \sum_{\alpha,\beta} \rho^{\alpha\beta}_\mu.
% \end{align}

% The sums in \eqref{eq:SS_simp} generally involve a large number of
% terms; in practice, many of these terms will be equal to zero.  We
% can, however, impose additional restrictions on these sums in such a
% way as to throw out all terms that are zero, and keep only those that
% are not.  Remembering that $s$ counts the number of spins that are
% addressed by both operators $P_{\v j}$ and $P_{\v k}$ in
% \eqref{eq:SS_PP}, the first restriction we can impose comes from
% recognizing that for given $\tilde\S_{\v m},\tilde\S_{\v n}$, the
% overlap $s$ is bounded as
% \begin{align}
%   \max\set{0,\abs{\v m}+\abs{\v n}-N}
%   \le s \le \min\set{N,\abs{\v m}+\abs{\v n}}.
% \end{align}
% In addition, the matrix of structure constants $\eta_{\mu\nu\kappa}$
% may generally contain zeros that we can preemptively avoid with
% restrictions on $\v r$ and $\v\rho$.  Recalling that $r_{\mu\nu}$
% counts the number of operators addressed by
% $s_\mu s_\nu=\sum_\kappa\eta_{\mu\nu\kappa} s_\kappa$, if
% $\eta_{\mu\nu\kappa}=0$ for all $\kappa$, i.e.~$s_\mu s_\nu=0$, then
% any term with $r_{\mu\nu}>0$ will vanish.  Our second restriction on
% the sums in \eqref{eq:SS_simp} is therefore to fix $r_{\mu\nu}=0$ for
% all $\mu,\nu$ with $s_\mu s_\nu=0$.  Finally, we can avoid the
% nullifying effect of zeros from the factor
% $\v\eta^{\v\rho}=\prod_{\mu,\nu,\kappa}
% \p{\eta_{\mu\nu\kappa}}^{\rho^{\mu\nu}_\kappa}$ by fixing
% $\rho^{\mu\nu}_\kappa=0$ for all $\mu,\nu,\kappa$ with
% $\eta_{\mu\nu\kappa}=0$.  These restrictions are sufficient to ensure
% that all individual terms in \eqref{eq:SS_simp} are nonzero, although
% they do not rule out the possibility of several terms canceling out.

% TODO: comment on the use of this new basis vs.~the old one.


\bibliography{\jobname}

\end{document}
