\documentclass[preprint,superscriptaddress]{revtex4-2}

% linking references
\usepackage{hyperref}
\hypersetup{
  breaklinks=true,
  colorlinks=true,
  linkcolor=blue,
  urlcolor=cyan,
}

%%% symbols, notations, etc.
\usepackage{physics,braket,bm,commath,amssymb}
\renewcommand{\t}{\text} % text in math mode
\newcommand{\f}[2]{\dfrac{#1}{#2}} % shorthand for fractions
\newcommand{\p}[1]{\left( #1 \right)} % parenthesis
\renewcommand{\sp}[1]{\left[ #1 \right]} % square parenthesis
\renewcommand{\v}{\bm} % bold vectors
\newcommand{\uv}[1]{\bm{\hat{#1}}} % unit vectors

\renewcommand{\abs}[1]{\lvert #1 \rvert}

\newcommand{\bk}{\braket} % shorthand for braket notation
\newcommand{\Bk}{\Braket}

\newcommand{\ul}{\underline} % shorthand for underline

\newcommand{\A}{\mathcal{A}}
\newcommand{\B}{\mathcal{B}}
\newcommand{\D}{\mathcal{D}}
\newcommand{\E}{\mathcal{E}}
\newcommand{\J}{\mathcal{J}}
\renewcommand{\O}{\mathcal{O}}
\newcommand{\Q}{\mathcal{Q}}
\renewcommand{\S}{\mathcal{S}}
\newcommand{\U}{\mathcal{U}}

\newcommand{\C}{\mathbb{C}}
\newcommand{\N}{\mathbb{N}}

\newcommand{\z}{\text{z}}
\newcommand{\x}{\text{x}}
\newcommand{\y}{\text{y}}
\newcommand{\Z}{\text{Z}}
\newcommand{\X}{\text{X}}
\newcommand{\Y}{\text{Y}}
\newcommand{\bmu}{{\bar\mu}}
\newcommand{\bnu}{{\bar\nu}}

\usepackage{dsfont} % for identity operator
\newcommand{\1}{\mathds{1}}

\newcommand{\up}{\uparrow}
\newcommand{\dn}{\downarrow}

\renewcommand{\a}{\alpha} % free index
\renewcommand{\b}{\beta} % free index

%%% figures
\usepackage{graphicx} % for figures
\usepackage[caption=false]{subfig} % subfigures (via \subfloat[]{})
\newcommand{\sref}[1]{\protect\subref{#1}} % for referencing subfigures
\graphicspath{{./figures/}} % set path for all figures

% remove extra white space inside enumerate environments
\usepackage[inline]{enumitem}
\setenumerate{label=(\roman*)}
\setlist[enumerate]{leftmargin=*}

\setlength{\parindent}{0pt}
\setlength{\parskip}{1ex}

% for colored text
\usepackage[dvipsnames]{xcolor}
\newcommand{\blue}[1]{{\color{blue} #1}}
\newcommand{\red}[1]{{\color{red} #1}}
\newcommand{\green}[1]{{\color{ForestGreen} #1}}

%%%%%%%%%%%%%%%%%%%%%%%%%%%%%%%%%%%%%%%%%%%%%%%%%%%%%%%%%%%%%%%%%%%%%%
\begin{document}

Dear Editors,

Please find attached our revised manuscript AJ11834, titled
``Short-time expansion of Heisenberg operators in open collective
quantum spin systems'', for consideration for publication in Physical
Review A.  Below, we address the questions and comments made by our
referees, and detail the corresponding changes that we have made to
our manuscript.  We hope that our responses and changes address the
referees' questions to a satisfactory degree, leaving no more
outstanding questions or concerns about our work.

Excerpts from the referee reports are written in \blue{blue}, excerpts
from the previous version of our manuscript are written in \red{red},
and excerpts from the revised version of our manuscript are written in
\green{green}.  Unless explicitly stated otherwise, all equation
numbers, page numbers, etc.~refer to those in the revised manuscript.

%%%%%%%%%%%%%%%%%%%%%%%%%%%%%%%%%%%%%%%%%%%%%%%%%%%%%%%%%%%%%%%%%%%%%%
\section{First referee}

\begin{enumerate}
\item Concerning the first referee's summary and comments:

  \blue{The authors have studied open collective spin systems with
    generic Markovian decoherence and presented a new method to
    calculate short-time correlators. The method is based on an
    expansion of Heisenberg operators and truncation of it for short
    times (TST). The number of particles that the presented method can
    deal with are shown to be rather high. The method is meant to work
    well with highly-correlated systems. Spin squeezing, two-time
    correlation functions, and out-of-time-ordered correlators are
    calculated as an application of the new method. As technical parts
    of the paper, authors calculate all the ingredients, e.g., the
    structure constants of a collective spin operator algebra, which
    are necessary to apply the method.}

  \blue{Although the developed method works only at short times, it
    can be fruitful for applications in the field of quantum
    metrology. One of the key points that makes the paper interesting
    is the capability of the TST expansion to compute the spin
    correlators in a regime (strong correlations) that other methods
    cannot practically access. In the regime that it works, the method
    is efficient and very promising for future applications.}

  \blue{Overall, the paper is well written and the presented method
    and following results deserve publication in Phys.~Rev.~A. In
    particular, I expect it to be utilized in the spin squeezed
    systems. I just have the following question:}

  We thank the referee for their summary of our work, and for the
  conclusion that our results deserve publication in Phys.~Rev.~A.  We
  similarly hope that our method finds useful applications in spin
  squeezed systems, as well as others.  We address the referee's
  question below.


\item Concerning the first referee's question:

  \blue{In the last paragraph of page 4, which continues to page 5,
    authors explain how unphysical result can be marked by and
    unphysical squeezing parameter. In the same paragraph authors list
    three possible mechanisms for breakdown. Is there a fundamental
    relationship between all of the three breakdown mechanisms and the
    value of the squeezing parameter? Asked differently, could it be
    possible that one of the three breakdown mechanisms happens
    without the squeezing parameter going unphysical?}

  The referee brings up a good question about diagnosing the breakdown
  of our method (TST).  In fact, there is {\it no} fundamental
  relationship between the breakdown of TST and an unphysical
  squeezing parameter ($\xi^2<0$), and it {\it is} possible for
  breakdown to occur without the squeezing parameter going unphysical.
  Instead, there is a direct relationship between the breakdown of TST
  and the value of individual correlators $\bk{\hat S_{\v n}}$ that
  are used to compute the squeezing parameter $\xi^2$.  The breakdown
  of TST affects (non-negligible) contributions to correlators
  $\bk{\hat\S_{\v n}}$ that grow with the time $t$ raised to some
  large power, so after breakdown these correlators rapidly take on
  unphysical values.  It is often the case that unphysical correlators
  will lead to an unphysical squeezing parameter, but not always.  The
  breakdown of TST should therefore be diagnosed by inspecting the
  correlators used to compute the squeezing parameter, rather than by
  inspecting the squeezing parameter itself.

  In order to clarify the diagnosis of breakdown for the TST
  expansion, we have rearranged and added text to the paragraph in
  question, replacing (end of page 4 through some of page 5 in the old
  text):

  \red{The main lesson from Figure 1 is that the TST expansion yields
    essentially exact results right up until a sudden and drastic
    departure that can be diagnosed by inspection.  {\bf The breakdown
      of the TST expansion in Figure 1 is marked by an unphysical
      squeezing parameter $\xi^2<0$, which occurs due to individual
      correlators taking unphysical values with
      $\abs{\bk{\hat\S_{\v n}\p{t}}}\gtrsim S^{\abs{\v n}}$.  Going up
      through order $M=70$ in the TST expansion does not significantly
      increase the breakdown time $t_{\t{break}}^{(M)}$, and in some
      cases even shortens $t_{\t{break}}^{(M)}$.}  The sudden and
    drastic departure from virtually exact results is consistent with
    the limitations of the TST expansion discussed at the end of
    Section II.  Specifically, we identify three possible mechanisms
    for breakdown:
    \begin{enumerate*}
    \item a rapid growth in the order $M$ necessary for the TST
      expansion to converge,
    \item the growth of numerical errors in excessively large terms of
      the TST expansion, and
    \item the formal breakdown of the perturbative expansion in the
      time $t$.
    \end{enumerate*}
    In all of these cases, a detailed cancellation eventually ceases
    to occur between large terms at high orders in the TST expansion.
    These large terms grow with the time $t$ raised to some large
    power (as high as $M$), and therefore rapidly yield wildly
    unphysical results.  In contrast to other approximate methods such
    as the cumulant expansion[58], the TST expansion can thus diagnose
    its own breakdown, which is an important feature when working in
    parameter regimes that are inaccessible via other means to compute
    correlators.}

  by (top of page 5 in the new text):

  \green{The main lesson from Figure 1 is that the TST expansion
    yields essentially exact results right up until a sudden and
    drastic departure that can be diagnosed by inspection.  {\bf The
      breakdown of the TST expansion in Figure 1 induces an unphysical
      squeezing parameter $\xi^2<0$.  In general, however, there is no
      fundamental relationship between the breakdown of the TST
      expansion and the conditions for a physical squeezing parameter
      $\xi^2$.  A proper diagnosis of breakdown therefore requires
      inspection of the correlators $\bk{\hat\S_{\v n}\p{t}}$ used to
      compute the squeezing parameter $\xi^2$, which upon breakdown
      will rapidly take unphysical values with
      $\abs{\bk{\hat\S_{\v n}\p{t}}}\gtrsim S^{\abs{\v n}}$ (see
      Appendix L for an example).}  The sudden and drastic departure
    from virtually exact results is consistent with the limitations of
    the TST expansion discussed at the end of Section II.
    Specifically, we identify three possible mechanisms for breakdown:
    \begin{enumerate*}
    \item a rapid growth in the order $M$ necessary for the TST
      expansion to converge,
    \item the growth of numerical errors in excessively large terms of
      the TST expansion, and
    \item the formal breakdown of the perturbative expansion in the
      time $t$.
    \end{enumerate*}
    In all of these cases, a detailed cancellation eventually ceases
    to occur between large terms at high orders in the TST expansion.
    These large terms grow with the time $t$ raised to some large
    power (as high as $M$), and therefore rapidly yield wildly
    unphysical results.  In contrast to other approximate methods such
    as the cumulant expansion[58], the TST expansion can thus diagnose
    its own breakdown, which is an important feature when working in
    parameter regimes that are inaccessible via other means to compute
    correlators.  {\bf Note that, due to the breakdown mechanisms of
      the TST expansion, going up through order $M=70$ does not
      significantly increase the breakdown time $t_{\t{break}}^{(M)}$
      in Figure 1, and in some cases even shortens
      $t_{\t{break}}^{(M)}$.}}

  Note that we used bold text to help identify the changes above; the
  manuscript itself does not make these changes bold.  We have also
  added a short new appendix (L) to provide an example of breakdown
  without an unphysical squeezing parameter.  We hope that these
  changes clarify subtleties related to the referee's question.

\end{enumerate}

%%%%%%%%%%%%%%%%%%%%%%%%%%%%%%%%%%%%%%%%%%%%%%%%%%%%%%%%%%%%%%%%%%%%%%
\section{Second referee}

\begin{enumerate}
\item Concerning the second referee's summary and comments:

  \blue{In this work, the authors compute the dynamics of collective
    spin systems with decoherence via uncorrelated decay of individual
    spins by solving the operators’ Heisenberg’s equation of motion
    approximately.  It is carried out in the following way: Heisenberg
    operators are formally expanded into a Taylor series, which is
    then truncated to a finite order for short evolution times. The
    authors benchmark the method by computing spin squeezing, two-time
    correlation functions, and out-of-time-ordered correlators.

    The paper is clearly written, and the derivations appear to be
    sound.  However, there are some points need to be clarified before
    this paper can be accepted for publication.}

  We thank the referee for their succinct summary of our method, and
  for the the recognition that our manuscript is well written, and our
  derivations sound.  We address the refere's points below, and hope
  that the corresponding changes to our manuscript deem our work
  acceptable for publication in Physical Review A.


\item Concerning the first point, about novelty:

  \blue{1. The authors declare that they present a ‘new’
    method. However, it seems that the method as described resembles
    (if not the same as) the well-known Mori formalism, or memory
    function method, developed by Mori, Berne, Lee, Grigolini and
    others [see for example H.~Mori, Prog.~Theor.~Phys.~34 399
    (1965); James F Annett, et al.~Condens.~Matter 6, 645(1994)],
    where the Liouvillian superoperator is used to provide a compact
    representation of the time evolution of operators under
    Heisenberg’s equation of motion, and followed up with a continued
    fraction expansion. I hope the authors could acknowledge these
    previous studies and be careful with the claiming of ‘new’
    method.}

  We thank the referee for drawing our attention to the Mori formalism
  (\url{https://doi.org/10/cgtkbb}) and the related work by Annett et
  al.~(\url{https://doi.org/10/d79ngr}).  Indeed, there are some
  interesting connections between these works and ours, as pointed out
  by the referee.  For example, all of these methods rely on some sort
  of formal solution to the equations of motion for Heisenberg
  operators.  Nonetheless, we strongly believe that our methods are
  substantively distinct from those of these past works, and expand on
  some differences below.  In acknowledgment of the resemblance and
  connections to past works, however, we have also followed the
  referee's suggestion and changed all instances of ``\red{a new
    method}'' in our manuscript to ``\green{an efficient method}''.
  We have also added references to the Mori formalism and follow-up
  works (outlined below).

  At a high level, the truncated short-time (TST) expansion for a
  Heisenberg operator, Eq.~(10) in our manuscript, essentially
  involves only products of (i) initial-time expectation values of
  operators chosen from a fixed basis, and (ii) matrix elements of the
  time derivative operator $T$ (essentially the Liouvillian), as well
  as powers of $T$.  In order to evaluate the TST expansion, our
  method makes use of the structure constants of a collective spin
  operator algebra.  The Mori formalism, meanwhile, arrives at an
  expansion of Heisenberg operators (see Eq.~2.29 of
  \url{https://doi.org/10/cgtkbb}) that involves (i) Laplace
  transforms, (ii) operators, $f_j$, that are constructed recursively
  from the initial Heisenberg operator of interest, and (iii)
  coefficients, $C_j$, that are defined using the operators $f_j$ and
  additional Laplace transforms.  In order to evaluate this expansion,
  the Mori formalism provides a continued-fraction representation of
  the coefficients $C_j$.  Perhaps additional work might illuminate
  the connections between these two methods, but at face value they
  are quite different.

  Some of the work by Annett et al.~in \url{https://doi.org/10/d79ngr}
  has a closer resemblance to ours, in that Eq.~(25) therein is,
  similar to the TST expansion, a series expansion organized in powers
  of the time $t$.  Even here, however, Eq.~(25) concerns only one
  component of an (initially localized) Heisenberg operator $Q(t)$,
  namely the projection of $Q(t)$ onto $Q(0)$.  Furthermore, the
  solution of Eq.~(25) involves resolving $Q(t)$ into frequency
  components $R(\omega)$, representing the resolvents $R(\omega)$ by a
  continued fraction with some free parameters $\set{a_n,b_n}$, and
  then solving for the parameters $\set{a_n,b_n}$.  Our TST expansion
  considers explicitly non-local operators, solves for the entirety of
  these operators (as opposed to only one component), and involves
  neither frequency components nor continued fractions.  We hope that
  these considerations clarify how the TST expansion is distinct from
  past works.

  Despite these differences, we acknowledge that there may be some
  interesting connections between the above works and ours, and that
  investigating these connections may lead to a refinement of our
  work.  We have therefore changed the following sentence in the
  introduction (top left of page 2):

  \red{Our method is based on a short-time expansion of exact
    solutions to the equations of motion for Heisenberg operators.
    Evaluating this expansion requires ...}

  to:

  \green{Our method is based on a formal solution to the equations of
    motion for Heisenberg operators, thereby bearing some resemblance
    to the Mori formalism[59] and related work[60].  Specifically, we
    expand a formal solution for a Heisenberg operator into a Taylor
    series whose truncation can yield negligible error at sufficiently
    short times.  Evaluating the resulting expansion requires ...}

  We have also added a reference to these works to the discussion of
  the limitations of the TST expansion at the end of the theory
  section of our manuscript (right column of page 3), changing the
  text:

  \red{Precisely characterizing the implications of these last two
    considerations for the TST expansion requires additional analysis
    that we defer to future work.  As we show from benchmarks of the
    TST expansion in Section III, however, a detailed understanding of
    these limitations is not necessary ...}

  to:

  \green{Precisely characterizing the implications of these last two
    considerations for the TST expansion requires additional analysis
    that we defer to future work.  {\bf An investigation of
      connections between the TST expansion and past work related to
      the Mori formalism[59, 60], for example, might answer questions
      about the breakdown and convergence of the TST expansion.}  As
    we show from benchmarks of the TST expansion in Section III,
    however, a detailed understanding of {\bf breakdown} is not
    necessary ...}

  We hope that the referee considers these additions a sufficient
  acknowledgment of the Mori formalism and related work.


\item Concerning the second question, about decoherence:

  \blue{2. I am not sure if the paper properly takes into account
    previous results about the influence of decoherence on spin
    squeezing. First, there exists many mechanisms for decoherence,
    while decoherence considered in this paper is due to uncorrelated
    decay of individual spins. This is not clarified in the
    introduction. In fact, atom loss represents an important mechanism
    of decoherence in spin squeezing experiments, but is not
    considered in this paper. Could the method studied here also be
    applied to atom loss? It would be helpful if this point is
    mentioned. Second, for the specific decoherence model considered,
    the authors only cite a few studies about the influence of such
    decoherence on spin squeezing from one-axis twisting (OAT). There
    are many studies about the effect of decoherence on spin
    squeezing.}

  The referee makes a fair point that we claim to present a method for
  simulating ``generic'' Markovian decoherence, when in fact we only
  address certain (typical) forms of Markovian decoherence.  To avoid
  misleading readers about the contents of our paper, we have
  therefore replaced the words ``\red{generic Markovian decoherence}''
  by ``\green{typical forms of Markovian decoherence}'' in the
  abstract, as well as the last paragraph of the introduction.

  We would like to point out that our work {\it does} address more
  than just uncorrelated decay of individual spins: we also address
  uncorrelated excitation and dephasing of individual spins, as well
  as arbitrary coherent or incoherent combinations of these processes.
  Similarly, we have done all of the work necessary to account for
  arbitrary combinations of {\it collective} decay, excitation, and
  dephasing in Appendices D-G, as mentioned in the paragraph below
  Eq.~(4) of our manuscript.

  The referee is correct, though, that our original manuscript does
  not discuss atom loss, which is an important decoherence mechanism
  in some experimental realizations of collective spin models.  We
  have therefore added the following discussion about atom loss to the
  theory section our manuscript (bottom right of page 2):

  \green{We note that particle loss is an important decoherence
    mechanism in many experimental realizations of collective spin
    models[41].  In principle, a spin model has no notion of the
    particle annihilation operators that generate particle loss, and
    therefore cannot capture this effect directly.  Nonetheless, for a
    system initially composed of $N$ particles, the effect of particle
    loss can be emulated with $O(1/N)$ error by the dissipator
    $\check\D_{\t{loss}}$ defined by
    $\check\D_{\t{loss}}\hat\S_{\v m}=-\abs{\v m}\hat\S_{\v m}$, where
    $\abs{\v m}\equiv\sum_\alpha m_\alpha$ (see Appendix H).
    Furthermore, the effect particle loss can be accounted for exactly
    by (i) introducing an additional index on spin operators,
    $\hat\S_{\v m}\to\hat\S_{N\v m}$, to keep track of different
    sectors of fixed particle number within a multi-particle Fock
    space, and (ii) constructing jump operators that appropriately
    couple spin operators within different particle-number sectors.
    We defer a detailed exact accounting of particle loss to future
    work.}

  The new appendix provides mathematical justification for our claim
  that the dissipator $\check\D_{\t{loss}}$ emulates particle loss
  with $O(1/N)$ error.  We hope that these additions do a satisfactory
  job of addressing particle loss.

  As a last point, concerning the number of references in our work to
  past studies about the influence of decoherence on spin squeezing in
  the one-axis twisting (OAT) model: the small number of such
  references is due to the fact that we do not have any particular
  discussions of the influence of decoherence on spin squeezing with
  OAT, and do not consider such discussions relevant to our current
  work.  Even with decoherence (of the types primarily considered in
  our work), the OAT model is an exactly analytically solvable (see
  \url{https://doi.org/gcsjdp}).  Approximate numerical methods such
  as that presented in our manuscript are therefore inappropriate
  tools for the study of OAT.  The primary role of OAT in our work is
  to serve as a simple and exactly solvable benchmark of our numerical
  methods, rather than an object of study in and of itself.  To
  clarify this point, we have changed the second paragraph of section
  III on spin squeezing, benchmarking, and breakdown (top of page 4),
  from:

  \red{Note that the OAT model is a special case of the zero-field
    Ising model, whose quantum dynamics admits an exact analytic
    solution even in the presence of decoherence[62].  We will
    benchmark our calculations using these analytical OAT results
    wherever applicable (see Appendix J, as well as the Supplementary
    Material of Ref.~[14]).}

  to:

  \green{Note that the OAT model is a special case of the zero-field
    Ising model, whose quantum dynamics admits an exact analytic
    solution even in the presence of decoherence[64].  The approximate
    and numerics-oriented TST expansion is therefore an inappropriate
    tool for studying the OAT model, which will merely serve as an
    exactly solvable benchmark of our methods.  Wherever applicable,
    we will provide exact results for the OAT model (see Appendix K,
    as well as the Supplementary Material of Ref.~[14]).}


\item Concerning the final question, about squeezing with strong
  decoherence:

  \blue{3. I am surprised by the results from the strong decoherence
    regimes in Fig.~2, which shows that twist and-turn (TNT) model is
    much more robust than OAT and TAT. Is this physical or just an
    artifact from numerical errors. To rule out the latter, I suggest
    the authors check the results in the strong decoherence regime by
    studying a smaller system, where Monte Carlo simulation is
    accessible and can be used as a comparison. If this result is
    indeed physical, if would be very helpful if the authors could
    comment on the reason?}

  We thank the referee for their careful attention to our results.  We
  agree that our findings warrant a comment on why TNT performs better
  than OAT or TAT in the presence of decoherence.  We have therefore
  rewritten some of the last paragraph of section III (end of page 5,
  beginning of page 6), replacing:

  \red{The results in Figure 2 show that TNT is much more robust to
    single-spin decoherence than OAT or TAT, a finding that could not
    necessarily be deduced from the weak-decoherence simulations
    presented in Figure 1.  Strong-decoherence computations of this
    sort were used to put lower bounds...}

  by:

  \green{The results in Figure 2 show that the TNT model can generate
    more squeezing than the OAT or TAT models in the presence of
    strong decoherence.  The better performance of TNT is in part a
    consequence of the fact that TNT initially generates squeezing at
    a faster rate than OAT or TAT, thereby allowing it to produce more
    squeezing before the degrading effects of decoherence kick in.  We
    corroborate this finding with quantum trajectory simulations of a
    small system in Appendix M.  Strong-decoherence computations of
    the sort used for Figure 2 put lower bounds...}

  The new appendix includes Monte Carlo simulations of a small system
  with sufficiently strong decoherence to reproduce the improved
  performance of TNT.  We hope that these additions address the
  referee's questions about figure 2 to a satisfactory degree.

\end{enumerate}

%%%%%%%%%%%%%%%%%%%%%%%%%%%%%%%%%%%%%%%%%%%%%%%%%%%%%%%%%%%%%%%%%%%%%%
\section{Miscellaneous changes}

\begin{enumerate}
\item In our discussions about the breakdown of the TST expansion in
  Sections II and III, we use the word ``unphysical'' to essentially
  mean ``not physically possible''.  In Section I, meanwhile, we
  comment about a method of simulation called the cumulant expansion,
  which breaks down in the presence of genuinely multi-body
  correlations.  In such a case, the cumulant expansion yields results
  that are {\it physically possible}, but {\it incorrect}.  For
  consistent use of language, we have therefore changed the following
  sentence in section I (bottom right of page 1):

  \red{The growth of genuinely multi-body correlations, however,
    eventually causes the cumulant expansion to yield unphysical
    results with no clear signature of failure.}

  to:

  \green{The growth of genuinely multi-body correlations, however,
    eventually causes the cumulant expansion to yield incorrect
    results with no clear signature of failure.}


\item One of our previous changes to the manuscript defined
  ``\green{$\abs{\v m}\equiv\sum_\alpha m_\alpha$}'' (bottom of page 2
  in the new text).  We have therefore removed a subsequent definition
  of this notation, ``\red{where
    $\abs{\v\ell}\equiv\sum_\alpha\ell_\alpha$}'', which previously
  appeared in the following paragraph (middle right column of page 2
  in the old text).

\end{enumerate}

\end{document}
