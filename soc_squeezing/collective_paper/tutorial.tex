\section{Computing correlators with the truncated short-time (TST)
  expansion}
\label{sec:tutorial}

Here we provide a pedagogical tutorial for computing correlators using
the truncated short-time TST expansion.  For concreteness, we
nominally consider $N$ spins evolving under the one-axis twisting
(OAT) Hamiltonian
\begin{align}
  H_{\t{OAT}} = \chi S_\z^2,
\end{align}
additionally subject to spontaneous single-spin decay at rate
$\gamma_-$, with jump operators $\J_-=\set{s_-^{(j)}:j=1,2,\cdots,N}$.
The equation of motion for a Heisenberg operator
$\p{S_+^\ell S_\z^m S_-^n}\p{t}$ is
\begin{align}
  \f{d}{dt} \Bk{S_+^\ell S_\z^m S_-^n}
  = i\chi\Bk{\sp{S_\z^2, S_+^\ell S_\z^m S_-^n}_-}
  + \gamma_- \Bk{\D\p{\J_-}\p{S_+^\ell S_\z^m S_-^n}},
  \label{eq:tutorial_EOM}
\end{align}
where we have suppressed the explicit time dependence of operators for
brevity.  Using the results in appendices \ref{sec:general_product}
and \ref{sec:decay_single} respectively to evaluate the commutator
$\sp{S_\z^2, S_+^\ell S_\z^m S_-^n}_-$ and dissipator
$\D\p{\J_-} \p{S_+^\ell S_\z^m S_-^n}$ in \eqref{eq:tutorial_EOM}, we
can expand
\begin{multline}
  \f{d}{dt} \bk{S_+^\ell S_\z^m S_-^n} \\
  = i\chi \bk{\p{\ell-n} S_+^\ell \p{\ell+n+2S_\z} S_\z^m S_-^n}
  + \gamma_- \Bk{S_+^\ell \sp{\p{S+S_\z}\p{-1+S_\z}^m
      - \p{S+\f{\ell+n}{2}+S_\z} S_\z^m} S_-^n}.
  \label{eq:tutorial_EOM_expanded}
\end{multline}
In practice, we do not want to keep track of such an expansion by
hand, especially in the case of e.g.~the two-axis twisting (TAT) and
twist-and-turn (TNT) models with more general types of decoherence,
for which the analogue of \eqref{eq:tutorial_EOM_expanded} may take
several lines just to write out in full.  Defining the operators
$\S_{\v m}\equiv S_+^{m_+} S_\z^{m_\z} S_-^{m_-}$ with
$\v m\equiv\p{m_+,m_\z,m_-}$ for shorthand, we note that the vector
space spanned by $\set{\S_{\v m}}$ is closed under time evolution.  We
therefore expand
\begin{align}
  \f{d}{dt} \bk{\S_{\v n}}
  = \bk{T \S_{\v n}}
  = \sum_{\v m} \bk{\S_{\v m}} T_{\v m\v n},
  \label{eq:tutorial_EOM_general}
\end{align}
where $T$ is a superoperator that generates time evolution for
Heisenberg operators.  In the present example, the matrix elements
$T_{\v m\v n}\in\C$ of $T$ are defined by
\eqref{eq:tutorial_EOM_expanded} and \eqref{eq:tutorial_EOM_general}.
For any Hamiltonian $H$ with decoherence characterized by sets of jump
operators $\J$ and decoherence rates $\gamma_\J$, the matrix elements
$T_{\v m\v n}$ are more generally defined by
\begin{align}
  T \S_{\v n}
  = i\sp{H,\S_{\v n}}_-
  + \sum_\J \gamma_\J \D\p{\J} \S_{\v n}
  = \sum_{\v m} \S_{\v m} T_{\v m\v n}.
\end{align}
The results in Appendices \ref{sec:general_product},
\ref{sec:decoherence_single}, and \ref{sec:decoherence_collective} can
be used to write model-agnostic codes that compute matrix elements
$T_{\v m\v n}$, taking a particular Hamiltonian $H$ and decoherence
processes $\set{\p{\J,\gamma_\J}}$ as inputs.

In order to compute a quantity such as spin squeezing, we need to
compute correlators of the form $\bk{\S_{\v n}\p{t}}$, where for
clarity we will re-introduce the explicit time dependence of
Heisenberg operators $\S_{\v n}\p{t}$.  The order-$M$ truncated
short-time (TST) expansion takes
\begin{align}
  \bk{\S_{\v n}\p{t}}
  = \bk{e^{tT}\S_{\v n}\p{0}}
  = \sum_{k\ge0} \f{t^k}{k!} \bk{T^k\S_{\v n}\p{0}}
  = \sum_{k\ge0} \f{t^k}{k!}
  \sum_{\v m} \bk{\S_{\v m}\p{0}} T^k_{\v m\v n}
  \to \sum_{k=0}^M \f{t^k}{k!}
  \sum_{\v m} \bk{\S_{\v m}\p{0}} T^k_{\v m\v n},
  \label{eq:tutorial_TST}
\end{align}
where $T^k_{\v m\v n}$ are matrix elements of the $k$-th time
derivative operator $T^k$, given by
\begin{align}
  T^0_{\v m\v n} \equiv
  \begin{cases}
    1 & \v m = \v n, \\
    0 & \t{otherwise}
  \end{cases},
  &&
  T^1_{\v m\v n} \equiv T_{\v m\v n},
  &&
  T^{k>1}_{\v m\v n}
  \equiv \sum_{\v p_1,\v p_2,\cdots,\v p_{k-1}}
  T_{\v m\v p_{k-1}} \cdots T_{\v p_3\v p_2}
  T_{\v p_2\v p_1} T_{\v p_1\v n}.
\end{align}
Matrix elements $T^k_{\v m\v n}$ and initial-time expectation values
$\bk{\S_{\v m}\p{0}}$ are thus computed as needed for any particular
correlator $\bk{\S_{\v n}\p{t}}$ of interest, and combined according
to \eqref{eq:tutorial_TST}.  Note that initial-time expectation values
$\bk{\S_{\v m}\p{0}}$ are an {\it input} to the TST expansion, and
need to be computed separately for any initial state of interest;
expectation values with respect to spin-polarized (Gaussian) states
are provided in Appendix \ref{sec:initial_conditions}.  In practice,
we further collect terms in \eqref{eq:tutorial_TST} to write
\begin{align}
  \bk{\S_{\v n}\p{t}} \to \sum_{k=0}^M c_{\v n k} t^k,
  &&
  c_{\v n k}
  \equiv \f1{k!} \sum_{\v m} \bk{\S_{\v m}\p{0}} T^k_{\v m\v n},
  \label{eq:tutorial_corr}
\end{align}
where $c_{\v n k}$ are time-independent coefficients for the expansion
of $\bk{\S_{\v n}\p{t}}$.  After computing the coefficients
$c_{\v n k}$, there is only negligible computational overhead to
compute the correlator $\bk{\S_{\v n}\p{t}}$ for any time $t$.
