\section{Changing operator bases, and a general product for algebras
  of permutationally-symmetric operators}
\label{sec:new_basis}

For a collective spin system composed of many small spins, the
operators $\S_{\v m}\equiv S_+^{m_+} S_\z^{m_\z} S_-^{m_-}$ contain
$\abs{\v m}$-body operators (for $\abs{\v m}\equiv\sum_\mu m_\mu$),
but also carry lots of ``baggage'' in the form of $k$-body operators
with $k<\abs{\v m}$.  To cut down on this overhead, we try to change
our basis for collective spin operators from $\S_{\v m}$ to the
purely-$\abs{\v m}$-body operators
\begin{align}
  \tilde\S_{\v m} \equiv \sum_{\p{\v j;\v m}} P_{\v j},
  &&
  P_{\v j} \equiv \prod_{\mu,a} s_\mu^{\p{j^\mu_a}},
  \label{eq:S_P}
\end{align}
where $\v m \equiv \p{m_+,m_\z,m_-}\in\mathbb{N}_0^3$ with $m_\mu$ the
number of $s_\mu$ operators in each term of $\S_{\v m}$; $\v j$ is an
ordered list of indices for the spins addressed by $P_{\v j}$; and the
sum over $\p{\v j;\v m}$ denotes a sum over all possible $\v j$ for
which all indices in $\v j$ are distinct and
$\abs{\set{j^\mu_a}} = m_\mu$, such that $j^\mu_a$ indexes the $a$-th
spin addressed by an $s_\mu$ operator in $P_{\v j}$.

In order to make use of the operators $\tilde S_{\v m}$, we will need
to compute products of the form $\tilde S_{\v m} \tilde S_{\v n}$.  We
therefore expand such a product into terms that have exactly $s$ spins
addressed by both operators:
\begin{align}
  \tilde\S_{\v m} \tilde\S_{\v n}
  = \sum_{s\ge0} \sum_{\p{\v j,\v k;\v m,\v n,s}} P_{\v j} P_{\v k},
  &&
  \sum_{\p{\v j,\v k;\v m,\v n,s}} X \equiv
  \sum_{\substack{\p{\v j;\v m},\p{\v k;\v n} \\
      \abs{\set{j_\a^\mu}\cap\set{k_\b^\nu}}=s}} X.
  \label{eq:SS_PP}
\end{align}
Collecting terms in which $r_{\mu\nu}$ of the $s_\mu$ operators in
$P_{\v j}$ address the same spin as an $s_\nu$ operator in $P_{\v k}$,
we have
\begin{align}
  \tilde\S_{\v m} \tilde\S_{\v n}
  = \sum_{s\ge0} \sum_{\p{\v r;\v m,\v n,s}} f_{\v m\v n\v r}
  \sum_{\p{\v J,\v K;\v m,\v n,\v r}} P_{\v J} Q_{\v K},
  \label{eq:SS_PQ}
\end{align}
where the sum over $\p{\v r;\v m,\v n,s}$ denotes a sum over all
values of $\v r$ with the restrictions
\begin{align}
  r_{\mu\nu} \ge 0 ~\forall~ \mu,\nu,
  &&
  \sum_\nu r_{\mu\nu} \le m_\mu,
  &&
  \sum_\mu r_{\mu\nu} \le n_\nu,
  &&
  \sum_{\mu,\nu} r_{\mu\nu} = s;
  \label{eq:rest_r}
\end{align}
the factor $f_{\v m\v n\v r}$ counts the number of ways to pair
operators in $P_{\v j}$ and $P_{\v k}$ according to $\v r$, or
\begin{align}
  f_{\v m\v n\v r}
  &\equiv \sp{\prod_{\mu,\nu}
    { m_\mu - \sum_{\rho<\nu} r_{\mu\rho} \choose r_{\mu\nu} }
    { n_\mu - \sum_{\rho<\nu} r_{\rho\mu} \choose r_{\nu\mu} }}
  \sp{\prod_{\mu,\nu} r_{\mu\nu}!} \\
  &= \sp{\prod_\mu \f{m_\mu!}{\p{m_\mu-\sum_\rho r_{\mu\rho}}!}
    \f{n_\mu!}{\p{n_\mu-\sum_\rho r_{\rho\mu}}!}}
  \sp{\prod_{\mu,\nu} r_{\mu\nu}!}^{-1};
\end{align}
$\v J$ and $\v K$ are respectively ordered lists of indices for spins
addressed by one and two single-spin operators in a term of the
product $P_{\v j} P_{\v k}$ with fixed $\v r$; the sum over
$\p{\v J,\v K;\v m,\v n,\v r}$ denotes a sum over all values of
$\v J,\v K$ with the restrictions
\begin{align}
  \abs{\set{J_a^\mu}}
  = m_\mu + n_\mu - \sum_\rho\p{r_{\mu\rho}+r_{\rho\mu}},
  &&
  \abs{\set{K^{\mu\nu}_a}} = r_{\mu\nu},
  &&
  \set{J_a^\lambda} \cap \set{K_b^{\mu\nu}} = \varnothing,
\end{align}
with $\varnothing$ denoting the empty set; and finally, similarly to
$P_{\v j}$ in \eqref{eq:S_P} we define
\begin{align}
  Q_{\v K} \equiv \prod_{\mu,\nu,a}
  s_\mu^{(K^{\mu\nu}_a)} s_\nu^{(K^{\mu\nu}_a)}.
\end{align}
As the sum in \eqref{eq:SS_PQ} is invariant under permutation of the
indices in $\v K$, we can safely neglect keeping track of these
indices and simply write
\begin{align}
  Q_{\v K}
  = \prod_{\mu,\nu} \prod_{a=1}^{r_{\mu\nu}}
  s_\mu^{(K^{\mu\nu}_a)} s_\nu^{(K^{\mu\nu}_a)}
  \simeq \bigotimes_{\mu,\nu} \bigotimes_{a=1}^{r_{\mu\nu}} s_\mu s_\nu
  = \bigotimes_{\mu,\nu} \bigotimes_{a=1}^{r_{\mu\nu}}
  \sum_\rho \eta_{\mu\nu\rho} s_\rho,
  \label{eq:Q_K_eta}
\end{align}
where $\simeq$ denotes equality up to a re-indexing of spins and
additional tensor factors of the single-spin identity operator
(i.e.~$\1$); we have introduced explicit dependence on
$r_{\mu\nu}=\abs{\set{K^{\mu\nu}_a}}$ for brevity; and
$\eta_{\mu\nu\rho}$ is a structure constant defined by
$s_\mu s_\nu=\sum_\rho\eta_{\mu\nu\rho}s_\rho$.  Unlike the sums over
$\mu,\nu$ in most of the work above, the sum over $\rho$ here includes
an index for the identity operator $s_0\equiv\1$.  Distributing the
product of sums into a sum of products gives us
\begin{align}
  Q_{\v K}
  \simeq \sum_{\v\rho} \bigotimes_{\mu,\nu,a}
  \eta_{\mu\nu\rho^{\mu\nu}_a} s_{\rho^{\mu\nu}_a}
  = \sum_{\v\rho} \sp{\prod_{\mu,\nu,a}\eta_{\mu\nu\rho^{\mu\nu}_a}}
  \bigotimes_{\mu,\nu,a} s_{\rho^{\mu\nu}_a},
  \label{eq:Q_K_rho}
\end{align}
where implicitly $\abs{\set{\rho^{\mu\nu}_a}}=r_{\mu\nu}$, as
$\p{\mu,\nu,a}$ essentially index a factor in \eqref{eq:Q_K_eta} and
$\rho^{\mu\nu}_a$ indexes one of the terms in that factor.  Letting
$\tilde\rho^{\mu\nu}_\kappa$ denote the number of elements in the
index list
$\p{\rho^{\mu\nu}_1,\rho^{\mu\nu}_2,\cdots,\rho^{\mu\nu}_{r_{\mu\nu}}}$
that are equal to $\kappa$, we observe that two terms in
\eqref{eq:Q_K_rho} with, say, $\v\rho=\v\rho_1$ and $\v\rho=\v\rho_2$
are equal up to a permutation of indices if
$\tilde{\v\rho}_1=\tilde{\v\rho}_2$.  The degeneracy (under
permutation) of terms in \eqref{eq:Q_K_rho}, i.e.~the number of
$\v\rho$ that are consistent with $\tilde{\v\rho}$, is
\begin{align}
  g_{\tilde{\v\rho}}
  \equiv \prod_{\mu,\nu,\kappa}
  { r_{\mu\nu} - \sum_{\lambda<\kappa} \tilde\rho^{\mu\nu}_\lambda
    \choose \tilde\rho^{\mu\nu}_\kappa }
  = \sp{\prod_{\mu,\nu} r_{\mu\nu}!}
  \sp{\prod_{\mu,\nu,\kappa} \tilde\rho^{\mu\nu}_\kappa!}^{-1}.
\end{align}
Defining additionally
\begin{align}
  \v\eta^{\tilde{\v\rho}}
  \equiv \prod_{\mu,\nu,\kappa}
  \p{\eta_{\mu\nu\kappa}}^{\tilde\rho^{\mu\nu}_\kappa},
  \label{eq:eta_rho}
\end{align}
we can collect like factors and group together equivalent terms in
\eqref{eq:Q_K_rho} to write
\begin{align}
  Q_{\v K}
  \simeq \sum_{\tilde{\v\rho}} g_{\tilde{\v\rho}} \v\eta^{\tilde{\v\rho}}
  \bigotimes_{\kappa}
  s_\kappa^{\otimes\sum_{\mu,\nu}\tilde\rho^{\mu\nu}_\kappa}
  \simeq \sum_{\tilde{\v\rho}} g_{\tilde{\v\rho}} \v\eta^{\tilde{\v\rho}}
  \1^{\otimes\sum_{\mu,\nu}\tilde\rho^{\mu\nu}_0}
  \otimes \bigotimes_{\kappa\ne0}
  s_\kappa^{\otimes\sum_{\mu,\nu}\tilde\rho^{\mu\nu}_\kappa},
  \label{eq:Q_K_rho_simp}
\end{align}
where again implicitly
$\sum_\kappa\tilde\rho^{\mu\nu}_\kappa=r_{\mu\nu}$ for consistency
with \eqref{eq:Q_K_eta} and \eqref{eq:Q_K_rho}, and we have explicitly
factored out the identity operators in each term of $Q_{\v K}$.

We have now simplified $Q_{\v K}$ sufficiently to substitute it back
into \eqref{eq:SS_PQ}, which (removing the tilde from $\tilde{\v\rho}$
to simplify notation) gives us
\begin{align}
  \tilde\S_{\v m} \tilde\S_{\v n}
  = \sum_{s\ge0} \sum_{\p{\v r;\v m,\v n,s}} f_{\v m\v n\v r}
  \sum_{\p{\v\rho;\v r}} g_{\v\rho} \v\eta^{\v\rho}
  c_{\v\ell_{\v m\v n\v r\v\rho}\v\rho}
  \tilde\S_{\v\ell_{\v m\v n\v r\v\rho}},
  \label{eq:SS_simp}
\end{align}
where the sum over $\p{\v\rho;\v r}$ denotes a sum over all values of
$\v\rho$ and with the restrictions
\begin{align}
  \abs{\set{\rho^{\mu\nu}_\kappa}} = \abs{\set{s_\kappa}},
  &&
  \sum_{\kappa} \rho^{\mu\nu}_\kappa = r_{\mu\nu};
\end{align}
the combinatorial factor
\begin{align}
  c_{\v\ell\v\rho}
  \equiv \f{\p{N - \abs{\v\ell}}!}
  {\p{N - \abs{\v\ell} - \sum_{\mu,\nu}\rho^{\mu\nu}_0}!}
\end{align}
accounts for the number of terms that are equivalent up to a
permutation of indices involving the identity operators in
\eqref{eq:Q_K_rho_simp}; and the number of $s_\mu$ operators in each
term of $\tilde\S_{\v\ell_{\v m\v n\v r\v\rho}}$ is
\begin{align}
  \ell_{\v m\v n\v r\v\rho;\mu}
  = m_\mu + n_\mu - \sum_\kappa \p{r_{\mu\kappa}+r_{\kappa\mu}}
  + \sum_{\a,\b} \rho^{\a\b}_\mu.
\end{align}

The sums in \eqref{eq:SS_simp} generally involve a large number of
terms; in practice, many of these terms will be equal to zero.  We
can, however, impose additional restrictions on these sums in such a
way as to throw out all terms that are zero, and keep only those that
are not.  Remembering that $s$ counts the number of spins that are
addressed by both operators $P_{\v j}$ and $P_{\v k}$ in
\eqref{eq:SS_PP}, the first restriction we can impose comes from
recognizing that for given $\tilde\S_{\v m},\tilde\S_{\v n}$, the
overlap $s$ is bounded as
\begin{align}
  \max\set{0,\abs{\v m}+\abs{\v n}-N}
  \le s \le \min\set{N,\abs{\v m}+\abs{\v n}}.
\end{align}
In addition, the matrix of structure constants $\eta_{\mu\nu\kappa}$
may generally contain zeros that we can preemptively avoid with
restrictions on $\v r$ and $\v\rho$.  Recalling that $r_{\mu\nu}$
counts the number of operators addressed by
$s_\mu s_\nu=\sum_\kappa\eta_{\mu\nu\kappa} s_\kappa$, if
$\eta_{\mu\nu\kappa}=0$ for all $\kappa$, i.e.~$s_\mu s_\nu=0$, then
any term with $r_{\mu\nu}>0$ will vanish.  Our second restriction on
the sums in \eqref{eq:SS_simp} is therefore to fix $r_{\mu\nu}=0$ for
all $\mu,\nu$ with $s_\mu s_\nu=0$.  Finally, we can avoid the
nullifying effect of zeros from the factor
$\v\eta^{\v\rho}=\prod_{\mu,\nu,\kappa}
\p{\eta_{\mu\nu\kappa}}^{\rho^{\mu\nu}_\kappa}$ by fixing
$\rho^{\mu\nu}_\kappa=0$ for all $\mu,\nu,\kappa$ with
$\eta_{\mu\nu\kappa}=0$.  These restrictions are sufficient to ensure
that all individual terms in \eqref{eq:SS_simp} are nonzero, although
they do not rule out the possibility of several terms canceling out.

% TODO: comment on the use of this new basis vs.~the old one.
