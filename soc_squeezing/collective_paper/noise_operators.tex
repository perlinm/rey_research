\section{Heisenberg operators in open quantum systems}
\label{sec:noise}

Here we explain the origin and character of the mean-zero ``noise''
operators $\E_\O\p{t}$ that appear in the expansion of a Heisenberg
operator $\O\p{t}=\sum_{\v m}\O_{\v m}\p{t}\S_{\v m}+\E_\O\p{t}$ with
time-dependent coefficients $\O_{\v m}\p{t}$ for time-independent
(Schr\"odinger) operators $\hat\S_{\v m}$.  Our discussion clarify why
noise operators play no role in our calculation of expectation values
of the form $\bk{\O\p{t}}$ and $\bk{\O\Q\p{t}}$, despite the fact that
noise operators generally {\it do} need to be considered in the
calculation of more general multi-time correlators in open quantum
systems\cite{blocher2019quantum}.

In any closed quantum system with initial state $\rho$ and propagator
$U\p{t}$, such that the state at time $t$ is
$\rho\p{t}\equiv U\p{t}\rho U^\dag\p{t}$, time-dependent Heisenberg
operators $\O\p{t}$ are uniquely defined from time-independent
Schr\"odinger operators $\O$ by
\begin{align}
  \bk{\O\p{t}} \equiv \tr\sp{\rho\p{t}\O} = \tr\sp{\rho\O\p{t}}.
  \label{eq:heisenberg_condition}
\end{align}
Enforcing \eqref{eq:heisenberg_condition} for {\it arbitrary} initial
states $\rho$ forces $\O\p{t}=U^\dag\p{t}\O U\p{t}$.  In an open
quantum system, however, the definition of a Heisenberg operator is
not so straightforward.  Open systems can often be understood as
subsystems of a larger closed system.  Consider therefore an open
system $S$ with environment $E$, a joint initial state $\rho_{SE}$,
and propagator $U_{SE}\p{t}$.  The reduced state $\rho_S\p{t}$ of $S$
at time $t$ is
\begin{align}
  \rho_S\p{t}
  \equiv \tr_E\sp{\rho_{SE}\p{t}}
  = \tr_E\sp{U_{SE}\p{t} \rho_{SE} U_{SE}^\dag\p{t}}
  \equiv \U_{\ul S}\p{t} \rho_S,
  \label{eq:state_S}
\end{align}
where $\rho_S\equiv\rho_S\p{0}$ is a time-independent state of $S$ in
the Heisenberg picture, $\ul S$ denotes the space of operators on $S$,
and the quantum channel $\U_{\ul S}\p{t}$ has the
decomposition\cite{rivas2012time}
\begin{align}
  \U_{\ul S}\p{t} \rho_S
  = \sum_j \U_S^{(j)}\p{t} \rho_S {\U_S^{(j)}}^\dag\p{t}
\end{align}
with ordinary operators $\U_S^{(j)}\p{t}$ on $S$.  We can therefore
expand
\begin{align}
  \bk{\O_S\p{t}}
  = \tr\sp{\rho_S\p{t}\O_S}
  = \tr\sp{\U_{\ul S}\p{t} \rho_S \O_S}
  = \tr\sp{\rho_S \U_{\ul S}^\dag\p{t} \O_S}
  = \tr\sp{\rho_S \ul{\O_S}\p{t}}
  = \bk{\ul{\O_S}\p{t}},
  \label{eq:heisenberg_correlator}
\end{align}
where we define the operator
\begin{align}
  \ul{\O_S}\p{t}
  \equiv \U_{\ul S}^\dag\p{t} \O_S
  = \sum_j {\U_S^{(j)}}^\dag\p{t} \O_S \U_S^{(j)}\p{t}.
  \label{eq:heisenberg_wrong}
\end{align}
We thus find that substituting $\ul{\O_S}\p{t}$ in place of
$\O_S\p{t}$ suffices for the calculation of correlators
$\bk{\O_S\p{t}}$, thereby accounting for the validity of the equation
of motion in \eqref{eq:EOM}.  As we show below, this substitution also
suffices for the calculation of two-time correlators of the form
$\bk{\O_S\Q_S\p{t}}$ when the environment $E$ is Markovian.

The problem with {\it defining} Heisenberg operators $\O_S\p{t}$ by
$\ul{\O_S}\p{t}$ only becomes evident when considering products of
Heisenberg operators.  One would like for the product of two
Heisenberg operators $\O_S\p{t}$ and $\Q_S\p{t}$ to satisfy
$\O_S\p{t}\Q_S\p{t}=\p{\O_S\Q_S}\p{t}$.  This intuition can be
formalized by observing that
\begin{align}
  \bk{\O_S\p{t}}
  = \tr\sp{\rho_{SE}\p{t} \p{\O_S\otimes\1_E}}
  = \tr\sp{\rho_{SE} \p{\O_S\otimes\1_E}\p{t}}
  = \bk{\p{\O_S\otimes\1_E}\p{t}},
  \label{eq:heisenberg_expansion}
\end{align}
where $\1_E$ is the identity operator on $E$, expectation values of
Heisenberg operators on system $A\in\set{S,E,SE}$ are taken with
respect to the state $\rho_A$, and
\begin{align}
  \p{\O_S\otimes\1_E}\p{t}
  \equiv U_{SE}^\dag\p{t} \p{\O_S\otimes\1_E} U_{SE}\p{t}.
  \label{eq:heisenberg_extension}
\end{align}
By expanding Heisenberg operators similarly to
\eqref{eq:heisenberg_expansion} and \eqref{eq:heisenberg_extension},
we then find
\begin{align}
  \bk{\O_S\p{t}\Q_S\p{t}}
  = \bk{\p{\O_S\otimes\1_E}\p{t} \p{\Q_S\otimes\1_E}\p{t}}
  = \bk{\p{\O_S\Q_S\otimes\1_E}\p{t}}
  = \bk{\p{\O_S\Q_S}\p{t}}.
\end{align}
The expression in \eqref{eq:heisenberg_wrong}, however, makes it clear
that generally
$\ul{\O_S}\p{t} \ul{\Q_S}\p{t}\ne\p{\ul{\O_S\Q_S}}\p{t}$.  To correct
for this discrepancy, we define
\begin{align}
  \O_S\p{t} \equiv \ul{\O_S}\p{t} + \E_{\O_S}\p{t}
  \label{eq:noise_def}
\end{align}
in terms of new ``noise'' operators $\E_{\O_S}\p{t}$ that are
essentially defined to enforce the consistency of operator products
such as $\O_S\p{t}\Q_S\p{t}=\p{\O_S\Q_S}\p{t}$. Self-consistency
forces noise operators to be mean-zero, as
\begin{align}
  \bk{\E_{\O_S}\p{t}} = \bk{\O_S\p{t}} - \bk{\ul{\O_S}\p{t}} = 0.
\end{align}
Furthermore, if the environment $E$ is Markovian, then noise operators
are also uncorrelated with initial-time observables,
i.e.~$\bk{\O_S\E_{\Q_S}\p{t}}=0$, which means that noise operators can
be neglected in the calculation of two-time correlators of the form
$\bk{\O_S\Q_S\p{t}}$.  To see why, we observe that a Markovian
environment is essentially defined to satisfy
\begin{align}
  \rho_{SE}\p{t} = U_{SE}\p{t} \rho_{SE} U_{SE}^\dag\p{t}
  \approx \rho_S\p{t}\otimes\rho_E
  = \U_{\ul S}\p{t} \rho_S\otimes\rho_E,
  \label{eq:markov}
\end{align}
with $\rho_E$ a time-independent steady state of the environment.
Enforcing \eqref{eq:markov} for all states $\rho_S$ implies that
\begin{align}
  U_{SE}\p{t} \rho_{SE} \p{\O_S\otimes\1_E} U_{SE}^\dag\p{t}
  \approx \U_{\ul S}\p{t}\p{\rho_S\O_S}\otimes\rho_E.
  \label{eq:makov_op}
\end{align}
We can therefore expand
\begin{align}
  \bk{\O_S\Q_S\p{t}}
  &= \tr\sp{\rho_{SE} \p{\O_S\otimes\1_E}
    U_{SE}^\dag\p{t} \p{\Q_S\otimes\1_E} U_{SE}\p{t}} \\
  &= \tr\sp{U_{SE}\p{t} \rho_{SE} \p{\O_S\otimes\1_E}
    U_{SE}^\dag\p{t} \p{\Q_S\otimes\1_E}},
\end{align}
and invoke the Markov approximation in \eqref{eq:makov_op} to find
that
\begin{align}
  \bk{\O_S\Q_S\p{t}}
  \approx \tr\sp{\U_{\ul S}\p{t}\p{\rho_S\O_S} \Q_S}
  = \tr\sp{\rho_S \O_S \U_{\ul S}^\dag\p{t} \Q_S}
  = \bk{\O_S\ul{\Q_S}\p{t}},
\end{align}
which implies
\begin{align}
  \bk{\O_S\E_{\Q_S}\p{t}}
  = \bk{\O_S\Q_S\p{t}} - \bk{\O_S\ul{\Q_S}\p{t}}
  \approx 0.
\end{align}
Noise operators thus play no role in the calculation of correlators
such as $C\p{t}$ in \eqref{eq:two_time}.  In contrast, noise operators
generally {\it do} play a role in the calculation of multi-time
correlators of the form
$\bk{\prod_j\O_S^{(j)}\p{t_j}}$\cite{blocher2019quantum}.
Furthermore, these calculations generally require additional
assumptions about the environment.  To keep our discussion simple and
general, we therefore exclude the effects of noise terms in Section
\ref{sec:multi_time}.
