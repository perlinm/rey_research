\documentclass[aps,11pt,notitlepage,nofootinbib]{revtex4-1}

% linking references
\usepackage{hyperref}
\hypersetup{
  breaklinks=true,
  colorlinks=true,
  linkcolor=blue,
  filecolor=magenta,
  urlcolor=cyan,
}

%%% symbols, notations, etc.
\usepackage{physics,braket,bm,commath,amssymb}
\renewcommand{\t}{\text} % text in math mode
\newcommand{\f}[2]{\dfrac{#1}{#2}} % shorthand for fractions
\newcommand{\p}[1]{\left(#1\right)} % parenthesis
\renewcommand{\sp}[1]{\left[#1\right]} % square parenthesis
\renewcommand{\set}[1]{\left\{#1\right\}} % curly parenthesis
\renewcommand{\v}{\bm} % bold vectors
\newcommand{\uv}[1]{\v{\hat{#1}}} % unit vectors
\renewcommand{\c}{\cdot} % inner product
\newcommand{\bk}{\Braket} % shorthand for braket notation

\newcommand{\C}{\mathcal{C}}
\newcommand{\D}{\mathcal{D}}
\newcommand{\F}{\mathcal{F}}
\newcommand{\I}{\mathcal{I}}
\newcommand{\J}{\mathcal{J}}
\renewcommand{\O}{\mathcal{O}}
\renewcommand{\S}{\mathcal{S}}

\newcommand{\N}{\mathbb{N}}

\newcommand{\z}{\text{z}}
\newcommand{\x}{\text{x}}
\newcommand{\y}{\text{y}}
\newcommand{\Z}{\text{Z}}
\newcommand{\X}{\text{X}}
\newcommand{\Y}{\text{Y}}
\newcommand{\bmu}{{\bar\mu}}
\newcommand{\bnu}{{\bar\nu}}

\usepackage{dsfont} % for identity operator
\newcommand{\1}{\mathds{1}}

\usepackage[inline]{enumitem} % for inline enumeration

%%% figures
\usepackage{graphicx} % for figures
\usepackage{grffile} % help latex properly identify figure extensions
\usepackage[caption=false]{subfig} % subfigures (via \subfloat[]{})
\graphicspath{{./figures/}} % set path for all figures

% for strikeout text
% normalem included to prevent underlining titles in the bibliography
\usepackage[normalem]{ulem}

% for leaving notes in the text
\newcommand{\note}[1]{\textcolor{red}{#1}}


\begin{document}

\title{Short-time simulation of open collective quantum spin systems}

\author{Michael A. Perlin}
\email{mika.perlin@gmail.com}
\author{Ana Maria Rey}
\affiliation{JILA, National Institute of Standards and Technology and
  University of Colorado, 440 UCB, Boulder, Colorado 80309, USA}
\affiliation{Center for Theory of Quantum Matter, 440 UCB, Boulder,
  Colorado 80309, USA}
\affiliation{Department of Physics, University of Colorado, 390 UCB,
  Boulder, Colorado 80309, USA}

\begin{abstract}
  Abstract goes here.
\end{abstract}

\maketitle


\section{Introduction}

Collective spin systems attract considerable attention for providing a
window into many-body quantum phenomena that are both experimentally
accessible\cite{fernholz2008spin, takano2009spin,
  riedel2010atomchipbased, gross2010nonlinear, martin2013quantum,
  bohnet2016quantum, hosten2016quantum, norcia2018cavitymediated} and
metrologically useful\cite{wineland1992spin, kitagawa1993squeezed,
  ma2011quantum}.  Theoretical interest in such systems dates back to
the mid-twentieth century with the introduction of the
Lipkin-Meshkov-Glick (LMG) model as a toy model for testing many-body
approximation methods in contemporary nuclear
physics\cite{lipkin1965validity, meshkov1965validity,
  glick1965validity}.  Since then, collective system systems have
additionally featured in theoretical studies of quantum
criticality\cite{latorre2005entanglement, alcalde2007functional,
  morrison2008collective, sarandy2009classical, wang2012quantum,
  majd2014lmg}, non-equilibrium
phenomena\cite{morrison2008dissipationdriven,
  bhattacherjee2014nonequilibrium, klinder2015dynamical,
  zhiqiang2017nonequilibrium, smale2018observation}, and most notably
spin squeezing\cite{wineland1992spin, kitagawa1993squeezed,
  ma2011quantum}.  With the development of advanced atomic trapping,
cooling, and control techniques, collective spin systems have been
experimentally realized in atomic ensembles\cite{fernholz2008spin,
  takano2009spin}, Bose-Einstein
condensates\cite{riedel2010atomchipbased, gross2010nonlinear},
ultracold Fermi gasses\cite{martin2013quantum}, trapped
ions\cite{bohnet2016quantum}, and optical
cavities\cite{hosten2016quantum, norcia2018cavitymediated}.

One of the primary current motivations for studying collective spin
systems is their capability to dynamically generate spin-squeezed
states, which enable measurement of phase angles $\phi$ with an error
$\Delta\phi$ below the standard quantum limit
$\Delta\phi\sim1/\sqrt{N}$ set by quantum projection noise from the
measurement of $N$ independent particles\cite{ma2011quantum}.  The
nonlinear dynamics necessary to generate spin-squeezed states can be
implemented with long-range collisional, photon-, or phonon-mediated
interactions, typically resulting in a one-axis twising (OAT) model
described by the Hamiltonian\cite{kitagawa1993squeezed}
\begin{align}
  H_{\t{OAT}} = \chi S_\z^2,
  \label{eq:OAT}
\end{align}
where $S_\z\equiv\sum_{j=1}^N\sigma_\z^{(j)}/2$ is a collective
spin-$z$ operator defined in terms of Pauli-$z$ operators
$\sigma_\z^{(j)}$ addressing spin $j$.  The OAT model prepares a
spin-squeezed state that allows measurement of a phase angle $\phi$
with error $\Delta\phi\sim N^{-2/3}$ at times
$t\sim\chi^{-1}N^{-2/3}$\cite{ma2011quantum}.  Although a truly
collective spin model requires uniform, all-to-all interactions, as
long as measurements do not distinguish between constituent particles
even non-uniform systems can be effectively described by a uniform
model with renormalized parameters\cite{hu2015entangled}.  Various
works have proposed improvements over OAT through the addition of
auxiliary control fields\cite{law2001coherent, liu2011spin,
  huang2015twoaxis, muessel2015twistandturn} or a modified
vacuum\cite{hu2017vacuum}, thereby realizing different collective spin
models.

While the OAT model admits exact
solutions\cite{foss-feig2013nonequilibrium}, the same is not true of
more general, non-Ising Hamiltonians.  Nonetheless, in the absence of
decoherence, permutation symmetry and total spin conservation divide
the total Hilbert space of a collective spin system into superseletion
sectors that grow only linearly with system size, thereby admitting
efficient classical simulation of its dynamics.  If decoherence is
present, but sufficiently weak, dynamics of collective spin systems
are still numerically solvable via ``quantum jump'' Monte Carlo
methods\cite{plenio1998quantumjump, zhang2018montecarlo} (also known
as ``quantum trajectory'' or ``Monte Carlo wavefunction'' methods)
that can reproduce all expectation values of interest.  When
decoherence is strong, however, or when the number of jump operators
grows extensively with system size (e.g.~for single-spin decay), these
Monte Carlo methods can take a prohibitively long time to converge, as
simulations become dominated by incoherent jumps that generate large
numbers of distinct trajectories over which to average.

In this work, we develop a new method to simulate short-time dynamics
of collective spin systems with permutationally-symmetric decoherence.
This method is based on a short-time expansion of exact solutions to
the equations of motion for Heisenberg operators.  Evaluating this
expansion requires use of the structure factors of a collective spin
operator algebra; the calculation of these structure factors is one of
the main technical results of this work.


\section{Theory}

In this section we provide the basic theory for our method to compute
expectation values of collective spin operators, deferring lengthy
derivations to the Appendices.  We consider systems composed of $N$
distinct spin-1/2 particles.  Defining individual spin-1/2 operators
$s_{n=\x,\y,\z}\equiv\sigma_n/2$ and
$s_\pm\equiv s_\x\pm is_\y=\sigma_\pm$ with Pauli operators
$\sigma_n$, we denote an operator which acts with $s_n$ on the spin
indexed by $j$ and trivially (i.e.~with the identity $\1$) on all
other spins by $s_n^{(j)}$.  We then define the collective spin
operators $S_n\equiv\sum_{j=1}^Ns_n^{(j)}$ for
$n\in\set{\x,\y,\z,+,-}$.  Identifying the set $\set{\S_{\v m}}$ as a
basis for all collective spin operators, with
$\v m\equiv\p{m_+,m_\z,m_-}\in\N_0^3$ and
$\S_{\v m}\equiv S_+^{m_+} S_\z^{m_\z} S_-^{m_-}$, we can expand all
collective spin Hamiltonians in the form
\begin{align}
  H = \sum_{\v m} h_{\v m} \S_{\v m},
  \label{eq:H_general}
\end{align}
where Hermiticity requires that $h_{m_+,m_\z,m_-}=h_{m_-,m_\z,m_+}^*$.
The time evolution of a particular Heisenberg operator $\S_{\v m}$
under a Hamiltonian of the form in Eq.~\eqref{eq:H_general} is then
given by
\begin{align}
  \f{d}{dt}\S_{\v m}
  = i\sum_{\v n}h_{\v n}\sp{\S_{\v n},\S_{\v m}}_-
  + \sum_\J \gamma_\J \D\p{\J;\S_{\v m}}
  \equiv \sum_{\v n} T_{\v m\v n} \S_{\v n},
  \label{eq:EOM}
\end{align}
where $\sp{X,Y}_\pm\equiv XY\pm YX$, $\J$ is a set of jump operators
with a corresponding decoherence rate $\gamma_\J$, and
\begin{align}
  \D\p{\J;\O}
  \equiv \sum_{J\in\J} \p{J^\dag \O J - \f12\sp{J^\dag J,\O}_+}
\end{align}
is the Heisenberg picture dissipator, or Lindblad superoperator, for
all jump operators $J\in\J$.  Decoherence via uncorrelated decay of
individual spins at a rate $\gamma_-$, for example, would be described
by the set of jump operators
$\J_-=\set{\sigma_-^{(j)}:j=1,2,\cdots,N}$ with
$\gamma_{\J_-}=\gamma_-$.  The commutator in Eq.~\eqref{eq:EOM} can be
determined from the structure factors of $f_{\v n\v m\v\ell}$ of the
general product
$\S_{\v n}\S_{\v m}=\sum_{\v\ell}f_{\v n\v m\v\ell}S_{\v\ell}$ worked
out in Appendix \ref{sec:prod_general}, while the effects of
decoherence from jump operators (i.e.~elements of $\J$) of the form
$\gamma^{(j)}=\sum_n\gamma_ns_n^{(j)}$ and $\Gamma=\sum_n\Gamma_nS_n$
are respectively worked out in Appendices \ref{sec:decoherence_single}
and \ref{sec:decoherence_collective}.  These ingredients are
sufficient to compute matrix elements $T_{\v m\v n}$ of the time
derivative operator $d/dt$ in Eq.~\eqref{eq:EOM} in most cases of
interest.

The time derivative operator $d/dt$ will generally couple spin
operators $\S_{\v m}$ to spin operators $\S_{\v n}$ with higher
``weight'', i.e.~with $\abs{\v n}>\abs{\v m}$, where
$\abs{\v\ell}\equiv\sum_\lambda\ell_\lambda$.  The growth of operator
weight signifies the growth of many-body correlations.  In practice,
keeping track of this growth will eventually require more
computational resources than are available, meaning we must somehow
truncate our equations of motion.  The simplest truncation strategy
would be to take
\begin{align}
  \f{d}{dt} \S_{\v m}
  \to \sum_{w\p{\v n}<W} T_{\v m\v n} \S_{\v n}
  \label{eq:weight_truncation}
\end{align}
for some weight measure $w$, e.g.~$w\p{\v n}=\abs{\v n}$, and cutoff
$W$.  The truncation in Eq.~\eqref{eq:weight_truncation} will close
the system of differential equations defined by Eq.~\eqref{eq:EOM},
and allow us to solve it using standard numerical methods.  Initial
conditions for this system of differential equations, i.e.~expectation
values of collective spin operators with respect to states that are
simple to prepare experimentally, are provided in Appendix
\ref{sec:initial_conditions}.

The truncation strategy in Eq.~\eqref{eq:weight_truncation} has a few
limitations:
\begin{enumerate*}[label=(\roman*)]
\item simulating a system of differential equations for a large number
  of operators can be quite time-consuming,
\item the weight measure $w$ may need to be chosen carefully, as the
  optimal measure is generally system-dependent,
\item simulation results can only be trusted up to the time at which
  the initial value of operators $\S_{\v m}$ with weight
  $w\p{\v m}\ge W$ have a non-negligible contribution to operators of
  interest.\label{pt:limitation}
\end{enumerate*}
The last limitation in particular unavoidably applies in some form to
any method tracking only a subset of all relevant operators.  We
therefore devise an alternate truncation strategy built around
limitation \ref{pt:limitation}.

We can expand Heisenberg operators in a Taylor series about $t=0$ to
write
\begin{align}
  \bk{\S_{\v m}}
  = \sum_{k\ge0} \f{t^k}{k!} \bk{\f{d^k}{dt^k} \S_{\v m}}_{t=0}
  = \sum_{k\ge0} \f{t^k}{k!}
  \sum_{\v n} T^k_{\v m\v n} \bk{\S_{\v n}}_{t=0},
  \label{eq:time_series}
\end{align}
where the matrix elements $T^k_{\v m\v n}$ of the $k$-th time
derivative are
\begin{align}
  T^0_{\v m\v n} \equiv \delta_{\v m\v n},
  &&
  T^1_{\v m\v n} \equiv T_{\v m\v n},
  &&
  T^{k>1}_{\v m\v n}
  \equiv \sum_{\v p_1,\v p_2,\cdots,\v p_{k-1}}
  T_{\v m\v p_1} T_{\v p_1\v p_2} T_{\v p_2\v p_3}
  \cdots T_{\v p_{k-1}\v n},
\end{align}
with $\delta_{\v m\v n}=1$ if $\v m=\v n$ and 0 otherwise.  For
sufficiently short times, we can truncate the series in
Eq.~\eqref{eq:time_series} by taking
\begin{align}
  \bk{\S_{\v m}}
  \to \sum_{k=0}^M \f{t^k}{k!}
  \sum_{\v n} T^k_{\v m\v n} \bk{\S_{\v n}}_{t=0}.
\end{align}
Unlike the truncation in Eq.~\eqref{eq:weight_truncation}, the nonzero
matrix elements $T^k_{\v m\v n}$ for $k=0,1,\cdots,M$ now tell us
which operators are relevant for computing the expectation value
$\bk{\S_{\v m}}$ to a fixed order $M$.  Note that the relation
$\S_{\v m}^\dag=\S_{\v m^*}$ for $\v m^*\equiv\p{m_-,m_\z,m_+}$ can be
exploited to cut both the number of matrix elements $T_{\v m\v n}$ and
expectation values $\bk{\S_{\v m}}_{t=0}$ that we need to explicitly
compute roughly in half.


\section{Benchmarking}

Our measure of spin squeezing is
\begin{align}
  \xi^2 \equiv \f{N}{\abs{\bk{\v S}}^2}
  \min_{\uv n\perp\bk{\v S}} \bk{\p{\v S\c\uv n}^2},
  \label{eq:squeezing}
\end{align}
where $N$ is the total number of spins, $\v S$ is the collective spin
vector, and the minimization is performed over all unit vectors
$\uv n$ in the plane orthogonal to the mean spin vector $\bk{\v S}$.
This squeezing parameter is entirely determined by one- and two-spin
correlators of the form $\bk{S_\alpha}$ and $\bk{S_\alpha S_\beta}$.


\section{Discussion}

Many-body correlators and cumulants \`a la
Ref.~\cite{schweigler2017solving}?


\section{Conclusion}



\bibliography{\jobname}

\newpage
\appendix

\section{Exact results for the one-axis twisting model}
\label{sec:OAT}

The one-axis twisting (OAT) Hamiltonian for $N$ spin-1/2 particles
takes the form
\begin{align}
  H_{\t{OAT}}
  = \chi S_\z^2
  = \f12 \chi \sum_{j<k} \sigma_\z^{(j)} \sigma_\z^{(k)} + \f14 N \chi,
\end{align}
where $\sigma_\z^{(j)}$ represents a Pauli-$z$ operator acting on spin
$j$.  This Hamiltonian is a special case of the Ising Hamiltonian
previously solved in Ref.~[\citenum{foss-feig2013nonequilibrium}] via
exact, analytical treatment of the quantum jump Monte Carlo method for
computing expectation values.  The solution therein accounts for
coherent evolution in addition to decoherence via uncorrelated
single-spin decay, excitation, and dephasing respectively at rates
$\gamma_-$, $\gamma_+$, and $\gamma_\z$ (denoted by $\Gamma_{\t{ud}}$,
$\Gamma_{\t{du}}$, and $\Gamma_{\t{el}}$ in
Ref.~[\citenum{foss-feig2013nonequilibrium}]).  Letting $S\equiv N/2$
and $\mu,\nu\in\set{+1,-1}$, we adapt expectation values computed in
Ref.~[\citenum{foss-feig2013nonequilibrium}] for the initial state
$\ket{\X}$ satisfying $S_\x\ket{\X}=S\ket{\X}$ evolving under
$H_{\t{OAT}}$, finding
\begin{align}
  \bk{S_+}
  &= S e^{-\Gamma t} \Phi\p{\chi,t}^{N-1}, \label{eq:S+_OAT} \\
  \bk{S_\mu S_\z}
  &= -\f{\mu}{2}\bk{S_\mu} + S \p{S-\f12} e^{-\Gamma t}
  \Psi\p{\mu\chi,t} \Phi\p{\chi,t}^{N-2}, \\
  \bk{S_\mu S_\nu}
  &= \delta_{\mu,-\nu} \p{S + \mu\bk{S_\z}}
  + S \p{S-\f12} e^{-2\Gamma t}
  \Phi\p{\sp{\mu+\nu}\chi,t}^{N-2}, \label{eq:SS+-_OAT}
\end{align}
where
\begin{align}
  \Phi\p{X,t}
  &\equiv e^{-\lambda t} \sp{\cos\p{t\sqrt{s_X^2-r}}
    + \lambda t~\t{sinc}\p{t\sqrt{s_X^2-r}}},
  \\
  \Psi\p{X,t}
  &\equiv e^{-\lambda t} \p{is_X-\gamma}t~
  \t{sinc}\p{t\sqrt{s_X^2-r}},
\end{align}
for
\begin{align}
  \gamma \equiv -\f12 \p{\gamma_+ - \gamma_-},
  &&
  \lambda \equiv \f12 \p{\gamma_+ + \gamma_-},
  &&
  r \equiv \gamma_+ \gamma_-,
  &&
  \Gamma \equiv \f12\gamma_\z + \lambda,
  &&
  s_X \equiv X + i\gamma,
\end{align}
The last ingredients we need to compute spin squeezing at any time are
$\bk{S_\z}$ and $\bk{S_\z^2}$.  As these operators commute with the
OAT Hamiltonian, their evolution is governed entirely by decoherence
(see Appendix \ref{sec:decay_single}), which means
\begin{align}
  \f{d}{dt} S_\z
  &= S\p{\gamma_+-\gamma_-} - \p{\gamma_++\gamma_-} S_\z,
  \\
  \f{d}{dt}\p{S_\z^2}
  &= S\p{\gamma_++\gamma_-} + 2\p{\gamma_+-\gamma_-}\p{S-\f12} S_\z
  - 2 \p{\gamma_++\gamma_-} S_\z^2.
\end{align}
The initial conditions $\bk{S_\z}_{t=0}=0$ and $\bk{S_\z^2}_{t=0}=S/2$
then imply
\begin{align}
  \bk{S_\z}
  = S\p{\f{\gamma_+-\gamma_-}{\gamma_++\gamma_-}}
  \p{1-e^{-\p{\gamma_+ + \gamma_-} t}},
  &&
  \bk{S_\z^2} = S \sp{\f12 + \p{S-\f12} \f{\bk{S_\z}^2}{S^2}}.
  \label{eq:Sz_OAT}
\end{align}
The expectation values in \eqref{eq:S+_OAT}-\eqref{eq:SS+-_OAT} and
\eqref{eq:Sz_OAT} are sufficient to compute the spin squeezing at any
time throughout evolution of the initial state $\ket{\X}$ under
$H_{\t{OAT}}$.


\section{Basic spin operator identities}
\label{sec:identities}

The appendices in this work make ubiquitous use of various spin
operator identities; we collect and derive some basic identities here
for reference.  Note that despite the working definition of collective
spin operators from $S_n=\sum_js_n^{(j)}$, the identities we will
derive involving only collective spin operators apply just as well to
large-spin operators that cannot be expressed as the sum of individual
spin-1/2 operators.  The elementary commutation relations between spin
operators are, with $\bmu\equiv-\mu\in\set{+1,-1}$ for brevity,
\begin{align}
  \sp{s_\z^{(j)},s_\mu^{(k)}}_-
  &= \delta_{jk} \mu s_\mu^{(j)},
  &
  \sp{S_\z,s_\mu^{(j)}}_-
  &= \sp{s_\z^{(j)},S_\mu}_- = \mu s_\mu^{(j)},
  &
  \sp{S_\z,S_\mu}_-
  &= \mu S_\mu,
  \label{eq:comm_z_base} \\
  \sp{s_\mu^{(j)},s_\bmu^{(k)}}_-
  &= \delta_{jk} 2 \mu s_\z^{(j)},
  &
  \sp{S_\mu,s_\bmu^{(j)}}_-
  &= \sp{s_\mu^{(j)},S_\bmu}_- = 2 \mu s_\z^{(j)},
  &
  \sp{S_\mu,S_\bmu}_-
  &= 2 \mu S_\z.
  \label{eq:comm_mu_base}
\end{align}
These relations can be used to inductively compute identities
involving powers of collective spin operators.  By pushing through one
spin operator at a time, we can find
\begin{align}
  \p{\mu S_\z}^m s_\mu^{(j)}
  = \p{\mu S_\z}^{m-1} s_\mu^{(j)} \p{1 + \mu S_\z}
  = \p{\mu S_\z}^{m-2} s_\mu^{(j)} \p{1 + \mu S_\z}^2
  = \cdots
  = s_\mu^{(j)} \p{1 + \mu S_\z}^m,
  \label{eq:push_z_mu_Ss}
\end{align}
and
\begin{align}
  \mu s_\z^{(j)} S_\mu^m
  = S_\mu \mu s_\z^{(j)} S_\mu^{m-1} + s_\mu^{(j)} S_\mu^{m-1}
  = \cdots
  = S_\mu^m \mu s_\z^{(j)} + ms_\mu^{(j)} S_\mu^{m-1},
  \label{eq:push_z_mu_sS}
\end{align}
where we will generally find it nicer to express results in terms of
$\mu s_\z^{(j)}$ and $\mu S_\z$ rather than $s_\z^{(j)}$ and $S_\z$.
Summing over the single-spin index $j$ in both of the cases above
gives us the purely collective-spin versions of these identities:
\begin{align}
  \p{\mu S_\z}^m S_\mu = S_\mu \p{1 + \mu S_\z}^m,
  &&
  \mu S_\z S_\mu^m = S_\mu^m \p{m + \mu S_\z},
  \label{eq:push_z_mu_single}
\end{align}
where we can repeat the process of pushing through individual $S_\z$
operators $\ell$ times to get
\begin{align}
  \p{\mu S_\z}^\ell S_\mu^m
  = \p{\mu S_\z}^{\ell-1} S_\mu^m \p{m + \mu S_\z}
  = \p{\mu S_\z}^{\ell-2} S_\mu^m \p{m + \mu S_\z}^2
  = \cdots
  = S_\mu^m \p{m + \mu S_\z}^\ell.
  \label{eq:push_z_mu}
\end{align}
Multiplying \eqref{eq:push_z_mu} through by $\p{\mu\nu}^\ell$ (for
$\nu\in\set{+1,-1}$) and taking its Hermitian conjugate, we can say
that more generally
\begin{align}
  \p{\nu S_\z}^\ell S_\mu^m
  = S_\mu^m \p{\mu\nu m+\nu S_\z}^\ell,
  &&
  S_\mu^m \p{\nu S_\z}^\ell
  = \p{-\mu\nu m+\nu S_\z}^\ell S_\mu^m.
\end{align}
Finding commutation relations between powers of transverse spin
operators, i.e.~$S_\mu$ and $S_\bmu$, turns out to be considerably
more difficult than the cases we have worked out thus far.  We
therefore save this work for Appendix \ref{sec:comm_transverse}.


\section{Commutation relations between powers of transverse spin
  operators}
\label{sec:comm_transverse}

To find commutation relations between powers of transverse collective
spin operators, we first compute
\begin{align}
  S_\mu^m s_\bmu^{(j)}
  &= S_\mu^{m-1}s_\bmu^{(j)} S_\mu
  + S_\mu^{m-1} 2\mu s_\z^{(j)} \\
  &= S_\mu^{m-2} s_\bmu^{(j)} S_\mu^2
  + S_\mu^{m-2} 2\mu s_\z^{(j)} S_\mu
  + S_\mu^{m-1} 2\mu s_\z^{(j)} \\
  &= s_\bmu^{(j)} S_\mu^m
  + \sum_{k=0}^{m-1} S_\mu^k 2\mu s_\z^{(j)} S_\mu^{m-k-1}
  \label{eq:push_mu_Ss_start}.
\end{align}
While \eqref{eq:push_mu_Ss_start} gives us the commutator
$\sp{S_\mu^m,s_\bmu^{(j)}}_-$, we would like to enforce an ordering on
products of spin operators, which will ensure that we only keep track
of operators that are linearly independent.  We choose (for now) to
impose an ordering with all $s_\bmu^{(j)}$ operators on the left, and
all $s_\z^{(j)}$ operators on the right, as such an ordering will
prove convenient for the calculations in this section.  This choice of
ordering compels us to expand
\begin{align}
  \sum_{k=0}^{m-1} S_\mu^k 2\mu s_\z^{(j)} S_\mu^{m-k-1}
  &= \sum_{k=0}^{m-1} S_\mu^k
  \sp{2\p{m-k-1} s_\mu^{(j)} S_\mu^{m-k-2}
    + S_\mu^{m-k-1} 2\mu s_\z^{(j)}} \\
  &= m \p{m-1} s_\mu^{(j)} S_\mu^{m-2}
  + m S_\mu^{m-1} 2\mu s_\z^{(j)},
\end{align}
which implies
\begin{align}
  S_\mu^m s_\bmu^{(j)}
  = s_\bmu^{(j)} S_\mu^m + m \p{m-1} s_\mu^{(j)} S_\mu^{m-2}
  + m S_\mu^{m-1} 2\mu s_\z^{(j)},
  \label{eq:push_mu_Ss}
\end{align}
and in turn
\begin{align}
  S_\mu^m S_\bmu = S_\bmu S_\mu^m
  + m S_\mu^{m-1} \p{m - 1 + 2\mu S_\z}.
  \label{eq:push_mu_single}
\end{align}
As the next logical step, we take on the task of computing
\begin{align}
  S_\mu^m S_\bmu^n
  &= S_\mu^{m-1} S_\bmu^n S_\mu
  + n \sp{S_\mu^{m-1} S_\bmu^{n-1} \p{1 - n + 2\mu S_\z}} \\
  &= S_\bmu^n S_\mu^m
  + n \sum_{k=0}^{m-1} S_\mu^{m-k-1} S_\bmu^{n-1}
  \p{1 - n + 2\mu S_\z} S_\mu^k,
\end{align}
which implies
\begin{align}
  \sp{S_\mu^m, S_\bmu^n}_-
  = C_{mn;\mu}
  \equiv n \sum_{k=0}^{m-1} S_\mu^{m-k-1} S_\bmu^{n-1}
  \p{1 - n + 2\mu S_\z} S_\mu^k.
\end{align}
We now need rearrange the operators in $C_{mn;\mu}$ into a standard
order, which means pushing all $S_\z$ operators to the right and, for
the purposes of this calculation, all $S_\bmu$ operators to the left.
We begin by pushing $S_\mu^k$ to the left of $S_\z$, which takes
$2\mu S_\z\to 2\mu S_\z+2k$, and then push $S_\mu^{m-k-1}$ to the
right of $S_\bmu^{n-1}$, giving us
\begin{align}
  C_{mn;\mu}
  &= n \sum_{k=0}^{m-1}
  \p{S_\bmu^{n-1} S_\mu^{m-k-1} + C_{m-k-1,n-1;\mu}} S_\mu^k
  \p{2k + 1 - n + 2\mu S_\z} \\
  &= D_{mn;\mu}
  + n \sum_{k=0}^{m-2} C_{m-k-1,n-1;\mu}
  S_\mu^k \p{2k + 1 - n + 2\mu S_\z},
  \label{eq:C_mn}
\end{align}
where we have dropped the last ($k=m-1$) term in the remaining sum
because $C_{m-k-1,n-1;\mu}=0$ if $k=m-1$, and
\begin{align}
  D_{mn;\mu}
  \equiv mn S_\bmu^{n-1} S_\mu^{m-1} \p{m - n + 2\mu S_\z}.
  \label{eq:D_mn}
\end{align}
To our despair, we have arrived in \eqref{eq:C_mn} at a {\it
  recursive} formula for $C_{mn;\mu}$.  Furthermore, we have not even
managed to order all spin operators, as $C_{m-k-1,n-1;\mu}$ contains
$S_\z$ operators which are to the left of $S_\mu^k$.  To sort all spin
operators once and for all, we define
\begin{align}
  C_{mn;\mu}^{(k)} \equiv C_{m-k,n;\mu} S_\mu^k,
  &&
  D_{mn;\mu}^{(k)} \equiv D_{m-k,n;\mu} S_\mu^k,
\end{align}
which we can expand as
\begin{align}
  D_{mn;\mu}^{(k)}
  &= \p{m-k}n S_\bmu^{n-1} S_\mu^{m-k-1}
  \p{m-k-n+2\mu S_\z} S_\mu^k \\
  &= \p{m-k}n S_\bmu^{n-1} S_\mu^{m-1} \p{k+m-n+2\mu S_\z},
  \label{eq:D_mn_k}
\end{align}
and
\begin{align}
  C_{mn;\mu}^{(k)}
  &= D_{m-k,n;\mu} S_\mu^k + n \sum_{j=0}^{m-k-2}
  C_{m-k-j-1,n-1;\mu} S_\mu^j \p{2j+1-n+2\mu S_\z} S_\mu^k \\
  &= D_{mn;\mu}^{(k)} + n \sum_{j=0}^{m-k-2}
  C_{m-k-j-1,n-1;\mu} S_\mu^{j+k} \p{2j+2k+1-n+2\mu S_\z} \\
  &= D_{mn;\mu}^{(k)} + n \sum_{j=0}^{m-k-2}
  C_{m-1,n-1;\mu}^{(k+j)} \p{2\sp{j+k}+1-n+2\mu S_\z} \\
  &= D_{mn;\mu}^{(k)} + n \sum_{j=k}^{m-2}
  C_{m-1,n-1;\mu}^{(j)} \p{2j+1-n+2\mu S_\z}.
  \label{eq:C_mn_k}
\end{align}
While the resulting expression in \eqref{eq:C_mn_k} strongly resembles
that in \eqref{eq:C_mn}, there is one crucial difference: all spin
operators in \eqref{eq:C_mn_k} have been sorted into a standard order.
We can now repeatedly substitute $C_{mn;\mu}^{(k)}$ into itself, each
time decreasing $m$ and $n$ by 1, until one of $m$ or $n$ reaches
zero.  Such repeated substitution yields the expansion
\begin{align}
  C_{mn;\mu}
  = C_{mn;\mu}^{(0)}
  = D_{mn;\mu}
  + \sum_{p=1}^{\min\set{m,n}-1} E_{mn;\mu}^{(p)},
  \label{eq:C_mn_E}
\end{align}
where the first two terms in the sum over $p$ are
\begin{align}
  E_{mn;\mu}^{(1)}
  &= n \sum_{k=0}^{m-2} D_{m-1,n-1;\mu}^{(k)} \p{2k+1-n+2\mu S_\z}, \\
  E_{mn;\mu}^{(2)}
  &= n \sum_{k_1=0}^{m-2} \p{n-1} \sum_{k_2=k_1}^{m-3}
  D_{m-2,n-2;\mu}^{(k_2)} \p{2k_2+2-n+2\mu S_\z} \p{2k_1+1-n+2\mu S_\z},
\end{align}
and more generally for $p>1$,
\begin{align}
  E_{mn;\mu}^{(p)}
  = \f{n!}{\p{n-p}!}
  \sum_{k_1=0}^{m-2} \sum_{k_2=k_1}^{m-3} \cdots\sum_{k_p=k_{p-1}}^{m-p-1}
  D_{m-p,n-p;\mu}^{(k_p)} \prod_{j=1}^p \p{2k_j+j-n+2\mu S_\z}.
  \label{eq:E_mn_p}
\end{align}
In principle, the expressions in \eqref{eq:D_mn_k} and
\eqref{eq:C_mn_E}-\eqref{eq:E_mn_p} can be used to evaluate the
commutator $\sp{S_\mu^m,S_\bmu^n}_- = C_{mn;\mu}$, but these result
are -- to put it lightly -- quite a mess: the expression for
$E_{mn;\mu}^{(p)}$ in \eqref{eq:E_mn_p} involves a sum over $p$
mutually dependent intermediate variables, each term of which
additionally contains a product of $p$ factors.  We therefore devote
the rest of this section to simplifying our result for the commutator
$\sp{S_\mu^m,S_\bmu^n}_-$.

Observing that in \eqref{eq:E_mn_p} we always have
$0\le k_1\le k_2\le\cdots\le k_p\le m-p-1$, we can rearrange the order
of the sums and relabel $k_p\to\ell$ to get
\begin{align}
  E_{mn;\mu}^{(p)}
  = \f{n!}{\p{n-p}!}
  \sum_{\ell=0}^{m-p-1} D_{m-p,n-p;\mu}^{(\ell)} \p{2\ell+F_{np;\mu}}
  \sum_{\p{\v k,p-1,\ell}} \prod_{j=1}^{p-1} \p{2k_{p-j}-j+F_{np;\mu}},
  \label{eq:E_mn_p_sum}
\end{align}
where for shorthand we define
\begin{align}
  F_{np;\mu} \equiv p - n + 2\mu S_\z,
  &&
  \sum_{\p{\v k,q,\ell}} X \equiv
  \sum_{k_1=0}^\ell \sum_{k_2=k_1}^\ell
  \cdots \sum_{k_q=k_{q-1}}^\ell X.
\end{align}
We now further define
\begin{align}
  f_{np\ell;\mu}\p{k,q} \equiv \p{\ell-k+q} \p{\ell+k-q+F_{np;\mu}},
\end{align}
and evaluate sums successively over $k_{p-1},k_{p-2},\cdots,k_1$,
finding
\begin{align}
  \sum_{\p{\v k,p-1,\ell}} \prod_{j=1}^{p-1} \p{2k_{p-j}-j+F_{np;\mu}}
  &= \sum_{\p{\v k,p-2,\ell}}
  \prod_{j=2}^{p-1} \p{2k_{p-j}-j+F_{np;\mu}}
  f_{np\ell;\mu}\p{k_{p-2},1} \\
  &= \f1{\p{r-1}!} \sum_{\p{\v k,p-r,\ell}}
  \prod_{j=r}^{p-1} \p{2k_{p-j}-j+F_{np;\mu}}
  \prod_{q=1}^{r-1} f_{np\ell;\mu}\p{k_{p-r},q} \\
  &= \f1{\p{p-1}!} \prod_{q=1}^{p-1} f_{np\ell;\mu}\p{0,q} \\
  &= { \ell + p - 1 \choose p - 1 }
  \prod_{q=1}^{p-1} \p{\ell-q+F_{np;\mu}}.
\end{align}
Substitution of this result together with $D_{m-p,n-p;\mu}^{(\ell)}$
using \eqref{eq:D_mn_k} into \eqref{eq:E_mn_p_sum} then gives us
\begin{align}
  E_{mn;\mu}^{(p)}
  = \f{n!}{\p{n-p-1}!} S_\bmu^{n-p-1} S_\mu^{m-p-1} G_{mnp;\mu}
\end{align}
with
\begin{align}
  G_{mnp;\mu}
  &\equiv \sum_{\ell=0}^{m-p-1} { \ell + p - 1 \choose p - 1 }
  \p{m-p-\ell} \p{\ell+m-p+F_{np;\mu}}
  \p{2\ell + F_{np;\mu}}
  \prod_{q=1}^{p-1} \p{\ell-q+F_{np;\mu}} \\
  &= { m \choose p + 1 } \prod_{q=0}^p \p{m-p-q+F_{np;\mu}}.
\end{align}
We can further simplify
\begin{align}
  \prod_{q=0}^p \p{m-p-q+F_{np;\mu}}
  = \prod_{q=0}^p \p{m-n-q+2\mu S_\z}
  = \sum_{q=0}^{p+1} \p{-1}^{p+1-q}
  { p+1 \brack q } \p{m-n+2\mu S_\z}^q,
\end{align}
where ${ n \brack k }$ is an unsigned Stirling number of the first
kind, and finally
\begin{align}
  \sum_{q=0}^p \p{-1}^{p-q} { p \brack q } \p{m-n+2\mu S_\z}^q
  &= \sum_{q=0}^p \p{-1}^{p-q} { p \brack q } \sum_{\ell=0}^q
  { q \choose \ell } \p{m-n}^{q-\ell} \p{2\mu S_\z}^\ell \\
  &= \sum_{\ell=0}^p 2^\ell \sum_{q=\ell}^p \p{-1}^{p-q}
  { p \brack q } { q \choose \ell } \p{m-n}^{q-\ell} \p{\mu S_\z}^\ell.
\end{align}
Putting everything together, we finally have
\begin{align}
  E_{mn;\mu}^{(p-1)}
  = p! { m \choose p } { n \choose p }
  S_\bmu^{n-p} S_\mu^{m-p}
  \sum_{\ell=0}^p \epsilon_{mn}^{p\ell} \p{\mu S_\z}^\ell,
\end{align}
with
\begin{align}
  \epsilon_{mn}^{p\ell}
  \equiv 2^\ell \sum_{q=\ell}^p \p{-1}^{p-q}
  { p \brack q } { q \choose \ell } \p{m-n}^{q-\ell},
\end{align}
where in this final form $E_{mn;\mu}^{(0)} = D_{mn;\mu}$, which
together with the expansion for $C_{mn;\mu}$ in \eqref{eq:C_mn_E}
implies that
\begin{align}
  \sp{S_\mu^m, S_\bmu^n}_-
  = \sum_{p=1}^{\min\set{m,n}}
  p! { m \choose p } { n \choose p } S_\bmu^{n-p} S_\mu^{m-p}
  \sum_{\ell=0}^p \epsilon_{mn}^{p\ell} \p{\mu S_\z}^\ell,
  \label{eq:comm_mu}
\end{align}
and
\begin{align}
  S_\mu^m S_\bmu^n
  = \sum_{p=0}^{\min\set{m,n}}
  p! { m \choose p } { n \choose p } S_\bmu^{n-p} S_\mu^{m-p}
  \sum_{\ell=0}^p \epsilon_{mn}^{p\ell} \p{\mu S_\z}^\ell,
  \label{eq:push_mu_bmu}
\end{align}
If we wish to order products of collective spin operators with $S_\z$
in between $S_\bmu$ and $S_\mu$, then
\begin{align}
  S_\mu^m S_\bmu^n
  = \sum_{p=0}^{\min\set{m,n}} p! { m \choose p } { n \choose p }
  S_\bmu^{n-p} Z_{mn;\bmu}^{(p)} S_\mu^{m-p},
\end{align}
where
\begin{align}
  Z_{mn;\bmu}^{(p)}
  \equiv \sum_{\ell=0}^p \epsilon_{mn}^{p\ell}
  \p{-\sp{m-p} + \mu S_\z}^\ell
  = \sum_{\ell=0}^p \epsilon_{mn}^{p\ell}
  \p{-1}^\ell \p{\sp{m-p} + \bmu S_\z}^\ell
  = \sum_{q=0}^p \zeta_{mn}^{pq} \p{\bmu S_\z}^q,
  \label{eq:Z_mnp}
\end{align}
with
\begin{align}
  \zeta_{mn}^{pq}
  \equiv \sum_{\ell=q}^p \epsilon_{mn}^{p\ell} \p{-1}^\ell
  { \ell \choose q } \p{m-p}^{\ell-q}
  = \p{-1}^p 2^q \sum_{s=q}^p
  { p \brack s } { s \choose q } \p{m+n-2p}^{s-q}.
  \label{eq:zeta_mnpq}
\end{align}


\section{Product of arbitrary ordered collective spin operators}
\label{sec:prod_general}

The most general product of collective spin operators that we need to
compute is
\begin{align}
  \S^{pqr}_{\ell mn;\mu}
  = S_\mu^p \p{\mu S_\z}^q S_\bmu^r
  S_\mu^\ell \p{\mu S_\z}^m S_\bmu^n
  = \sum_{k=0}^{\min\set{r,\ell}} k! { r \choose k } { \ell \choose k }
  S_\mu^{p+\ell-k} \tilde Z_{qr\ell m;\mu}^{(k)} S_\bmu^{r+n-k},
  \label{eq:general_product}
\end{align}
where
\begin{align}
  \tilde Z_{qr\ell m;\mu}^{(k)}
  &\equiv \p{\ell-k+\mu S_\z}^q
  Z_{r\ell;\mu}^{(k)} \p{r-k+\mu S_\z}^m \\
  &= \sum_{a=0}^k \zeta_{r\ell}^{ka}
  \sum_{b=0}^q \p{\ell-k}^{q-b} { q \choose b }
  \sum_{c=0}^m \p{r-k}^{m-c} { m \choose c }
  \p{\mu S_\z}^{a+b+c},
\end{align}
is defined in terms of $Z_{r\ell;\mu}^{(k)}$ and $\zeta_{r\ell}^{ka}$
as respectively given in \eqref{eq:Z_mnp} and \eqref{eq:zeta_mnpq}.
The (anti-)commutator of two ordered products of collective spin
operators is then
\begin{align}
  \sp{S_\mu^p \p{\mu S_\z}^q S_\bmu^r,
    S_\mu^\ell \p{\mu S_\z}^m S_\bmu^n}_\pm
  = \S^{pqr}_{\ell mn;\mu} \pm \S^{\ell mn}_{pqr;\mu}.
\end{align}


\section{Sandwich identities for single-spin decoherence calculations}
\label{sec:sandwich_single}

In this section we derive several identities which will be necessary
for computing the effects of single-spin decoherence on ordered
products of collective spin operators.  These identities all involve
sandwiching a collective spin operator between operators that act on
individual spins only, and summing such terms over all individual spin
indices.  Our general strategy will be to use commutation relations to
push single-spin operators together, and then evaluate the sum to
arrive at an expression in terms of collective spin operators only.

We first compute sums of single-spin operators sandwiching
$\p{\mu S_\z}^m$, which will in turn be used to compute similar sums
sandwiching $S_\mu^\ell \p{\mu S_\z}^m S_\bmu^n$.  Up to Hermitian
conjugation, the unique cases are, for $\mu,\nu\in\set{+1,-1}$,
\begin{align}
  \sum_j s_\z^{(j)} \p{\mu S_\z}^m s_\z^{(j)}
  &= \sum_j s_\z^{(j)} s_\z^{(j)} \p{\mu S_\z}^m
  = \f14 \sum_j \1_j \p{\mu S_\z}^m
  = \f12 S \p{\mu S_\z}^m, \\
  \sum_j s_\z^{(j)} \p{\mu S_\z}^m s_\nu^{(j)}
  &= \p{\mu S_\z}^m \sum_j s_\z^{(j)} s_\nu^{(j)}
  = \f12 \p{\mu S_\z}^m \nu S_\nu
  = \f12 \nu S_\nu \p{\mu\nu+\mu S_\z}^m, \\
  \sum_j s_\nu^{(j)} \p{\mu S_\z}^m s_\nu^{(j)}
  &= \sum_j s_\nu^{(j)} s_\nu^{(j)} \p{\mu\nu+\mu S_\z}^m
  = 0, \\
  \sum_j s_\bnu^{(j)} \p{\mu S_\z}^m s_\nu^{(j)}
  &= \sum_j s_\bnu^{(j)} s_\nu^{(j)} \p{\mu\nu+\mu S_\z}^m
  = \p{S-\nu S_\z} \p{\mu\nu+\mu S_\z}^m.
\end{align}
We are now equipped to derive similar identities for general ordered
products of collective spin operators.  Making heavy use of identities
\eqref{eq:push_z_mu_sS} and \eqref{eq:push_mu_Ss} to push single-spin
operators through transverse collective-spin operators, we again work
through all combinations that are unique up to Hermitian conjugation,
finding
\begin{align}
  \sum_j s_\z^{(j)} S_\mu^\ell \p{\mu S_\z}^m S_\bmu^n s_\z^{(j)}
  &= \f12 \p{S-\ell-n} S_\mu^\ell \p{\mu S_\z}^m S_\bmu^n
  + \ell n S_\mu^{\ell-1} \p{S+\mu S_\z}
  \p{-1+\mu S_\z}^m S_\bmu^{n-1},
  \label{eq:san_z_z} \allowdisplaybreaks \\
  \sum_j s_\z^{(j)} S_\mu^\ell \p{\mu S_\z}^m S_\bmu^n s_\mu^{(j)}
  &= \f12 \mu S_\mu^{\ell+1} \p{1+\mu S_\z}^m S_\bmu^n
  - \mu n \p{S-\ell-\f12\sp{n-1}} S_\mu^\ell
  \p{\mu S_\z}^m S_\bmu^{n-1} \notag \\
  &\qquad - \mu\ell n\p{n-1} S_\mu^{\ell-1}
  \p{S+\mu S_\z} \p{-1+\mu S_\z}^m S_\bmu^{n-2},
  \label{eq:san_z_mu} \allowdisplaybreaks \\
  \sum_j s_\z^{(j)} S_\mu^\ell \p{\mu S_\z}^m S_\bmu^n s_\bmu^{(j)}
  &= -\f12 \mu S_\mu^\ell \p{\mu S_\z}^m S_\bmu^{n+1}
  + \mu \ell S_\mu^{\ell-1} \p{S+\mu S_\z} \p{-1+\mu S_\z}^m S_\bmu^n,
  \label{eq:san_z_bmu} \allowdisplaybreaks \\
  \sum_j s_\mu^{(j)} S_\mu^\ell \p{\mu S_\z}^m S_\bmu^n s_\mu^{(j)}
  &= n S_\mu^{\ell+1} \p{\mu S_\z}^m S_\bmu^{n-1}
  - n\p{n-1} S_\mu^\ell \p{S+\mu S_\z} \p{-1+\mu S_\z}^m S_\bmu^{n-2},
  \label{eq:san_mu_mu} \allowdisplaybreaks \\
  \sum_j s_\mu^{(j)} S_\mu^\ell \p{\mu S_\z}^m S_\bmu^n s_\bmu^{(j)}
  &= S_\mu^\ell \p{S+\mu S_\z}\p{-1+\mu S_\z}^m S_\bmu^n,
  \label{eq:san_mu_bmu} \allowdisplaybreaks \\
  \sum_j s_\bmu^{(j)} S_\mu^\ell \p{\mu S_\z}^m S_\bmu^n s_\mu^{(j)}
  &= S_\mu^\ell \p{S - \ell - n - \mu S_\z}
  \p{1+\mu S_\z}^m S_\bmu^n \notag \\
  &\qquad + \ell n \p{2S - \ell - n + 2}
  S_\mu^{\ell-1} \p{\mu S_\z}^m S_\bmu^{n-1} \notag \\
  &\qquad + \ell n \p{\ell-1} \p{n-1} S_\mu^{\ell-2} \p{S+\mu S_\z}
  \p{-1+\mu S_\z}^m S_\bmu^{n-2}.
  \label{eq:san_bmu_mu}
\end{align}


\section{Uncorrelated, permutationally-symmetric single-spin
  decoherence}
\label{sec:decoherence_single}

In this section we work out the effects of permutationally-symmetric
decoherence of individual spins on ordered products of collective spin
operators.  For shorthand, we define
\begin{align}
  \D\p{\gamma;\O} \equiv \D\p{\set{\gamma^{(j)}};\O}
  = \sum_j\p{{\gamma^{(j)}}^\dag \O \gamma^{(j)}
    - \f12\sp{{\gamma^{(j)}}^\dag \gamma^{(j)}, \O}_+},
\end{align}
where $\gamma$ is an operator that acts on a single spin, and
$\gamma^{(j)}$ is an operator that acts with $\gamma$ on spin $j$ and
the identity on all other spins.


\subsection{Decay-type decoherence}
\label{sec:decay_single}

The effect of decoherence via uncorrelated decay ($\mu=-1$) or
excitation ($\mu=1$) of individual spins is described by
\begin{align}
  \D\p{s_\mu; \O}
  = \sum_j\p{s_\bmu^{(j)} \O s_\mu^{(j)}
    - \f12\sp{s_\bmu^{(j)} s_\mu^{(j)},\O}_+}
  = \sum_j s_\bmu^{(j)} \O s_\mu^{(j)}
  - S \O + \f{\mu}{2} \sp{S_\z, \O}_+.
\end{align}
In order to determine the effect of this decoherence on ordered
products of collective spin operators, we expand the anti-commutator
\begin{align}
  \sp{S_\z, S_\mu^\ell \p{\mu S_\z}^m S_\bmu^n}_+
  = S_\z S_\mu^\ell \p{\mu S_\z}^m S_\bmu^n
  + S_\mu^\ell \p{\mu S_\z}^m S_\bmu^n S_\z
  = \mu S_\mu^\ell\p{\ell+n+2\mu S_\z} \p{\mu S_\z}^m S_\bmu^n,
\end{align}
which implies, using \eqref{eq:san_mu_bmu},
\begin{align}
  \D\p{s_\bmu; S_\mu^\ell \p{\mu S_\z}^m S_\bmu^n}
  = S_\mu^\ell \p{S+\mu S_\z}\p{-1+\mu S_\z}^m S_\bmu^n
  - S_\mu^\ell\sp{S + \f12\p{\ell+n} + \mu S_\z}
  \p{\mu S_\z}^m S_\bmu^n,
  \label{eq:decay_diff}
\end{align}
and, using \eqref{eq:san_bmu_mu},
\begin{align}
  \D\p{s_\mu; S_\mu^\ell \p{\mu S_\z}^m S_\bmu^n}
  &= S_\mu^\ell \p{S - \ell - n - \mu S_\z} \p{1+\mu S_\z}^m S_\bmu^n
  - S_\mu^\ell\sp{S - \f12\p{\ell+n} - \mu S_\z}
  \p{\mu S_\z}^m S_\bmu^n \notag \\
  &\qquad + \ell n \p{2S - \ell - n + 2}
  S_\mu^{\ell-1} \p{\mu S_\z}^m S_\bmu^{n-1} \notag \\
  &\qquad + \ell n \p{\ell-1} \p{n-1} S_\mu^{\ell-2} \p{S + \mu S_\z}
  \p{-1+\mu S_\z}^m S_\bmu^{n-2}.
  \label{eq:decay_same}
\end{align}
Decoherence via jump operators $s_\bmu^{(j)}$ only couples operators
$S_\mu^\ell \p{\mu S_\z}^m S_\bmu^n$ to operators
$S_\mu^\ell \p{\mu S_\z}^{m'} S_\bmu^n$ with $m'\le m$.  Decoherence
via jump operators $s_\mu^{(j)}$, meanwhile, makes operators
$S_\mu^\ell \p{\mu S_\z}^m S_\bmu^n$ ``grow'' in $m$ through the last
term in \eqref{eq:decay_same} with $k=m$, although the sum $\ell+m+n$
does not grow.


\subsection{Dephasing}
\label{sec:dephasing_single}

The effect of decoherence via single-spin dephasing is described by
\begin{align}
  \D\p{s_\z; \O}
  = \sum_j\p{s_\z^{(j)} \O s_\z^{(j)}
    - \f12\sp{s_\z^{(j)} s_\z^{(j)},\O}_+}
  = \sum_j s_\z^{(j)} \O s_\z^{(j)} - \f12 S \O.
\end{align}
From \eqref{eq:san_z_z}, we then have
\begin{align}
  \D\p{s_\z; S_\mu^\ell \p{\mu S_\z}^m S_\bmu^n}
  = -\f12\p{\ell+n} S_\mu^\ell \p{\mu S_\z}^m S_\bmu^n
  + \ell n S_\mu^{\ell-1} \p{S + \mu S_\z}
  \p{-1 + \mu S_\z}^m S_\bmu^{n-1}.
\end{align}
Decoherence via single-spin dephasing makes operators
$S_\mu^\ell \p{\mu S_\z}^m S_\bmu^n$ ``grow'' in $m$, although the sum
$\ell+m+n$ does not grow.


\subsection{The general case}
\label{sec:general_single}

The most general type of single-spin decoherence is described by
\begin{align}
  \D\p{\gamma;\O}
  = \sum_j\p{{\gamma^{(j)}}^\dag \O \gamma^{(j)}
    - \f12\sp{{\gamma^{(j)}}^\dag \gamma^{(j)}, \O}_+},
  &&
  \gamma \equiv \gamma_\z s_\z + \gamma_+ s_+ + \gamma_- s_-.
  \label{eq:D_general_single}
\end{align}
To simplify \eqref{eq:D_general_single}, we expand
\begin{align}
  \gamma^\dag \O \gamma
  = \abs{\gamma_\z}^2 s_\z \O s_\z
  + \sum_\mu \p{\abs{\gamma_\mu}^2 s_\bmu \O s_\mu
    + \gamma_\bmu^* \gamma_\mu s_\mu \O s_\mu
    + \gamma_\z^* \gamma_\mu s_\z \O s_\mu
    + \gamma_\bmu^* \gamma_\z s_\mu \O s_\z},
\end{align}
and
\begin{align}
  \gamma^\dag \gamma
  = \f14 \abs{\gamma_\z}^2
  + \f12 \sum_\mu \sp{\abs{\gamma_\mu}^2 \p{1-2\mu s_\z}
    + \mu \p{\gamma_\z^*\gamma_\mu - \gamma_\bmu^*\gamma_\z} s_\mu},
\end{align}
which implies
\begin{align}
  \D\p{\gamma;\O}
  &= \sum_{X\in\set{\z,+,-}} \abs{\gamma_X}^2 \D\p{s_X;\O}
  + \sum_{\mu,j}
  \p{\gamma_\bmu^* \gamma_\mu s_\mu^{(j)} \O s_\mu^{(j)}
    + \gamma_\z^* \gamma_\mu s_\z^{(j)} \O s_\mu^{(j)}
    + \gamma_\bmu^* \gamma_\z s_\mu^{(j)} \O s_\z^{(j)}}
  \notag \\
  &\qquad -\f14 \sum_\mu \mu
  \p{\gamma_\z^*\gamma_\mu - \gamma_\bmu^*\gamma_\z} \sp{S_\mu, \O}_+.
\end{align}
In order to compute the effect of this decoherence on ordered products
of collective spin operators, we expand the anti-commutator
\begin{align}
  \sp{S_\mu, S_\mu^\ell \p{\mu S_\z}^m S_\bmu^n}_+
  = S_\mu^{\ell+1} \sp{\p{\mu S_\z}^m+\p{1+\mu S_\z}^m} S_\bmu^n
  - n S_\mu^\ell \p{n-1+2\mu S_\z} \p{\mu S_\z}^m S_\bmu^{n-1}.
  \label{eq:S_mu_acomm}
\end{align}
Recognizing a resemblance between \eqref{eq:S_mu_acomm} and
\eqref{eq:san_z_mu}, we can collect terms to simplify
\begin{align}
  \sum_j s_\z^{(j)} S_\mu^\ell \p{\mu S_\z}^m S_\bmu^n s_\mu^{(j)}
  - \f14 \mu \sp{S_\mu, S_\mu^\ell \p{\mu S_\z}^m S_\bmu^n}_+
  = K_{\ell mn;\mu} + L_{\ell mn;\mu}
  \label{eq:dec_z_mu}
\end{align}
and likewise
\begin{align}
  \sum_j s_\mu^{(j)} S_\mu^\ell \p{\mu S_\z}^m S_\bmu^n s_\z^{(j)}
  + \f14 \mu \sp{S_\mu, S_\mu^\ell \p{\mu S_\z}^m S_\bmu^n}_+
  = K_{\ell mn;\mu} + M_{\ell mn;\mu}
  \label{eq:dec_mu_z}
\end{align}
with
\begin{align}
  K_{\ell mn;\mu}
  &\equiv \f14 \mu S_\mu^{\ell+1}
  \sp{\p{1+\mu S_\z}^m-\p{\mu S_\z}^m} S_\bmu^n, \\
  L_{\ell mn;\mu}
  &\equiv -\mu n S_\mu^\ell \sp{S-\ell-\f34\p{n-1}-\f12\mu S_\z}
  \p{\mu S_\z}^m S_\bmu^{n-1} \notag \\
  &\qquad - \mu\ell n\p{n-1} S_\mu^{\ell-1}
  \p{S+\mu S_\z} \p{-1+\mu S_\z}^m S_\bmu^{n-2}, \\
  M_{\ell mn;\mu}
  &\equiv \mu n S_\mu^\ell \sp{\p{S+\mu S_\z}\p{-1+\mu S_\z}^m
    - \f12\p{\f12\sp{n-1}+\mu S_\z}\p{\mu S_\z}^m} S_\bmu^{n-1}.
\end{align}
Defining for completion
\begin{align}
  P_{\ell mn;\mu}
  &\equiv \sum_j s_\mu^{(j)} S_\mu^\ell
  \p{\mu S_\z}^m S_\bmu^n s_\mu^{(j)} \\
  &= n S_\mu^{\ell+1} \p{\mu S_\z}^m S_\bmu^{n-1}
  - n\p{n-1} S_\mu^\ell \p{S+\mu S_\z} \p{-1+\mu S_\z}^m S_\bmu^{n-2},
\end{align}
and
\begin{align}
  Q_{\ell mn;\mu}^{(\gamma)}
  \equiv \gamma_\bmu^* \gamma_\mu P_{\ell mn;\mu}
  + \p{\gamma_\z^* \gamma_\mu + \gamma_\bmu^* \gamma_\z}
  K_{\ell mn;\mu}
  + \gamma_\z^* \gamma_\mu L_{\ell mn;\mu}
  + \gamma_\bmu^* \gamma_\z M_{\ell mn;\mu},
  \label{eq:Q_single}
\end{align}
we finally have
\begin{align}
  \D\p{\gamma; S_\mu^\ell \p{\mu S_\z}^m S_\bmu^n}
  = \sum_{X\in\set{\z,+,-}} \abs{\gamma_X}^2
  \D\p{s_X; S_\mu^\ell \p{\mu S_\z}^m S_\bmu^n}
  + Q_{\ell mn;\mu}^{(\gamma)} + \sp{Q_{nm\ell;\mu}^{(\gamma)}}^\dag.
\end{align}
Note that the sum $\ell+m+n$ for operators
$S_\mu^\ell \p{\mu S_\z}^m S_\bmu^n$ does not grow under this type of
decoherence.


\section{Sandwich identities for collective-spin decoherence
  calculations}
\label{sec:sandwich_collective}

In analogy with the work in Appendix \ref{sec:sandwich_single}, in
this section we work out sandwich identities necessary for
collective-spin decoherence calculations.  The simplest cases are
\begin{align}
  S_\mu S_\mu^\ell \p{\mu S_\z}^m S_\bmu^n S_\bmu
  &= S_\mu^{\ell+1} \p{\mu S_\z}^m S_\bmu^{n+1},
  \allowdisplaybreaks \\
  S_\mu S_\mu^\ell \p{\mu S_\z}^m S_\bmu^n S_\z
  &= \mu S_\mu^{\ell+1} \p{n+\mu S_\z} \p{\mu S_\z}^m S_\bmu^n,
  \allowdisplaybreaks \\
  S_\z S_\mu^\ell \p{\mu S_\z}^m S_\bmu^n S_\z
  &= S_\mu^\ell \sp{\ell n + \p{\ell+n} \mu S_\z + \p{\mu S_\z}^2}
  \p{\mu S_\z}^m S_\bmu^n.
\end{align}
With a bit more work, we can also find
\begin{align}
  S_\mu^\ell \p{\mu S_\z}^m S_\bmu^n S_\mu
  &= S_\mu^{\ell+1} \p{1+\mu S_\z}^m S_\bmu^n
  - n S_\mu^\ell \p{n-1+2\mu S_\z} \p{\mu S_\z}^m S_\bmu^{n-1},
\end{align}
which implies
\begin{align}
  S_\mu S_\mu^\ell \p{\mu S_\z}^m S_\bmu^n S_\mu
  &= S_\mu^{\ell+2} \p{1+\mu S_\z}^m S_\bmu^n
  - n S_\mu^{\ell+1} \p{n-1+2\mu S_\z} \p{\mu S_\z}^m S_\bmu^{n-1},
  \allowdisplaybreaks \\
  S_\z S_\mu^\ell \p{\mu S_\z}^m S_\bmu^n S_\mu
  &= \mu S_\mu^{\ell+1} \p{\ell+1+\mu S_\z} \p{1+\mu S_\z}^m S_\bmu^n
  \notag \\
  &\qquad - \mu n S_\mu^\ell
  \sp{\ell\p{n-1} + \p{2\ell+n-1}\mu S_\z + 2\p{\mu S_\z}^2}
  \p{\mu S_\z}^m S_\bmu^{n-1}.
\end{align}
Finally, we compute
\begin{align}
  S_\bmu S_\mu^\ell \p{\mu S_\z}^m S_\bmu^n S_\mu
  &= \sp{S_\mu^\ell S_\bmu - \ell S_\mu^{\ell-1} \p{\ell-1+2\mu S_\z}}
  \p{\mu S_\z}^m
  \sp{S_\mu S_\bmu^n - n \p{n-1+2\mu S_\z} S_\bmu^{n-1}} \notag \\
  &= S_\mu^\ell S_\bmu \p{\mu S_\z}^m S_\mu S_\bmu^n \notag \\
  &\qquad - S_\mu^\ell
  \sp{\ell\p{\ell+1} + n\p{n+1}+2\p{\ell+n}\mu S_\z}
  \p{1+\mu S_\z}^m S_\bmu^n \notag \\
  &\qquad + \ell n S_\mu^{\ell-1}
  \sp{\p{\ell-1}\p{n-1}+2\p{\ell+n-2}\mu S_\z + 4\p{\mu S_\z}^2}
  \p{\mu S_\z}^m S_\bmu^{n-1},
\end{align}
where
\begin{multline}
  S_\bmu \p{\mu S_\z}^m S_\mu
  = S_\bmu S_\mu \p{1+\mu S_\z}^m
  = \p{S_\mu S_\bmu - 2\mu S_\z} \p{1+\mu S_\z}^m \\
  = S_\mu \p{2+\mu S_\z}^m S_\bmu - 2\mu S_\z \p{1+\mu S_\z}^m,
\end{multline}
so
\begin{align}
  S_\bmu S_\mu^\ell \p{\mu S_\z}^m S_\bmu^n S_\mu
  &= S_\mu^{\ell+1} \p{2+\mu S_\z}^m S_\bmu^{n+1} \notag \\
  &\qquad - S_\mu^\ell
  \sp{\ell\p{\ell+1} + n\p{n+1}+2\p{\ell+n+1}\mu S_\z}
  \p{1+\mu S_\z}^m S_\bmu^n \notag \\
  &\qquad + \ell n S_\mu^{\ell-1}
  \sp{\p{\ell-1}\p{n-1}+2\p{\ell+n-2}\mu S_\z + 4\p{\mu S_\z}^2}
  \p{\mu S_\z}^m S_\bmu^{n-1}.
\end{align}


\section{Collective spin decoherence}
\label{sec:decoherence_collective}

In this section we work out the effects of decoherence via collective
jump operators on ordered products of collective spin operators.  For
shorthand, we define
\begin{align}
  \D\p{\Gamma;\O}
  \equiv \D\p{\set{\Gamma};\O}
  = \Gamma^\dag \O \Gamma - \f12\sp{\Gamma^\dag \Gamma, \O}_+,
\end{align}
where $\Gamma$ is a collective spin operator with unit weight, i.e.~a
linear combination of $S_\z$, $S_\mu$, and $S_\bmu$.

\subsection{Decay-type decoherence and dephasing}
\label{sec:decay_dephasing_collective}

Making use of the results in Appendix \ref{sec:sandwich_collective},
we find that the effects of collective decay-type decoherence on
ordered products of collective spin operators are given by
\begin{align}
  \D\p{S_\bmu; S_\mu^\ell \p{\mu S_\z}^m S_\bmu^n}
  &= -S_\mu^{\ell+1} \sp{\p{1+\mu S_\z}^m - \p{\mu S_\z}^m}
  S_\bmu^{n+1} \notag \\
  &\qquad + \f12 S_\mu^\ell \sp{\ell\p{\ell-1} + n\p{n-1}
    + 2\p{\ell+n}\mu S_\z} \p{\mu S_\z}^m S_\bmu^n,
\end{align}
and
\begin{align}
  \D\p{S_\mu; S_\mu^\ell \p{\mu S_\z}^m S_\bmu^n}
  &= S_\mu^{\ell+1} \sp{\p{2+\mu S_\z}^m-\p{1+\mu S_\z}^m}
  S_\bmu^{n+1} \notag \\
  &\qquad - S_\mu^\ell
  \sp{\ell\p{\ell+1} + n\p{n+1}+2\p{\ell+n+1}\mu S_\z}
  \p{1+\mu S_\z}^m S_\bmu^n \notag \\
  &\qquad + \f12 S_\mu^\ell
  \sp{\ell\p{\ell+1} + n\p{n+1}+2\p{\ell+n+2}\mu S_\z}
  \p{\mu S_\z}^m S_\bmu^n \notag \\
  &\qquad + \ell n S_\mu^{\ell-1}
  \sp{\p{\ell-1}\p{n-1}+2\p{\ell+n-2}\mu S_\z + 4\p{\mu S_\z}^2}
  \p{\mu S_\z}^m S_\bmu^{n-1}.
\end{align}
Similarly, we can work out that the effect of collective spin
dephasing is given by
\begin{align}
  \D\p{S_\z; S_\mu^\ell \p{\mu S_\z}^m S_\bmu^n}
  = -\f12 \p{\ell-n}^2 S_\mu^\ell \p{\mu S_\z}^m S_\bmu^n.
\end{align}


\subsection{The general case}
\label{sec:general_collective}

More generally, we may need to consider collective-spin jump operators
of the form
\begin{align}
  \Gamma \equiv \Gamma_\z S_\z + \Gamma_+ S_+ + \Gamma_- S_-,
\end{align}
whose decoherence effects are determined by
\begin{align}
  \Gamma^\dag \O \Gamma
  = \abs{\Gamma_\z}^2 S_\z \O S_\z
  + \sum_\mu \p{\abs{\Gamma_\mu}^2 S_\bmu \O S_\mu
    + \Gamma_\bmu^* \Gamma_\mu S_\mu \O S_\mu
    + \Gamma_\z^* \Gamma_\mu S_\z \O S_\mu
    + \Gamma_\bmu^* \Gamma_\z S_\mu \O S_\z},
\end{align}
and
\begin{align}
  \Gamma^\dag \Gamma
  = \abs{\Gamma_\z}^2 S_\z^2
  + \sum_\mu \p{\abs{\Gamma_\mu}^2 S_\bmu S_\mu
    + \Gamma_\z^*\Gamma_\mu S_\z S_\mu
    + \Gamma_\bmu^* \Gamma_\z S_\mu S_\z
    + \Gamma_\bmu^* \Gamma_\mu S_\mu^2},
\end{align}
which implies
\begin{align}
  \D\p{\Gamma;\O}
  &= \sum_{X\in\set{\z,+,-}} \abs{\Gamma_X}^2 \D\p{S_X;\O}
  + \sum_\mu \p{\Gamma_\bmu^* \Gamma_\mu S_\mu \O S_\mu
    + \Gamma_\z^* \Gamma_\mu S_\z \O S_\mu
    + \Gamma_\bmu^* \Gamma_\z S_\mu \O S_\z}
  \notag \\
  &\qquad -\f12 \sum_\mu\p{\Gamma_\bmu^* \Gamma_\mu \sp{S_\mu^2, \O}_+
    + \Gamma_\z^*\Gamma_\mu \sp{S_\z S_\mu, \O}_+
    + \Gamma_\bmu^* \Gamma_\z \sp{S_\mu S_\z, \O}_+}.
\end{align}
In order to compute the effect of this decoherence on ordered products
of collective spin operators, we expand the anti-commutators
\begin{align}
  \sp{S_\mu^2, S_\mu^\ell \p{\mu S_\z}^m S_\bmu^n}_+
  &= S_\mu^{\ell+2} \sp{\p{2+\mu S_\z}^m+\p{\mu S_\z}^m} S_\bmu^n
  - 2n S_\mu^{\ell+1} \p{n+2\mu S_\z} \p{1+\mu S_\z}^m S_\bmu^{n-1}
  \notag \\
  &\qquad + n\p{n-1} S_\mu^\ell \sp{\p{n-1}\p{n-2}
    + 2\p{2n-3}\mu S_\z + 4\p{\mu S_\z}^2} \notag \\
  &\qquad\qquad\qquad\qquad \times \p{\mu S_\z}^m S_\bmu^{n-2},
  \allowdisplaybreaks \\
  \sp{S_\z S_\mu, S_\mu^\ell \p{\mu S_\z}^m S_\bmu^n}_+
  &= \mu S_\mu^{\ell+1} \sp{\p{\ell+1+\mu S_\z}\p{\mu S_\z}^m
    + \p{n+1+\mu S_\z} \p{1+\mu S_\z}^m } S_\bmu^n \notag \\
  &\qquad - \mu n S_\mu^\ell \sp{n \p{n-1}
    + \p{3n-1}\mu S_\z + 2\p{\mu S_\z}^2} \p{\mu S_\z}^m S_\bmu^{n-1},
  \allowdisplaybreaks \\
  \sp{S_\mu S_\z, S_\mu^\ell \p{\mu S_\z}^m S_\bmu^n}_+
  &= \mu S_\mu^{\ell+1} \sp{\p{\ell+\mu S_\z}\p{\mu S_\z}^m
    + \p{n+\mu S_\z} \p{1+\mu S_\z}^m} S_\bmu^n \notag \\
  &\qquad - \mu n S_\mu^\ell \sp{\p{n-1}^2
    + 3\p{n-1}\mu S_\z + 2\p{\mu S_\z}^2} \p{\mu S_\z}^m S_\bmu^{n-1}.
\end{align}
Collecting terms and defining
\begin{align}
  \Gamma_{\z,\mu}^{(\pm)}
  &\equiv \f12\p{\Gamma_\z^* \Gamma_\mu \pm \Gamma_\bmu^* \Gamma_\z},
  \allowdisplaybreaks \\
  \tilde L_{\ell mn;\mu}^{(\Gamma)}
  &\equiv \mu \sp{\p{\ell-n+\f12} \Gamma_{\z,\mu}^{(+)}
    + \p{\ell+\f12} \Gamma_{\z,\mu}^{(-)}}
  S_\mu^{\ell+1} \p{1+\mu S_\z}^m S_\bmu^n \notag \\
  &\qquad -\mu \sp{\p{\ell-n+\f12} \Gamma_{\z,\mu}^{(+)}
    + \p{n+\f12} \Gamma_{\z,\mu}^{(-)}}
  S_\mu^{\ell+1} \p{\mu S_\z}^m S_\bmu^n \notag \\
  &\qquad + \mu \Gamma_{\z,\mu}^{(-)}
  S_\mu^{\ell+1} \mu S_\z
  \sp{\p{1+\mu S_\z}^m - \p{\mu S_\z}^m} S_\bmu^n,
  \allowdisplaybreaks \\
  \tilde M_{\ell mn;\mu}^{(\Gamma)}
  &= -\mu n\p{n-1} \sp{\p{\ell-n+\f12} \Gamma_{\z,\mu}^{(+)}
    + \p{\ell-\f12} \Gamma_{\z,\mu}^{(-)}}
  S_\mu^\ell \p{\mu S_\z}^m S_\bmu^{n-1} \notag \\
  &\qquad - 2\mu n \sp{\p{\ell-n+\f12} \Gamma_{\z,\mu}^{(+)}
    + \p{\ell+\f12n-1} \Gamma_{\z,\mu}^{(-)}}
  S_\mu^\ell \p{\mu S_\z}^{m+1} S_\bmu^{n-1} \notag \\
  &\qquad - 2\mu n \Gamma_{\z,\mu}^{(-)}
  S_\mu^\ell \p{\mu S_\z}^{m+2} S_\bmu^{n-1},
  \allowdisplaybreaks \\
  \tilde P_{\ell mn;\mu}
  &\equiv -\f12 S_\mu^{\ell+2}
  \sp{\p{2+\mu S_\z}^m - 2\p{1+\mu S_\z}^m + \p{\mu S_\z}^m}
  S_\bmu^n \notag \\
  &\qquad + n S_\mu^{\ell+1} \sp{\p{n+2\mu S_\z} \p{1+\mu S_\z}^m
    - \p{n-1+2\mu S_\z} \p{\mu S_\z}^m}
  S_\bmu^{n-1} \notag \\
  &\qquad -n\p{n-1} S_\mu^\ell
  \sp{\f12\p{n-1}\p{n-2} + \p{2n-3}\mu S_\z + 2\p{\mu S_\z}^2}
  \p{\mu S_\z}^m S_\bmu^{n-2},
  \allowdisplaybreaks \\
  \tilde Q_{\ell mn;\mu}^{(\Gamma)}
  &\equiv \Gamma_\bmu^* \Gamma_\mu \tilde P_{\ell mn;\mu}
  + \tilde L_{\ell mn;\mu}^{(\Gamma)}
  + \tilde M_{\ell mn;\mu}^{(\Gamma)},
\end{align}
we then have
\begin{align}
  \D\p{\Gamma; S_\mu^\ell \p{\mu S_\z}^m S_\bmu^n}
  = \sum_{X\in\set{\z,+,-}} \abs{\Gamma_X}^2
  \D\p{S_X; S_\mu^\ell \p{\mu S_\z}^m S_\bmu^n}
  + \tilde Q_{\ell mn;\mu}^{(\Gamma)}
  + \sp{\tilde Q_{nm\ell;\mu}^{(\Gamma)}}^\dag.
\end{align}
Note that the sum $\ell+m+n$ for operators
$S_\mu^\ell \p{\mu S_\z}^m S_\bmu^n$ grows by one if $\Gamma_\mu\ne0$
or $\Gamma_\bmu\ne0$, and does not grow otherwise.


\section{Initial conditions}
\label{sec:initial_conditions}

Initial conditions for all collective spin operators with respect to
(spin-polarized) Gaussian states in the Dicke manifold spanned by
$\set{\ket{m}:S_\z\ket{m}=m\ket{m}}$ can be determined from the
expansions
\begin{align}
  S_\z = \sum_{k=-S}^S k \op{k},
  &&
  S_\mu = \sum_{k=-S+\delta_{\mu,-1}}^{S-\delta_{\mu,1}}
  g_\mu\p{S,k} \op{k+\mu}{k},
\end{align}
with $\mu\in\set{+1,-1}$ and
\begin{align}
  g_\mu\p{S,k} \equiv \sqrt{\p{S-\mu k}\p{S+\mu k+1}}.
\end{align}
Given the initial states
\begin{align}
  \ket{\pm\Z} \equiv \ket{\pm S},
  &&
  \ket{\X} \equiv 2^{-N/2} \sum_{k=-S}^S
  { N \choose S+k }^{1/2} \ket{k},
\end{align}
where $S_\x\ket{\X}=S\ket{\X}$, in terms of $\nu\in\set{+1,-1}$ and
$\bmu\equiv-\mu$ we then have
\begin{align}
  \bk{\nu\Z|S_\mu^\ell \p{\mu S_\z}^m S_\bmu^n|\nu\Z}
  = \delta_{\ell n} \times
  \begin{cases}
     \p{S-n}^m \f{N! n!}{\p{N-n}!} & \mu = \nu, \\
     \delta_{n,0} \p{-S}^m & \mu \ne \nu,
  \end{cases}
\end{align}
1and
\begin{align}
  \bk{\X|S_\mu^\ell \p{\mu S_\z}^m S_\bmu^n|\X}
  &=  \f{\mu^m}{2^N} \sum_{k=-S+\delta_{\mu,-1}\max\set{\ell,n}}
  ^{S-\delta_{\mu,1}\max\set{\ell,n}} k^m
  \sp{{ N \choose S+k+\mu\ell } { N \choose S+k+\mu n }}^{1/2}
  \notag \\
  &\qquad\qquad\qquad\qquad
  \times \prod_{p=0}^{\ell-1} g_\mu\p{S,k+\mu p}
  \prod_{q=0}^{n-1} g_\bmu\p{S,k+\mu\sp{n-q}} \\
  &= \mu^m \f{N!}{2^N} \sum_{k=-S+\delta_{\mu,-1}\max\set{\ell,n}}
  ^{S-\delta_{\mu,1}\max\set{\ell,n}}
  \f{k^m \p{S-\mu k}!}
  {\p{S+\mu k}!\p{S-\ell-\mu k}!\p{S-n-\mu k}!} \\
  &= \f{N!}{2^N} \sum_{k=-S}^{S-\max\set{\ell,n}}
  \f{k^m \p{S-k}!}{\p{S+k}!\p{S-\ell-k}!\p{S-n-k}!} \\
  &= \p{-1}^m\f{N!}{2^N} \sum_{k=0}^{N-\max\set{\ell,n}}
  \f{\p{S-k}^m \p{N-k}!}{k!\p{N-\ell-k}!\p{N-n-k}!}.
\end{align}
Furthermore, using the fact that
\begin{align}
  e^{i\phi S_\z} S_\mu e^{-i\phi S_\z} = e^{i\mu\phi} S_\mu,
\end{align}
we can define the state
\begin{align}
  \ket{\tilde\phi} \equiv e^{-i\phi S_\z}\ket{\X},
  &&
  \p{\cos\phi S_\x + \sin\phi S_\y} \ket{\tilde\phi}
  = S \ket{\tilde\phi},
\end{align}
and say that
\begin{align}
  \bk{\tilde\phi|S_\mu^\ell \p{\mu S_\z}^m S_\bmu^n|\tilde\phi}
  = e^{i\mu\p{\ell-n}\phi} \bk{\X|S_\mu^\ell \p{\mu S_\z}^m S_\bmu^n|\X}.
\end{align}


\section{Changing bases for collective spin operators}

For a collective spin system composed of many small spins, the
operators $\S_{\v m}\equiv S_+^{m_+} S_\z^{m_\z} S_-^{m_-}$ contain
genuinely $\abs{\v m}$-body operators (for
$\abs{\v m}\equiv\sum_\mu m_\mu$), but also carry lots of ``baggage''
in the form of $k$-body operators with $k<\abs{\v m}$.  To cut down on
this overhead, we try to change our basis for collective spin
operators from $\S_{\v m}$ to the purely-$\abs{\v m}$-body operators
\begin{align}
  \tilde\S_{\v m} \equiv \sum_{\p{\v j;\v m}} P_{\v j},
  &&
  P_{\v j} \equiv \prod_{\mu,a} s_\mu^{\p{j^\mu_a}},
  \label{eq:S_P}
\end{align}
where $\v m \equiv \p{m_+,m_\z,m_-}\in\mathbb{N}_0^3$ with $m_\mu$ the
number of $s_\mu$ operators in each term of $\S_{\v m}$; $\v j$ is an
ordered list of indices for the spins addressed by $P_{\v j}$; and the
sum over $\p{\v j;\v m}$ denotes a sum over all possible $\v j$ for
which all indices in $\v j$ are distinct and
$\abs{\set{j^\mu_a}} = m_\mu$, such that $j^\mu_a$ indexes the $a$-th
spin addressed by a $s_\mu$ operator in $P_{\v j}$.

We now expand a product of basis operators $\tilde\S_{\v m}$ and
$\tilde\S_{\v n}$ into terms that have exactly $s$ spins addressed by
both operators:
\begin{align}
  \tilde\S_{\v m} \tilde\S_{\v n}
  = \sum_{s\ge0} \sum_{\p{\v j,\v k;\v m,\v n,s}} P_{\v j} P_{\v k},
  &&
  \sum_{\p{\v j,\v k;\v m,\v n,s}} X \equiv
  \sum_{\substack{\p{\v j;\v m},\p{\v k;\v n} \\
      \abs{\set{j_\alpha^\mu}\cap\set{k_\beta^\nu}}=s}} X.
  \label{eq:SS_PP}
\end{align}
Collecting terms in which $r_{\mu\nu}$ of the $s_\mu$ operators in
$P_{\v j}$ address the same spin as an $s_\nu$ operator in $P_{\v k}$,
we have
\begin{align}
  \tilde\S_{\v m} \tilde\S_{\v n}
  = \sum_{s\ge0} \sum_{\p{\v r;\v m,\v n,s}} f_{\v m\v n\v r}
  \sum_{\p{\v J,\v K;\v m,\v n,\v r}} P_{\v J} Q_{\v K},
  \label{eq:SS_PQ}
\end{align}
where the sum over $\p{\v r;\v m,\v n,s}$ denotes a sum over all
values of $\v r$ with the restrictions
\begin{align}
  r_{\mu\nu} \ge 0 ~\forall~ \mu,\nu,
  &&
  \sum_\nu r_{\mu\nu} \le m_\mu,
  &&
  \sum_\mu r_{\mu\nu} \le n_\nu,
  &&
  \sum_{\mu,\nu} r_{\mu\nu} = s;
  \label{eq:rest_r}
\end{align}
the factor $f_{\v m\v n\v r}$ counts the number of ways to pair
operators in $P_{\v j}$ and $P_{\v k}$ according to $\v r$, or
\begin{align}
  f_{\v m\v n\v r}
  &\equiv \sp{\prod_{\mu,\nu}
    { m_\mu - \sum_{\rho<\nu} r_{\mu\rho} \choose r_{\mu\nu} }
    { n_\mu - \sum_{\rho<\nu} r_{\rho\mu} \choose r_{\nu\mu} }}
  \sp{\prod_{\mu,\nu} r_{\mu\nu}!} \\
  &= \sp{\prod_\mu \f{m_\mu!}{\p{m_\mu-\sum_\rho r_{\mu\rho}}!}
    \f{n_\mu!}{\p{n_\mu-\sum_\rho r_{\rho\mu}}!}}
  \sp{\prod_{\mu,\nu} r_{\mu\nu}!}^{-1};
\end{align}
$\v J$ and $\v K$ are respectively ordered lists of indices for spins
addressed by one and two single-spin operators in a term of the
product $P_{\v j} P_{\v k}$ with fixed $\v r$; the sum over
$\p{\v J,\v K;\v m,\v n,\v r}$ denotes a sum over all values of
$\v J,\v K$ with the restrictions
\begin{align}
  \abs{\set{J_a^\mu}}
  = m_\mu + n_\mu - \sum_\rho\p{r_{\mu\rho}+r_{\rho\mu}},
  &&
  \abs{\set{K^{\mu\nu}_a}} = r_{\mu\nu},
  &&
  \set{J_a^\lambda} \cap \set{K_b^{\mu\nu}} = \varnothing,
\end{align}
with $\varnothing$ denoting the empty set; and finally, similarly to
$P_{\v j}$ in \eqref{eq:SS_P} we define
\begin{align}
  Q_{\v K} \equiv \prod_{\mu,\nu,a}
  s_\mu^{(K^{\mu\nu}_a)} s_\nu^{(K^{\mu\nu}_a)}.
\end{align}
As the sum in \eqref{eq:SS_PQ} is invariant under permutation of the
indices in $\v K$, we can safely neglect keeping track of these
indices and simply write
\begin{align}
  Q_{\v K}
  = \prod_{\mu,\nu} \prod_{a=1}^{r_{\mu\nu}}
  s_\mu^{(K^{\mu\nu}_a)} s_\nu^{(K^{\mu\nu}_a)}
  \simeq \bigotimes_{\mu,\nu} \bigotimes_{a=1}^{r_{\mu\nu}} s_\mu s_\nu
  = \bigotimes_{\mu,\nu} \bigotimes_{a=1}^{r_{\mu\nu}}
  \sum_\rho \eta_{\mu\nu\rho} s_\rho,
  \label{eq:Q_K_eta}
\end{align}
where $\simeq$ denotes equality up to a re-indexing of spins; we have
introduced explicit dependence on
$r_{\mu\nu}=\abs{\set{K^{\mu\nu}_a}}$ for brevity; and
$\eta_{\mu\nu\rho}$ is a structure factor defined by
$s_\mu s_\nu=\sum_\rho\eta_{\mu\nu\rho}s_\rho$.  Unlike the sums over
$\mu,\nu$ in most of the work above, the sum over $\rho$ here includes
an index for the identity operator $s_0\equiv\1$.  Distributing the
product of sums into a sum of products gives us
\begin{align}
  Q_{\v K}
  \simeq \sum_{\v\rho} \bigotimes_{\mu,\nu,a}
  \eta_{\mu\nu\rho^{\mu\nu}_a} s_{\rho^{\mu\nu}_a}
  = \sum_{\v\rho} \sp{\prod_{\mu,\nu,a}
    \p{\eta_{\mu\nu\rho^{\mu\nu}_a}}^{\rho^{\mu\nu}_a}}
  \bigotimes_{\mu,\nu,a} s_{\rho^{\mu\nu}_a},
  \label{eq:Q_K_rho}
\end{align}
where implicitly $\abs{\set{\rho^{\mu\nu}_a}}=r_{\mu\nu}$, as
$\p{\mu,\nu,a}$ essentially index a factor in \eqref{eq:Q_K_eta} and
$\rho^{\mu\nu}_a$ indexes one of the terms in that factor.  Letting
$\tilde\rho^{\mu\nu}_\kappa$ denote the number of elements in
$\p{\rho^{\mu\nu}_1,\rho^{\mu\nu}_2,\cdots,\rho^{\mu\nu}_{r_{\mu\nu}}}$
that are equal to $\kappa$, we observe that two terms in
\eqref{eq:Q_K_rho} with, say, $\v\rho=\v\rho_1$ and $\v\rho=\v\rho_2$
are equal up to a permutation of indices if
$\tilde{\v\rho}_1=\tilde{\v\rho}_2$.  The degeneracy of terms in
\eqref{eq:Q_K_rho}, i.e.~the number of $\v\rho$ which are consistent
with $\tilde{\v\rho}$, is
\begin{align}
  g_{\tilde{\v\rho}}
  \equiv \prod_{\mu,\nu,\kappa}
  { r_{\mu\nu} - \sum_{\lambda<\kappa} \tilde\rho^{\mu\nu}_\lambda
    \choose \tilde\rho^{\mu\nu}_\kappa }
  = \prod_{\mu,\nu}
  \f{r_{\mu\nu}!}{\prod_\kappa\tilde\rho^{\mu\nu}_\kappa!}.
\end{align}
Defining additionally
\begin{align}
  \v\eta_{\tilde{\v\rho}}
  \equiv \prod_{\mu,\nu,\kappa}
  \p{\eta_{\mu\nu\kappa}}^{\tilde\rho^{\mu\nu}_\kappa},
  \label{eq:eta_rho}
\end{align}
we can collect like factors and group together equivalent terms in
\eqref{eq:Q_K_rho} to write
\begin{align}
  Q_{\v K}
  \simeq \sum_{\tilde{\v\rho}} g_{\tilde{\v\rho}} \v\eta_{\tilde{\v\rho}}
  \bigotimes_{\kappa}
  s_\kappa^{\otimes\sum_{\mu,\nu}\tilde\rho^{\mu\nu}_\kappa}
  \simeq \sum_{\tilde{\v\rho}} g_{\tilde{\v\rho}} \v\eta_{\tilde{\v\rho}}
  \1^{\otimes\sum_{\mu,\nu}\tilde\rho^{\mu\nu}_0}
  \otimes \bigotimes_{\kappa\ne0}
  s_\kappa^{\otimes\sum_{\mu,\nu}\tilde\rho^{\mu\nu}_\kappa},
  \label{eq:Q_K_rho_simp}
\end{align}
where again implicitly
$\sum_\kappa\tilde\rho^{\mu\nu}_\kappa=r_{\mu\nu}$ for consistency
with \eqref{eq:Q_K_eta} and \eqref{eq:Q_K_rho}, and we have explicitly
factored out the identity operators in each term of $Q_{\v K}$.

We have now simplified $Q_{\v K}$ sufficiently to substitute it back
into \eqref{eq:SS_PQ}, which (removing the tilde from $\tilde{\v\rho}$
to simplify notation) gives us
\begin{align}
  \tilde\S_{\v m} \tilde\S_{\v n}
  = \sum_{s\ge0} \sum_{\p{\v r;\v m,\v n,s}} f_{\v m\v n\v r}
  \sum_{\p{\v\rho;\v r}} g_{\v\rho} \v\eta_{\v\rho}
  c_{\v\ell_{\v m\v n\v r\v\rho}\v\rho}
  \tilde\S_{\v\ell_{\v m\v n\v r\v\rho}},
  \label{eq:SS_simp}
\end{align}
where the sum over $\p{\v\rho;\v r}$ denotes a sum over all values of
$\v\rho$ and with the restrictions
\begin{align}
  \abs{\set{\rho^{\mu\nu}_\kappa}} = \abs{\set{s_\kappa}},
  &&
  \sum_{\kappa} \rho^{\mu\nu}_\kappa = r_{\mu\nu};
\end{align}
the combinatorial factor
\begin{align}
  c_{\v\ell\v\rho}
  \equiv \f{\p{N - \abs{\v\ell}}!}
  {\p{N - \abs{\v\ell} - \sum_{\mu,\nu}\rho^{\mu\nu}_0}!}
\end{align}
accounts for the number of terms which are equivalent up to a
permutation of indices involving the identity operators in
\eqref{eq:Q_K_rho_simp}; and the number of $s_\mu$ operators in each
term of $\tilde\S_{\v\ell_{\v m\v n\v r\v\rho}}$ is
\begin{align}
  \ell_{\v m\v n\v r\v\rho;\mu}
  = m_\mu + n_\mu - \sum_\kappa \p{r_{\mu\kappa}+r_{\kappa\mu}}
  + \sum_{\alpha,\beta} \rho^{\alpha\beta}_\mu.
\end{align}

The sums in \eqref{eq:SS_simp} generally involve a large number of
terms; in practice, many of these terms will be equal to zero.  We
can, however, impose additional restrictions on these sums in such a
way as to throw out terms which vanish, and keep only those which do
not.  Remembering that $s$ counts the number of spins which are
addressed by both operators $P_{\v j}$ and $P_{\v k}$ in
\eqref{eq:SS_PP}, the first restriction we can impose comes from
recognizing that for given $\tilde\S_{\v m},\tilde\S_{\v n}$, the
overlap $s$ is bounded as
\begin{align}
  \max\set{0,\sum_\mu\p{m_\mu+n_\mu}-N}
  \le s \le \min\set{N,\sum_\mu\p{m_\mu+n_\mu}}.
\end{align}
In addition, the structure factor $\eta_{\mu\nu\kappa}$ may generally
contain zeros that we can preemptively avoid with restrictions on
$\v r$ and $\v\rho$.  Recalling that $r_{\mu\nu}$ counts the number of
operators addressed by
$s_\mu s_\nu=\sum_\kappa\eta_{\mu\nu\kappa} s_\kappa$, if
$\eta_{\mu\nu\kappa}=0$ for all $\kappa$, i.e.~$s_\mu s_\nu=0$, then
any term with $r_{\mu\nu}>0$ will vanish.  Our second restriction on
the sums in \eqref{eq:SS_simp} is therefore to fix $r_{\mu\nu}=0$ for
all $\mu,\nu$ with $s_\mu s_\nu=0$.  Finally, we can avoid the
nullifying effect of zeros from $\eta_{\mu\nu\kappa}$
$\v\eta_{\v\rho}\propto\p{\eta_{\mu\nu\kappa}}^{\rho^{\mu\nu}_\kappa}$
by fixing $\rho^{\mu\nu}_\kappa=0$ for all $\mu,\nu,\kappa$ with
$\eta_{\mu\nu\kappa}=0$.  These restrictions are sufficient to ensure
that all individual terms in \eqref{eq:SS_simp} are nonzero, although
they do not rule out the possibility of several terms canceling out.

TODO: comment on the use of this new basis vs.~the old one.

\end{document}