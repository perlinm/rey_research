\documentclass[aps,notitlepage,nofootinbib,11pt]{revtex4-1}

% linking references
\usepackage{hyperref}
\hypersetup{
  breaklinks=true,
  colorlinks=true,
  linkcolor=blue,
  filecolor=magenta,
  urlcolor=cyan,
}

%%% symbols, notations, etc.
\usepackage{physics,braket,bm,commath,amssymb} % physics and math
\renewcommand{\t}{\text} % text in math mode
\newcommand{\f}[2]{\dfrac{#1}{#2}} % shorthand for fractions
\newcommand{\p}[1]{\left(#1\right)} % parenthesis
\renewcommand{\sp}[1]{\left[#1\right]} % square parenthesis
\renewcommand{\set}[1]{\left\{#1\right\}} % curly parenthesis
\renewcommand{\v}{\bm} % bold vectors
\newcommand{\uv}[1]{\v{\hat{#1}}} % unit vectors
\renewcommand{\c}{\cdot} % inner product
\newcommand{\bk}{\Braket} % shorthand for braket notation
\renewcommand{\d}{\text{d}} % "d" for integration measure

\newcommand{\D}{\mathcal{D}}
\renewcommand{\O}{\mathcal{O}}

\usepackage{dsfont} % for identity operator
\newcommand{\1}{\mathds{1}}

\usepackage[inline]{enumitem} % for inline enumeration

%%% figures
\usepackage{graphicx} % for figures
\usepackage{grffile} % help latex properly identify figure extensions
\usepackage[caption=false]{subfig} % subfigures (via \subfloat[]{})
\graphicspath{{./figures/}} % set path for all figures

% for strikeout text
% normalem included to prevent underlining titles in the bibliography
\usepackage[normalem]{ulem}

% for leaving notes in the text
\newcommand{\note}[1]{\textcolor{red}{#1}}


\begin{document}

\title{Solving Collective Spin Hamiltonians with Decoherence}

\author{Michael A. Perlin}

\maketitle

Previous work has shown how to realize various spin-squeezing
Hamiltonians with ultracold atomic systems.  While it is
straightforward to simulate the dynamics generated by these
Hamiltonians in the context of a closed quantum system, here we try to
develop an understanding of their the squeezing behavior in the
presence of decoherence.  In particular, we will be considering
dynamics induced by the collective one-axis twisting (OAT), two-axis
twisting (TAT), and transverse field (TVF) Hamiltonians of the form
\begin{align}
  H_{\t{OAT}} = \chi S_z^2,
  &&
  H_{\t{TAT}} = \chi \p{S_z^2 - S_y^2},
  &&
  H_{\t{TVF}} = \chi S_z^2 - \Omega S_x,
\end{align}
for spins which decay in an uncorrelated manner at a rate $\gamma$.
The evolution of any Heisenburg operator $\O$ with a Hamiltonian $H$
is given by
\begin{align}
  \f{d}{dt} \O
  = i\sp{H,\O}_- + \gamma\D\p{\O},
  &&
  \D\p{\O} \equiv \sum_j\p{\sigma_j^+\O\sigma_j^-
    - \f12\sp{\sigma_j^+\sigma_j^-,\O}_+},
  \label{eq:EOM}
\end{align}
where $\sp{A,B}_\pm\equiv AB\pm BA$ and $j$ indexes an individual
spin.  Our measure of squeezing is
\begin{align}
  \xi^2 \equiv \f{N}{\abs{\bk{\v S}}^2}
  \min_{\uv n\perp\bk{\v S}} \bk{\p{\v S\c\uv n}^2},
  \label{eq:squeezing}
\end{align}
where $N$ is the total number of spins, $\v S$ is the collective spin
vector, and the minimization is performed over all unit vectors
$\uv n$ in the plane orthogonal to the mean spin vector $\bk{\v S}$.
This squeezing parameter is entirely determined by components
$\bk{S_\alpha}$ of the mean spin vector in addition to collective
spin-spin correlators of the form $\bk{S_\alpha S_\beta}$; our task is
therefore to compute these expectation values in all scenarios of
interest.


\section{One-axis twisting}

The case of the one-axis twisting is simplest to consider, as we can
write the Hamiltonian in the form
\begin{align}
  H_{\t{OAT}} = \f14 \chi \sum_{j,k} \sigma_j^z \sigma_k^z
  = \f12 \chi \sum_{j<k} \sigma_j^z \sigma_k^z + \f14 N \chi,
\end{align}
which admits an exact solution previously worked out by Foss-Feig et
al.~in ref.~\cite{foss-feig2013nonequilibrium}.  Adapting exact
expectation values with respect to the initial state $\ket{X}$ which
has all spins pointing in $x$, in terms of $\mu,\nu\in\set{+1,-1}$ we
find that
\begin{align}
  \bk{S_+}
  &= \sum_j\bk{\sigma_j^+}
  = \f{N}{2} e^{-\gamma t/2} \Phi\p{\chi,t}^{N-1}, \label{eq:S+_OAT} \\
  \bk{S_\mu S_z}
  &= \f12\sum_j\bk{\sigma_j^\mu\sigma_j^z}
  + \f12\sum_{j\ne k} \bk{\sigma_j^\mu \sigma_k^z} \\
  &= -\f{\mu}{2}\bk{S_\mu} + \f14 N \p{N-1} e^{-\gamma t/2}
  \Psi\p{\mu\chi,t} \Phi\p{\chi,t}^{N-2}, \\
  \bk{S_\mu S_\nu}
  &= \sum_j \bk{\sigma_j^\mu \sigma_j^\nu}
  + \sum_{j\ne k} \bk{\sigma_j^\mu \sigma_k^\nu} \\
  &= \delta_{\mu,-\nu} \p{\f{N}{2} + \mu \bk{S_z}}
  + \f14 N \p{N-1} e^{-\gamma t}
  \Phi\p{\sp{\mu+\nu}\chi,t}^{N-2}, \label{eq:SS+-_OAT}
\end{align}
where
\begin{align}
  \Phi\p{\zeta,t} \equiv e^{-\gamma t/2}
  \sp{\cos\p{s_\zeta t}+\f12\gamma t~\t{sinc}\p{s_\zeta t}},
  &&
  \Psi\p{\zeta,t} \equiv e^{-\gamma t/2}
  \p{is_\zeta-\gamma/2}t~ \t{sinc}\p{s_\zeta t},
\end{align}
and $s_\zeta \equiv \zeta + i2\gamma$.  The last ingredients we need
to compute spin squeezing at any time are $\bk{S_z}$ and $\bk{S_z^2}$.
As these operators commute with the OAT Hamiltonian, their evolution
is governed entirely by decoherence, such that
\begin{align}
  \f{d}{dt}\p{S_z^n}
  = \gamma \D\p{S_z^n}
  = \gamma \p{S + S_z} \sum_{k=0}^{n-1}
  { n \choose k } \p{-1}^{n-k} S_z^k,
\end{align}
where $S\equiv N/2$ and $\D\p{S_z^n}$ is worked out in Appendix
\ref{sec:decoherence}.  In particular,
\begin{align}
  \f{d}{dt} S_z = -\gamma \p{S + S_z},
  &&
  \f{d}{dt}\p{S_z^2} = \gamma \p{S + S_z} \p{1 - 2S_z}.
\end{align}
The initial conditions $\bk{S_z}_{t=0}=0$ and $\bk{S_z^2}_{t=0}=S/2$
then imply
\begin{align}
  \bk{S_z} = S\p{e^{-\gamma t}-1},
  &&
  \bk{S_z^2} = S \p{1 - \f12 e^{-\gamma t}} e^{-\gamma t} + \bk{S_z}^2.
  \label{eq:Sz_OAT}
\end{align}
The expectation values in \eqref{eq:S+_OAT}-\eqref{eq:SS+-_OAT} and
\eqref{eq:Sz_OAT} are sufficient to compute the squeezing parameter
$\xi^2$ in \eqref{eq:squeezing} at any time throughout evolution of
the initial state $\ket{X}$ under $H_{\t{OAT}}$.


\subsection{Cumulant expansion}

Unlike the one-axis twisting Hamiltonian, the two-axis twisting and
transverse-field Hamiltonians do not admit exact analytical
solutions. We will therefore solve these Hamiltonians approximately
using cumulant expansions.  The basic idea is to find equations of
motion for operators of interest, and at some point truncate
high-order terms to arrive at a closed, reasonably-sized system of
equations.  To benchmark this method, we will test it against exact
one-axis twisting results.

In order to find equations of motion for $S_\mu$, $S_\mu S_z$, and
$S_\mu S_\nu$ (i.e.~where $\mu,\nu\in\set{+1,-1}$), we first compute
the commutator
\begin{align}
  \sp{S_z^2, S_+^\ell S_-^m S_z^n}_-
  = \p{S_z^2 S_+^\ell S_-^m - S_+^\ell S_-^m S_z^2} S_z^n,
\end{align}
where we can use the commutation relation
$\sp{S_z,S_\mu^k}_-=\mu k S_\mu$ (see Appendix \ref{sec:identities})
to push spin-$z$ operators (i.e.~$S_z$) to the right, as
\begin{align}
  S_z S_+^\ell S_-^m
  = S_+^\ell \p{\ell + S_z} S_-^m
  = \ell S_+^\ell S_-^m + S_+^\ell S_z S_-^m
  = S_+^\ell S_-^m \p{\ell - m + S_z},
\end{align}
which implies
\begin{align}
  \sp{S_z^2, S_+^\ell S_-^m S_z^n}_-
  = S_+^\ell S_-^m \p{\sp{\ell - m + S_z}^2 - S_z^2} S_z^n
  = S_+^\ell S_-^m \p{\sp{\ell-m}^2 + 2\sp{\ell-m} S_z} S_z^n.
\end{align}
For the OAT Hamiltonian, we then have that
\begin{align}
  \f{d}{dt} \p{S_+^\ell S_-^m S_z^n}
  = i\chi S_+^\ell S_-^m \p{\sp{\ell-m}^2 + 2\sp{\ell-m} S_z} S_z^n
  + \gamma \D\p{S_+^\ell S_-^m S_z^n},
\end{align}
where $\D\p{S_+^\ell S_-^m S_z^n}$ is worked out in Appendix
\ref{sec:decoherence}.  In order to compute squeezing of the OAT
Hamiltonian in the presence of single-spin decay, we therefore need to
solve the systems of equations defined by
\begin{align}
  \f{d}{dt} \p{S_\mu S_z^n}
  = i\chi S_\mu \p{1 + 2\mu S_z} S_z^n
  + \gamma S_\mu \sp{\p{S - \delta_{\mu,-1} + S_z} \sum_{k=0}^{n-1}
    { n \choose k } \p{-1}^{n-k} S_z^k - \f12 S_z^n}
\end{align}
and
\begin{multline}
  \f{d}{dt} \p{S_\mu S_\nu S_z^n}
  = i\chi S_\mu S_\nu \p{\sp{\mu+\nu}^2 + 2 \sp{\mu+\nu} S_z} S_z^n \\
  + \gamma S_\mu S_\nu
  \sp{\p{S - \delta_{\mu,-1} - \delta_{\nu,-1} + S_z} \sum_{k=0}^{n-1}
    { n \choose k } \p{-1}^{n-k} S_z^k - S_z^n},
\end{multline}
for $\mu\ge\nu$.


\bibliography{\jobname}

\newpage
\appendix

\section{Spin operator identities}
\label{sec:identities}

This text frequently makes use of various spin operator identities; we
collect some particularly useful identities here for reference.
\begin{align}
  \f12 \sp{\sigma_j^z,\sigma_k^\mu}_- = \delta_{jk} \mu \sigma_j^\mu,
  &&
  \sp{S_z,\sigma_j^\mu}_- = \f12 \sp{\sigma_j^z,S_\mu}_-
  = \mu \sigma_j^\mu,
  &&
  \sp{S_z,S_\mu}_- = \mu S_\mu
  \label{eq:comm_z_mu}
\end{align}
\begin{align}
  \sp{\sigma_j^\mu,\sigma_k^\nu}_-
  = \delta_{jk} \delta_{\mu,-\nu} \mu \sigma_j^z,
  &&
  \sp{S_\mu,\sigma_j^\nu}_- = \sp{\sigma_j^\mu,S_\nu}_-
  = \delta_{\mu,-\nu} \mu \sigma_j^z,
  &&
  \sp{S_\mu,S_\nu}_- = 2\delta_{\mu,-\nu} \mu S_z
  \label{eq:comm_mu_nu}
\end{align}
\begin{align}
  S_z S_\mu^n
  = \p{\mu S_\mu + S_\mu S_z} S_\mu^{n-1}
  = \mu S_\mu^n + S_\mu S_z S_\mu^{n-1}
  = 2\mu S_\mu^n + S_\mu^2 S_z S_\mu^{n-2}
  = S_\mu^n \p{\mu n + S_z}
  \label{eq:comm_z_mu_n}
\end{align}
\begin{align}
  \sigma_j^+ \sigma_j^- = \f12\p{\1_j+\sigma_j^z}
  &&
  \sum_j \sigma_j^+ \sigma_j^- = S + S_z.
\end{align}
Here $S\equiv N/2$.


\section{Decoherence of Heisenberg spin operators}
\label{sec:decoherence}

The decoherence operator for uncorrelated single-spin decay at a
uniform rate is
\begin{align}
  \D\p{\O}
  \equiv \sum_j\p{\sigma_j^+\O\sigma_j^-
    - \f12\sp{\sigma_j^+\sigma_j^-,\O}_+}
  = \sum_j \sigma_j^+\O\sigma_j^- - S \O - \f12 \sp{S_z,\O}_+.
\end{align}
We are interested in the effect of this decoherence on the dynamics of
collective spin operators, all of which are determined by operators of
the form $S_+^\ell S_-^m S_z^n$.  We thus expand
\begin{align}
  \D\p{S_+^\ell S_-^m S_z^n}
  = \sum_j \sigma_j^+ S_+^\ell S_-^m S_z^n \sigma_j^-
  - S S_+^\ell S_-^m S_z^n - \f12 \sp{S_z, S_+^\ell S_-^m S_z^n}_+.
  \label{eq:D_z+-}
\end{align}
Considering the first term, we can use a commutation relations from
\eqref{eq:comm_z_mu} to push through and simplify the jump operators
as
\begin{align}
  \sum_j \sigma_j^+ S_+^\ell S_-^m S_z^n \sigma_j^-
  = \sum_j S_+^\ell \sigma_j^+ \sigma_j^- S_-^m \p{S_z-1}^n
  = S_+^\ell \p{S + S_z} S_-^m
  \sum_{k=0}^n { n \choose k } \p{-1}^{n-k} S_z^k.
\end{align}
We now need to push the $S_-^m$ to the left, which we can do using the
commutation relation $\sp{S_z,S_\mu^m}=\mu mS_\mu$ derived in
\eqref{eq:comm_z_mu_n}:
\begin{align}
  \sum_j \sigma_j^+ S_+^\ell S_-^m S_z^n \sigma_j^-
  = S_+^\ell S_-^m \p{S - m + S_z}
  \sum_{k=0}^n { n \choose k } \p{-1}^{n-k} S_z^k.
\end{align}
Considering now the last term in \eqref{eq:D_z+-}, we can expand
\begin{align}
  \f12 \sp{S_z, S_+^\ell S_-^m S_z^n}_+
  = \f12 \p{S_z S_+^\ell S_-^m + S_+^\ell S_-^m S_z} S_z^n,
\end{align}
where we can again use commutation relation
$\sp{S_z,S_\mu^m}=\mu mS_\mu$ to push $S_z$ to the right and get
\begin{align}
  S_z S_+^\ell S_-^m
  = S_+^\ell \p{\ell + S_z} S_-^m
  = S_+^\ell S_-^m \p{\ell - m + S_z},
\end{align}
which implies
\begin{align}
  \f12 \sp{S_z, S_+^\ell S_-^m S_z^n}_+
  = \f12 S_+^\ell S_-^m \p{\ell - m + 2 S_z} S_z^n
  = S_+^\ell S_-^m \p{\f12\sp{\ell-m} + S_z} S_z^n.
\end{align}
Putting everything together, we have that
\begin{align}
  \D\p{S_+^\ell S_-^m S_z^n}
  &= S_+^\ell S_-^m \sp{\p{S - m + S_z} \sum_{k=0}^n
    { n \choose k } \p{-1}^{n-k} S_z^k
    - \p{S + \f12\sp{\ell-m} + S_z} S_z^n} \\
  &= S_+^\ell S_-^m \sp{\p{S - m + S_z} \sum_{k=0}^{n-1}
    { n \choose k } \p{-1}^{n-k} S_z^k - \f12 \p{\ell+m} S_z^n}.
\end{align}

\end{document}
