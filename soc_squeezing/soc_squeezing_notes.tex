\documentclass[aps,notitlepage,nofootinbib,11pt]{revtex4-1}

% linking references
\usepackage{hyperref}
\hypersetup{
  breaklinks=true,
  colorlinks=true,
  linkcolor=blue,
  filecolor=magenta,
  urlcolor=cyan,
}

%%% symbols, notations, etc.
\usepackage{physics,braket,bm,commath,amssymb} % physics and math
\renewcommand{\t}{\text} % text in math mode
\newcommand{\f}[2]{\dfrac{#1}{#2}} % shorthand for fractions
\newcommand{\p}[1]{\left(#1\right)} % parenthesis
\renewcommand{\sp}[1]{\left[#1\right]} % square parenthesis
\renewcommand{\set}[1]{\left\{#1\right\}} % curly parenthesis
\renewcommand{\v}{\bm} % bold vectors
\renewcommand{\c}{\cdot} % inner product
\newcommand{\bk}{\braket} % shorthand for braket notation
\newcommand{\Bk}{\Braket} % shorthand for braket notation

\renewcommand{\d}{\text{d}} % "d" for integration measure
\newcommand{\g}{\text{g}} % ground / excited electronic states
\newcommand{\e}{\text{e}}

\newcommand{\B}{\mathcal{B}}
\newcommand{\D}{\mathcal{D}}
\newcommand{\E}{\mathcal{E}}
\newcommand{\G}{\mathcal{G}}
\newcommand{\I}{\mathcal{I}}
\newcommand{\J}{\mathcal{J}}
\renewcommand{\L}{\mathcal{L}}
\renewcommand{\O}{\mathcal{O}}
\renewcommand{\P}{\mathcal{P}}
\newcommand{\Q}{\mathcal{Q}}

\usepackage{dsfont} % for identity operator
\newcommand{\1}{\mathds{1}}


\usepackage[inline]{enumitem} % for inline enumeration

%%% figures
\usepackage{graphicx} % for figures
\usepackage{grffile} % help latex properly identify figure extensions
\usepackage[caption=false]{subfig} % subfigures (via \subfloat[]{})
\graphicspath{{./figures/}} % set path for all figures

% for strikeout text
% normalem included to prevent underlining titles in the bibliography
\usepackage[normalem]{ulem}

% for leaving notes in the text
\newcommand{\note}[1]{\textcolor{red}{#1}}



\begin{document}

\title{Spin-orbit-coupling-induced squeezing in the optical lattice
  clock}

\author{Michael A. Perlin}

\maketitle

These are some notes about trying to induce spin squeezing in an
optical lattice clock by combining spin-orbit coupling, interactions,
and external drives.  We assume, for now, that
\begin{enumerate*}[label=(\roman*)]
\item atoms are fermionic in nature with an alkali-earth(-like)
  electronic structure,
\item atoms are nuclear-spin-polarized, and
\item the optical lattice is quasi-one-dimensional, with tight
  (i.e.~ground-state) confinement in transverse directions.
\end{enumerate*}


\section{Spin-orbit coupling}

We start with two-level atoms loaded into the ground band of an
optical lattice, and consider interrogating these atoms by a linearly
polarized plane-wave laser tuned to the relevant electronic (atomic)
transition, with wavenumber projection $\phi$ onto the lattice axis.
After performing a gauge transformation which shifts the on-axis
momenta of ground (excited) states by $\phi/2$ ($-\phi/2$), and
eliminating (i.e.~turning off) the interrogation laser, one can arrive
at the Hamiltonian
\begin{align}
  H_{\t{SOC}}
  = -2J_0\sum_{q,s} \cos\p{q+s\phi/2} \hat c_{qs}^\dag \hat c_{qs}
  \label{eq:H_SOC_start}
\end{align}
where $q$ indexes quasi-momentum along the lattice axis,
$s\in\set{\g\leftrightarrow-1,\e\leftrightarrow1}$ labels the
electronic state, $J_0$ is the (positive) ground-band tunneling rate,
$c_{q\sigma}$ is a fermionic annihilation operator, and we work in
units with the lattice spacing $a=1$.  Defining
\begin{align}
  \1_q \equiv \hat c_{q,\g}^\dag \hat c_{q,\g}
  + \hat c_{q,\e}^\dag \hat c_{q,\e},
  &&
  \sigma_q^j \equiv \sum_{\alpha,\beta}
  \hat c_{q\alpha}^\dag \sigma^j_{\alpha\beta} \hat c_{q\beta},
  \label{eq:pseudospin}
\end{align}
for Pauli matrices $\sigma^j$ (i.e.~such that $\sigma_q^j$ is a
pseudo-spin-1/2 Pauli operator on the electronic degrees of freedom)
and expanding the cosine in \eqref{eq:H_SOC_start}, we thus find
\begin{align}
  H_{\t{SOC}}
  = -\sum_q \p{\epsilon_q \1_q + \f12 h_q \sigma_q^z},
  &&
  \epsilon_q \equiv 2J_0 \cos\p{\phi/2} \cos q,
  &&
  h_q \equiv -4J_0 \sin\p{\phi/2} \sin q.
\end{align}
When all dynamics conserve the occupations of all quasi-momenta, the
first term in this Hamiltonian merely contributes a constant energy
shift which we can safely neglect, leaving us with the effective
inhomogeneous-field Hamiltonian
\begin{align}
  H_{\t{SOC}} = -\sum_q h_q s_q^z,
  &&
  s_q^z \equiv \f12 \sigma_q^z
  \label{eq:H_SOC}
\end{align}


\section{Interactions}

At ultracold temperatures inter-atomic interactions are dominated by
$s$-wave collisions, which are captured by the Hamiltonian
\begin{align}
  H_{\t{int}} = G \int \d^3x~
  \hat\psi_\e^\dag\p{x} \hat\psi_\g^\dag\p{x}
  \hat\psi_\g\p{x} \hat\psi_\e\p{x},
  &&
  G \equiv \f{4\pi a_{\e\g^-}}{m_A},
\end{align}
where $\hat\psi_\sigma$ is a fermionic field operator for atoms in
electronic state $\sigma\in\set{\g,\e}$, $a_{eg}^-$ is an scattering
length, and $m_A$ is the mass of a single atom.  Considering only
occupation of the ground band in a lattice, we can expand the field
operators as
\begin{align}
  \hat\psi_\sigma\p{x} = \sum_q \phi_q\p{x} \hat c_{q\sigma},
\end{align}
for momentum-space (Bloch) wavefunctions $\phi_q$ and fermionic
annihilation operators $\hat c_{q\sigma}$.  The interaction
Hamiltonian then becomes
\begin{align}
  H_{\t{int}} = G \sum_{p,q,r,s} K^{pq}_{rs}
  \hat c_{r,\e}^\dag \hat c_{s,\g}^\dag \hat c_{q,\g} \hat c_{p,\e},
  &&
  K^{pq}_{rs} \equiv \int \d^3x~
  \phi_r\p{x}^* \phi_s\p{x}^* \phi_q\p{x} \phi_p\p{x}.
\end{align}
In a lattice with $L$ sites centered on positions $x_j$ (i.e.~for
$j=1,\cdots L$ indexing lattice site) and localized (Wannier)
wavefunctions $w_j$, we can expand
$\phi_p\p{x}=L^{-1/2}\sum_je^{-ipx_j}w_j\p{x}$, which implies
\begin{align}
  K^{pq}_{rs} = L^{-2} \sum_{j,k,\ell,m} \int \d^3x~
  \exp\sp{-i\p{px_j+qx_k-rx_\ell-sx_m}}
  w_\ell\p{x}^* w_m\p{x}^* w_k\p{x} w_j\p{x}.
\end{align}
If we now assume that we can neglect integrals without $j=k=\ell=m$,
which is equivalent to neglecting inter-site interactions and
interaction-assisted hopping, then we have
\begin{align}
  K^{pq}_{rs} \approx L^{-2} \sum_j \int \d^3x~
  \exp\sp{-i\p{p+q-r-s}x_j} \abs{w_j\p{x}}^4,
\end{align}
where a stationary phase approximation now forces $p+q-r-s=0$ (mod
$2\pi$ in units with the lattice spacing $a=1$), which is equivalent
to conservation of total momentum.  In terms of the Kronecker delta
$\delta_{p+q,r+s}$ enforcing $p+q=r+s$, it follows that
\begin{align}
  K^{pq}_{rs} \approx \delta_{p+q,r+s} L^{-2}
  \sum_j \int \d^3x~ \abs{w_j\p{x}}^4
  = \delta_{p+q,r+s} L^{-1} \int \d^3x~ \abs{w_0\p{x}}^4,
\end{align}
which motivates the definitions
\begin{align}
  U \equiv G \int \d^3x~ \abs{w_0\p{x}}^4,
\end{align}
in order to express the interaction Hamiltonian in the simple form
\begin{align}
  H_{\t{int}} = \f{U}{L} \sum_{p,q,r,s}
  \delta_{p+q,r+s} \hat c_{r,\e}^\dag \hat c_{s,\g}^\dag
  \hat c_{q,\g} \hat c_{p,\e}.
  \label{eq:H_int_full}
\end{align}


\section{Collective spin model}

If the single-particle energy spacings are large compared to the
strength of inter-particle interactions, or equivalently if
$J_0\gtrsim U$, then by the secular approximation we can neglect terms
in $H_{\t{int}}$ which do not conserve the sum of single-particle
energies.  By solving this single-particle energy conservation
condition, one finds that we must have $\set{p,q}\ne\set{r,s}$; that
is, atoms in different clock states can only interact via
\begin{enumerate*}[label=(\roman*)]
\item direct density-density terms, and
\item terms which exchange their momenta.
\end{enumerate*}
The surviving terms in \eqref{eq:H_int_full} are thus
\begin{align}
  H_{\t{int}}
  = \f{U}{L} \sum_{p,q}
  \p{\hat c_{p,\e}^\dag \hat c_{q,\g}^\dag \hat c_{q,\g} \hat c_{p,\e}
    + \hat c_{q,\e}^\dag \hat c_{p,\g}^\dag \hat c_{q,\g} \hat c_{p,\e}}
  = \f{U}{L} \sum_{p,q}
  \p{\hat c_{q,\g}^\dag \hat c_{q,\g} \hat c_{p,\e}^\dag \hat c_{p,\e}
    - \hat c_{q,\e}^\dag \hat c_{q,\g} \hat c_{p,\g}^\dag \hat c_{p,\e}},
\end{align}
where we can use the single-particle pseudo-spin-1/2 Pauli operators
in \eqref{eq:pseudospin} to write
\begin{align}
  H_{\t{int}} = \f{U}{L} \sum_{p,q}
  \sp{\p{\f{\1_q-\sigma_q^z}{2}} \p{\f{\1_p+\sigma_p^z}{2}}
    - \sigma_q^+ \sigma_p^-}.
\end{align}
Up to a global shift in energy, this result is equivalently
\begin{align}
  H_{\t{int}}
  = - \f14 \f{U}{L} \sum_{p,q} \v{\sigma}_q\c\v{\sigma}_p
  = - \f{U}{L} \sum_{p,q} \v s_q \c\v s_p
  = - \f{U}{L} \v S \c \v S,
\end{align}
where
\begin{align}
  \v{\sigma}_p
  \equiv \p{\sigma_p^x,\sigma_p^y,\sigma_p^z},
  &&
  \v s_p \equiv \p{s_p^x,s_P^y,s_p^z},
  &&
  \v S \equiv \f12 \sum_p \v{\sigma}_p.
  \label{eq:H_int}
\end{align}
For a fixed particle number $N$, the collective spin Hamiltonian
$H_{\t{int}}$ has electronic eigenstates $\set{\ket{Sm}}$ with total
(pseudo-)spin $S$ and projection $m$ onto a quantization axis; the
corresponding energies are $\bk{Sm|H_{\t{int}}|S,m}=-\p{U/L}S\p{S+1}$.
In particular, the ground-state $S=N/2$ manifold is spanned by the
Dicke states
\begin{align}
  \ket{m} \equiv \ket{N/2,m}
  = {N \choose N/2+m}^{-1/2} S_+^{N/2+m} \ket{\g}^{\otimes N},
  &&
  S_+ \equiv \sum_n \sigma_n^+,
  \label{eq:dicke_states}
\end{align}
where $n$ indexes an individual atom,
$\sigma_n^+\equiv\hat c_{n,\e}\hat c_{n,\g}$ is an individual
spin-raising operator, and $N/2+m=0,1,\cdots,N$ is the number of
electronic state excitations in the state $\ket{m}$
\cite{swallows2011suppression}.  In words, the Dicke state $\ket{m}$
is a uniform superposition of all states with $N/2+m$ total electronic
excitations.


\section{Effective spin squeezing}
\label{sec:OAT}

In total, free evolution of atoms on a lattice is governed by the
Hamiltonian
\begin{align}
  H_{\t{free}}
  = H_{\t{int}} + H_{\t{SOC}}
  = -\f{U}{L} \v S\c\v S - \sum_n h_n s_n^z,
\end{align}
for $n$ indexing an individual atom.  When the ``local'' fields $h_n$
are small in magnitude compared to the collective spin gap $\eta U$
and we initialize all atoms in the ground-state subspace of
$H_{\t{int}}$, we can treat the action of the spin-orbit coupling
Hamiltonian perturbatively, which yields the effective free-evolution
Hamiltonian (see Appendix \ref{sec:squeezing_derivation} or Peiru's
notes)
\begin{align}
  H_{\t{free,eff}} = -\bar h S_z + \chi S_z^2,
\end{align}
where
\begin{align}
  \bar h \equiv \f1N \sum_n h_n,
  &&
  \tilde h^2 \equiv \f1N \sum_n \p{h_n - \bar h}^2,
  &&
  \chi \equiv \f{\tilde h^2}{\p{N-1}\eta U}.
\end{align}
At temperatures $T\gg J_0$, the mean effective field $\bar h=0$, while
for $T\sim J_0$ the collective rotation due to the $S_z$ term can be
eliminated by use of a rotating frame.  The free-evolution Hamiltonian
thus simplifies further to the one-axis twisting Hamiltonian
\begin{align}
  H_{\t{OAT}}^z = \chi S_z^2.
  \label{eq:H_OAT}
\end{align}
An example set of parameters and values for one-axis twisting via
$H_{\t{OAT}}^z$ with strontium-87 in a 1-D magic-wavelength lattice is
provided in Table \ref{tab:parameters}, along with the optimal
one-axis twisting squeezing time
$t_{\t{opt}}^{\t{OAT}}\sim N^{-2/3}\chi^{-1}\sim\eta
N^{1/3}=N^{4/3}/L$.  Lattice depths are provided in units of the
lattice recoil energy $E_R$.

\begin{table}[h]
  \centering
  \caption{Example parameters for spin squeezing via one-axis
    twisting.}
  \label{tab:parameters}
  \begin{tabular}{|l|c|l|}
    \hline
    Parameter & Symbol & Value \\ \hline\hline
    Primary lattice depth & $V_0$ & 4 $E_R$ \\
    Transverse lattice depths & $V_T$ & 60 $E_R$ \\
    Lattice sites & $L$ & 100 \\
    Atom number & $N$ & 50 \\
    SOC strength & $\phi$ & $\pi/50$ \\ \hline\hline
    Tunneling rate & $J_0$ & $\approx300\times2\pi$ Hz \\
    On-site interaction strength & $U$ & $\approx1.4\times2\pi$ kHz \\
    Squeezing strength & $\chi$ & $\approx21\times2\pi$ mHz \\
    Optimal squeezing time & $t_{\t{opt}}^{\t{OAT}}$
    & $\approx0.59$ seconds \\ \hline
  \end{tabular}
\end{table}


\section{Two-axis twisting: pulsed drive}

While one-axis twisting provides a spin fluctuation noise floor which
scales with atom number as $\sim1/N^{2/3}$, it is possible to beat
this limit and achieve a noise floor scaling as $\sim1/N$ via
so-called two-axis twisting.  We can transform the one-axis twisting
Hamitlonian in \eqref{eq:H_OAT} into an effective two-axis twisting
Hamiltonian by using a pulsed drive sequence developed in
ref.~\cite{liu2011spin}.  The necessary drive sequence requires the
pulsed application of a clock laser, which turns on a collective spin
Hamiltonian
\begin{align}
  H_{\t{clock}} = -\Omega \sum_n s_n^x = -\Omega S_x.
\end{align}
If $\abs{\Omega}\gg N\chi$, then we can apply $H_{\t{clock}}$ for
short times $\theta/\abs{\Omega}\ll\p{N\chi}^{-1}$ (during which we
can neglect $H_{\t{eff}}$) to realize collective rotations of the form
$\exp\p{\pm i\theta S_x}$.  Thus
\begin{enumerate*}[label=(\roman*)]
\item acting with $\exp\sp{i\p{\pi/2}S_x}$,
\item waiting for a time $2\tau$,
\item acting with $\exp\sp{-i\p{\pi/2}S_x}$, and
\item waiting for a time $\tau$
\end{enumerate*}
realizes the unitary
\begin{align}
  U_1 = \exp\p{-i\tau\chi S_z^2} \exp\p{-i\f{\pi}{2}S_x}
  \exp\p{-i2\tau\chi S_z^2} \exp\p{i\f{\pi}{2}S_x}
  = \exp\p{-i\tau\chi S_z^2} \exp\p{-i2\tau\chi S_y^2}.
\end{align}
If $\tau\chi\ll1$, then
\begin{align}
  U_1 \approx \exp\sp{-i\tau\chi\p{S_z^2 + 2S_y^2}}
  = \exp\sp{-i\tau\chi\p{S^2 + S_y^2 - S_x^2}}
  \simeq \exp\sp{-i\tau\chi\p{S_y^2 - S_x^2}},
\end{align}
where $\simeq$ denotes equality up to an overall phase, which holds
above because $S^2$ is conserved by all dynamics.  Repeating the above
sequence $n$ times for a time $t=3n\tau$ thus results in the unitary
\begin{align}
  U = \exp\sp{-in\tau\chi\p{S_y^2 - S_x^2}}
  = \exp\sp{-i\p{t/3}\chi\p{S_y^2 - S_x^2}},
\end{align}
which is equivalently generated by evolution for a time $t$ under the
two-axis countertwisting Hamiltonian
\begin{align}
  H_{\t{TAT}}^{y,x} = \f{\chi}{3}\p{S_y^2 - S_x^2}.
  \label{eq:H_TAT_pulse}
\end{align}
Despite the factor of 1/3 in the effective two-axis countertwisting
strength, the optimal squeezing time for this two-axis twisting
Hamiltonian is generally close to that for one-axis twisting with
$H_{\t{OAT}}^z$ in \eqref{eq:H_OAT} \cite{liu2011spin}.

\note{[MP: In practice, of course, the gates $\exp\p{i\theta S_x}$
  take a nonzero amount of time to implement.  For the purposes of
  simulation, we can probably approximate these gates by square pulses
  centered on the appropriate time.  I can do an error analysis on
  this sequence to determine the effect of the $\O\p{\tau^2\chi^2}$
  terms in the exponent of $U_1$, but it is probably simpler to just
  try this out in simulations.]}

\note{[MP: I expect this pulsed-drive protocol to be worse than the
  continuous-drive protocol below.  If we decide that we are not
  interested in the pulsed drive, we can merely mention it and throw
  in a citation of ref.~\cite{liu2011spin} without devoting an entire
  section to the subject.]}


\section{Two-axis twisting: continuous drive}

Rather than pulsing the clock laser, we can apply a continuous drive
to realize the Hamiltonian
\begin{align}
  H_{\t{drive}} = H_{\t{free}} + H_{\t{clock}}
  = -\f{U}{L} \v S\c\v S - \bar h S_z - \Omega S_x - \sum_n b_n s_n^z,
\end{align}
where the drive strength $\Omega$ may generally depend on time.  As
shown in Appendix \ref{sec:effective_squeezing_drive}, the addition of
a drive with $\abs{\Omega}\ll\eta\abs{U}=N\abs{U}/L$ does not modify
the squeezing behavior of the spin-orbit coupling terms in
$H_{\t{drive}}$, whose effective action on the $S=N/2$ manifold is
captured by
\begin{align}
  H_{\t{drive,eff}} = -\bar h S_z - \Omega S_x + \chi S_z^2.
  \label{eq:H_drive_eff}
\end{align}
To follow the continuous-drive two-axis twisting prescription in
ref.~\cite{huang2015twoaxis}, we now
\begin{enumerate*}[label=(\roman*)]
\item blue-detune the clock laser by $\bar h$ and shift into an
  appropriate rotating frame to eliminate the $\sim S_z$ term in
  \eqref{eq:H_drive_eff},
\item modulate the drive as $\Omega\p{t}=\beta\omega\cos\p{\omega t}$
  with $\omega\gg N\chi$, and
\item move into the rotating frame of $-\Omega\p{t}S_x$.
\end{enumerate*}
After a secular approximation which relies on $\omega\gg N\chi$, this
procedure results in the effective Hamiltonian
\begin{align}
  H_{\t{drive,eff}}^{\t{mod}}
  = \f{\chi}{2} \p{\sp{\J_0\p{2\beta}+1} S_z^2
    - \sp{\J_0\p{2\beta}-1} S_y^2},
\end{align}
where $\J_0$ is the zero-order Bessel function of the first kind.
Choosing a modulation index $\beta$ such that $\J_0\p{2\beta}=1/3$ or
$\J_0\p{2\beta}=-1/3$ then respectively results in the two-axis
twisting Hamiltonians
\begin{align}
  H_{\t{TAT}}^{z,x}
  = \f{\chi}{3} \p{2 S_z^2 + S_y^2}
  \simeq \f{\chi}{3} \p{S_z^2 - S_x^2},
  &&
  H_{\t{TAT}}^{y,x}
  = \f{\chi}{3} \p{S_z^2 + 2 S_y^2}
  \simeq \f{\chi}{3}\p{S_y^2 - S_x^2},
  \label{eq:H_TAT_drive}
\end{align}
where $\simeq$ denotes equality up to a global energy shift. Given the
set of parameters in Table \ref{tab:parameters}, the two-axis twisting
Hamiltonian $H_{\t{TAT}}^{z,x}$ can be achieved with a drive frequency
$\omega=N\abs{\chi U/L}^{1/2}\approx27\times2\pi$ Hz and modulation
index $\beta=0.906$.  The optimal squeezing time for this protocol is
$t_{\t{opt}}^{\t{TAT}}\approx0.82$ seconds.  A comparison of squeezing
through $H_{\t{OAT}}^z$ and $H_{\t{TAT}}^{z,x}$ for the initial state
$\ket{Y}\equiv\bigotimes_n\p{\ket{\g}_n+i\ket{\e}_n}/\sqrt2$ via the
parameters in Table \ref{tab:parameters} is provided in Figure
\ref{fig:squeezing_comparison}.  The figure of merit for squeezing is
the normalized minimal variance of spin in the plane orthogonal to the
mean spin vector $\bk{\v S}$:
\begin{align}
  \xi^2 \equiv \f{N}{\abs{\bk{\v S}}^2}
  \min_{\hat{\v n}\perp\bk{\v S}} \bk{\p{\v S\c\hat{\v n}}^2}
\end{align}
for $\abs{\hat{\v n}}=1$.

\begin{figure}
  \centering \includegraphics{squeezing_comparison_TAT.pdf}
  \caption{Comparison of squeezing through $H_{\t{OAT}}^z$ and
    $H_{\t{TAT}}^{z,x}$ via the parameters in Table
    \ref{tab:parameters} and the continuous two-axis twisting protocol
    with $\omega=27\times2\pi$ Hz and $\beta=0.906$.}
  \label{fig:squeezing_comparison}
\end{figure}





\section{Discussions}

\note{[MP: We should clarify in the notes what exactly the conditions
  are for the effective Hamiltonians we derive, i.e.~the spin model
  and the one-axis twisting Hamiltonian.]}

\note{[MP] Future ideas:} Make use of a cavity to enhance the one-axis
twisting strength \cite{hu2017vacuum}?


\appendix

\section{Effective spin squeezing in the presence of a weak drive}
\label{sec:squeezing_derivation}

Suppose we have a Hamiltonian of the form
\begin{align}
  H = H_0 + V,
  &&
  H_0 = - \f{U}{L} \v S\c\v S,
  &&
  V = - \sum_n h_n s_n^z - \Omega S_x,
\end{align}
and we consider $N$-particle states initially in the ground-state
manifold $\G_0$ of $H_0$, which have total spin $S=N/2$.  If the
largest eigenvalue of $V$ is smaller in magnitude than half of the
collective spin gap $N\abs{U}/L=\eta\abs{U}$, i.e.~the energy gap
under $H_0$ between $\G_0$ and its orthogonal complement $\E_0$, then
we can formally develop a perturbative treatment for the action of $V$
on $\G_0$.  Such a treatment yields an effective Hamiltonian on $\G_0$
of the form $H_{\t{eff}}=\sum_pH_{\t{eff}}^{(p)}$, where
$H_{\t{eff}}^{(p)}$ is order $p$ in $V$.  Letting $\P_0$ ($\Q_0$) be a
projector onto $\G_0$ ($\E_0$) and $X$ denote any operator on
$\G_0\cup\E_0$ (i.e.~the entire Hilbert space), we define the
superoperators
\begin{align}
  \D X \equiv \P_0 X \P_0 + \Q_0 X \Q_0,
  &&
  \O X \equiv \P_0 X \Q_0 + \Q_0 X \P_0,
\end{align}
which select the diagonal ($\D$) and off-diagonal ($\O$) parts of $X$
with respect to $\G_0$ and $\E_0$, and
\begin{align}
  \L X \equiv \sum_{\alpha,\beta}
  \f{\op{\alpha}\O X\op{\beta}}{E_\alpha-E_\beta},
  &&
  \t{where}
  &&
  H_0 = \sum_\alpha E_\alpha \op\alpha.
\end{align}
The first few terms in the expansion of the effective Hamiltonian
$H_{\t{eff}}$ are then, as derived in
ref.~\cite{bravyi2011schrieffer},
\begin{align}
  H_{\t{eff}}^{(0)} = \P_0 H_0 \P_0,
  &&
  H_{\t{eff}}^{(1)} = \P_0 V \P_0,
  &&
  H_{\t{eff}}^{(2)} = -\f12 \P_0 \sp{\O V,\L V} \P_0.
  \label{eq:H_eff_012}
\end{align}
If we add a constant to $H_0$ such that $E_\psi=0$ for all
$\ket\psi\in\G_0$, or equivalently we measure all energies relative to
that of the ground-state manifold $\G_0$ with respect to $H_0$, it
immediately follows that $H_{\t{eff}}^{(0)}=0$.  To calculate
$H_{\t{eff}}^{(1)}$, we note that the ground-state manifold $\G_0$ is
spanned by the Dicke states $\ket{m}$ defined in
\eqref{eq:dicke_states}, in terms of which we can expand the
collective spin-$z$ operator as $S_z=\sum_mm\op{m}$.  We can likewise
expand the collective spin-$x$ operator $S_x$ in terms of $x$-oriented
Dicke states $\ket{m_x}$ as $S_x=\sum_mm\op{m_x}$.  The ground-state
projector $\P_0$ onto $\G_0$ can be expanded in either basis as
$\P_0=\sum_m\op{m}=\sum_m\op{m_x}$.  Defining the mean and residual
fields
\begin{align}
  \bar h \equiv \f1N \sum_n h_n,
  &&
  b_n \equiv h_n - \bar h,
\end{align}
we can then write
\begin{align}
  V = -\sum_n \p{b_n+\bar h} s_n^z - \Omega S_x
  = -\sum_n b_n s_n^z - \bar h S_z - \Omega S_x,
\end{align}
and in turn
\begin{align}
  H_{\t{eff}}^{(1)}
  = \P_0\p{-\sum_n b_n s_n^z - \bar h S_z - \Omega S_x} \P_0
  = -\sum_n b_n \P_0 s_n^z\P_0 - \bar h S_z - \Omega S_x,
\end{align}
where we used the fact that $\P_0 S_{j=z,x} \P_0 = S_j$.  By
construction, the residual fields are mean-zero, i.e.~$\sum_nb_n=0$.
Using the particle-exchange symmetry of the Dicke states, we can thus
expand
\begin{align}
  \sum_n b_n \P_0 s_n^z \P_0
  = \sum_{n,m,m'} b_n \op{m} s_n^z \op{m'}
  = \sum_n b_n \sum_{m,m'} \op{m} s_1^z \op{m'}
  = 0,
\end{align}
which implies
\begin{align}
  H_{\t{eff}}^{(1)} = - \bar h S_z - \Omega S_x.
\end{align}
To calculate the second-order effective Hamiltonian
$H_{\t{eff}}^{(2)}$, we let $\B_0\p{\E_0}$ denote an eigenbasis of
$H_0$ for the excited subspace $\E_0$, define the operator
\begin{align}
  \I \equiv \sum_{\ket\alpha\in\B_0\p{\E_0}} \f{\op\alpha}{E_\alpha},
\end{align}
which sums over projections onto excited states with corresponding
energetic suppression factors, and expand
\begin{align}
  \L X = \O\p{\L X}
  = \Q_0 \L X \P_0 + \P_0 \L X \Q_0
  = \I X \P_0 - \P_0 X \I.
\end{align}
The expression for $H_{\t{eff}}^{(2)}$ in \eqref{eq:H_eff_012} then
simplifies to
\begin{align}
  H_{\t{eff}}
  = -\f12 \P_0 \p{\sp{\O V, \I V \P_0} - \sp{\O V, \P_0 V \I}} \P_0
  = -\P_0 V \I V \P_0.
\end{align}
The only part of $V$ which is off-diagonal with respect to the ground-
and excited-state manifolds $\G_0$ and $\E_0$ is $-\sum_nb_ns_n^z$,
and the individual operators in this term can only change the total
spin $S$ by at most 1.  It is therefore sufficient to expand $\I$ in a
basis for states with total spin $S=N/2-1$, which is provided by the
spin-wave states
\begin{align}
  \ket{mk}
  \equiv \ket{N/2-1,m,k}
  \equiv {N-1 \choose \p{N/2-m}\p{N/2-m+1}}^{-1/2}
  \sum_{n=1}^N e^{2\pi ikn/N} s_n^+ \ket{N/2,m-1},
\end{align}
for $k=1,2,\cdots,N-1$ \cite{swallows2011suppression}.  Using the fact that all spin-$z$ operators preserve the projection of total spin onto the $z$ axis, we then have that
\begin{align}
  H_{\t{eff}}^{(2)}
  = -\f1{\eta U} \sum_{m,m',k,n,n'} b_n b_{n'}
  \Bk{m| s_n^z \op{m'k} s_{n'}^z |m} \op{m}.
  \label{eq:general_H_eff_2}
\end{align}
The relevant matrix elements between the Dicke states and the
spin-wave states are \cite{swallows2011suppression}
\begin{align}
  \bk{m|s_n^z|m'k}
  = e^{2\pi i k n/N} \sqrt{\f{(N/2)^2-m^2}{N^2 (N-1)}}~ \delta_{m,m'},
\end{align}
which implies
\begin{align}
  H_{\t{free,eff}}^{(2)}
  = -\f1{\eta U} \sum_m \f{(N/2)^2-m^2}{N^2 (N-1)} \op{m}
  \sum_{k,n,n'} b_n b_{n'} e^{2\pi ik\p{n-n'}/N}.
\end{align}
Using the fact that $\sum_nb_n=0$, we can expand
\begin{align}
  \sum_{k,n,n'} b_n b_{n'} e^{2\pi ik\p{n-n'}/N}
  = \sum_{n,n'} b_n b_{n'} \sum_{k=1}^{N-1} e^{2\pi ik\p{n-n'}/N}
  = \sum_{n,n'} b_n b_{n'} \sum_{k=0}^{N-1} e^{2\pi ik\p{n-n'}/N},
\end{align}
where the sum over $k$ vanishes for $n\ne n'$ and equals $N$ when
$n=n'$, so
\begin{align}
  \sum_{k,n,n'} b_n b_{n'} e^{2\pi ik\p{n-n'}/N}
  = N \sum_n b_n^2 = N^2 \tilde h^2,
  &&
  \tilde h^2 \equiv \f1N \sum_n b_n^2 = \f1N \sum_n \p{h_n - \bar h}^2.
  \label{eq:sum_knn}
\end{align}
We therefore have that
\begin{align}
  H_{\t{free,eff}}^{(2)}
  = -\sum_m \f{\p{N/2}^2-m^2}{\p{N-1}\eta U}~ \tilde h^2 \op{m},
\end{align}
where the term $\p{N/2}^2$ term contributes a global energy shift
which we can neglect, while the $m^2$ term is proportional to
$m^2\op{m}=S_z^2$.  In total, the effective Hamiltonian is thus
\begin{align}
  H_{\t{free,eff}} = - \bar h S_z - \Omega S_x + \chi S_z^2,
  &&
  \chi \equiv \f{\tilde h^2}{\p{N-1}\eta U}.
\end{align}

\bibliography{\jobname}

\end{document}
