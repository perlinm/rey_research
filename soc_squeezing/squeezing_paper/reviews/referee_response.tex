\documentclass[reprint,onecolumn,12pt]{revtex4-2}

%%% linking references
\usepackage{hyperref}
\hypersetup{
  colorlinks=true,
  linkcolor=blue,
  urlcolor=cyan,
}

%%% physics and math packages
\usepackage{physics} % general physics package
\usepackage{amssymb} % math fonts and symbols
\usepackage{bm} % for making math symbols bold
\usepackage{braket} % for nice brackets (e.g. |X>)
\usepackage{mathtools} % for \coloneqq
\usepackage{graphicx} % for figures
\graphicspath{{./figures/}} % set path for all figures

%%% shorthands to use inside math environments
\renewcommand{\t}{\text} % text in math mode
\newcommand{\f}[2]{\dfrac{#1}{#2}} % shorthand for fractions
\newcommand{\p}[1]{\left(#1\right)} % parenthesis
\renewcommand{\sp}[1]{\left[#1\right]} % square parenthesis
\renewcommand{\set}[1]{\left\{#1\right\}} % square parenthesis
\renewcommand{\c}{\cdot} % inner product
\renewcommand{\d}{\text{d}} % integration measure, e.g. d^3 x
\renewcommand{\v}{\bm} % bold symbols (e.g. for vectors)
\newcommand{\uv}[1]{\hat{\bm #1}} % unit vectors, or vector operators
\newcommand{\bk}{\Braket}
\renewcommand{\ket}{\Ket}
\renewcommand{\bra}{\Bra}

\newcommand{\B}{\mathcal{B}}
\newcommand{\E}{\mathcal{E}}
\newcommand{\G}{\mathcal{G}}
\newcommand{\I}{\mathcal{I}}
\newcommand{\J}{\mathcal{J}}
\renewcommand{\L}{\mathcal{L}}
\renewcommand{\O}{\mathcal{O}}
\renewcommand{\P}{\mathcal{P}}
\newcommand{\Q}{\mathcal{Q}}
\renewcommand{\S}{\mathcal{S}}

\newcommand{\z}{\text{z}}
\newcommand{\x}{\text{x}}
\newcommand{\y}{\text{y}}

\newcommand{\up}{\uparrow}
\newcommand{\dn}{\downarrow}


\usepackage{enumitem}
\setlist[enumerate]{leftmargin=*}
\setlist[itemize]{leftmargin=*}
\usepackage[dvipsnames]{xcolor}
\newcommand{\blue}[1]{{\color{blue} #1}}
\newcommand{\red}[1]{{\color{red} #1}}
\newcommand{\green}[1]{{\color{ForestGreen} #1}}


\begin{document}

\section*{Response to referee}

We thank the referee for taking the time to carefully read our
manuscript, and for providing thoughtful and constructive feedback.
Below, we address the comments and questions brought up by the
referee.  We hope that the referee finds our responses and revisions
satisfactory, deeming our manuscript acceptable for publication in
Physical Review X.

Excerpts from the referee reports are written in \blue{blue}, excerpts
from our original manuscript are written in \red{red}, and excerpts
from the revised version of our manuscript is written in
\green{green}.  All page, equation, and reference numbers generally
refer to those in our original manuscript, unless explicitly stated
otherwise.


\begin{enumerate}
\item Concerning:

  \blue{The main result of the manuscript is the theoretical
    development and characterization of a new protocol for the
    generation of spin squeezing in optical atomic clocks, to be
    employed for boosting the precision of clock experiments beyond
    the standard quantum limit. The topic, encompassing AMO physics
    and quantum physics, is certainly highly interesting. The
    implementation of these ideas in an experiment would constitute a
    very important advancement in the broad field of quantum sensing
    and metrology, as the application of nonclassical states for
    quantum enhancement of the measurement precision is the subject of
    major efforts in quantum technologies nowadays.}

  \blue{The proposal is new, although it is substantially based on
    previous theoretical and experimental works by the groups led by
    the two senior authors (J. Ye and A. M. Rey). On the theoretical
    side, the amount of technological novelty seems limited, as the
    authors are not developing radically new theoretical tools or
    approaches. What is new is the idea that the ``ingredients'' which
    have already been demonstrated can be mixed up, with the proper
    protocol, to generate substantial, useful entanglement. On the
    experimental side, the implementation seems feasible and the
    consequences would be highly relevant, as the promised levels of
    spin squeezing have been never reported yet in any experiment of
    the atomic clock kind. Balancing these considerations, I find the
    subject overall appropriate for PRX, mostly in view of the
    importance of an eventual experimental demonstration of the
    proposed protocol.}

  \blue{The paper is written very well and the level of presentation
    fits the target of PRX, with a first paragraph that serves as a
    quite useful introduction for non-specialists. In the same spirit,
    I have appreciated the decision of the authors to focus on the
    one-axis twisting (OAT) protocol, as it is less effective but more
    pedagogical than the two-axis twisting (TAT) protocol they also
    consider, leaving the latter almost completely in the
    appendixes. Although the interested reader will be forced to read
    the appendixes to understand how the TAT works, I think that this
    choice is positive overall. With the more technical parts moved to
    the appendixes and the general physics presented in the main text,
    the paper seems suitable for a generic audience.}

  \blue{So, on the side of presentation and significance, I find the
    manuscript adequate for PRX.}

  \blue{Before giving a final assessment on the actual validity of the
    manuscript, I invite the authors to provide clarifications on the
    following points:}

  We thank the referee for summarizing the nature and impact of our
  work, deeming the subject of our manuscript ``appropriate for PRX'',
  and noting that our manuscript is ``written very well'' with a
  ``presentation [that] fits the target of PRX''.  We hope that the
  revisions to our manuscript outlined below adequately address any
  questions and reservations that the referee may have, deeming our
  revised manuscript acceptable for publication in PRX.


\item Concerning:

  \blue{1) The authors always consider the case of a homogeneously
    trapped atomic gas, whereas in current 3D optical clock
    implementations there is a significant quasi-harmonic trapping
    induced by the focused magic-optical-lattice beams. Could the
    authors comment on the relevance of this effect for their spin
    squeezing protocol? Since the squeezing arises from the dephasing
    of spatially delocalized wavefunctions, the trap, modifying those
    wavefunctions, might introduce unwanted effects. For instance, the
    spectrum of the B term in the Hamiltonian would change with
    respect to the simple Cos form of Eq. (2): with the upper bound of
    the spectrum upshifted by the harmonic potential energy, the
    (global) interaction gap U would be shrinked, potentially leading
    to a mixing of different Dicke manifolds. In addition, the lattice
    filling f will be continuously changing all over the trap,
    diminishing in the outer regions and potentially causing problems
    for the weak SOC limit assumption. It is not clear to me whether a
    LDA approach would be sufficient to demonstrate the efficiency of
    the protocol in the presence of the trap, as the wavefunctions are
    delocalized. If yes, which assumptions have to be made on the
    system?}



\item Concerning:

  \blue{2) The authors always consider the effects of purely elastic
    interactions. However, in previous works it was shown that
    collisions involving two-electron atoms in the 1S0 and 3P0 states
    can have a substantial inelastic character, with sizable atom loss
    rates. The authors should comment on the influence of such
    processes on their protocol. Can they provide estimates on the
    reduction of spin squeezing when these processes are considered?
    Of course, this effect could be circumvented by considering the
    alternative squeezing protocol relying on nuclear spin states and
    Raman transitions, very briefly mentioned in the last paragraph
    before the conclusions.  Regarding the latter point, it would be
    interesting to know something more about the transfer of
    entanglement from one degree of freedom (nuclear spin) to the
    other (electronic state).}



\item Concerning:

  \blue{3) What happens after the squeezing protocol is never
    mentioned. Even briefly (for the sake of non-experts), the authors
    should explain how the ensuing clock interrogation protocol would
    be implemented. I guess that it would be carried out with a simple
    increase of the lattice depth to stop the effect of the squeezing
    Hamiltonian, followed by a plain clock interrogation sequence in
    the Lamb-Dicke regime. Is this enough or are there additional
    subtleties to be discussed?}

  We thank the referee for their careful consideration of how our
  squeezing protocol would be used in practice.  Indeed, the
  accessibility of our work would benefit greatly from a discussion of
  how to interface our squeezing protocol with an ensuing clock
  interrogation protocol.  To address this point, we have added a
  written a new appendix for our manuscript (Appendix D, on page 10 of
  the revised manuscript):
  % todo: make sure appendix and page numbers are correct

  \green{[Insert Appendix D here, once we finalize it.]}

  In order to distinguish Fermi-Hubbard Hamiltonians with and without
  SOC, we have added an explicit dependence of the Fermi-Hubbard
  Hamiltonians in Eqns.~(1) and (2) on the SOC angle $\phi$ by
  changing \red{$\hat H_{\t{FH}}$} and \red{$\hat H_{\t{FH,single}}$}
  to \green{$\hat H_{\t{FH}}^{(\phi)}$} and
  \green{$\hat H_{\t{FH,single}}^{(\phi)}$}.

  We refer readers to this new appendix at the end of the caption to
  Fig.~1 on page 3, adding the text ``\green{(see Appendix D)}''; as
  well as after Eq.~(5) on page 4, replacing the text:

  \red{Note that the above protocol concerns only the preparation of a
    spin-squeezed state, which ideally would be used as an input state
    for a follow-up clock in- terrogation protocol. SOC can be turned
    off during the latter protocol by properly re-aligning the clock
    laser.}

  by:

  \green{The above protocol concerns only the preparation of a
    spin-squeezed state, which would then be used as an input state
    for a follow-up clock interrogation protocol without SOC.  We
    discuss subtleties of interfacing the squeezing protocol with a
    subsequent clock interrogation protocol in Appendix D.}

  We have also added two sentences to the main text in order to make
  readers aware of the pulsed drive TAT protocol.  Specifically, we
  have changed the text (immediately preceding section III on page 4):

  \red{... TAT remains approximately constant until reaching
    Heisenberg-limited amount of spin squeezing with
    $\xi^2_{\t{opt}}\sim N^{-1}$[37].  Following the prescription in
    Ref.~[38], ...}

  which now reads:

  \green{... TAT remains approximately constant until reaching
    Heisenberg-limited amount of spin squeezing with
    $\xi^2_{\t{opt}}\sim N^{-1}$[17, 37].}

  \green{There are two general strategies for converting OAT into TAT:
    by use of a pulsed[38] or continuous[39] drive protocol.  For
    simplicity, we consider the latter in our work.  Following the
    prescription in Ref.~[39], ...}
  % TODO: make sure reference numbers are correct

  We hope that these changes help clarify subtleties related to
  interfacing our squeezing protocol with a subsequent clock
  interrogation protocol.


\item Concerning:

  \blue{4) It is not completely clear whether the results of Fig.~4 in
    the presence of decoherence have been obtained for a fixed
    evolution time (the authors mention a 1s timescale in the main
    text) or they represent the best spin squeezing values reached at
    an optimal squeezing time? Is there still an optimal squeezing
    time in the presence of decoherence or a different, maybe
    asymptotic, evolution?}

  We thank the referee for finding an ambiguity in our presentation of
  the results in Fig.~4.  To clarify that optimal squeezing occurs at
  a particular time that depends on system and decoherence parameters,
  we have modified the caption of Fig.~4 (on page 6), such that the
  text:

  \red{{\bf Squeezing with decoherence} ... The time scales for
    optimal squeezing with decoherence are generally no greater than
    those in Fig.~3.}

  now reads:

  \green{{\bf Optimal squeezing with decoherence} ... Optimal
    squeezing times in the presence of decoherence are generally
    smaller than the corresponding times shown in Fig.~3, as
    decoherence generally degrades squeezing before it reaches the
    decoherence-free maximum.  Sample plots of squeezing over time for
    particular choices of lattice size ($\ell$) and depth
    ($V_0/E_{\t{R}}$) are provided in the Supplementary
    Material[37].}
  % TODO: make sure reference number is correct

  Note that Fig.~3 shows the optimal squeezing time as a function of
  lattice depth ($V_0/E_{\t{R}}$) and system size ($\ell$).  We hope
  that these changes clarify the fact that optimal squeezing generally
  occurs at a particular time that depends on system parameters.


\item Concerning:

  \blue{5) A final minor point. When the authors indicate that the
    motional degrees of freedom are frozen (for instance, in the new
    paragraph after Eq. 5), it would be quite helpful (for readers who
    are not familiar with previous works) to stress again that
    localization is in momentum space, while the wavefunctions in real
    space are extended.}

  We thank the referee for pointing out a potential source of
  confusion from our use of language when we say ``frozen motional
  degrees of freedom''.  Indeed, the atoms in our setup may still
  tunnel from site to site; these atoms are only ``frozen'' in
  momentum space.  In order to clarify this point, we revised all
  references in our manuscript to frozen motional degrees of freedom,
  in particular changing the following text in the caption to Fig.~1
  (on page 3):

  \red{({\bf c}) If interactions are sufficiently weak, all motional
    degrees of freedom are frozen and atoms are effectively pinned to
    fixed quasi-momentum modes $\v q$.}

  to:

  \green{({\bf c}) If interactions are sufficiently weak, all motional
    degrees of freedom become frozen in momentum space, with atoms
    effectively pinned to fixed quasi-momentum modes $\v q$.}

  and adding a parenthetical to the following text after Eq.~(5) on
  page 4:

  \red{... one cannot assume frozen motional degrees of freedom and
    map the Fermi-Hubbard model onto a spin model.}

  which now reads:

  \green{... one cannot assume frozen motional degrees of freedom
    (i.e.~with atoms pinned to fixed quasi-momentum modes) and map the
    Fermi-Hubbard model to a spin model.}


\end{enumerate}

\end{document}
