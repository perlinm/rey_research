\documentclass[preprint,superscriptaddress]{revtex4-2}

%%% linking references
\usepackage{hyperref}
\hypersetup{
  colorlinks=true,
  linkcolor=blue,
  urlcolor=cyan,
}

%%% physics and math packages
\usepackage{physics} % general physics package
\usepackage{amssymb} % math fonts and symbols
\usepackage{bm} % for making math symbols bold
\usepackage{braket} % for nice brackets (e.g. |X>)
\usepackage{mathtools} % for \coloneqq

%%% shorthands to use inside math environments
\renewcommand{\t}{\text} % text in math mode
\newcommand{\f}[2]{\dfrac{#1}{#2}} % shorthand for fractions
\newcommand{\p}[1]{\left(#1\right)} % parenthesis
\renewcommand{\sp}[1]{\left[#1\right]} % square parenthesis
\renewcommand{\set}[1]{\left\{#1\right\}} % square parenthesis
\renewcommand{\c}{\cdot} % inner product
\renewcommand{\d}{\text{d}} % integration measure, e.g. d^3 x
\renewcommand{\v}{\bm} % bold symbols (e.g. for vectors)
\newcommand{\uv}[1]{\hat{\bm #1}} % unit vectors, or vector operators
\newcommand{\bk}{\Braket}
\renewcommand{\ket}{\Ket}
\renewcommand{\bra}{\Bra}

\newcommand{\B}{\mathcal{B}}
\newcommand{\E}{\mathcal{E}}
\newcommand{\G}{\mathcal{G}}
\newcommand{\I}{\mathcal{I}}
\newcommand{\J}{\mathcal{J}}
\renewcommand{\L}{\mathcal{L}}
\renewcommand{\O}{\mathcal{O}}
\renewcommand{\P}{\mathcal{P}}
\newcommand{\Q}{\mathcal{Q}}
\renewcommand{\S}{\mathcal{S}}

\newcommand{\z}{\text{z}}
\newcommand{\x}{\text{x}}
\newcommand{\y}{\text{y}}

\newcommand{\up}{\uparrow}
\newcommand{\dn}{\downarrow}

% remove extra white space inside enumerate environments
\usepackage{enumitem}
\setlist[enumerate]{leftmargin=*}

\setlength{\parindent}{0pt}
\setlength{\parskip}{1ex}

% for colored text
\usepackage[dvipsnames]{xcolor}
\newcommand{\blue}[1]{{\color{blue} #1}}
\newcommand{\red}[1]{{\color{red} #1}}
\newcommand{\green}[1]{{\color{ForestGreen} #1}}

%%%%%%%%%%%%%%%%%%%%%%%%%%%%%%%%%%%%%%%%%%%%%%%%%%%%%%%%%%%%%%%%%%%%%%
\begin{document}

Dear Editors of Physical Review Research,

Please find attached our revised manuscript, ``Engineering spin squeezing in a 3D optical lattice with interacting spin-orbit-coupled fermions'', for consideration for publication in Physical Review Research.
Below, we address the recommendations and criticisms made by past referees in the peer-review process for Physical Review X, and detail the corresponding changes that we have made to our manuscript.

We were pleased to find that the referee reports were in unanimous agreement about the quality and content of the manuscript, calling it ``well written'' and saying that ``it offers a pathway towards pushing the record level of precision achievable for atomic clocks''.

The most important feedback from previous referees concerned the potential of our manuscript to attract broader interest outside the optical lattice clock community.
We have addressed these concerns in our responses to the referees, as well as corresponding changes to the manuscript.
In short, we believe that developing realistic proposal to achieve a genuine quantum advantage in a world-class sensor unavoidably requires focusing on a specific platform.
Our work is therefore well placed in the context of a broad scientific effort to a achieve a quantum advantage in emergent technologies.
We hope that with these clarifications, our paper can be accepted for publication in Physical Review Research.

Excerpts from the referee reports are written in \blue{blue}, excerpts from the previous version of our manuscript are written in \red{red}, and excerpts from the revised version of our manuscript are written in \green{green}.

%%%%%%%%%%%%%%%%%%%%%%%%%%%%%%%%%%%%%%%%%%%%%%%%%%%%%%%%%%%%%%%%%%%%%%
\section{First referee}

\blue{The authors replied to my previous report with exhaustive and
  mostly convincing explanations.}

\blue{In the revised version of the manuscript the novelty of their
  results is now clearly explained, which strengthens the impact of
  the work.}

\blue{The reply to my specific questions, e.g.~on the importance of
  the underlying trapping potential and robustness of the achievable
  spin squeezing in a realistic setting, is quite detailed and
  accompanied by significant extensions of the supplementary
  materials. The new version of the manuscript now contains important
  details on the practical implementation of the suggested protocol,
  while keeping an appropriate format for PRX. In light of the
  authors' reply and of the revisions made, I can now fully recommend
  the manuscript for publication.}

We thank the Referee for their positive appreciation of our work, and for the recommendation for publication in PRX.
We appreciate their comment that the revisions we made to our manuscript throughout the review process with PRX ``clearly explained the novelty of our results'' and ``strengthen[ed] the impact of our work''.

We hope that the editors of Physical Review Research agree with this Referee's assessment.

%%%%%%%%%%%%%%%%%%%%%%%%%%%%%%%%%%%%%%%%%%%%%%%%%%%%%%%%%%%%%%%%%%%%%%
\section{Second referee}
\label{sec:R2}

\begin{enumerate}
\item Concerning:
  \label{pt:R2.scope}

  \blue{The proposal details a protocol to use the weakly interacting,
    weakly spin-orbit coupled regime of fermions in an optical lattice
    to generate spin squeezing for increased precision in the
    state-of-the-art optical lattice clock.  The individual components
    are achievable with current technologies and have all been
    experimentally demonstrated, with many theoretical and
    experimental advancements from the groups led by the senior
    authors, A. M. Rey and J. Ye. The manuscript is in general
    well-written and accessible to the motivated reader, with
    comprehensive appendices.}

  \blue{This work represents current efforts to enhance measurement
    sensitivity through non-classical states and promises considerable
    levels of spin squeezing for the system under consideration. In
    particular, because it focuses so closely on the current
    state-of-the-art optical lattice clock, it offers a pathway
    towards pushing the record level of precision achievable for
    atomic clocks.}

  \blue{At the same time, because the work focuses so closely on this
    particular lattice clock, the experimentalists most interested in
    this work are likely two of the authors and their colleagues. This
    is my biggest concern about the impact of this work: while it does
    open up one new research opportunity, at the immediate level this
    scope seems limited. I ask that the authors address this point in
    full and the more detailed points below.}

  We appreciate the recognition that our manuscript is ``well-written
  and accessible to the motivated reader, with comprehensive
  appendices''.

  Indeed, our work focuses on current state-of-the-art optical lattice
  clocks, since it aims to push current record levels of precision
  achievable with classical sensors.  Without identifying a concrete
  system with world-class sensing capabilities, our effort would
  reduce to a mere proof-of-principle concept which, despite having
  some merit, would lack any practical utility.  Focusing on a
  specific platform is unavoidable for any {\it realistic} and {\it
    practical} proposal to advance the frontier of metrology.  Our
  work therefore fits well into the context of a worldwide scientific
  effort to achieve a quantum advantage in emergent technologies.

  To emphasize the impact of our work in this new paradigm of enhanced
  quantum metrology with correlated quantum states, and thereby to
  further illustrate the relevance of our proposal to a broader
  scientific community, we have rewritten the last paragraph of our
  introduction (left column of page 2), which previously read:

  \red{The intriguing and non-trivial observation reported here is
    that weakly interacting fermions in a lattice with dynamics
    described by the Hubbard model --- which thus obey fermionic
    quantum statistics and interact via purely contact collisions ---
    can feature all-to-all collective spin interactions that naturally
    generate spin-squeezed states in a manner robust to realistic
    experimental imperfections. In the spirit of quantum simulation,
    our work illustrates that short-range interacting systems can be
    used to emulate dynamical magnetic behaviors found in arrays of
    long-range interacting systems (such as arrays of trapped ions[36]
    or cavity QED systems[37, 38] with interactions mediated by
    phonons or photons, respectively), thus enhancing their potential
    as a metrological resource. Although surprising, this result is in
    line with the recent understandings in quantum research that
    explore how the interplay between well understood simple phenomena
    can give rise to emergent new behaviors. In this direction, our
    work emphasizes that the precise control of interactions, which
    becomes possible in 3D optical lattice systems with quantized
    motional degrees of freedom in all three directions, is a key for
    the implementation of the protocols studied here, which may enable
    new quantum technologies.}

  and now reads:

  \green{Despite an abundance of proof-of-principle experiments with
    entangled states[24, 39], so far only the remarkable example of
    LIGO[40, 41] has demonstrated a quantum advantage in a
    state-of-the-art quantum sensing or measurement system.  The new
    generation of 3D optical lattice systems have fully quantized
    motional degrees of freedom[4], allowing for precise control of
    collisional interactions.  We demonstrate how these interactions
    can naturally give rise to metrologically useful correlated
    many-body fermionic states, opening a path to not only generate
    entanglement, but also harness it to achieve a quantum advantage
    in a world-class sensor.  Such an advance will ultimately deliver
    gains to real-world applications including timekeeping,
    navigation, communication, and our understanding of the
    fundamental laws of nature[42].}

  Similarly, we have changed the following sentence in the
  Conclusions:

  \red{To our knowledge, this is the first proposal to use atomic
    collisions for entanglement generation in state-of-the art atomic
    clocks, opening a path for the practical improvement of
    world-leading quantum sensors using correlated many-body fermionic
    states.}

  to:

  \green{To our knowledge, this is the first proposal to use quantum
    correlations in a many-body fermionic system to push
    state-of-the-art quantum sensors beyond the independent-particle
    regime, thereby achieving a genuine quantum advantage.}

  We hope that these changes highlight the broader scope and relevance
  of our work.

%%%%%%%%%%%%%%%%%%%%%%%%%%%%%%%%%%%%%%%%
\item Concerning:

  \blue{Abstract (L: line)}

  \blue{1. L15: The statement ``This capability exemplifies a new
    paradigm of using driven non-equilibrium systems to overcome
    current limitations in quantum metrology...'' goes unaddressed in
    the main text. The manuscript could be improved if this larger
    impact and its context were made clear, perhaps in the last
    paragraph of the introduction or in the conclusion.}

  We thank the referee for pointing out that our manuscript could be
  improved by following through on the subject of applying driven
  non-equilibrium systems to quantum metrology.  Accordingly, we have
  added the following text to the introduction, preceding the last
  paragraph (left column of page 2):

  \green{This capability mirrors efforts in other settings, such as NV
    centers[36, 37] and trapped ions[38], to enhance quantum metrology
    through the use of driven non-equilibrium phenomena.}

  As the subject of driven non-equilibrium does not play a central
  role in our manuscript, we have also rewritten the last sentence in
  the abstract to de-emphasize this point somewhat, replacing the
  text:

  \red{This capability exemplifies a new paradigm of using driven
    non-equilibrium systems to overcome current limitations in quantum
    metrology, allowing OLCs to enter a new regime of enhanced sensing
    with correlated quantum states.}

  by:

  \green{This capability allows OLCs to enter a new era of quantum
    enhanced sensing using correlated quantum states of driven
    non-equilibrium systems.}

%%%%%%%%%%%%%%%%%%%%%%%%%%%%%%%%%%%%%%%%
\item Concerning:

  \blue{Introduction (P: paragraph, L: line, skipping equations)}

  \blue{1. P4L11: While the orientation of the clock laser and optical
    lattice depth were parameters that were varied to examine the
    resulting optimum level of squeezing, strictly speaking these do
    not need to be actively controlled (beyond pointing/intensity
    stabilization) to generate spin squeezing itself; they just need
    to be set to a single value -- unless I am missing
    something. Perhaps this could be reworded.}

  Indeed, our squeezing protocol does not require {\it active} control
  over clock laser orientation and optical lattice depth.  To clarify
  this point, we have changed the text (page 2, top left paragraph):

  \red{To generate spin squeezing, our protocol only requires control
    over...}

  to:

  \green{To generate spin squeezing, our protocol only requires the
    capability to fix...}

%%%%%%%%%%%%%%%%%%%%%%%%%%%%%%%%%%%%%%%%%%%%%%%%%%
\item Concerning comments about the last paragraph of our introduction
  prior to revision:

  \blue{2. P5L4: The formulation ``...interact via purely contact
    collisions'' is a bit odd. Do you mean, ``solely interact via
    contact collisions''?}

  \blue{3. P5L8: The sentence ``In the spirit of quantum
    simulation... potential as a metrological resource'' seems a bit
    odd to me. Is the point that systems with short-range interactions
    can simulate long-range interacting systems, or that systems with
    short-range interactions can be used in metrology (because of
    effective long-range interactions)? Currently it reads as though
    the second point is being made, in which case I would remove ``In
    the spirit of quantum simulation'' and the parenthetical remark
    because metrology seems to be the main focus. Otherwise I would
    remove ``thus enhancing their potential as a metrological
    resource.''}

  \blue{4. P5L20: ``...which becomes possible in 3D optical lattice
    systems with quantized motional degrees of freedom in all three
    directions...''  As opposed to 3D optical lattices \_without\_
    quantized motional degrees of freedom in all three directions? The
    authors likely mean something along the lines of ``as a result of
    quantized motional degrees of freedom...''}

  These points are no longer applicable, as we have removed the
  relevant paragraph from our manuscript entirely, as detailed in
  point \ref{pt:R2.scope} of our response above.

%%%%%%%%%%%%%%%%%%%%%%%%%%%%%%%%%%%%%%%%%%%%%%%%%%
\item Concerning:

  \blue{From the Fermi-Hubbard model to one-axis and two-axis
    twisting}

  \blue{1. P9L11: In writing ``...up through U/J=8'', it sounds as if
    there is disagreement beyond U/J=8. To prevent this confusion, I
    would reword the statement.}

  To clarify this point, we have changed the text:

  \red{... up through $U/J=8$.}

  to:

  \green{... up through (and exceeding) $U/J=8$.}

%%%%%%%%%%%%%%%%%%%%%%%%%%%%%%%%%%%%%%%%%%%%%%%%%%
\item Concerning:

  \blue{2. P9L17: In the sentence beginning with ``Note that we choose
    filling f=5/6...'' it may be good to explicitly state that the
    optimum squeezing occurs at half-filling f=1.}

  To clarify this point, we have changed the sentence:

  \red{Note that we chose filling $f=5/6$ to demonstrate that our
    protocol should work even in this highly hole-doped case; in
    practice, optimized experiments are capable of achieving fillings
    closer to $f=1$.}

  to:

  \green{Note that we chose filling $f=5/6$ to demonstrate that our
    protocol should work, albeit sub-optimally, even in this highly
    hole-doped case; in practice, optimized experiments are capable of
    achieving fillings closer to the optimal $f=1$.}

%%%%%%%%%%%%%%%%%%%%%%%%%%%%%%%%%%%%%%%%%%%%%%%%%%
\item Concerning:

  \blue{3. FIG 2: In (b.ii), is it possible to use the same vertical
    axis range as (a.ii)? This would facilitate comparison between the
    two fillings.}

  We have modified figure 2 of the main text accordingly.

%%%%%%%%%%%%%%%%%%%%%%%%%%%%%%%%%%%%%%%%%%%%%%%%%%
\item Concerning:

  \blue{Experimental preparation of a squeezed state and practical
    considerations}

  \blue{1. P1: Rewording this paragraph could be a good opportunity to
    frame the current work beyond the single scope of the
    state-of-the-art OLC.}

  We thank the referee for identifying an opportunity for us to frame
  our work in a broader context.  Accordingly, we have replaced the
  first two sentences of Section III:

  \red{Here we discuss specific implementations of the above protocols
    in the state-of-the-art 3D $^{87}$Sr optical lattice clock (OLC).
    This system has demonstrated the capability to load a quantum
    degenerate gas into a 3D lattice at the ``magic wavelength''
    ($\lambda_{\t{lattice}}=2a\approx813$ nm) for which both the
    ground ($^1S_0,\dn$) and first excited ($^3P_0,\up$) electronic
    states of the atoms experience the same optical potential[4].}

  by:

  \green{Thus far, we have largely considered the general preparation
    of spin-squeezed states with the Fermi-Hubbard model.  Here, we
    discuss the specific implementation of the above protocols in the
    state-of-the-art 3D $^{87}$Sr optical lattice clock (OLC).  If
    successful, such an implementation would (to our knowledge) for
    the first time break through the proof-of-principle stage of spin
    squeezing efforts, and achieve a genuine metrological enhancement
    of a world-class quantum sensor.}

  \green{As required for our protocol, 3D $^{87}$Sr OLC has
    demonstrated the capability to load a quantum degenerate gas into
    a 3D lattice at the ``magic wavelength''
    ($\lambda_{\t{lattice}}=2a\approx813$ nm) for which both the
    ground ($^1S_0,\dn$) and first excited ($^3P_0,\up$) electronic
    states (i.e.~the ``clock states'') of the atoms experience the
    same optical potential[4].}

%%%%%%%%%%%%%%%%%%%%%%%%%%%%%%%%%%%%%%%%%%%%%%%%%%
\item Concerning:

  \blue{2. FIG 3: It may be helpful to the reader to add a dotted line
    to (b) and (c) indicating the boundary where the optical squeezing
    times for OAT and TAT are equal, to emphasize how TAT is faster
    for N$\sim>$500 atoms as noted in the caption.}

  We have added a dotted line to figure 3 marking the boundary at
  which optimal squeezing times for OAT and TAT are equal.
  Accordingly, we have modified the last sentence in the caption of
  figure 3, from:

  \red{... TAT ... achieves this squeezing faster for $N\gtrsim500$
    atoms.}

  to:

  \green{... TAT ... achieves optimal squeezing faster for
    $N\gtrsim400$ atoms, as denoted by a dotted line in panels ({\bf
      b}, {\bf c}).}

  Note that the change from \red{$N\gtrsim500$} to
  \green{$N\gtrsim400$} reflects the boundary between faster optimal
  squeezing under OAT vs.~TAT more accurately.

%%%%%%%%%%%%%%%%%%%%%%%%%%%%%%%%%%%%%%%%%%%%%%%%%%
\item Concerning:

  \blue{3. P4L20: When citing the level of spin squeezing attainable
    with your protocol, including some context or comparison may be
    helpful for the non-expert to evaluate the impact of your
    protocol.}

  We thank the referee for realizing that our manuscript would benefit
  from putting the amount of spin squeezing we report into context
  with other results in the literature.  We have consequently split
  off the following text in Section III:

  \red{The results in Fig.~4 show that squeezing via OAT saturates
    with system size around $N\approx10^3$ ($\ell\approx30$), while
    TAT allows for continued squeezing gains through $N=10^4$
    ($\ell=100$).  Even with decoherence, our protocol may
    realistically generate $\sim10$--$14$ dB of spin squeezing in
    $\sim1$ second with $\sim10^2$--$10^4$ atoms in a 2D section of
    the lattice, which is compatible with the atom numbers and
    interrogation times of state-of-the-art optical lattice clocks[4,
    5].}

  into a separate paragraph that elaborates on levels of spin
  squeezing attained in the literature:

  \green{The results in Fig.~4 show that squeezing via OAT saturates
    with system size around $N\approx10^3$ ($\ell\approx30$), while
    TAT allows for continued squeezing gains through $N=10^4$
    ($\ell=100$).  Even with decoherence, our protocol may
    realistically generate $\sim10$--$14$ dB of spin squeezing in
    $\sim1$ second with $\sim10^2$--$10^4$ atoms in a 2D section of
    the lattice, which is compatible with the atom numbers and
    interrogation times of state-of-the-art optical lattice clocks[4,
    5].  This amount of spin squeezing exceeds those reported in the
    ground-state nuclear spin sublevels of a state-of-the-art
    ${}^{171}$Yb OLC ($\sim6.5$ dB)[55].  While the latter protocol
    might be used to transfer spin squeezing to the electronic clock
    state, to date there has been no demonstration of spin squeezing
    in an optical clock transition.}

%%%%%%%%%%%%%%%%%%%%%%%%%%%%%%%%%%%%%%%%%%%%%%%%%%
\item Concerning:

  \blue{4. FIG 4: For this figure, as well as for figures 2 and 3, it
    may be helpful to use different colors and colorbars, one
    color(bar) for squeezing and one for time, kept consistent
    throughout.}

  At the request of the referee, we have made the color maps used in
  figures 2, 3, and 4 all distinct.  We have done so by changing the
  color map in figure 3, so as to keep a larger number of perceptually
  distinct colors in figure 4 (which makes it easier to match a color
  to a numerical value by eye).

%%%%%%%%%%%%%%%%%%%%%%%%%%%%%%%%%%%%%%%%%%%%%%%%%%
\item Concerning:

  \blue{5. FIG 4 caption: It would be helpful to note that the maximum
    value of the colorbar is about 20 dB less than the x-range maximum
    in figure 3a.}

  At the request of the referee, we have added the following sentence
  to the caption of figure 4:

  \green{The decoherence considered in this work also limits maximally
    achievable squeezing to $\sim20$ dB less than the decoherence-free
    maxima shown in Fig.~3.}

%%%%%%%%%%%%%%%%%%%%%%%%%%%%%%%%%%%%%%%%%%%%%%%%%%
\item Concerning:

  \blue{6. P5L3: Please quantify what you mean by ``shallow
    lattices'', both here and at the end of Appendix H.}

  This comment concerns words ``shallow lattices'' in the first
  sentence of the text:

  \red{In addition to light scattering, $p$-wave losses from inelastic
    ${}^3 P_0$ collisions[17, 19, 50] can also degrade the maximum
    achievable spin squeezing, particularly in shallow lattices.  More
    details on $p$-wave losses are discussed in Appendix H, where we
    show that operating at lattices deeper than $8 E_{\t{R}}$ may be
    necessary to suppress the impact of inelastic collisions on spin
    squeezing, at the cost of slightly increasing light scattering.}

  The purpose of the first sentence, which contains the words
  ``shallow lattices'', was not to quantify relevant lattice depths,
  but rather to communicate the fact that $p$-wave losses generally
  get worse with increasing lattice depth.  The {\it following}
  sentence then quantifies the implications of $p$-wave losses for
  operating lattice depths.  Our wording previously made this
  structure unclear, so we have changed the above sentences to read:

  \green{In addition to light scattering, $p$-wave losses from
    inelastic ${}^3 P_0$ collisions[17, 19, 56], can also degrade the
    maximum achievable spin squeezing, which becomes more pronounced
    for shallower lattices.  More details on $p$-wave losses are
    discussed in Appendix H, where we show that operating at lattice
    depths $V_0\gtrsim 7 E_{\t{R}}$ may be necessary to suppress the
    impact of inelastic collisions on spin squeezing, at the cost of
    slightly increasing light scattering.}

  Note that the change from \red{$8E_{\t{R}}$} to \green{$7E_{\t{R}}$}
  better reflects our findings in Appendix H, which determined that
  useful squeezing is still achievable in the presence of with
  $p$-wave losses with $U/J\gtrsim6$, at which $V_0\gtrsim7E_{\t{R}}$.

  For similar reasons, we have changed the following text in Appendix
  H:

  \red{As can be seen, the decay rate $\gamma$ is largely suppressed
    as $V_0$ increases but is not completely negligible at shallow
    lattices.  To quantify the effect ...}

  to:

  \green{The decay rate $\gamma$ is suppressed exponentially with
    increasing lattice depth $V_0$.  To quantify the effect ...}

  and we have changed the text (in the last paragraph of Appendix H):

  \red{For shallow lattices, the effect of inelastic collisions can
    limit the spin squeezing to $\sim10$ dB.}

  to:

  \green{For shallow lattices ($V_0\lesssim7E_{\t{R}}$), the effect of
    inelastic collisions can limit the spin squeezing to $\sim10$ dB.}

\end{enumerate}

%%%%%%%%%%%%%%%%%%%%%%%%%%%%%%%%%%%%%%%%%%%%%%%%%%%%%%%%%%%%%%%%%%%%%%
\section{External consultant}

\begin{enumerate}
\item Concerning:

  \blue{The present manuscript is a theory proposal motivated very
    much by the existing, and highly optimized setup of a Sr lattice
    clock in the Ye lab at JILA. There is no question that this is a
    world leading setup. To push the accuracy further, one question is
    how -- in this existing setup -- one can generate spin squeezed
    states. I note that spin squeezing is probably relevant for clocks
    involving a relatively small number of atoms (like present trapped
    ion experiments), but for lattice clocks with a very large number
    the relevance is a question to ask. Let me also mention that spin
    squeezing has been demonstrated with cavities etc. (Thompson at
    JILA, Vuletic MIT and others), but its actual usefulness in an
    actual leading atomic clock still awaits demonstration, as far as
    I know.}

  \blue{What this paper does is to start from Hubbard models, and
    design One-Axis-Twist Hamiltonians (there is a huge literature on
    this) from the Hubbard Hamiltonian naturally appearing in the
    lattice clock in the limit of shallow lattices. This is probably a
    limit they find it necessary to work in because of other
    constraints. Their idea is to get an infinite-range interaction
    required for One-Axis-Twisting by inducing an energy gap so that
    Hubbard model reduces to the One-Axis-Twist
    Hamiltonian. Conceptual ideas like this is what I have seen
    before, e.g.~in
    \url{https://journals.aps.org/pra/abstract/10.1103/PhysRevA.80.032311},
    though in a different experimental context.}

  As the consultant says, spin squeezing has indeed been demonstrated
  in other experimental platforms, and despite numerous proposals such
  as
  \href{https://journals.aps.org/pra/abstract/10.1103/PhysRevA.80.032311}{PhysRevA.80.032311},
  the ``actual usefulness'' of spin squeezing still awaits
  demonstration.  In our view, this point highlights the relevance and
  importance of our work, which goes beyond proof-of-principle
  demonstrations that spin squeezing might be used to achieve some
  technological advantage.  Specifically, our work outlines the first
  (to our knowledge) {\it realistic} proposal to generate {\it useful}
  spin squeezing that would push the state-of-the-art in a world-class
  sensor.  To emphasize this point, we have rewritten the last
  paragraph of our introduction (as detailed in point
  \ref{pt:R2.scope} of Section \ref{sec:R2} above).

%%%%%%%%%%%%%%%%%%%%%%%%%%%%%%%%%%%%%%%%
\item Concerning:

  \blue{My conclusions are:}

  \blue{Pushing atomic clocks, and lattice clocks in particular, to a
    next level of accuracy, and exploring the use of spin squeezing,
    and in particular its use in an actual precision experiment, is
    highly relevant.}

  \blue{The present proposal might be a good way to go in the Jun Ye
    lab, and would thus have quite an impact if successful.}

  \blue{However, this paper is quite specialized to the JILA setup,
    and the question is if these are ideas which are conceptually new,
    and applicable in a broader context. Here my answer would be
    slightly mixed: using Spins Orbit Coupling to open gaps in the
    spectrum to make infinite range interactions is an interesting
    idea. But is it enough?}

  \blue{Of course, if they can demonstrate the existence of squeezing
    based on these ideas in a real experiment at JILA on whatever
    level, I would be unreservedly impressed and enthusiastic. As a
    theory paper alone, it might be a bit too specialized. It is the
    old question and dilemma for theorists: publish a theory paper by
    itself, or wait until experimentalists actually do it -- which
    might take a long time.}

  We appreciate the recognition that our proposal is ``highly
  relevant'', with the potential to have ``quite an impact''.  In
  light of this recognition, and the admission that the consultant
  would be ``unreservedly impressed and enthusiastic'' with a
  successful experimental implementation of our proposal, with all due
  respect we must express firm disagreement with the consultant's
  present takeaway from the ``old question and dilemma'' of whether
  theorists should publish their ideas, or wait until these ideas are
  implemented experimentally.  Due to the scarcity of realistic
  proposals for a practical quantum advantage in world-class sensors,
  we believe that our proposal and methods are worthwhile
  disseminating to a broad scientific audience that can scrutinize our
  ideas, and build on them to develop their own (possibly modified or
  improved) implementations.  We hope that the editors agree with our
  view on the importance to widely disseminate not only experimental
  results, but also theoretical proposals that guide experiments in
  ways that actually advance state-of-the-art sensors and pave the
  ground for the practical development and application of new
  technologies.

\end{enumerate}

%%%%%%%%%%%%%%%%%%%%%%%%%%%%%%%%%%%%%%%%%%%%%%%%%%%%%%%%%%%%%%%%%%%%%%
\section{Miscellaneous changes}

\begin{enumerate}
\item To clarify our high-level summary of the mechanism for spin
  squeezing, we have added the following sentence to the summary of
  our results in the introduction (second to last paragraph of Section
  I, top left of page 2 in the revised text):

  \green{Interactions thereby prolong inter-particle spin coherence
    through a spin-locking effect, which additionally transforms the
    dephasing effect of SOC into a collective spin squeezing process.}

\item To add structure to our manuscript, we have changed the title of
  Section II from ``\red{From the Fermi-Hubbard model to one-axis and
    two-axis twisting}'' to ``\green{Spin squeezing with the
    Fermi-Hubbard model}'', and we have added the sub-section headings
  ``\green{Model validity}'' (II.A) and ``\green{Two-axis twisting}''
  (II.B).

\item We have shortened the title of Section III, from
  ``\red{Experimental preparation of a squeezed state and practical
    considerations}'' to ``\green{Experimental implementation and
    practical considerations}''.

\item To clarify some notation, we have changed the last sentence
  before the conclusion, from:

  \red{Such a pulse can transfer $\ket{g, I=-7/2}$ to
    $\ket{e, I=-9/2}$ without affecting $\ket{g, I=-9/2}$.}

  to:

  \green{Such a pulse can transfer $\ket{g,-7/2}$ to $\ket{e,-9/2}$
    without affecting $\ket{g,-9/2}$, where $g$ and $e$ respectively
    denote the ground and excited (electronic) clock states.}

\item We have added the following text to the beginning of the caption
  to figure H1:

  \green{{\bf $p$-wave loss rates}.}
\end{enumerate}

\end{document}
