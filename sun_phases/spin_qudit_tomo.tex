\documentclass[notitlepage,twocolumn]{revtex4-2}

\usepackage{setspace} % to change text spacing

%%% linking references
\usepackage{hyperref}
\hypersetup{
  breaklinks=true,
  colorlinks=true,
  linkcolor=blue,
  filecolor=magenta,
  urlcolor=cyan,
}

%%% symbols, notations, etc.
\usepackage{physics,braket,bm,amssymb} % physics and math
\renewcommand{\t}{\text} % text in math mode
\newcommand{\f}[2]{\dfrac{#1}{#2}} % shorthand for fractions
\newcommand{\p}[1]{\left(#1\right)} % parenthesis
\renewcommand{\sp}[1]{\left[#1\right]} % square parenthesis
\renewcommand{\set}[1]{\{#1\}} % curly parenthesis
\renewcommand{\Set}[1]{\left\{#1\right\}} % curly parenthesis
\newcommand{\bk}{\Braket} % shorthand for braket notation
\renewcommand{\v}{\bm} % bold vectors
\newcommand{\uv}[1]{\bm{\hat{#1}}} % unit vectors
\renewcommand{\c}{\cdot} % inner product
\renewcommand{\i}{\mathrm{i}\mkern1mu} % imaginary unit

% double angle brackets
\newcommand{\bbk}[1]{\langle\!\langle #1 \rangle\!\rangle}
\newcommand{\Bbk}[1]
{\left\langle\!\!\left\langle #1 \right\rangle\!\!\right\rangle}

\usepackage{dsfont} % for identity operator
\newcommand{\1}{\mathds{1}}

\newcommand{\up}{\uparrow}
\newcommand{\dn}{\downarrow}
\newcommand{\x}{\text{x}}
\newcommand{\y}{\text{y}}
\newcommand{\z}{\text{z}}

\newcommand{\B}{\mathcal{B}}
\newcommand{\M}{\mathcal{M}}
\renewcommand{\O}{\mathcal{O}}
\newcommand{\Q}{\mathcal{Q}}
\newcommand{\R}{\mathcal{R}}
\newcommand{\T}{\mathcal{T}}
\newcommand{\Y}{\mathcal{Y}}

\newcommand{\D}{\mathfrak{D}}
\renewcommand{\dd}{\mathfrak{d}}
\renewcommand{\d}{\text{d}}

\newcommand{\SO}{\text{SO}}

\newcommand{\CC}{\mathbb{C}}
\newcommand{\RR}{\mathbb{R}}
\renewcommand{\SS}{\mathbb{S}}
\newcommand{\ZZ}{\mathbb{Z}}

\def\obra#1{\mathinner{({#1}|}}
\def\oket#1{\mathinner{|{#1})}}
\def\obk#1{\mathinner{({#1})}}
\def\oop#1#2{\oket{#1}\!\obra{#2}}

%%% figures
\usepackage{graphicx} % for figures
\graphicspath{{./figures/}} % set path for all figures

% to place figures in the correct section
\usepackage[section]{placeins}

\usepackage[inline]{enumitem} % in-line lists and \setlist{} (below)
\setlist[enumerate,1]{label={(\roman*)}} % default in-line numbering
\setlist{nolistsep} % more compact spacing between environments

%%% text markup
\usepackage{color} % text color
\newcommand{\red}[1]{{\color{red} #1}}
\newcommand{\todo}[1]{{\color{blue} To do: #1}}

%%%%%%%%%%%%%%%%%%%%%%%%%%%%%%%%%%%%%%%%%%%%%%%%%%%%%%%%%%%%%%%%%%%%%%
\begin{document}

\newcommand{\JILA}{JILA, National Institute of Standards and Technology and
  University of Colorado, 440 UCB, Boulder, Colorado 80309, USA}
\newcommand{\CTQM}{Center for Theory of Quantum Matter, University of Colorado, Boulder, CO, 80309, USA}

\newcommand{\thetitle}{Spin qudit tomography}

\title{\thetitle}
\author{Michael A.~Perlin}
\email{mika.perlin@gmail.com}
\author{Ana Maria Rey}
\affiliation{\JILA}
\affiliation{\CTQM}
\date{\today}

\keywords{qudits; spin qudits; quantum state tomography}

\begin{abstract}
  We consider the task of performing quantum state tomography on a $d$-state spin qudit, using only measurements of spin projection onto different quantization axes.
  By mapping this quantum tomography problem onto the classical problem of signal recovery on the sphere, we prove that full reconstruction of arbitrary mixed states is possible, and requires a minimum number of measurement axes, $r_d^{\t{min}}$, that is bounded by $2d-1\le r_d^{\t{min}}\le d^2$.
  We conjecture that $r_d^{\t{min}}=2d-1$, which we verify for all $d\le200$ using numerical experiments.
  We then provide an algorithm with $O\p{d^4}$ runtime for certifying an upper bound on the error with which measurements of spin projection can reconstruct an unknown quantum state.
  Our algorithm motivates a simple randomized tomography protocol, which we use to study the trade-off between the number of measurement axes and state reconstruction error at a fixed total number of measurements [\red{note: this part is not yet complete}].
  [\red{todo: maybe look into a simplified tomography procedure for pure states?]}
\end{abstract}

\maketitle

%%%%%%%%%%%%%%%%%%%%%%%%%%%%%%%%%%%%%%%%%%%%%%%%%%%%%%%%%%%%%%%%%%%%%%
\section{Introduction}

Quantum state tomography, or the task of reconstructing a quantum state by collecting and processing measurement data, is an essential primitive for quantum sensing, quantum simulation, and quantum information processing.
The central importance of quantum state tomography has led to the development of a variety of techniques based on compressed sensing \cite{gross2010quantum}, matrix product states \cite{cramer2010efficient}, maximum-likelihood estimation \cite{smolin2012efficient}, Bayesian inference \cite{huszar2012adaptive}, and neural networks \cite{torlai2018neuralnetwork}, among others \cite{christandl2012reliable, qi2013quantum, opatrny1997leastsquares}.
These techniques are typically developed in a general, information-theoretic setting, and make minimal assumptions about the physical medium of a quantum state.
As a consequence, even well-established techniques can nonetheless be ill-suited for physical platforms with unique or limited capabilities.

Due in part to advancements in experimental capabilities to address nuclear spins in ultracold atomic systems [\red{CITE}], as well as developments in molecular cooling techniques [\red{CITE}], a particular setting of growing interest is the spin qudit, or a multilevel quantum angular momentum degree of freedom.
Spin qudits can provide advantages over their qubit counterparts for quantum sensing \cite{evrard2019enhanced}, enable quantum simulations of SU($d$) magnetism \cite{banerjee2013atomic, cazalilla2014ultracold, zhang2014spectroscopic, scazza2014observation, goban2018emergence}, and offer unique capabilities for quantum error correction \cite{albert2020robust, gross2020encoding}.
In all cases, quantum state tomography is necessary to take full advantage of a spin qudit.

The problem of qudit tomography in particular is not new, with an extensive literature on a variety of techniques \cite{manko1997spin, thew2002qudit, flammia2005minimal, salazar2012quantum, sosa-martinez2017quantum, ha2018minimal, evrard2019enhanced, stefano2019set, palici2020oam}.
However, existing proposals either require the capability to perform essentially arbitrary operations on a qudit \cite{thew2002qudit, flammia2005minimal, salazar2012quantum, sosa-martinez2017quantum, ha2018minimal, stefano2019set, palici2020oam}, or rely on infinite-dimensional representations of a quantum spin \cite{manko1997spin, evrard2019enhanced}, resulting in tomographic protocols that can be highly inefficient or unachievable in practice.
In particular, the protocols based on infinite-dimensional representations of a quantum spin require only the capability to measure spin projection onto different spatial axes, which is usually straightforward with a spin qudit.
Nonetheless, these protocols obfuscate the basic requirements for performing full state tomography, provide no straightforward error bounds or guarantees of accuracy, and extract only a vanishing faction of the information contained in measurement data.

In this work, we provide an efficient framework for performing spin qudit tomography using only measurements of spin projection onto different spatial axes.
We first introduce, in Section \ref{sec:transition_ops}, a basis of operators that have a simple physical interpretation, and transform nicely under rotations in 3D space, thereby respecting the symmetries of the tomography task at hand.
We then map this quantum tomography problem onto the well-studied classical problem of signal recovery on the sphere in Section \ref{sec:signal_recovery}, thereby importing a host of existing literature and mathematical machinery that has been developed for this task \cite{mcewen2011novel, mcewen2011sampling, rauhut2011sparse, alem2012sparse, khalid2014optimaldimensionality}.
In particular, this mapping allows us to bound the minimum number of measurement axes necessary to perform full tomography on a $d$-level spin qudit, $r_d$, by $2d-1\le r_d\le d^2$.

[\red{finish intro}]

%%%%%%%%%%%%%%%%%%%%%%%%%%%%%%%%%%%%%%%%%%%%%%%%%%%%%%%%%%%%%%%%%%%%%%
\section{Spin transition operators and rotation matrices}
\label{sec:transition_ops}

Consider a $d$-state spin qudit with total spin $s\equiv\frac{d-1}{2}$.
The defining property of a spin qudit is the fact that it describes an angular momentum degree of freedom, which has specific implications for how a spin qudit should transform under the group SO(3) of rotations in 3D space.
Due to the central importance of these transformation rules for a spin qudit, we first identify a basis of operators that transforms nicely under 3D rotations.
One such basis is provided by the {\it transition operators} (also known as polarization operators \cite{kryszewski2006positivity, bertlmann2008bloch}), defined by
\begin{align}
  T_{\ell m} \equiv \sqrt{\f{2\ell+1}{2s+1}} \sum_{\mu,\nu=-s}^s
  \bk{s\mu;\ell m|s\nu} \op{\nu}{\mu},
  \label{eq:trans_op}
\end{align}
where $\bk{s\mu;\ell m|s\nu}$ is a Clebsh-Gordan coefficient that enforces $\ell\in\set{0,1,\cdots,2s}$ and $m\in\set{-\ell,-\ell+1,\cdots,\ell}$, such that there are $d^2$ operators in total.
For brevity, we will generally treat the value of $d$ (or $s$) as constant but arbitrary throughout this work, and suppress any explicit dependence of quantities or operators such as $T_{\ell m}$ on $d$.
The transition operators are orthonormal with respect to the trace inner product $\obk{\O|\Q}\equiv\tr\p{\O^\dag\Q}$, and transform nicely under conjugation:
\begin{align}
  \obk{T_{\ell m }|T_{\ell'm'}}
  = \delta_{\ell\ell'} \delta_{mm'},
  &&
  T_{\ell m}^\dag = \p{-1}^m T_{\ell,-m},
\end{align}
where $\delta_{kk'}\equiv 1$ if $k=k'$ and $0$ otherwise.
As a consequence, we can expand any density operator $\rho$ in the transition operator basis as
\begin{align}
  \rho = \sum_{\ell=0}^{2s} \sum_{m=-\ell}^\ell \rho_{\ell m} T_{\ell m},
  &&
  \rho_{\ell m}\equiv\obk{T_{\ell m}|\rho},
  \label{eq:trans_state}
\end{align}
where $\rho^\dag=\rho$ implies that $\rho_{\ell m}^*=\p{-1}^m\rho_{\ell,-m}$.
The transition operators can be physically interpreted as the transition $\ket\psi\to T_{\ell m}\ket\psi$ induced on the state $\ket\psi$ of a spin-$s$ system by the absorption of a spin-$\ell$ particle with spin projection $m$ onto a fixed quantization axis.
In analogy with the spherical harmonics $Y_{\ell m}$, we will refer to the indices $\ell$ and $m$ respectively as the {\it degree} and {\it order} of $T_{\ell m}$.

The transition operators are spherical tensor operators, whose degree $\ell$ is preserved under 3D rotations generated by the spin operators $S_\x,S_\y,S_\z$ (see Appendix \ref{sec:rotations}).
For any triplet of Euler angles $\v\omega=\p{\alpha_{\v\omega},\beta_{\v\omega},\gamma_{\v\omega}}$, we can therefore define the {\it fundamental} rotation operator
\begin{align}
  R\p{\v\omega} \equiv e^{-\i\alpha_{\v\omega}S_\z} e^{-\i\beta_{\v\omega}S_\y} e^{-\i\gamma_{\v\omega}S_\z},
  \label{eq:rot_op}
\end{align}
as well as the corresponding {\it adjoint} rotation operator $\R\p{\v\omega}$ defined by $\R\p{\v\omega}\O = R\p{\v\omega} \O R\p{\v\omega}^\dag$, and expand
\begin{align}
  T_{\v\omega\ell m} \equiv
  \R\p{\v\omega} T_{\ell m}
  = \sum_{n=-\ell}^\ell \D_{nm}^\ell\p{\v\omega} T_{\ell n}.
  \label{eq:trans_rot}
\end{align}
The adjoint rotation matrix elements $\D_{mn}^\ell\p{\v\omega}\equiv\obk{T_{\ell m}|\R\p{\v\omega}|T_{\ell n}}$ are an analog of the fundamental (Wigner) rotation matrix elements $D_{mn}^\ell\p{\v\omega}\equiv\bk{\ell m|R\p{\v\omega}|\ell n}$ that govern rotations of spin-$\ell$ particles and spherical harmonics $Y_{\ell m}$, where $\ket{\ell m}$ denotes the state of a spin-$\ell$ qudit with spin projection $m$ onto a quantization axis.

It will be useful to think of rotation matrix elements $D^\ell_{mn}$ and $\D^\ell_{mn}$ as scalar functions of Euler angles $\v\omega\in\Omega_3$.
As functions of $\v\omega$, the fundamental rotation matrix elements $D_{mn}^\ell$ satisfy the orthogonality relations \cite{brown2003rotational}
\begin{align}
  \int_{\Omega_3} \d\Omega_3\p{\v\omega}
  D^{\ell}_{mn}\p{\v\omega}^* D^{\ell'}_{m'n'}\p{\v\omega}
  = \f{8\pi^2}{2\ell+1} \delta_{\ell\ell'} \delta_{mm'} \delta_{nn'},
  \label{eq:ortho_full}
\end{align}
where integration is performed over all $\v\omega\in\Omega_3\simeq[0,2\pi]\times[0,\pi]\times[0,2\pi]$ with integration measure $\d\Omega_3\p{\v\omega} \equiv \d\alpha_{\v\omega} \d\beta_{\v\omega} \d\gamma_{\v\omega} \sin\beta_{\v\omega}$.
In particular, the restriction of $D^\ell_{m,0}$ to the sphere $\Omega_2$ at $\gamma_{\v\omega}=0$ recovers (up to complex conjugation) the spherical harmonics $Y_{\ell m}$, with orthogonality relations
\begin{align}
  \int_{\Omega_2} \d\Omega_2\p{\v v}
  D^{\ell}_{m,0}\p{\v v}^* D^{\ell'}_{m',0}\p{\v v}
  = \f{4\pi}{2\ell+1} \delta_{\ell\ell'} \delta_{mm'},
  \label{eq:ortho_sphere}
\end{align}
where integration is performed over all $\v v\in\Omega_2\simeq[0,2\pi]\times[0,\pi]$ with integration measure $\d\Omega_2\p{\v v} \equiv \d\alpha_{\v v} \d\beta_{\v v} \sin\beta_{\v v}$, and for ease of notation we implicitly associate doublets $\v v=\p{\alpha_{\v v},\beta_{\v v}}$ with the triplet $\p{\alpha_{\v v},\beta_{\v v},0}$, so that for example $R\p{\v v} = R\p{\alpha_{\v v},\beta_{\v v},0}$ and $T_{\v v\ell m} = T_{\p{\alpha_{\v v},\beta_{\v v},0},\ell m}$.
The same orthogonality relations in Eqs.~\eqref{eq:ortho_full} and \eqref{eq:ortho_sphere} are inherited by the adjoint rotation matrix elements $\D_{mn}^\ell$ through the fact $\R\p{\v\omega}$ is the adjoint action of $R\p{\v\omega}$ (see Appendix \ref{sec:ortho}), and the functions $\set{\D_{m,0}^\ell}$ restricted to $\Omega_2$ are similarly another orthogonal basis for the space $L^2\p{\Omega_2}$ of square-integrable functions on the sphere.

%%%%%%%%%%%%%%%%%%%%%%%%%%%%%%%%%%%%%%%%%%%%%%%%%%%%%%%%%%%%%%%%%%%%%%
\section{Spin tomography as signal recovery on the sphere}
\label{sec:signal_recovery}

Our goal is to reconstruct an arbitrary state of a spin qudit from measurements of spin projection onto different quantization axes.
We are thus nominally restricted to measuring projectors $\Pi_{\v v\mu} \equiv \op{\mu_{\v v}}$, where $\ket{\mu_{\v v}}\equiv R\p{\v v}\ket{\mu}$ is a state with spin projection $\mu$ onto a measurement axis determined by a point $\v v$ on the sphere $\Omega_2$.
For any fixed axis $\v v$, the sets $\set{\Pi_{\v v\mu}}$ and $\set{T_{\v v\ell,0}}$ are both orthonormal bases for the space of operators that are diagonal in the basis $\set{\ket{\mu_{\v v}}}$.
By a simple change of operator basis, measuring the projectors $\set{\Pi_{\v v\mu}}$ is therefore equivalent to measuring the transition operators $\set{T_{\v v\ell,0}}$, and provides data on the expectation values $\bk{T_{\v \ell,0}}_\rho$, where $\bk{X}_\rho\equiv\tr\p{\rho X}$.

Reconstructing an arbitrary density operator $\rho$ essentially requires us to find a set of axes $V=\set{\v v}$ and associated coefficients $C_{\ell mk}\p{\v v}$ that would allow us to recover any coefficient $\rho_{\ell m}$ of the expansion in Eq.~\eqref{eq:trans_state} through
\begin{align}
  \rho_{\ell m}^* = \bk{T_{\ell m}}_\rho
  = \sum_{\v v,k} C_{\ell mk}\p{\v v} \bk{T_{\v vk,0}}_\rho.
  \label{eq:state_recon}
\end{align}
Expanding the rotated transition operators $T_{\v vk,0}$ into a sum of un-rotated transition operators, according to Eq.~\eqref{eq:trans_rot}, we find that the recovery condition in Eq.~\eqref{eq:state_recon} is satisfied when
\begin{align}
  T_{\ell m}
  = \sum_{\v v,k,n} C_{\ell mk}\p{\v v} \D^k_{n,0}\p{\v v} T_{kn}.
\end{align}
Orthogonality of the transition operators motivates the ansatz $C_{\ell mk}\p{\v v}=\delta_{k\ell}C_{\ell m}\p{\v v}$, which in turn implies that
\begin{align}
  \sum_{\v v} C_{\ell m}\p{\v v} \D^\ell_{n,0}\p{\v v} = \delta_{mn}
  \label{eq:tomo_recovery}
\end{align}
for all $\ell$.

In fact, the problem of finding suitable axes $V$ and coefficients $C_{\ell m}\p{\v v}$ to satisfy Eq.~\eqref{eq:tomo_recovery}, solving which would enable the recovery of arbitrary spin qudit states, can be mapped onto the well-studied problem of signal recovery on the sphere \cite{mcewen2011novel, mcewen2011sampling, rauhut2011sparse, alem2012sparse, khalid2014optimaldimensionality}.
The signal recovery problem can be stated as follows: given a square-integrable function $f\in L^2\p{\Omega_2}$ on the sphere, with the harmonic expansion
\begin{align}
  f\p{\v v} = \sum_{\ell,m} f_{\ell m} B_{\ell m}\p{\v v},
\end{align}
where $f_{\ell m}$ are expansion coefficients and $\set{B_{\ell m}}$ is an orthogonal basis for $L^2\p{\Omega_2}$, find a set of points $V=\set{\v v}\subset\Omega_2$ and associated coefficients $C_{\ell m}\p{\v v}$ with which we can reconstruct $f$, or equivalently its coefficients $f_{\ell m}$, from knowledge of the function's value $f\p{\v v}$ at all points $\v v\in V$; that is
\begin{align}
  f_{\ell m} = \sum_{\v v} C_{\ell m}\p{\v v} f\p{\v v}
  = \sum_{\v v,k,n} C_{\ell m}\p{\v v} f_{kn} B_{kn}\p{\v v}.
\end{align}
Reconstruction of functions with arbitrary coefficients $f_{\ell m}$ then implies that
\begin{align}
  \sum_{\v v} C_{\ell m}\p{\v v} B_{kn}\p{\v v}
  = \delta_{k\ell} \delta_{mn},
  \label{eq:full_recovery}
\end{align}
which is a stronger version of the condition that we found for the spin qudit tomography problem in Eq.~\eqref{eq:tomo_recovery}.
We will refer to Eq.~\eqref{eq:full_recovery} as the {\it full recovery problem}, and Eq.~\eqref{eq:tomo_recovery} as the the {\it reduced recovery problem}.
Any solution to the full recovery problem with basis functions $B_{\ell m}=\D^\ell_{m,0}$ automatically solves the reduced recovery problem.

If
\begin{enumerate*}
\item the functions $B_{\ell m}$ are indexed by an integer degree $\ell\ge0$ that is preserved under rotations of the sphere and an integer order $m\in\set{-\ell,-\ell-1,\cdots,\ell}$, as with the spherical harmonics $Y_{\ell m}$ and $\D^\ell_{m,0}$, and
\item the function $f$ is {\it band-limited} at degree $L$, which is to say that $f_{\ell m}=0$ for all $\ell\ge L$,
\end{enumerate*}
then the full recovery problem in Eq.~\eqref{eq:full_recovery} is provably solvable with a suitable choice of $\abs{V}=L^2$ points \cite{freeden2008spherical, freeden2018spherical}.
The existence of these solutions to the full recovery problem in turn implies the existence of $d^2$ measurement axes that suffice to reconstruct arbitrary states of $d$-level spin qudit, whose possible states are essentially ``band-limited'' at degree $d$ \footnote{Spin qudit states can be faithfully represented by quasi-probability distributions on a sphere.  For a $d$-level spin, these quasi-probability distributions are band-limited at degree $d$.}.

Moreover, for any fixed degree $\ell$, finding solutions to the reduced recovery problem in Eq.~\eqref{eq:tomo_recovery} is equivalent to the recovery of a function in the degree-$\ell$ subspace of $L^2\p{\Omega_2}$, which is possible with $\abs{V}=2\ell+1$ samples (equal to the dimension of the degree-$\ell$ subspace) \cite{freeden2008spherical}.
In the case of spin qudit tomography, the degree $\ell$ takes a maximal value of $\ell_{\t{max}}\equiv d-1$, so state recovery requires at least as many measurement axes as there are transition operators with degree $\ell_{\t{max}}$, namely $2\ell_{\t{max}}+1=2d-1$.
Since
\begin{enumerate*}
\item there are only $2\ell+1\le2d-1$ transition operators of degree $\ell\le\ell_{\t{max}}$, and
\item measuring spin projections along any axis simultaneously provides measurement data for transition operators of {\it all} degrees $\ell\le\ell_{\t{max}}$,
\end{enumerate*}
one might expect that $2d-1$ suitably chosen measurement axes would always suffice to perform full tomography of a $d$-level spin qudit.
In the language of classical signal recovery, we would expect there to exist a set of $2\ell_{\t{max}}+1$ points on the sphere that enable the recovery of any function with fixed degree $\ell\le\ell_{\t{max}}$ (given prior knowledge of $\ell$) \footnote{We could further conjecture that if {\it any} set of points $V\subset\Omega_2$ enables the recovery of degree-$\ell_{\t{max}}$ functions on the sphere, then $V$ would allow recovery degree-$\ell$ functions with $\ell\le\ell_{\t{max}}$.  However, this condition is stronger than that necessary to guarantee full recovery of a $d$-level spin qudit state with $2d-1$ measurement axes.}.
Unfortunately, we have no general proof of this conjecture, which may require a deeper analysis of fixed-degree signal recovery on the sphere.
Nonetheless, we have numerically verified that $2d-1$ measurement axes suffice for all $d\le200$ (see Section \ref{sec:axes}).

In principle, any solution to the full recovery problem with a particular set of basis functions, such the spherical harmonics $Y_{\ell m}$, can be readily translated into a solution with a different set of basis functions, such as the adjoint rotation matrix elements $\D^\ell_{m,0}$ (see Appendix \ref{sec:basis_change}).
This translation enables importing a host of existing signal recovery algorithms \cite{mcewen2011novel, mcewen2011sampling, rauhut2011sparse, alem2012sparse, khalid2014optimaldimensionality} for the task of spin qudit tomography.
In practice, spin qudits typically have only a modest dimension $d$, allowing for simpler and optimized tomography protocols that are practical despite worse scaling with $d$.

%%%%%%%%%%%%%%%%%%%%%%%%%%%%%%%%%%%%%%%%%%%%%%%%%%%%%%%%%%%%%%%%%%%%%%
\section{Reconstruction error bound}
\label{sec:bound}

For the practically minded, the mere existence of solutions to a problem is less interesting than the exposition of a particular solution.
On a high level, a spin qudit tomography protocol needs to
\begin{enumerate*}
\item select a set of measurement axes,
\item collect measurement data on spin projection onto these axes, and then
  \label{tomo:measure}
\item process the collected data to reconstruct the state of the spin qudit.
\end{enumerate*}
Whereas step \ref{tomo:measure} can involve a host of platform-dependent technical challenges, here we discuss the steps to take before and after collecting measurement data.

To this end, we begin by asking a question: what is a ``good'' choice of measurement axes?
Intuitively, a good choice of axes should minimize the error with which one can reconstruct an unknown quantum state from associated measurement data.
If we can quantify this intuition, then we can try to optimize over different choices of measurement axes to find a set that (approximately) minimizes the error in reconstructed states.

A set of measurement axes $V=\set{\v v}$ nominally induces a set of projectors $\set{\Pi_{\v v\mu}^V}$ that will be measured in an experiment, where for clarify we index objects by $V$ to emphasize their dependence on the choice of measurement axes.
By a simple change of basis, measuring these projectors is equivalent to measuring the set of transition operators $\v T_V\equiv\set{T_{\v v\ell,0}}$.
The operators in $\v T_V$ can be used to construct the overlap (Gram) matrix
\begin{align}
  G_V \equiv \sum_{\v v,\v v',\ell}
  \obk{T_{\v v\ell,0}|T_{\v v'\ell,0}} \op{\v v\ell}{\v v'\ell},
  \label{eq:overlap_matrix}
\end{align}
where we neglect cross terms with different degrees $\ell$ because the corresponding inner products vanish, i.e.~$\obk{T_{\v v\ell,0}|T_{\v v'\ell',0}}=0$ for all $\ell\ne\ell'$.
If the operators in $\v T_V$ span the entire ($d^2$-dimensional) space of operators on a $d$-level spin qudit, which is a necessary and sufficient condition for $V$ to allow for full state tomography, then $G_V$ must have exactly $d^2$ nonzero eigenvalues.
Indexing these eigenvalues $G^V_k$ and the corresponding (normalized) eigenvectors $\v x^V_k\equiv(x^V_{k,1},x^V_{k,2},\cdots)$ by an integer $k\in\set{1,2,\cdots,d^2}$, we can construct the orthonormal operators
\begin{align}
  Q^V_k \equiv \sum_j g^V_{kj} T_j,
  &&
  g^V_{kj} \equiv \f{x^V_{kj}}{\sqrt{G^V_k}},
\end{align}
where for shorthand we use a combined index $j=\p{\v v_j,\ell_j}$ to specify both a measurement axis $\v v_j$ and a degree $\ell_j$, which identify the transition operator $T_j\equiv T_{\v v_j\ell_j,0}$.
These operators allow us to expand any state $\rho$ of a $d$-level spin qudit in the form
\begin{align}
  \rho = \sum_{k=1}^{d^2} \rho_k^V Q_k^V,
  &&
  \rho_k^V \equiv \bk{Q^V_k}_\rho.
\end{align}
Given empirical estimates $E_j$ of the expectation values $\bk{T_j}_\rho$, an empirical estimate $\tilde\rho_V$ of $\rho$ is then
\begin{align}
  \tilde\rho_V \equiv \sum_k \tilde\rho^V_k Q^V_k,
  \label{eq:reconstructed_state}
\end{align}
where
\begin{align}
  \tilde\rho^V_k \equiv \sum_j g^V_{kj} E_j
  \approx \sum_j g^V_{kj} \bk{T_j}_\rho
  = \bk{Q^V_k}_\rho
  = \rho^V_k.
\end{align}
The eigenvalues of the overlap matrix $G_V$ allow us to make concrete statements about the statistical error between the empirical estimate $\tilde\rho_V$ and the true state $\rho$.
Assume, for example, that the estimates $E_j$ are equal to $\bk{T_j}_\rho$ up to Gaussian noise with variance at most $\epsilon^2$, i.e.
\begin{align}
  E_j = \bk{T_j}_\rho + \epsilon_j,
  &&
  \bbk{\epsilon_j \epsilon_{j'}} \le \epsilon^2 \delta_{jj'},
\end{align}
where $\set{\epsilon_j}$ are independent random variables, and we use the double brackets $\bbk{\cdot}$ to denote statistical averaging.
The mean squared error with which $\tilde\rho^V_k$ approximates $\rho^V_k$ is then
\begin{align}
  \Bbk{\abs{\tilde\rho^V_k-\rho^V_k}^2}
  &= \Bbk{\p{\tilde\rho^V_k-\rho^V_k}^*\p{\tilde\rho^V_k-\rho^V_k}} \\
  &= \sum_{j,j'} \p{g^V_{kj}}^* g^V_{kj'}\,
  \bbk{\epsilon_j \epsilon_{j'}} \\
  &\le \sum_j \abs{g^V_{kj}}^2 \epsilon^2
  = \f{\epsilon^2}{G^V_k}.
\end{align}
Using the fact that the operators $Q^V_k$ are orthonormal, we can therefore bound the mean squared 2-norm (Euclidean) distance between $\tilde\rho_V$ and $\rho$ as
\begin{align}
  \Bbk{\norm{\tilde\rho_V-\rho}_2^2}
  = \sum_k \Bbk{\abs{\tilde\rho^V_k - \rho^V_k}^2}
  \le \epsilon^2 \M_V^2,
  \label{eq:bound}
\end{align}
where
\begin{align}
  \M_V^2 \equiv \sum_k \f1{G^V_k}
  \label{eq:scale}
\end{align}
is a scale determined by the set of measurement axes $V$.
Note that the sum in Eq.~\eqref{eq:scale} is performed over the $d^2$ largest eigenvalues of $G_V$, and therefore may be divergent if $G_V$ has fewer than $d^2$ nonzero eigenvalues.

The complexity of computing $\M_V^2$ is determined by the need to:
\begin{enumerate*}
\item compute all rotated transition operators $T_{\v v\ell,0}$, which by the expansion $T_{\v v\ell,0}=R\p{\v v}T_{\ell,0}R\p{\v v}^\dag$ takes $O(\abs{V}d^4)$ serial or $O(d^3)$ parallel runtime;
\item construct the overlap matrix $G_V$, whose block diagonal structure makes possible in $O(\abs{V}^2d^3)$ serial or $O(\abs{V}^2d^2)$ parallel runtime; and
\item diagonalize the blocks of $G_V$, which takes $O(\abs{V}^3d)$ serial or $O(\abs{V}^3)$ parallel runtime.
\end{enumerate*}
We discuss these calculations in more detail in Appendix \ref{sec:overlaps}.
Altogether, assuming $\abs{V}\propto d$ the bound in Eq.~\eqref{eq:bound} can be certified in $O(d^5)$ serial or $O(d^4)$ parallel runtime.
In practice, for modest values of $d$ we find that serial execution time is divided about evenly between computing rotated transition operators $T_{\v v\ell,0}$ and constructing the overlap matrix $G_V$, with a negligible amount of time spent on diagonalization (see Figure [\red{REF}]).  [\red{todo: add figure showing runtime for each routine as a function of $d$ with $\abs{V}=2d-1$; modify text appropriately after making the figure}].
We suspect that more efficient representations of the transition operators $T_{\ell m}$, namely as $\p{2\ell+1}$-component vectors rather than $d\times d$ matrices, together with appropriate changes to the representation of the adjoint rotation operator $\R\p{\v v}$, can reduce the complexity of computing $\M_V^2$ by a factor of $O(d)$.

%%%%%%%%%%%%%%%%%%%%%%%%%%%%%%%%%%%%%%%%%%%%%%%%%%%%%%%%%%%%%%%%%%%%%%
\section{Tomography protocols}
\label{sec:protocol}

The ability to certify a statistical error bound on the empirical estimate $\tilde\rho_V$ of an unknown quantum state $\rho$ motivates the following simple protocol for spin qudit tomography: select a random set of measurement axes $V$ by uniformly sampling points on the sphere \footnote{To sample a point $\p{\alpha,\beta}$ from the uniform distribution on the sphere (with azimuthal angle $\alpha$ and polar angle $\beta$), you can sample a point $\p{a,b}\in[0,1]\times[0,1]$ from the uniform distribution on the unit square, and then set $\alpha=2\pi a$ and $\beta=\arccos\p{1-2b}$.}, and variationally optimize the $2\abs{V}$ parameters in $V$ (two angles for each point $\v v\in V$) to minimize the error scale $\M_V$ defined in Eq.~\eqref{eq:scale}.
If $\abs{V}$ is too large for variational optimization to be practical, one can instead generate many random sets of measurement axes, and then choose the set that minimizes $\M_V$.
After choosing a set of measurement axes $V$, experimentally characterize the expectation values $\bk{T_j}_\rho$.
If the experimental measurements $E_j$ of $\bk{T_j}_\rho$ are accurate to within an error $\lesssim\epsilon$, then the state $\tilde\rho_V$ defined in Eq.~\eqref{eq:reconstructed_state} is an approximation of the true state $\rho$ to within a 2-norm distance $\lesssim\epsilon\M_V$.
Furthermore, if the empirical estimate $\tilde\rho_V$ has negative eigenvalues, its 2-norm distance from $\rho$ can be further reduced with maximum-likelihood corrections \cite{smolin2012efficient}.

Note that any information about $\rho$, obtained either from prior knowledge or preliminary measurement data, can be used to construct tailored or adaptive measurement protocols \cite{pereira2018adaptive} that are more efficient in terms of the number of measurements required to estimate $\rho$ to a fixed precision.
We leave the development of tailored and adaptive measurement protocols to future work.

As a last point, we mention that the error scale $\M_V$ can also evaluate the ``quality'' of measurement axes $V$ chosen with existing methods for sparse signal recovery on the sphere \cite{rauhut2011sparse, alem2012sparse}.
In practice, we find that uniform random sampling outperforms these methods, in the sense that it generates low-error sets of measurement axes with higher probability.
Two observations may explain the poor performance of these sparse signal recovery methods.
First, spin qudit tomography essentially boils down to recovering sparse signals with a particular harmonic structure, namely a fixed degree $\ell$.
The existing sparse signal recovery algorithms do not assume or account for any harmonic structure, and may therefore perform worse than expected on this particular subset of signals.
Second, spin qudit tomography requires recovering signals that are band-limited at degree $d$, but have $O(d)$ non-zero harmonic coefficients (out of $d^2$ coefficients in total).
The existing sparse signal recovery methods, on the other hand, may only perform well for signals with sparsity much smaller than $d$.
In any case, the fact that uniform sampling outperforms these methods only simplifies our randomized tomography protocol.

%%%%%%%%%%%%%%%%%%%%%%%%%%%%%%%%%%%%%%%%%%%%%%%%%%%%%%%%%%%%%%%%%%%%%%
\section{How many measurement axes?}
\label{sec:axes}

We conjecture, and numerically verify for $d\le200$, that measurements of spin projection along $2d-1$ axes suffice to reconstruct an arbitrary state of a $d$-level spin qudit (see Figure [\red{REF}]).
Even so, $2d-1$ measurement axes may not be the ``optimal'' number to use.
Increasing the number of measurement axes generally decreases the error scale $\M_V$, but at the cost of having to estimate more observables.
If we fix the total number of measurements (or experimental trials), then having to estimate more observables means that fewer measurements are devoted to estimating any particular observable, which in turn increases the error $\epsilon$ on those estimates.
This trade-off begs the question: how many measurement axes should be chosen to perform spin qudit tomography?

The reconstruction error bound in Eq.~\eqref{eq:bound} provides a straightforward answer: the number of measurement axes should be chosen to minimize the product $\epsilon^2\M_V^2$, which bounds the expected error in the estimate $\tilde\rho_V$ of an unknown state $\rho$.
If there are a total of $N$ spin projection measurements onto $r$ quantization axes, then $\sim N/r$ measurements are used to estimate spin projection onto any given axis.
The variance in estimates of spin projection is then $\epsilon^2\sim r/N$, or simply $\epsilon^2\sim r$ at a fixed total number of measurements $N$.
Altogether, the number of measurement axes should be chosen in such a way as to minimize $r\M_V^2$ with $\abs{V}=r$.




\bibliography{\jobname.bib}


\onecolumngrid
\appendix

%%%%%%%%%%%%%%%%%%%%%%%%%%%%%%%%%%%%%%%%%%%%%%%%%%%%%%%%%%%%%%%%%%%%%%
\section{Rotating transition operators}
\label{sec:rotations}

Defining
\begin{align}
  S_\z \equiv \sum_{\mu=-s}^s \mu \op{\mu},
  &&
  S_\pm \equiv \sum_{\mu=-s}^s
  \sqrt{s\p{s+1}-\mu\p{\mu\pm1}} \op{\mu\pm1}{\mu},
\end{align}
as well as
\begin{align}
  S_\x \equiv \f12\p{S_+ + S_-},
  &&
  S_\y \equiv -\f\i2\p{S_+-S_-},
  &&
  \v S \equiv \p{S_\x,S_\y,S_\z},
\end{align}
the spin vector $\v S$ generates rotations of a spin-$s$ system in 3D space.
Specifically, the operator $e^{-\i\theta\v S\cdot\uv n}$ rotates a spin-$s$ system by an angle $\theta$ about the unit vector $\uv n$.
Furthermore, observing that $S_\z=T_{1,0}$ and $S_\pm\propto T_{1,\pm1}$, we can use the product operator expansion of the transition operators (see Appendix \ref{sec:trans_prod}), the properties of Clebsch-Gordan coefficients, and a computer algebra system to simplify the commutators
\begin{align}
  \sp{S_\z,T_{\ell m}} = m\, T_{\ell m},
  &&
  \sp{S_\pm,T_{\ell m}} = \sqrt{\p{\ell\mp m}\p{\ell\pm m+1}}\, T_{\ell,m\pm1},
\end{align}
which implies that $T_{\ell m}$ is a spherical tensor operator, and in particular that its degree $\ell$ is preserved under rotations generated by $\v S$.
For any triplet of Euler angles $\v\omega=\p{\alpha_{\v\omega},\beta_{\v\omega},\gamma_{\v\omega}}$, we can therefore define the fundamental rotation operator
\begin{align}
  R\p{\v\omega} \equiv e^{-\i\alpha_{\v\omega}S_\z} e^{-\i\beta_{\v\omega}S_\y} e^{-\i\gamma_{\v\omega}S_\z},
\end{align}
as well as the corresponding adjoint rotation operator $\R\p{\v\omega}$ defined by $\R\p{\v\omega}\O = R\p{\v\omega} \O R\p{\v\omega}^\dag$, and expand
\begin{align}
  \R\p{\v\omega} T_{\v\omega\ell m}
  = \sum_{n=-\ell}^\ell \D_{nm}^\ell\p{\v\omega} T_{\ell n}.
\end{align}
The adjoint rotation matrix elements $\D_{mn}^\ell\p{\v\omega}\in\CC$ are
\begin{align}
  \D_{mn}^\ell\p{\v\omega}
  &\equiv \obk{T_{\ell m}|\R\p{\v\omega}|T_{\ell n}} \\
  &= \tr\p{T_{\ell m}^\dag e^{-\i\alpha_{\v\omega}S_\z} e^{-\i\beta_{\v\omega}S_\y} e^{-\i\gamma_{\v\omega}S_\z} T_{\ell n}
    e^{\i\gamma_{\v\omega}S_\z} e^{\i\beta_{\v\omega}S_\y} e^{\i\alpha_{\v\omega}S_\z}} \\
  &= e^{-\i\alpha_{\v\omega} m} \,\dd_{mn}^\ell\p{\beta_{\v\omega}}\, e^{-\i\gamma_{\v\omega} n},
\end{align}
and the {\it reduced} adjoint rotation matrix elements $\dd_{mn}^\ell\p{\beta}\in\RR$ are
\begin{align}
  \dd_{mn}^\ell\p{\beta}
  &\equiv \D_{mn}^\ell\p{0,\beta,0} \\
  &= \tr\p{T_{\ell m}^\dag e^{-\i\beta S_\y} T_{\ell n} e^{\i\beta S_\y}} \\
  &= \f{2\ell+1}{2s+1} \sum_{\mu,\nu=-s}^s
  \bk{s\mu;\ell m|s,\mu+m} \bk{s\nu;\ell n|s,\nu+n} \times
  \tr\p{\op{\mu}{\mu+m} e^{-\i\beta S_\y}
    \op{\nu+n}{\nu} e^{\i\beta S_\y}} \\
  &= \f{2\ell+1}{2s+1} \sum_{\mu,\nu=-s}^s
  \bk{s\mu;\ell m|s,\mu+m} \bk{s\nu;\ell n|s,\nu+n} \times
  d_{\mu\nu}^s\p{\beta} d_{\mu+m,\nu+n}^s\p{\beta}.
\end{align}
Here $d_{\mu\nu}^s\p{\beta}\equiv D^s_{\mu\nu}\p{0,\beta,0}\in\RR$ are
matrix elements of a reduced fundamental (Wigner) rotation matrix.
Note that the above product of reduced fundamental rotation matrix elements can be expanded as \cite{rose1957elementary}
\begin{align}
  d_{\mu\nu}^s\p{\beta} d_{\mu+m,\nu+n}^s\p{\beta}
  = \sum_{L=0}^{2s} \bk{s\mu;s,\mu+m|L,2\mu+m}
  \bk{s\nu;s,\nu+n|L,2\nu+n} d_{2\mu+m,2\nu+n}^L\p{\beta}.
\end{align}

%%%%%%%%%%%%%%%%%%%%%%%%%%%%%%%%%%%%%%%%%%%%%%%%%%%%%%%%%%%%%%%%%%%%%%
\section{Rotation matrix elements as orthogonal functions}
\label{sec:ortho}

As functions of Euler angle triplets $\v\omega=\p{\alpha_{\v\omega},\beta_{\v\omega},\gamma_{\v\omega}}$, the fundamental (Wigner) rotation matrix elements $D_{mn}^\ell$ and satisfy the orthogonality relations \cite{brown2003rotational}
\begin{align}
  \int_{\Omega_3} \d\Omega_3\p{\v\omega}
  D^{\ell}_{mn}\p{\v\omega}^* D^{\ell'}_{m'n'}\p{\v\omega}
  = \f{8\pi^2}{2\ell+1} \delta_{\ell\ell'} \delta_{mm'} \delta_{nn'},
  \label{eq:orthogonal}
\end{align}
where integration is performed over all $\v\omega\in\Omega_3\simeq[0,2\pi]\times[0,\pi]\times[0,2\pi]$ with integration measure $\d\Omega_3\p{\v\omega}\equiv\d\alpha_{\v\omega} \d\beta_{\v\omega} \d\gamma_{\v\omega} \sin\beta_{\v\omega}$.
This orthogonality relation is a consequence of the fact that, for a fixed degree $\ell$, the matrix
\begin{align}
  D^\ell \equiv \sum_{m,n} D^\ell_{mn} \op{m}{n}
\end{align}
is an irreducible unitary matrix representation of SO(3), where by unitary we mean that
\begin{align}
  \sum_k D^\ell_{km}\p{\v\omega}^* D^\ell_{kn}\p{\v\omega} = \delta_{mn}
  ~\t{for all}~\v\omega\in\Omega_3\simeq\SO(3).
\end{align}
Unitarity of $D^\ell$ follows from the fact that the matrix $D^\ell\p{\v\omega}$ is essentially the degree-$\ell$ block of the block-diagonal fundamental rotation matrix $R\p{\v\omega}$, which is itself unitary.
The Schur orthogonality relations for irreducible unitary representations of compact Lie groups then imply that \cite{cornwell1997group}
\begin{align}
  \int_{\Omega_3} \d\Omega_3\p{\v\omega}
  D^{\ell}_{mn}\p{\v\omega}^* D^{\ell'}_{m'n'}\p{\v\omega}
  = \f{\abs{\Omega_3}}{\dim\p{D^\ell}}
  \delta_{\ell\ell'} \delta_{mm'} \delta_{nn'},
  \label{eq:ortho_full_apndx}
\end{align}
where $\dim\p{D^\ell}=2\ell+1$ is the dimension of $D^\ell$, and analogously
\begin{align}
  \abs{\Omega_3} \equiv \int_{\Omega_3} \d\Omega_3\p{\v\omega} = 8\pi^2.
\end{align}
Orthogonality relations for adjoint matrix elements $\D^\ell_{mn}$ are derived identically from the fact that the matrix
\begin{align}
  \D^\ell \equiv \sum_{m,n} \D^\ell_{mn} \op{m}{n}
\end{align}
is an irreducible unitary representation of SO(3).
This property of $\D^\ell$ follows from the fact that it is an adjoint representation of SO(3), which is isomorphic to the standard representation $D^\ell$ \cite{hall2015lie}.

The decomposition $D^\ell_{mn}\p{\v\omega} = e^{-\i\alpha_{\v\omega}m} d^\ell_{mn}\p{\beta_{\v\omega}} e^{-\i\gamma_{\v\omega}n}$, where $d^\ell_{mn}\p{\beta}\equiv D^\ell_{mn}\p{0,\beta,0}$ is
a matrix element of a reduced fundamental (Wigner) rotation matrix \cite{rose1957elementary}, implies that setting $n'=n$ in in Eq.~\eqref{eq:ortho_full_apndx} makes the integrand independent of $\gamma_{\v\omega}$.
Evaluating the integral over $\gamma_{\v\omega}$ and dividing through by $2\pi$ then recovers the orthogonality relations
\begin{align}
  \int_{\Omega_2} \d\Omega_2\p{\v v}
  D^{\ell}_{mn}\p{\v v}^* D^{\ell'}_{m'n}\p{\v v}
  = \f{4\pi}{2\ell+1} \delta_{\ell\ell'} \delta_{mm'},
  \label{eq:ortho_sphere_apndx}
\end{align}
where integration is performed over all $\v v\in\Omega_2\simeq[0,2\pi]\times[0,\pi]$ with integration measure $\d\Omega_2\p{\v v} \equiv \d\alpha_{\v v} \d\beta_{\v v} \sin\beta_{\v v}$.
A similar decomposition of the adjoint rotation matrix elements $\D^\ell_{mn}\p{\v\omega}$ (see Appendix \ref{sec:rotations}) implies the identical orthogonality relations.

%%%%%%%%%%%%%%%%%%%%%%%%%%%%%%%%%%%%%%%%%%%%%%%%%%%%%%%%%%%%%%%%%%%%%%
\section{Changing bases for the signal recovery problem}
\label{sec:basis_change}

Here we show how one solution to the problem of signal recovery on the sphere with a particular basis of functions (such as the spherical harmonics $Y_{\ell m}$) can be translated into a solution with a different basis (such as the adjoint rotation matrix elements $\D^\ell_{m,0}$).
Consider a square-integrable function on the sphere, $f\in L^2\p{\Omega_2}$, with the harmonic expansion
\begin{align}
  f\p{\v v} = \sum_{\ell,m} f_{\ell m} B_{\ell m},
\end{align}
where $f_{\ell m}\in\CC$ is a coefficient and $\set{B_{\ell m}}$ is a complete basis for $L^2\p{\Omega_2}$.
Without loss of generality, we will assume that for a fixed degree $\ell$ the functions $\set{B_{\ell m}}$ span the degree-$\ell$ subspace of $L^2\p{\Omega_2}$, and that these functions satisfy orthogonality relations analogous to those in Eqs.~\eqref{eq:ortho_sphere} and \eqref{eq:ortho_sphere_apndx}.
A solution to the signal recovery problem amounts to a set of points $V=\set{\v v}\subset\Omega_2$ and coefficients $C_{\ell m}\p{\v v}$ for which
\begin{align}
  f_{\ell m} = \sum_{\v v} C_{\ell m}\p{\v v} f\p{\v v}.
  \label{eq:recovery_solution}
\end{align}
Consider now a different basis $\set{\tilde B_{\ell m}}$ for $L^2\p{\Omega_2}$, satisfying the same general assumptions that we made for $\set{B_{\ell m}}$, with which
\begin{align}
  f\p{\v v} = \sum_{\ell, m} \tilde f_{\ell m} \tilde B_{\ell m}.
\end{align}
Substituting the harmonic expansion of $f\p{\v v}$ in the original basis $\set{B_{\ell m}}$, we find that
\begin{align}
  \tilde f_{\ell m} = \int_{\Omega_2} \d\Omega_2\p{\v v}
  \tilde B_{\ell m}\p{\v v}^* f\p{\v v}
  = \sum_n G^\ell_{mn} f_{\ell n},
  &&
  G^\ell_{mn} \equiv \int_{\Omega_2} \d\Omega_2\p{\v v}
  \tilde B_{\ell m}\p{\v v}^* B_{\ell n}\p{\v v},
  \label{eq:basis_change}
\end{align}
where integration is performed over all $\v v\in\Omega_2\simeq[0,2\pi]\times[0,\pi]$ with integration measure $\d\Omega_2\p{\v v} \equiv \d\alpha_{\v v} \d\beta_{\v v} \sin\beta_{\v v}$.
Note that the inner product between functions $\tilde B_{\ell m}\p{\v v}$ and $B_{kn}\p{\v v}$ for $k\ne\ell$ vanishes because these functions live in orthogonal subspaces of $L^2\p{\Omega_2}$ \cite{freeden2008spherical}.
Substituting Eq.~\eqref{eq:recovery_solution} into Eq.~\eqref{eq:basis_change}, we find that
\begin{align}
  \tilde f_{\ell m}
  = \sum_n G^\ell_{mn} \sum_{\v v} C_{\ell n}\p{\v v} f\p{\v v}
  = \sum_{\v v} \tilde C_{\ell m}\p{\v v} f\p{\v v},
  &&
  \tilde C_{\ell m}\p{\v v} \equiv \sum_n G^\ell_{mn} C_{\ell n}\p{\v v}.
\end{align}
The coefficients $\tilde C_{\ell m}\p{\v v}$ thus solve the signal recovery problem with the new basis $\set{\tilde B_{\ell m}}$ for $L^2\p{\Omega_2}$.

%%%%%%%%%%%%%%%%%%%%%%%%%%%%%%%%%%%%%%%%%%%%%%%%%%%%%%%%%%%%%%%%%%%%%%
\section{Computing the overlap matrix}
\label{sec:overlaps}

[\red{todo: walk through the calculation of the error scale $\M_V$.}]

%%%%%%%%%%%%%%%%%%%%%%%%%%%%%%%%%%%%%%%%%%%%%%%%%%%%%%%%%%%%%%%%%%%%%%
\section{Transition operator product expansion}
\label{sec:trans_prod}

The transition operators on the $d$-dimensional Hilbert space of a spin-$s$ system (with $s\equiv\frac{d-1}{2}$) are defined by
\begin{align}
  T_{\ell m} \equiv \sqrt{\f{2\ell+1}{2s+1}} \sum_{\mu,\nu=-s}^s
  \bk{s\mu;\ell m|s\nu} \op{\nu}{\mu},
\end{align}
where $\bk{s\mu;\ell m|s\nu}$ is a Clebsh-Gordan coefficient that enforces $\ell\in\set{0,1,\cdots,2s}$ and $m\in\set{-\ell,-\ell+1,\cdots,\ell}$.
We wish to compute the coefficients of the operator product expansion
\begin{align}
  T_{\ell_1 m_1} T_{\ell_2 m_2}
  = \sum_{L,M} f_{\ell_1 m_1;\ell_2 m_2}^{LM} T_{LM},
  &&
  f_{\ell_1 m_1;\ell_2 m_2}^{LM}
  \equiv \obk{T_{LM} | T_{\ell_1 m_1} T_{\ell_2 m_2}}.
\end{align}
Using the symmetry properties of Clebsch-Gordan coefficients, namely
\begin{align}
  \bk{\ell_1 m_1; \ell_2 m_2| L M}
  &= \p{-1}^{\ell_2+m_2} \sqrt{\f{2L+1}{2\ell_1+1}}
  \bk{L,-M; \ell_2 m_2| \ell_1,-m_1} \\
  \bk{\ell_1 m_1; \ell_2 m_2| L M}
  &= \p{-1}^{\ell_1+\ell_2-L}
  \bk{\ell_1,-m_1; \ell_2,-m_2| L,-M},
\end{align}
we can find that the transition operators transform under conjugation as
\begin{align}
  T_{\ell m}^\dag
  = \sqrt{\f{2\ell+1}{2s+1}}
  \sum_{\mu,\nu} \p{-1}^m \bk{s\nu;\ell,-m|s\mu} \op{\mu}{\nu}
  = \p{-1}^m T_{\ell,-m},
\end{align}
which implies that
\begin{align}
  f_{\ell_1 m_1;\ell_2 m_2}^{LM}
  = \p{-1}^M \sqrt{\f{\p{2L+1}\p{2\ell_1+1}\p{2\ell_2+1}}
    {\p{2s+1}\p{2s+1}\p{2s+1}}}
  \sum_{\mu,\nu,\rho} \bk{s\nu;L,-M|s\mu}
  \bk{s\rho;\ell_1m_1|s\nu} \bk{s\mu;\ell_2m_2|s\rho}.
\end{align}
Replacing Clebsch-Gordan coefficients by Wigner 3-$j$ symbols with the identity
\begin{align}
  \bk{\ell_1 m_1; \ell_2 m_2| L M}
  = \p{-1}^{2\ell_2} \p{-1}^{L-M} \sqrt{2L+1}
  \begin{pmatrix}
    L & \ell_2 & \ell_1 \\
    -M & m_2 & m_1
  \end{pmatrix},
\end{align}
we can use the fact that $2\ell_2$ is (in our case) always even to expand
\begin{align}
  f_{\ell_1 m_1;\ell_2 m_2}^{LM}
  = \p{-1}^M \sqrt{\p{2L+1}\p{2\ell_1+1}\p{2\ell_2+1}}
  \sum_{\mu,\nu,\rho} \p{-1}^{3s-\mu-\nu-\rho}
  \begin{pmatrix}
    s & L & s \\
    -\mu & -M & \nu
  \end{pmatrix}
  \begin{pmatrix}
    s & \ell_1 & s \\
    -\nu & m_1 & \rho
  \end{pmatrix}
  \begin{pmatrix}
    s & \ell_2 & s \\
    -\rho & m_2 & \mu
  \end{pmatrix}.
\end{align}
This sum can be simplified by the introduction of Wigner 6-$j$ symbols, giving us
\begin{align}
  f_{\ell_1 m_1;\ell_2 m_2}^{LM}
  &= \p{-1}^{2s+M} \sqrt{\p{2L+1}\p{2\ell_1+1}\p{2\ell_2+1}}
  \begin{pmatrix}
    L & \ell_1 & \ell_2 \\
    M & -m_1 & -m_2
  \end{pmatrix}
  \begin{Bmatrix}
    L & \ell_1 & \ell_2 \\
    s & s & s
  \end{Bmatrix} \\
  &= \p{-1}^{2s+L} \sqrt{\p{2\ell_1+1}\p{2\ell_2+1}}
  \bk{\ell_1 m_1; \ell_2 m_2| LM}
  \begin{Bmatrix}
    \ell_1 & \ell_2 & L \\
    s & s & s
  \end{Bmatrix}.
\end{align}

\end{document}

%%% Local Variables:
%%% mode: latex
%%% TeX-master: t
%%% End:


signal recovery literature:
-- a $\sim2d^2$ signaling theorem \cite{mcewen2011novel}
-- a nice review of these methods \cite{mcewen2011sampling}
-- an empirically accurate (albeit without rigorous proofs/guarantees) $d^2$ sampling method \cite{khalid2014optimaldimensionality}
-- current status of the theory behind sparse signal recovery with random sampling \cite{rauhut2011sparse} (note: existing bounds are not believed to be tight)
-- nice numerical method for sparse signal recovery \cite{alem2012sparse}
-- useful textbook references \cite{freeden2008spherical, freeden2018spherical}
