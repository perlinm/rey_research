\documentclass[notitlepage,twocolumn]{revtex4-2}

\usepackage{setspace} % to change text spacing

%%% linking references
\usepackage{hyperref}
\hypersetup{
  breaklinks=true,
  colorlinks=true,
  linkcolor=blue,
  filecolor=magenta,
  urlcolor=cyan,
}

%%% symbols, notations, etc.
\usepackage{physics,braket,bm,amssymb} % physics and math
\renewcommand{\t}{\text} % text in math mode
\newcommand{\f}[2]{\dfrac{#1}{#2}} % shorthand for fractions
\newcommand{\p}[1]{\left(#1\right)} % parenthesis
\renewcommand{\sp}[1]{\left[#1\right]} % square parenthesis
\newcommand{\bk}{\Braket} % shorthand for braket notation
\renewcommand{\v}{\bm} % bold vectors
\newcommand{\uv}[1]{\bm{\hat{#1}}} % unit vectors
\renewcommand{\i}{\mathrm{i}\mkern1mu} % imaginary unit

% curly brackets for a set
\renewcommand{\set}[1]{\{#1\}} % curly parenthesis
\renewcommand{\Set}[1]{\left\{#1\right\}} % curly parenthesis

% double angle brackets
\newcommand{\bbk}[1]{\langle\!\langle #1 \rangle\!\rangle}
\newcommand{\Bbk}[1]
{\left\langle\!\!\left\langle #1 \right\rangle\!\!\right\rangle}

\newcommand{\x}{\text{x}}
\newcommand{\y}{\text{y}}
\newcommand{\z}{\text{z}}
\renewcommand{\d}{\text{d}}

\newcommand{\C}{\mathcal{C}}
\newcommand{\D}{\mathcal{D}}
\newcommand{\E}{\mathcal{E}}
\newcommand{\I}{\mathcal{I}}
\renewcommand{\L}{\mathcal{L}}
\newcommand{\N}{\mathcal{N}}
\renewcommand{\O}{\mathcal{O}}
\newcommand{\Q}{\mathcal{Q}}
\renewcommand{\S}{\mathcal{S}}

\newcommand{\ZZ}{\mathbb{Z}}

\def\obra#1{\mathinner{({#1}|}}
\def\oket#1{\mathinner{|{#1})}}
\def\obk#1{\mathinner{({#1})}}
\def\oop#1#2{\oket{#1}\!\obra{#2}}

\DeclareMathOperator{\cov}{cov}
\DeclareMathOperator{\diag}{diag}

\usepackage{dsfont} % for identity operator
\newcommand{\1}{\mathds{1}}

%%% figures
\usepackage{graphicx} % for figures
\graphicspath{{./figures/}} % set path for all figures

% to place figures in the correct section
\usepackage[section]{placeins}

\usepackage[inline]{enumitem} % in-line lists and \setlist{} (below)
\setlist[enumerate,1]{label={(\roman*)}} % default in-line numbering
\setlist{nolistsep} % more compact spacing between environments

%%% text markup
\usepackage{color} % text color
\newcommand{\red}[1]{{\color{red} #1}}

%%%%%%%%%%%%%%%%%%%%%%%%%%%%%%%%%%%%%%%%%%%%%%%%%%%%%%%%%%%%%%%%%%%%%%
\begin{document}

\newcommand{\JILA}{JILA, National Institute of Standards and Technology and
  University of Colorado, 440 UCB, Boulder, Colorado 80309, USA}
\newcommand{\CTQM}{Center for Theory of Quantum Matter, University of Colorado, Boulder, CO, 80309, USA}

\newcommand{\thetitle}{Spin qudit tomography}

\title{\thetitle}
\author{Michael A.~Perlin}
\email{mika.perlin@gmail.com}
\author{Diego Barberena}
\author{Ana Maria Rey}
\affiliation{\JILA}
\affiliation{\CTQM}
\date{\today}

\keywords{qudits; spin qudits; quantum state tomography}

\begin{abstract}
  We consider the task of performing quantum state tomography on a $d$-state spin qudit, using only measurements of spin projection onto different quantization axes.
  By mapping this quantum tomography problem onto the classical problem of signal recovery on the sphere, we prove that full reconstruction of arbitrary mixed states is possible, and requires a minimum number of measurement axes, $r_d^{\t{min}}$, that is bounded by $2d-1\le r_d^{\t{min}}\le d^2$.
  We conjecture that $r_d^{\t{min}}=2d-1$, which we verify for all $d\le200$ using numerical experiments.
  We then provide an algorithm with $O\p{rd^2}$ runtime for certifying an upper bound on the error with which measurements of spin projection along $r$ axes can reconstruct an unknown quantum state.
  Our algorithm motivates a simple randomized tomography protocol, which we use to study the trade-off between the number of measurement axes and state reconstruction error at a fixed total number of measurements [\red{note: this part is not yet complete}].
  [\red{todo: maybe look into a simplified tomography procedure for pure states?]}
\end{abstract}

\maketitle

%%%%%%%%%%%%%%%%%%%%%%%%%%%%%%%%%%%%%%%%%%%%%%%%%%%%%%%%%%%%%%%%%%%%%%
\section{Introduction}

Quantum state tomography, or the task of reconstructing a quantum state by collecting and processing measurement data, is an essential primitive for quantum sensing, quantum simulation, and quantum information processing.
The central importance of quantum state tomography has led to the development of a variety of techniques based on compressed sensing \cite{gross2010quantum}, matrix product states \cite{cramer2010efficient}, maximum-likelihood estimation \cite{smolin2012efficient}, Bayesian inference \cite{huszar2012adaptive}, and neural networks \cite{torlai2018neuralnetwork}, among others \cite{christandl2012reliable, qi2013quantum, opatrny1997leastsquares}.
These techniques are typically developed in a general, information-theoretic setting, and make minimal assumptions about the physical medium of a quantum state.
As a consequence, even well-established techniques can nonetheless be ill-suited for physical platforms with unique or limited capabilities.

Due in part to advancements in experimental capabilities to address nuclear spins in ultracold atomic systems [\red{CITE}], as well as developments in molecular cooling techniques [\red{CITE}], a particular setting of growing interest is the spin qudit, or a multilevel quantum angular momentum degree of freedom.
Spin qudits can provide advantages over their qubit counterparts for quantum sensing \cite{evrard2019enhanced}, enable quantum simulations of SU($d$) magnetism \cite{banerjee2013atomic, cazalilla2014ultracold, zhang2014spectroscopic, scazza2014observation, goban2018emergence}, and offer unique capabilities for quantum error correction \cite{albert2020robust, gross2020encoding}.
In all cases, quantum state tomography is necessary to take full advantage of a spin qudit.

The problem of qudit tomography in particular is not new, with an extensive literature on a variety of techniques \cite{manko1997spin, thew2002qudit, flammia2005minimal, salazar2012quantum, sosa-martinez2017quantum, ha2018minimal, evrard2019enhanced, stefano2019set, palici2020oam}.
However, existing proposals either require the capability to perform essentially arbitrary operations on a qudit \cite{thew2002qudit, flammia2005minimal, salazar2012quantum, sosa-martinez2017quantum, ha2018minimal, stefano2019set, palici2020oam}, or rely on infinite-dimensional representations of a quantum spin \cite{manko1997spin, evrard2019enhanced}, resulting in tomographic protocols that can be highly inefficient or unachievable in practice.
In particular, the protocols based on infinite-dimensional representations of a quantum spin require only the capability to measure spin projection onto different spatial axes, which is usually straightforward with a spin qudit.
Nonetheless, these protocols obfuscate the basic requirements for performing full state tomography, provide no straightforward error bounds or guarantees of accuracy, and extract only a vanishing faction of the information contained in measurement data.

In this work, we provide an efficient framework for performing spin qudit tomography using only measurements of spin projection onto different spatial axes.
We first introduce, in Section \ref{sec:transition_ops}, a basis of operators that have a simple physical interpretation, and transform nicely under rotations in 3D space, thereby respecting the symmetries of the tomography task at hand.
We then map this quantum tomography problem onto the well-studied classical problem of signal recovery on the sphere in Section \ref{sec:signal_recovery}, thereby importing a host of existing literature and mathematical machinery that has been developed for this task \cite{mcewen2011novel, rauhut2011sparse, alem2012sparse, khalid2014optimaldimensionality}.
In particular, this mapping allows us to bound the minimum number of measurement axes necessary to perform full tomography on a $d$-level spin qudit, $r_d^{\t{min}}$, by $2d-1\le r_d^{\t{min}}\le d^2$.

[\red{finish intro}]

%%%%%%%%%%%%%%%%%%%%%%%%%%%%%%%%%%%%%%%%%%%%%%%%%%%%%%%%%%%%%%%%%%%%%%
\section{Spin transition operators and 3D rotations}
\label{sec:transition_ops}

[\red{todo: investigate and comment on the connection between $T_{\ell m}$ and $Y_{\ell m}$?}]

Consider a $d$-state spin qudit with total spin $s\equiv\frac{d-1}{2}$.
The defining property of a spin qudit is the fact that it describes an angular momentum degree of freedom, which has specific implications for how a spin qudit should transform under the group SO(3) of rotations in 3D space.
Due to the central importance of these transformation rules for a spin qudit, we first identify a basis of operators that transforms nicely under 3D rotations.
One such basis is provided by the {\it transition operators} (also known as polarization operators \cite{kryszewski2006positivity, bertlmann2008bloch}), defined by
\begin{align}
  T_{\ell m} \equiv \sqrt{\f{2\ell+1}{2s+1}} \sum_{\mu,\nu=-s}^s
  \bk{s\mu;\ell m|s\nu} \op{\nu}{\mu},
  \label{eq:trans_op}
\end{align}
where $\bk{s\mu;\ell m|s\nu}$ is a Clebsh-Gordan coefficient that enforces $\ell\in\set{0,1,\cdots,d-1}$ and $m\in\set{-\ell,-\ell+1,\cdots,\ell}$, such that there are $d^2$ operators in total.
For brevity, we will generally treat the value of $d$ (or $s$) as constant but arbitrary throughout this work, and suppress any explicit dependence of quantities or operators such as $T_{\ell m}$ on $d$.
The transition operators are orthonormal with respect to the trace inner product, and transform nicely under conjugation:
\begin{align}
  \obk{T_{\ell m }|T_{\ell'm'}}
  = \delta_{\ell\ell'} \delta_{mm'},
  &&
  T_{\ell m}^\dag = \p{-1}^m T_{\ell,-m},
\end{align}
where for any $d\times d$ matrix $X=\sum_{\mu,\nu} X_{\mu\nu} \op{\mu}{\nu}$ we define the $d^2$-component vector $\oket{X} \equiv \sum_{\mu,\nu} X_{\mu\nu} \ket{\mu\nu}$; $\obra{X}$ is the conjugate transpose of $\oket{X}$, such that $\obk{X|Y}=\tr\p{X^\dag Y}$; and $\delta_{kk'}\equiv 1$ if $k=k'$ and $0$ otherwise.
These properties of the transition operators allow us to expand any density operator $\rho$ in the transition operator basis as
\begin{align}
  \rho = \sum_{\ell=0}^{d-1} \sum_{m=-\ell}^\ell
  \rho_{\ell m} T_{\ell m},
  &&
  \rho_{\ell m} \equiv \bk{T_{\ell m}^\dag}_\rho,
  \label{eq:trans_state}
\end{align}
where $\bk{X}_\rho\equiv\tr\p{\rho X}=\obk{\rho|X}$, and $\rho^\dag=\rho$ implies that $\rho_{\ell m}^*=\p{-1}^m\rho_{\ell,-m}$.
The transition operators can be physically interpreted as the transition $\ket\psi\to T_{\ell m}\ket\psi$ induced on the state $\ket\psi$ of a spin-$s$ system by the absorption of a spin-$\ell$ particle with spin projection $m$ onto a fixed quantization axis.
In analogy with the spherical harmonics $Y_{\ell m}$, we will refer to the index $\ell$ as the {\it degree} and $m$ as the {\it order} of $T_{\ell m}$.

The transition operators are spherical tensor operators, whose degree is preserved under 3D rotations generated by the spin operators $S_\x,S_\y,S_\z$.
Moreover, the degree-$\ell$ transition operators $T_{\ell m}$ transform identically to spin-$\ell$ particles under 3D rotations (see Appendix \ref{sec:rotations}).
For any triplet of Euler angles $\v\omega=\p{\alpha_{\v\omega},\beta_{\v\omega},\gamma_{\v\omega}}$, we can therefore define the rotation operator
\begin{align}
  R\p{\v\omega} \equiv e^{-\i\alpha_{\v\omega}S_\z} e^{-\i\beta_{\v\omega}S_\y} e^{-\i\gamma_{\v\omega}S_\z},
\end{align}
and expand rotations of the transition operators as
\begin{align}
  T_{\v\omega\ell m} \equiv
  R\p{\v\omega} T_{\ell m} R\p{\v\omega}^\dag
  = \sum_{n=-\ell}^\ell D_{nm}^\ell\p{\v\omega} T_{\ell n},
  \label{eq:trans_rot}
\end{align}
where
\begin{align}
  D_{mn}^\ell\p{\v\omega} \equiv \bk{\ell m|R\p{\v\omega}|\ell n}
\end{align}
are (Wigner) rotation matrix elements.
Throughout this work, we will primarily consider rotations determined by points on the sphere.
For ease of notation, we will therefore implicitly associate Euler angle doublets $\v v=\p{\alpha_{\v v},\beta_{\v v}}$ with the triplet $\p{\alpha_{\v v},\beta_{\v v},0}$, so that for example $R\p{\v v} = R\p{\alpha_{\v v},\beta_{\v v},0}$ and $T_{\v v\ell m} = T_{\p{\alpha_{\v v},\beta_{\v v},0},\ell m}$.

%%%%%%%%%%%%%%%%%%%%%%%%%%%%%%%%%%%%%%%%%%%%%%%%%%%%%%%%%%%%%%%%%%%%%%
\section{Spin tomography as signal recovery on the sphere}
\label{sec:signal_recovery}

[\red{todo: add figure for signal recovery problem?}]

Our goal is to reconstruct an arbitrary state $\rho$ of a spin qudit from measurements of spin projection onto different quantization axes.
We are thus nominally restricted to measuring projectors $\Pi_{\v v\mu} \equiv \op{\mu_{\v v}}$, where $\ket{\mu_{\v v}}\equiv R\p{\v v}\ket{\mu}$ is a state with spin projection $\mu$ onto a measurement axis $\v v$ determined by a point on the sphere.
For any fixed axis $\v v$, the sets $\set{\Pi_{\v v\mu}}$ and $\set{T_{\v v\ell,0}}$ are both orthonormal bases for the space of operators that are diagonal in the basis $\set{\ket{\mu_{\v v}}}$.
By a simple change of operator basis, measuring the projectors $\set{\Pi_{\v v\mu}}$ is therefore equivalent to measuring the transition operators $\set{T_{\v v\ell,0}}$, and provides data on the expectation values $\bk{T_{\v v\ell,0}}_\rho$.

Reconstructing an arbitrary density operator $\rho$ essentially requires us to find a set of axes $V=\set{\v v}$ and associated coefficients $C_{\ell mk}\p{\v v}$ that would allow us to recover any coefficient $\rho_{\ell m}$ in the expansion in Eq.~\eqref{eq:trans_state} through
\begin{align}
  \rho_{\ell m}^* = \bk{T_{\ell m}}_\rho
  = \sum_{\v v,k} C_{\ell mk}\p{\v v} \bk{T_{\v vk,0}}_\rho.
  \label{eq:state_recon}
\end{align}
Expanding the rotated transition operators $T_{\v vk,0}$ into a sum of un-rotated transition operators, according to Eq.~\eqref{eq:trans_rot}, we find that the recovery condition in Eq.~\eqref{eq:state_recon} is satisfied when
\begin{align}
  T_{\ell m}
  = \sum_{\v v,k,n} C_{\ell mk}\p{\v v}
  D^k_{n,0}\p{\v v} T_{kn}.
\end{align}
Orthogonality of the transition operators motivates the ansatz $C_{\ell mk}\p{\v v}=\delta_{k\ell}C_{\ell m}\p{\v v}$, which in turn implies that
\begin{align}
  \sum_{\v v} C_{\ell m}\p{\v v} D^\ell_{n,0}\p{\v v} = \delta_{mn}
  \label{eq:tomo_recovery}
\end{align}
for all $\ell$.

In fact, the problem of finding suitable axes $V$ and coefficients $C_{\ell m}\p{\v v}$ to satisfy Eq.~\eqref{eq:tomo_recovery}, solving which would enable the recovery of arbitrary spin qudit states, can be mapped onto the well-studied problem of signal recovery on the sphere \cite{mcewen2011novel, rauhut2011sparse, alem2012sparse, khalid2014optimaldimensionality}.
The signal recovery problem can be stated as follows: given a square-integrable function $f$ on the sphere, with the harmonic expansion
\begin{align}
  f\p{\v v} = \sum_{\ell,m} f_{\ell m} Y_{\ell m}\p{\v v},
\end{align}
where $f_{\ell m}$ are expansion coefficients and $Y_{\ell m}$ is the spherical harmonic of degree $\ell$ and order $m$, find a set of points $V=\set{\v v}$ and associated coefficients $\tilde C_{\ell m}\p{\v v}$ with which we can reconstruct $f$, or equivalently its coefficients $f_{\ell m}$, from knowledge of the function's value $f\p{\v v}$ at all points $\v v\in V$; that is
\begin{align}
  f_{\ell m} = \sum_{\v v} \tilde C_{\ell m}\p{\v v} f\p{\v v}
  = \sum_{\v v,k,n} \tilde C_{\ell m}\p{\v v} Y_{kn}\p{\v v} f_{kn}.
\end{align}
Reconstruction of functions with arbitrary coefficients $f_{\ell m}$ implies that
\begin{align}
  \sum_{\v v} \tilde C_{\ell m}\p{\v v} Y_{kn}\p{\v v}
  = \delta_{k\ell} \delta_{mn},
  \label{eq:full_recovery}
\end{align}
which is a stronger version of the condition that we found for the spin qudit tomography problem in Eq.~\eqref{eq:tomo_recovery}.
We will refer to Eq.~\eqref{eq:full_recovery} as the {\it full recovery problem}, and Eq.~\eqref{eq:tomo_recovery} as the the {\it reduced recovery problem}.
Due to the fact that $D^k_{n,0}\p{\v v} = \sqrt{\frac{4\pi}{2\ell+1}}\, Y_{kn}\p{\v v}^*$, any solution to the full recovery problem automatically solves the reduced recovery problem by setting $C_{\ell m}\p{\v v} = \sqrt{\frac{2\ell+1}{4\pi}}\, \tilde C_{\ell m}\p{\v v}^*$.
In principle, this mapping allows us to import a host of existing signal recovery algorithms \cite{mcewen2011novel, rauhut2011sparse, alem2012sparse, khalid2014optimaldimensionality} for the task of spin qudit tomography.
In practice, spin qudits typically have only a modest dimension $d$, which allows for simpler and optimized tomography protocols that are practical despite worse scaling with $d$ (see Section \ref{sec:protocol}).

If the function $f$ is {\it band-limited} at degree $L$, which is to say that $f_{\ell m}=0$ for all $\ell\ge L$, then the full recovery problem in Eq.~\eqref{eq:full_recovery} is provably solvable with a suitable choice of $\abs{V}=L^2$ points on the sphere \cite{freeden2008spherical, freeden2018spherical}.
The existence of these solutions to the full recovery problem in turn implies the existence of $d^2$ measurement axes that suffice to reconstruct arbitrary states of $d$-level spin qudit, whose possible states are essentially ``band-limited'' at degree $d$ \footnote{Spin qudit states can be faithfully represented by quasi-probability distributions on a sphere.  For a $d$-level spin, these quasi-probability distributions are band-limited at degree $d$.}.
Moreover, for any fixed degree $\ell$, finding solutions to the reduced recovery problem in Eq.~\eqref{eq:tomo_recovery} is equivalent to the recovery of a degree-$\ell$ function $f = \sum_m f_m Y_{\ell m}$, which is provably possible with $\abs{V}=2\ell+1$ samples \cite{freeden2008spherical}.
In the case of spin qudit tomography, the degree $\ell$ takes a maximal value of $\ell_{\t{max}}\equiv d-1$, so state recovery requires at least as many measurement axes as there are transition operators with degree $\ell_{\t{max}}$, namely $2\ell_{\t{max}}+1=2d-1$.

Since
\begin{enumerate*}
\item there are only $2\ell+1\le2d-1$ transition operators of degree $\ell\le\ell_{\t{max}}$, and
\item measuring spin projection along any single axis provides measurement data for transition operators of {\it all} degrees $\ell\le\ell_{\t{max}}$,
\end{enumerate*}
one might expect that $2d-1$ suitably chosen measurement axes would always suffice to perform full tomography of a $d$-level spin qudit.
In the language of classical signal recovery, we would expect there to exist a fixed set of $2\ell_{\t{max}}+1$ points on the sphere that enable the recovery of {\it any} function with fixed degree $\ell\le\ell_{\t{max}}$ (given prior knowledge of $\ell$) \footnote{We could further conjecture that if {\it any} set of points $V\subset\Omega_2$ enables the recovery of degree-$\ell_{\t{max}}$ functions on the sphere, then $V$ enables the recovery degree-$\ell$ functions with $\ell\le\ell_{\t{max}}$.  However, this condition is stronger than necessary to guarantee full recovery of a $d$-level spin qudit state with $2d-1$ measurement axes, which only requires the existence of {\it some} such set of points.}.
Unfortunately, we have no general proof of this conjecture, which requires a deeper analysis of fixed-degree signal recovery on the sphere.
Nonetheless, in the following section we numerically verify that $2d-1$ measurement axes suffice for all $d\le200$ [\red{todo: refer to Section \ref{sec:axes} instead}].

%%%%%%%%%%%%%%%%%%%%%%%%%%%%%%%%%%%%%%%%%%%%%%%%%%%%%%%%%%%%%%%%%%%%%%
\section{Tomographic reconstruction error}
\label{sec:bound}

For the practically minded, the mere existence of solutions to a problem is less interesting than the exposition of a particular solution.
On a high level, a spin qudit tomography protocol consists of
\begin{enumerate*}
\item selecting a set of measurement axes,
\item collecting measurement data on spin projection onto these axes, and then
  \label{tomo:measure}
\item processing the collected data to reconstruct the state of the spin qudit.
\end{enumerate*}
Whereas step \ref{tomo:measure} can involve a host of platform-dependent technical challenges, in the following two sections we discuss the steps to take before and after collecting measurement data.

To this end, we begin by asking a question: what is a ``good'' choice of measurement axes?
Intuitively, a good choice of axes should minimize the error with which one can reconstruct an unknown quantum state from associated measurement data.
If we can quantify this intuition, then we can optimize over different choices of measurement axes to find a set that (approximately) minimizes the error in reconstructed states.

A set of measurement axes $V=\set{\v v}$ nominally induces a set of projectors $\set{\Pi_{\v v\mu}}$ that will be measured in an experiment.
By a simple change of basis, measuring these projectors is equivalent to measuring the set of transition operators $\v T_V\equiv\set{T_{\v v\ell,0}}$.
By flattening the $d\times d$ matrices $T_{\v v\ell,0}$ in $\v T_V$ into $d^2$-component column vectors $\oket{T_{\v v\ell,0}}$, we can construct the {\it measurement matrix}
\begin{align}
  M_V \equiv \sum_{\v v,\ell} \oket{T_{\v v\ell,0}} \bra{\v v\ell},
  \label{eq:meas_mat}
\end{align}
Here $\v v$ and $\ell$ label a column of $M_V$, or equivalently label a standard basis vector $\ket{\v v\ell}$ of a $\p{\abs{V}\times d}$-dimensional vector space.
A necessary and sufficient condition for $V$ to allow for full state tomography is that the measured transition operators in $\v T_V$, or equivalently the columns of $M_V$, span the entire ($d^2$-dimensional) space of operators on a $d$-level spin qudit.
In this case $M_V$ must be full rank, with $d^2$ nonzero singular values.
Indexing these singular values $M^V_k$ and the corresponding (normalized) right singular vectors $\v x^V_k\equiv(x^V_{k,1},x^V_{k,2},\cdots,x^V_{k,\abs{V}})$ by an integer $k\in\set{1,2,\cdots,d^2}$, we can construct the orthonormal qudit operators
\begin{align}
  Q^V_k \equiv \sum_j q^V_{kj} T_j,
  &&
  \v q^V_k \equiv \f{\v x^V_k}{M^V_k},
\end{align}
where for shorthand we use a combined index $j=\p{\v v,\ell}$ to specify both a measurement axis $\v v$ and a degree $\ell$, which identify the transition operator $T_j\equiv T_{\v v\ell,0}$.
These operators allow us to expand any state $\rho$ of a $d$-level spin qudit in the form
\begin{align}
  \rho = \sum_{k=1}^{d^2} \rho_k^V Q_k^V,
  &&
  \rho_k^V \equiv \bk{Q^V_k}_\rho.
\end{align}
Given empirical estimates $\tilde T_j$ of the expectation values $\bk{T_j}_\rho$, an empirical estimate $\tilde\rho_V$ of $\rho$ is then
\begin{align}
  \tilde\rho_V \equiv \sum_k \tilde\rho^V_k Q^V_k,
  \label{eq:reconstructed_state}
\end{align}
where
\begin{align}
  \tilde\rho^V_k \equiv \sum_j q^V_{kj} \tilde T_j
  \approx \sum_j q^V_{kj} \bk{T_j}_\rho
  = \bk{Q^V_k}_\rho
  = \rho^V_k.
\end{align}
The measurement matrix $M_V$ allows us to make concrete statements about the statistical error between the empirical estimate $\tilde\rho_V$ and the true state $\rho$.
Assume, for example, that the estimates $\tilde T_j$ are equal to $\bk{T_j}_\rho$ up to uncorrelated noise with variance at most $\epsilon^2$:
\begin{align}
  \tilde T_j = \bk{T_j}_\rho + \epsilon_j,
  &&
  \bbk{\epsilon_j \epsilon_{j'}} \le \epsilon^2 \delta_{jj'},
  \label{eq:error_assumption}
\end{align}
where $\set{\epsilon_j}$ are independent random variables and we use the double brackets $\bbk{\cdot}$ to denote statistical averaging over experimental trials that estimate $\bk{T_j}_\rho$.
In this case, the mean squared error with which $\tilde\rho^V_k$ approximates $\rho^V_k$ is
\begin{align}
  \Bbk{\abs{\tilde\rho^V_k-\rho^V_k}^2}
  &= \Bbk{\p{\tilde\rho^V_k-\rho^V_k}^*\p{\tilde\rho^V_k-\rho^V_k}} \\
  &= \sum_{j,j'} \p{q^V_{kj}}^* q^V_{kj'}\,
  \bbk{\epsilon_j \epsilon_{j'}} \\
  &\le \sum_j \abs{q^V_{kj}}^2 \epsilon^2
  = \p{\f{\epsilon}{M^V_k}}^2.
\end{align}
Using the fact that the operators $Q^V_k$ are orthonormal, we can therefore bound the mean squared (Euclidean) distance between $\tilde\rho_V$ and $\rho$ as
\begin{align}
  \E_V\p{\rho}^2
  \equiv \Bbk{\norm{\tilde\rho_V-\rho}^2}
  \le \epsilon^2 \S_V^2,
  \label{eq:bound_eps}
\end{align}
where $\norm{X}^2 \equiv \obk{X|X} = \tr\p{X^\dag X}$ is the squared (Frobenius, or Hilbert-Schmidt) norm of $X$, and the {\it classical error scale}
\begin{align}
  \S_V^2 \equiv \sum_k\f1{\p{M^V_k}^2} = \norm{M_V^{-1}}^2,
\end{align}
is the squared norm of the right inverse $M_V^{-1}$ of $M_V$, satisfying $M_V M_V^{-1}=1$.
We refer to this error scale $\S_V^2$ as ``classical'' because the bound in Eq.~\eqref{eq:bound_eps} applies in the presence of classical sources of measurement error.
Note that the classical error scale $\S_V^2$ diverges if the measurement matrix $M_V$ is singular, in which case measuring spin projections along all axes in $V$ does not provide sufficient information to reconstruct arbitrary quantum states.

Computing the classical error scale $\S_V^2$ to certify the bound in Eq.~\eqref{eq:bound_eps} requires constructing the measurement matrix $M_V$ and finding its singular values.
The complexity of this task can be greatly reduced by the fact that the degree $\ell$ of a transition operator $T_{\ell m}$ is preserved under rotations, which implies that the unitary
\begin{align}
  U \equiv \sum_{\ell=0}^{d-1} \sum_{m=-\ell}^\ell
  \oket{T_{\ell m}} \bra{\ell m},
  \label{eq:trans_basis}
\end{align}
with vectors $\oket{T_{\ell m}}$ in a column indexed by integers $\p{\ell,m}$, block-diagonalizes the measurement matrix into $d$ blocks indexed by the degree $\ell$:
\begin{align}
  U^\dag M_V = \sum_{\ell=0}^{d-1} \op{\ell} \otimes M_{V\ell},
\end{align}
where the $\p{2\ell+1}\times\abs{V}$-sized blocks are
\begin{align}
  M_{V\ell} \equiv \sum_{m,\v v} \ket{m} \obk{T_{\ell m}|T_{\v v\ell,0}} \bra{\v v}
  = \sum_{m,\v v} D^\ell_{m,0}\p{\v v} \op{m}{\v v}.
\end{align}
As the singular values of $M_V$ are invariant under unitary transformations, it follows that
\begin{align}
  \S_V^2 = \sum_\ell \S_{V\ell}^2,
  &&
  \S_{V\ell}^2 \equiv \norm{M_{V\ell}^{-1}}^2,
\end{align}
where $M_{V\ell}^{-1}$ is the right inverse of $M_{V\ell}$.
Constructing the block $M_{V\ell}$ and computing its singular values takes at most $O(\abs{V}d^2)$ time.
If we assume that $\abs{V}\sim d$, then computing $\S_V^2$ to certify the bound in Eq.~\eqref{eq:bound_eps} takes $O(d^4)$ serial or $O(d^3)$ parallel runtime (see Figure \ref{fig:times}).

\begin{figure}
  \centering
  \includegraphics{qudit_errors/error_times.pdf}
  \caption{Serial runtime to compute the classical error scale $\S_V^2$ with $\abs{V}=2d-1$ randomly chosen measurement axes.
    Each point is an average over 100 calculations or 5 minutes of runtime, whichever comes first.
    Dashed lines shows a fit to runtime $t=c d^\alpha$ for all $d\ge100$, finding $\alpha\approx 3.7$.
    These results do not include fixed $O(d^3)$ runtimes for pre-computing quantities that can be recycled for every new choice of measurement axes $V$.}
  \label{fig:times}
\end{figure}

The assumption that observables can be estimated up to uncorrelated noise with maximal variance $\epsilon^2$, summarized by Eq.~\eqref{eq:error_assumption}, is reasonable when measurement error is dominated by classical sources of experimental noise.
However, this assumption breaks down when measurement error is limited by fundamental quantum (shot) noise.
We relax the assumption of Eq.~\eqref{eq:error_assumption} in Appendix \ref{sec:improved_bound}, where we instead assume that $\tilde\rho_V$ is built from $n$ independent measurements of spin projection along every axis $\v v\in V$, with shot noise the dominant source of error.
In this case, the constraints that all $\bk{\Pi_{\v v\mu}}_\rho\ge0$ and $\tr\p\rho=1$ allow us to bound the mean squared distance between $\tilde\rho_V$ and $\rho$ as
\begin{align}
  \E_V\p{\rho}^2 < \f{\epsilon_V^2}{n},
  &&
  \epsilon_V^2 \equiv \sum_\ell \Gamma_\ell^2 \S_{V\ell}^2,
  \label{eq:bound}
\end{align}
where the {\it quantum error scale} $\epsilon_V^2$ is defined in terms of (half of) the spectral range of $T_{\ell,0}$:
\begin{align}
  \Gamma_\ell \equiv \f{\max_\mu t_{\ell\mu} - \min_\mu t_{\ell\mu}}{2},
  &&
  t_{\ell\mu} \equiv \bk{\mu|T_{\ell,0}|\mu}.
\end{align}
If $d$ is even, then $\Gamma_\ell=\max_\mu t_{\ell\mu}$.
For comparison with previous error bound in Eq.~\eqref{eq:bound_eps}, we note that $\epsilon_V^2<\S_V^2/2$, so the previous bound still holds with the replacement $\epsilon^2\to1/2n$.
The factors $\Gamma_\ell^2$ are quick to compute and can be recycled for every new choice of axes $V$, so the complexity of computing $\epsilon_V^2$ is identical to that of $\S_V^2$.

Though straightforward to compute, the bound in Eq.~\eqref{eq:bound} is not tight, as it is acquired by bounding the statistical variance $\epsilon_{\v v\ell}^2$ in the empirical estimate $\tilde T_{\v v\ell,0}$ of $\bk{T_{\v v\ell,0}}_\rho$ as $\bbk{\epsilon_{\v v\ell}^2}\le\Gamma_\ell^2$.
Though the individual bounds on $\bbk{\epsilon_{\v v\ell}^2}$ for each axis $\v v$ and degree $\ell$ are tight, these bounds cannot all be achieved simultaneously.
There is therefore still room for improvement on the bound in Eq.~\eqref{eq:bound} by maximizing $\E_V$ over the set of all qudit states $\rho$.
We also note that the (squared) reconstruction error bound in Eq.~\eqref{eq:bound} obeys the ``standard quantum limit'' of $\sim1/n$ scaling in the number of measurements.
In principle, this scaling can be improved to $\sim1/n^2$ by preparing and measuring entangled copies of many qudits \cite{giovannetti2006quantum}.

The quantum error scale $\epsilon_V^2$ provides a pessimistic upper bound on statistical error, which can be calculated without prior knowledge of the true qudit state $\rho$.
The actual error in the reconstruction $\tilde\rho_V$ of a particular state $\rho$ may be considerably smaller.
The exact mean squared distance between $\tilde\rho_V$ and $\rho$, derived in Appendix \ref{sec:exact}, can be written in the form
\begin{align}
  \E_V\p{\rho}^2
  = \f1n \sum_\ell \sp{\bk{\chi_{V\ell}|\rho_\ell}
    - \bk{\rho_\ell|\N_{V\ell}|\rho_\ell}},
  \label{eq:error}
\end{align}
where $\ket{\rho_\ell} \equiv \sum_m \rho_{\ell m}\ket{m}$, and $\ket{\chi_{V\ell}}$ and $\N_{V\ell}$ are defined below.
While the true error in $\tilde\rho_V$ cannot be known exactly without knowing $\rho$, this error can be estimated by computing the right-hand side of Eq.~\eqref{eq:error} with the replacement $\rho\to\tilde\rho_V$.

The matrix $\N_{V\ell}$ that appears in Eq.~\eqref{eq:error} is defined in terms of the measurement matrix $M_{V\ell}$ and its right inverse $M_{V\ell}^{-1}$ by
\begin{align}
  \N_{V\ell} \equiv M_{V\ell} \diag\sp{M_{V\ell}^{-1}
    \p{M_{V\ell}^{-1}}^\dag} M_{V\ell}^\dag,
\end{align}
where $\diag\sp{X}$ sets all off-diagonal parts of $X$ to zero.
The vector $\ket{\chi_{V\ell}}\equiv\sum_m \chi^V_{\ell m} \ket{m}$ in Eq.~\eqref{eq:error} is defined by
\begin{align}
  \chi^V_{LM} &\equiv
  \sum_\ell \obk{\N_{V\ell}|\D_M|\I g_{L\ell}}, \\
  g_{L\ell} &\equiv \sum_{m,m'}
  \obk{T_{L,m+m'}|T_{\ell m}T_{\ell m'}} \op{m}{m'}, \\
  \I &\equiv \sum_m \p{-1}^m \op{-m}{m}, \\
  \D_M &\equiv \sum_{m,m'} \delta_{M,m'-m} \op{mm'}.
\end{align}
Here $g_{L\ell}$ is a matrix of structure constants for the transition operator algebra (see Appendix \ref{sec:trans_prod}), and $\D_M$ simply picks off the $M$-th diagonal of the matrix it acts on, such that $\obk{\N_{V\ell}|\D_M|\I g_{L\ell}}$ is an inner product of the $M$-th diagonal of $\I g_{L\ell}$ with the $\p{-M}$-th diagonal of $\N_{V\ell}$.

[\red{todo: comment on code availability}]

%%%%%%%%%%%%%%%%%%%%%%%%%%%%%%%%%%%%%%%%%%%%%%%%%%%%%%%%%%%%%%%%%%%%%%
\section{Tomography protocols}
\label{sec:protocol}

The ability to certify a statistical error bound on the empirical estimate $\tilde\rho_V$ of an unknown quantum state $\rho$ motivates the following simple protocol for spin qudit tomography: select a random set of measurement axes $V$ by uniformly sampling points on the sphere \footnote{To sample a point $\p{\alpha,\beta}$ from the uniform distribution on the sphere (with azimuthal angle $\alpha$ and polar angle $\beta$), you can sample a point $\p{a,b}\in[0,1]\times[0,1]$ from the uniform distribution on the unit square, and then set $\alpha=2\pi a$ and $\beta=\arccos\p{1-2b}$.}, and optimize (using any standard minimization algorithm) the $2\abs{V}$ parameters in $V$ (two angles for each point $\v v\in V$) to minimize the quantum error scale $\epsilon_V^2$ defined in Eq.~\eqref{eq:bound}.
If $\abs{V}$ is too large for such optimization to be practical, one can instead generate many random sets of measurement axes, and then choose the set with the smallest quantum error scale $\epsilon_V^2$.
After choosing a set of measurement axes $V$, (experimentally) make $n$ measurements of spin projection along every axis $\v v\in V$, and set $\tilde\Pi_{\v v\mu}\approx\bk{\Pi_{\v v\mu}}_\rho$ to the fraction of times in which the measurement outcome was $\mu$.
The estimates $\tilde\Pi_{\v v\mu}$ of $\bk{\Pi_{\v v\mu}}_\rho$ determine estimates of the expectation values $\bk{T_{\v v\ell,0}}_\rho$, which can be used to construct an estimate $\tilde\rho_V$ of $\rho$, defined in Eq.~\eqref{eq:reconstructed_state}.
If $\tilde\rho_V$ has negative eigenvalues, its distance from $\rho$ can be reduced with maximum-likelihood corrections \cite{smolin2012efficient}.
Finally, the error in $\tilde\rho_V$ can be estimated by computing the right-hand side of Eq.~\eqref{eq:error} with the replacement $\rho_V\to\tilde\rho_V$.

Note that any information about $\rho$, obtained either from prior knowledge or preliminary measurement data, can be used to construct tailored or adaptive measurement protocols \cite{pereira2018adaptive} that are more efficient in terms of the number of measurements required to estimate $\rho$ to a fixed precision.
We leave the development of tailored and adaptive measurement protocols to future work.

As a last point, we mention that the quantum error scale $\epsilon_V^2$ can also be used to evaluate the ``quality'' of measurement axes $V$ chosen with existing methods for sparse signal recovery on the sphere \cite{rauhut2011sparse, alem2012sparse}.
In practice, we find that uniform random sampling outperforms these methods, in the sense that it generates low-error-scale sets of measurement axes with higher probability [\red{todo: add figure?}].
Two observations may explain the poor performance of these sparse signal recovery methods.
First, spin qudit tomography essentially boils down to recovering sparse signals with a particular harmonic structure, namely a fixed degree $\ell$.
The existing sparse signal recovery algorithms do not assume or account for any harmonic structure, and may therefore perform worse than expected on this particular subset of signals.
Second, spin qudit tomography requires recovering signals that are band-limited at degree $d$, but have $O(d)$ non-zero harmonic coefficients (out of $d^2$ coefficients in total).
The existing sparse signal recovery methods, on the other hand, may only perform well for signals with sparsity much smaller than $d$.
In any case, the fact that uniform sampling outperforms these methods only simplifies our randomized tomography protocol.

%%%%%%%%%%%%%%%%%%%%%%%%%%%%%%%%%%%%%%%%%%%%%%%%%%%%%%%%%%%%%%%%%%%%%%
\section{How many measurement axes?}
\label{sec:axes}

We conjecture, and numerically verified for $d\le200$, that measurements of spin projection along $2d-1$ axes suffice to reconstruct an arbitrary state of a $d$-level spin qudit.
Even so, $2d-1$ measurement axes may not be the optimal number to use.
Increasing the number of measurement axes generally decreases the quantum error scale $\epsilon_V^2$, but at the cost of having to estimate more observables.
At a fixed total number of measurements, increasing the number of measurement axes $\abs{V}$ reduces the number of measurements $n$ devoted to each axis $\v v\in V$.
This trade-off begs the question: how many measurement axes should be chosen to perform spin qudit tomography?

The reconstruction error bound in Eq.~\eqref{eq:bound} nominally provides a straightforward answer: at a fixed total number of measurements, $N=n\abs{V}$, the number of measurement axes should be chosen to minimize $\epsilon_V^2/n\propto\epsilon_V^2\abs{V}$.



\bibliography{\jobname.bib}


\onecolumngrid
\appendix

%%%%%%%%%%%%%%%%%%%%%%%%%%%%%%%%%%%%%%%%%%%%%%%%%%%%%%%%%%%%%%%%%%%%%%
\section{Rotating transition operators}
\label{sec:rotations}

Denoting the state of a spin-$s$ particle spin spin projection $\mu$ onto a quantization axis by $\ket{s\mu}$, we define
\begin{align}
  S_\z \equiv \sum_{\mu=-s}^s \mu \op{s\mu},
  &&
  S_\pm \equiv \sum_{\mu=-s}^s
  \sqrt{s\p{s+1}-\mu\p{\mu\pm1}} \op{s,\mu\pm1}{s\mu},
  \label{eq:spin_ops}
\end{align}
as well as
\begin{align}
  S_\x \equiv \f12\p{S_+ + S_-},
  &&
  S_\y \equiv -\f\i2\p{S_+-S_-},
  &&
  \v S \equiv \p{S_\x,S_\y,S_\z}.
\end{align}
The spin vector $\v S$ generates rotations of a spin-$s$ system in 3D space.
Specifically, the operator $e^{-\i\theta\v S\cdot\uv n}$ rotates a spin-$s$ system by an angle $\theta$ about the unit vector $\uv n$.

Observing that $S_\z=T_{1,0}$ and $S_\pm\propto T_{1,\pm1}$, we can use the product operator expansion of the transition operators (see Appendix \ref{sec:trans_prod}), properties of Clebsch-Gordan coefficients, properties of Wigner $6$-$j$ symbols, and a computer algebra system to simplify the commutators
\begin{align}
  \sp{S_\z,T_{\ell m}} = m\, T_{\ell m},
  &&
  \sp{S_\pm,T_{\ell m}} = \sqrt{\ell\p{\ell+1}-m\p{m\pm 1}}\, T_{\ell,m\pm1},
  \label{eq:spin_trans}
\end{align}
which implies that $T_{\ell m}$ is a spherical tensor operator, whose degree degree $\ell$ is preserved under rotations generated by $\v S$.
Moreover, by comparing Eqs.~\eqref{eq:spin_ops} and \eqref{eq:spin_trans} we see that the transition operators $T_{\ell m}$ transform identically to spin-$\ell$ particles under the adjoint action of the spin operators $S_\z$ and $S_\pm$.
For any triplet of Euler angles $\v\omega=\p{\alpha_{\v\omega},\beta_{\v\omega},\gamma_{\v\omega}}$, we can therefore define the fundamental rotation operator
\begin{align}
  R\p{\v\omega} \equiv e^{-\i\alpha_{\v\omega}S_\z} e^{-\i\beta_{\v\omega}S_\y} e^{-\i\gamma_{\v\omega}S_\z},
\end{align}
and expand
\begin{align}
  T_{\v\omega\ell m}
  \equiv R\p{\v\omega} T_{\ell m} R\p{\v\omega}^\dag
  = \sum_{n=-\ell}^\ell D_{nm}^\ell\p{\v\omega} T_{\ell n},
\end{align}
where
\begin{align}
  D_{mn}^\ell\p{\v\omega} \equiv \bk{\ell m|R\p{\v\omega}|\ell n}
  = \obk{T_{\ell m}|R\p{\v\omega}\otimes R\p{\v\omega}^*|T_{\ell n}}
\end{align}
are matrix elements of the rotation operator $R\p{\v\omega}$ for spin-$\ell$ particles.

%%%%%%%%%%%%%%%%%%%%%%%%%%%%%%%%%%%%%%%%%%%%%%%%%%%%%%%%%%%%%%%%%%%%%%
\section{An improved reconstruction error bound}
\label{sec:improved_bound}

In Section \ref{sec:bound} of the main text, we provided a reconstruction error bound using the assumption of Eq.~\eqref{eq:error_assumption}, namely that expectation values derived from spin projection measurements can be estimated up to uncorrelated errors with maximal variance $\epsilon^2$.
This assumption is reasonable if measurement error is dominated by experimental sources of noise, and it yields a simple derivation of the reconstruction bound in Eq.~\eqref{eq:bound_eps}.
Nonetheless, there are two problems with the assumption of Eq.~\eqref{eq:error_assumption}:
\begin{enumerate*}
\item there is no a priori guarantee for the value of $\epsilon$, which must be inferred from experimental outcomes, and
\item the assumption that all errors are uncorrelated is unjustified (and generally false).
\end{enumerate*}
Here, we relax the assumption of Eq.~\eqref{eq:error_assumption} and derive an explicit error bound in terms of the qudit dimension $d$ and the number of spin projection measurements made along every measurement axis.

To this end, we fix a particular set of measurement axes $V$, and consider performing $n$ measurements of spin projection along every axis $\v v\in V$, for a total of $N=\abs{V}\times n$ measurements.
Such a procedure is equivalent to making $N$ local measurements of the $N$-fold product state $\rho^{\otimes N}$.
For convenience, we index the tensor factors of $\rho^{\otimes N}$ by the integers $\p{i,j}$, with $i\in\set{1,2,\cdots,\abs{V}}$ specifying a measurement axis $\v v_i\in V$, and $j\in\set{1,2,\cdots,n}$ specifying the copy of $\rho$ prepared for the $j$-th measurement spin projection along a particular axis.
We then define the projectors $\Pi_{i\mu}\equiv\op{\mu_{\v v_i}}$ onto states $\ket{\mu_{\v v_i}}$ with definite spin projection $\mu$ along axis $\v v_i\in V$, and define $\Pi_{i\mu}^j$ to be an $N$-qudit operator with $\Pi_{i\mu}$ on the $\p{i,j}$-th tensor factor and the identity elsewhere.
We denote the experimental outcome of measuring $\Pi_{i\mu}$ in the $\p{i,j}$-th copy of $\rho$ by $\tilde \Pi_{i\mu}^j\in\set{0,1}$.
In other words, $\tilde \Pi_{i\mu}^j$ is the ``single-shot estimate'' of $\Pi_{i\mu}$, with $\tilde \Pi_{i\mu}^j=1$ if outcome $\mu$ was observed on the $\p{i,j}$-th experimental trial, and $\tilde \Pi_{i\mu}^j=0$ otherwise.
An empirical estimate of the expectation value $\bk{\Pi_{i\mu}}_\rho$ is provided by the fraction of times that outcome $\mu$ was observed when measuring spin projection along axis $\v v_i$, that is
\begin{align}
  \tilde \Pi_{i\mu} \equiv \f1n \sum_{j=1}^n \tilde \Pi_{i\mu}^j
  \approx \f1n \sum_{j=1}^n \tr\p{\rho^{\otimes N} \Pi_{i\mu}^j}
  = \tr\p{\rho \Pi_{i\mu}}.
\end{align}
For reasons that will become clear shortly, it will be useful to think of $\tilde \Pi_{i\mu}$ as an empirical estimate of $\bk{\bar \Pi_{i\mu}}_{\rho^{\otimes N}}$, where
\begin{align}
  \bar \Pi_{i\mu} \equiv \f1n \sum_{j=1}^n \Pi_{i\mu}^j
  \label{eq:mean_proj}
\end{align}
is the average of $\Pi_{i\mu}$ applied to all copies of $\rho$ for which spin projection is measured along the axis $\v v_i$.

%%%%%%%%%%%%%%%%%%%%%%%%%%%%%%%%%%%%%%%%%%%%%%%%%%
\subsection{Covariance of errors in the spin-projection basis}

Finite sampling error (i.e.~shot noise) generally induces statistical error $\epsilon_\O$ into the empirical estimate $\tilde\O$ of an observable $\O$:
\begin{align}
  \epsilon_\O \equiv \tilde\O - \bk{\O},
\end{align}
where the single brackets $\bk{\cdot}$ denote an expectation value with respect to the measured quantum state.
On average, this statistical error will be zero, which is to say that
\begin{align}
  \bbk{\epsilon_\O} = \Bbk{\tilde\O - \bk{\O}}
  = \bk{\O - \bk{\O}}
  = 0,
\end{align}
where the double brackets $\bbk{\cdot}$ to denote statistical averaging over experimental trials.
However, the covariance between statistical errors $\epsilon_\O$ and $\epsilon_\Q$ on the empirical estimates $\tilde\O$ and $\tilde\Q$ of observables $\O$ and $\Q$ is
\begin{align}
  \bbk{\epsilon_\O \epsilon_\Q}
  = \Bbk{\p{\tilde\O - \bk{\O}} \p{\tilde\Q - \bk{\Q}}}
  = \bk{\p{\O - \bk{\O}} \p{\Q - \bk{\Q}}}
  = \bk{\O\Q} - \bk{\O} \bk{\Q}.
\end{align}
In the context of spin qudit tomography, we can therefore define the statistical error
\begin{align}
  \epsilon_{i\mu}
  \equiv \tilde \Pi_{i\mu} - \bk{\Pi_{i\mu}}_{\rho}
  = \tilde \Pi_{i\mu} - \bk{\bar\Pi_{i\mu}}_{\rho^{\otimes N}}
\end{align}
in the empirical estimate of $\bk{\Pi_{i\mu}}_{\rho}$ or $\bk{\bar\Pi_{i\mu}}_{\rho^{\otimes N}}$, and use Eq.~\eqref{eq:mean_proj} to expand
\begin{align}
  \bbk{\epsilon_{i\mu} \epsilon_{i'\mu'}}
  = \bk{\bar\Pi_{i\mu} \bar\Pi_{i'\mu'}}_{\rho^{\otimes N}}
  - \bk{\bar\Pi_{i\mu}}_{\rho^{\otimes N}}
  \bk{\bar\Pi_{i'\mu'}}_{\rho^{\otimes N}}
  = \f1{n^2} \sum_{j,j'=1}^n
  \sp{\bk{\Pi_{i\mu}^j \Pi_{i'\mu'}^{j'}}_{\rho^{\otimes N}}
    - \bk{\Pi_{i\mu}^j}_{\rho^{\otimes N}}
    \bk{\Pi_{i'\mu'}^{j'}}_{\rho^{\otimes N}}}.
  \label{eq:proj_cov_start}
\end{align}
If $\p{i,j}\ne\p{i',j'}$, then $\Pi_{i\mu}^j$ and $\Pi_{i'\mu'}^{j'}$ address different tensor factors of the product state $\rho^{\otimes N}$, so the expectation value of their product factorizes due to the identity $\tr\sp{\p{A\otimes B}\p{A'\otimes B'}} = \tr\p{AA'}\times\tr\p{BB'}$.
This factorization can also be seen as a consequence of the fact that if $\p{i,j}\ne\p{i',j'}$, then $\Pi_{i\mu}^j$ and $\Pi_{i'\mu'}^{j'}$ are ``spatially separated'' on $\rho^{\otimes N}$, which means that their expectation values cannot have quantum correlations.
The terms in Eq.~\eqref{eq:proj_cov_start} with $\p{i,j}\ne\p{i',j'}$ therefore vanish, so
\begin{align}
  \bbk{\epsilon_{i\mu} \epsilon_{i'\mu'}}
  &= \delta_{ii'} \times \f1{n^2} \sum_{j=1}^n \sp{\bk{\Pi_{i\mu}^j \Pi_{i\mu'}^j}_{\rho^{\otimes N}}
    - \bk{\Pi_{i\mu}^j}_{\rho^{\otimes N}}
    \bk{\Pi_{i\mu'}^j}_{\rho^{\otimes N}}} \\
  &= \delta_{ii'} \times \f1n \sp{\bk{\Pi_{i\mu} \Pi_{i\mu'}}_\rho
    - \bk{\Pi_{i\mu}}_\rho \bk{\Pi_{i\mu'}}_\rho} \\
  &= \delta_{ii'} \times \f1n \cov_\rho\p{\Pi_{i\mu}, \Pi_{i\mu'}},
\end{align}
where $\cov_\rho\p{X,Y} \equiv \bk{XY}_\rho - \bk{X}_\rho \bk{Y}_\rho$.

%%%%%%%%%%%%%%%%%%%%%%%%%%%%%%%%%%%%%%%%%%%%%%%%%%
\subsection{Variance of errors in the transition operator basis}

Rather than the statistical errors $\epsilon_{i\mu} \equiv \tilde\Pi_{i\mu} - \bk{\Pi_{i\mu}}_\rho$ in the estimates $\tilde\Pi_{i\mu}$ of the projectors $\Pi_{i\mu}$, we now consider the statistical errors $\epsilon_{i\ell} \equiv \tilde T_{i\ell} - \bk{T_{i\ell}}_\rho$ in the estimates $\tilde T_{i\ell}$ of the transition operators $T_{i\ell} \equiv T_{\v v_i\ell,0}$.
We can expand the transition operators $T_{i\ell}$ as a sum over projectors $\Pi_{i\mu}$ as
\begin{align}
  T_{i\ell} = \sum_\mu t_{\ell\mu} \Pi_{i\mu},
  &&
  t_{\ell\mu} \equiv \bk{\mu|T_{\ell,0}|\mu}
  = \sqrt{\f{2\ell+1}{d}} \bk{s\mu;\ell,0|s\mu},
\end{align}
and likewise $\tilde T_{i\ell} = \sum_\mu t_{\ell\mu} \tilde\Pi_{i\mu}$.
The covariance between errors in the transition operator basis is then
\begin{align}
  \bbk{\epsilon_{i\ell} \epsilon_{i'\ell'}}
  = \sum_{\mu,\mu'} t_{\ell\mu} t_{\ell'\mu'}
  \bbk{\epsilon_{i\mu} \epsilon_{i'\mu'}}
  = \delta_{ii'} \times \f1n
  \sum_{\mu,\mu'} t_{\ell\mu} t_{\ell'\mu'}
  \cov_\rho\p{\Pi_{i\mu}, \Pi_{i\mu'}}
  = \delta_{ii'} \times \f1n \cov_\rho\p{T_{i\ell}, T_{i\ell'}},
\end{align}
where we used the fact that the covariance $\cov_\rho\p{X,Y}$ is linear in both $X$ and $Y$.
Due to the appearance of $\delta_{ii'}$ above and the orthogonality of rotated transition operators $T_{i\ell}$ and $T_{i'\ell'}$ with degrees $\ell\ne\ell'$, only the variance $\bbk{\epsilon_{i\ell}^2}$ ultimately contributes to reconstruction error.
We therefore seek to find an upper bound on $\bbk{\epsilon_{i\ell}^2}$.

To this end, we define the probability $p^i_\mu\equiv\bk{\Pi_{i\mu}}_\rho$, collect these probabilities into the classical probability distribution $p^i = \p{p^i_{-s}, p^i_{-s+1}, \cdots, p^i_s}$, and define the vector $t_\ell \equiv \p{t_{\ell,-s},t_{\ell,-s+1},\cdots,t_{\ell s}}$ of all $t_{\ell\mu}\equiv\bk{\mu|T_{\ell,0}|\mu}$.
We then observe that
\begin{align}
  \bbk{\epsilon_{i\ell}^2} = \f1n \times \sigma_{p^i}^2\p{t_\ell},
  &&
  \sigma_p^2\p{X}
  \equiv \sum_\mu p_\mu X_\mu^2 - \p{\sum_\mu p_\mu X_\mu}^2,
\end{align}
where $\sigma_p^2\p{X}$ is the weighted variance of $X$.
This variance is maximal when $p$ has equal weight on the largest and smallest values of $X$, which implies that
\begin{align}
  \sigma_p^2\p{t_\ell} \le \Gamma_\ell^2,
  &&
  \Gamma_\ell \equiv \f{\max_\mu t_{\ell\mu} - \min_\mu t_{\ell\mu}}{2},
  &&
  \t{so}
  &&
  \bbk{\epsilon_{i\ell}^2} \le \f1n \times \Gamma_\ell^2.
\end{align}
Note that this bound on $\bbk{\epsilon_{i\ell}^2}$ is tight, as equality is achieved by the state
\begin{align}
  \rho_i^\star = \f12\p{\Pi_{i\mu_{\t{max}}} + \Pi_{i\mu_{\t{min}}}}.
\end{align}
To find a nicer analytical bound on $\bbk{\epsilon_{i\ell}^2}$, we can use normalization of the transition operators, $\obk{T_{i\ell}|T_{i\ell}}=\sum_\mu t_{\ell\mu}^2 = 1$, and the fact that all probabilities $p_\mu\le1$ to bound
\begin{align}
  \sigma_p^2\p{t_\ell} \le \sum_\mu p_\mu t_{\ell\mu}^2
  \le \sum_\mu t_{\ell\mu}^2 = 1,
  &&
  \t{so}
  &&
  \bbk{\epsilon_{i\ell}^2} < \f1n.
\end{align}
We can get a tighter analytical bound on $\bbk{\epsilon_{i\ell}^2}$ by considering the fact that $t_{\ell\mu}^2 = t_{\ell,-\mu}^2$ due to the symmetries of the Clebsch-Gordan coefficients.
Denoting the index that maximizes $t_{\ell\mu}^2$ by $\mu_{\t{max}}$, it follows that if $\mu_{\t{max}}\ne0$ then
\begin{align}
  \sigma_p^2\p{t_\ell} \le \sum_\mu p_\mu t_{\ell\mu}^2
  \le t_{\ell\mu_{\t{max}}}^2
  = \f12 \p{t_{\ell\mu_{\t{max}}}^2 + t_{\ell,-\mu_{\t{max}}}^2}
  \stackrel{\mu_{\t{max}}\ne0}{\le} \f12 \sum_\mu t_{\ell\mu}^2
  = \f12.
\end{align}
If $\mu_{\t{max}}=0$, then similarly
\begin{align}
  t_{\ell\mu_{\t{max}}}^2 + 2 t_{\ell\mu_{\t{min}}}^2
  = t_{\ell\mu_{\t{max}}}^2 + t_{\ell\mu_{\t{min}}}^2
  + t_{\ell,-\mu_{\t{min}}}^2
  \stackrel{\mu_{\t{max}}=0}{\le}
  \sum_\mu t_{\ell\mu}^2 = 1,
  &&
  \t{so}
  &&
  \abs{t_{\ell\mu_{\t{min}}}}
  \stackrel{\mu_{\t{max}}=0}{\le}
  \sqrt{\f{1 - t_{\ell\mu_{\t{max}}}^2}{2}},
\end{align}
where $\mu_{\t{min}}$ is the index that minimizes $t_{\ell\mu}^2$.
This inequality lets us bound
\begin{align}
  \Gamma_\ell
  = \f12 \p{t_{\ell\mu_{\t{max}}} - t_{\ell\mu_{\t{min}}}}
  \le \f12 \p{t_{\ell\mu_{\t{max}}} + \abs{t_{\ell\mu_{\t{min}}}}}
  \stackrel{\mu_{\t{max}}=0}{\le}
  \f12 t_{\ell\mu_{\t{max}}}
  + \f12\sqrt{\f{1 - t_{\ell\mu_{\t{max}}}^2}{2}}
  \equiv \lambda\p{t_{\ell\mu_{\t{max}}}}.
\end{align}
If is straightforward to show that $\lambda\p{x}$ is maximally $\lambda^\star\equiv \max_x \lambda\p{x} = \sqrt{3/8}$, so
\begin{align}
  \Gamma_\ell^2
  \stackrel{\mu_{\t{max}}=0}{\le} \p{\lambda^\star}^2
  = \f38 < \f12.
\end{align}
Altogether, we thus find that in all cases
\begin{align}
  \sigma_p^2\p{t_\ell} \le \f12,
  &&
  \t{so}
  &&
  \bbk{\epsilon_{i\ell}} \le \f1{2n}.
\end{align}

%%%%%%%%%%%%%%%%%%%%%%%%%%%%%%%%%%%%%%%%%%%%%%%%%%
\subsection{Revisiting the reconstruction error bound}
\label{sec:revisiting}

We now revisit the derivation of reconstruction error in Section \ref{sec:bound} to make use of the bounds on variances $\bbk{\epsilon_{i\ell}^2}$.
To recap, for a set of measurement axes $V=\set{\v v}$ and degrees $\ell\in\set{0,1,\cdots,d-1}$ we construct the measurement matrix
\begin{align}
  M_V \equiv \sum_{\v v,\ell} \oket{T_{\v v\ell,0}} \bra{\v v\ell},
\end{align}
which can be block diagonalized as
\begin{align}
  U^\dag M_V = \sum_\ell \op{\ell} \otimes M_{V\ell},
  &&
  U \equiv \sum_{\ell,m} \oket{T_{\ell m}} \bra{\ell m},
  &&
  M_{V\ell} = \sum_{m,\v v} D^\ell_{m,0}\p{\v v} \op{m}{\v v},
\end{align}
where $D^\ell_{mn}\p{\v v}\equiv\bk{\ell m|R\p{\v v}|\ell n}$ is a (Wigner) rotation matrix element for a spin-$\ell$ particle.
The block-diagonal structure of $M_V$ allows us to index its singular values $M^V_{\ell m}$ and corresponding (normalized) right singular vectors $\v x^V_{\ell m} = (x^V_{\ell m,1}, x^V_{\ell m,2}, \cdots )$ by the indices $\p{\ell,m}$, where the integer $\abs{m}\le\ell$.
These singular vectors and values define the orthonormal operators
\begin{align}
  Q^V_{\ell m} \equiv \sum_i q^V_{\ell mi} T_{i\ell},
  &&
  \v q^V_{\ell m} \equiv \f{\v x^V_{\ell m}}{M^V_{\ell m}},
\end{align}
where $i\in\set{1,2,\cdots,\abs{V}}$ indexes an axis $\v v_i\in V$.
The state $\rho$ can be expanded in the basis of these operators as
\begin{align}
  \rho = \sum_{\ell,m} \bk{Q^V_{\ell m}}_\rho Q^V_{\ell m},
\end{align}
and the estimates $\tilde T_{i\ell}$ of $\bk{T_{i\ell}}_\rho$ can be used to construct the following estimate $\tilde\rho_V$ of $\rho$:
\begin{align}
  \tilde\rho_V \equiv \sum_{\ell,m}
  \sp{\sum_i q^V_{\ell mi} \tilde T_{i\ell}} Q^V_{\ell m}
  \approx \sum_{\ell,m}
  \sp{\sum_i q^V_{\ell mi} \bk{T_{i\ell}}_\rho} Q^V_{\ell m}
  = \sum_{\ell,m} \bk{Q^V_{\ell m}}_\rho Q^V_{\ell m}
  = \rho.
\end{align}
Recalling that $\epsilon_{i\ell} \equiv \tilde T_{i\ell} - \bk{T_{i\ell}}_\rho$, we can use orthonormality of all $Q^V_{\ell m}$ to expand the mean squared distance between $\tilde\rho_V$ and $\rho$ as
\begin{align}
  \E_V\p{\rho}^2 \equiv \Bbk{\norm{\tilde\rho_V-\rho}^2}
  = \sum_{\ell,m,i,i'} \p{q^V_{\ell mi}}^* q^V_{\ell mi'} \bbk{\epsilon_{i\ell} \epsilon_{i'\ell}}
  = \sum_{\ell,m,i} \abs{q^V_{\ell mi}}^2 \bbk{\epsilon_{i\ell}^2}
  < \f1n \sum_\ell \Gamma_\ell^2 \S_{V\ell}^2,
  \label{eq:bound_apndx}
\end{align}
where we used the fact that $\bbk{\epsilon_{i\ell}^2}\le\Gamma_\ell^2/n$, and
\begin{align}
  \sum_{m,i} \abs{q^V_{\ell mi}}^2
  = \sum_m \f1{\p{M^V_{\ell m}}^2}
  = \norm{M_{V\ell}^{-1}}
  = \S_{V\ell}^2.
\end{align}
The fact that $\bbk{\epsilon_{i\ell}^2}<1/2n$ also implies that
\begin{align}
  \E_V\p{\rho}^2 < \f1{2n} \sum_\ell \S_{V\ell}^2 = \f{\S_V^2}{2n}.
\end{align}
Note that the bound in Eq.~\eqref{eq:bound_apndx} is not tight, as the individual bounds on the variances $\bbk{\epsilon_{i\ell}^2}$ cannot all be achieved simultaneously.
There is therefore still room for improvement on the bound in Eq.~\eqref{eq:bound} by maximizing $\E_V$ over the set of all qudit states $\rho$.

%%%%%%%%%%%%%%%%%%%%%%%%%%%%%%%%%%%%%%%%%%%%%%%%%%%%%%%%%%%%%%%%%%%%%%
\section{Exact reconstruction error}
\label{sec:exact}

Here we find exact expressions for reconstruction error, which can be used to estimate the error in a given reconstruction $\tilde\rho_V$ of an unknown state $\rho$ after performing tomography.
To this end, we start with Eq.~\eqref{eq:bound_apndx} from Appendix \ref{sec:revisiting} to write
\begin{align}
  \E_V\p{\rho}^2
  = \sum_{\ell,m,i,i'} \p{q^V_{\ell mi}}^* q^V_{\ell mi'} \bbk{\epsilon_{i\ell}\epsilon_{i'\ell}}
  = \f1n \sum_{\ell,m,i} \abs{\tilde{\v q}_{\ell i}}^2 \cov_\rho\p{T_{i\ell},T_{i\ell}},
  &&
  \abs{\tilde{\v q}_{\ell i}}^2 = \sum_m \abs{q_{\ell mi}}^2,
  \label{eq:error_SM_start}
\end{align}
where $\tilde{\v q}_{\ell i} = (q_{\ell,-\ell,i},q_{\ell,-\ell+1,i},\cdots q_{\ell,\ell,i})$, and we used the fact that $\bbk{\epsilon_{i\ell}\epsilon_{i'\ell}} = \delta_{ii'}\times\cov_\rho\p{T_{i\ell},T_{i\ell}}/n$.
Identifying the singular value decomposition $M_{V\ell} = U_{V\ell} \Sigma_{V\ell} W_{V\ell}^\dag$, we then we observe that vector $\tilde{\v q}_{\ell i}$ is essentially the $i$-th row of $\Sigma_{V\ell}^{-1} W_{V\ell}^\dag$, which is to say that $\tilde{\v q}_{\ell i} = \Sigma_{V\ell}^{-1} W_{V\ell}^\dag \ket{\v v_i}$.
We can therefore simplify
\begin{align}
  \abs{\tilde{\v q}_{\ell i}}^2
  = \bk{\v v_i | W_{V\ell} \Sigma_{V\ell}^{-2} W_{V\ell}^\dag | \v v_i}
  = \bk{\v v_i| M_{V\ell}^{-1} \p{M_{V\ell}^{-1}}^\dag | \v v_i}.
\end{align}
Using the fact that all $T_{i\ell}=T_{i\ell}^\dag$, we can also expand
\begin{align}
  \cov_\rho\p{T_{i\ell},T_{i\ell}}
  = \cov_\rho\p{T_{i\ell}^\dag,T_{i\ell}}
  = \sum_{m,m'} D^\ell_{m,0}\p{\v v_i}^* D^\ell_{m',0}\p{\v v_i}
  \cov_\rho\p{T_{\ell m}^\dag, T_{\ell m'}},
\end{align}
which implies that
\begin{align}
  \E_V\p{\rho}^2
  = \f1n \sum_{i,\ell,m,m'}
  D^\ell_{m',0}\p{\v v_i} \abs{\tilde{\v q}_{\ell i}}^2
  D^\ell_{m,0}\p{\v v_i}^*
  \cov_\rho\p{T_{\ell m}^\dag, T_{\ell m'}}.
\end{align}
Altogether, this reconstruction error can be expressed more compactly by defining the {\it covariance matrix}
\begin{align}
  \C_\ell\sp{\rho} \equiv
  \sum_{m,m'} \cov_\rho\p{T_{\ell m}^\dag, T_{\ell m'}} \op{m}{m'},
\end{align}
and the {\it noise matrix}
\begin{align}
  \N_{V\ell} \equiv M_{V\ell} \diag\sp{M_{V\ell}^{-1}
    \p{M_{V\ell}^{-1}}^\dag} M_{V\ell}^\dag,
\end{align}
where $\diag\sp{X}$ sets all off-diagonal entries of $X$ to zero, in terms of which
\begin{align}
  \E_V\p{\rho}^2 = \f1n \sum_\ell \obk{\N_{V\ell}|\C_\ell\sp{\rho}},
  \label{eq:error_SM_mid}
\end{align}
where $\obk{X|Y}=\tr\p{X^\dag Y}$ is a trace inner product.

The result in Eq.~\eqref{eq:error_SM_mid} essentially expresses reconstruction error as a weighted sum of the covariances $\cov_\rho\p{T_{\ell m},T_{\ell m'}}$, where the weights are given by the corresponding matrix elements of the noise matrix $\N_{V\ell}$.
This expression is perhaps the most physically meaningful form of the reconstruction error $\E_V\p{\rho}^2$ that we will consider in this work, but in practice it turns out that Eq.~\eqref{eq:error_SM_mid} is inconvenient and inefficient to evaluate for any given state $\rho$.
To find a more practical expression of reconstruction error, we use the fact that
\begin{align}
  \bk{T_{\ell m}^\dag}_\rho
  = \obk{\rho|T_{\ell m}^\dag}
  = \tr\p{\rho T_{\ell m}^\dag}
  = \tr\p{T_{\ell m}^\dag \rho}
  = \obk{T_{\ell m}|\rho},
\end{align}
to expand the covariance matrix as
\begin{align}
  \C_\ell\sp{\rho}
  &= \sum_{m,m'} \op{m}{m'} \sp{\obk{\rho|T_{\ell m}^\dag T_{\ell m'}}
    - \obk{\rho|T_{\ell m}^\dag} \obk{\rho|T_{\ell m'}}} \\
  &= \sum_{m,m'} \op{m}{m'} \sp{\obk{T_{\ell m'}^\dag T_{\ell m}|\rho}
    - \obk{T_{\ell m}|\rho} \obk{T_{\ell m'}^\dag|\rho}} \\
  &= \sum_{m,m'} \op{m}{m'}\I \sp{\obk{T_{\ell m'} T_{\ell m}|\rho}
    - \obk{T_{\ell m}|\rho} \obk{T_{\ell m'}|\rho}}
\end{align}
where we define the inversion operator $\I\equiv\sum_m \p{-1}^m\op{-m}{m}$.
We then expand the product $T_{\ell m'} T_{\ell m}$ as
\begin{align}
  \obk{T_{\ell m'} T_{\ell m}|\rho}
  = \sum_L g_{\ell m'm}^L \obk{T_{L,m'+m}|\rho},
  &&
  g_{\ell m'm}^L \equiv \obk{T_{L,m'+m} | T_{\ell m'} T_{\ell m}}
  = f_{\ell m';\ell m}^{L,m'+m},
\end{align}
where the (real) factors $f_{\ell m';\ell m}^{L,m'+m}$ are provided in Appendix \ref{sec:trans_prod}.
Substituting the covariance matrix back into Eq.~\eqref{eq:error_SM_mid} and replacing $\obk{T_{\ell m}|\rho}\to\rho_{\ell m}$, we get
\begin{align}
  \E_V\p{\rho}^2
  = \f1n \sp{\sum_{\ell,m} \p{\chi^V_{\ell m}}^* \rho_{\ell m}
  - \sum_{\ell,m,m'} \bk{m'|\I\N_{V\ell}|m}
  \rho_{\ell m} \rho_{\ell m'}},
  \label{eq:error_comps}
\end{align}
where
\begin{align}
  \chi^V_{LM}
  &\equiv \sum_{\ell,m,m'} \delta_{M,m'+m}
  \bk{m'|\I\N_{V\ell}|m}^* g^L_{\ell m'm} \\
  &= \sum_{\ell,m,m'} \delta_{M,-m'+m}
  \bk{m|\N_{V\ell}|m'} \p{-1}^{m'} g^L_{\ell,-m',m} \\
  &= \sum_\ell \obk{\N_{V\ell}|\D_M|\I g_{L\ell}}
\end{align}
can be written in terms of the matrices
\begin{align}
  g_{L\ell} \equiv \sum_{m,m'} g^L_{\ell m'm} \op{m'}{m},
  &&
  \D_M \equiv \sum_{m,m'} \delta_{M,-m'+m} \op{m'm}.
\end{align}
Here $\D_M$ simply picks off the $M$-th diagonal of the matrix it acts on, such that $\obk{\N_{V\ell}|\D_M|\I g_{L\ell}}$ is an inner product of the $M$-th diagonal of $\I g_{L\ell}$ with the $\p{-M}$-th diagonal of $\N_{V\ell}$.
Defining the $\p{2\ell+1}$-component vectors
\begin{align}
  \ket{\rho_\ell} \equiv \sum_m \rho_{\ell m} \ket{m},
  &&
  \ket{\chi_{V\ell}} \equiv \sum_{\ell,m} \chi^V_{\ell m} \ket{m},
\end{align}
we can write the expansion in Eq.~\eqref{eq:error_comps} in the vectorized form
\begin{align}
  \E_V\p{\rho}^2
  = \f1n \sum_\ell \sp{\bk{\chi_{V\ell}|\rho_\ell}
    - \bk{\rho_\ell|\N_{V\ell}|\rho_\ell}}.
  \label{eq:error_SM}
\end{align}

%%%%%%%%%%%%%%%%%%%%%%%%%%%%%%%%%%%%%%%%%%%%%%%%%%
\subsection{Comments on a tight reconstruction error bound}

In principle, maximizing the reconstruction error in Eq.~\eqref{eq:error_SM} over all qudit states $\rho$ would provide a tight upper bound on reconstruction error for any set of axes $V$.
To simplify this task somewhat, we first maximize Eq.~\eqref{eq:error_SM} over all $\rho$ with $\tr\p\rho=1$: this maximum occurs at a ``state'' $\rho_V^\star$ whose components are defined by
\begin{align}
  \ket{\rho_{V\ell}^\star} \stackrel{\ell\ne0}{\equiv}
  \f12 \N_{V\ell}^{-1} \ket{\chi_{V\ell}},
  &&
  \ket{\rho_{V,0}^\star} \equiv \f1{\sqrt{d}} \ket{0}.
\end{align}
The corresponding maximum of $\E_V$ is given by
\begin{align}
  \E_V\p{\rho_V^\star}^2 = \f1n \sum_{\ell>0}
  \sp{\f14 \bk{\chi_{V\ell}|\N_{V\ell}^{-1}|\chi_{V\ell}}
  - \f1d \tr\p{\N_{V\ell}}},
\end{align}
where the $\tr\p{\N_{V\ell}}$ terms above come from simplifying the $\ell=0$ terms of Eq.~\eqref{eq:error_SM} with $\rho\to\rho_V^\star$.
While $\E_V\p{\rho_V^\star}^2$ provides an upper bound on $\E_V\p{\rho}^2$ over all $\rho$ with $\tr\p{\rho}=1$, this bound turns out to be useless in practice, because $\rho_V^\star$ will generally be a non-physical ``state'' with negative eigenvalues.
To find tight bound on $\E_V\p{\rho}^2$ over the state of physical qudit states $\rho$, we also need to enforce the constraint that $\rho$ have no negative eigenvalues.
Equipped with $\rho_V^\star$ and $\E_V\p{\rho_V^\star}^2$, we can expand
\begin{align}
  \E_V\p{\rho}^2 = \E_V\p{\rho_V^\star}^2
  - \f1n D_V\p{\rho,\rho_V^\star}^2,
  &&
  D_V\p{X,Y}^2 \equiv
  \sum_\ell \bk{\p{X-Y}_\ell | \N_{V\ell} | \p{X-Y}_\ell},
\end{align}
where $Z_\ell\equiv\sum_m \obk{T_{\ell m}|Z} \ket{m}$ is a vector of the degree-$\ell$ components of $Z$.
Finding the maximum of $\E_V$ over all qudit states $\rho$ thus amounts to finding the closest physical qudit state $\rho$ to $\rho_V^\star$, with distance measured by the metric $D_V$ induced by the noise matrices $\N_{V\ell}$.
This minimization problem is essentially solved for the case of $\N_{V\ell}=1$ in Ref.~\cite{smolin2012efficient}, but it is not obvious how to minimize $D_V\p{\rho,\rho_V^\star}^2$ for general $\N_{V\ell}$.
We leave the general problem of minimizing $D_V\p{\rho,\rho_V^\star}$ to future work, noting that this task may be assisted by the positivity conditions derived in Ref.~\cite{kryszewski2006positivity}.
A loose upper bound, which may nonetheless be tighter than that in Eq.~\eqref{eq:bound} of the main text, might be found by the fact that physical qudit states $\rho$ satisfy $\obk{\rho|\rho}\le1$.

%%%%%%%%%%%%%%%%%%%%%%%%%%%%%%%%%%%%%%%%%%%%%%%%%%%%%%%%%%%%%%%%%%%%%%
\section{Transition operator product expansion}
\label{sec:trans_prod}

The transition operators on the $d$-dimensional Hilbert space of a spin-$s$ system (with $s\equiv\frac{d-1}{2}$) are defined by
\begin{align}
  T_{\ell m} \equiv \sqrt{\f{2\ell+1}{2s+1}} \sum_{\mu,\nu=-s}^s
  \bk{s\mu;\ell m|s\nu} \op{\nu}{\mu},
\end{align}
where $\bk{s\mu;\ell m|s\nu}$ is a Clebsh-Gordan coefficient that enforces $\ell\in\set{0,1,\cdots,2s}$ and $m\in\set{-\ell,-\ell+1,\cdots,\ell}$.
We wish to compute the coefficients of the operator product expansion
\begin{align}
  T_{\ell_1 m_1} T_{\ell_2 m_2}
  = \sum_{L,M} f_{\ell_1 m_1;\ell_2 m_2}^{LM} T_{LM},
  &&
  f_{\ell_1 m_1;\ell_2 m_2}^{LM}
  \equiv \obk{T_{LM} | T_{\ell_1 m_1} T_{\ell_2 m_2}}.
\end{align}
Using the symmetry properties of Clebsch-Gordan coefficients, namely
\begin{align}
  \bk{\ell_1 m_1; \ell_2 m_2| L M}
  &= \p{-1}^{\ell_2+m_2} \sqrt{\f{2L+1}{2\ell_1+1}}
  \bk{L,-M; \ell_2 m_2| \ell_1,-m_1} \\
  \bk{\ell_1 m_1; \ell_2 m_2| L M}
  &= \p{-1}^{\ell_1+\ell_2-L}
  \bk{\ell_1,-m_1; \ell_2,-m_2| L,-M},
\end{align}
we can find that the transition operators transform under conjugation as
\begin{align}
  T_{\ell m}^\dag
  = \sqrt{\f{2\ell+1}{2s+1}}
  \sum_{\mu,\nu} \p{-1}^m \bk{s\nu;\ell,-m|s\mu} \op{\mu}{\nu}
  = \p{-1}^m T_{\ell,-m},
\end{align}
which implies that
\begin{align}
  f_{\ell_1 m_1;\ell_2 m_2}^{LM}
  = \p{-1}^M \sqrt{\f{\p{2L+1}\p{2\ell_1+1}\p{2\ell_2+1}}
    {\p{2s+1}\p{2s+1}\p{2s+1}}}
  \sum_{\mu,\nu,\rho} \bk{s\nu;L,-M|s\mu}
  \bk{s\rho;\ell_1m_1|s\nu} \bk{s\mu;\ell_2m_2|s\rho}.
\end{align}
Replacing Clebsch-Gordan coefficients by Wigner 3-$j$ symbols with the identity
\begin{align}
  \bk{\ell_1 m_1; \ell_2 m_2| L M}
  = \p{-1}^{2\ell_2} \p{-1}^{L-M} \sqrt{2L+1}
  \begin{pmatrix}
    L & \ell_2 & \ell_1 \\
    -M & m_2 & m_1
  \end{pmatrix},
\end{align}
we can use the fact that $2\ell_2$ is always even (because $\ell_2$ is always an integer) to expand
\begin{align}
  f_{\ell_1 m_1;\ell_2 m_2}^{LM}
  = \p{-1}^M \sqrt{\p{2L+1}\p{2\ell_1+1}\p{2\ell_2+1}}
  \sum_{\mu,\nu,\rho} \p{-1}^{3s-\mu-\nu-\rho}
  \begin{pmatrix}
    s & L & s \\
    -\mu & -M & \nu
  \end{pmatrix}
  \begin{pmatrix}
    s & \ell_1 & s \\
    -\nu & m_1 & \rho
  \end{pmatrix}
  \begin{pmatrix}
    s & \ell_2 & s \\
    -\rho & m_2 & \mu
  \end{pmatrix}.
\end{align}
This sum can be simplified by the introduction of Wigner 6-$j$ symbols, giving us
\begin{align}
  f_{\ell_1 m_1;\ell_2 m_2}^{LM}
  &= \p{-1}^{2s+M} \sqrt{\p{2L+1}\p{2\ell_1+1}\p{2\ell_2+1}}
  \begin{pmatrix}
    L & \ell_1 & \ell_2 \\
    M & -m_1 & -m_2
  \end{pmatrix}
  \begin{Bmatrix}
    L & \ell_1 & \ell_2 \\
    s & s & s
  \end{Bmatrix} \\
  &= \p{-1}^{2s+L} \sqrt{\p{2\ell_1+1}\p{2\ell_2+1}}
  \bk{\ell_1 m_1; \ell_2 m_2| LM}
  \begin{Bmatrix}
    \ell_1 & \ell_2 & L \\
    s & s & s
  \end{Bmatrix}.
\end{align}

\end{document}

%%% Local Variables:
%%% mode: latex
%%% TeX-master: t
%%% End:


signal recovery literature:
-- a $\sim2d^2$ signaling theorem \cite{mcewen2011novel}
-- a nice review of these methods \cite{mcewen2011sampling}
-- an empirically accurate (albeit without rigorous proofs/guarantees) $d^2$ sampling method \cite{khalid2014optimaldimensionality}
-- current status of the theory behind sparse signal recovery with random sampling \cite{rauhut2011sparse} (note: existing bounds are not believed to be tight)
-- nice numerical method for sparse signal recovery \cite{alem2012sparse}
-- useful textbook references \cite{freeden2008spherical, freeden2018spherical}
