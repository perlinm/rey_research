\documentclass[nofootinbib,twocolumn]{revtex4-2}

%%% linking references
\usepackage[dvipsnames]{xcolor}
\usepackage{hyperref}
\hypersetup{
  breaklinks=true,
  colorlinks=true,
  allcolors=BlueViolet,
}

%%% symbols, notations, etc.
\usepackage{physics,braket,bm,amssymb} % physics and math
\renewcommand{\t}{\text} % text in math mode
\newcommand{\f}[2]{\dfrac{#1}{#2}} % shorthand for fractions
\newcommand{\p}[1]{\left(#1\right)} % parenthesis
\renewcommand{\sp}[1]{\left[#1\right]} % square parenthesis
\renewcommand{\set}[1]{\left\{#1\right\}} % curly parenthesis
\newcommand{\bk}{\Braket} % shorthand for braket notation
\renewcommand{\v}{\bm} % bold vectors
\renewcommand{\c}{\cdot} % inner product
\renewcommand{\d}{\partial} % partial derivative
\renewcommand{\dd}{\text{d}} % for infinitesimals
\renewcommand{\i}{\mathrm{i}\mkern1mu} % imaginary unit

% double angle brackets
\newcommand{\bbk}[1]{\langle\!\langle #1 \rangle\!\rangle}
\newcommand{\Bbk}[1]
{\left\langle\!\!\left\langle #1 \right\rangle\!\!\right\rangle}

\usepackage{dsfont} % for identity operator
\newcommand{\1}{\mathds{1}}

% more shorthands
\newcommand{\up}{\uparrow}
\newcommand{\dn}{\downarrow}
\newcommand{\x}{\text{x}}
\newcommand{\y}{\text{y}}
\newcommand{\z}{\text{z}}
\newcommand{\g}{\text{g}}
\newcommand{\e}{\text{e}}
\renewcommand{\SS}{\mathbb{S}}
\newcommand{\ZZ}{\mathbb{Z}}
\newcommand{\A}{\mathcal{A}}
\newcommand{\C}{\mathcal{C}}
\newcommand{\I}{\mathcal{I}}
\renewcommand{\O}{\mathcal{O}}
\renewcommand{\P}{\mathcal{P}}

\newcommand{\xx}{\x\x}
\newcommand{\xxi}{\x\x_\i}
\newcommand{\X}{\text{X}}
\newcommand{\XX}{\X\X}
\newcommand{\XXI}{\X\X_\i}
\newcommand{\su}{\mathfrak{su}}
\newcommand{\MF}{\text{MF}}
\renewcommand{\ss}{\bar{\v s}\c\bar{\v s}}
\DeclareMathOperator{\diag}{diag}

%%% for in-line enumerated lists
\usepackage[inline]{enumitem}
\setlist[enumerate,1]{label={(\roman*)}} % default in-line numbering
\setlist{nolistsep} % more compact spacing between environments

% use alphanumeric tags for footnotes
\renewcommand*{\thefootnote}{\alph{footnote}}

%%% figures
\usepackage{graphicx} % for figures
\graphicspath{{./figures/}} % set path for all figures

% for text markup
\newcommand{\red}[1]{{\color{red} #1}}

%%%%%%%%%%%%%%%%%%%%%%%%%%%%%%%%%%%%%%%%%%%%%%%%%%%%%%%%%%%%%%%%%%%%%%
\begin{document}
\interfootnotelinepenalty=10000 % don't split footnotes between pages

\title{Multilevel spin models in cold atomic systems}
\author{Michael A. Perlin}
% \orcid{https://orcid.org/0000-0002-9316-1596}
\email{mika.perlin@gmail.com}
\author{Diego Barbarena}
\author{Robert J.~Lewis-Swan}
% \orcid{https://orcid.org/0000-0002-0201-281X}
\author{Ana Maria Rey}
% \orcid{http://orcid.org/0000-0001-7176-9413}
\affiliation{JILA, National Institute of Standards and Technology and University of Colorado, 440 UCB, Boulder, Colorado 80309, USA}
\affiliation{Center for Theory of Quantum Matter, University of Colorado, Boulder, CO, 80309, USA}

\date{\today}

\begin{abstract}
  Abstract.
\end{abstract}

\maketitle

%%%%%%%%%%%%%%%%%%%%%%%%%%%%%%%%%%%%%%%%%%%%%%%%%%%%%%%%%%%%%%%%%%%%%%
\section{Introduction}
\label{sec:intro}

SU($n$) symmetries play an important role in physics.
Underpinning much of high energy physics, the SU($n$) gauge theory known as Yang-Mills theory is central to our understanding of the electroweak and strong forces.
Extensions of Yang-Mills and SU($n$) symmetry feature in the most well-studied examples of holographic duality \cite{maldacena1999largen} and the connection between entanglement and gravity \cite{ryu2006holographic} through the anti-de Sitter/conformal field theory (AdS/CFT) correspondence.
In a condensed matter setting, SU(2) appears ubiquitously as a symmetry of the Hubbard model, with important consequences for the study of quantum magnetism and high temperature superconductivity \cite{lee2006doping}.
The extension of SU(2) Hubbard and spin models to SU($n$) has led to predictions of exotic phases of matter such as valence bond solids \cite{read1989valencebond, rokhsar1990quadratic, kaul2012lattice, hermele2011topological} and chiral spin liquids \cite{hermele2009mott, hermele2011topological, chen2016syntheticgaugefield, nataf2016chiral}, as well as the potential to perform universal topological quantum computation \cite{freedman2004class, nayak2008nonabelian} and other phenomena \cite{nataf2014exact, nataf2016exact}.
Furthermore, the consideration of disordered SU($n$) spin models has opened analytically tractable avenues for studying quantum chaos and information scrambling \cite{sachdev1993gapless, bentsen2019integrable}.

The tremendous theoretical significance of SU($n$) symmetries makes it all the more exciting that they appear naturally in experimental atomic, molecular, and optical (AMO) platforms \cite{gorshkov2010twoorbital, beverland2016realizing, cazalilla2014ultracold, taie2012su, hofrichter2016direct, cappellini2014direct, scazza2014observation, zhang2014spectroscopic, goban2018emergence, perlin2019effective} with exquisite degrees of microscopic control.
In the simplest example, this symmetry arises through the independence of atomic orbital and interaction parameters on the $n$ nuclear spin states of alkaline-earth(-like) atoms, with e.g.~$n=10$ in the case of ${}^{87}$Sr \cite{cazalilla2014ultracold}.
As a result, experiments can directly probe the SU($n$) Hubbard model, leading to experimental observations of SU($n$) Hubbard phases and phase transitions \cite{taie2012su, hofrichter2016direct}, two-orbital SU($n$) magnetism \cite{cappellini2014direct, scazza2014observation, zhang2014spectroscopic}, and multi-body SU($n$)-symmetric interactions \cite{goban2018emergence, perlin2019effective}.
In the spirit of quantum simulation, further investigations of this sort can play an important role in understanding SU($n$) symmetries' consequences for fundamental questions in physics, as well as their practical use in technological applications.
For example, SU(2)-symmetric spin interactions can be harnessed to develop quantum sensors that surpass classical limits on measurement precision \cite{he2019engineering, perlin2020spin}.
The exciting prospect of similarly exploiting larger SU($n$) symmetries to achieve a technological advantage is still an unexplored avenue of research with enormous potential.

In this work, we ... \red{[finish intro, summarize paper]}
% the (un)interested reader can ...
% main results in/below Eqs. ...

%%%%%%%%%%%%%%%%%%%%%%%%%%%%%%%%%%%%%%%%%%%%%%%%%%%%%%%%%%%%%%%%%%%%%%
\section{From lattice fermions to an SU($n$) spin model}
\label{sec:spin_model}

Here we review the realization of a collective SU($n$) spin model in a system of ultracold alkaline-earth(-like) atoms on an optical lattice.
Without external driving fields, the evolution of such atoms in their electronic ground state is governed by single-body kinetic and two-body interaction Hamiltonians of the form
\begin{align}
  H_{\t{kin}}
  &= -\f{J}{2} \sum_{\bk{j,j'},\mu} c_{j\mu}^\dag c_{j'\mu} + \t{h.c.}, \\
  H_{\t{int}}
  &= \f{U_0}{2} \sum_{j,\mu,\nu}
  c_{j\mu}^\dag c_{j\mu} c_{j\nu}^\dag c_{j\nu},
\end{align}
where $\bk{j,j'}$ denotes neighboring lattice sites $j$ and $j'$; $\mu,\nu\in\set{s,s-1,\cdots,-s}$ index orthogonal spin states of a spin-$s$ nucleus, with $s=\frac{n-1}{2}$ (e.g.~$s=\frac{9}{2}$ in the case of ${}^{87}$Sr with $10$ nuclear spin states); $c_{j\mu}$ is a fermionic annihilation operator, $J$ is a tunneling amplitude; and $U_0$ is a two-body on-site interaction energy.
In the present work, we neglect inter-site interactions and interaction-assisted hopping, which may become relevant in a sufficiently shallow lattice, namely when $J\gtrsim E_{\t{R}}$, where $E_{\t{R}}$ is the atomic lattice recoil energy.
For simplicity, we now assume a periodic lattice of $L$ sites, and expand the on-site fermionic operators in terms of operators addressing (quasi-)momentum modes $q$, $c_{j\mu}=\frac1{\sqrt{L}}\sum_k e^{-\i q\c j} c_{q\mu}$, finding that
\begin{align}
  H_{\t{kin}}
  &= -J\sum_{q,\mu} \cos\p{q} c_{q\mu}^\dag c_{q\mu}, \\
  H_{\t{int}}
  &= \f{U}{2N} \sum_{k,\ell,p,q,\mu,\nu}
  c_{k\mu}^\dag c_{\ell\mu} c_{p\nu}^\dag c_{q\nu}
  \times \delta_{k+p,\ell+q},
  \label{eq:H_int_momenta}
\end{align}
where $N$ is the total number of atoms on the lattice, we define $U\equiv U_0\times N/L$ for convenience, $\delta_{k+p,\ell+q}=1$ if $k+p=\ell+q$ and zero otherwise (enforcing conservation of momentum), and we work in units with lattice spacing $a=1$.

If the interaction energy $U_0$ is smaller than the single-particle bandwidth $\sim J$, then the mode-changing collisions in $H_{\t{int}}$ become off-resonant, motivating the frozen-mode approximation $\set{k,p}=\set{\ell,q}$ (which we assume for the remainder of this work)\footnote{Note that the frozen-mode approximation neglects correlated momentum-hopping terms of the form $c_{\pi-p,\mu}^\dag c_{\pi-q,\mu} c_{p\nu}^\dag c_{q\nu}$, which conserve both momentum and energy.
We defer a careful treatment of these terms to future work, noting only that they vanish on the manifold of permutationally-symmetric spin states with one atom per lattice site.}.
The operator content of terms with $k=\ell$ and $p=q$ takes the form $c_{\ell\mu}^\dag c_{\ell\mu} c_{q\nu}^\dag c_{q\nu}$, so these terms contribute an overall energy $\frac12NU$ that can be neglected in the absence of coherent atom number fluctuations.
Defining the spin operators $s_{\mu\nu q}\equiv c_{q\mu}^\dag c_{q\nu}$, the remaining terms of the kinetic and interaction Hamiltonians become
\begin{align}
  H_{\t{kin}} &= -J\sum_{q,\mu} \cos\p{q} s_{\mu\mu q},
  \label{H_kin_start} \\
  H_{\t{int}} &= -\f{U}{2N} \sum_{p,q,\mu,\nu} s_{\mu\nu p} s_{\nu\mu q}.
  \label{eq:H_int_start}
\end{align}
Throughout this work, we will assume that atomic modes are singly-occupied, e.g.~due to the initialization of a spin-polarized state with one atom per lattice site, in which multiple occupation of an atomic mode is forbidden by fermionic statistics (Pauli exclusion).
In this case, we can neglect the fermionic origins of $H_{\t{int}}$ and instead treat our system as a bona fide spin model.
Note that the ``kinetic'' terms of this spin model ($H_{\t{kin}}$) are proportional to the identity operator, and can be neglected for the time being.
The fermionic origin of these terms will return to play an important role when we consider spin-orbit coupling in Section \ref{sec:SOC}.

To further simplify the interaction Hamiltonian $H_{\t{int}}$ and write it in a form suggestively similar to SU(2) spin alignment, we now construct the operator-valued spin matrix
\begin{align}
  \v s_q \equiv \sum_{\mu,\nu} s_{\mu\nu q} \op{\mu}{\nu},
\end{align}
and for any pair of such operator-valued matrices $\v A,\v B$, we define the inner product
\begin{align}
  \v A \c \v B \equiv \Tr_{\t{aux}}\p{\v A^\dag\v B}
  = \sum_{\mu,\nu} A_{\mu\nu}^\dag B_{\mu\nu},
\end{align}
where the trace is performed over the auxiliary matrix degrees of freedom.
These definitions allow us to write the spin Hamiltonian in Eq.~\eqref{eq:H_int_start} as
\begin{align}
  H_{\t{int}} = -\f{U}{2N} \sum_{p,q} \v s_p\c\v s_q
  = -\f{U}{2N}\v S\c\v S,
  &&
  \v S \equiv \sum_q \v s_q.
  \label{eq:H_int}
\end{align}
Here $\v S$ is a collective spin matrix, analogous to the collective spin vector $\vec S=\p{S_\x,S_\y,S_\z}$ in the case of SU(2) \cite{he2019engineering}, with $\v S\c\v S \simeq 2 \vec S\c\vec S = 2 \sum_{\alpha\in\set{\x,\y,\z}}S_\alpha^2$ when $n=2$ (here $\simeq$ denotes equality after modding out identity terms).

\begin{figure}
\centering
\includegraphics[width=\linewidth]{spin_model.pdf}
\caption{
(a) Ultracold atoms on a lattice of $L$ sites can tunnel between neighboring lattice sites at a rate $J$, and locally repel each other with interaction energy $U_0$.
(b,c) When the interaction energy $U_0$ is smaller than the single-particle bandwidth $\sim J$, the frozen-mode approximation enables writing the interaction Hamiltonian as a spin model consisting of exchange terms $\v s_p\c\v s_q$, which swap the states two spins pinned to momentum modes $p,q$.
(d) Interactions thereby open an energy gap $U=U_0\times N/L$ between the manifold of permutationally-symmetric states of $N$ spins, and the orthogonal complement of states that break spin-permutation symmetry.
}
\label{fig:spin_model}
\end{figure}

We now discuss the spin Hamiltonian $H_{\t{int}}$ in Eq.~\eqref{eq:H_int}.
The operator $\v s_p\c\v s_q$ simply swaps the nuclear spin states of two atoms pinned to modes $p,q$.
The term $-\v s_p\c\v s_q$ thereby assigns a definite energy of $-1$ ($+1$) to a pair of spins that are symmetric (anti-symmetric) under exchange.
In this sense, $\v s_p\c\v s_q$ is analogous to the enforcement of SU(2) spin alignment by ferromagnetic interactions, which similarly assigns (distinct) definite energies to anti-symmetric spin-0 singlet $\ket{\up\dn}-\ket{\dn\up}$ and the symmetric spin-1 triplets $\set{\ket{\up\up},\ket{\dn\dn},\ket{\up\dn}+\ket{\dn\up}}$.
By summing over all pair-wise exchange terms $\v s_p\c\v s_q$, the interaction Hamiltonian $H_{\t{int}}$ energetically enforces a permutational symmetry among all spins, opening an energy gap $U$ between the manifold of all permutationally-symmetric (PS) states and the orthogonal complement of excited (e.g.~spin-wave) states that break permutational symmetry.
See Figure \ref{fig:spin_model} for a summary of this section thus far.

In the case of SU(2), the PS manifold is precisely the Dicke manifold of collective states $\ket{m_\z}$ with total spin $S=\frac{N}{2}$ and definite spin projection $m_\z\in\set{S,S-1,\cdots,-S}$ onto a fixed quantization axis.
Equivalently, Dicke states $\ket{m_\z}=\ket{m_\up,m_\dn}$ can be labeled by a definite number of spins $m_\up=S+m_\z$ ($m_\dn=S-m_\z$) pointing up (down) along the spin quantization axis, with $m_\up+m_\dn=N$.
In the general case of SU($n$), the PS manifold is similarly spanned by states $\ket{m_s,m_{s-1},\cdots,m_{-s}}$ with a definite number $m_\mu$ of spins in state $\mu$, and $\sum_\mu m_\mu=N$.
The dimension of the PS manifold is equal to the number of ways of assigning $N$ identical spins to $n$ distinct internal states, or ${N+n-1 \choose n-1} \sim N^{n-1}$.

External fields or additional interactions that respect permutational symmetry can induce nontrivial dynamics within the PS manifold.
Even if additional Hamiltonian terms break permutational symmetry, as long as the coupling to states outside of the PS manifold is small (compared to the energy gap $U$) these terms can be restricted to the PS manifold perturbatively (see Appendix \ref{sec:pert_theory}).
Simulating dynamics within the PS manifold requires calculating matrix elements $\bk{\ell|\O|m}$ of spin operators $\O$ with respect to PS states $\ket\ell,\ket{m}$; we discuss this calculation in Appendix \ref{sec:PS_ops}.

Finally, we take a moment to discuss individual $n$-level spins.
Whenever possible, it is desirable to find geometric representations of abstract mathematical objects, such as the state of a quantum system.
Geometric representations facilitate the development of understanding and intuitions that can be difficult to extract from algebraic expressions alone.
For this reason, the state of a two-level spin, or a qubit, is commonly represented by a point on (or within) the Bloch sphere.
More generally, the state $\ket\psi$ of a $n$-level spin can be represented by a probability distribution\footnote{Strictly speaking, in order to make $Q_\psi$ (and $Q_\rho$) a probability distribution, one has to divide it by a normalization factor $\frac{4\pi}{d}$.
We omit this normalization factor for convenience.} $Q_\psi$ on the sphere $\SS^2$.
The value $Q_\psi\p{\v v}$ at a point $\v v\in\SS^2$ equal to the overlap of $\ket\psi$ with a pure state $\ket{\v v}$ maximally polarized along $\v v$: $Q_\psi\equiv\abs{\bk{\v v|\psi}}^2$ (see Figure \ref{fig:spin_dist}).
In the case of a mixed state $\rho$, this distribution is defined by $Q_\rho\equiv\bk{\v v|\rho|\v v}$.
The dimension $n$ is reflected in $Q_\rho$ by the fact that it can be decomposed into spherical harmonics $Y_{\ell m}$ with degree $\ell<n$ (that is, $Q_\rho$ is ``low resolution'' or ``band-limited'' at degree $n$) \cite{perlin2020qudit}, as well as in the normalization $\int_{\v v\in\SS^2}\dd^2\v v\,Q_\rho\p{\v v}=\frac{4\pi}{d}$.
Closely related spherical representations of multilevel spin states (and operators) are discussed in Ref.~\cite{li2013weylwignermoyal}.

\begin{figure}
\centering
\includegraphics[width=\linewidth]{qudit_rep.pdf}
\caption{
Whereas the state of a two-level spin (qubit) can be represented by a point on (or inside) the Bloch sphere, the state of a $n$-level spin is more generally represented by a probability distribution on the sphere, decomposable into spherical harmonics $Y_{\ell m}$ of degree $\ell<n$.
The distribution shown for $n=10$ corresponds to a Haar-random pure state.
}
\label{fig:spin_dist}
\end{figure}

In fact, $Q_\rho$ is essentially the Husimi-$Q$ function commonly used (e.g.~in the spin squeezing community \cite{ma2011quantum}) to represent collective states of spin-$\frac{1}{2}$ particles in the Dicke manifold.
This correspondence is made precise by identifying the Hilbert space of a single $n$-level spin with the Dicke manifold of $n-1$ spin-$\frac{1}{2}$ particles.
The Hilbert space of a 10-level (spin-$\frac92$) nucleus of a ${}^{87}$Sr atom, for example, can be thought of as the Dicke manifold of 9 ``excess'' proton and neutron spins.

%%%%%%%%%%%%%%%%%%%%%%%%%%%%%%%%%%%%%%%%%%%%%%%%%%%%%%%%%%%%%%%%%%%%%%
\section{Spin-orbit coupling and gauge transformation}
\label{sec:SOC}

We wish to study the dynamical behaviors of experimentally realizable SU($n$) spin models.
In order to ensure validity of the $\v S\c\v S$ spin Hamiltonian in Eq.~\eqref{eq:H_int}, we consider initial preparation of a spin-polarized state\footnote{We remind the reader that validity of the spin model requires that each momentum mode is at most singly-occupied, which is guaranteed for a spin-polarized state due to Pauli blocking.}, which has the added benefit of being simple to prepare experimentally.
However, a spin polarized state is an eigenstate of the $\v S\c\v S$ spin Hamiltonian, so we need additional ingredients for nontrivial behavior.
Here, we consider the addition of spin-orbit coupling (SOC) induced by external driving fields.

Before discussing SOC for $n$-level fermions, we first consider the well-studied case of two-level SOC on a one-dimensional lattice \cite{wall2016synthetic, bromley2018dynamics}.
In this case, SOC is induced by an external driving field that imprints a phase $e^{-\i\phi j}$ on lattice site $j$, or equivalently imparts a momentum kick $q\to q+\phi$, upon the absorption of a photon:
\begin{align}
  H_{\t{drive}}^{(\phi)}
  = \f{\Omega}{2} \sum_q c_{q+\phi,\up}^\dag c_{q,\dn} + \t{h.c.}.
  \label{eq:drive_2}
\end{align}
Identifying a numerical spin index $\mu=+\frac12$ ($-\frac12$) with the state $\up$ ($\dn$), this drive Hamiltonian can be diagonalized in its momentum index $q$ by the gauge transformation $c_{q\mu}^\dag\to c_{q-\mu\phi,\mu}^\dag$ (equivalently $c_{j\mu}^\dag\to e^{\i\phi\mu j} c_{j\mu}^\dag$), which takes
\begin{align}
  H_{\t{drive}}^{(\phi)} \to H_{\t{drive}} \equiv \Omega S_\x,
  &&
  S_\x \equiv \sum_q s_{\x,q},
  \label{eq:drive_trans}
\end{align}
where $S_\x$ is a collective spin-$x$ operator, and $s_{\x,q}=\frac12 c_{q,\up}^\dag c_{q,\dn} + \t{h.c.}$ for two-level spins.

The two-level drive in Eq.~\eqref{eq:drive_2} is typically implemented by coupling pseudo-spin electronic states of an atom with an external laser \cite{wall2016synthetic, bromley2018dynamics}.
Though the coupling between laser fields and nuclear spins is generally too weak to drive nuclear spin transitions directly, it will nonetheless be useful to consider the $n$-level generalization of the two-level drive.
This generalization will capture the important features of SOC for $n$-level nuclei, while avoiding complications that we defer to Section \ref{sec:controls}.

The spin-$s$ generalization of Eq.~\eqref{eq:drive_2} is
\begin{align}
  H_{\t{drive}}^{(\phi)} = \Omega_0 \sum_{q,\mu} \bk{s,\mu;1,1|s,\mu+1}
  c_{q+\phi,\mu+1}^\dag c_{q\mu} + \t{h.c.},
\end{align}
where $\mu\in\set{s,s-1,\cdots,-s}$, and $\bk{s,\mu;1,1|s,\mu+1}$ is a Clebsch-Gordan coefficient.
The gauge gauge transformation $c_{q\mu}^\dag\to c_{q-\mu\phi,\mu}^\dag$ again takes $H_{\t{drive}}^{(\phi)}\to \Omega S_\x$ as in Eq.~\eqref{eq:drive_trans}, where now
\begin{align}
  \Omega = \f{\Omega_0}{\xi},
  &&
  \xi \equiv -\sqrt{\f{s(s+1)}{2}},
\end{align}
and
\begin{align}
  s_{\x,q} \equiv \xi \sum_\mu \bk{s,\mu;1,1|s,\mu+1} c_{q,\mu+1}^\dag c_{q\mu} + \t{h.c.}
  \label{eq:spin_x_op}
\end{align}
is an $n$-level spin-$x$ operator for spin $q$, with which $e^{-\i\theta s_{\x,q}}$ rotates the state of spin $q$ about the $x$ axis by an angle $\theta$ (literally rotating the sphere in Figure \ref{fig:spin_dist}).
Despite complicating details, the change from two-level to $n$-level spins thus results in a drive Hamiltonian with an identical physical interpretation: after moving into an appropriate ``gauged frame'', this drive simply rotates all spins about the $x$ axis at a rate $\Omega$.

\begin{figure}
\centering
\includegraphics{soc_panels.pdf}
\caption{
Spin-orbit coupling for 2-level ({\bf a},{\bf c}) and 4-level ({\bf b},{\bf d}) spins.
Colors indicate different spin projections $\mu$.
In the ``lab frame'' ({\bf a},{\bf b}), kinetic energy is insensitive to spin, but the drive imparts a momentum kick $q\to q+\phi$ into any atom that absorbs a photon.
Changing to the ``gauged frame'' ({\bf c},{\bf d}), essentially by shifting the momentum label $q$ for each spin state $\mu$, makes the drive diagonal in the momentum index, but comes at the cost of making kinetic energy spin-dependent.
}
\label{fig:soc_panels}
\end{figure}

Of course, spin-orbit coupling cannot be ``gauged away'' entirely.
Making a gauge transformation to simplify the drive comes at the cost of making kinetic energy spin-dependent, taking
\begin{align}
  H_{\t{kin}} \to H_{\t{kin}}^{(\phi)}
  \equiv -J \sum_q \cos\p{q+\mu\phi} s_{\mu\mu q},
\end{align}
as visualized in Figure \ref{fig:soc_panels}.
To better interpret this Hamiltonian, we can write it in the simplified form
\begin{align}
  H_{\t{kin}}^{(\phi)}
  = -J \sum_q
  \sp{\cos\p{q} w_{+,q}^{(\phi)} - \sin\p{q} w_{+,q}^{(\phi)}},
\end{align}
where
\begin{align}
  w_{+,q}^{(\phi)} &\equiv \sum_\mu \cos\p{\mu\phi} s_{\mu\mu q}, \\
  w_{-,q}^{(\phi)} &\equiv \sum_\mu \sin\p{\mu\phi} s_{\mu\mu q}.
\end{align}
For two-level spins, $w_{+,q}^{(\phi)}$ is proportional to the identity operator, and $w_{+,q}^{(\phi)}=2\sin\p{\phi/2} s_{q,\z}$, where $s_{\z,q}=\sum_\mu \mu\, s_{\mu\mu q}$ is a spin-$z$ operator for spin $q$, so the kinetic Hamiltonian in the gauged frame describes a (synthetic) inhomogeneous magnetic field:
\begin{align}
  \left. H_{\t{kin}}^{(\phi)} \right|_{n=2}
  = 2\sin\p{\phi/2} J \sum_q \sin\p{q} s_{\z,q}.
\end{align}
When $n>2$, an inhomogeneous magnetic field is likewise recovered in the limit of weak SOC, $s\phi\ll1$, in which case
\begin{align}
  \left. H_{\t{kin}}^{(\phi)} \right|_{s\phi\ll1}
  = J\phi \sum_q \sin\p{q} s_{\z,q} + O\p{(s\phi)^2}.
\end{align}
As the strength of SOC is increased, this Hamiltonian acquires terms with higher powers of $s_{\z,q}$, up to $s_{\z,q}^{n-1}$.

In principle, making the gauge transformation $c_{q\mu}^\dag\to c_{q-\mu\phi,\mu}^\dag$ also transforms the interaction Hamiltonian as $H_{\t{int}} \to H_{\t{int}}^{(\phi)}$, where
\begin{align}
  H_{\t{int}}^{(\phi)}
  \equiv -\f{U}{2N} \sum_{p,q,\mu,\nu}
  c_{p-\mu\phi,\mu}^\dag c_{p-\nu\phi,\nu}
  c_{q-\nu\phi,\nu}^\dag c_{q-\mu\phi,\mu}.
\end{align}
However, $H_{\t{int}}^{(\phi)}$ cannot be exactly reduced to a spin Hamiltonian that addresses the same spin degrees of freedom as $H_{\t{kin}}^{(\phi)}$.
Nonetheless, for sufficiently weak SOC (small $s\phi$) the effect of $H_{\t{int}}^{(\phi)}$ may be approximated by $H_{\t{int}}^{(0)}=H_{\t{int}}$, which has previously been benchmarked for SU(2)-symmetric interactions \cite{he2019engineering, smale2019observation}.
To ensure that $H_{\t{kin}}^{(\phi)}$ does not become trivial as $\phi\to0$, we can simultaneously take $J\to\infty$ ($J$ large compared to $U$) while keeping $J\phi$ constant.
Altogether, the interacting spin Hamiltonian in the gauged frame becomes
\begin{align}
  H_{\t{spin}} = -\f{U}{2N} \v S\c\v S + J\phi \sum_q \sin\p{q} s_{\z,q},
  \label{eq:H_spin}
\end{align}
consisting of a spin-locking $\v S\c\v S$ term that energetically favors permutational symmetry, and an inhomogeneous magnetic field that causes inter-spin dephasing.

%%%%%%%%%%%%%%%%%%%%%%%%%%%%%%%%%%%%%%%%%%%%%%%%%%%%%%%%%%%%%%%%%%%%%%
\section{Control fields, observables, and initial states}
\label{sec:controls}

We now take a closer look at external control fields, which determine the observables we can access and the initial states we can prepare.
To this end, we consider addressing the atoms with several lasers detuned by $\Delta$ below an electronic ${^1}\t{S}_0\to {^3}\t{P}_0$ transition of individual alkaline-earth-like atoms on a one-dimensional lattice with tight transverse confinement.
Each laser is indexed by an axis of propagation, $\v v$, and a drive amplitude $\Omega_{\v v,+}$ ($\Omega_{\v v,-}$) for right-circularly (left-circularly) polarized light.
An electronic ${^1}\t{S}_0\to {^3}\t{P}_0$ transition is thus accompanied by a nuclear spin transition $\mu_{\v v}\to\mu_{\v v}\pm1$, where $\mu_{\v v}$ is the projection of nuclear spin onto the axis $\v v$.
Furthermore, a laser pointing along $\v v$ will generally imprint a phase $e^{-\i\kappa\v v\c\v\ell j}$ on lattice site $j$, where $\kappa$ is the laser's wavenumber (in units with lattice spacing $a=1$) and $\v\ell$ is a fixed unit vector parallel to the lattice.
Altogether, the multi-laser drive Hamiltonian can be written in the form
\begin{align}
  H_{\t{drive}}^{\t{full}}
  = \sum_{j,\v v,\sigma} \Omega_{\v v\sigma}
  \p{e^{-\i\kappa\v v\c\v\ell j} s_{\v v\sigma j}\otimes\op{\e}{\g}_j + \t{h.c.}}
  + \Delta N_\e,
  \label{eq:drive_full}
\end{align}
where $\sigma\in\set{+1,-1}$ indexes a polarization of light; $s_{\v v\sigma j}$ is a $n$-level spin-raising ($\sigma=+1$) or spin-lowering ($\sigma=-1$) operator along axis $\v v$ for atom $j$ (clarified below); $\ket\g_j$ and $\ket\e_j$ respectively denote the ground (${^1}\t{S}_0$) and excited (${^3}\t{P}_0$) electronic states of an atom on site $j$; and $N_\e=\1\otimes\sum_j\op\e_j$ is the number of excited atoms (with $\1$ the identity operator on all spin degrees of freedom).

To define the spin-raising/lowering operators $s_{\v v\sigma j}$, we first define $s_{\x,j}$ similarly to $s_{\x,q}$ in Eq.~\eqref{eq:spin_x_op} with $q\leftrightarrow j$.
The spin-raising operator $s_{+,j}$ is then twice the spin-raising part of $s_{\x,j}$, just as in the case of SU(2) where $s_\x=\frac12\op{\up}{\dn}+\t{h.c.}$, and the spin-lowering operator is $s_{-,j}\equiv s_{+,j}^\dag$.
The operator $s_{\v v,+\,j}$ that raises spin along $\v v$ is acquired by a rotation of $s_{+,j}$ that maps the north pole to $\v v$:
\begin{align}
  s_{\v v,+} \equiv U_{\v v} s_+ U_{\v v}^\dag,
  &&
  U_{\v v} \equiv e^{-\i\alpha_{\v v}s_\z} e^{-\i\beta_{\v v}s_\y},
  \label{eq:spin_rot}
\end{align}
where we have dropped the site index $j$ for brevity; $\alpha_{\v v}$ and $\beta_{\v v}$ are the azimuthal and polar angles of $\v v$; and $s_\y\equiv-\frac\i2s_++\t{h.c.}$.
Note that the operators $(s_\x,s_\y,s_\z)$ form an $\su(2)$ algebra, so given a particular axis $\v v$ one can expand $s_{\v v,+}=\sum_m c_{\v vm} s_m$ with coefficients $c_{\v vm}$ that are independent of the dimension $n$.
In particular, one is always free to set $n=2$ and expand Eq.~\eqref{eq:spin_rot} with spin-$\frac12$ operators (or Pauli matrices) to find the coefficients $c_{\v vm}$.

\begin{figure}
\centering
\includegraphics[width=\linewidth]{3LD.pdf}
\caption{
Sketch of the three-laser drive used to address nuclear spins on a one-dimensional lattice.
Two counter-propagating lasers with right-circular polarization and amplitudes $\Omega_\pm$ point at an angle $\theta$ to the lattice axis.
A third, linearly polarized laser with amplitude $\Omega_0$ points in a direction orthogonal to both the lattice and the other driving lasers.
Absorbing a photon from the laser with amplitude $\Omega_m$ induces a transition $(\g,\mu)\to(\e,\mu+m)$ for the (electronic, nuclear spin) state of an atom, where nuclear spin is quantized along the axis of propagation for the $m=+1$ laser.
}
\label{fig:3LD}
\end{figure}

In order to make the drive Hamiltonian in Eq.~\eqref{eq:drive_full} more tractable, we orient the lattice along the $z$ axis, $\v\ell=(0,0,1)$, and consider a three-laser drive with a geometry sketched in Figure \ref{fig:3LD}.
In this setup, two counter-propagating right-circularly polarized lasers point along $\v v_\pm=\pm(0,-\sin\theta,\cos\theta)$, at an angle $\theta$ to the lattice axis, and a third linearly-polarized laser points along $\v v_0=(1,0,0)$.
Choosing $\v v_+$ as the spin quantization axis, we can define new drive amplitudes $\Omega_0\equiv-(\Omega_{\v v_0,+}+\Omega_{\v v_0,-})$ and $\Omega_\pm\equiv\pm\Omega_{\v v_\pm,+}$, as well as a SOC angle $\phi\equiv\kappa\v v_+\c\v\ell=\kappa\cos\theta$, in terms of which the drive Hamiltonian becomes
\begin{align}
  H_{\t{3LD}}^{\t{full}}
  = \sum_{j,m} \Omega_m
  \p{e^{-\i m\phi j} s_{mj} \otimes\op{\e}{\g}_j + \t{h.c.}}
  + \Delta N_\e,
\end{align}
where $m\in\set{+1,0,-1}$ indexes the laser pointing along $\v v_m$; and $s_{0,j}\equiv s_{\z,j}$ for shorthand.
In the far-detuned limit $\abs{\Delta}\gg\abs{\Omega_m}$, a second-order perturbative treatment of electronic excitations ($\ket\e$) yields an effective drive Hamiltonian that only addresses ground-state nuclear spins.
After a gauge transformation $c_{j\mu}^\dag\to e^{\i\phi\mu j} c_{j\mu}^\dag$ (equivalently $s_{mj}\to e^{\i m\phi j}s_{mj}$), we can then make the simplifying assumption that all $\Omega_m$ are real to write
\begin{align}
  H_{\t{3LD}} = \sum_j H_{\t{3LD},j}^{\t{single}},
  \label{eq:drive_all}
\end{align}
where $H_{\t{3LD},j}^{\t{single}}$ denotes the action of $H_{\t{3LD}}^{\t{single}}$ on spin $j$:
\begin{multline}
  H_{\t{3LD}}^{\t{single}}
  = \tilde\Omega_+ \tilde\Omega_- s_\z
  + \tilde\Omega_0 \tilde\Omega_- s_\x
  + \tilde\Omega_0 \tilde\Omega_+ \p{s_\z s_\x  + s_\x s_\z} \\
  - \tilde\Omega_0^2 s_\z^2 - \tilde\Omega_+^2 s_\x^2
  - \tilde\Omega_-^2 s_\y^2,
  \label{eq:drive_single}
\end{multline}
with
\begin{align}
  \tilde\Omega_0 \equiv -\f{\Omega_0}{\sqrt\Delta},
  &&
  \tilde\Omega_\pm \equiv \f{\Omega_+\pm\Omega_-}{\sqrt\Delta}.
\end{align}

\begin{table}
\centering
\caption{
Drive Hamiltonians (left column) that can be implemented with different amplitude-matching conditions (right column), some of which are specified by an arbitrary sign $\sigma\in\set{+1,-1}$.
The drives shown here are equal to that Eq.~\eqref{eq:drive_single} up to a possible energy shift of $s_\x^2+s_\y^2+s_\z^2=s(s+1)$, and come in mutually commuting pairs: a drive with $\Omega_m=1$ and $\Omega_n=0$ for both $n\ne m$ commutes with the drive in which $\Omega_m=0$ and both $\abs*{\Omega_n}=1$.
}
\vspace{.5em}
\begin{tabular}{c|c}
  $H_{\t{drive}}^{\t{single}}$
  & $(\tilde\Omega_0,\tilde\Omega_+,\tilde\Omega_-)$
  \\ \hline\hline
  $-s_\z^2$ & $\p{1,0,0}$
  \\ \hline
   $-s_\x^2$ & $\p{0,1,0}$
  \\ \hline
  $-s_\y^2$ & $\p{0,0,1}$
  \\ \hline
  $\sigma s_\z + s_\z^2$ & $\p{0,1,\sigma}$
  \\ \hline
  $\sigma s_\x + s_\x^2$ & $\p{1,0,\sigma}$
  \\ \hline
  $\sigma\p{s_\z s_\x+s_\x s_\z} + s_\y^2$ & $\p{1,\sigma,0}$
  \\ \hline
  $\pm s_\z \pm \sigma s_\x + \sigma \p{s_\z s_\x + s_\x s_\z}$
  & $\p{1,\sigma,\pm\sigma}$
\end{tabular}
\label{tab:drives}
\end{table}

Eqs.~\eqref{eq:drive_all} and \eqref{eq:drive_single} are the main technical results of this section.
There are three important observations to make about these results.
First, the fact that $H_{\t{3LD}}$ acts identically on all spins means we can freely replace the site index $j$ with a momentum index $q$ (as can be verified by substituting $c_{j\mu}=\frac1{\sqrt{L}}\sum_k e^{-\i q\c j} c_{q\mu}$), which is important to ensure that this drive addresses the same spin degrees of freedom as the spin Hamiltonian $H_{\t{spin}}$ in Eq.~\eqref{eq:H_spin}.
Second, each of $\tilde\Omega_0,\tilde\Omega_+,\tilde\Omega_-$ can be tuned independently by changing the amplitudes of the driving lasers; some particular Hamiltonians for specific values of these amplitudes are shown in Table \ref{tab:drives}.
Third, due to the appearance of mutually commuting pairs of Hamiltonians in Table \ref{tab:drives}, specifically $-s_\alpha^2$ and $\pm s_\alpha+s_\alpha^2$ for $\alpha\in\set{\z,\x}$, the three-laser drive admits pulse sequences that exactly implement arbitrary SU(2) (spatial) rotations of the form $e^{-\i\chi\vec n\c\vec s}$, where $\chi$ is a rotation angle, $\vec n$ is a rotation axis, and $\vec s\equiv(s_\x,s_\y,s_\z)$.
The capability to perform arbitrary spatial rotations, together with the capability to measure the number of atoms with spin projection $\mu$ onto a fixed quantization axis, $\bk{S_{\mu\mu}}$ (where $S_{\mu\nu}=\sum_js_{\mu\nu j}$), implies the capability to reconstruct all components of the mean collective spin matrix $\bk{\v S}=\sum_{\mu\nu}\bk{S_{\mu\nu}}\op{\mu}{\nu}$ via spin qudit tomography \cite{newton1968measurability, perlin2020qudit}.
Moreover, we expect that advanced quantum optimal control techniques (similar to those of Refs.~\cite{anderson2015accurate, lucarelli2018quantum}) can be used to implement arbitrary SU($n$) rotations by designing suitable time-dependent drive amplitudes $\Omega_m$.

Finally, we comment on the preparation of initial states.
Initial states are nominally prepared in the ``lab frame'', and must be transformed according to the gauge transformation $c_{q\mu}^\dag\to c_{q-\mu\phi,\mu}^\dag$ prior to evolution under the ``gauged frame'' spin Hamiltonian $H_{\t{spin}}$ in Eq.~\eqref{eq:H_spin} or the three-laser drive $H_{\t{3LD}}$ in Eq.~\eqref{eq:drive_all}.
We assume the capability to prepare an initial state in which all spins are maximally polarized along $\v v_+$, i.e.~$\ket{\v v_+}^{\otimes N}=\ket{s}^{\otimes N}$, which is unaffected by the gauge transformation (up to a global phase).
The three-laser then allows us to rotate this state into one that is polarized along any spatial axis (in the gauged frame).
In addition, when $n>2$ the three-laser drive allows us to prepare product states with nontrivial intra-spin correlations.
For example, when $n$ is even we can prepare an $N$-fold product of the state
\begin{align}
  \frac{\ket{s} + \ket{-s}}{\sqrt{2}}
  \simeq e^{-\i\frac{\pi}{2}\p{s_\y + s_\y^2}} \ket{s},
\end{align}
where $\simeq$ denotes equality up to a global phase.
This state has a vanishing mean spin vector, $\bk{s_\x}=\bk{s_\y}=\bk{s_\z}=0$, but variances $\bk{s_\x^2}=\bk{s_\y^2}=s/2$ and $\bk{s_\z^2}=s^2$.

%%%%%%%%%%%%%%%%%%%%%%%%%%%%%%%%%%%%%%%%%%%%%%%%%%%%%%%%%%%%%%%%%%%%%%
\section{Mean-field theory and dynamical phases}
\label{sec:mean_field}

We now return to the spin Hamiltonian $H_{\t{spin}}$ in Eq.~\eqref{eq:H_spin}, which is valid in the weak-SOC limit $s\phi\ll1$, and henceforth work exclusively in the ``gauged frame'' of $H_{\t{spin}}$ and the three-laser drive $H_{\t{3LD}}$ in Eq.~\eqref{eq:drive_all}.
At the mean-field (MF) level, up to constant energy shifts the undriven spin Hamiltonian becomes
\begin{align}
  H_\MF = \sum_q\sp{-U\bk{\bar{\v s}}\c\v s_q
    + J\phi \sum_q \sin\p{q} s_{\z,q}}.
  \label{eq:H_MF}
\end{align}
where $\bar{\v s}\equiv\frac1N\sum_q\v s_q$ is the average spin matrix.
We assume that all momenta $q\in\ZZ_N\times 2\pi/N$ are occupied.
Fixing the atom number $N$, the spin Hamiltonian has one free parameter: the reduced field $J\phi/U$ that determines the relative strength of the single-particle and interaction terms.
One should therefore expect distinct dynamical behaviors when $J\phi/U\ll1$, in which case strong spin-locking interactions should give rise to a long-range ordered phase, as opposed to $J\phi/U\gg1$, in which case long-range order should be destroyed by the strong inhomogeneous magnetic field.

To investigate these behaviors quantitatively, we examine time-averaged observables of the form
\begin{align}
  \bbk{\O}_\MF = \lim_{T\to\infty} \f1T \int_0^T \dd t \bk{\O\p{t}}_\MF,
\end{align}
where $\bk{\O\p{t}}_\MF$ is the mean-field value of observable $\O$ at time $t$ (see Appendices \ref{sec:MFT} and \ref{sec:bosons} for details about our mean-field simulations).
Specifically, we consider the time-averaged magnetization
\begin{align}
  m_\MF \equiv \abs{\bbk{\vec\sigma}_\MF},
  &&
  \vec\sigma \equiv \f1{Ns} \times \p{S_\x,S_\y,S_\z},
\end{align}
where $S_\alpha \equiv \sum_q s_{\alpha,q}$, and the time-averaged interaction energy
\begin{align}
  \bbk{\ss}_\MF = \f1{N^2} \times \bbk{\v S\c\v S}_\MF.
\end{align}
By design, these non-negative quantities are normalized to a maximal value of one, independent of the system size $N$ or spin dimension $n$.
In the remainder of this section we will assume that $n$ is even, both for the sake of experimental relevance (most relevant atomic nuclei are fermionic) and to avoid complications from parity effects.

\subsection{Spin-polarized initial state}

\begin{figure}
\centering
\includegraphics{mean_mag_int_X.pdf}
\caption{
Time-averaged magnetization $m_\MF$ and interaction energy $\bbk{\ss}_\MF$ for different spin dimensions $n$ (indicated in the legend) as determined by mean-field simulations of $N=100$ spins initially in the $x$-polarized state $\ket\X$ for a time $T=10^5/U$.
Insets show same data after normalizing $J\phi/U\to J\phi/U \times (n/2)^{1/3}$, as well as transforming $m_\MF$ and $\bbk{\ss}_\MF$ according to Eq.~\eqref{eq:rescale}.
}
\label{fig:mean_mag_int_X}
\end{figure}

Figure \ref{fig:mean_mag_int_X} shows the time-averages of the magnetization $m_\MF$ and interaction energy $\bbk{\ss}_\MF$ as computed by mean-field simulations of $N=100$ spins initially in the $x$-polarized state $\X\equiv\ket\x^{\otimes N}$, where
\begin{align}
  \ket\x \equiv e^{-\i\frac{\pi}{2}s_\y} \ket{s}
  = \f1{2^s} \sum_\mu { 2s \choose s+\mu }^{1/2} \ket{\mu}.
  \label{eq:state_x}
\end{align}
Here ${ m \choose k }$ is a binomial coefficient.
As expected, the spin model exhibits a mean-field dynamical phase transition between an ordered phase at small $J\phi/U$ and a disordered phase at large $J\phi/U$.
The ordered phase has a non-zero magnetization $m_\MF$ and an interaction energy $\bbk{\ss}_\MF$ that asymptotically approach their maximal values as $J\phi/U\to0^+$.
The disordered phase has no (time-averaged) magnetization, $m_\MF=0$, but the interaction energy $\bbk{\ss}_\MF$ nonetheless indicates persistent nontrivial inter-spin correlations when $n>2$.
These nontrivial correlations vanish as $J\phi/U\to\infty$, in which case $\bbk{\ss}_\MF$ approaches the minimal value allowed by conservation laws (clarified below).
By minimizing the reduced field $J\phi/U$ for which $m_\MF=0$, we numerically find that the transition between ordered and disordered phases occurs at a critical field $\p{J\phi/U}_{\t{crit}}=\p{n/2}^{-\alpha}$ with $\alpha\approx1/3$ (see Figure \ref{fig:crit_fields_X}).

\begin{figure}
\centering
\includegraphics{crit_fields_X.pdf}
\caption{
The critical value of $\p{J\phi/U}_{\t{crit}}$ as determined by mean-field simulations of $N=100$ spins initially in the $x$-polarized state $\ket\X$.
A single-parameter fit to $\p{J\phi/U}_{\t{crit}}=\p{n/2}^{-\alpha}$ finds $\alpha=0.333(5)$, and $\alpha=1/3$ is consistent with all mean-field results to within an uncertainty determined by the resolution of $J\phi/U$ in mean-field simulations.
}
\label{fig:crit_fields_X}
\end{figure}

As shown in insets of Figure \ref{fig:mean_mag_int_X}, mean-field results for different spin dimensions $n$ collapse onto each other when normalizing the field $J\phi/U$ to its critical value, $\p{J\phi/U}\to\p{J\phi/U}\times\p{n/2}^{1/3}$, and rescaling
\begin{align}
  m_\MF \to \f{m_\MF}{\gamma\p{n/2}},
  &&
  \bbk{\ss}_\MF \to \f{\bbk{\ss}_\MF-\gamma\p{n}}{1-\gamma\p{n}},
  \label{eq:rescale}
\end{align}
where
\begin{align}
  \gamma\p{k} \equiv \f{\Gamma\p{k-\frac12}}{\sqrt\pi\,\Gamma\p{k}}
  \stackrel{k\ge2}{\approx} \f1{\sqrt{\pi(k-1)}}.
  \label{eq:gamma}
\end{align}
The rescaling of magnetization and interaction energy can be understood by considering their limiting behavior as $J\phi/U\to\infty$ or $J\phi/U\to0^+$.

In the strong-field limit $J\phi/U\to\infty$, we can ignore interactions and treat spins as though they simply precess at different rates.
The time-averaged transverse magnetization $m_\MF$ then trivially vanishes as $J\phi/U\to\infty$.
The interaction energy $\bk{\ss}_\MF = \bk{\bar{\v s}}_\MF \c \bk{\bar{\v s}}_\MF$, meanwhile, has contributions from
\begin{enumerate*}
\item the diagonal parts of the mean spin matrix $\bk{\bar{\v s}}_\MF$, which are conserved by inhomogeneous spin precession, and
\item the off-diagonal parts of $\bk{\bar{\v s}}_\MF$, whose oscillations average to zero when evaluating the time average in $\bbk{\ss}_\MF$.
\end{enumerate*}
Altogether, the interaction energy $\bbk{\ss}_\MF$ in the strong-field limit is determined by the time-independent diagonal part $\diag\bk{\bar{\v s}}_\MF = \diag\op{\x}$, namely
\begin{align}
  \lim_{J\phi/U\to\infty} \bbk{\ss}_\MF
  = \Tr\sp{\p{\diag\op{\x}}^2}
  = \gamma\p{n}.
  \label{eq:lim_ss}
\end{align}
The same result can be obtained by computing the time-averaged interaction energy of two spins precessing at different rates.

In the weak-field limit $J\phi/U\to0^+$, the spin-locking $\v S\c\v S$ interactions of the spin Hamiltonian $H_{\t{spin}}$ energetically restrict dynamics to the permutationally-symmetric (PS) manifold.
At second order in perturbation theory, the effective spin Hamiltonian within the PS manifold becomes (see Appendix \ref{sec:pert_theory})
\begin{align}
  H_{\t{spin}}^{\t{eff}}
  = \f{\p{J\phi}^2}{\p{N-1}U}
  \times \sp{S_\z^2 - N\sum_q s_{\z,q}^2},
\end{align}
whose mean-field version is
\begin{align}
  H_{\t{MFT}}^{\t{eff}} = -\f{\p{J\phi}^2}{U} \sum_q s_{\z,q}^2,
\end{align}
where we have used the fact that the axial magnetizations $\bk{s_{\z,q}}=\frac1N\bk{S_\z}$ within the PS manifold, and $\bk{S_\z}=0$ is conserved by $H_{\t{spin}}$.
The weak-field effective Hamiltonian preserves permutational symmetry, so $\bbk{\ss}_\MF\to1$ as $J\phi/U\to0^+$.
Moreover, both the spin Hamiltonian $H_{\t{spin}}$ and the initial state $\ket\X$ are invariant (up to global phase) under the action of $R_\x \equiv e^{-\i\pi S_\x}\sum_{q>0}\v s_q\c\v s_{-q}$ (i.e.~a $\pi$ rotation about the $x$ axis followed by a permutation of spins $q\leftrightarrow -q$), so
\begin{align}
  \bk{S_\y} = \bk{R_\x S_\y R_\x^\dag} = -\bk{S_\y} = 0
  \label{eq:sym_y}
\end{align}
at all times.
Conservation of $\bk{S_\z}=\bk{S_\y}=0$ and permutational symmetry together imply that the magnetization $m_\MF$ is determined by the time-average of $\sigma_\x\equiv s_\x/s$ for a single spin:
\begin{align}
  \lim_{J\phi/U\to0^+} m_\MF
  = \abs{\lim_{T\to\infty} \f1T \int_0^T \dd\tau
  \bk{\x|\sigma_\x\p{\tau}|\x}},
\end{align}
where
\begin{align}
  \sigma_\x\p{\tau} = e^{\i\tau s_\z^2} \sigma_\x e^{-\i\tau s_\z^2}.
\end{align}
By mapping the Hilbert space of an $n$-level spin onto the PS (Dicke) manifold of $n-1$ spin-$\frac12$ particles, we can adapt exact analytical results for the Ising model \cite{foss-feig2013nonequilibrium}\footnote{See Appendix K of Ref.~\cite{perlin2020shorttime} for a simpler adaptation of the analytics in Ref.~\cite{foss-feig2013nonequilibrium} to the one-axis twisting model $H_{\t{OAT}}=\chi s_\z^2$.} to find that
\begin{align}
  \bk{\x|\sigma_\x\p{\tau}|\x} = \p{\cos\tau}^{n-2},
\end{align}
so for even $n$
\begin{align}
  \lim_{J\phi/U\to0^+} m_\MF
  = \f1{2\pi} \int_0^{2\pi} \dd\tau \p{\cos\tau}^{n-2}
  = \gamma\p{\f{n}{2}}.
\end{align}

\subsection{Multi-cat initial states}

We now consider the same setup as above, but with the initial ``multi-cat'' states $\ket\XX\equiv\ket\xx^{\otimes N}$ and $\ket\XXI\equiv\ket\xxi^{\otimes N}$, where
\begin{align}
  \ket\xx &\equiv \f{\ket{\x} + \ket{-\x}}{\sqrt{2}}
  \simeq e^{-\i\frac{\pi}{2}s_\y^2} \ket{-s}, \\
  \ket\xxi &\equiv \f{\ket{\x} + \p{-1}^s \ket{-\x}}{\sqrt{2}}
  \simeq e^{-\i\frac{\pi}{2}s_\x} \ket\xx.
\end{align}
Here $\simeq$ denotes equality up to a global phase, and $\ket{-\x}$ is a state polarized along $-x$, defined similarly to $\ket\x$ in Eq.~\eqref{eq:state_x}:
\begin{align}
  \ket{-\x} \equiv e^{-\i\frac{\pi}{2}s_\y}\ket{-s}
  = \f1{2^s} \sum_\mu \p{-1}^{s+\mu}
  { 2s \choose s+\mu }^{1/2} \ket{\mu}.
\end{align}

\begin{figure}
\centering
\includegraphics{mean_mag_int_XX.pdf}
\caption{
A near-replica of Figure \ref{fig:mean_mag_int_X} for the initial state $\ket\XX$.
The inset for interaction energy $\bbk{\ss}_\MF$ in Figure \ref{fig:mean_mag_int_X} subtracts off the minimal value of $\bbk{\ss}_\MF$ and rescales to lie on the interval $[0,1]$, as prescribed by Eq.~\eqref{eq:rescale}.
Here the subtracting and rescaling is identical, but with a minimal value of $2\gamma\p{n}$ rather than $\gamma\p{n}$.
}
\label{fig:mean_mag_int_XX}
\end{figure}

\begin{figure}
\centering
\includegraphics{mean_mag_int_XXI.pdf}
\caption{
A replica of Figure \ref{fig:mean_mag_int_X} for the initial state $\ket\XXI$.
Insets show the same data shifted and rescaled identically to Figure \ref{fig:mean_mag_int_X}.
}
\label{fig:mean_mag_int_XXI}
\end{figure}

Similarly to Figure \ref{fig:mean_mag_int_X}, Figures \ref{fig:mean_mag_int_XX} and \ref{fig:mean_mag_int_XXI} respectively show the time-averaged magnetization $m_\MF$ and interaction energy $\bbk{\ss}_\MF$ throughout mean-field dynamics of the initial states $\ket\XX$ and $\ket\XXI$.
These figures exclude the trivial case of spin dimension $n=2$, for which $\ket\xx=\ket{-s}$ is an eigenstate of $H_{\t{spin}}$ and $\ket\xxi=e^{-i\frac{\pi}{2}s_\z}\ket\x$ is still an equatorially-polarized state.
The first and perhaps most interesting observation to make about Figures \ref{fig:mean_mag_int_XX} and \ref{fig:mean_mag_int_XXI} is that they are different, signifying the importance of intra-spin coherences for the dynamical behavior of multilevel spin models.

Unlike the case of an initial spin-polarized state $\ket\X$, the mean-field dynamics of the initial multi-cat state $\ket\XX$ (Figure \ref{fig:mean_mag_int_XX}) exhibit no sharp transition between distinct dynamical phases: the time-averaged magnetization $m_\MF=0$ for all values of the field $J\phi/U$, and the interaction energy $\bbk{\ss}_\MF$ smoothly crosses over from a maximal value of 1 to a minimal value of $2\gamma\p{n}$.
The minimal value of $\bbk{\ss}_\MF$ approached as $J\phi/U\to\infty$ can be explained using arguments identical to those above Eq.~\eqref{eq:lim_ss}, which now imply that
\begin{align}
  \lim_{J\phi/U\to\infty} \bbk{\ss}_\MF
  = \Tr\sp{\p{\diag\op{\xx}}^2}
  = 2\gamma\p{n}.
\end{align}
The vanishing magnetization $m_\MF=0$ in Figure \ref{fig:mean_mag_int_XX} is protected by symmetries of $H_{\t{spin}}$ and $\ket\XX$.
For all initial states that we have considered, the value of $\bk{S_\z}=0$ is conserved by the spin Hamiltonian $H_{\t{spin}}$.
Moreover, as discussed above Eq.~\eqref{eq:sym_y}, the invariance of $H_{\t{spin}}$ and all initial states under the action of $R_\x\equiv e^{-\i\pi S_\x}\sum_{q>0}\v s_q\c\v s_{-q}$ implies that $\bk{S_\y}=0$ at all times.
The state $\ket\XX$ is additionally invariant under the action of $R_\y\equiv e^{-\i\pi S_\y}\sum_{q>0}\v s_q\c\v s_{-q}$, which analogously ensures that $\bk{S_\x}=0$, and thereby $m_\MF=0$.

Turning now to mean-field results for the initial multi-cat state $\ket\XXI$ in Figure \ref{fig:mean_mag_int_XXI}, we remark that the magnetization $m_\MF$ and interaction energy $\bbk{\ss}_\MF$ behave identically to those for the initial spin-polarized state $\ket\X$ in Figure \ref{fig:mean_mag_int_X}.




\bibliography{main.bib}
\onecolumngrid
\appendix

%%%%%%%%%%%%%%%%%%%%%%%%%%%%%%%%%%%%%%%%%%%%%%%%%%%%%%%%%%%%%%%%%%%%%%
\section{Perturbation theory for SU($n$) ferromagnets}
\label{sec:pert_theory}

\red{[to be completed]}

%%%%%%%%%%%%%%%%%%%%%%%%%%%%%%%%%%%%%%%%%%%%%%%%%%%%%%%%%%%%%%%%%%%%%%
\section{Restricting spin operators to the permutationally-symmetric manifold}
\label{sec:PS_ops}

Here we provide the restriction of a general $M$-body spin operator $\O$ to the permutationally-symmetric (PS) manifold of $N$ spins (each with $n$ internal states).
Denoting the projector onto the PS manifold by $\P$, our task is essentially to find the coefficients of the expansion
\begin{align}
  \P \O_M \P = \sum_{a,b\in\A_n\p{N}} \bk{a|\O_M|b} \op{a}{b},
\end{align}
where $\A_n\p{N}$ is the set of all ways to assign $N$ (identical) spins to $n$ (distinct) states, such that for any $a\in\A_n\p{N}$ the state $\ket{a}=\ket{a_1,a_2,\cdots,a_n}$ is labeled by the occupation number $a_\mu$ of state $\mu$, with $\sum_\mu a_\mu=N$.
Written out explicitly,
\begin{align}
  \ket{a} = \f1{\sqrt{\C\p{a}}}
  \sum_{\substack{\t{distinct}\\\t{permutations}\\\Pi\,\t{of}\,\tilde a}}
  \Pi\ket{\tilde a},
  &&
  \ket{\tilde a} \equiv \bigotimes_\mu \ket{\mu}^{\otimes a_\mu},
  &&
  \C\p{a} \equiv \f{\p{\sum_\mu a_\mu}!}{\prod_\nu a_\nu!}.
\end{align}
Here $\C\p{a}$ is a multinomial coefficient that counts the number of distinct ways to permute the tensor factors of the ``standard-ordered'' state $\ket{\tilde a}$, enforcing $\bk{a|a}=1$.
Using these states, with some combinatorics we can expand
\begin{align}
  \bk{a|\O_M|b} =
  \sum_{\substack{\alpha,\beta\in\A_n\p{M}\\\alpha\le a,\beta\le b}}
  \delta_{a-\alpha,b-\beta}
  \sqrt{\f{\C\p{\alpha}\C\p{a-\alpha}\C\p{\beta}\C\p{b-\beta}}
    {\C\p{a}\C\p{b}}}
  \, \bk{\alpha|\O_M|\beta},
  \label{eq:multi_body_eval}
\end{align}
where the restriction $\alpha\le a$ and the difference $a-\alpha$ are evaluated element-wise, i.e.~$\alpha\le a\implies\alpha_\mu\le a_\mu$ and $\p{a-\alpha}_\mu=a_\mu-\alpha_\mu$ for all $\mu$; and $\delta_{cd}=1$ if $c=d$ and zero otherwise.
We sum over both $\alpha$ and $\beta$ above merely to keep the expression symmetric with respect to transposition $\p{a,\alpha}\leftrightarrow\p{b,\beta}$; in practice, one can simply sum over $\alpha\in\A_n\p{M}$ and set $\beta=b-a+\alpha$, throwing out terms with any $\beta_\mu<0$.
Note that, by slight abuse of notation, the operator $\O_M$ on the left of \eqref{eq:multi_body_eval} acts on an arbitrary choice of $M$ spins (out of $N$), whereas the operator $\O_M$ on the right of \eqref{eq:multi_body_eval} is simply an $M$-spin operator, with matrix elements $\bk{\alpha|\O_M|\beta}$ evaluated with respect to the PS $M$-spin states $\ket\alpha,\ket\beta\in\A_n\p{M}$.

%%%%%%%%%%%%%%%%%%%%%%%%%%%%%%%%%%%%%%%%%%%%%%%%%%%%%%%%%%%%%%%%%%%%%%
\section{Mean-field theory}
\label{sec:MFT}

Here we describe the mean-field theory used to simulate the spin Hamiltonian
\begin{align}
  H_{\t{spin}} = -\f{U}{2N}\v S\c\v S + J\phi \sum_q \sin\p{q} s_{\z,q}
\end{align}
in Eq.~\eqref{eq:H_spin} of the main text.
We begin by decomposing individual spin operators into Schwinger bosons as $s_{\mu\nu q} = b_{\mu q}^\dag b_{\nu q}$, such that the spin Hamiltonian becomes
\begin{align}
  H_{\t{spin}} \to H_{\t{boson}}
  = -\f{U}{2N} \sum_{p,q,\mu,\nu}
  b_{\mu p}^\dag b_{\nu p} b_{\nu q}^\dag b_{\mu q}
  + J\phi \sum_{q,\mu} \sin\p{q} \mu\, b_{\mu q}^\dag b_{\mu q}.
\end{align}
The Heisenberg equations of motion for the Schwinger boson operators are (see Appendix \ref{sec:bosons})
\begin{align}
  \i \partial_t b_{\mu q}
  = -\f{U}{N} \sum_{\nu,p} b_{\nu p}^\dag b_{\mu p} b_{\nu q}
  + J\phi \sin\p{q} \mu\, b_{\mu q}.
\end{align}
Our mean-field theory then treats all boson operators in these equations of motion as complex numbers, $b_{\mu q}\to\bk{b_{\mu q}}_\MF$, with the initial value $\bk{b_{\mu q}\p{t=0}}_\MF$ equal to the initial amplitude of spin $q$ in state $\mu$.
Specifically, for an $N$-fold product state of the form $\ket\psi=\bigotimes_q\sum_\mu\psi_{\mu q}\ket{\mu}$ we set $\bk{b_{\mu q}\p{t=0}}_\MF = \psi_{\mu q}$.
For pure initial product states, this mean-field treatment of the boson operators $b_{\mu q}$ is mathematically equivalent to a mean-field treatment of the spin operators $s_{\mu\nu q}$, as in Eq.~\eqref{eq:H_MF}, but reduces the number of variables to keep track of by a factor of $\sim n$.

%%%%%%%%%%%%%%%%%%%%%%%%%%%%%%%%%%%%%%%%%%%%%%%%%%%%%%%%%%%%%%%%%%%%%%
\section{Schwinger boson equations of motion for quadratic spin Hamiltonians}
\label{sec:bosons}

Here we decompose a quadratic spin Hamiltonian into Schwinger bosons, and derive the equations of motion for the resulting boson operators.
We begin with a general spin Hamiltonian of the form
\begin{align}
  H = \sum_{\substack{\mu,\nu,\rho,\sigma\\j<k}}
  g^{\mu\nu j}_{\rho\sigma k} s_{\mu\nu j} s_{\rho\sigma k}
  + \sum_{\mu,\nu,j} \epsilon_{\mu\nu j} s_{\mu\nu j},
  \label{eq:quadratic_spin}
\end{align}
where $\mu,\nu$ index orthogonal states of an $n$-level spin; $j,k$ index one of $N$ spins; $g^{\mu\nu j}_{\rho\sigma k}$ and $\epsilon_{\mu\nu j}$ are scalars; and $s_{\mu\nu j}=\op{\mu}{\nu}_j$ is a transition operator for spin $j$.
Strictly speaking, Eq.~\eqref{eq:quadratic_spin} only defines the couplings $g^{\mu\nu j}_{\rho\sigma k}$ for $j<k$, so we enforce $g^{\mu\nu k}_{\rho\sigma j}=g^{\mu\nu j}_{\rho\sigma k}$ and $g^{\mu\nu j}_{\rho\sigma j}=0$ for completion.
Decomposing spin operators into Schwinger bosons as $s_{\mu\nu j}=b_{\mu j}^\dag b_{\nu j}$, where $b_{\nu j}$ a annihilates a boson of type $\nu$ on site $j$, we can write this Hamiltonian as
\begin{align}
  H = \sum_{\substack{\mu,\nu,\rho,\sigma\\j<k}}
  g^{\mu\nu j}_{\rho\sigma k}
  b_{\mu j}^\dag b_{\nu j} b_{\rho k}^\dag b_{\sigma k}
  + \sum_{\mu,\nu,j} \epsilon_{\mu\nu j} b_{\mu j}^\dag b_{\nu j}.
\end{align}
The Heisenberg equations of motion for the boson operators are then
\begin{align}
  \i \partial_t b_{\alpha\ell} = \sp{b_{\alpha\ell}, H}
  &= \sum_{\substack{\mu,\nu,\rho,\sigma\\j<k}}
  g^{\mu\nu j}_{\rho\sigma k}
  \sp{b_{\alpha\ell}, b_{\mu j}^\dag b_{\nu j} b_{\rho k}^\dag b_{\sigma k}}
  + \sum_{\mu,\nu,j} \epsilon_{\mu\nu j}
  \sp{b_{\alpha\ell}, b_{\mu j}^\dag b_{\nu j}} \\
  &= \sum_{\mu,\nu,\rho,\sigma,k} g^{\mu\nu\ell}_{\rho\sigma k}
  \sp{b_{\alpha\ell}, b_{\mu\ell}^\dag b_{\nu\ell}}
  b_{\rho k}^\dag b_{\sigma k}
  + \sum_{\mu,\nu} \epsilon_{\mu\nu\ell}
  \sp{b_{\alpha\ell}, b_{\mu\ell}^\dag b_{\nu\ell}} \\
  &= \sum_{\mu,\nu} \p{\sum_{\rho,\sigma,k}
    g^{\mu\nu\ell}_{\rho\sigma k} b_{\rho k}^\dag b_{\sigma k}
    + \epsilon_{\mu\nu\ell}}
  \sp{b_{\alpha\ell}, b_{\mu\ell}^\dag b_{\nu\ell}}
\end{align}
where
\begin{align}
  \sp{b_{\alpha\ell}, b_{\mu\ell}^\dag b_{\nu\ell}}
  = \delta_{\alpha\mu} \delta_{\alpha\nu} b_{\alpha\ell}
  + \delta_{\alpha\mu} \p{1-\delta_{\alpha\nu}} b_{\nu\ell}
  = \delta_{\alpha\mu} b_{\nu\ell},
\end{align}
so
\begin{align}
  \i \partial_t b_{\alpha\ell}
  = \sum_\nu \p{\sum_{\rho,\sigma,k}
    g^{\alpha\nu\ell}_{\rho\sigma k} b_{\rho k}^\dag b_{\sigma k}
    + \epsilon_{\alpha\nu\ell}} b_{\nu\ell}.
\end{align}
In the case of uniform SU($n$)-symmetric interactions of the form $\frac{g}{2}\v S\c\v S$ and a diagonal external field, we have
\begin{align}
  g^{\alpha\nu\ell}_{\rho\sigma k}
  = g \times \delta_{\alpha\sigma} \delta_{\nu\rho},
  &&
  \epsilon_{\alpha\nu\ell}
  = \epsilon_{\alpha\ell} \times \delta_{\alpha\nu}
\end{align}
so
\begin{align}
  \i \partial_t b_{\alpha\ell}
  = g \sum_{\nu,k} b_{\nu k}^\dag b_{\alpha k} b_{\nu\ell}
  + \epsilon_{\alpha\ell} b_{\alpha\ell}.
\end{align}

%%%%%%%%%%%%%%%%%%%%%%%%%%%%%%%%%%%%%%%%%%%%%%%%%%%%%%%%%%%%%%%%%%%%%%
\section{Lax vector analysis}

We start with the spin Hamiltonian
\begin{align}
  H_{\t{spin}}
  = -\f{U}{2N} \sum_{\mu,\nu} S_{\mu\nu} S_{\nu\mu}
  + J\phi \sum_q \sin\p{q} s_{\z,q},
\end{align}
where $S_{\mu\nu} = \sum_q s_{\mu\nu q}$.
The single-body operators involved have normalizations
\begin{align}
  \tr\p{s_{\mu\nu q}^\dag s_{\mu\nu q}} = 1
  &&
  \t{and}
  &&
  \tr\p{s_{\z,q}^\dag s_{\z,q}}
  = \sum_\mu \mu^2
  = \f1{12} (n+1) n (n-1)
  \equiv \eta^2.
\end{align}
The Lax formulation requires all single-body operators involved to have the same normalization, so we substitute $s_{\tilde\z,q}\equiv s_{\z,q}/\eta$ to expand
\begin{align}
  U^{-1} H_{\t{spin}}
  = -\f1{2N} \sum_{\mu,\nu} S_{\mu\nu} S_{\nu\mu}
  + \eta h \sum_q \sin\p{q} s_{\tilde\z,q},
  &&
  \t{where}
  &&
  h \equiv \f{J\phi}{U}.
\end{align}
In the thermodynamic limit $N\to\infty$, the ``normalized'' Lax vector $\vec\ell\p{z}$ associated with $H_{\t{spin}}$ has components
\begin{align}
  \ell_\alpha\p{z}
  = \lim_{N\to\infty} \f1N \sum_q \f{s_{\alpha,q}}{z-\sin q}
    + \delta_{\alpha,\tilde\z} \, \eta h,
\end{align}
where $\alpha$ indexes elements of basis $\set{s_\alpha}$ of self-adjoint generators of SU($n$), with normalization $\tr\p{s_\alpha^2}=1$.
Within the permutationally symmetric manifold, we can replace $s_{\alpha,q}\to\bar s_\alpha\equiv\frac1N\sum_q s_{\alpha,q}$ at the cost of $O(1/N)$ errors that vanish as $N\to\infty$, so
\begin{align}
  \ell_\alpha\p{z}
  = \I\p{z} \bar s_\alpha
  + \delta_{\alpha,\tilde\z} \, \eta h,
\end{align}
where
\begin{align}
  \I\p{z} \equiv \lim_{N\to\infty} \f1N \sum_q \f1{z-\sin\p{q}}
  = \f1{2\pi} \int_{-\pi}^\pi \f{\dd q}{z-\sin\p{q}}
  = \f1{\sqrt{z^2-1}} \times
  \begin{cases}
    +1 & \arg\p{z} \in (-\pi/2,\pi/2] \\
    -1 & \t{otherwise}
  \end{cases}
  &&
  \t{for}
  &&
  z \notin \sp{-1,1}.
\end{align}
The ``squared norm'' of the Lax vector is therefore
\begin{align}
  \vec\ell\p{z}^2
  = \sum_\alpha \ell_\alpha\p{z}^2
  = \I\p{z}^2 \sum_{\alpha\ne\tilde\z} \bar s_\alpha^2
  + \sp{\I\p{z} \bar s_{\tilde\z} + \eta h}^2,
\end{align}
where we can define $Q^2\equiv\sum_\alpha \bar s_\alpha^2$ to simplify
\begin{align}
  \vec\ell\p{z}^2
  = \I\p{z}^2 \p{Q^2 - \bar s_{\tilde\z}^2}
  + \sp{\I\p{z} \bar s_{\tilde\z} + \eta h}^2
  = \I\p{z}^2 Q^2 + \eta^2 h^2
  + 2 \I\p{z} \eta h \bar s_{\tilde\z}.
\end{align}
For initial states with $\bar s_\z=0$, we thus find that
\begin{align}
  \vec\ell\p{z}^2 = \f{Q^2}{z^2-1} + \eta^2 h^2,
\end{align}
which is zero when\footnote{Strictly speaking, the zeros in Eq.~\eqref{eq:lax_zeros} occur at values of $z$ at which $\I\p{z}$ is undefined.
We avoid this issue by analytically continuing $\I\p{z}^2$ to the interval $z\in(-1,1)$.}
\begin{align}
  z = \pm \sqrt{1 - \p{\f{Q}{\eta h}}^2}.
  \label{eq:lax_zeros}
\end{align}
These roots change character when $z=0$, suggesting that the critical field $h_{\t{crit}}$ separating dynamical phases satisfies
\begin{align}
  h_{\t{crit}}^2 = \f{Q^2}{\eta^2}.
\end{align}
For a permutationally symmetric state, up to vanishing $O(1/N)$ corrections we can expand
\begin{align}
  Q^2 = \sum_\alpha \bar s_\alpha^2
  = \sum_{\mu,\nu} \bar s_{\mu\nu} \bar s_{\nu\mu} - \f1n
  = 1 - \f1n
  = \f{n-1}{n},
\end{align}
which implies that
\begin{align}
  h_{\t{crit}}^2 = \f{n-1}{n} \times \f{12}{(n+1)n(n-1)}
  = \f{12}{n^2\p{n+1}}.
\end{align}
This Lax analysis correctly predicts that $h_{\t{crit}}=1$ when $n=2$, but otherwise predicts $h_{\t{crit}}\sim n^{-3/2}$, which is inconsistent with the finding that $h_{\t{crit}}\sim n^{-1/3}$ in mean-field simulations (see Figure \ref{fig:crit_fields_X} of the main text).

\end{document}

%%% Local Variables:
%%% mode: latex
%%% TeX-master: t
%%% End:
