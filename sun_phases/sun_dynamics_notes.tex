\documentclass[nofootinbib,notitlepage,11pt]{revtex4-2}

%%% linking references
\usepackage{hyperref}
\hypersetup{
  breaklinks=true,
  colorlinks=true,
  linkcolor=blue,
  filecolor=magenta,
  urlcolor=cyan,
}

%%% header / footer
\usepackage{fancyhdr} % easier header and footer management
\pagestyle{fancy} % page formatting style
\fancyhf{} % clear all header and footer text
\renewcommand{\headrulewidth}{0pt} % remove horizontal line in header
\usepackage{lastpage} % for referencing last page
\cfoot{\thepage~of \pageref{LastPage}} % "x of y" page labeling

%%% symbols, notations, etc.
\usepackage{physics,braket,bm,amssymb} % physics and math
\renewcommand{\t}{\text} % text in math mode
\newcommand{\f}[2]{\dfrac{#1}{#2}} % shorthand for fractions
\newcommand{\p}[1]{\left(#1\right)} % parenthesis
\renewcommand{\sp}[1]{\left[#1\right]} % square parenthesis
\renewcommand{\set}[1]{\left\{#1\right\}} % curly parenthesis
\newcommand{\bk}{\Braket} % shorthand for braket notation
\renewcommand{\v}{\bm} % bold vectors
\newcommand{\uv}[1]{\bm{\hat{#1}}} % unit vectors
\newcommand{\av}{\vec} % arrow vectors
\renewcommand{\c}{\cdot} % inner product
\renewcommand{\d}{\partial} % partial derivative
\renewcommand{\dd}{\text{d}} % for infinitesimals
\renewcommand{\i}{\mathrm{i}\mkern1mu} % imaginary unit

\usepackage{dsfont} % for identity operator
\newcommand{\1}{\mathds{1}}

\usepackage{mathtools} % for \coloneqq

\newcommand{\up}{\uparrow}
\newcommand{\dn}{\downarrow}

\newcommand{\x}{\text{x}}
\newcommand{\y}{\text{y}}
\newcommand{\z}{\text{z}}
\newcommand{\X}{\text{X}}
\newcommand{\DCS}{\text{DCS}}

\newcommand{\B}{\mathcal{B}}
\newcommand{\D}{\mathcal{D}}
\newcommand{\E}{\mathcal{E}}
\renewcommand{\H}{\mathcal{H}}
\newcommand{\I}{\mathcal{I}}
\newcommand{\J}{\mathcal{J}}
\newcommand{\M}{\mathcal{M}}
\newcommand{\N}{\mathcal{N}}
\renewcommand{\O}{\mathcal{O}}
\renewcommand{\P}{\mathcal{P}}
\newcommand{\Q}{\mathcal{Q}}
\newcommand{\R}{\mathcal{R}}
\newcommand{\T}{\mathcal{T}}
\renewcommand{\S}{\mathcal{S}}
\newcommand{\V}{\mathcal{V}}
\newcommand{\Z}{\mathcal{Z}}

\newcommand{\EE}{\mathbb{E}}
\newcommand{\RR}{\mathbb{R}}
\renewcommand{\SS}{\mathbb{S}}
\newcommand{\ZZ}{\mathbb{Z}}

\newcommand{\PS}{\text{PS}}
\newcommand{\col}{\underline}

\let\var\relax
\DeclareMathOperator{\var}{var}

\usepackage[inline]{enumitem} % in-line lists and \setlist{} (below)
\setlist[enumerate,1]{label={(\roman*)}} % default in-line numbering
\setlist{nolistsep} % more compact spacing between environments

%%% text markup
\usepackage{color} % text color
\newcommand{\red}[1]{{\color{red} #1}}

%%%%%%%%%%%%%%%%%%%%%%%%%%%%%%%%%%%%%%%%%%%%%%%%%%%%%%%%%%%%%%%%%%%%%%
\begin{document}
\thispagestyle{fancy}

% todo: find roots of lax vector

\title{Collective SU($n$) spin model dynamics}%
\author{Michael A. Perlin}%
\date{\today}

\maketitle

We first consider a general quadratic spin Hamiltonian of the form
\begin{align}
  H = \f1N \sum_{\substack{\mu,\nu,\rho,\sigma\\i<j}}
  h^{\mu\nu i}_{\rho\sigma j} s_{\mu\nu i} s_{\rho\sigma j}
  + \sum_{\mu,\nu,i} \epsilon_{\mu\nu i} s_{\mu\nu i},
  \label{eq:spin}
\end{align}
where $\mu,\nu\in\set{I,I-1,\cdots,-I}$ with $I\equiv\p{n-1}/2$ index orthogonal states of an $n$-level spin; $i,j\in\ZZ_N$ index one of $N$ spins; $h^{\mu\nu i}_{\rho\sigma j}$ and $\epsilon_{\mu\nu i}$ are scalars; and $s_{\mu\nu i}\coloneqq\op{\mu}{\nu}_i$ is a transition operator for spin $i$.
If interactions are SU($n$) symmetric then $h^{\mu\nu i}_{\rho\sigma k} = h_{ik} \delta_{\mu\sigma}\delta_{\nu\rho}$, in which case
\begin{align}
  H = \f1N \sum_{i<j}
  h_{ij} \Pi_{ij}
  + \sum_{\mu,\nu,i} \epsilon_{\mu\nu i} s_{\mu\nu i},
  &&
  \Pi_{ij} \equiv \sum_{\mu,\nu} s_{\mu\nu i} s_{\nu\mu j}.
  \label{eq:spin_sun}
\end{align}

%%%%%%%%%%%%%%%%%%%%%%%%%%%%%%%%%%%%%%%%%%%%%%%%%%%%%%%%%%%%%%%%%%%%%% 
\section{Schwinger boson representation}

We can write the Hamiltonian in \eqref{eq:spin} using Schwinger bosons as
\begin{align}
  H = \f1N \sum_{\substack{\mu,\nu,\rho,\sigma\\i<j}}
  h^{\mu\nu i}_{\rho\sigma j}
  b_{\mu i}^\dag b_{\nu i} b_{\rho j}^\dag b_{\sigma j}
  + \sum_{\mu,\nu,i} \epsilon_{\mu\nu i} b_{\mu i}^\dag b_{\nu i},
\end{align}
where $b_{\mu i}$ annihilates a boson of type $\mu$ on site $i$.
The Heisenberg equations of motion for these operators are\footnote{The Hamiltonian in \eqref{eq:spin} only defines $h^{\mu\nu j}_{\rho\sigma k}$ for $j<k$.
  To simplify expressions in the remainder of this work, we therefore additionally define $h^{\mu\nu k}_{\rho\sigma j}=h^{\rho\sigma j}_{\mu\nu k}$ and $h^{\mu\nu i}_{\rho\sigma i}=0$.}
\begin{align}
  \i \d_t b_{\alpha i} = \sp{b_{\alpha i}, H}
  &= \f1N \sum_{\substack{\mu,\nu,\rho,\sigma\\j<k}}
  h^{\mu\nu j}_{\rho\sigma k}
  \sp{b_{\alpha i}, b_{\mu j}^\dag b_{\nu j} b_{\rho k}^\dag b_{\sigma k}}
  + \sum_{\mu,\nu,j} \epsilon_{\mu\nu j}
  \sp{b_{\alpha i}, b_{\mu j}^\dag b_{\nu j}} \\
  &= \f1N \sum_{\mu,\nu,\rho,\sigma,k}
  h^{\mu\nu j}_{\rho\sigma k}
  \sp{b_{\alpha i}, b_{\mu i}^\dag b_{\nu i}} b_{\rho k}^\dag b_{\sigma k}
  + \sum_{\mu,\nu} \epsilon_{\mu\nu i}
  \sp{b_{\alpha i}, b_{\mu i}^\dag b_{\nu i}} \\
  &= \sum_{\mu,\nu} \p{\f1N \sum_{\rho,\sigma,k}
    h^{\mu\nu i}_{\rho\sigma k} b_{\rho k}^\dag b_{\sigma k}
    + \epsilon_{\mu\nu i}}
  \sp{b_{\alpha i}, b_{\mu i}^\dag b_{\nu i}}
\end{align}
where
\begin{align}
  \sp{b_{\alpha i}, b_{\mu i}^\dag b_{\nu i}}
  = \delta_{\mu\alpha} \delta_{\nu\alpha} b_{\alpha i}
  + \delta_{\mu\alpha} \p{1-\delta_{\nu\alpha}} b_{\nu i}
  = \delta_{\mu\alpha} b_{\nu i},
\end{align}
so
\begin{align}
  \i \d_t b_{\alpha i}
  = \sum_\nu \p{\f1N \sum_{\rho,\sigma,k}
    h^{\alpha\nu i}_{\rho\sigma k} b_{\rho k}^\dag b_{\sigma k}
    + \epsilon_{\alpha\nu i}} b_{\nu i}.
\end{align}

%%%%%%%%%%%%%%%%%%%%%%%%%%%%%%%%%%%%%%%%%%%%%%%%%%%%%%%%%%%%%%%%%%%%%%
\section{Initial states}

We can essentially prepare $N$-fold tensor products $\ket\Psi\equiv\ket\psi^{\otimes N}$ of three types of initial product states $\ket\psi$.
First and foremost, we can prepare states $\ket{\mu}$ of definite spin projection $\mu=\set{I,I-1,\cdots,-I}$ onto a quantization axis.
In addition, we can prepare arbitrary polarized states defined by the polar and azimuthal angles $\theta,\phi$:
\begin{align}
  \ket{\theta,\phi}
  \equiv e^{i\phi s_\z} e^{i\theta s_\y} \ket\up
  = \sum_\mu \zeta_\mu\p{\theta} e^{-\i\mu\phi} \ket\mu,
  \label{eq:polarized_state}
\end{align}
where $\ket\up\equiv\ket{I}$ is a state polarized along the $z$ axis, and\footnote{The expansion of $\ket{\theta,\phi}$ in \eqref{eq:polarized_state} and \eqref{eq:polarized_proj} is acquired by representing an $n$-level spin by $n-1$ qubits restricted to the (Dicke) manifold of permutationally symmetric states.}
\begin{align}
  \zeta_\mu\p{\theta} \equiv { 2I \choose I+\mu }^{1/2}
  \cos\p{\f{\theta}{2}}^{I+\mu} \sin\p{\f{\theta}{2}}^{I-\mu}.
  \label{eq:polarized_proj}
\end{align}
Finally, for even $n$ we can also prepare the following superposition of polarized states:
\begin{align}
  \ket{\DCS}
  \equiv \f1{\sqrt{2}}
  \p{e^{\i\gamma} \ket{\alpha,\beta}
    + e^{-\i\gamma} \ket{\alpha,-\beta}}
  = \sum_\mu \zeta_\mu\p{\alpha}
  \times \sqrt{2}\, \cos\p{\mu\beta-\gamma} \ket\mu
  \label{eq:double_state}
\end{align}
with
\begin{align}
  \alpha = \f{\pi}{2} + \arcsin\p{\f13},
  &&
  \beta = \f{\pi}{3},
  &&
  \gamma = -\f{2\pi}{3}\, I.
\end{align}

%%%%%%%%%%%%%%%%%%%%%%%%%%%%%%%%%%%%%%%%%%%%%%%%%%%%%%%%%%%%%%%%%%%%%%
\section{Non-interacting limit of a realistic 1D model}

We consider the single-particle Hamiltonian
\begin{align}
  H = -J \sum_{\mu,q} \cos\p{q + \mu\varphi} s_{\mu\mu q},
\end{align}
where $q$ is a quasi-momentum (in units with lattice spacing $a=1$); $\mu\in\set{I,I-1,\cdots,-I}$ is a nuclear spin projection onto a quantization axis, with $I\equiv\p{n-1}/2$ the total nuclear spin; $\varphi$ is a spin-orbit-coupling angle; and $J$ is essentially the single-particle bandwidth (proportional to the nearest-neighbor tunneling rate, $J/2$).
As dynamical order parameters, we consider the operators
\begin{align}
  s_{\mu\nu} \equiv \f1N \sum_q s_{\mu\nu q},
  &&
  s^2 \equiv \v s^\dag\cdot \v s
  = \f1{N^2} \sum_{\mu,\nu,p,q} s_{\mu\nu p} s_{\nu\mu q}.
\end{align}
In the Heisenberg picture, these operators at time $t$ are
\begin{align}
  s_{\mu\nu}\p{t} = \f1N \sum_q s_{\mu\nu q}
  \times e^{\i\eta_{\mu\nu q} J t},
  &&
  s^2\p{t} = \f1{N^2} \sum_{\mu,\nu,p,q} s_{\mu\nu p} s_{\nu\mu q}
  \times e^{\i \p{\eta_{\mu\nu p}-\eta_{\mu\nu q}} J t},
\end{align}
where
\begin{align}
  \eta_{\mu\nu q}
  \equiv -\sp{\cos\p{q+\mu\varphi} - \cos\p{q+\nu\varphi}}
  = 2 \sin\p{\varphi_{\mu\nu}^-} \sin\p{q+\varphi_{\mu\nu}^+},
  &&
  \varphi_{\mu\nu}^\pm \equiv \f{\mu\pm\nu}{2}\,\varphi.
\end{align}
In particular, we are interested in the expectation values
\begin{align}
  \bk{\Psi|s_{\mu\nu}\p{t}|\Psi}
  = \f1N \bk{\psi\op{\mu}{\nu}\psi} \sum_q e^{\i\eta_{\mu\nu q} J t}
  = \bk{\psi\op{\mu}{\nu}\psi} \chi_{\mu\nu}\p{\varphi},
\end{align}
and
\begin{align}
  \bk{\Psi|s^2\p{t}|\Psi}
  &= \f1{N^2} \sum_{\mu,\nu} \abs{\bk{\psi\op{\mu}{\nu}\psi}}^2
  \sum_{p\ne q} e^{\i\p{\eta_{\mu\nu p}-\eta_{\mu\nu q}} J t} + \f{n}{N} \\
  &= \sum_{\mu,\nu} \abs{\bk{\psi\op{\mu}{\nu}\psi}}^2
  \abs{\chi_{\mu\nu}\p{\varphi}}^2 + \f{n-1}{N},
\end{align}
where
\begin{align}
  \chi_{\mu\nu}\p{\varphi} \equiv \f1N \sum_q e^{\i\eta_{\mu\nu q} J t}
  = \f1N \sum_q e^{\i 2\sin\p{\varphi_{\mu\nu}^-} \sin\p{q+\varphi_{\mu\nu}^+} J t}.
\end{align}
As $N\to\infty$, this sum is well approximated by an integral:
\begin{align}
  \chi_{\mu\nu}\p{\varphi} \approx \f1{2\pi} \int_{-\pi}^\pi \dd\theta\,
  e^{\i 2\sin\p{\varphi_{\mu\nu}^-} \sin\p{\theta+\varphi_{\mu\nu}^+} J t}
  = \J_0\sp{2 \sin\p{\varphi_{\mu\nu}^-} J t},
\end{align}
where $\J_0$ is the zero-order Bessel function of the first kind.
To leading order in $N\gg1$, we thus find that
\begin{align}
  \bk{\Psi|s_{\mu\nu}\p{t}|\Psi}
  &\approx \bk{\psi\op{\mu}{\nu}\psi}
  \J_0\sp{2\sin\p{\varphi_{\mu\nu}^-}Jt},
  \\
  \bk{\Psi|s^2\p{t}|\Psi}
  &\approx \sum_{\mu,\nu} \abs{\bk{\psi\op{\mu}{\nu}\psi}}^2
  \J_0\sp{2\sin\p{\varphi_{\mu\nu}^-}Jt}^2.
  \label{eq:ops_time}
\end{align}

%%%%%%%%%%%%%%%%%%%%%%%%%%%%%%%%%%%%%%%%%%%%%%%%%%
\subsection{Limiting cases}

The constant (time independent) terms in \eqref{eq:ops_time} have $\mu=\nu$ (in which case all $\varphi_{\mu\nu}^-=0$), so up to $\O(1/N)$ corrections
\begin{align}
  \bk{\Psi|s_{\mu\nu}^{\t{const}}|\Psi}
  \approx \delta_{\mu\nu} \abs{\bk{\mu|\psi}}^2,
  &&
  \bk{\Psi|s^2_{\t{const}}|\Psi}
  \approx \sum_\mu \abs{\bk{\mu|\psi}}^4 \equiv r_\psi,
  \label{eq:ops_const}
\end{align}
where the quantity $r_\psi$ is sometimes called the inverse participation ratio of $\ket\psi$ with respect to the basis $\set{\ket\mu}$ of states with definite projection onto the $z$ axis.
Figure \ref{fig:limiting_vals} shows several values of $r_\psi$ for different $n$.

At ``short'' times $t$ for which $\sin\p{I\varphi}Jt\ll1$, we can Taylor expand the Bessel functions in \eqref{eq:ops_time} about $t=0$ to get
\begin{align}
  \bk{\Psi|s_{\mu\nu}\p{t}|\Psi}
  &\approx \bk{\psi\op{\mu}{\nu}\psi}
  \times \sp{1 - \sin\p{\varphi_{\mu\nu}^-}^2 \p{J t}^2}, \\
  \bk{\Psi|s^2\p{t}|\Psi}
  &\approx 1 - 2 \p{Jt}^2 \sum_{\mu,\nu}
  \sin\p{\varphi_{\mu\nu}^-}^2 \abs{\bk{\psi\op{\mu}{\nu}\psi}}^2.
\end{align}
In the limit of a small angle $\varphi$ for which $I\varphi\ll1$, we can further simplify
\begin{align}
  \bk{\Psi|s_{\mu\nu}\p{t}|\Psi}
  &\approx \bk{\psi\op{\mu}{\nu}\psi}
  \times \sp{1 - \p{\f{\mu-\nu}{2I}}^2
    \p{I\varphi J t}^2}, \\
  \bk{\Psi|s^2\p{t}|\Psi}
  &\approx 1 - \var_\psi\p{Z} \p{I\varphi J t}^2,
  \label{eq:small_time_angle}
\end{align}
where
\begin{align}
  \var_\psi\p{Z} \equiv \bk{\psi|Z^2|\psi} - \bk{\psi|Z|\psi}^2,
  &&
  Z \equiv \sum_\mu \f{\mu}{I} \op{\mu}.
\end{align}
Note that the result in \eqref{eq:small_time_angle} is valid for all $t\ll\p{I\varphi J}^{-1}$, which may be a long time given the small-angle assumption $I\varphi\ll1$.
Figure \ref{fig:limiting_vals} shows several values of $\var_\psi\p{Z}$ for different $n$.

\begin{figure}
  \centering
  \includegraphics{figures/oscillations/limiting_vals.pdf}
  \caption{Several values of $r_\psi$ and $\var_\psi\p{Z}$ for
    different $n$.  Computed with respect to the states
    $\ket\X\equiv\ket{\pi/2,0}$ and $\ket{\DCS}$, defined in
    \eqref{eq:polarized_state} and \eqref{eq:double_state}.}
  \label{fig:limiting_vals}
\end{figure}

%%%%%%%%%%%%%%%%%%%%%%%%%%%%%%%%%%%%%%%%%%%%%%%%%%%%%%%%%%%%%%%%%%%%%%
\section{Interacting regime of a realistic 1D model}

We now consider a realistic 1D model with interactions, in which the Hamiltonian is
\begin{align}
  H = -\f{U}{N} \v S^\dag\cdot \v S
  - J \sum_{\mu,q} \cos\p{q + \mu\varphi} s_{\mu\mu q}.
\end{align}
This Hamiltonian can be written in the simplified form
\begin{align}
  H = -\f{U}{N} \v S^\dag\cdot \v S
  - J \sum_q \p{\cos q\, w_{\varphi,q}^+ - \sin q\, w_{\varphi,q}^-},
\end{align}
where the diagonal single-spin operators
\begin{align}
  w_\varphi^+ \equiv \sum_\mu \cos\p{\mu\varphi} \op{\mu},
  &&
  w_\varphi^- \equiv \sum_\mu \sin\p{\mu\varphi} \op{\mu},
\end{align}
are respectively even ($+$) and odd ($-$) under spin inversion, $\mu\to-\mu$.

%%%%%%%%%%%%%%%%%%%%%%%%%%%%%%%%%%%%%%%%%%%%%%%%%%%%%%%%%%%%%%%%%%%%%%
\section{Nuclear spin tomography}

Here we discuss the task of performing full tomography on the average reduced single-particle density operator of a collection of nuclear spins.
As this task will be performed using collective measurements and homogeneous control fields, for ease of language we will consider the equivalent problem of performing tomography on a single nuclear spin.
We have essentially two ingredients at our disposal: projective measurements of spin onto a fixed quantization axis, and a three-laser drive that addresses nuclear spins via off-resonant coupling to an excited electronic state.
To be concrete, we can directly measure projectors $\op{\mu}$ onto states of definite spin projection $\mu$ onto the $z$ axis.
While the three-laser drive gives us access to a variety of nuclear spin Hamiltonians, most notably it allows us to implement arbitrary SU(2) rotations of the form $e^{-\i\v\theta\c\v S}$, with $\v S\equiv\p{S_\x,S_\y,S_\z}$ a vector of spin operators and $\v\theta\equiv\p{\theta_\x,\theta_\y,\theta_\z}$ an arbitrary rotation vector.
These rotations essentially allow us to measure projectors $\Pi_{\v v\mu} \equiv \op{\mu_{\v v}}$ onto states of definite spin projection $\mu$ onto an arbitrary quantization axis $\v v\in\SS_2/\ZZ_2$.

Question: given a collection of projectors $\Pi\equiv\set{\Pi_j}$ and measurement outcomes $M\equiv\set{M_j}$, with $M_j$ an empirical estimate of $\bk{\Pi_j}_\rho=\tr\p{\rho\Pi_j}$, how do we determine whether the measurement data $M$ is sufficient to reconstruct $\rho$?
Answer: reconstruction of $\rho$ is possible if $\Pi$ spans the entire space $\B_n$ of operators on the Hilbert space of an $n$-level spin.
Given an orthonormal basis $\set{Q_\alpha}$ of self-adjoint operators spanning $\B_n$, we can expand
\begin{align}
  \rho = \sum_\alpha \rho_\alpha Q_\alpha,
  &&
  \rho_\alpha \equiv \bk{Q_\alpha}_\rho.
\end{align}
If $\Pi$ spans $\B_n$, then we can find a set of real numbers $c_{\alpha j}\in\RR$ for which
\begin{align}
  Q_\alpha = \sum_j c_{\alpha j} \Pi_j.
\end{align}
An empirical estimate $\tilde\rho$ of $\rho$ is then\footnote{This estimate can be improved by defining $\tilde\rho$ as the least-squares fit to the set of linear equations $\tr\p{\tilde\rho\,\Pi_j} = M_j$, and performing maximum-likelihood corrections to $\tilde\rho$ \cite{smolin2012efficient}.}
\begin{align}
  \tilde\rho = \sum_\alpha \tilde\rho_\alpha Q_\alpha,
  &&
  \tilde\rho_\alpha \equiv \sum_j c_{\alpha j} M_j
  \approx \sum_j c_{\alpha j} \bk{\Pi_j}_\rho
  = \bk{Q_\alpha}_\rho
  = \rho_\alpha.
\end{align}
As it turns out, numerical experiments with $n\le20$ suggest that in order to perfurm full tomography of $\rho$, it suffices to measure all projectors $\Pi_{\v v\mu}$ onto states of definite spin $\mu$ along $2n-1$ different axes $\v v$.
The question remains: which axes should we choose?
As a first pass, we can simply choose a set of $2n-1$ random axes, $V\equiv\set{\v v}$, by uniformly sampling points on the sphere.
These random axes define a set of $\p{2n-1}\times n$ projectors $\Pi_V=\set{\Pi^V_j}$, where for shorthand we use a combined index $j=\p{\v v_j,\mu_j}$ that specifies both a measurement axis $\v v_j$ and an outcome $\mu_j$, and for clarity we index objects by $V$ to emphasize their dependence on the choice of measurement axes.
The projectors $\Pi_V$ can be used to construct $n^2$ linearly independent operators by constructing a matrix $B_V$ with entries
\begin{align}
  \sp{B_V}_{ij} = \tr\p{\Pi^V_i \Pi^V_j}.
\end{align}
Indexing the (normalized) eigenvectors and the corresponding eigenvalues of $B_V$ by $\alpha$, respectively $\v x^V_\alpha$ and $m^V_\alpha$, for all $\alpha$ with eigenvalue $m^V_\alpha\ne0$ we can then construct the normalized and linearly independent operators
\begin{align}
  Q^V_\alpha \equiv \f1{\sqrt{m^V_\alpha}} \sum_j x^V_{\alpha j} \Pi^V_j.
\end{align}
To summarize: randomly choosing a set $V$ of $2n-1$ axes will induce a set of measurement operators $\Pi_V$ from which we can construct a complete basis $\set{Q^V_\alpha}$ of operators for $\B_n$.
However, some choices of axes will be better than others: a good choice of axes $V$ will maximize the amount of information that the corresponding measurement data $M_V=\set{M^V_j}$ will contain about the expectation values $\bk{Q^V_\alpha}_\rho$ used to reconstruct $\rho$.
The amount of information that $M_V$ contains about $\bk{Q^V_\alpha}_\rho$ is determined, in part, by the magnitude of the eigenvalue $m^V_\alpha$: if $m^V_\alpha$ is small, then $M_V$ will generally contain little information about $\bk{Q^V_\alpha}_\rho$, and vice versa\footnote{Think about the limit $m^V_\alpha\to0$, in which case the fact that $Q^V_\alpha$ is normalized implies that the operator $\sum_j x^V_{\alpha j} \Pi^V_j$ must have a vanishing norm.
  A vanishing norm can only occur due to fine-tuned cancellations that make the expectation value of $\sum_j x^V_{\alpha j} \Pi^V_j$ difficult to estimate from noisy data on $\bk{\Pi^V_j}_\rho$.}.
Rather than choosing axes randomly, we can therefore choose a set of axes that minimizes some cost function $C\p{m_V}$ favoring large eigenvalues $m_V\equiv\set{m^V_\alpha}$.
A reasonable cost function that treats all eigenvalues on equal footing, for example, is
\begin{align}
  C_{\t{inv}}\p{m_V} \equiv \sum_\alpha \f1{m^V_\alpha},
\end{align}
where the sum is implicitly performed over the $n^2$ largest eigenvalues of $B_V$ (as all other eigenvalues are guaranteed to be zero).
Note that any information about $\rho$, obtained either from prior knowledge or preliminary measurement data, can be used to construct an even better choice of measurement axes, leading to the possibility of adaptive measurement protocols \cite{pereira2018adaptive} that are more efficient in terms of the number of measurements required to estimate $\rho$ to a fixed precision.
We leave the details of tailored and adaptive measurement protocols to future work.

\bibliography{sun_dynamics_notes.bib}

\end{document}

%%% Local Variables:
%%% mode: latex
%%% TeX-master: t
%%% End:
