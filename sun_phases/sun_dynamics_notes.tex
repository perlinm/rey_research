\documentclass[nofootinbib,notitlepage,11pt]{revtex4-2}

%%% linking references
\usepackage{hyperref}
\hypersetup{
  breaklinks=true,
  colorlinks=true,
  linkcolor=blue,
  filecolor=magenta,
  urlcolor=cyan,
}

%%% header / footer
\usepackage{fancyhdr} % easier header and footer management
\pagestyle{fancy} % page formatting style
\fancyhf{} % clear all header and footer text
\renewcommand{\headrulewidth}{0pt} % remove horizontal line in header
\usepackage{lastpage} % for referencing last page
\cfoot{\thepage~of \pageref{LastPage}} % "x of y" page labeling

%%% symbols, notations, etc.
\usepackage{physics,braket,bm,amssymb} % physics and math
\renewcommand{\t}{\text} % text in math mode
\newcommand{\f}[2]{\dfrac{#1}{#2}} % shorthand for fractions
\newcommand{\p}[1]{\left(#1\right)} % parenthesis
\renewcommand{\sp}[1]{\left[#1\right]} % square parenthesis
\renewcommand{\set}[1]{\left\{#1\right\}} % curly parenthesis
\newcommand{\bk}{\Braket} % shorthand for braket notation
\renewcommand{\v}{\bm} % bold vectors
\newcommand{\uv}[1]{\bm{\hat{#1}}} % unit vectors
\newcommand{\av}{\vec} % arrow vectors
\renewcommand{\c}{\cdot} % inner product
\renewcommand{\d}{\partial} % partial derivative
\renewcommand{\dd}{\text{d}} % for infinitesimals
\renewcommand{\i}{\mathrm{i}\mkern1mu} % imaginary unit

\usepackage{dsfont} % for identity operator
\newcommand{\1}{\mathds{1}}

\usepackage{mathtools} % for \coloneqq

\newcommand{\up}{\uparrow}
\newcommand{\dn}{\downarrow}

\newcommand{\x}{\text{x}}
\newcommand{\y}{\text{y}}
\newcommand{\z}{\text{z}}

\newcommand{\B}{\mathcal{B}}
\newcommand{\D}{\mathcal{D}}
\newcommand{\E}{\mathcal{E}}
\renewcommand{\H}{\mathcal{H}}
\newcommand{\I}{\mathcal{I}}
\newcommand{\J}{\mathcal{J}}
\newcommand{\M}{\mathcal{M}}
\newcommand{\N}{\mathcal{N}}
\renewcommand{\O}{\mathcal{O}}
\renewcommand{\P}{\mathcal{P}}
\newcommand{\Q}{\mathcal{Q}}
\newcommand{\R}{\mathcal{R}}
\newcommand{\T}{\mathcal{T}}
\renewcommand{\S}{\mathcal{S}}
\newcommand{\V}{\mathcal{V}}
\newcommand{\X}{\mathcal{X}}
\newcommand{\Z}{\mathcal{Z}}

\newcommand{\EE}{\mathbb{E}}
\renewcommand{\SS}{\mathbb{S}}
\newcommand{\ZZ}{\mathbb{Z}}

\newcommand{\PS}{\text{PS}}
\newcommand{\col}{\underline}

\usepackage[inline]{enumitem} % in-line lists and \setlist{} (below)
\setlist[enumerate,1]{label={(\roman*)}} % default in-line numbering
\setlist{nolistsep} % more compact spacing between environments

%%% text markup
\usepackage{color} % text color
\newcommand{\red}[1]{{\color{red} #1}}

%%%%%%%%%%%%%%%%%%%%%%%%%%%%%%%%%%%%%%%%%%%%%%%%%%%%%%%%%%%%%%%%%%%%%%
\begin{document}
\thispagestyle{fancy}

% todo: find roots of lax vector

\title{Collective SU($n$) spin model dynamics}%
\author{Michael A. Perlin}%
\date{\today}

\maketitle

We first consider a general quadratic spin Hamiltonian of the form
\begin{align}
  H_{\t{spin}} = \f1N \sum_{\substack{\mu,\nu,\rho,\sigma\\i<j}}
  h^{\mu\nu i}_{\rho\sigma j} s_{\mu\nu i} s_{\rho\sigma j}
  + \sum_{\mu,\nu,i} \epsilon_{\mu\nu i} s_{\mu\nu i},
  \label{eq:spin}
\end{align}
where $\mu,\nu\in\ZZ_n$ index orthogonal states of a single spin;
$i,j\in\ZZ_N$ index one of $N$ spins; $h^{\mu\nu i}_{\rho\sigma j}$
and $\epsilon_{\mu\nu i}$ are scalars; and
$s_{\mu\nu i}\coloneqq\op{\mu}{\nu}_i$ is a transition operator for
spin $i$.  If interactions are SU($n$) symmetric then
$h^{\mu\nu i}_{\rho\sigma k} = h_{ik}
\delta_{\mu\sigma}\delta_{\nu\rho}$, in which case
\begin{align}
  H_{\t{spin}} = \f1N \sum_{\substack{\mu,\nu,\rho,\sigma\\i<j}}
  h_{ij} \Pi_{ij}
  + \sum_{\mu,\nu,i} \epsilon_{\mu\nu i} s_{\mu\nu i},
  &&
  \Pi_{ij} \equiv \sum_{\mu,\nu} s_{\mu\nu i} s_{\rho\sigma j}.
  \label{eq:spin_sun}
\end{align}

%%%%%%%%%%%%%%%%%%%%%%%%%%%%%%%%%%%%%%%%%%%%%%%%%%%%%%%%%%%%%%%%%%%%%% 
\section{Schwinger boson representation}

We can write the Hamiltonian in \eqref{eq:spin} using Schwinger bosons
as
\begin{align}
  H_{\t{boson}} = \f1N \sum_{\substack{\mu,\nu,\rho,\sigma\\i<j}}
  h^{\mu\nu i}_{\rho\sigma j}
  b_{\mu i}^\dag b_{\nu i} b_{\rho j}^\dag b_{\sigma j}
  + \sum_{\mu,\nu,i} \epsilon_{\mu\nu i} b_{\mu i}^\dag b_{\nu i},
\end{align}
where $b_{\mu i}$ annihilates a boson of type $\mu$ on site $i$.  The
Heisenberg equations of motion for these operators are\footnote{The
  Hamiltonian in \eqref{eq:spin} only defines
  $h^{\mu\nu j}_{\rho\sigma k}$ for $j<k$.  To simplify expressions in
  the remainder of this work, we therefore additionally define
  $h^{\mu\nu k}_{\rho\sigma j}=h^{\rho\sigma j}_{\mu\nu k}$ and
  $h^{\mu\nu i}_{\rho\sigma i}=0$.}
\begin{align}
  \i \d_t b_{\alpha i} = \sp{b_{\alpha i}, H_{\t{boson}}}
  &= \f1N \sum_{\substack{\mu,\nu,\rho,\sigma\\j<k}}
  h^{\mu\nu j}_{\rho\sigma k}
  \sp{b_{\alpha i}, b_{\mu j}^\dag b_{\nu j} b_{\rho k}^\dag b_{\sigma k}}
  + \sum_{\mu,\nu,j} \epsilon_{\mu\nu j}
  \sp{b_{\alpha i}, b_{\mu j}^\dag b_{\nu j}} \\
  &= \f1N \sum_{\mu,\nu,\rho,\sigma,k}
  h^{\mu\nu j}_{\rho\sigma k}
  \sp{b_{\alpha i}, b_{\mu i}^\dag b_{\nu i}} b_{\rho k}^\dag b_{\sigma k}
  + \sum_{\mu,\nu} \epsilon_{\mu\nu i}
  \sp{b_{\alpha i}, b_{\mu i}^\dag b_{\nu i}} \\
  &= \sum_{\mu,\nu} \p{\f1N \sum_{\rho,\sigma,k}
    h^{\mu\nu i}_{\rho\sigma k} b_{\rho k}^\dag b_{\sigma k}
    + \epsilon_{\mu\nu i}}
  \sp{b_{\alpha i}, b_{\mu i}^\dag b_{\nu i}}
\end{align}
where
\begin{align}
  \sp{b_{\alpha i}, b_{\mu i}^\dag b_{\nu i}}
  = \delta_{\mu\alpha} \delta_{\nu\alpha} b_{\alpha i}
  + \delta_{\mu\alpha} \p{1-\delta_{\nu\alpha}} b_{\nu i}
  = \delta_{\mu\alpha} b_{\nu i},
\end{align}
so
\begin{align}
  \i \d_t b_{\alpha i}
  = \sum_\nu \p{\f1N \sum_{\rho,\sigma,k}
    h^{\alpha\nu i}_{\rho\sigma k} b_{\rho k}^\dag b_{\sigma k}
    + \epsilon_{\alpha\nu i}} b_{\nu i}.
\end{align}

%%%%%%%%%%%%%%%%%%%%%%%%%%%%%%%%%%%%%%%%%%%%%%%%%%%%%%%%%%%%%%%%%%%%%%
\section{Non-interacting limit of a realistic 1D model}

Consider the single-particle Hamiltonian
\begin{align}
  H = -\f{J}{2} \sum_{\mu,q} \cos\p{q + \mu\phi} s_{\mu\mu q},
\end{align}
where $q$ is a quasi-momentum (in units with lattice spacing $a=1$);
$\mu\in\set{-I,-I+1,\cdots,I}$ is a nuclear spin projection onto a
quantization axis, with $I\equiv\p{n-1}/2$ the total nuclear spin;
$\phi$ is a spin-orbit-coupling angle, and $J$ is the single-particle
bandwidth (proportional to the tunneling rate, $J/4$).  We wish to
compute an expectation value of the operator
\begin{align}
  S^2\p{t}
  \equiv e^{it H} \p{\v S^\dag \c\v S} e^{-it H}
  = \sum_{\mu,\nu,p,q} s_{\mu\nu p} s_{\nu\mu q} \times
  e^{itJ\p{\eta_{\mu\nu p}-\eta_{\mu\nu q}}},
\end{align}
where
\begin{align}
  \eta_{\mu\nu p}
  \equiv -\f12 \sp{\cos\p{p+\mu\phi} - \cos\p{p+\nu\phi}}
  = \sin\p{\phi_{\mu\nu}^-} \sin\p{p+\phi_{\mu\nu}^+},
  &&
  \phi_{\mu\nu}^\pm \equiv \f{\mu\pm\nu}{2}\,\phi.
\end{align}
If $\ket\Psi=\ket\psi^{\otimes N}$, then
\begin{align}
  \bk{\Psi|S^2\p{t}|\Psi}
  &= \sum_{\mu,\nu} \abs{\bk{\psi\op{\mu}{\nu}\psi}}^2
  \sum_{p\ne q} e^{itJ\p{\eta_{\mu\nu p}-\eta_{\mu\nu q}}} + n N \\
  &= \sum_{\mu,\nu} \abs{\bk{\psi\op{\mu}{\nu}\psi}}^2
  \sum_{p,q} e^{itJ\p{\eta_{\mu\nu p}-\eta_{\mu\nu q}}} + \p{n-1} N \\
  &= N^2 \sum_{\mu,\nu} \abs{\bk{\psi\op{\mu}{\nu}\psi}}^2
  \abs{\xi_{\mu\nu}\p{\phi}}^2 + \p{n-1} N,
\end{align}
where
\begin{align}
  \xi_{\mu\nu}\p{\phi} \equiv \f1N \sum_q e^{itJ\eta_{\mu\nu q}}
  = \f1N \sum_q e^{itJ\sin\p{\phi_{\mu\nu}^-} \sin\p{q+\phi_{\mu\nu}^+}}.
\end{align}
As $N\to\infty$, this sum is well approximated by an integral:
\begin{align}
  \xi_{\mu\nu}\p{\phi} \approx \f1{2\pi} \int_{-\pi}^\pi \dd\theta\,
  e^{itJ\sin\p{\phi_{\mu\nu}^-} \sin\p{\theta+\phi_{\mu\nu}^+}}
  = \J_0\sp{tJ\sin\p{\phi_{\mu\nu}^-}},
\end{align}
where $\J_0$ is the zero-order Bessel function of the first kind.  To
leading order in $N\gg1$, we thus find that
\begin{align}
  \bk{\Psi|S^2\p{t}|\Psi}
  \approx N^2 \sum_{\mu,\nu} \abs{\bk{\psi\op{\mu}{\nu}\psi}}^2
  \J_0\sp{tJ\sin\p{\phi_{\mu\nu}^-}}^2.
\end{align}
The time-independent terms in the sum have $\mu=\nu$ (in which case
$\phi_{\mu\nu}^-=0$), so
\begin{align}
  \bk{\Psi|S^2_{\t{const}}|\Psi}
  \approx N^2 \sum_{\mu,\mu} \abs{\bk{\psi\op{\mu}{\mu}\psi}}^2
  = N^2 r,
  &&
  r \equiv \sum_\mu \abs{\bk{\mu|\psi}}^4.
  \label{eq:S^2_const}
\end{align}
Up to an inconsequential choice of gauge, we can prepare a state of
the form
\begin{align}
  \ket\psi = \f1{\sqrt{2}}
  \p{e^{i\beta} \ket{\theta,\alpha}
    + e^{-i\beta} \ket{\theta,-\alpha}},
  &&
  \ket{\theta,\alpha} \equiv e^{i\alpha s_\z} e^{i\theta s_\y} \ket\up,
\end{align}
where $\theta,\alpha$ are respectively polar and azimuthal angles of
the polarized state $\ket{\theta,\alpha}$ of a single spin; the spin
operators $s_\z,s_\y$ generate rotations about the $z,y$ axes, with
e.g.~$s_\z\equiv\sum_\mu \mu\op\mu$; and $\ket\up$ is a state
polarized along the $z$ axis.  More specifically, the state that we
can prepare has
\begin{align}
  \theta = \f{\pi}{2} + \arcsin\p{\f13},
  &&
  \alpha = \f{\pi}{3},
  &&
  \beta = -\f{2\pi}{3}\, I.
\end{align}
In order to evaluate \eqref{eq:S^2_const}, we therefore
expand\footnote{The expansion in \eqref{eq:polarized_proj} is acquired
  by representing an $n$-level spin by $n-1$ qubits restricted to the
  (Dicke) manifold of permutationally symmetric states.}
\begin{align}
  \bk{\mu|\theta,\alpha} = { 2I \choose I+\mu }^{1/2}
  \cos\p{\f{\theta}{2}}^{I+\mu} \sin\p{\f{\theta}{2}}^{I-\mu}
  e^{-i\mu\alpha},
  \label{eq:polarized_proj}
\end{align}
which implies that
\begin{align}
  r = \sum_\mu { 2I \choose I+\mu }^2
  \sp{\cos\p{\f{\theta}{2}}^{I+\mu} \sin\p{\f{\theta}{2}}^{I-\mu}}^4
  \times 4 \cos\p{\mu\alpha-\beta}^4.
\end{align}
Figure \ref{fig:const_vals} shows several values of $r$ for different
$n$.

\begin{figure}
  \centering
  \includegraphics{figures/oscillations/const_vals.pdf}
  \caption{Several values of
    $r \equiv\sum_\mu\abs{\bk{\mu|\psi}}^4 =
    \lim_{N\to\infty}\bk{\Psi|S^2_{\t{const}}|\Psi}/N^2$ for different
    $n$.}
  \label{fig:const_vals}
\end{figure}

\end{document}

%%% Local Variables:
%%% mode: latex
%%% TeX-master: t
%%% End:
