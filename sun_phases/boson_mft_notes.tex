\documentclass[nofootinbib,notitlepage,11pt]{revtex4-2}

%%% linking references
\usepackage{hyperref}
\hypersetup{
  breaklinks=true,
  colorlinks=true,
  linkcolor=blue,
  filecolor=magenta,
  urlcolor=cyan,
}

%%% header / footer
\usepackage{fancyhdr} % easier header and footer management
\pagestyle{fancy} % page formatting style
\fancyhf{} % clear all header and footer text
\renewcommand{\headrulewidth}{0pt} % remove horizontal line in header
\usepackage{lastpage} % for referencing last page
\cfoot{\thepage~of \pageref{LastPage}} % "x of y" page labeling

%%% symbols, notations, etc.
\usepackage{physics,braket,bm,amssymb} % physics and math
\renewcommand{\t}{\text} % text in math mode
\newcommand{\f}[2]{\dfrac{#1}{#2}} % shorthand for fractions
\newcommand{\p}[1]{\left(#1\right)} % parenthesis
\renewcommand{\sp}[1]{\left[#1\right]} % square parenthesis
\renewcommand{\set}[1]{\left\{#1\right\}} % curly parenthesis
\newcommand{\bk}{\Braket} % shorthand for braket notation
\renewcommand{\v}{\bm} % bold vectors
\newcommand{\uv}[1]{\bm{\hat{#1}}} % unit vectors
\newcommand{\av}{\vec} % arrow vectors
\renewcommand{\c}{\cdot} % inner product
\renewcommand{\d}{\partial} % partial derivative
\renewcommand{\dd}{\text{d}} % for infinitesimals
\renewcommand{\i}{\mathrm{i}\mkern1mu} % imaginary unit

\usepackage{dsfont} % for identity operator
\newcommand{\1}{\mathds{1}}

\newcommand{\up}{\uparrow}
\newcommand{\dn}{\downarrow}

\newcommand{\x}{\text{x}}
\newcommand{\y}{\text{y}}
\newcommand{\z}{\text{z}}

\newcommand{\B}{\mathcal{B}}
\newcommand{\D}{\mathcal{D}}
\newcommand{\E}{\mathcal{E}}
\renewcommand{\H}{\mathcal{H}}
\newcommand{\I}{\mathcal{I}}
\newcommand{\J}{\mathcal{J}}
\newcommand{\M}{\mathcal{M}}
\newcommand{\N}{\mathcal{N}}
\renewcommand{\O}{\mathcal{O}}
\renewcommand{\P}{\mathcal{P}}
\newcommand{\Q}{\mathcal{Q}}
\newcommand{\R}{\mathcal{R}}
\newcommand{\T}{\mathcal{T}}
\renewcommand{\S}{\mathcal{S}}
\newcommand{\V}{\mathcal{V}}
\newcommand{\X}{\mathcal{X}}
\newcommand{\Z}{\mathcal{Z}}

\newcommand{\EE}{\mathbb{E}}
\renewcommand{\SS}{\mathbb{S}}
\newcommand{\ZZ}{\mathbb{Z}}

\newcommand{\PS}{\text{PS}}
\newcommand{\col}{\underline}

\usepackage[inline]{enumitem} % in-line lists and \setlist{} (below)
\setlist[enumerate,1]{label={(\roman*)}} % default in-line numbering
\setlist{nolistsep} % more compact spacing between environments

%%% text markup
\usepackage{color} % text color
\newcommand{\red}[1]{{\color{red} #1}}

%%%%%%%%%%%%%%%%%%%%%%%%%%%%%%%%%%%%%%%%%%%%%%%%%%%%%%%%%%%%%%%%%%%%%%
\begin{document}
\thispagestyle{fancy}

\title{Schwinger boson MFT for an SU($n$) spin model}%
\author{Michael A. Perlin}%
\date{\today}

\maketitle

Suppose we have a spin Hamiltonian of the form
\begin{align}
  H_{\t{spin}} = \v S^\dag\c\v S
  + \sum_{\mu,\nu,i} \epsilon_{\mu\nu i} \op{\mu}{\nu}_i
\end{align}
where $\v S$ is a vector of collective SU($n$) operators;
$\mu,\nu\in\ZZ_n$ index orthogonal states of an individual spin;
$i\in\ZZ_N$ indexes one of $N$ spins; $\epsilon_{\mu\nu i}$ is a
scalar; and $\op{\mu}{\nu}_i$ is a transition operator for spin $i$.
We can write this Hamiltonian in terms of Schwinger bosons as
\begin{align}
  H = \sum_{\substack{\mu,\nu\\i<j}}
  a_{\mu i}^\dag a_{\nu i} a_{\nu j}^\dag a_{\mu j}
  + \sum_{\mu,\nu,i} \epsilon_{\mu\nu i} a_{\mu i}^\dag a_{\nu i},
\end{align}
where $a_{\mu i}$ is a bosonic annihilation operator that satisfies
$a_{\mu i}^2=0$.  The Heisenberg equations of motion for these
operators are
\begin{align}
  \i \d_t a_{\mu i}
  = \sp{a_{\mu i},H}
  &= \sum_{\substack{\rho,\sigma\\j<k}}
  \sp{a_{\mu i}, a_{\rho j}^\dag a_{\sigma j}
    a_{\sigma k}^\dag a_{\rho k}}
  + \sum_{\rho,\sigma,j} \epsilon_{\rho\sigma j}
  \sp{a_{\mu i}, a_{\rho j}^\dag a_{\sigma j}} \\
  &= \sum_{\substack{\rho,\sigma\\j\ne i}} a_{\rho j}^\dag a_{\sigma j}
  \sp{a_{\mu i}, a_{\sigma i}^\dag a_{\rho i}}
  + \sum_{\rho,\sigma} \epsilon_{\rho\sigma i}
  \sp{a_{\mu i}, a_{\rho i}^\dag a_{\sigma i}},
\end{align}
where, for $j\ne i$,
\begin{align}
  \sum_{\rho,\sigma} a_{\rho j}^\dag a_{\sigma j}
  \sp{a_{\mu i}, a_{\sigma i}^\dag a_{\rho i}}
  = a_{\mu j}^\dag a_{\mu j}
  \sp{a_{\mu i}, a_{\mu i}^\dag a_{\mu i}}
  + \sum_{\rho\ne\sigma} a_{\rho j}^\dag a_{\sigma j}
  \sp{a_{\mu i}, a_{\sigma i}^\dag a_{\rho i}}
  = a_{\mu j}^\dag a_{\mu j}
  + \sum_{\rho\ne\mu} a_{\rho j}^\dag a_{\mu j} a_{\rho i},
\end{align}
and
\begin{align}
  \sum_{\rho,\sigma} \epsilon_{\rho\sigma i}
  \sp{a_{\mu i}, a_{\rho i}^\dag a_{\sigma i}}
  = \epsilon_{\mu\mu i}
  \sp{a_{\mu i}, a_{\mu i}^\dag a_{\mu i}}
  + \sum_{\rho\ne\sigma} \epsilon_{\rho\sigma i}
  \sp{a_{\mu i}, a_{\rho i}^\dag a_{\sigma i}}
  = \epsilon_{\mu\mu i}
  + \sum_{\sigma\ne\mu} \epsilon_{\mu\sigma i} a_{\sigma i},
\end{align}
so
\begin{align}
  \i \d_t a_{\mu i}
  = \sum_{j\ne i} \p{a_{\mu j}^\dag a_{\mu j}
  + \sum_{\nu\ne\mu} a_{\nu j}^\dag a_{\mu j} a_{\nu i}}
  + \epsilon_{\mu\mu i}
  + \sum_{\nu\ne\mu} \epsilon_{\mu\nu i} a_{\nu i}.
\end{align}


% todo: find roots of lax vector (numerically)

\end{document}

%%% Local Variables:
%%% mode: latex
%%% TeX-master: t
%%% End:
