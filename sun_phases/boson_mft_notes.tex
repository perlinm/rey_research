\documentclass[nofootinbib,notitlepage,11pt]{revtex4-2}

%%% linking references
\usepackage{hyperref}
\hypersetup{
  breaklinks=true,
  colorlinks=true,
  linkcolor=blue,
  filecolor=magenta,
  urlcolor=cyan,
}

%%% header / footer
\usepackage{fancyhdr} % easier header and footer management
\pagestyle{fancy} % page formatting style
\fancyhf{} % clear all header and footer text
\renewcommand{\headrulewidth}{0pt} % remove horizontal line in header
\usepackage{lastpage} % for referencing last page
\cfoot{\thepage~of \pageref{LastPage}} % "x of y" page labeling

%%% symbols, notations, etc.
\usepackage{physics,braket,bm,amssymb} % physics and math
\renewcommand{\t}{\text} % text in math mode
\newcommand{\f}[2]{\dfrac{#1}{#2}} % shorthand for fractions
\newcommand{\p}[1]{\left(#1\right)} % parenthesis
\renewcommand{\sp}[1]{\left[#1\right]} % square parenthesis
\renewcommand{\set}[1]{\left\{#1\right\}} % curly parenthesis
\newcommand{\bk}{\Braket} % shorthand for braket notation
\renewcommand{\v}{\bm} % bold vectors
\newcommand{\uv}[1]{\bm{\hat{#1}}} % unit vectors
\newcommand{\av}{\vec} % arrow vectors
\renewcommand{\c}{\cdot} % inner product
\renewcommand{\d}{\partial} % partial derivative
\renewcommand{\dd}{\text{d}} % for infinitesimals
\renewcommand{\i}{\mathrm{i}\mkern1mu} % imaginary unit

\usepackage{dsfont} % for identity operator
\newcommand{\1}{\mathds{1}}

\usepackage{mathtools} % for \coloneqq

\newcommand{\up}{\uparrow}
\newcommand{\dn}{\downarrow}

\newcommand{\x}{\text{x}}
\newcommand{\y}{\text{y}}
\newcommand{\z}{\text{z}}

\newcommand{\B}{\mathcal{B}}
\newcommand{\D}{\mathcal{D}}
\newcommand{\E}{\mathcal{E}}
\renewcommand{\H}{\mathcal{H}}
\newcommand{\I}{\mathcal{I}}
\newcommand{\J}{\mathcal{J}}
\newcommand{\M}{\mathcal{M}}
\newcommand{\N}{\mathcal{N}}
\renewcommand{\O}{\mathcal{O}}
\renewcommand{\P}{\mathcal{P}}
\newcommand{\Q}{\mathcal{Q}}
\newcommand{\R}{\mathcal{R}}
\newcommand{\T}{\mathcal{T}}
\renewcommand{\S}{\mathcal{S}}
\newcommand{\V}{\mathcal{V}}
\newcommand{\X}{\mathcal{X}}
\newcommand{\Z}{\mathcal{Z}}

\newcommand{\EE}{\mathbb{E}}
\renewcommand{\SS}{\mathbb{S}}
\newcommand{\ZZ}{\mathbb{Z}}

\newcommand{\PS}{\text{PS}}
\newcommand{\col}{\underline}

\usepackage[inline]{enumitem} % in-line lists and \setlist{} (below)
\setlist[enumerate,1]{label={(\roman*)}} % default in-line numbering
\setlist{nolistsep} % more compact spacing between environments

%%% text markup
\usepackage{color} % text color
\newcommand{\red}[1]{{\color{red} #1}}

%%%%%%%%%%%%%%%%%%%%%%%%%%%%%%%%%%%%%%%%%%%%%%%%%%%%%%%%%%%%%%%%%%%%%%
\begin{document}
\thispagestyle{fancy}

\title{Schwinger boson MFT for an SU($n$) spin model}%
\author{Michael A. Perlin}%
\date{\today}

\maketitle

Suppose we have a general quadratic spin Hamiltonian of the form
\begin{align}
  H_{\t{spin}} = \f1N \sum_{\substack{\mu,\nu,\rho,\sigma\\i<j}}
  h^{\mu\nu i}_{\rho\sigma j} s_{\mu\nu i} s_{\rho\sigma j}
  + \sum_{\mu,\nu,i} \epsilon_{\mu\nu i} s_{\mu\nu i},
\end{align}
where $\mu,\nu\in\ZZ_n$ index orthogonal states of a single spin;
$i,j\in\ZZ_N$ index one of $N$ spins; $h^{\mu\nu i}_{\rho\sigma j}$
and $\epsilon_{\mu\nu i}$ are scalars; and
$s_{\mu\nu i}\coloneqq\op{\mu}{\nu}_i$ is a transition operator for
spin $i$.  We can write this Hamiltonian in terms of Schwinger bosons
as
\begin{align}
  H = \f1N \sum_{\substack{\mu,\nu,\rho,\sigma\\i<j}}
  h^{\mu\nu i}_{\rho\sigma j}
  b_{\mu i}^\dag b_{\nu i} b_{\rho j}^\dag b_{\sigma j}
  + \sum_{\mu,\nu,i} \epsilon_{\mu\nu i} b_{\mu i}^\dag b_{\nu i},
\end{align}
where $b_{\mu i}$ annihilates a boson of type $\mu$ on site $i$.  The
Heisenberg equations of motion for these operators are
\begin{align}
  \i \d_t b_{\alpha i} = \sp{b_{\alpha i}, H}
  &= \f1N \sum_{\substack{\mu,\nu,\rho,\sigma\\j<k}}
  h^{\mu\nu j}_{\rho\sigma k}
  \sp{b_{\alpha i}, b_{\mu j}^\dag b_{\nu j} b_{\rho k}^\dag b_{\sigma k}}
  + \sum_{\mu,\nu,j} \epsilon_{\mu\nu j}
  \sp{b_{\alpha i}, b_{\mu j}^\dag b_{\nu j}} \\
  &= \f1N \sum_{\mu,\nu,\rho,\sigma} \sum_{k\ne i}
  h^{\mu\nu j}_{\rho\sigma k}
  \sp{b_{\alpha i}, b_{\mu i}^\dag b_{\nu i}} b_{\rho k}^\dag b_{\sigma k}
  + \sum_{\mu,\nu} \epsilon_{\mu\nu i}
  \sp{b_{\alpha i}, b_{\mu i}^\dag b_{\nu i}} \\
  &= \sum_{\mu,\nu} \p{\f1N \sum_{\rho,\sigma} \sum_{k\ne i}
    h^{\mu\nu i}_{\rho\sigma k} b_{\rho k}^\dag b_{\sigma k}
    + \epsilon_{\mu\nu i}}
  \sp{b_{\alpha i}, b_{\mu i}^\dag b_{\nu i}}
\end{align}
where
\begin{align}
  \sp{b_{\alpha i}, b_{\mu i}^\dag b_{\nu i}}
  = \delta_{\mu\alpha} \delta_{\nu\alpha} b_{\alpha i}
  + \delta_{\mu\alpha} \p{1-\delta_{\nu\alpha}} b_{\nu i}
  = \delta_{\mu\alpha} b_{\nu i},
\end{align}
so
\begin{align}
  \i \d_t b_{\alpha i}
  = \sum_\nu \p{\f1N \sum_{\rho,\sigma} \sum_{k\ne i}
    h^{\alpha\nu i}_{\rho\sigma k} b_{\rho k}^\dag b_{\sigma k}
    + \epsilon_{\alpha\nu i}} b_{\nu i}.
\end{align}
If interactions are SU($n$) symmetric, then
$h^{\alpha\nu i}_{\rho\sigma k} = h_{ik}
\delta_{\alpha\sigma}\delta_{\nu\rho}$, so
\begin{align}
  \i \d_t b_{\alpha i}
  = \sum_\nu \p{\f1N \sum_{k\ne i} h_{ik} b_{\nu k}^\dag b_{\alpha k}
    + \epsilon_{\alpha\nu i}} b_{\nu i}.
\end{align}


% todo: find roots of lax vector (numerically)

\end{document}

%%% Local Variables:
%%% mode: latex
%%% TeX-master: t
%%% End:
