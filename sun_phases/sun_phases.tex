\documentclass[nofootinbib,twocolumn]{revtex4-2}

%%% linking references
\usepackage[dvipsnames]{xcolor}
\usepackage{hyperref}
\hypersetup{
  breaklinks=true,
  colorlinks=true,
  allcolors=BlueViolet,
}

%%% symbols, notations, etc.
\usepackage{physics,braket,bm,amssymb} % physics and math
\renewcommand{\t}{\text} % text in math mode
\newcommand{\f}[2]{\dfrac{#1}{#2}} % shorthand for fractions
\newcommand{\p}[1]{\left(#1\right)} % parenthesis
\renewcommand{\sp}[1]{\left[#1\right]} % square parenthesis
\renewcommand{\set}[1]{\left\{#1\right\}} % curly parenthesis
\newcommand{\bk}{\Braket} % shorthand for braket notation
\renewcommand{\v}{\bm} % bold vectors
\renewcommand{\c}{\cdot} % inner product
\renewcommand{\d}{\partial} % partial derivative
\renewcommand{\dd}{\text{d}} % for infinitesimals
\renewcommand{\i}{\mathrm{i}\mkern1mu} % imaginary unit

% double angle brackets
\newcommand{\bbk}[1]{\langle\!\langle #1 \rangle\!\rangle}
\newcommand{\Bbk}[1]
{\left\langle\!\!\left\langle #1 \right\rangle\!\!\right\rangle}

\usepackage{dsfont} % for identity operator
\newcommand{\1}{\mathds{1}}

% more shorthands
\newcommand{\up}{\uparrow}
\newcommand{\dn}{\downarrow}
\newcommand{\x}{\text{x}}
\newcommand{\y}{\text{y}}
\newcommand{\z}{\text{z}}
\newcommand{\g}{\text{g}}
\newcommand{\e}{\text{e}}
\newcommand{\su}{\mathfrak{su}}
\renewcommand{\SS}{\mathbb{S}}

%%% figures
\usepackage{graphicx} % for figures
\graphicspath{{./figures/}} % set path for all figures

% for text markup
\newcommand{\red}[1]{{\color{red} #1}}

% use alphanumeric tags for footnotes
\renewcommand*{\thefootnote}{\alph{footnote}}

%%%%%%%%%%%%%%%%%%%%%%%%%%%%%%%%%%%%%%%%%%%%%%%%%%%%%%%%%%%%%%%%%%%%%%
\begin{document}

\title{Multilevel spin models in cold atomic systems}
\author{Michael A. Perlin}
% \orcid{https://orcid.org/0000-0002-9316-1596}
\email{mika.perlin@gmail.com}
\author{Diego Barbarena}
\author{Robert J.~Lewis-Swan}
% \orcid{https://orcid.org/0000-0002-0201-281X}
\author{Ana Maria Rey}
% \orcid{http://orcid.org/0000-0001-7176-9413}
\affiliation{JILA, National Institute of Standards and Technology and University of Colorado, 440 UCB, Boulder, Colorado 80309, USA}
\affiliation{Center for Theory of Quantum Matter, University of Colorado, Boulder, CO, 80309, USA}

\date{\today}

\begin{abstract}
  Abstract.
\end{abstract}

\maketitle

%%%%%%%%%%%%%%%%%%%%%%%%%%%%%%%%%%%%%%%%%%%%%%%%%%%%%%%%%%%%%%%%%%%%%%
\section{Introduction}
\label{sec:intro}

SU($n$) symmetries play an important role in physics.
Underpinning much of high energy physics, the SU($n$) gauge theory known as Yang-Mills theory is central to our understanding of the electroweak and strong forces.
Extensions of Yang-Mills and SU($n$) symmetry feature in the most well-studied examples of holographic duality \cite{maldacena1999largen} and the connection between entanglement and gravity \cite{ryu2006holographic} through the anti-de Sitter/conformal field theory (AdS/CFT) correspondence.
In a condensed matter setting, SU(2) appears ubiquitously as a symmetry of the Hubbard model, with important consequences for the study of quantum magnetism and high temperature superconductivity \cite{lee2006doping}.
The extension of SU(2) Hubbard and spin models to SU($n$) has led to predictions of exotic phases of matter such as valence bond solids \cite{read1989valencebond, rokhsar1990quadratic, kaul2012lattice, hermele2011topological} and chiral spin liquids \cite{hermele2009mott, hermele2011topological, chen2016syntheticgaugefield, nataf2016chiral}, as well as the potential to perform universal topological quantum computation \cite{freedman2004class, nayak2008nonabelian} and other phenomena \cite{nataf2014exact, nataf2016exact}.
Furthermore, the consideration of disordered SU($n$) spin models has opened analytically tractable avenues for studying quantum chaos and information scrambling \cite{sachdev1993gapless, bentsen2019integrable}.

The tremendous theoretical significance of SU($n$) symmetries makes it all the more exciting that they appear naturally in experimental atomic, molecular, and optical (AMO) platforms \cite{gorshkov2010twoorbital, beverland2016realizing, cazalilla2014ultracold, taie2012su, hofrichter2016direct, cappellini2014direct, scazza2014observation, zhang2014spectroscopic, goban2018emergence, perlin2019effective} with exquisite degrees of microscopic control.
In the simplest example, this symmetry arises through the independence of atomic orbital and interaction parameters on the $n$ nuclear spin states of alkaline-earth(-like) atoms, with e.g.~$n=10$ in the case of ${}^{87}$Sr \cite{cazalilla2014ultracold}.
As a result, experiments can directly probe the SU($n$) Hubbard model, leading to experimental observations of SU($n$) Hubbard phases and phase transitions \cite{taie2012su, hofrichter2016direct}, two-orbital SU($n$) magnetism \cite{cappellini2014direct, scazza2014observation, zhang2014spectroscopic}, and multi-body SU($n$)-symmetric interactions \cite{goban2018emergence, perlin2019effective}.
In the spirit of quantum simulation, further investigations of this sort can play an important role in understanding SU($n$) symmetries' consequences for fundamental questions in physics, as well as their practical use in technological applications.
For example, SU(2)-symmetric spin interactions can be harnessed to develop quantum sensors that surpass classical limits on measurement precision \cite{he2019engineering, perlin2020spin}.
The exciting prospect of similarly exploiting larger SU($n$) symmetries to achieve a technological advantage is still an unexplored avenue of research with enormous potential.

\red{[finish intro, summarize paper]}
% (un)interested reader can ...
% main results in/below Eqs. ...

%%%%%%%%%%%%%%%%%%%%%%%%%%%%%%%%%%%%%%%%%%%%%%%%%%%%%%%%%%%%%%%%%%%%%%
\section{From lattice fermions to an SU($n$) spin model}
\label{sec:spin_model}

Here we review the realization of a collective SU($n$) spin model in a system of ultracold alkaline-earth(-like) atoms on an optical lattice.
Without external driving fields, the evolution of such atoms in their electronic ground state is governed by single-body kinetic and two-body interaction Hamiltonians of the form
\begin{align}
  H_{\t{kin}}
  &= -\f{J}{2} \sum_{\bk{j,j'},\mu} c_{j\mu}^\dag c_{j'\mu} + \t{h.c.}, \\
  H_{\t{int}}
  &= \f{U_0}{2} \sum_{j,\mu,\nu}
  c_{j\mu}^\dag c_{j\mu} c_{j\nu}^\dag c_{j\nu},
\end{align}
where $\bk{j,j'}$ denotes neighboring lattice sites $j$ and $j'$; $\mu,\nu\in\set{s,s-1,\cdots,-s}$ index orthogonal spin states of a spin-$s$ nucleus, with $s=\frac{n-1}{2}$ (e.g.~$s=\frac{9}{2}$ in the case of ${}^{87}$Sr with $10$ nuclear spin states); $c_{j\mu}$ is a fermionic annihilation operator, $J$ is a tunneling amplitude; and $U_0$ is a two-body on-site interaction energy.
In the present work, we neglect inter-site interactions and interaction-assisted hopping, which may become relevant in a sufficiently shallow lattice, namely when $J\gtrsim E_{\t{R}}$, where $E_{\t{R}}$ is the atomic lattice recoil energy.
For simplicity, we now assume a periodic lattice of $L$ sites, and expand the on-site fermionic operators in terms of operators addressing (quasi-)momentum modes $q$, $c_{j\mu}=\frac1{\sqrt{L}}\sum_k e^{-\i q\c j} c_{q\mu}$, finding that
\begin{align}
  H_{\t{kin}}
  &= -J\sum_q \cos\p{q} c_{q\mu}^\dag c_{q\mu}, \\
  H_{\t{int}}
  &= \f{U}{2N} \sum_{k,\ell,p,q,\mu,\nu}
  c_{k\mu}^\dag c_{\ell\mu} c_{p\nu}^\dag c_{q\nu}
  \times \delta_{k+p,\ell+q},
  \label{eq:H_int_momenta}
\end{align}
where $N$ is the total number of atoms on the lattice, we define $U\equiv U_0\times N/L$ for convenience, $\delta_{k+p,\ell+q}=1$ if $k+p=\ell+q$ and $0$ otherwise (enforcing conservation of momentum), and we work in units with lattice spacing $a=1$.

If the interaction energy $U_0$ is smaller than the single-particle bandwidth $\sim J$, then the mode-changing collisions in $H_{\t{int}}$ become off-resonant, motivating the frozen-mode approximation $\set{k,p}=\set{\ell,q}$ (which we assume for the remainder of this work).
The operator content of terms with $k=\ell$ and $p=q$ takes the form $c_{\ell\mu}^\dag c_{\ell\mu} c_{q\nu}^\dag c_{q\nu}$, so these terms contribute an overall energy $\frac12NU$ that can be neglected in the absence of coherent atom number fluctuations.
Defining the spin operators $s_{\mu\nu q}\equiv c_{q\mu}^\dag c_{q\nu}$, the remaining terms of the kinetic and interaction Hamiltonians become
\begin{align}
  H_{\t{kin}} &= -J\sum_q \cos\p{q} s_{\mu\mu q},
  \label{H_kin_start} \\
  H_{\t{int}} &= -\f{U}{2N} \sum_{p,q,\mu,\nu} s_{\mu\nu p} s_{\nu\mu q}.
  \label{eq:H_int_start}
\end{align}
Throughout this work, we will assume that atomic modes are singly-occupied, e.g.~due to the initialization of a spin-polarized state with one atom per lattice site, in which multiple occupation of an atomic mode is forbidden by Pauli exclusion.
In this case, we can neglect the fermionic origins of $H_{\t{int}}$ and instead treat our system as a bona fide spin model.

To further simplify notation and write the interaction Hamiltonian in a form suggestively similar to SU(2) spin alignment, we now construct the operator-valued spin matrix
\begin{align}
  \v s_q \equiv \sum_{\mu,\nu} s_{\mu\nu q} \op{\mu}{\nu},
\end{align}
and for any pair of such operator-valued matrices $\v A,\v B$, we define the inner product
\begin{align}
  \v A \c \v B \equiv \Tr\p{\v A^\dag\v B}
  = \sum_{\mu,\nu} A_{\mu\nu}^\dag B_{\mu\nu},
\end{align}
where the trace is performed only over the auxiliary matrix degrees of freedom.
These definitions allow us to write the spin Hamiltonian in Eq.~\eqref{eq:H_int_start} as
\begin{align}
  H_{\t{int}} = -\f{U}{2N} \sum_{p,q} \v s_p\c\v s_q
  = -\f{U}{2N}\v S\c\v S,
  &&
  \v S \equiv \sum_q \v s_q.
  \label{eq:H_int}
\end{align}
Here $\v S$ is a collective spin matrix, analogous to the collective spin vector $\vec S=\p{S_\x,S_\y,S_\z}$ in the case of SU(2) \cite{he2019engineering}.

We now discuss the spin Hamiltonian $H_{\t{int}}$ in Eq.~\eqref{eq:H_int}.
The operator $\v s_p\c\v s_q$ simply swaps the nuclear spin states of two atoms pinned to modes $p,q$.
The term $\v s_p\c\v s_q$ thereby assigns a definite energy of $+1$ ($-1$) to a pair of spins that are symmetric (anti-symmetric) under exchange.
In this sense, $\v s_p\c\v s_q$ is analogous to the enforcement of SU(2) spin alignment, which similarly assigns (distinct) definite energies to anti-symmetric spin-0 singlet $\ket{\up\dn}-\ket{\dn\up}$ and the symmetric spin-1 triplets $\set{\ket{\up\up},\ket{\dn\dn},\ket{\up\dn}+\ket{\dn\up}}$.

By summing over all pair-wise exchange terms $\v s_p\c\v s_q$, the interaction Hamiltonian $H_{\t{int}}$ energetically enforces a permutational symmetry among all spins, opening an energy gap $U$ between the manifold of all permutationally-symmetric (PS) states and the orthogonal complement of excited (e.g.~spin-wave) states that break permutational symmetry.
In the case of SU(2), the PS manifold is precisely the Dicke manifold of collective states $\ket{m_\z}$ with total spin $S=\frac{N}{2}$ and definite spin projection $m_\z\in\set{S,S-1,\cdots,-S}$ onto a fixed quantization axis.
Equivalently, Dicke states $\ket{m_\z}=\ket{m_\up,m_\dn}$ can be labeled by a definite number of spins $m_\up=S+m_\z$ ($m_\dn=S-m_\z$) pointing up (down) along the spin quantization axis, with $m_\up+m_\dn=N$.
In the general case of SU($n$), the PS manifold is similarly spanned by states $\ket{m_s,m_{s-1},\cdots,m_{-s}}$ with a definite number $m_\mu$ of spins in state $\mu$, and $\sum_\mu m_\mu=N$.
The dimension of the PS manifold, or the ground-state degeneracy of $H_{\t{int}}$, is equal to the number of ways of assigning $N$ identical spins to $n$ distinct internal states, or ${N+n-1 \choose n-1} \sim N^{n-1}$.

\begin{figure}
\centering
\includegraphics[width=\linewidth]{sketches/spin_model.pdf}
\caption{
(a) Ultracold atoms on a lattice can tunnel between neighboring lattice sites at a rate $J$, and locally repel each other with interaction energy $U_0$.
(b,c) When the interaction energy $U_0$ is smaller than the single-particle bandwidth $\sim J$, the frozen-mode approximation enables writing the interaction Hamiltonian as a spin model consisting of exchange terms $\v s_p\c\v s_q$, which swap the states two spins pinned to momentum modes $p,q$.
(d) Interactions thereby open an energy gap $U$ between the manifold of permutationally-symmetric states, and the orthogonal complement of states that break permutational symmetry.
}
\label{fig:spin_model}
\end{figure}

Finally, we take a moment to discuss individual $n$-level spins.
Whenever possible, it is desirable to find geometric representations of abstract mathematical objects, such as the state of a quantum system.
Geometric representations facilitate the development of understanding and intuitions that can be difficult to extract from algebraic expressions.
For this reason, the state of a two-level spin, or a qubit, is commonly represented by a point on (or within) the Bloch sphere.
More generally, the state $\ket\psi$ of a $n$-level spin can be represented by a probability distribution\footnote{Strictly speaking, in order to make $Q_\psi$ (and $Q_\rho$) a probability distribution, one has to divide it by a normalization factor $\frac{4\pi}{d}$.
We omit this normalization factor for convenience.} $Q_\psi$ on the sphere $\SS^2$.
The value $Q_\psi\p{\v v}$ at a point $\v v\in\SS^2$ equal to the overlap of $\ket\psi$ with a pure state $\ket{\v v}$ maximally polarized along $\v v$: $Q_\psi\equiv\abs{\bk{\v v|\psi}}^2$ (see Figure \ref{fig:spin_dist}).
In the case of a mixed state $\rho$, this distribution is defined by $Q_\rho\equiv\bk{\v v|\rho|\v v}$.
The dimension $n$ is reflected in $Q_\rho$ by the fact that it can be decomposed into spherical harmonics $Y_{\ell m}$ with degree $\ell<n$ (that is, $Q_\rho$ is ``low resolution'' or ``band-limited'' at degree $n$) \cite{perlin2020qudit}, as well as in the normalization $\int_{\v v\in\SS^2}\dd^2\v v\,Q_\rho\p{\v v}=\frac{4\pi}{d}$.

\begin{figure}
\centering
\includegraphics[width=\linewidth]{sketches/qudit_rep.pdf}
\caption{
Whereas the state of a two-level spin (qubit) can be represented by a point on (or inside) the Bloch sphere, the state of a $n$-level spin is more generally represented by a probability distribution on the sphere, decomposable into spherical harmonics $Y_{\ell m}$ of degree $\ell<n$.
The distribution shown for $n=10$ corresponds to a Haar-random pure state.
}
\label{fig:spin_dist}
\end{figure}

In fact, $Q_\rho$ is essentially the Husimi-$Q$ function commonly used (e.g.~in the spin squeezing community \cite{ma2011quantum}) to represent collective states of spin-$\frac{1}{2}$ particles in the Dicke manifold.
This correspondence is made precise by identifying the Hilbert space of a single $n$-level spin with the Dicke manifold of $n-1$ spin-$\frac{1}{2}$ particles.
The Hilbert space of a 10-level (spin-$\frac92$) nucleus of a ${}^{87}$Sr atom, for example, can be thought of as the Dicke manifold of 9 ``excess'' proton and neutron spins.
This representation of a $n$-level quantum system is a useful picture to keep in mind for angular-momentum-like degrees of freedom, and will be used freely throughout this work.

%%%%%%%%%%%%%%%%%%%%%%%%%%%%%%%%%%%%%%%%%%%%%%%%%%%%%%%%%%%%%%%%%%%%%%
\section{Spin-orbit coupling and gauge transformation}
\label{sec:SOC}

We wish to study the dynamical behaviors of experimentally realizable SU($n$) spin models.
In order to ensure validity of the $\v S\c\v S$ spin Hamiltonian in Eq.~\eqref{eq:H_int}, we consider initial preparation of a spin-polarized state\footnote{We remind the reader that validity of the spin model requires that each momentum mode is at most singly-occupied, which is guaranteed in a spin-polarized state due to Pauli blocking.}, which has the added benefit of being simple to prepare experimentally.
However, a spin polarized state is an eigenstate of both the kinetic and interaction spin Hamiltonians in Section \ref{sec:spin_model}, so we need additional ingredients for nontrivial behavior.
Here, we consider the addition of spin-orbit coupling (SOC) induced by external driving fields.

Before discussing SOC for $n$-level fermions, we first consider the well-studied case of two-level SOC on a one-dimensional lattice \cite{wall2016synthetic, bromley2018dynamics}.
In this case, SOC is induced by an external driving field that imprints a phase $e^{-\i\phi j}$ on lattice site $j$, or equivalently imparts a momentum kick $q\to q+\phi$, upon the absorption of a photon:
\begin{align}
  H_{\t{drive}}^{(\phi)}
  = \f{\Omega}{2} \sum_q c_{q+\phi,\up}^\dag c_{q,\dn} + \t{h.c.}.
  \label{eq:drive_2}
\end{align}
Identifying a numerical spin index $\mu=+\frac12$ ($-\frac12$) with the state $\up$ ($\dn$), this drive Hamiltonian can be diagonalized in its momentum index $q$ by the gauge transformation $c_{q\mu}^\dag\to c_{q-\mu\phi,\mu}^\dag$ (equivalently $c_{j\mu}^\dag\to e^{\i\phi\mu j} c_{j\mu}^\dag$), which takes
\begin{align}
  H_{\t{drive}}^{(\phi)} \to H_{\t{drive}} \equiv \Omega S_\x,
  &&
  S_\x \equiv \sum_q s_{\x,q},
  \label{eq:drive_trans}
\end{align}
where $S_\x$ is a collective spin-$x$ operator, and $s_{\x,q}=\frac12 c_{q,\up}^\dag c_{q,\dn} + \t{h.c.}$ for two-level spins.

The two-level drive in Eq.~\eqref{eq:drive_2} is typically implemented by coupling pseudo-spin electronic states of an atom with an external laser \cite{wall2016synthetic, bromley2018dynamics}.
Though the coupling between laser fields and nuclear spins is generally too weak to drive nuclear spin transitions directly, it will nonetheless be useful to consider the $n$-level generalization of the two-level drive.
This generalization will capture the important features of SOC for $n$-level nuclei, while avoiding complications that we defer to Section \ref{sec:controls}.

The spin-$s$ generalization of Eq.~\eqref{eq:drive_2} is
\begin{align}
  H_{\t{drive}}^{(\phi)} = \Omega_0 \sum_{q,\mu} \bk{s,\mu;1,1|s,\mu+1}
  c_{q+\phi,\mu+1}^\dag c_{q\mu} + \t{h.c.},
\end{align}
where $\mu\in\set{s,s-1,\cdots,-s}$, and $\bk{s,\mu;1,1|s,\mu+1}$ is a Clebsch-Gordan coefficient.
The gauge gauge transformation $c_{q\mu}^\dag\to c_{q-\mu\phi,\mu}^\dag$ again takes $H_{\t{drive}}^{(\phi)}\to \Omega S_\x$ as in Eq.~\eqref{eq:drive_trans}, where now
\begin{align}
  \Omega = \f{\Omega_0}{\xi},
  &&
  \xi \equiv -\sqrt{\f{s(s+1)}{2}},
\end{align}
and
\begin{align}
  s_{\x,q} \equiv \xi \sum_\mu \bk{s,\mu;1,1|s,\mu+1} c_{q,\mu+1}^\dag c_{q\mu} + \t{h.c.}
  \label{eq:spin_x_op}
\end{align}
is an $n$-level spin-$x$ operator for spin $q$, with which $e^{-\i\theta s_{\x,q}}$ rotates the state of spin $q$ about the $x$ axis by an angle $\theta$.
Despite complicating details, the change from two-level to $n$-level spins thus results in a drive Hamiltonian with an identical physical interpretation: after moving into an appropriate ``gauged frame'', this drive simply rotates all spins about the $x$ axis at a rate $\Omega$.

\begin{figure}
\centering
\includegraphics{sketches/soc_panels.pdf}
\caption{
Spin-orbit coupling for 2-level ({\bf a},{\bf c}) and 4-level ({\bf b},{\bf d}) spins.
Colors indicate different spin projections $\mu$.
In the ``lab frame'' ({\bf a},{\bf b}), kinetic energy is insensitive to spin, but the drive imparts a momentum kick $q\to q+\phi$ into any atom that absorbs a photon.
Changing to the ``gauged frame'' ({\bf c},{\bf d}) makes the drive diagonal in momentum, but comes at the cost of making kinetic energy spin-dependent.
}
\label{fig:soc_panels}
\end{figure}

Of course, spin-orbit coupling cannot be ``gauged away'' entirely.
Making a gauge transformation to simplify the drive comes at the cost of making kinetic energy spin dependent, taking
\begin{align}
  H_{\t{kin}} \to H_{\t{kin}}^{(\phi)}
  \equiv -J \sum_q \cos\p{q+\mu\phi} s_{\mu\mu q},
\end{align}
as visualized in Figure \ref{fig:soc_panels}.
To better interpret this Hamiltonian, we can write it in the simplified form
\begin{align}
  H_{\t{kin}}^{(\phi)}
  = -J \sum_q
  \sp{\cos\p{q} w_{+,q}^{(\phi)} - \sin\p{q} w_{+,q}^{(\phi)}},
\end{align}
where
\begin{align}
  w_{+,q}^{(\phi)} &\equiv \sum_\mu \cos\p{\mu\phi} s_{\mu\mu q}, \\
  w_{-,q}^{(\phi)} &\equiv \sum_\mu \sin\p{\mu\phi} s_{\mu\mu q}.
\end{align}
For two-level spins, $w_{+,q}^{(\phi)}$ is proportional to the identity operator, and $w_{+,q}^{(\phi)}=2\sin\p{\phi/2} s_{q,\z}$, where $s_{\z,q}=\sum_\mu \mu\, s_{\mu\mu q}$ is a spin-$z$ operator for spin $q$, so the kinetic Hamiltonian in the gauged frame describes a (synthetic) inhomogeneous magnetic field:
\begin{align}
  \left. H_{\t{kin}}^{(\phi)} \right|_{d=2}
  = 2\sin\p{\phi/2} J \sum_q \sin\p{q} s_{\z,q}.
\end{align}
When $n>2$, an inhomogeneous magnetic field is likewise recovered in the limit of weak SOC, $s\phi\ll1$, in which case
\begin{align}
  \left. H_{\t{kin}}^{(\phi)} \right|_{s\phi\ll1}
  = J\phi \sum_q \sin\p{q} s_{\z,q} + O\p{(s\phi)^2}.
\end{align}
As the strength of SOC is increased, this Hamiltonian acquires terms with higher powers of $s_{\z,q}$, up to $s_{\z,q}^{n-1}$.

In principle, making the gauge transformation $c_{q\mu}^\dag\to c_{q-\mu\phi,\mu}^\dag$ also transforms the interaction Hamiltonian as
\begin{align}
  H_{\t{int}} \to H_{\t{int}}^{(\phi)}
  \equiv -\f{U}{2N} \sum_{p,q,\mu,\nu}
  c_{p-\mu\phi,\mu}^\dag c_{p-\nu\phi,\nu}
  c_{q-\nu\phi,\nu}^\dag c_{q-\mu\phi,\mu}.
\end{align}
However, $H_{\t{int}}^{(\phi)}$ cannot be exactly reduced to a spin Hamiltonian that addresses the same spin degrees of freedom as $H_{\t{kin}}^{(\phi)}$.
Nonetheless, in the weak SOC limit $s\phi\ll1$ the effect of $H_{\t{int}}^{(\phi)}$ may be approximated by $H_{\t{int}}^{(0)}=H_{\t{int}}$, which has previously been benchmarked for SU(2)-symmetric interactions \cite{he2019engineering, smale2019observation}.
To ensure that $H_{\t{kin}}^{(\phi)}$ does not become trivial as $\phi\to0$, we can simultaneously take $J\to\infty$ while keeping $J\phi$ constant.
Altogether, the interacting spin Hamiltonian in the gauged frame becomes
\begin{align}
  H_{\t{spin}} = -\f{U}{2N} \v S\c\v S + J\phi \sum_q \sin\p{q} s_{\z,q}.
  \label{eq:H_spin}
\end{align}

%%%%%%%%%%%%%%%%%%%%%%%%%%%%%%%%%%%%%%%%%%%%%%%%%%%%%%%%%%%%%%%%%%%%%%
\section{Control fields, observables, and initial states}
\label{sec:controls}

We now take a closer look at external control fields, which determine the observables we can access and initial states that we can prepare.
To this end, we consider addressing the atoms with several lasers detuned by $\Delta$ below an electronic ${^1}\t{S}_0\to {^3}\t{P}_0$ transition of individual alkaline-earth-like atoms on a one-dimensional lattice with tight transverse confinement.
Each laser is indexed by an axis of propagation, $\v v$, and a drive amplitude $\Omega_{\v v,+}$ ($\Omega_{\v v,-}$) for right-circularly (left-circularly) polarized light.
An electronic ${^1}\t{S}_0\to {^3}\t{P}_0$ transition is thus accompanied by a nuclear spin transition $\mu_{\v v}\to\mu_{\v v}\pm1$, where $\mu_{\v v}$ is the projection of nuclear spin onto the axis $\v v$.
Furthermore, a laser pointing along $\v v$ will generally imprint a phase $e^{-\i\kappa\v v\c\v\ell j}$ on lattice site $j$, where $\kappa$ is the laser's wavenumber (in units with lattice spacing $a=1$) and $\v\ell$ is a fixed unit vector parallel to the lattice.
Altogether, the multi-laser drive Hamiltonian can be written in the form
\begin{multline}
  H_{\t{drive}}^{\t{full}}
  = \sum_{j,\v v,\sigma} \Omega_{\v v\sigma}
  \p{e^{-\i\kappa\v v\c\v\ell j} s_{\v v\sigma j}\otimes\op{\e}{\g}_j + \t{h.c.}} \\
  + \Delta \sum_j \1_j \otimes \op{\e}_j,
  \label{eq:drive_full}
\end{multline}
where $\sigma\in\set{+1,-1}$ indexes a polarization of light; $s_{\v v\sigma j}$ is a $n$-level spin-raising ($\sigma=+1$) or spin-lowering ($\sigma=-1$) operator along axis $\v v$ for atom $j$ (clarified below); $\ket\g_j$ and $\ket\e_j$ respectively denote the ground (${^1}\t{S}_0$) and excited (${^3}\t{P}_0$) electronic states of an atom on site $j$; $\1_j\equiv\sum_\mu s_{\mu\mu j}$ is an identity operator for the spin of atom $j$; and we work in a rotating frame in which the energy difference between ground and excited electronic states is $\Delta$.

To define the spin-raising/lowering operators $s_{\v v\sigma j}$, we first define $s_{\x,j}$ similarly to $s_{\x,q}$ in Eq.~\eqref{eq:spin_x_op} with $q\leftrightarrow j$.
The spin-raising operator $s_{+,j}$ is then twice the spin-raising part of $s_{\x,j}$, just as in the case of SU(2) where $s_\x=\frac12\op{\up}{\dn}+\t{h.c.}$, and the spin-lowering operator is $s_{-,j}\equiv s_{+,j}^\dag$.
The operator $s_{\v v,+\,j}$ that raises spin along $\v v$ is acquired by a rotation of $s_{+,j}$ that maps the north pole to $\v v$:
\begin{align}
  s_{\v v,+} \equiv U_{\v v} s_+ U_{\v v}^\dag,
  &&
  U_{\v v} \equiv e^{-\i\alpha_{\v v}s_\z} e^{-\i\beta_{\v v}s_\y},
  \label{eq:spin_rot}
\end{align}
where we have dropped the site index $j$ for brevity; $\alpha_{\v v}$ and $\beta_{\v v}$ are the azimuthal and polar angles of $\v v$; and $s_\y\equiv-\frac\i2s_++\t{h.c.}$.
Note that the operators $(s_\x,s_\y,s_\z)$ form an $\su(2)$ algebra, so given a particular axis $\v v$ one can expand $s_{\v v,+}=\sum_m c_{\v vm} s_m$ with coefficients $c_{\v vm}$ that are independent of the dimension $n$.
In particular, one is always free to set $n=2$ and expand Eq.~\eqref{eq:spin_rot} with spin-$\frac12$ operators (or Pauli matrices) to find the coefficients $c_{\v vm}$.

\begin{figure}
\centering
\includegraphics[width=\linewidth]{sketches/3LD.pdf}
\caption{
Sketch of the three-laser drive used to address nuclear spins on a one-dimensional lattice.
Two counter-propagating lasers with right-circular polarization and amplitudes $\Omega_\pm$ point at an angle $\theta$ to the lattice axis.
A third, linearly polarized laser with amplitude $\Omega_0$ points in a direction orthogonal to both the lattice and the other driving lasers.
Absorbing a photon from the laser with amplitude $\Omega_m$ induces a transition $(\g,\mu)\to(\e,\mu+m)$ for the (electronic, nuclear spin) state of an atom, where nuclear spin is quantized along the axis of propagation for the $m=+1$ laser.
}
\label{fig:3LD}
\end{figure}

In order to make the drive Hamiltonian in Eq.~\eqref{eq:drive_full} more tractable, we orient the lattice along the $z$ axis, $\v\ell=(0,0,1)$, and consider a three-laser drive with a geometry sketched in Figure \ref{fig:3LD}.
In this setup, two counter-propagating right-circularly polarized lasers point in directions at an angle $\theta$ with the lattice axis, $\v v_\pm=\pm(0,-\sin\theta,\cos\theta)$, and a third linearly-polarized laser points along $\v v_0=(1,0,0)$.
Choosing $\v v_+$ as the spin quantization axis, we can define new drive amplitudes $\Omega_0\equiv-(\Omega_{\v v_0,+}+\Omega_{\v v_0,-})$ and $\Omega_\pm\equiv\pm\Omega_{\v v_\pm,+}$, as well as a SOC angle $\phi\equiv\kappa\v v_+\c\v\ell=\kappa\cos\theta$, in terms of which the drive Hamiltonian becomes
\begin{multline}
  H_{\t{3LD}}^{\t{full}}
  = \sum_{j,m} \Omega_m
  \p{e^{-\i m\phi j} s_{mj} \otimes\op{\e}{\g}_j + \t{h.c.}} \\
  + \Delta \sum_j \1_j \otimes \op{\e}_j,
\end{multline}
where $m\in\set{+1,0,-1}$ indexes the laser pointing along $\v v_m$; and $s_{0,j}\equiv s_{\z,j}$ for shorthand.
In the far-detuned limit $\abs{\Delta}\gg\abs{\Omega_m}$, a second-order perturbative treatment of electronic excitations ($\ket\e$) yields an effective drive Hamiltonian that only addresses ground-state nuclear spins.
After a gauge transformation $c_{j\mu}^\dag\to e^{\i\phi\mu j} c_{j\mu}^\dag$ (equivalently $s_{mj}\to e^{\i m\phi j}s_{mj}$), we can then make the simplifying assumption (or requirement) that all $\Omega_m$ are real to write
\begin{align}
  H_{\t{3LD}} = \sum_j H_{\t{3LD},j}^{\t{single}},
  \label{eq:drive_all}
\end{align}
where $H_{\t{3LD},j}^{\t{single}}$ denotes the action of $H_{\t{3LD}}^{\t{single}}$ on spin $j$:
\begin{multline}
  H_{\t{3LD}}^{\t{single}}
  = \tilde\Omega_+ \tilde\Omega_- s_\z
  + \tilde\Omega_0 \tilde\Omega_- s_\x
  + \tilde\Omega_0 \tilde\Omega_+ \p{s_\z s_\x  + s_\x s_\z} \\
  - \tilde\Omega_0^2 s_\z^2 - \tilde\Omega_+^2 s_\x^2
  - \tilde\Omega_-^2 s_\y^2,
  \label{eq:drive_single}
\end{multline}
with
\begin{align}
  \tilde\Omega_0 \equiv -\f{\Omega_0}{\sqrt\Delta},
  &&
  \tilde\Omega_\pm \equiv \f{\Omega_+\pm\Omega_-}{\sqrt\Delta}.
\end{align}

\begin{table}
\centering
\caption{
Drive Hamiltonians (left column) that can be implemented with different amplitude-matching conditions (right column), some of which are specified by a sign $\sigma\in\set{+1,-1}$.
The drives shown here are equal to that Eq.~\eqref{eq:drive_single} up to a possible energy shift of $s_\x^2+s_\y^2+s_\z^2=s(s+1)$, and come in mutually commuting pairs: a drive with $\Omega_m=1$ and $\Omega_n=0$ for both $n\ne m$ commutes with the drive in which both $\abs*{\Omega_n}=1$ but $\Omega_m=0$.
}
\vspace{.5em}
\begin{tabular}{c|c}
  $H_{\t{drive}}^{\t{single}}$
  & $(\tilde\Omega_0,\tilde\Omega_+,\tilde\Omega_-)$
  \\ \hline\hline
  $-s_\z^2$ & $\p{1,0,0}$
  \\ \hline
   $-s_\x^2$ & $\p{0,1,0}$
  \\ \hline
  $-s_\y^2$ & $\p{0,0,1}$
  \\ \hline
  $\sigma s_\z + s_\z^2$ & $\p{0,1,\sigma}$
  \\ \hline
  $\sigma s_\x + s_\x^2$ & $\p{1,0,\sigma}$
  \\ \hline
  $\sigma\p{s_\z s_\x+s_\x s_\z} + s_\y^2$ & $\p{1,\sigma,0}$
  \\ \hline
  $\pm s_\z \pm \sigma s_\x + \sigma \p{s_\z s_\x + s_\x s_\z}$
  & $\p{1,\sigma,\pm\sigma}$
\end{tabular}
\label{tab:drives}
\end{table}

Eqs.~\eqref{eq:drive_all} and \eqref{eq:drive_single} are the main technical results of this section.
There are three important observations to make about these results.
First, the fact that $H_{\t{3LD}}$ acts identically on all spins means we can freely replace the site index $j$ with a momentum index $q$ (as can be verified by substituting $c_{j\mu}=\frac1{\sqrt{L}}\sum_k e^{-\i q\c j} c_{q\mu}$), which is important to ensure that this drive addresses the same spin degrees of freedom as the spin Hamiltonian $H_{\t{spin}}$ in Eq.~\eqref{eq:H_spin}.
Second, each of $\tilde\Omega_0,\tilde\Omega_+,\tilde\Omega_-$ can be tuned independently by changing the amplitudes of the driving lasers; some particular Hamiltonians for specific values of these amplitudes are shown in Table \ref{tab:drives}.
Third, due to the appearance of mutually commuting pairs of Hamiltonians in Table \ref{tab:drives}, specifically $-s_\alpha^2$ and $\pm s_\alpha+s_\alpha^2$ for $\alpha\in\set{\z,\x}$, the three-laser drive admits pulse sequences that can implement arbitrary SU(2) (spatial) rotations of the form $e^{-\i\chi\vec n\c\vec s}$, where $\chi$ is a rotation angle, $\vec n$ is a rotation axis, and $\vec s\equiv(s_\x,s_\y,s_\z)$.
The capability to perform arbitrary spatial rotations, together with the capability to measure the number of atoms with spin projection $\mu$ onto a fixed quantization axis, $\bk{S_{\mu\mu}}$ (where $S_{\mu\nu}=\sum_js_{\mu\nu j}$), implies the capability to reconstruct all components of the mean collective spin matrix $\bk{\v S}=\sum_{\mu\nu}\bk{S_{\mu\nu}}\op{\mu}{\nu}$ via spin qudit tomography \cite{newton1968measurability, perlin2020qudit}.
We leave open the question of whether and how a suitable choice of time-dependent amplitudes $\tilde\Omega_m$ in Eq.~\eqref{eq:drive_single} enables generating arbitrary SU($n$) rotations.

Finally, we comment on the preparation of initial states.
Initial states are nominally prepared in the ``lab gauge'', and must be transformed according to the gauge transformation $c_{j\mu}^\dag\to e^{\i\phi\mu j} c_{j\mu}^\dag$ prior to evolution under the ``gauged'' spin Hamiltonian $H_{\t{spin}}$ in Eq.~\eqref{eq:H_spin}.
We assume the capability to prepare an initial state in which all spins are maximally polarized along $\v v$, i.e.~$\ket{\v v}^{\otimes N}=\ket{s}^{\otimes N}$, which is unaffected by the gauge transformation (up to a global phase).
The three-laser then allows us to rotate this state into one that is polarized along any spatial axis (in the gauged frame).
In addition, when $n>2$ the three-laser drive allows us to prepare product states with nontrivial intra-spin correlations.
For example, by choosing $H_{\t{drive}}^{\t{single}}\propto s_\x^2$ (see Table \ref{tab:drives}), we can transform the polarized state $\ket{s}^{\otimes N}$ into $\p{\ket{+s}+\ket{-s}}^{\otimes N}$, where every spin is in a coherent superposition of states pointing in opposite directions along $\v v$.
When $n>2$, this state has a vanishing mean spin vector, $\bk{S_\x}=\bk{S_\y}=\bk{S_\z}=0$ (where $S_\alpha = \sum_j s_{\alpha j}$), but $\bk{S_\z^2}=Ns^2$.

%%%%%%%%%%%%%%%%%%%%%%%%%%%%%%%%%%%%%%%%%%%%%%%%%%%%%%%%%%%%%%%%%%%%%%
\section{A simple model}
\label{sec:simple_model}



% use/simulate sinusoidal dispersion
% initial state in |X>
% repeat analysis with different initial state
% -- look at class of initial states (e.g. back-to-back, tipping angle)

\bibliography{\jobname.bib}

\onecolumngrid
\appendix

%%%%%%%%%%%%%%%%%%%%%%%%%%%%%%%%%%%%%%%%%%%%%%%%%%%%%%%%%%%%%%%%%%%%%%
\section{Schwinger boson equations of motion}
\label{sec:bosons}

Here we decompose a quadratic spin Hamiltonian into Schwinger bosons, and derive the equations of motion for the resulting boson operators.
We begin with a general spin Hamiltonian of the form
\begin{align}
  H = \sum_{\substack{\mu,\nu,\rho,\sigma\\j<k}}
  h^{\mu\nu j}_{\rho\sigma k} s_{\mu\nu j} s_{\rho\sigma k}
  + \sum_{\mu,\nu,j} \epsilon_{\mu\nu j} s_{\mu\nu j},
  \label{eq:spin}
\end{align}
where $\mu,\nu$ index orthogonal states of an $n$-level spin; $j,k$ index one of $N$ spins; $h^{\mu\nu j}_{\rho\sigma k}$ and $\epsilon_{\mu\nu j}$ are scalars; and $s_{\mu\nu j}=\op{\mu}{\nu}_j$ is a transition operator for spin $j$.
Strictly speaking, Eq.~\eqref{eq:spin} only defines the couplings $h^{\mu\nu j}_{\rho\sigma k}$ for $j<k$, so we enforce $h^{\mu\nu k}_{\rho\sigma j}=h^{\mu\nu j}_{\rho\sigma k}$ and $h^{\mu\nu j}_{\rho\sigma j}=0$ for completion.
Decomposing spin operators into Schwinger bosons as $s_{\mu\nu j}=b_{\mu j}^\dag b_{\nu j}$, where $b_{\nu j}$ a annihilates a boson of type $\nu$ on site $j$, we can write this Hamiltonian as
\begin{align}
  H = \sum_{\substack{\mu,\nu,\rho,\sigma\\j<k}}
  h^{\mu\nu j}_{\rho\sigma k}
  b_{\mu j}^\dag b_{\nu j} b_{\rho k}^\dag b_{\sigma k}
  + \sum_{\mu,\nu,j} \epsilon_{\mu\nu j} b_{\mu j}^\dag b_{\nu j}.
\end{align}
The Heisenberg equations of motion for the boson operators are then
\begin{align}
  \i \d_t b_{\alpha\ell} = \sp{b_{\alpha\ell}, H}
  &= \sum_{\substack{\mu,\nu,\rho,\sigma\\j<k}}
  h^{\mu\nu j}_{\rho\sigma k}
  \sp{b_{\alpha\ell}, b_{\mu j}^\dag b_{\nu j} b_{\rho k}^\dag b_{\sigma k}}
  + \sum_{\mu,\nu,j} \epsilon_{\mu\nu j}
  \sp{b_{\alpha\ell}, b_{\mu j}^\dag b_{\nu j}} \\
  &= \sum_{\mu,\nu,\rho,\sigma,k} h^{\mu\nu\ell}_{\rho\sigma k}
  \sp{b_{\alpha\ell}, b_{\mu\ell}^\dag b_{\nu\ell}}
  b_{\rho k}^\dag b_{\sigma k}
  + \sum_{\mu,\nu} \epsilon_{\mu\nu\ell}
  \sp{b_{\alpha\ell}, b_{\mu\ell}^\dag b_{\nu\ell}} \\
  &= \sum_{\mu,\nu} \p{\sum_{\rho,\sigma,k}
    h^{\mu\nu\ell}_{\rho\sigma k} b_{\rho k}^\dag b_{\sigma k}
    + \epsilon_{\mu\nu\ell}}
  \sp{b_{\alpha\ell}, b_{\mu\ell}^\dag b_{\nu\ell}}
\end{align}
where
\begin{align}
  \sp{b_{\alpha\ell}, b_{\mu\ell}^\dag b_{\nu\ell}}
  = \delta_{\alpha\mu} \delta_{\alpha\nu} b_{\alpha\ell}
  + \delta_{\alpha\mu} \p{1-\delta_{\alpha\nu}} b_{\nu\ell}
  = \delta_{\alpha\mu} b_{\nu\ell},
\end{align}
so
\begin{align}
  \i \d_t b_{\alpha\ell}
  = \sum_\nu \p{\sum_{\rho,\sigma,k}
    h^{\alpha\nu\ell}_{\rho\sigma k} b_{\rho k}^\dag b_{\sigma k}
    + \epsilon_{\alpha\nu\ell}} b_{\nu\ell}.
\end{align}
In the case of uniform SU($n$)-symmetric interactions of the form $\frac{h}{2}\v S\c\v S$ and a diagonal external field, we have
\begin{align}
  h^{\alpha\nu\ell}_{\rho\sigma k}
  = h \times \delta_{\alpha\sigma} \delta_{\nu\rho},
  &&
  \epsilon_{\alpha\nu\ell}
  = \epsilon_{\alpha\ell} \times \delta_{\alpha\nu}
\end{align}
so
\begin{align}
  \i \d_t b_{\alpha\ell}
  = h \sum_{\nu,k} b_{\nu k}^\dag b_{\alpha k} b_{\nu\ell}
  + \epsilon_{\alpha\ell} b_{\alpha\ell}.
\end{align}

\end{document}

%%% Local Variables:
%%% mode: latex
%%% TeX-master: t
%%% End:
