\documentclass[aspectratio=43]{beamer}

\setbeamertemplate{navigation symbols}{}
\addtobeamertemplate{navigation symbols}{}{%
    \usebeamerfont{footline}%
    \usebeamercolor[fg]{footline}%
    \hspace{1em}%
    \insertframenumber/\inserttotalframenumber
}
\setbeamertemplate{caption}[numbered]


%%% beamer theme
\usepackage{beamerthemelined}%
\setbeamertemplate{headline}[text line]%
{%
  \vbox{%
    \vskip3pt%
    \vskip1.5pt%
    \insertvrule{0.4pt}{beamerstructure!50!averagebackgroundcolor}%
    \vskip1.5pt%
    \tinyline{\color{beamerstructure}\insertsubsection\hfill}
    \insertvrule{4pt}{beamerstructure!50!averagebackgroundcolor}%
  }%
}
% \setbeamertemplate{footline}{} % clear footer

%%% bibliography
% normally cite with \footfullcite{-ref-}
% inside column environment, cite with
% \footnote[frame]{\fullcite{-ref-}}
% bibtex backend necessary for using a style
\usepackage[backend=bibtex,style=nature,isbn=false,url=false]{biblatex}
\addbibresource{\jobname.bib}


%%% modify indentation of footnotes
\makeatletter
\renewcommand<>\beamer@framefootnotetext[1]{%
  \global\setbox\beamer@footins\vbox{%
    \hsize\framewidth
    \textwidth\hsize
    \columnwidth\hsize
    \unvbox\beamer@footins
    \reset@font\footnotesize
    \@parboxrestore
    \protected@edef\@currentlabel
         {\csname p@footnote\endcsname\@thefnmark}%
    \color@begingroup
      \uncover#2{\@makefntext{%
        \rule\z@\footnotesep\ignorespaces\parbox[t]{.9\textwidth}{#1\@finalstrut\strutbox}\vskip1sp}}%
    \color@endgroup}%
}
\makeatother

% letters instead of numbers for marking footnotes
\renewcommand*{\thefootnote}{\alph{footnote}}


%%% symbols, notations, etc.
\usepackage{physics,braket,bm,amssymb} % physics and math
\renewcommand{\t}{\text} % text in math mode
\newcommand{\f}[2]{\dfrac{#1}{#2}} % shorthand for fractions
\newcommand{\p}[1]{\left(#1\right)} % parenthesis
\renewcommand{\sp}[1]{\left[#1\right]} % square parenthesis
\renewcommand{\set}[1]{\left\{#1\right\}} % curly parenthesis
\renewcommand{\v}{\bm} % bold vectors
\newcommand{\uv}[1]{\hat{\v{#1}}} % unit vectors
\newcommand{\bk}{\Braket} % shorthand for braket notation

\newcommand{\g}{\text{g}}
\newcommand{\e}{\text{e}}
\renewcommand{\c}{\hat{c}}
\newcommand{\n}{\hat{n}}
\renewcommand{\H}{\mathcal{H}}
\renewcommand{\P}{\mathcal{P}}


% color definitions (used in a figure)
\usepackage{xcolor}
\definecolor{orange}{RGB}{255,127,14}
\definecolor{green}{RGB}{44,160,44}
\newcommand{\mblue}[1]{\mathcolor{blue}{#1}}
\newcommand{\morange}[1]{\mathcolor{orange}{#1}}
\newcommand{\mgreen}[1]{\mathcolor{green}{#1}}

% proper coloring inside math environment
\makeatletter
\def\mathcolor#1#{\@mathcolor{#1}}
\def\@mathcolor#1#2#3{
  \protect\leavevmode
  \begingroup
    \color#1{#2}#3
  \endgroup
}
\makeatother

\usefonttheme[onlymath]{serif} % math font

% figures, ets.
\usepackage{graphicx}
\graphicspath{{./figures/}}
\usepackage[caption=false,labelformat=empty]{subfig} % subfigures
\usepackage{caption} % caption text options
\captionsetup{textfont={small}}
\captionsetup{labelformat=empty}

\DeclareGraphicsRule{.ai}{pdf}{.ai}{} % treat .ai figures as .pdf

% diagrams
\usepackage{tikz,tikz-feynman}
\tikzset{
  baseline = (current bounding box.center)
}
\usetikzlibrary{math}
\usetikzlibrary{patterns}
\usetikzlibrary{decorations.pathmorphing}
\tikzfeynmanset{
  compat = 1.1.0,
  every feynman = {/tikzfeynman/small}
}
\newcommand{\shrink}[1]{\scalebox{0.8}{#1}} % for smaller diagrams


\title{Effective multi-body SU($N$)-symmetric interactions of
  ultracold fermions on a lattice}%
\author[M. A. Perlin]{M. A. Perlin, A. Goban, R. B. Hutson,
  G. E. Marti, S. L. Campbell, J. Ye, and A. M. Rey}%
\institute[JILA]{JILA, University of Colorado at Boulder, NIST, and
  CTQM}%
\date{May 2018}

\titlegraphic{
  ~\hfill
  $\vcenter{\hbox{\includegraphics[width=2cm]{jila_logo.eps}}}$
  \hfill
  $\vcenter{\hbox{\includegraphics[width=1.5cm]{nsf_logo.ai}}}$
  \hfill
  $\vcenter{\hbox{\includegraphics[width=1.5cm]{afosr_logo.jpg}}}$
  \hfill
  $\vcenter{\hbox{\includegraphics[width=1.5cm]{darpa_logo.jpg}}}$
  \hfill~
}


\begin{document}

\frame{\titlepage
}
\begin{frame}
  \frametitle{Alkaline-earth(-like) atoms}
  \begin{itemize}
  \item Ultra-narrow optical transition
    (${}^1S_0 \leftrightarrow {}^3P_0$)
    \begin{itemize}
    \item Long lifetimes
    \end{itemize}
  \item[]

  \item ${}^1S_0$, ${}^3P_0$ orbitals decoupled from $N$ nuclear spin
    levels
    \begin{itemize}
    \item SU($N$)-symmetric interactions
    \end{itemize}
  \item[]

  \item High density, ultracold fermionic samples on a 3-D
    lattice\footnote[frame]{\fullcite{goban2018emergence}}
    \begin{itemize}
    \item Stongly-interacting regime
    \end{itemize}
  \end{itemize}
\end{frame}

\begin{frame}
  \frametitle{Motivations}
  \begin{columns}
    \begin{column}{0.65\textwidth}
      \begin{itemize}
      \item Most precise atomic clocks!
      \item[]

      \item Quantum simulation of lattice gauge
        theories\footnote[frame]{\fullcite{rico2018nuclear}}
      \item[]

      \item SU($N$) quantum magnetism
        \begin{itemize}
        \item Novel topological phases of
          matter\footnote[frame]{\fullcite{hermele2011topological}}
        \end{itemize}
      \end{itemize}
    \end{column}
    \begin{column}{0.35\textwidth}
      \begin{figure}
        \centering
        $\vcenter{\hbox{
            \includegraphics[width=0.8\columnwidth]{qcd.png}}}$
        \caption{Simulating QCD?}
      \end{figure}
    \end{column}
  \end{columns}
\end{frame}

\begin{frame}
  \frametitle{Two-body interactions of ultracold alkali-earth atoms}
  \begin{itemize}
  \item {\color{blue}{Electronic orbital}} and {\color{green}{nuclear
        spin}} and states
    \begin{itemize}
    \item Number ops.~$\n_{\mblue{s}}$, fermionic
      ops.~$\c_{\mgreen{\mu}\mblue{s}}$, $\mblue{\g}\equiv{}^1S_0$,
      $\mblue{\e}\equiv{}^3P_0$
    \end{itemize}
  \end{itemize}
  \begin{align*}
    H_{\t{int}} = \overbrace{
      \sum_{\mblue{s}\in\set{\mblue{\g},\mblue{\e}}} V_{\mblue{s}}
      \underbrace{
        \n_{\mblue{s}} \p{\n_{\mblue{s}}-1}
      }_{\substack{\t{\# pairs with}\\\t{orbital $\mblue{s}$}}}
    }^{\t{intra-orbital interactions}}
    + \overbrace{
      V_{\t{dir}} \underbrace{
        \n_{\mblue{\e}} \n_{\mblue{\g}}
      }_{\t{``direct''}}
      + V_{\t{ex}} \sum_{\mgreen{\mu},\mgreen{\nu}}
      \underbrace{
        \c_{\mgreen{\mu},\mblue{\e}}^\dag \c_{\mgreen{\nu},\mblue{\g}}^\dag
        \c_{\mgreen{\nu},\mblue{\e}} \c_{\mgreen{\mu},\mblue{\g}}
      }_{\t{``exchange''}}
    }^{\t{inter-orbital interactions}}
  \end{align*}
  \begin{itemize}
  \item Exchange $\to$ two-body eigenstates
    $\ket{\mblue{\e\g}\pm\mblue{\g\e}} \otimes
    \ket{\mgreen{\t{nuclear~state}}}$
  \item[]

  \item SU($\mgreen{N}$) symmetry $\to$ many-body interaction energies
    $E_{\mblue{\e\g\cdots\g}^\pm}$
  \end{itemize}
\end{frame}

\begin{frame}
  \frametitle{Electronic excitation spectrum: sketch}
  \begin{align*}
    \begin{tikzpicture}[
      scale = 0.45,
      trans/.style = {thick,<->,shorten >=2pt,shorten <=2pt,>=stealth}
      ]
      \tikzmath{
        \h = 1.5; % vertical scale of lattice
        \hh = 4*\h;
        \gx = 1.6;
        \ggx = 1.3;
        \egmx = \ggx+0.1;
        \egpx = \ggx-0.1;
        \gggx = 1;
        \eggmx = \gggx+0.1;
        \eggpx = \gggx-0.1;
        \gy = \h*cos(180*\gx/2);
        \ggy = \h*cos(180*\ggx/2);
        \egmy = \h*cos(180*\egmx/2);
        \egpy = \h*cos(180*\egpx/2);
        \gggy = \h*cos(180*\gggx/2);
        \eggmy = \h*cos(180*\eggmx/2);
        \eggpy = \h*cos(180*\eggpx/2);
      }
      % draw lattices and g energy levels
      \foreach \y in {0,\hh}
      \foreach \x in {0,1,2,3,4}
      {
        \draw (4*\x,\h+\y) cos (4*\x+1,0+\y) sin (4*\x+2,-\h+\y)
        cos(4*\x+3,0+\y) sin (4*\x+4,\h+\y);
        \draw (4*\x+\gx,\gy+\y) -- (4*\x+4-\gx,\gy+\y);
      }
      % draw gg and ggg energy levels
      \foreach \x in {1,2}
      {
        \draw (4*\x+\ggx,\ggy) -- (4*\x+4-\ggx,\ggy);
        \draw (4*\x+8+\gggx,\gggy) -- (4*\x+12-\gggx,\gggy);
      }
      % draw eg and egg energy levels
      \draw (4+\egmx,\egmy+\hh) -- (8-\egmx,\egmy+\hh);
      \draw (8+\egpx,\egpy+\hh) -- (12-\egpx,\egpy+\hh);
      \draw (12+\eggmx,\eggmy+\hh) -- (16-\eggmx,\eggmy+\hh);
      \draw (16+\eggpx,\eggpy+\hh) -- (20-\eggpx,\eggpy+\hh);
      % draw g/e atoms
      \filldraw[fill = blue] (2,\gy) circle (5pt);
      \filldraw[fill = red] (2,\gy+\hh) circle (5pt);
      % draw gg/eg atoms
      \filldraw[fill = blue] (6-2/3+\ggx/3,\ggy) circle (5pt);
      \filldraw[fill = blue] (6+2/3-\ggx/3,\ggy) circle (5pt);
      \filldraw[fill = blue] (6-2/3+\ggx/3,\egmy+\hh) circle (5pt);
      \filldraw[fill = red] (6+2/3-\ggx/3,\egmy+\hh) circle (5pt);
      \filldraw[fill = blue] (10-2/3+\ggx/3,\ggy) circle (5pt);
      \filldraw[fill = blue] (10+2/3-\ggx/3,\ggy) circle (5pt);
      \filldraw[fill = blue] (10-2/3+\ggx/3,\egpy+\hh) circle (5pt);
      \filldraw[fill = red] (10+2/3-\ggx/3,\egpy+\hh) circle (5pt);
      % draw ggg/egg atoms
      \filldraw[fill = blue] (14-1+\gggx/2,\gggy) circle (5pt);
      \filldraw[fill = blue] (14,\gggy) circle (5pt);
      \filldraw[fill = blue] (14+1-\gggx/2,\gggy) circle (5pt);
      \filldraw[fill = blue] (14-1+\gggx/2,\eggmy+\hh) circle (5pt);
      \filldraw[fill = blue] (14,\eggmy+\hh) circle (5pt);
      \filldraw[fill = red] (14+1-\gggx/2,\eggmy+\hh) circle (5pt);
      \filldraw[fill = blue] (18-1+\gggx/2,\gggy) circle (5pt);
      \filldraw[fill = blue] (18,\gggy) circle (5pt);
      \filldraw[fill = blue] (18+1-\gggx/2,\gggy) circle (5pt);
      \filldraw[fill = blue] (18-1+\gggx/2,\eggpy+\hh) circle (5pt);
      \filldraw[fill = blue] (18,\eggpy+\hh) circle (5pt);
      \filldraw[fill = red] (18+1-\gggx/2,\eggpy+\hh) circle (5pt);
      % draw arrows and labels for g --> e excitation
      \foreach \y in {0,\hh}
      {
        \draw[dashed] (-1,\gy+\y) -- (\gx,\gy+\y);
      }
      \draw[trans] (-1,\gy) -- (-1,\gy+\hh)
      node[midway,left]{$\omega_0$};
      % draw state labels
      \node[below] at (2,-\h) {$\ket{\g}$};
      \node[below] at (2,-\h+\hh) {$\ket{\e}$};
      \node[below] at (6,-\h) {$\ket{\g\g}$};
      \node[below] at (6,-\h+\hh) {$\ket{\e\g^-}$};
      \node[below] at (10,-\h) {$\ket{\g\g}$};
      \node[below] at (10,-\h+\hh) {$\ket{\e\g^+}$};
      \node[below] at (14,-\h) {$\ket{\g\g\g}$};
      \node[below] at (14,-\h+\hh) {$\ket{\e\g\g^-}$};
      \node[below] at (18,-\h) {$\ket{\g\g\g}$};
      \node[below] at (18,-\h+\hh) {$\ket{\e\g\g^+}$};
      % draw interaction energy labels
    \end{tikzpicture}
  \end{align*}

  ~\hfill $\omega_0=$ single-atom ${}^1S_0\to{}^3P_0$ excitation
  energy \hfill~
\end{frame}

\begin{frame}
  \frametitle{Electronic excitation spectrum: data}
  ~\vfill ~\hfill Color $=$ atom number \hfill \uncover<2>{Dashed
    lines $=$ two-body theory} \hfill~
  \begin{overlayarea}{\textwidth}{\textheight}
  \begin{figure}
    \centering
    \includegraphics<1>[width=0.8\textwidth]%
    {sweep_expt_V54_refs_1_blank.pdf}%
    \includegraphics<2>[width=0.8\textwidth]%
    {sweep_expt_V54_refs_1.pdf}
  \end{figure}
  \end{overlayarea}
\end{frame}

\begin{frame}
  \frametitle{Multi-atom interaction shifts}
  \begin{itemize}
  \item Experimental measurements inconsistent with {\bf two-body}
    theory for {\bf ground-state} atoms
  \item[]

  \item Ultracold temperatures: $T\ll E_{\t{vibration}}$
  \item[]

  \item Interaction vs. vibrational energy scales:
    $E_{\t{int}}\lesssim E_{\t{vibration}}$
  \item[]

  \item Perturbative treatment of interactions $H_{\t{int}}$
  \end{itemize}
\end{frame}

\begin{frame}
  \frametitle{Second order effects: three-body interactions}
  ~\vfill
  ~\hfill Color $=$ nuclear spin \hfill~
  \begin{align*}
    \begin{tikzpicture}
      [int/.style={decorate, decoration=snake, color=red, ultra thick}]
      \tikzmath{
        \x0 = 1.5; \y0 = 2;
        \y1 = 0.5; \y2 = 1; \y3 = 1.5;
        \a = sqrt(\y0/\x0^2);
        \x1 = sqrt(\y1/\a); \x2 = sqrt(\y2/\a); \x3 = sqrt(\y3/\a);
        \xC = 0; \xL = -\x1/2; \xR = \x1/2;
        \z1 = (\y1 + \y2) / 2;
        \z2 = (\y2 + \y3) / 2;
      }
      \draw (0,0) parabola (\x0,\y0);
      \draw (0,0) parabola (-\x0,\y0);
      \draw (-\x1,\y1) -- (\x1,\y1);
      \draw (-\x2,\y2) -- (\x2,\y2);
      \draw (-\x3,\y3) -- (\x3,\y3);
      \draw[int] (\xL,\y1) -- (\xC,\y1);
      \filldraw[fill = blue] (\xL,\y1) circle (3pt);
      \filldraw[fill = green] (\xC,\y1) circle (3pt);
      \draw[dotted] (\xC,\y3) circle (3pt);
      \draw[->] (\xC,\z1) -- (\xC,\z2);
      \filldraw[fill = orange] (\xR,\y1) circle (3pt);
    \end{tikzpicture}
    \to
    \begin{tikzpicture}
      [int/.style={decorate, decoration=snake, color=red, ultra thick}]
      \tikzmath{
        \x0 = 1.5; \y0 = 2;
        \y1 = 0.5; \y2 = 1; \y3 = 1.5;
        \a = sqrt(\y0/\x0^2);
        \x1 = sqrt(\y1/\a); \x2 = sqrt(\y2/\a); \x3 = sqrt(\y3/\a);
        \xC = 0; \xL = -\x1/2; \xR = \x1/2;
        \z1 = (\y1 + \y2) / 2;
        \z2 = (\y2 + \y3) / 2;
      }
      \draw (0,0) parabola (\x0,\y0);
      \draw (0,0) parabola (-\x0,\y0);
      \draw (-\x1,\y1) -- (\x1,\y1);
      \draw (-\x2,\y2) -- (\x2,\y2);
      \draw (-\x3,\y3) -- (\x3,\y3);
      \draw[int] (\xC,\y3) -- (\xR,\y1);
      \filldraw[fill = blue] (\xL,\y1) circle (3pt);
      \filldraw[fill = green] (\xC,\y3) circle (3pt);
      \draw[dotted] (\xC,\y1) circle (3pt);
      \draw[->] (\xC,\z2) -- (\xC,\z1);
      \filldraw[fill = orange] (\xR,\y1) circle (3pt);
    \end{tikzpicture}
    \to
    \begin{tikzpicture}
      [int/.style={decorate, decoration=snake, color=red, thick}]
      \tikzmath{
        \x0 = 1.5; \y0 = 2;
        \y1 = 0.5; \y2 = 1; \y3 = 1.5;
        \a = sqrt(\y0/\x0^2);
        \x1 = sqrt(\y1/\a); \x2 = sqrt(\y2/\a); \x3 = sqrt(\y3/\a);
        \xC = 0; \xL = -\x1/2; \xR = \x1/2;
      }
      \draw (0,0) parabola (\x0,\y0);
      \draw (0,0) parabola (-\x0,\y0);
      \draw (-\x1,\y1) -- (\x1,\y1);
      \draw (-\x2,\y2) -- (\x2,\y2);
      \draw (-\x3,\y3) -- (\x3,\y3);
      \filldraw[fill = blue] (\xL,\y1) circle (3pt);
      \filldraw[fill = green] (\xC,\y1) circle (3pt);
      \filldraw[fill = orange] (\xR,\y1) circle (3pt);
    \end{tikzpicture}
  \end{align*}
  \vfill
  \uncover<2>{
    \begin{align*}
      \begin{array}{ccc}
        \overbrace{
        \mgreen{\c}^\dag \morange{\c}^\dag
        \morange{\c} \mgreen{\hat{\mathcal{E}}}
        }^{\substack{\t{second}\\\t{``interaction''}}}
        ~
        \overbrace{
        \mblue{\c}^\dag \mgreen{\hat{\mathcal{E}}}^\dag
        \mgreen{\c} \mblue{\c}
        }^{\substack{\t{first}\\\t{``interaction''}}}
        & ~
        & \overbrace{
          \mblue{\c}^\dag \mgreen{\c}^\dag \morange{\c}^\dag
          \morange{\c} \mgreen{\c} \mblue{\c}
          }^{\substack{\t{effective ground-state}\\\t{3-body interaction}}} \\
        ~
        & \leftrightarrow
        & ~ \\
        \begin{tikzpicture}
          \begin{feynman}[small]
            \vertex (v1);
            \vertex[above left = of v1] (f1);
            \vertex[below left = of v1] (f2);
            \vertex[right = 4em of v1] (v2);
            \vertex[above right = of v1] (f3);
            \vertex[below left = of v2] (f4);
            \vertex[below right = of v2] (f5);
            \vertex[above right = of v2] (f6);
            \diagram* {
              (f1) --[fermion, color = blue] (v1)
              --[fermion, color = blue] (f3),
              (f2) --[fermion, color = green] (v1)
              --[charged scalar, color = green] (v2),
              (f4) --[fermion, color = orange] (v2)
              --[fermion, color = orange] (f5),
              (v2) --[fermion, color = green] (f6), };
          \end{feynman}
        \end{tikzpicture}
        & ~
        & \begin{tikzpicture}
          \begin{feynman}
            \vertex[blob] (v) {};
            \vertex[above left = of v] (f1);
            \vertex[left = of v] (f2);
            \vertex[below left = of v] (f3);
            \vertex[above right = of v] (f4);
            \vertex[right = of v] (f5);
            \vertex[below right = of v] (f6);
            \diagram* {
              (f1) --[fermion, color = blue] (v)
              --[fermion, color = blue] (f4),
              (f2) --[fermion, color = green] (v)
              --[fermion, color = green] (f5),
              (f3) --[fermion, color = orange] (v)
              --[fermion, color = orange] (f6) };
          \end{feynman}
        \end{tikzpicture}
      \end{array}
    \end{align*}
  }
\end{frame}

\begin{frame}
  \frametitle{Higher order effects: diagram zoo}
  ~\vfill
  3-body
  \hfill
  $\sim$
  \hfill
  $\shrink{
    \begin{tikzpicture}
      \begin{feynman}
        \vertex (v1);
        \vertex[below right = of v1] (v2);
        \vertex[above right = of v2] (v3);
        \vertex[above left = of v1] (f1);
        \vertex[left = of v1] (f2);
        \vertex[below left = of v2] (f3);
        \vertex[above right = of v3] (f4);
        \vertex[right = of v3] (f5);
        \vertex[below right = of v2] (f6);
        \diagram* {
          (f1) -- (v1) --[scalar] (v3) -- (f4),
          (f2) -- (v1) --[scalar] (v2) --[scalar] (v3) -- (f5),
          (f3) -- (v2) -- (f6), };
      \end{feynman}
    \end{tikzpicture}}$
  \hfill
  $\shrink{
    \begin{tikzpicture}
      \begin{feynman}
        \vertex (v1);
        \vertex[right = of v1] (v2);
        \vertex[below right = of v2] (v3);
        \vertex[above left = of v1] (f1);
        \vertex[below left = of v1] (f2);
        \vertex[below left = of v3] (f3);
        \vertex[above right = of v2] (f4);
        \vertex[above right = of v3] (f5);
        \vertex[below right = of v3] (f6);
        \diagram* {
          (f1) -- (v1) --[half left, scalar] (v2) -- (f4),
          (f2) -- (v1) --[half right, scalar] (v2) -- (v3) -- (f5),
          (f3) -- (v3) -- (f6), };
      \end{feynman}
    \end{tikzpicture}}$
  \hfill~
  \vfill
  4-body
  \hfill
  $\sim$
  \hfill
  $\shrink{
    \begin{tikzpicture}
      \begin{feynman}
        \vertex (v1);
        \vertex[above right = 1.5em of v1] (v2);
        \vertex[below right = 2.5em of v1] (v3);
        \vertex[above left = of v1] (f1);
        \vertex[below left = of v1] (f2);
        \vertex[above left = of v2] (f3);
        \vertex[below left = of v3] (f4);
        \vertex[above right = of v2] (f5);
        \vertex[right = of v2] (f6);
        \vertex[right = of v3] (f7);
        \vertex[below right = of v3] (f8);
        \diagram* {
          (f1) -- (v1) --[scalar] (v2) -- (f6),
          (f2) -- (v1) --[scalar] (v3) -- (f7),
          (f3) -- (v2) -- (f5),
          (f4) -- (v3) -- (f8),
        };
      \end{feynman}
    \end{tikzpicture}}$
  \hfill
  $\shrink{
    \begin{tikzpicture}
      \begin{feynman}
        \vertex (v1);
        \vertex[below right = 2.5em of v1] (v3);
        \vertex[below left = 1.7em of v3] (v2);
        \vertex[above left = of v1] (f1);
        \vertex[left = of v1] (f2);
        \vertex[left = of v2] (f3);
        \vertex[below left = of v2] (f4);
        \vertex[above right = of v1] (f5);
        \vertex[above right = of v3] (f6);
        \vertex[below right = of v3] (f7);
        \vertex[below right = of v2] (f8);
        \diagram* {
          (f1) -- (v1) -- (f5),
          (f2) -- (v1) -- (v3) -- (f6),
          (f3) -- (v2) --[scalar] (v3) -- (f7),
          (f4) -- (v2) -- (f8), };
      \end{feynman}
    \end{tikzpicture}}$
  \hfill
  $\shrink{
    \begin{tikzpicture}
      \begin{feynman}
        \vertex (v1);
        \vertex[below right = of v1] (v2);
        \vertex[below right = of v2] (v3);
        \vertex[above left = of v1] (f1);
        \vertex[below left = of v1] (f2);
        \vertex[below left = of v2] (f3);
        \vertex[below left = of v3] (f4);
        \vertex[above right = of v1] (f5);
        \vertex[above right = of v2] (f6);
        \vertex[above right = of v3] (f7);
        \vertex[below right = of v3] (f8);
        \diagram* {
          (f1) -- (v1) -- (f5),
          (f2) -- (v1) -- (f5),
          (v1) --[scalar] (v2) --[scalar] (v3) -- (f8),
          (f3) -- (v2) -- (f6),
          (f4) -- (v3) -- (f7),
        };
      \end{feynman}
    \end{tikzpicture}}$
  \hfill~
  \vfill
  \uncover<2->{
    % \begin{align*}
    %   H_{\t{int}}^{\t{eff}} = \sum_{M\ge2} H_{M\t{-body}}
    %   &&
    %   \t{No new free parameters!}
    % \end{align*}
    \begin{align*}
      H_{M\t{-body}} =
      \underbrace{
        \sum_{\abs{\set{\mgreen{\mu_j}}}=M}
      }_{\substack{\t{all nuclear spin}\\\t{combinations}}}
      \underbrace{
        H_{\mgreen{\mu_1},\mgreen{\mu_2}}^{(M)}
      }_{\substack{\t{2-body}\\\t{interaction}}}
      \times
      \underbrace{
        \n_{\mgreen{\mu_3},\mblue{\g}} \cdots \n_{\mgreen{\mu_M},\mblue{\g}}
      }_{\substack{\t{ground-state}\\\t{spectator atoms}}}
      &&
      \t{\uncover<3>{No free parameters!}}
    \end{align*}
  }
\end{frame}

\begin{frame}
  \frametitle{Electronic excitation spectrum: theory\footnote{(To
      appear on arXiv soon)} and
    experiment\footfullcite{goban2018emergence}}
  \only<1>{
    \begin{figure}
      \centering
      \subfloat[Two-body theory]
      {\includegraphics[width=0.49\textwidth]
        {sweep_expt_V54_refs_1.pdf}}
      \subfloat[Multi-body theory (third order)]
      {\includegraphics[width=0.49\textwidth]
        {sweep_expt_V54_refs_3.pdf}}
    \end{figure}
  }
  \only<2>{\includegraphics[width=\textwidth]{shifts_summary.pdf}}
\end{frame}

\begin{frame}
  \frametitle{Summary and outlook}
  \begin{itemize}
  \item Optical lattice experiments with alkaline-earth atoms:
    low-temperature, high-density, strongly-interacting limit
  \item[]

  \item Multi-body SU($N$)-symmetric interactions
    \begin{itemize}
    \item Low-energy effective theory
    \end{itemize}
  \item[]

  \item Future directions
    \begin{itemize}
    \item Shallow lattice, super-exchange dynamics
    \end{itemize}
  \end{itemize}
  \vfill
  \uncover<2->{~\hfill Experimental side: Q03:3 Thu.~Grand B}
  \vfill
  \uncover<3>{~\hfill {\bf Thank you} \hfill~}
\end{frame}

\end{document}

Emmanuel

Wolfram Ketter
- Mott insulator shell structure
