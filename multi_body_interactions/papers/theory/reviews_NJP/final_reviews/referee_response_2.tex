\documentclass[preprint,showkeys,nofootinbib]{revtex4-1}


% linking references
\usepackage{hyperref}
\hypersetup{
  breaklinks=true,
  colorlinks=true,
  linkcolor=blue,
  urlcolor=cyan,
}


% general physics / math packages and commands
\usepackage{physics,amsmath,amssymb,braket,dsfont}
\renewcommand{\t}{\text} % text in math mode
\newcommand{\f}{\dfrac} % shorthand for fractions
\newcommand{\p}[1]{\left(#1\right)} % parenthesis
\renewcommand{\sp}[1]{\left[#1\right]} % square parenthesis
\renewcommand{\set}[1]{\left\{#1\right\}} % curly parenthesis
\newcommand{\bk}{\braket} % shorthand for braket

\renewcommand{\d}{\text{d}}
\newcommand{\g}{\text{g}}
\newcommand{\e}{\text{e}}
\newcommand{\x}{\text{x}}
\newcommand{\y}{\text{y}}
\newcommand{\z}{\text{z}}

\renewcommand{\c}{\hat{c}}
\newcommand{\n}{\hat{n}}

\newcommand{\A}{\mathcal{A}}
\newcommand{\B}{\mathcal{B}}
\newcommand{\D}{\mathcal{D}}
\newcommand{\E}{\mathcal{E}}
\newcommand{\G}{\mathcal{G}}
\renewcommand{\H}{\mathcal{H}}
\newcommand{\I}{\mathcal{I}}
\newcommand{\K}{\mathcal{K}}
\renewcommand{\L}{\mathcal{L}}
\newcommand{\M}{\mathcal{M}}
\newcommand{\N}{\mathcal{N}}
\renewcommand{\O}{\mathcal{O}}
\renewcommand{\P}{\mathcal{P}}
\newcommand{\Q}{\mathcal{Q}}
\renewcommand{\S}{\mathcal{S}}
\newcommand{\U}{\mathcal{U}}

\newcommand{\1}{\mathds{1}}

\newcommand{\mA}{m_{\text{A}}} % symbol for the mass of an atom


% "left vector" arrow; requires tikz package
\newcommand{\lvec}[1]
{\reflectbox{\ensuremath{\vec{\reflectbox{\ensuremath{#1}}}}}}


% figures
\usepackage{graphicx} % for figures
\usepackage{grffile} % help latex properly identify figure extensions
\graphicspath{{./figures/}} % set path for all figures
\usepackage[caption=false]{subfig} % subfigures (via \subfloat[]{})


% inline lists
\usepackage[inline]{enumitem}


% for feynman diagrams
\usepackage{tikz,tikz-feynman}
\tikzset{
  baseline = (current bounding box.center)
}
\tikzfeynmanset{
  compat = 1.1.0,
  every feynman = {/tikzfeynman/small}
}
\newcommand{\shrink}[1]{\scalebox{0.8}{#1}} % for smaller diagrams


% color definitions (used in a figure)
\usepackage{xcolor}
\definecolor{lightblue}{RGB}{31,119,180}
\definecolor{orange}{RGB}{255,127,14}
\definecolor{green}{RGB}{44,160,44}
\definecolor{lightred}{RGB}{214,39,40}


% proper coloring inside math environment
\makeatletter
\def\mathcolor#1#{\@mathcolor{#1}}
\def\@mathcolor#1#2#3{
  \protect\leavevmode
  \begingroup
    \color#1{#2}#3
  \endgroup
}
\makeatother
\newcommand{\bmu}{\mathcolor{lightblue}{\mu}}
\newcommand{\onu}{\mathcolor{orange}{\nu}}
\newcommand{\grho}{\mathcolor{green}{\rho}}
\newcommand{\re}{\mathcolor{lightred}{\text{e}}}


% leave a note in the text, visible in the compiled document
\newcommand{\note}[1]{\textcolor{red}{#1}}

% for strikeout text
% normalem included to prevent underlining titles in the bibliography
\usepackage[normalem]{ulem}


\newcommand{\blue}[1]{\textcolor{blue}{#1}}
\newcommand{\red}[1]{\textcolor{red}{#1}}
\newcommand{\green}[1]{\textcolor{green}{#1}}


\begin{document}

\section*{Referee 2}

We thank the referee for taking the time to read and reconsider our
revised manuscript.  Concerning the comments:

\blue{I read the replies from the authors and other report and
  recommend the publication.}

\blue{There are a few discussions on ``renormalization'' schemes in
  the article and in Referee 1's report.  Let me comment on the scheme
  in the manuscript. The scheme used here (more precisely
  ``regularization'' scheme) is a quite standard and has been commonly
  used in many-body studies since the beginning of many-body
  diagrammatic calculations. It is based on a k-space ``pseudo''
  potential.  From the point of view of low energy few-body physics
  studied here, it is indeed equivalent to the other well known
  standard real space ``differential pseudo'' potential Referee A
  might have in mind (I speculate that what he/she meant by ``pseudo
  potential''). But more importantly, it can be easily and
  straightforwardly generalized to arbitrary dimensions and can be
  integrated into studies of complex/sophisticated collective
  many-body effects associated with Fermi surfaces and/or sound waves
  in symmetry breaking states.}

We thank the referee for their recommendation to publish our
manuscript, and for their input to help clarify a point of confusion
in our exchanges with referee 1.  While our original response to
referee 1 was imprecise in conflating the term ``renormalization
scheme'' with ``regularization scheme'', we have gone through to make
sure that our manuscript uses these terms properly.

\end{document}
