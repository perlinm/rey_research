\documentclass[preprint]{revtex4-1}

% linking references
\usepackage{hyperref}
\hypersetup{
  breaklinks=true,
  colorlinks=true,
  linkcolor=blue,
  urlcolor=cyan,
}

% general physics / math packages and commands
\usepackage{physics,amsmath,amssymb,braket,dsfont}
\renewcommand{\t}{\text} % text in math mode
\newcommand{\f}{\dfrac} % shorthand for fractions
\newcommand{\p}[1]{\left(#1\right)} % parenthesis
\renewcommand{\sp}[1]{\left[#1\right]} % square parenthesis
\renewcommand{\set}[1]{\left\{#1\right\}} % curly parenthesis
\newcommand{\bk}{\braket} % shorthand for braket

\renewcommand{\d}{\text{d}}
\newcommand{\g}{\text{g}}
\newcommand{\e}{\text{e}}
\newcommand{\x}{\text{x}}
\newcommand{\y}{\text{y}}
\newcommand{\z}{\text{z}}

\renewcommand{\c}{\hat{c}}
\newcommand{\n}{\hat{n}}

\newcommand{\A}{\mathcal{A}}
\newcommand{\B}{\mathcal{B}}
\newcommand{\D}{\mathcal{D}}
\newcommand{\E}{\mathcal{E}}
\newcommand{\G}{\mathcal{G}}
\renewcommand{\H}{\mathcal{H}}
\newcommand{\I}{\mathcal{I}}
\newcommand{\K}{\mathcal{K}}
\renewcommand{\L}{\mathcal{L}}
\newcommand{\M}{\mathcal{M}}
\newcommand{\N}{\mathcal{N}}
\renewcommand{\O}{\mathcal{O}}
\renewcommand{\P}{\mathcal{P}}
\newcommand{\Q}{\mathcal{Q}}
\renewcommand{\S}{\mathcal{S}}
\newcommand{\U}{\mathcal{U}}
\newcommand{\1}{\mathds{1}}

% inline lists
\usepackage[inline]{enumitem}
\setlist[enumerate]{leftmargin=*} % nice margins for enumerate


% for feynman diagrams
\usepackage{tikz,tikz-feynman}
\tikzset{
  baseline = (current bounding box.center)
}
\tikzfeynmanset{
  compat = 1.1.0,
  every feynman = {/tikzfeynman/small}
}
\newcommand{\shrink}[1]{\scalebox{0.8}{#1}} % for smaller diagrams


% color definitions (used in a figure)
\usepackage{xcolor}
\newcommand{\blue}[1]{\textcolor{blue}{#1}}
\newcommand{\red}[1]{\textcolor{red}{#1}}
\newcommand{\green}[1]{\textcolor{green}{#1}}
\definecolor{lightblue}{RGB}{31,119,180}
\definecolor{orange}{RGB}{255,127,14}
\definecolor{green}{RGB}{44,160,44}
\definecolor{lightred}{RGB}{214,39,40}


% for strikeout text
% normalem included to prevent underlining titles in the bibliography
\usepackage[normalem]{ulem}


\begin{document}

\section*{Referee 2}

We thank the referee for taking the time to carefully read our
manuscript, and for providing thoughtful and constructive feedback.
Below, we address the comments and concerns brought up by the referee.
We hope that the referee finds our responses and revisions
satisfactory, deeming our manuscript acceptable for publication in New
Journal of Physics.

Excerpts from the referee reports are written in \blue{blue}, excerpts
from our original manuscript are written in \red{red}, and excerpts
from the revised version of our manuscript is written in
\green{green}.  All page, line, equation, and reference numbers
generally refer to those in our original manuscript, unless explicitly
stated otherwise.


\begin{enumerate}
\item Concerning:

  \blue{Largely motivated by the experimental spectroscopy data on
    87Sr atoms, the authors of manuscript NJP-109533 have studied
    effective few-body interactions in mesoscopic Sr atomic clusters
    confined by optical lattice potentials and applied the effective
    Hamiltonian to characterize the spectra of interacting Sr atoms.
    The authors have also compared the result of theory with data and
    shown general consistency.}

  \blue{The main results are summarized in section III and IV.  In
    section II, the authors presented the detailed discussions on the
    effective interactions via the standard Schrieffer-Wolff
    transformation. The low energy manifold where the orbital motion
    is fixed in the ground state is separated from excited manifolds
    involving excited orbitals. The low energy physics therefore is
    purely driven by the internal SU(N) spin dynamics and the excited
    orbitals will further induce few-body interaction between SU(N)
    spins.  These can be systematically studied at least in the weakly
    interacting systems or in the perturbative limit where the single
    particle orbital motion is well gapped and Schrieffer-Wolff
    transformation is applicable. In section IV, the authors benched
    their numerical results against the data on 87Sr and emphasized
    the situation where the induced effective interactions might play
    a substantial role.}

  \blue{I find the article is well written and results have been
    carefully organized and are well presented.  Although at
    qualitatively level, these higher order effects are relatively
    small (not surprising as perturbative effects) and do not modify
    the results in a drastic way, they do offer better agreements with
    experimental data. And in the extreme cases in Fig.5 (3+, 4+, 5+
    lines), numerical values of these effects o(G\^{}3) can be as
    larger as 10-20\% (to my bare eyes) although in this limit it
    becomes less clear whether the contributions not included are also
    substantial.}

  \blue{A few suggestions the authors can take into account to improve
    the presentation.}

  We thank the referee for their summary of our work, and for
  remarking that our ``article is well written'', and that our
  ``results have been carefully organized are well presented''.  We
  hope that the revisions we outline below address the referee's
  suggestions to a satisfactory degree.


\item Concerning:

  \blue{1) In the caption of Fig 1, the authors shall include the
    dashed solid line definitions to make it self-contained.}

  We thank the referee for the suggestion to make figure 1 more
  self-contained.  Following this suggestion, we have added the
  definition of solid and dashed lines to the caption of figure 1, as
  well as a sentence describing how to translate a diagram into a
  sequence of operators.  In total, we have replaced the text (page
  11, line 13):

  \red{... $n>0$ and $\c_{\mu s}\equiv\c_{0,\mu s}$.  For the sake
    presentation, ...}

  by:

  \green{... $n>0$ and $\c_{\mu s}\equiv\c_{0,\mu s}$.  Diagrams are
    read from left to right to construct a sequence of operators from
    right to left.  Solid (dashed) lines represent field operators
    acting on the lowest (arbitrary) motional states.  For the sake
    presentation, ...}


\item Concerning the closely related comments/questions::

  \blue{2) Section IV, the author shall add discussions about why some
    states are more affected than the others by the 4-body interaction
    etc.  Data are clearly presented and well compared with
    experiments but there seems to be a lack of central message. What
    is the main lesson? After all, they are high order effects so
    naively and logically one would say they play little role in the
    system. But on other hand, numerically in some cases they appear
    to be amplified up to 20\%, not so small anymore.  Is it
    accidental for 3+, 4+, 5+ states or there are some more
    fundamental reasons for that?}

  \blue{3) Related to 2), when the correction becomes 10-20\%, shall
    we expect that higher order effects of order of o(G\^{}4) etc also
    become important?}

  We thank the referee for their careful reading of our results in
  section IV, and for pointing out that our manuscript would greatly
  benefit from a deeper discussion of figures 4 and 5.  The referee
  asks questions that follow naturally from an inspection of these
  figures, which our manuscript should address directly.

  Indeed, there is a good reason why multi-body interaction energy
  corrections are more prominent for the symmetric ($3+$, $4+$, $5+$)
  states than for the asymmetric ($3-$, $4-$, $5-$) ones.
  Specifically, there is a competition between contributions of
  opposite sign to the interaction energy of asymmetric states.
  Concerning higher-order effects, it is not obvious when exactly
  further corrections ($\sim G^4$) should become relevant.  Computing
  higher-order corrections requires more systematic methods than those
  used in our manuscript, and we are simply unsure whether considering
  these corrections is worthwhile without first performing a more
  detailed analysis of systematic errors in both theory and
  experiment.

  To address these questions and add a more pointed discussion of the
  lessons from figures 4 and 5, we have rearranged, rewritten, and
  added new text to the following two paragraphs (page 23, line 54):

  \red{Identifying peaks in excitation spectra such as in figure 3
    constitutes a measurement of many-body excitation energies, which
    was performed in ref.~[34] to detect signatures of effective
    multi-body interactions.  Figure 4 shows a comparison between
    experimental measurements of the many-body excitation energies
    $\Delta_{NX}$ for $N\in\set{3,4,5}$ at various lattice depths
    $\U$, and the corresponding values of $\Delta_{NX}$ predicted by
    the low-energy effective theory at different orders in the
    coupling constants.  A summary of the data in figure 4 is provided
    in figure 5.  These figures show a clear improvement of the
    low-energy effective theory we have developed with increasing
    order in the coupling constants, in particular demonstrating the
    need to consider multi-body interactions in a single-band model of
    the atoms without motional excitations.}

  \red{We expect that the dominant source of error in the third-order
    effective theory comes from the fact that we neglect the
    inter-site matrix elements of all Hamiltonians.  This source of
    error is discussed in Appendix E, and leads to theoretical
    uncertainties represented by error bars in figure 4.  As these
    uncertainties are sometimes unable to account for the residual
    disagreement between experimental measurements and the effective
    theory, better agreement requires either a better understanding of
    experimental procedures and error, a better understanding of
    theoretical uncertainties, or consideration of yet higher-order
    terms in the effective theory.  Due to a combinatorial explosion
    in the number of diagrams which appear at increasing perturbative
    orders of the effective theory, we need more systematic methods to
    compute effective multi-body Hamiltonians at fourth order; we
    therefore leave this calculation for future work.}

  which now read:

  \green{Identifying peaks in excitation spectra such as in figure 3
    constitutes a measurement of many-body excitation energies, which
    was performed in ref.~[34] to detect signatures of effective
    multi-body interactions.  Figure 4 shows a comparison between (i)
    experimental measurements of the many-body excitation energies
    $\Delta_{NX}$ for all $\p{N,X}\in\set{3,4,5}\times\set{+,-}$ at
    various lattice depths $\U$, and (ii) the corresponding values of
    $\Delta_{NX}$ predicted by the low-energy effective theory at
    different orders in the coupling constants.  A known source of
    error in our effective theory comes from neglecting the inter-site
    matrix elements of all Hamiltonians.  This error is discussed in
    Appendix E, and leads to theoretical uncertainties represented by
    error bars on the $\O\p{G^3}$ theory in figure 4.  A summary of
    figure 4 is provided in figure 5.  We note that many-body
    interaction energy shifts are smaller for asymmetric ($-$) states
    than symmetric ($+$) ones due to the competition between
    contributions of opposite sign in the asymmetric case (see rows 2
    and 3 of table I).}

  \green{The results in figures 4 and 5 highlight a few important
    points about ultracold, high-density ${}^{87}$Sr experiments and
    our low-energy effective theory.  First, these experiments exhibit
    clear signatures of multi-body interactions, as evidenced by a
    stark disagreement between the observed many-body excitation
    energies $\Delta_{NX}$ and those that are predicted by the
    two-body $\O\p{G}$ theory.  Multi-body interactions are thus
    crucial for understanding these high-density experiments in the
    context of a single-band Hubbard model, which naturally arises in
    the zero-temperature limit when all atoms occupy their motional
    ground state.  Second, the inter-atomic interactions in these
    experiments are strong enough to require going beyond leading
    order for the description of multi-body interactions in the
    low-energy effective theory.  As experiments begin to work with
    more atoms and shallower lattices, therefore, reliably predicting
    interaction energies may require going to yet higher orders in
    perturbation theory.  Due to a combinatorial explosion of the
    number of diagrams which appear at increasing orders in the
    effective theory, however, we need more systematic methods to
    compute effective multi-body Hamiltonians at fourth order.  In any
    case, we are agnostic as to whether such a calculation would
    provide better agreement between experiment and theory without
    first performing a more detailed analysis of systematic errors.}

  We hope that this revision has clarified the questions brought up by
  the referee, and provided a central message for the results in
  figures 4 and 5.


\item Concerning:

  \blue{4) At a phenomenological level, the author shall describe, in
    connection to the experiment, the range of the perturbation
    parameter [ a\_X K/Delta] that measures the two-particle
    interaction energy. For Sr atoms with typical background
    scattering lengths, this is usually small but does depend on the
    lattice depth quite sensitively.  The authors shall provide the
    typical numerical range of this dimensionless interaction energy
    for the optical lattices (up to 70E\_R in Fig 5) they studied (and
    experimentally studied). This will help readers to better
    appreciate their numerical results (surprising or anticipated).}




\end{enumerate}

\end{document}
