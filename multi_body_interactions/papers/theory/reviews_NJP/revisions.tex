\documentclass[preprint]{revtex4-1}

% linking references
\usepackage{hyperref}
\hypersetup{
  breaklinks=true,
  colorlinks=true,
  linkcolor=blue,
  urlcolor=cyan,
}

% general physics / math packages and commands
\usepackage{physics,amsmath,amssymb,braket,dsfont,bm}
\renewcommand{\t}{\text} % text in math mode
\newcommand{\f}{\dfrac} % shorthand for fractions
\newcommand{\p}[1]{\left(#1\right)} % parenthesis
\renewcommand{\sp}[1]{\left[#1\right]} % square parenthesis
\renewcommand{\set}[1]{\left\{#1\right\}} % curly parenthesis
\newcommand{\bk}{\braket} % shorthand for braket

\newcommand{\g}{\text{g}}
\newcommand{\e}{\text{e}}
\newcommand{\x}{\text{x}}
\newcommand{\y}{\text{y}}
\newcommand{\z}{\text{z}}

\renewcommand{\c}{\hat{c}}
\newcommand{\n}{\hat{n}}

\newcommand{\A}{\mathcal{A}}
\newcommand{\B}{\mathcal{B}}
\newcommand{\D}{\mathcal{D}}
\newcommand{\E}{\mathcal{E}}
\newcommand{\G}{\mathcal{G}}
\renewcommand{\H}{\mathcal{H}}
\newcommand{\I}{\mathcal{I}}
\newcommand{\K}{\mathcal{K}}
\renewcommand{\L}{\mathcal{L}}
\newcommand{\M}{\mathcal{M}}
\newcommand{\N}{\mathcal{N}}
\renewcommand{\O}{\mathcal{O}}
\renewcommand{\P}{\mathcal{P}}
\newcommand{\Q}{\mathcal{Q}}
\renewcommand{\S}{\mathcal{S}}
\newcommand{\U}{\mathcal{U}}
\newcommand{\1}{\mathds{1}}

% inline lists
\usepackage[inline]{enumitem}
\setlist[enumerate,1]{label={(\roman*)}}

% color definitions (used in a figure)
\usepackage{xcolor} \newcommand{\blue}[1]{\textcolor{blue}{#1}}
\newcommand{\red}[1]{\textcolor{red}{#1}}
\newcommand{\green}[1]{\textcolor{green}{#1}}
\definecolor{lightblue}{RGB}{31,119,180}
\definecolor{orange}{RGB}{255,127,14}
\definecolor{green}{RGB}{44,160,44}
\definecolor{lightred}{RGB}{214,39,40}


% for strikeout text
% normalem included to prevent underlining titles in the bibliography
\usepackage[normalem]{ulem}


\begin{document}

Dear editors,

In this document, we summarize and comment on all revisions made to
our manuscript since its initial submission.  Revisions made in
response to comments and recommendations made by Referee 1 (Referee 2)
are numbered R1.$X$ (R2.$X$).  In addition, we summarize a few minor
revisions made to refine wording or fix errors in the original
manuscript; these revisions are numbered R3.$X$.

Excerpts from the referee reports are written in \blue{blue}, excerpts
from our original manuscript are written in \red{red}, and new text in
the revised version of our manuscript is written in \green{green}.
All page, line, and reference numbers generally refer to those in our
original manuscript.


\section*{Revisions prompted by Referee 1}

\begin{enumerate}[label=(R1.\arabic*)]
\item
\end{enumerate}


\section*{Revisions prompted by Referee 2}

\begin{enumerate}[label=(R2.\arabic*)]
\item
\end{enumerate}


\section*{Miscellaneous revisions}

\begin{enumerate}[label=(R3.\arabic*)]
\item Page 3, line 24 of our original manuscript previously said:

  \red{Last year (2017), a new generation of OLCs became operational
    at JILA}

  Reflecting the fact that it is now 2019, the new version of the
  manuscript says:

  \green{More recently (2017), a new generation of OLCs became
    operational at JILA}


\item Page 5, line 40 of our original manuscript previously said:

  \red{motional state $n\in\mathbb{N}_0^3$ with nuclear spin
    $\mu\in\set{-I,I+1,\cdots,I}$}

  This text contains a sign error; in the version of our manuscript,
  this text reads:

  \green{motional state $n\in\mathbb{N}_0^3$ with nuclear spin
    $\mu\in\set{-I,-I+1,\cdots,I}$}


\item Page 5, lines 43-44 of our original manuscript previously said:

  \red{In a harmonic trap approximation we would have
    $E_n=\p{3/2+n_x+n_y+n_z}\omega$}

  For clarity, we have modified this text so that all subscripts
  denoting free variables (e.g.~the $n$ in $E_n$, as $n$ can here be
  any element of $\mathbb{N}_0^3$) are italic, while all subscripts
  denoting fixed symbols (e.g.~the $x$ in $n_x$, as here $x$ denotes a
  fixed spatial axis) are in roman font.  The new text thus reads:

  \green{In a harmonic trap approximation we would have
    $E_n=\p{3/2+n_\x+n_\y+n_\z}\omega$}

  We have similarly replaced all appearances of $m_A$ (denoting the
  mass of a single atom) with $m_{\text{A}}$; all $E_R$ (denoting the
  lattice photon recoil energy) with $E_{\t{R}}$; all
  $k_L,k_L^x,k_L^y,k_L^z$ (denoting the wavenumber of the lattice
  laser and its projection onto three principal axes) by
  $k_{\text{L}}, k_{\text{L}}^\x, k_{\text{L}}^\y, k_{\text{L}}^\z$;
  all $\U_x,\U_y,\U_z$ (denoting lattice depths along principal axes)
  respectively by $\U_\x,\U_\y,\U_\z$; and all
  $T_\mu^z,T_\mu^x,S_{\N X}^z,S_{\N X}^x$ (denoting single-atom and
  many-body pseudospin operators) respectively by
  $T_\mu^\z,T_\mu^\x,S_{\N X}^\z,S_{\N X}^\x$.

  Finally, we have fixed a few discrepancies in our general convention
  that multiple subscripts on a given mathematical symbol should only
  be separated by a comma if at least one of these subscripts has a
  fixed value (as e.g.~in $K^{k\ell}_{mn}$, $K^{\ell,3}_{2,5}$,
  $\c_{\mu n}$, and $\c_{\mu,\g}$).  To this end, in our last appendix
  (page 40, line 17) we have replaced all instances of $U_{M,X}$ and
  $U_{M,X}^{(p)}$ respectively with $U_{MX}$ and $U_{MX}^{(p)}$.


\item On page 7, line 17 we previously had the text:

  \red{The existence of such a gauge is guaranteed by the analytic
    properties of Wannier orbitals $\phi_{in}$ [42].}

  With the exception of this sentence, our manuscript has generally
  reserved the word ``orbital'' to refer to electronic degrees of
  freedom of atoms, and ``motional state'' to refer to atoms' spatial
  degrees of freedom.  To avoid potential confusion, we have therefore
  modified the above sentence to read:

  \green{The existence of such a gauge is guaranteed by the analytic
    properties of Wannier wavefunctions $\phi_{in}$ [42].}


\item The last two lines of page 10 and line 26 of page 11 previously
  said:

  \red{Effective $M$-body interaction terms at order $p$ in
    $H_{\t{int}}$ can be represented by directed, weakly connected
    graphs (i.e.~which are connected when ignoring edge direction)
    containing $p$ internal vertices.}

  Though the final expressions in our perturbation theory can indeed
  be represented by weakly connected graphs, intermediate terms that
  cancel out may be represented by graphs that are not connected, as
  e.g.~in Eq.~(35) on page 17.  For clarity, we have therefore
  modified the above text to read:

  \green{Effective $M$-body interaction terms at order $p$ in
    $H_{\t{int}}$ can be represented by directed graphs containing $p$
    internal vertices.}


\item To clean up repetitive wording, we have replaced the sentence
  (page 12, line 45):

  \red{To leading order in the coupling constants, it follows that the
    counter-terms are second order in the coupling constants,
    i.e. $\tilde G\sim G^2$.}

  with:

  \green{To leading order, it follows that the counter-terms are
    second order in the coupling constants, i.e.~$\tilde G\sim G^2$.}


\item To make our first appendix more self-contained, we have defined
  $\P_0$ therein, such that the text (page 30, line 38):

  \red{Letting $\Q_0\equiv1-\P_0$ denote the projector onto $\E_0$ and
    $X$ ...}

  now reads:

  \green{Letting $\P_0$ denote the projector onto $\G_0$,
    $\Q_0\equiv\1-\P_0$ denote the projector onto $\E_0$, and $X$ ...}


\item Reference 34 of the original manuscript has been published since
  the original submission of our manuscript.  We have updated this
  reference accordingly in the bibliography of our revised manuscript.

\end{enumerate}

\end{document}