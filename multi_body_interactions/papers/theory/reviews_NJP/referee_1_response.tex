\documentclass[preprint]{revtex4-1}

% linking references
\usepackage{hyperref}
\hypersetup{
  breaklinks=true,
  colorlinks=true,
  linkcolor=blue,
  urlcolor=cyan,
}

% general physics / math packages and commands
\usepackage{physics,amsmath,amssymb,braket,dsfont,bm}
\renewcommand{\t}{\text} % text in math mode
\newcommand{\f}{\dfrac} % shorthand for fractions
\newcommand{\p}[1]{\left(#1\right)} % parenthesis
\renewcommand{\sp}[1]{\left[#1\right]} % square parenthesis
\renewcommand{\set}[1]{\left\{#1\right\}} % curly parenthesis
\newcommand{\bk}{\braket} % shorthand for braket

\newcommand{\g}{\text{g}}
\newcommand{\e}{\text{e}}
\newcommand{\x}{\text{x}}
\newcommand{\y}{\text{y}}
\newcommand{\z}{\text{z}}

\renewcommand{\c}{\hat{c}}
\newcommand{\n}{\hat{n}}

\newcommand{\A}{\mathcal{A}}
\newcommand{\B}{\mathcal{B}}
\newcommand{\D}{\mathcal{D}}
\newcommand{\E}{\mathcal{E}}
\newcommand{\G}{\mathcal{G}}
\renewcommand{\H}{\mathcal{H}}
\newcommand{\I}{\mathcal{I}}
\newcommand{\K}{\mathcal{K}}
\renewcommand{\L}{\mathcal{L}}
\newcommand{\M}{\mathcal{M}}
\newcommand{\N}{\mathcal{N}}
\renewcommand{\O}{\mathcal{O}}
\renewcommand{\P}{\mathcal{P}}
\newcommand{\Q}{\mathcal{Q}}
\renewcommand{\S}{\mathcal{S}}
\newcommand{\U}{\mathcal{U}}
\newcommand{\1}{\mathds{1}}

% inline lists
\usepackage[inline]{enumitem}
\setlist[enumerate]{leftmargin=*} % nice margins for enumerate


% color definitions (used in a figure)
\usepackage{xcolor}
\newcommand{\blue}[1]{\textcolor{blue}{#1}}
\newcommand{\red}[1]{\textcolor{red}{#1}}
\newcommand{\green}[1]{\textcolor{green}{#1}}
\definecolor{lightblue}{RGB}{31,119,180}
\definecolor{orange}{RGB}{255,127,14}
\definecolor{green}{RGB}{44,160,44}
\definecolor{lightred}{RGB}{214,39,40}


% for strikeout text
% normalem included to prevent underlining titles in the bibliography
\usepackage[normalem]{ulem}


\begin{document}

\section*{Referee 1}

We thank the referee for taking the time to carefully read our
manuscript, and for providing thoughtful and constructive feedback.
Below, we address the comments and concerns brought up by the referee.
We hope that the referee finds our responses and revisions
satisfactory, deeming our manuscript acceptable for publication in New
Journal of Physics.

Excerpts from the referee reports are written in \blue{blue}, excerpts
from our original manuscript are written in \red{red}, and new text in
the revised version of our manuscript is written in \green{green}.
All page, line, and reference numbers generally refer to those in our
original manuscript.


\begin{enumerate}
\item Concerning:

  \blue{The authors study the effective on-site interactions for the
    description of several fermionic particles confined in a very deep
    optical lattice. The focus is on alkaline earth atoms, where the
    s-wave interaction strength is independent on the nuclear spin,
    and on all atoms in the lowest state of the trapping
    potential. The manuscript is an extended version on the
    theoretical tools applied in the comparison between experiment and
    theory of Ref. [34].  The analysis is very important and
    timely. However, in the present form, the manuscript has some
    significant short comings, which have to be improved before a
    decision about publication can be made.}

  \blue{One of the main points of criticism is that the manuscript is
    very poorly written and very difficult to access. A few examples
    are listed below:}

  We thank the referee for the succinct summary of our work, and for
  recognizing that our analysis is ``very important and timely''.  We
  hope that thew revisions outlined below have made our manuscript
  more accessible, and that we have addressed shortcomings identified
  by the reviewer to a satisfactory degree.


\item Concerning the comment:

  \blue{(i) The authors try to be rather general in their description
    such as title, abstract and introduction. During the description
    of the results, they start to apply several approximations, which
    are neither mentioned before or justified.  As an example: the
    authors claim to derive a general low energy theory for orbital
    and nuclear spin stats. However, in Eq.(10) the authors suddenly
    restrict the analysis to a low density of excitations in the ``e''
    orbital state. This restriction is never explained nor motivated.}

  \label{pt:general}

  The referee is correct that we do not ``derive a general low energy
  theory for orbital and nuclear spin states''.  In our abstract, for
  example, we merely claim that (page 1, line 27)

  \red{Motivated in particular by recent experiments ... we develop a
    low-energy effective theory ... which exhibits emergent multi-body
    SU($N$)-symmetric interactions ...}

  In our introduction, meanwhile, the very first sentence discussing
  the technical contributions of our manuscript (page 4, line 19)
  explicitly states that

  \red{In this work, we investigate the first experimental
    capabilities with ultracold fermionic AEAs to prepare high-density
    samples in a 3-D lattice with multiple occupation of individual
    lattice sites [34].}

  We similarly begin both our abstract and introduction with a
  discussion of specific experiments, and make frequent reference to
  experimental developments and capabilities.  We thus believe that
  our presentation is faithful to the technical content of our
  manuscript; we do not make any claim of having developed a
  completely general theory two-orbital multi-body SU($N$)-symmetric
  interactions.  We claim only to have developed a low-energy theory
  that (i) is relevant for current experiments, and (ii) exhibits
  multi-body SU($N$)-symmetric interactions.  For these reasons, we do
  indeed introduce several approximations even in the first paragraph
  of our theory overview (section II, page 5).  We believe that
  theoretical work motivated by specific experiments is generally
  justified in making approximations that are relevant to the
  parameter regimes of these experiments.

  Nonetheless, we fully recognize and agree with the referee on the
  importance of making the scope of our technical contributions clear
  (as the referee comments later in point \ref{pt:restrictions}).  The
  fact that the referee at any point believed we will present a
  general theory without applying several experimentally-motivated
  approximations clearly signifies that we need to be more explicit in
  our aims.  Consequently, we have modified the second sentence of our
  original abstract, from (page 1, line 27)

  \red{Motivated in particular by recent experiments ...}

  to

  \green{In order to analyze recent experiments ...}

  We have also clarified an important point in our introduction,
  replacing the following text (page 4, line 22)

  \red{Specifically, we consider the preparation of isolated few-body
    systems in the deep-lattice limit, ...}

  with

  \green{Specifically, we consider ground-state preparation of
    isolated few-body systems in the deep-lattice limit, ...}

  The point on ground-state preparation in the deep lattice limit is
  important for a restriction we make later concerning single
  occupation of nuclear spin states on individual lattice sites.  This
  restriction (discussed in point \ref{pt:single_nuclear} below)
  together with the experimentally-motivated approximations made in
  the first paragraph of the theory overview (section II, page 5)
  constitute all key restrictions of our theory.

  We hope that these changes, together with those discussed in the
  examples below (points \ref{pt:orbital_state} and
  \ref{pt:single_nuclear}), clarify the motivation and scope of our
  manuscript, such that a reader is not led to believe that we present
  a general theory without applying several well-motivated
  approximations that are relevant to current experiments.


\item Concerning the specific example:

  \blue{However, in Eq.(10) the authors suddenly restrict the analysis
    to a low density of excitations in the ``e'' orbital state. This
    restriction is never explained nor motivated.}

  \label{pt:orbital_state}

  It is true, as the referee points out, that in Eq.~(10) (page 8) we
  present an effective Hamiltonian addressing only states with at most
  one orbital excitation (``e'') per lattice site.  We motivate this
  restriction immediately after Eq.~(8) in our theory overview, as
  soon as we begin a technical discussion of multi-body interactions
  (page 7, line 20):

  \red{We are interested in the ground and low-lying orbital
    eigenstates of several interacting atoms on a single lattice site,
    as such states can be readily prepared and coherently addressed in
    current experiments with ultracold atoms [34].}

  We acknowledge that the referee found the paragraph containing this
  sentence confusing, and have consequently rewritten the paragraph
  between Eqs.~(8) and (9) entirely (see point \ref{pt:confusion}).
  In any case, our actual results in section III (page 8) take the
  general form of Eq.~(9) (page 7), and make no restrictions on the
  number of orbital excitations per lattice site.  The expression in
  Eq.~(10) is merely a simplified restriction of our technical
  results, made (i) to highlight important features of the low-lying
  orbital excitations in our multi-body SU($N$) theory, and (ii) to
  facilitate comparison with optical lattice clock experiments.

  To clarify this point, we have made more explicit the fact that our
  actual results take the form of Eq.~(9), and added additional
  discussion of Eq.~(10).  Specifically, we have replaced the
  following text (page 7, line 48):

  \red{Our low-energy effective theory exhibits SU($N$)-symmetric
    multi-body interactions, such that the effective interaction
    Hamiltonian can be written in the form
    \begin{align}
      H_{\t{int}}^{\t{eff}} = \sum_{M=2}^{2I+1} H_M, \tag{9}
    \end{align}
    where $H_M$ is an $M$-body Hamiltonian, $I$ is the total nuclear
    spin of each atom (e.g.~$I=9/2$ for ${}^{87}$Sr), and the sum
    terminates at $2I+1$ because this is the largest number of atoms
    which may initially occupy a single lattice site.  Focusing on the
    low-lying orbital state excitations of this theory, we find that
    SU($N$) symmetry allows us to express multi-body Hamiltonians in
    the simple form}

  with

  \green{Our low-energy effective theory exhibits SU($N$)-symmetric
    multi-body interactions, such that the effective interaction
    Hamiltonian can be written in the form
    \begin{align}
      H_{\t{int}}^{\t{eff}} = \sum_{M=2}^{2I+1} \sum_{p\ge1} H_M^{(p)},
      \tag{9}
    \end{align}
    where $H_M^{(p)}$ is an $M$-body Hamiltonian of order $p$ in the
    coupling constants $G^{qr}_{st}$, $I$ is the total nuclear spin of
    each atom (e.g.~$I=9/2$ for ${}^{87}$Sr), and the sum terminates
    at $2I+1$ because this is the largest number of atoms which may
    initially occupy a single lattice site.  We explicitly compute all
    $M$-body Hamiltonians $H_M\equiv\sum_p H_M^{(p)}$ through order
    $p=3$, yielding effective two-, three-, and four-body
    interactions.  To facilitate understanding and comparison with
    experimental measurements of density-dependent atomic energy
    spectra [34], in which states with multiple orbital excitations on
    a single lattice site are spectroscopically off-resonant, we also
    consider a restriction of our theory to states with at most one
    orbital excitation per lattice site.  In this case, we find that
    SU($N$) symmetry allows us to express multi-body Hamiltonians in
    the simple form}

  We have also modified the following sentence in our summary (page
  28, line 43):

  \red{The SU($N$) symmetry of $M$-body Hamiltonians allowed us to
    express them in a simple form, and fully characterize their
    corresponding eigenstates and spectra.}

  to:

  \green{When considering the low-lying orbitalexcitations of our
    effective theory, the SU($N$) symmetry of $M$-body Hamiltonians
    allowed us to express them in a simple form, and to fully
    characterize their eigenstates and spectra.}

  Finally, the new appendix we have added as a part of point
  \ref{pt:description} (below) contains additional discussion of the
  restriction to at most one orbital excitation per lattice site.

  We hope that these changes clarify the scope of our work, in
  particular highlighting that the theory developed in section III
  applies to any number of orbital excitations, despite the
  restriction to the single-excitation case for the sake of comparison
  with experiments in section IV (page 18).


\item Concerning the next example:

  \blue{In addition, the restriction to a single occupation of the
    nuclear spin state is never motivated or justified but just
    introduced at some point.}

  \label{pt:single_nuclear}

  The referee is correct that we restrict our theory to the case of
  single occupation of nuclear spin states on any given lattice site.
  We do so because (i) this regime is precisely that relevant to
  current experiments with ultracold atoms, which begin with
  ground-state preparation of the atoms, and (ii) this restriction
  greatly simplifies our theory.  We explained both the motivation and
  the justification for this restriction in the overview of our theory
  (page 7, line 40):

  \red{As our work is motivated by progress in experiments which
    initialize all atoms in identical (i.e.~ground) orbital and
    motional states, to simplify our theory we assume that any $N$
    atoms on a single lattice site occupy $N$ nuclear spin states.
    Multiple occupation of a single nuclear spin state on the same
    lattice site is initially forbidden by fermionic statistics, and
    subsequently cannot be violated in the absence of tunneling and
    hyperfine coupling.}

  Deferring the referee's later comment that this text was confusing
  to point \ref{pt:confusion}, we emphasize the fact that our theory
  is the first to include hyperfine structure when considering
  multi-body interactions of ultracold atoms on a lattice.  Of course,
  a more general theory relaxing the above restriction (i.e.~on the
  occupation of nuclear spin states) will eventually be necessary for
  future experiments with a shallower lattice or additional control
  fields.  Such a general theory will need to consider inter-site
  effects or hyperfine coupling (or, less likely, non-ground-state
  preparation protocols), as these are the mechanisms by which a
  nuclear spin state can become multiply-occupied on a single lattice
  site.  Given the that these mechanisms (i) are largely irrelevant
  for the experiments that prompted this work, and (ii) introduce
  major complications into the study of SU($N$) multi-body
  interactions, we believe that it is appropriate to leave such
  considerations for future work.

  For clarification, we have modified the sentence on page 7, line 43
  from

  \red{Multiple occupation of a single nuclear spin state on the same
    lattice site is initially forbidden by fermionic statistics, and
    subsequently cannot be violated in the absence of tunneling and
    hyperfine coupling.}

  to

  \green{Ground-state preparation implies that multiple occupation of
    a single nuclear spin state on any given lattice site is initially
    forbidden by fermionic statistics, and subsequently cannot occur
    in the absence of inter-site effects or hyperfine coupling.}

  We hope that this modification, together with the modifications
  mentioned in point \ref{pt:general}, clarifies our justification for
  restricting to the case of single occupation of nuclear spin states.


\item Concerning the following comments:

  \blue{The referee understands, that these restrictions are justified
    for the comparison with the experiment in Ref.[34]. However, it
    should be clearly pointed, that the analysis is not general and
    adapted to the description to these experiments in the abstract
    and introduction.}

  \label{pt:restrictions}

  We thank the referee for pointing out the potential to mislead
  readers on the generality of our theory, and hope that the new
  version of our manuscript does a better job of calibrating the
  expectations of a reader with regard to restrictions and
  approximations that we make.  In particular, we hope that the
  current text makes clear to readers that we are developing a theory
  for the purpose of understanding current optical lattice clock
  experiments, and not to understand SU($N$) multi-body interactions
  in full generality.


\item Concerning the description of the experiment with which we
  compare:

  \blue{Furthermore, a short description of the experiment would be
    mandatory for an independent manuscript.}

  \label{pt:description}

  We thank the referee for pointing out that a description of relevant
  experimental procedures would make our manuscript more
  self-contained, and we recognize the value in doing so.  We have
  therefore added a new appendix to our manuscript, summarizing all
  relevant experimental procedures:

  \green{{\bf Appendix A:~Summary of experimental procedures}}

  \green{The work in this paper is closely tied to the experimental
    work in ref.~[34]; for quick reference, we here provide a short
    summary of the experimental procedures used therein.  The
    experiment begins by preparing a degenerate gas of $10^4$-$10^5$
    (fermionic) ${}^{87}$Sr atoms, in a uniform mixture of their ten
    nuclear spin states, at $\sim0.1$ of their fermi temperature (10
    to 20 nanokelvin) [8, 52].  This gas is loaded into a primitive
    cubic optical lattice at the ``magic wavelength'' for which both
    ground (${}^1S_0$) and first-excited (${}^3P_0$) electronic
    orbital states of the atoms experience the same lattice potential.
    Lattice depths along the principal axes of the lattice are
    comparable in magnitude, with a geometric mean that can be varied
    from 30 to 80 $E_{\t{R}}$, where
    $E_{\t{R}}\approx3.5\times2\pi~\t{kHz}$ is the lattice photon
    recoil energy of the atoms.  These lattice depths are sufficiently
    large as to neglect tunneling on the time scales relevant to the
    experiment.  Atoms are then addressed by an external (``clock'')
    laser with an ultranarrow (26 mHz) linewidth, detuned by $\Delta$
    from the single-atom ${}^1S_0-{}^3P_0$ transition frequency.  When
    $\Delta=0$, non-interacting atoms in singly-occupied lattice sites
    thus undergo Rabi oscillations.  After a chosen time, the
    experiment removes all ground-state (${}^1S_0$) atoms from the
    lattice and uses absorption imaging to identify the remaining
    excited-state (${}^3P_0$) atoms.  As discussed in section IV, a
    similar procedure with $\Delta\ne0$ leads to the spectrum shown in
    figure 3, thereby constituting a measurement of the many-body
    atomic excitation spectrum.  The many-body atomic excitation
    spectrum is different from the single-body atomic excitation
    spectrum due to the presence of inter-atomic interactions.  We
    note that this spectroscopic protocol addresses only
    singly-excited orbital states of any lattice site.  Doubly-excited
    states are manifestly off-resonant for single-photon absorption
    processes, while multi-photon absorption processes are always
    off-resonant due to (i) the non-linearity (on the scale of kHz) of
    on-site electronic orbital excitation energies (resulting from
    non-uniform coupling constants $G_X$), combined with (ii) the
    ultranarrow linewidth (on the scale of mHz) of the interrogation
    laser.}

  We have also added, at the end of the first paragraph in our
  introduction that discusses our technical contributions, a sentence
  directing readers to the above summary of experimental procedures.
  Specifically, the text (page 4, line 31)

  \red{... through the exquisite capabilities with OLCs.}

  now reads

  \green{... through the exquisite capabilities with OLCs.  We briefly
    summarize the experimental procedures relevant to our work in
    Appendix A.}

  We have likewise directed readers to the new appendix at the
  beginning of the experiment/theory comparisons in section IV (page
  18), such that the first sentence of this section, previously

  \red{Current experiments with ultracold ${}^{87}$Sr on a lattice can
    coherently address ground states and their low-lying orbital
    excitations for up to five atoms per lattice site [34].}

  now reads

  \green{Current experiments with ultracold ${}^{87}$Sr on a lattice
    can coherently address ground states and their low-lying orbital
    excitations for up to five atoms per lattice site [34] (see
    Appendix A for a summary of relevant experimental procedures).}

  We hope that these additions provide a satisfactory description of
  the experimental procedures in ref.~[34], deeming our manuscript
  sufficiently self-contained for independent publication.


\item Concerning:

  \blue{(ii) The paragraph between Eq.(8) and Eq.(9) is unreadable and
    extremely hard to understand. E.g., the following sentences ``Even
    at zero temperature, therefore, the presence of excited motional
    states modifies many-body atomic ...'', or ``As our work is
    motivated by progress in experiments which initialize all atoms in
    identical (i.e. ground) orbital and motional states, to simplify
    our theory we assume that ...''.}

  \label{pt:confusion}

  We thank the referee for their careful reading of our manuscript,
  and for pointing out that our discussion in the paragraph between
  Eqs.~(8) and (9) is poorly worded.  The paragraph in question
  appears on page 7 lines 20--47:

  \red{We are interested in the ground and low-lying orbital
    eigenstates of several interacting atoms on a single lattice site,
    as such states can be readily prepared and coherently addressed in
    current experiments with ultracold atoms [34].  Although these
    experiments operate at temperatures sufficiently low for
    negligible thermal occupation of excited motional states, simply
    neglecting excited single-particle motional states fails to
    reproduce the observed spectra of multiply-occupied lattice sites.
    The reason for this failure lies in the fact that the bare
    two-body Hamiltonian $H_{\t{int}}$ in (7) is not diagonal in the
    basis of single-particle motional states.  Even at zero
    temperature, therefore, the presence of excited motional states
    modifies many-body atomic energy spectra when the magnitude of
    off-diagonal elements of $H_{\t{int}}$ approach the scale of
    low-lying single-particle motional state energy gaps.  To address
    this problem, we develop a low-energy effective theory for
    ultracold two-level fermionic atoms on a lattice.  As our work is
    motivated by progress in experiments which initialize all atoms in
    identical (i.e.~ground) orbital and motional states, to simplify
    our theory we assume that any $N$ atoms on a single lattice site
    occupy $N$ nuclear spin states.  Multiple occupation of a single
    nuclear spin state on the same lattice site is initially forbidden
    by fermionic statistics, and subsequently cannot be violated in
    the absence of tunneling and hyperfine coupling.}

  We have re-written this paragraph in its entirety, replacing it with
  the following text:

  \green{Current experiments with ${}^{87}$Sr can prepare up to five
    atoms atoms in the same (ground) orbital state on a single lattice
    site, and coherently address states with a single orbital
    excitation per lattice site (see Appendix A) [34].  The
    temperature of the atoms in these experiments is much smaller than
    single-atom motional excitation energies $\Delta_n\equiv E_n-E_0$
    for $n>0$, which implies a negligible thermal occupation of
    excited motional states.  A naive calculation of the multi-body
    orbital excitation spectrum, however, assuming that atoms only
    occupy motional ground states with spatial wavefunctions
    $\phi_{i,0}$, fails to reproduce the experimentally observed
    orbital excitation spectrum of multiply-occupied lattice sites.
    The reason that such a calculation fails to reproduce the observed
    excitation spectrum has to do with the fact that the two-body
    Hamiltonian $H_{\t{int}}$ in (7) is not diagonal in the basis of
    single-particle motional eigenstates (i.e.~$\phi_{in}$).  For
    interacting atoms, the true motional ground states are therefore
    admixtures of the non-interacting motional eigenstates; when the
    off-diagonal elements of $H_{\t{int}}$ are comparable in magnitude
    to the excitation energies $\Delta_n$, this admixture can have
    non-negligible weight outside the non-interacting motional
    ground-state subspace.  In order to compute the correct many-body
    orbital excitation spectrum, we develop a low-energy effective
    theory for such atoms.  As our work is motivated by experiments
    that initialize all atoms in identical (i.e.~ground) orbital and
    motional states, to simplify our theory we assume that any $N$
    atoms on a single lattice site occupy $N$ nuclear spin states.
    Ground-state preparation implies that multiple occupation of a
    single nuclear spin state on any given lattice site is initially
    forbidden by fermionic statistics, and subsequently cannot occur
    in the absence of inter-site effects or hyperfine coupling.}

  We hope that the new text is more readable and easier to understand.


\item Concerning:

  \blue{(iii) Eq. (10) and Eq. (42) are identical.}

  Indeed, Eq.~(10) (page 8) and Eq.~(42) (page 18) of our manuscript
  are identical.  The purpose of Eq.~(10) is to provide a preview and
  some brief discussion of one of our main results.  This same
  equation, however, is central to the discussions in section IV, so
  we provide this equation again in Eq.~(42) for quick reference.  We
  believe that the small cost in space is worth the added benefit of
  decoupling section IV (page 18) from section II (page 5), both (i)
  from a stylistic standpoint, and (ii) to save readers the need to
  locate Eq.~(10), and repeatedly flip back and forth between
  disparate sections of the manuscript.

  With these considerations in mind, we leave the choice of whether it
  is okay to have two identical equations up to the editors.  If the
  editors decide we must remove Eq.~(42), we would replace the
  following text (page 18, line 48):

  \red{Due to the SU($N$) symmetry of inter-atomic interactions,
    manifest in the fact that all coupling constants are independent
    of nuclear spin, we can write all effective $M$-body Hamiltonians
    which address these states in the form
    \begin{align}
      H_M = \sum_{\abs{\set{\mu_j}}=M}
      \p{U_{M,\g} \prod_{\alpha=1}^M \n_{\mu_\alpha,\g}
        + U_{M,+} \n_{\mu_1,\e} \prod_{\alpha=2}^M \n_{\mu_\alpha,\g}
        + U_{M,-} \c_{\mu_1,\g}^\dag \c_{\mu_2,\e}^\dag
        \c_{\mu_2,\g} \c_{\mu_1,\e} \prod_{\alpha=3}^M \n_{\mu_\alpha,\g}},
    \end{align}
    where $\n_{\mu s}\equiv \c_{\mu s}^\dag\c_{\mu s}$ is a number
    operator, and the coefficients $U_X$ can be determined from the
    coupling constants $G_Y$ and prefactors $\alpha_Z^{(p)}$ of the
    effective $M$-body Hamiltonians derived in section III (see
    Appendix G).}

  with

  \green{Due to the SU($N$) symmetry of inter-atomic interactions,
    manifest in the fact that all coupling constants are independent
    of nuclear spin, we can write all effective $M$-body Hamiltonians
    which address these states in the form of (10).}

  Note that we have not made this replacement in the re-submitted
  version of our manuscript, but will do so at the editors' request.


\item Concerning the interaction pseudopotential:

  \blue{(iv) It is well established that Eq.(2) is incorrect as the
    s-wave interaction in three-dimensions is described by a
    pseudo-potential and not a delta-function interaction. It is
    rather misleading that the authors to not point out the problem at
    the relevant position, but only afterwards require a
    renormalization of the divergencies without explaining the reason
    for their appearance.}

  There are two related matters to address in this comment:~one
  concerning the use of an interaction pseudo-potential, and another
  concerning divergences that appear in our effective theory.  We
  first address the question of a pseudo-potential.

  The referee is right that Eq.~(2) (page 6) is not the ``correct''
  form of the two-body interaction between two ultracold atoms.
  Rather, Eq.~(2) is a standard approximate form for two-body
  interactions (see, for example, Eqs.~(1) of
  refs.~[\href{https://www.nature.com/articles/nphys3061}{17}],
  [\href{https://iopscience.iop.org/article/10.1088/1367-2630/11/10/103033/meta}{21}],
  [\href{https://iopscience.iop.org/article/10.1088/1367-2630/14/5/053037/meta}{35}],
  [\href{https://iopscience.iop.org/article/10.1088/1367-2630/11/9/093022/meta}{37}],
  [\href{https://www.nature.com/articles/nphys1535}{41}],
  [\href{https://journals.aps.org/pra/abstract/10.1103/PhysRevA.87.033601}{52}],
  and
  [\href{https://link.springer.com/article/10.1023\%2FA\%3A1018705520999}{53}];
  and Eqs.~(4) and (5) of
  ref.~[\href{https://journals.aps.org/pra/abstract/10.1103/PhysRevA.90.043631}{36}])
  that appears in an effective field theory (EFT) description of
  low-energy scattering.  In general, an EFT description of
  interactions would also include terms with derivatives of field
  operators, taking the form of e.g.~Eqs.~(2) and (4) of
  ref.~[\href{https://journals.aps.org/pra/abstract/10.1103/PhysRevA.90.043631}{36}];
  scattering is thus generally momentum-dependent, implying the need
  to use a momentum-dependent (commonly referred to as an
  energy-dependent) interaction pseudo-potential.  The precise form of
  the dominant low-energy correction to the zero-momentum contact
  interaction in EFT is well-known (see
  refs.~[\href{https://journals.aps.org/pra/abstract/10.1103/PhysRevA.65.043613}{55}]
  and
  [\href{https://journals.aps.org/pra/abstract/10.1103/PhysRevA.59.1998}{57}]
  for studies of this correction, as well as
  refs.~[\href{https://iopscience.iop.org/article/10.1088/1367-2630/14/5/053037/meta}{35}],
  [\href{https://journals.aps.org/pra/abstract/10.1103/PhysRevA.90.043631}{36}],
  and
  [\href{https://iopscience.iop.org/article/10.1088/1367-2630/11/9/093022/meta}{37}]
  for their application to EFT calculations), and amounts to using an
  effective scattering length
  \begin{align*}
    a_{\t{eff}} = a - \f12 r_{\t{eff}} a^2 k^2,
  \end{align*}
  where $a$ is the zero-momentum scattering length, $r_{\t{eff}}$
  ($\sim a$ our case) is an effective range of the leading-order
  momentum-dependent interaction, and $k$ is the relative momentum of
  two atoms.  We include these corrections in our all of our numerical
  results, as discussed in the last paragraph of section III B (page
  13, line 24):

  \red{Second, the overview we provided in section II ...  neglects
    the effect of momentum-dependent two-body scattering. ... these
    interactions are third order in the coupling constants $G_X$.
    ... [and therefore] will not appear in any of our three- and
    four-body calculations ... The primary interest of our work
    ... concerns effective $M$-body interactions for $M\ge3$; we
    therefore defer the calculation of momentum-dependent two-body
    interactions to Appendix D.}

  A more detailed discussion of momentum-dependent corrections (and
  thereby our use of an energy-dependent pseudo-potential) is provided
  in Appendix D (page 36).

  Given the standard use of a zero-momentum contact interaction in the
  relevant literature, we did not consider it necessary to complicate
  our discussion in the overview of our manuscript (section II) by
  introducing an energy-dependent pseudo-potential.  In light of the
  referee's comment, however, we now see that our omission of this
  subtlety following Eq.~(2) may confuse some readers, or mislead them
  about the nature of two-body interactions.  In order to avoid such
  confusion without introducing major complications to our overview,
  we have therefore appended the following sentence to the paragraph
  containing Eq.~(5) (page 6, before line 36):

  \green{Note that (2) and (5) neglect momentum-dependent corrections
    to $s$-wave interactions, which are generally captured (in a
    low-energy limit) by the use of an energy-dependent interaction
    pseudo-potential [42].  These corrections will appear only at
    third order in our effective theory, as corrections to two-body
    interaction energies, and do not affect our discussions of
    multi-body interactions.  We therefore defer further discussion of
    momentum-dependent corrections to the end of section III B, and
    their calculation to Appendix E.}

  For an immediate reference on the subject matter, we have also added
  ref.~[55] (now ref.~[42] in the revised manuscript) to the end of
  the sentence on page 13, line 24, previously:

  \red{Second, the overview we provided in section II, and
    consequently our result in (17), neglects the effect of
    momentum-dependent two-body scattering.}

  and now:

  \green{Second, the overview we provided in section II, and
    consequently our result in (17), neglects the effect of
    momentum-dependent two-body scattering [42].}

  We hope that these additions clarify the point about our use of an
  energy-dependent pseudo-potential.


\item Concerning the second half of the previous comment by the
  referee, on divergences and renormalization:

  \blue{It is rather misleading that the authors to not point out the
    problem at the relevant position, but only afterwards require a
    renormalization of the divergencies without explaining the reason
    for their appearance.}

  We thank the referee for pointing out a potential point of confusion
  in our manuscript due to the appearance of divergent terms that must
  be regulated (in our case, by the introduction of counter-terms).
  To address this point of confusion, we have replaced the following
  text (page 12, line 25):

  \red{In general, the sums over excited states in the loop diagrams
    of (15) will diverge [35], which implies the need to renormalize
    the coupling constants in our theory.  Even if these diagrams do
    not diverge, we would like to express the effective two-body
    interactions in terms of net, physically observable two-body
    coupling constants as on the right-hand side of (15), rather than
    in terms of long sums at all orders of the bare coupling
    constants.  We therefore renormalize our coupling constants by
    introducing counter-terms $\tilde G_X$ into the effective
    interaction Hamiltonians via $G_X\to G_X+\tilde G_X$.  The values
    of these counter-terms are fixed by enforcing that $G_X$ is equal
    to the net two-body interaction strength.}

  with:

  \green{On physical grounds, the net effective two-body interaction
    must clearly be finite, but individual sums over the excited
    states in loop diagrams of (15) may generally diverge [35].  These
    divergences ultimately appear due to our use of effective field
    theory to describe inter-atomic interactions in (2), rather than a
    detailed microscopic description of two-atom scattering.
    Divergences of this sort are a generic feature of field theories,
    and can be dealt with using standard techniques, in particular
    through renormalization.  We therefore renormalize our coupling
    constants by introducing counter-terms $\tilde G_X$ into the
    effective interaction Hamiltonians via $G_X\to G_X+\tilde G_X$.
    The values of counter-terms can be fixed by enforcing that $G_X$
    is equal to the net two-body interaction strength, which has the
    added benefit of allowing us to express the effective two-body
    interactions in terms of net, physically observable two-body
    coupling constants on the right-hand side of (15), rather than in
    terms of long sums at all orders of the bare coupling constants.}

  We hope that this discussion helps clarify confusions regarding the
  appearance of divergences in our effective theory.


\item Concerning the effective theory corrections to many-body energy
  eigenstates:

  \blue{(v) On page 10: For the sentence ``Finally, the effective
    theory involves no corrections to the many-body energy
    eigenstates.'' the referee can again only guess, what the authors
    would like to express, but as it is written it is just plain
    wrong. The states between the non-interacting and the interacting
    theory are connected by a unitary transformation and are therefore
    not identical to the non-interacting ones. }

  We thank the referee for pointing out a subtle point that we need to
  clarify about effective theories constructed with a Schrieffer-Wolff
  transformation.  It is true, as the referee states, that the energy
  eigenstates $\set{\ket\alpha}$ of an interacting theory are not
  identical to those of a non-interacting theory,
  $\set{\ket{\alpha_0}}$, and that these sets of energy eigenstates
  are related by a unitary transformation
  $U=\sum_\alpha \op{\alpha_0}{\alpha}$.  While the non-interacting
  eigenstates $\set{\ket{\alpha_0}}$ are typically simple to identify,
  however, the interacting eigenstates $\set{\ket\alpha}$ are
  generally difficult to represent in the same basis as the
  Hamiltonian $H=H_0+H_{\t{int}}$ of the interacting theory, where in
  our case $H_0$ ($H_{\t{int}}$) contains only single-body (two-body)
  terms.  The purpose of an effective theory is then to construct an
  {\it effective Hamiltonian} $H_{\t{eff}}$ with two key
  properties:~(i) $H_{\t{eff}}$ is diagonal in the ``simple'' basis
  $\set{\ket{\alpha_0}}$, and (ii) the spectrum of $H_{\t{eff}}$ is
  identical to that of the true Hamiltonian $H$.  That is, if the
  non-interacting and interacting Hamiltonians respectively have
  decompositions $H_0 = \sum_\alpha E_\alpha \op{\alpha_0}$ and
  $H = \sum_\alpha \E_\alpha \op\alpha$, an effective Hamiltonian
  takes the form
  $H_{\t{eff}} = U H U^\dag = \sum_\alpha \E_\alpha \op{\alpha_0}$.
  The energy eigenstates of the effective theory are therefore
  identical to those of the non-interacting theory, namely
  $\set{\ket{\alpha_0}}$.

\end{enumerate}


\end{document}