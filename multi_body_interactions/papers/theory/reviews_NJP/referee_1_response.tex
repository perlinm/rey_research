\documentclass[preprint]{revtex4-1}

% linking references
\usepackage{hyperref}
\hypersetup{
  breaklinks=true,
  colorlinks=true,
  linkcolor=blue,
  urlcolor=cyan,
}

% general physics / math packages and commands
\usepackage{physics,amsmath,amssymb,braket,dsfont,bm}
\renewcommand{\t}{\text} % text in math mode
\newcommand{\f}{\dfrac} % shorthand for fractions
\newcommand{\p}[1]{\left(#1\right)} % parenthesis
\renewcommand{\sp}[1]{\left[#1\right]} % square parenthesis
\renewcommand{\set}[1]{\left\{#1\right\}} % curly parenthesis
\newcommand{\bk}{\braket} % shorthand for braket

\newcommand{\g}{\text{g}}
\newcommand{\e}{\text{e}}
\newcommand{\x}{\text{x}}
\newcommand{\y}{\text{y}}
\newcommand{\z}{\text{z}}

\renewcommand{\c}{\hat{c}}
\newcommand{\n}{\hat{n}}

\newcommand{\A}{\mathcal{A}}
\newcommand{\B}{\mathcal{B}}
\newcommand{\D}{\mathcal{D}}
\newcommand{\E}{\mathcal{E}}
\newcommand{\G}{\mathcal{G}}
\renewcommand{\H}{\mathcal{H}}
\newcommand{\I}{\mathcal{I}}
\newcommand{\K}{\mathcal{K}}
\renewcommand{\L}{\mathcal{L}}
\newcommand{\M}{\mathcal{M}}
\newcommand{\N}{\mathcal{N}}
\renewcommand{\O}{\mathcal{O}}
\renewcommand{\P}{\mathcal{P}}
\newcommand{\Q}{\mathcal{Q}}
\renewcommand{\S}{\mathcal{S}}
\newcommand{\U}{\mathcal{U}}
\newcommand{\1}{\mathds{1}}

% inline lists
\usepackage[inline]{enumitem}
\setlist[enumerate]{leftmargin=*} % nice margins for enumerate


% color definitions (used in a figure)
\usepackage{xcolor}
\newcommand{\blue}[1]{\textcolor{blue}{#1}}
\newcommand{\red}[1]{\textcolor{red}{#1}}
\newcommand{\green}[1]{\textcolor{green}{#1}}
\definecolor{lightblue}{RGB}{31,119,180}
\definecolor{orange}{RGB}{255,127,14}
\definecolor{green}{RGB}{44,160,44}
\definecolor{lightred}{RGB}{214,39,40}


% for strikeout text
% normalem included to prevent underlining titles in the bibliography
\usepackage[normalem]{ulem}


\begin{document}

\section*{Referee 1}

We thank the referee for taking the time to carefully read our
manuscript, and for providing thoughtful and constructive feedback.
Below, we address the comments and concerns brought up by the referee.
We hope that the referee finds our responses and revisions
satisfactory, deeming our manuscript acceptable for publication in New
Journal of Physics.

Excerpts from the referee reports are written in \blue{blue}, excerpts
from our original manuscript are written in \red{red}, and new text in
the revised version of our manuscript is written in \green{green}.
All page, line, and reference numbers generally refer to those in our
original manuscript.

\begin{enumerate}
\item Concerning:

  \blue{The authors study the effective on-site interactions for the
    description of several fermionic particles confined in a very deep
    optical lattice. The focus is on alkaline earth atoms, where the
    s-wave interaction strength is independent on the nuclear spin,
    and on all atoms in the lowest state of the trapping
    potential. The manuscript is an extended version on the
    theoretical tools applied in the comparison between experiment and
    theory of Ref. [34].  The analysis is very important and
    timely. However, in the present form, the manuscript has some
    significant short comings, which have to be improved before a
    decision about publication can be made.}

  \blue{One of the main points of criticism is that the manuscript is
    very poorly written and very difficult to access. A few examples
    are listed below:}

  We thank the referee for the succinct summary of our work, and for
  recognizing that our analysis is ``very important and timely''.  We
  hope that thew revisions outlined below have made our manuscript
  more accessible, and that we have addressed shortcomings identified
  by the reviewer to a satisfactory degree.


\item Concerning the generality of our multi-body theory:

  \blue{(i) The authors try to be rather general in their description
    such as title, abstract and introduction. During the description
    of the results, they start to apply several approximations, which
    are neither mentioned before or justified.  As an example: the
    authors claim to derive a general low energy theory for orbital
    and nuclear spin stats. However, in Eq.(10) the authors suddenly
    restrict the analysis to a low density of excitations in the ``e''
    orbital state. This restriction is never explained nor motivated.
    In addition, the restriction to a single occupation of the nuclear
    spin state is never motivated or justified but just introduced at
    some point. The referee understands, that these restrictions are
    justified for the comparison with the experiment in
    Ref.[34]. However, it should be clearly pointed, that the analysis
    is not general and adapted to the description to these experiments
    in the abstract and introduction. Furthermore, a short description
    of the experiment would be mandatory for an independent
    manuscript.}

  \label{pt:limitations}

  We thank the referee for pointing out the potential to mislead
  readers on the generality of our theory.  The referee is correct
  that we do not ``derive a general low energy theory for orbital and
  nuclear spin states'', and we agree on the importance of making the
  scope of our technical contributions clear.  In particular, our
  theory does not consider multiple occupation of a nuclear spin state
  on any lattice site.  To facilitate comparison with experiment, we
  additionally consider a restriction of our theory to the space of
  states with at most one orbital excitation per lattice site, as
  e.g.~in Eq.~(10) (page 8) and section IV (page 18).  The full theory
  developed in section III (page 8), however, is completely general
  with respect to the number of orbital excitations per lattice site.
  We also thank the referee for pointing out that a short description
  of relevant experimental procedures would make our manuscript more
  self-contained, and we recognize the value in doing so.

  To clarify the limitations of our theory, we have added a sentence
  to our abstract, such that the original text (page 1, line 34):

  \red{... nuclear spin levels.  We fully characterize...}

  now reads:

  \green{... nuclear spin levels.  Our theory is limited to the
    experimental regime of (i) a deep lattice, with (ii) at most one
    atom occupying each nuclear spin state on any lattice site, where
    the latter restriction is a consequence of initial ground-state
    preparation.  We fully characterize...}

  In the introduction, we have modified the following sentence (in the
  first paragraph discussing our technical contributions) (page 4,
  line 22):

  \red{Specifically, we consider the preparation of isolated few-body
    systems in the deep-lattice limit, ...}

  which now reads:

  \green{Specifically, we consider ground-state preparation of
    isolated few-body systems in the deep-lattice limit, ...}

  and we have appended an additional sentence to the same paragraph,
  previously ending with the text (page 4, line 31):

  \red{... through the exquisite capabilities with OLCs.}

  and now:

  \green{... through the exquisite capabilities with OLCs.  Our theory
    is limited to the experimental regime of at most one atom
    occupying each nuclear spin state on any lattice site, which is a
    consequence initial ground-state preparation.}

  We have also rearranged and rewritten some of the following
  paragraph (page 4, line 33):

  \red{Though effective multi-body interactions have previously been
    studied in the context of harmonically [35, 36] and
    lattice-confined [37] neutral bosons prepared in a single
    hyperfine state, our work deals for the first time with fermions
    that have internal degrees of freedom and multiple collisional
    parameters.  Nonetheless, due to the SU($N$) symmetry of these
    collisions, we are able to find a simple way to express effective
    multi-body Hamiltonians and fully characterize their low-lying
    many-body energy eigenstates.  While some past work has detected
    experimental signatures of multi-body interactions in the form of
    quantum phase revivals [38], we perform a direct comparison of the
    many-body interaction energies predicted by our low-energy
    effective theory with experimental measurements of
    density-dependent atomic energy spectra performed in ref.~[34],
    similarly to the measurements with bosons performed in ref.~[39].}

  which now reads:

  \green{Though effective multi-body interactions have previously been
    studied in the context of harmonically [35, 36] and
    lattice-confined [37] neutral bosons prepared in a single
    hyperfine state, our work deals for the first time with fermions
    that have internal degrees of freedom and multiple collisional
    parameters. We perform a direct comparison between the many-body
    interaction energies predicted by our low-energy effective theory
    and the experimental measurements of the density-dependent orbital
    excitation spectra performed in ref.~[34], similarly to the
    measurements with bosons performed in ref.~[38]. To facilitate
    this comparison of excitation spectra and characterize the
    low-lying excitations in our effective theory, we consider a
    restriction of the theory to states with at most one orbital
    excitation per lattice site, and find that the SU($N$) symmetry of
    atomic collisions allow effective multi-body interactions to take
    a remarkably simple form. We note that past work has also detected
    experimental signatures of multi-body interactions in the form of
    quantum phase revivals [39].}

  We also clarify the limitations of our theory in the overview
  between Eqs.~(8) and (9), as discussed later in point
  \ref{pt:confusion}, and have replaced the following text (page 7,
  line 48):

  \red{Our low-energy effective theory exhibits SU($N$)-symmetric
    multi-body interactions, such that the effective interaction
    Hamiltonian can be written in the form
    \begin{align}
      H_{\t{int}}^{\t{eff}} = \sum_{M=2}^{2I+1} H_M, \tag{9}
    \end{align}
    where $H_M$ is an $M$-body Hamiltonian, $I$ is the total nuclear
    spin of each atom (e.g.~$I=9/2$ for ${}^{87}$Sr), and the sum
    terminates at $2I+1$ because this is the largest number of atoms
    which may initially occupy a single lattice site.  Focusing on the
    low-lying orbital state excitations of this theory, we find that
    SU($N$) symmetry allows us to express multi-body Hamiltonians in
    the simple form}

  with

  \green{Our low-energy effective theory exhibits SU($N$)-symmetric
    multi-body interactions, such that the effective interaction
    Hamiltonian can be written in the form
    \begin{align}
      H_{\t{int}}^{\t{eff}} = \sum_{M=2}^{2I+1} \sum_{p\ge1} H_M^{(p)},
      \tag{9}
    \end{align}
    where $H_M^{(p)}$ is an $M$-body Hamiltonian of order $p$ in the
    coupling constants $G_X$, $I$ is the total nuclear spin of each
    atom (e.g.~$I=9/2$ for ${}^{87}$Sr), and the sum terminates at
    $2I+1$ because this is the largest number of atoms which may
    initially occupy a single lattice site.  We explicitly compute all
    $M$-body Hamiltonians $H_M\equiv\sum_p H_M^{(p)}$ through order
    $p=3$, yielding effective two-, three-, and four-body
    interactions.  To (i) facilitate a comparison with the
    experimental measurements of many-body orbital excitation spectra
    performed in ref.~[34] and (ii) characterize the low-lying
    excitations in our effective theory, we also consider a
    restriction of the multi-body Hamiltonians $H_M$ to states with at
    most one orbital excitation per lattice site.  We find that the
    SU($N$) symmetry of atomic collisions allows us to express
    multi-body Hamiltonians in the simple form}

  In section IV (page 18), we have changed the first sentence from
  (page 18, line line 45):

  \red{Current experiments with ultracold ${}^{87}$Sr on a lattice can
    coherently address ground states and their low-lying orbital
    excitations for up to five atoms per lattice site [34].  Due to
    the SU($N$) symmetry of inter-atomic interactions, manifest in the
    fact that all coupling constants are independent of nuclear spin,
    we can write all effective $M$-body Hamiltonians which address
    these states in the form}

  to:

  \green{Current experiments with ultracold ${}^{87}$Sr on a lattice
    can coherently address ground states and single orbital
    excitations of up to five atoms per lattice site [34].  Due to the
    SU($N$) symmetry of inter-atomic interactions, manifest in the
    fact that all coupling constants are independent of nuclear spin,
    a restriction of the $M$-body Hamiltonians $H_M=\sum_p H_M^{(p)}$
    to the subspace of experimentally addressed states takes the form}

  Finally, we have modified the following text in our summary (page
  28, line 40):

  \red{Working in the deep-lattice limit, we have derived a low-energy
    effective theory of such atoms. This theory exhibits emergent
    multi-body interactions that inherit the SU($N$) symmetry of the
    atoms' bare, two-body interactions. The SU($N$) symmetry of
    $M$-body Hamiltonians in the effective theory has allowed us to
    express them in a simple form, and fully characterize their
    corresponding eigenstates and spectra.}

  to:

  \green{Working in the deep-lattice limit and the experimental regime
    of at most one atom occupying each nuclear spin state on any
    lattice site, we have derived a low-energy effective theory for
    such atoms.  Our theory exhibits emergent multi-body interactions
    that inherit the SU($N$) symmetry of atoms' bare, two-body
    interactions.  Considering a restriction of our theory to the
    subspace of at most one orbital excitation per lattice site, we
    found that the SU($N$) symmetry of $M$-body Hamiltonians allowed
    us to express them in a simple form, and to fully characterize
    their eigenstates and spectra.}

  We hope that these changes clarify the scope of our work, in
  particular highlighting that (i) our theory is limited to the case
  of at most one atom per nuclear spin state per lattice site, and
  (ii) the theory in section III applies to any number of orbital
  excitations, despite the restriction to the single-excitation case
  for the sake of comparison with experiments in section IV.

  Concerning a description of the experiment in ref.~[34], we have
  added a summary of the relevant experimental procedures to the
  beginning of section II, such that the previous opening (page 5,
  line 19):

  \red{Consider a collection of fermionic atoms with total nuclear
    spin $I$ and two atomic orbital states loaded into a
    translationally invariant 3-D optical lattice.  While an external
    trapping potential will generally break discrete translational
    symmetry of the lattice, any background inhomogeneity can be made
    negligible by spectroscopically addressing a sufficiently small
    region of the lattice [34].  Throughout this paper, we work
    strictly in the deep-lattice regime with negligible tunneling
    between lattice sites.  We also assume a so-called
    ``magic-wavelength'' lattice for which both atomic orbital states
    experience identical lattice potentials [40].}

  now reads:

  \green{The work in this paper is closely tied to the experimental
    work in ref.~[34]; we begin with a short summary of the relevant
    experimental procedures therein.  The experiment begins by
    preparing a degenerate gas of $10^4$-$10^5$ (fermionic)
    ${}^{87}$Sr atoms in a uniform mixture of their ten nuclear spin
    states and at $\sim0.1$ of their fermi temperature ($\sim10$
    nanokelvin) [8, 40].  This gas is loaded into a primitive cubic
    optical lattice at the ``magic wavelength'' for which both ground
    (${}^1S_0$) and first-excited (${}^3P_0$) electronic orbital
    states of the atoms experience the same lattice potential [41].
    Lattice depths along the principal axes of the lattice are roughly
    equal in magnitude, with a geometric mean that can be varied from
    30 to 80 $E_{\t{R}}$, where
    $E_{\t{R}}\approx3.5\times2\pi~\t{kHz}$ is the lattice photon
    recoil energy of the atoms (with the reduced Planck constant
    $\hbar=1$ throughout this paper).  These lattice depths are
    sufficiently large as to neglect tunneling on the time scales
    relevant to the experiment.  The temperature of the atoms is also
    low enough to neglect thermal occupation of states outside the
    ground-state manifold.}

  \green{Once loaded into an optical lattice lattice, atoms are
    addressed by an external (``clock'') interrogation laser with an
    ultranarrow (26 mHz) linewidth, detuned by $\Delta$ from the
    single-atom ${}^1S_0-{}^3P_0$ transition frequency $\omega_0$.
    After a fixed interrogation time, the experiment turns off the
    interrogation laser, removes all ground-state (${}^1S_0$) atoms
    from the lattice, and uses absorption imaging to count any
    remaining excited-state (${}^3P_0$) atoms.  Non-interacting atoms
    in singly-occupied lattice sites undergo Rabi oscillations when
    $\Delta=0$ during the interrogation time.  The orbital excitation
    spectrum of multiply-occupied lattice sites, however, is shifted
    by inter-atomic interactions, which results in spectroscopic peaks
    (i.e.~local maxima in excited-state atom counts) away from
    $\Delta=0$.  A sweep across different detunings $\Delta$ (on the
    scale of inter-atomic interaction energies) thus constitutes a
    measurement of the many-body orbital excitation spectrum.  We note
    that this spectroscopic protocol addresses only singly-excited
    orbital states of any lattice site.  Doubly-excited states are
    manifestly off-resonant for single-photon absorption processes as
    $\abs{\Delta}\ll\omega_0$, while multi-photon absorption processes
    are always off-resonant due to (i) the non-linearity ($\sim$kHz)
    of orbital excitation energies for a collection of interacting
    atoms, and (ii) the ultranarrow linewidth ($\sim$mHz) of the
    interrogation laser.}

  \green{Consider now a collection of fermionic atoms with total
    nuclear spin $I$ and two atomic orbital states loaded into a
    translationally invariant 3-D optical lattice.  While an external
    trapping potential will generally break discrete translational
    symmetry of the lattice, any background inhomogeneity can be made
    negligible by spectroscopically addressing a sufficiently small
    region of the lattice [34].  Throughout this paper, we work
    strictly in the deep-lattice regime with negligible tunneling
    between lattice sites, and assume that both atomic orbital states
    experience identical lattice potentials.}

  To reflect the addition of new text, we have modified the
  description of section II on page 4, line 52, previously:

  \red{In section II we provide an overview of the relevant one- and
    two-body physics of ultracold atoms on a lattice, and summarize
    our main result of effective multi-body interactions between these
    atom.}

  and now:

  \green{In section II we summarize the experimental procedures
    relevant to our work, provide an overview one- and two-body
    physics of ultracold atoms on a lattice, and preview our main
    technical results.}

  We have also removed the now redundant parenthetical text on page 5,
  line 45:

  \red{(with the reduced Planck constant $\hbar=1$ throughout this
    paper)}

  We hope that the additional text in section II provides a
  satisfactory description of the relevant experimental procedures in
  ref.~[34], deeming our manuscript sufficiently self-contained for
  independent publication.


\item Concerning:

  \blue{(ii) The paragraph between Eq.(8) and Eq.(9) is unreadable and
    extremely hard to understand. E.g., the following sentences ``Even
    at zero temperature, therefore, the presence of excited motional
    states modifies many-body atomic ...'', or ``As our work is
    motivated by progress in experiments which initialize all atoms in
    identical (i.e. ground) orbital and motional states, to simplify
    our theory we assume that ...''.}

  \label{pt:confusion}

  We thank the referee for their careful reading of our manuscript,
  and for pointing out that our discussion in the paragraph between
  Eqs.~(8) and (9) is poorly worded.  The paragraph in question
  appears on page 7, lines 20--47:

  \red{We are interested in the ground and low-lying orbital
    eigenstates of several interacting atoms on a single lattice site,
    as such states can be readily prepared and coherently addressed in
    current experiments with ultracold atoms [34].  Although these
    experiments operate at temperatures sufficiently low for
    negligible thermal occupation of excited motional states, simply
    neglecting excited single-particle motional states fails to
    reproduce the observed spectra of multiply-occupied lattice sites.
    The reason for this failure lies in the fact that the bare
    two-body Hamiltonian $H_{\t{int}}$ in (7) is not diagonal in the
    basis of single-particle motional states.  Even at zero
    temperature, therefore, the presence of excited motional states
    modifies many-body atomic energy spectra when the magnitude of
    off-diagonal elements of $H_{\t{int}}$ approach the scale of
    low-lying single-particle motional state energy gaps.  To address
    this problem, we develop a low-energy effective theory for
    ultracold two-level fermionic atoms on a lattice.  As our work is
    motivated by progress in experiments which initialize all atoms in
    identical (i.e.~ground) orbital and motional states, to simplify
    our theory we assume that any $N$ atoms on a single lattice site
    occupy $N$ nuclear spin states.  Multiple occupation of a single
    nuclear spin state on the same lattice site is initially forbidden
    by fermionic statistics, and subsequently cannot be violated in
    the absence of tunneling and hyperfine coupling.}

  We have re-written this paragraph in its entirety, replacing it with
  the following text:

  \green{Current experiments with ${}^{87}$Sr can prepare up to five
    atoms atoms in the same (ground) orbital state on a single lattice
    site, and coherently address states with a single orbital
    excitation per lattice site [34].  At ultracold temperatures well
    below the single-atom motional excitation energies
    $\Delta_n\equiv E_n-E_0$ for $n>0$, atoms only occupy their
    motional ground state in the lattice.  For this reason, it is
    common to map the description of these atoms onto a single-band
    Hubbard model that captures all dynamics within the subspace of
    motional ground states, i.e.~with single-atom wavefunctions
    $\phi_{i,0}$.  When enough atoms occupy the same lattice site,
    however, their interaction energy can approach the scale of the
    motional excitation energies $\Delta_n$.  In this case, normally
    far-off-resonant terms in the interaction Hamiltonian of (7) that
    create atoms in excited motional states,
    e.g.~$\sim\c_{n\mu s}^\dag \c_{0,\nu t}^\dag \c_{0,\nu r}
    \c_{0,\mu q}$ with $n>0$, become relevant.  The true motional
    ground state of a collection of interacting atoms is then an
    admixture of the non-interacting motional eigenstates, and a naive
    Hubbard model that assumes atomic wavefunctions $\phi_{i,0}$ will
    fail to reproduce the interacting atoms' orbital excitation
    spectrum.}

  \green{In order to understand interacting atoms' orbital excitation
    spectrum in the context of a zero-temperature Hubbard-type model
    that eliminates non-dynamical (i.e.~ground-state) motional degrees
    of freedom, we develop a low-energy effective theory of
    interacting AEAs in a deep lattice.  To simplify our theory, we
    assume that any $N$ atoms on a single lattice site occupy $N$
    nuclear spin states.  Ground-state preparation of the atoms in an
    experiment implies that multiple occupation of a single nuclear
    spin state on any given lattice site is initially forbidden by
    fermionic statistics, and subsequently cannot occur in the absence
    of inter-site effects or hyperfine coupling between nuclear spin
    states.}

  Accordingly, we have modified language in section III that makes
  reference to the re-written paragraph, changing (page 9, line 45):

  \red{The naive idea of neglecting all excited states and using
    $H_{\t{int}}$ directly to describe the spectrum of single-particle
    motional ground states is thus equivalent to truncating our
    expansion for $H_{\t{int}}^{\t{eff}}$ at first order.}

  to:

  \green{Writing down a single-band Hubbard model that simply neglects
    excited atomic motional states and uses $H_{\t{int}}$ directly to
    describe the orbital spectrum of interacting atoms is thus
    equivalent to truncating our expansion for $H_{\t{int}}^{\t{eff}}$
    at first order.}

  We hope that the new text is more readable and easier to understand.


\item Concerning:

  \blue{(iii) Eq. (10) and Eq. (42) are identical.}

  As the referee points out, Eqs.~(10) and (42) in the original
  version of our manuscript are identical.  In order to avoid
  redundancy while still providing a preview of our main results, we
  have replaced the following text in section II (page 8, line 10):

  \red{\begin{align} H_M = \sum_{\abs{\set{\mu_j}}=M} \p{U_{M,\g}
        \prod_{\alpha=1}^M \n_{\mu_\alpha,\g} + U_{M,+} \n_{\mu_1,\e}
        \prod_{\alpha=2}^M \n_{\mu_\alpha,\g} + U_{M,-}
        \c_{\mu_1,\g}^\dag \c_{\mu_2,\e}^\dag \c_{\mu_2,\g}
        \c_{\mu_1,\e} \prod_{\alpha=3}^M \n_{\mu_\alpha,\g}},
      \tag{10}
    \end{align}
    where $\c_{\mu s}\equiv\c_{0,\mu s}$ is a fermionic annihilation
    operator for an atom in the motional ground state with nuclear
    spin $\mu$ and orbital state $s$;
    $\n_{\mu s}\equiv \c_{\mu s}^\dag \c_{\mu s}$ is a number
    operator; and the sum is performed over all choices of nuclear
    spins $\mu_j$ with $j=1,2,\cdots,M$ for which all $\mu_j$ are
    distinct, or equivalently all choices of $\mu_j$ for which the set
    $\set{\mu_j}$ contains $M$ elements, for a total of
    ${2I+1\choose M}\times\p{M!}$ nuclear spin combinations.  The
    coefficients $U_X$ can be expressed in terms of the coupling
    constants of the bare two-body interaction Hamiltonian (see
    Appendix F).  The key feature of the $M$-body interactions in (10)
    is that they take a form identical to the two-body interactions
    between motional ground-state atoms, but with the addition of
    $M-2$ spectator atoms.}

  by:

  \green{\begin{align}
      H_M = \sum_{\abs{\set{\mu_j}}=M}
      H_2^{(\mu_1,\mu_2)} \prod_{\alpha=3}^M \n_{\mu_\alpha,\g},
      \tag{10}
    \end{align}
    where $H_2^{(\mu_1,\mu_2)}$ is a two-body Hamiltonian addressing
    atoms with nuclear spin $\mu_1,\mu_2$;
    $\n_{\mu s}=\c_{\mu s}^\dag\c_{\mu s}$ is a number operator for
    atoms with nuclear state $\mu$ and orbital state $s$; and the sum
    is performed over all choices of nuclear spins $\mu_j$ with
    $j=1,2,\cdots,M$ for which all $\mu_j$ are distinct, or
    equivalently all choices of $\mu_j$ for which the set
    $\set{\mu_j}$ contains $M$ elements, for a total of
    ${2I+1\choose M}\times\p{M!}$ nuclear spin combinations.  The key
    feature of the $M$-body interactions in (10) is that they
    ultimately take the same form as two-body interactions, but with
    the addition of $M-2$ spectator atoms.}

  and have modified Eq.~(42) to take the explicit form advertised in
  the new Eq.~(10), changing (page 18, line 54):

  \red{\begin{align}
      H_M = \sum_{\abs{\set{\mu_j}}=M}
      \p{U_{M,\g} \prod_{\alpha=1}^M \n_{\mu_\alpha,\g}
        + U_{M,+} \n_{\mu_1,\e} \prod_{\alpha=2}^M \n_{\mu_\alpha,\g}
        + U_{M,-} \c_{\mu_1,\g}^\dag \c_{\mu_2,\e}^\dag
        \c_{\mu_2,\g} \c_{\mu_1,\e} \prod_{\alpha=3}^M \n_{\mu_\alpha,\g}},
      \tag{42}
    \end{align}}

  to:

  \green{\begin{align}
      H_M = \sum_{\abs{\set{\mu_j}}=M}
      \p{U_{M,\g} \n_{\mu_1,\g} \n_{\mu_2,\g}
        + U_{M,+} \n_{\mu_1,\e} \n_{\mu_2,\g}
        + U_{M,-} \c_{\mu_1,\g}^\dag \c_{\mu_2,\e}^\dag
        \c_{\mu_2,\g} \c_{\mu_1,\e}}
      \prod_{\alpha=3}^M \n_{\mu_\alpha,\g},
      \tag{42}
    \end{align}}


\item Concerning the use of an interaction pseudopotential:

  \blue{(iv) It is well established that Eq.(2) is incorrect as the
    s-wave interaction in three-dimensions is described by a
    pseudo-potential and not a delta-function interaction. It is
    rather misleading that the authors to not point out the problem at
    the relevant position, but only afterwards require a
    renormalization of the divergencies without explaining the reason
    for their appearance.}

  We thank the referee for their attention to detail, and for pointing
  out the potential to mislead readers with respect to the nature of
  $s$-wave interactions by our expression of Eq.~(2).  We have
  therefore added the following paragraph to section II of our
  manuscript, following the re-statement of Eq.~(2) in Eq.~(5)
  (i.e.~the following text has been inserted in between lines 35 and
  36 of page 6):

  \green{Note that the Hamiltonian in (5) is not the true microscopic
    interaction Hamiltonian of AEAs, but rather a generic form for a
    low-energy effective field theoretic description of two-body
    interactions [17, 21, 35--37, 42--44].  There are therefore two
    important points to keep in mind concerning our use of (5) to
    describe two-body interactions.  First, the use of effective field
    theory generically gives rise to divergences that must be dealt
    with either through regularization, e.g.~of the zero-range
    interaction potential implicitly assumed in the expression of (5)
    [45], or through renormalization of the coupling constants in the
    theory.  We chose the latter approach, as we will in any case find
    it convenient to renormalize the coupling constants in the
    effective theory developed in section III.  Second, (5) is only
    the first term in a low-energy expansion of two-body interactions
    in effective field theory, which generally includes additional
    terms containing derivatives of field operators.  Derivative terms
    correspond to the dependence of two-body scattering on the
    relative momentum $k$ of particles involved, with $k\to0$ in the
    zero-energy limit.  In the present case of $s$-wave scattering,
    the leading dependence of the two-body interaction Hamiltonian on
    the relative momentum $k$ can be captured by use of an
    energy-dependent pseudo-potential, which amounts to using a
    $k$-dependent effective scattering length [46].  This effective
    scattering length can be determined by expanding the $s$-wave
    collisional phase shift in powers of the relative momentum $k$
    [45, 47].  Details of this expansion will depend on the
    characteristic length scale of finite-range interactions.  In our
    work, these corrections to (5) will be relevant only for the
    calculation of two-body interaction energies, appearing at third
    order in the coupling constants $G_X$.  As we are primarily
    interested in $M$-body interactions for $M\ge3$, we defer this
    calculation to Appendix D.  We note that our approach of using an
    unregularized contact potential, renormalizing coupling constants,
    and separately accounting for momentum-dependent scattering is
    essentially the same as that used for similar calculations in
    refs.~[35--37].}

  Expanding on our use of renormalization, we have changed the
  following text in section III B (page 12, line 15):

  \red{In general, the sums over excited states in the loop diagrams
    of (15) will diverge [35], which implies the need to renormalize
    the coupling constants in our theory.  Even if these diagrams do
    not diverge, we would like to express the effective two-body
    interactions in terms of net, physically observable two-body
    coupling constants as on the right-hand side of (15), rather than
    in terms of long sums at all orders of the bare coupling
    constants.  We therefore renormalize our coupling constants by
    introducing counter-terms $\tilde G_X$ into the effective
    interaction Hamiltonians via $G_X\to G_X+\tilde G_X$.  The values
    of these counter-terms are fixed by enforcing that $G_X$ is equal
    to the net two-body interaction strength.}

  with:

  \green{On physical grounds, the net effective two-body interaction
    must clearly be finite, but individual sums over the excited
    states in loop diagrams of (15) may generally diverge [35].  These
    divergences ultimately appear due to our use of effective field
    theory to describe inter-atomic interactions in (2) and (5),
    rather than a detailed microscopic description of two-atom
    scattering.  Divergences of this sort are a generic feature of
    field theories, and can be dealt with using standard techniques
    such as renormalization.  We therefore renormalize our coupling
    constants by introducing counter-terms $\tilde G_X$ into the
    effective interaction Hamiltonians via $G_X\to G_X+\tilde G_X$.
    The values of counter-terms can be fixed by enforcing that $G_X$
    is equal to the net two-body interaction strength, which has the
    added benefit of allowing us to express the effective two-body
    interactions in terms of net, physically observable two-body
    coupling constants on the right-hand side of (15), rather than in
    terms of long sums at all orders of the bare coupling constants.}

  We have also modified the wording in our discussion of corrections
  to Eqs.~(2) and (5) at the end of section III B, replacing the text
  (page 13, line 24):

  \red{Second, the overview we provided in section II, and
    consequently our result in (17), neglects the effect of
    momentum-dependent two-body scattering. When the effective range
    of momentum-dependent interactions is comparable to the scattering
    lengths $a_X$, as is the case for ultracold ${}^{87}$Sr, these
    interactions are third order in the coupling constants $G_X$. Just
    as the $\O\p{G^2}$ counter-terms do not affect $M$-body
    interactions until next-to-leading order (NLO), the $\O\p{G^3}$
    momentum-dependent interactions do not come into play until next-
    next-leading order (NNLO).}

  by:

  \green{Second, our result in (17) neglects the effect of
    momentum-dependent two-body scattering.  When the effective range
    of interactions is comparable to the scattering lengths $a_X$, as
    is the case for ultracold ${}^{87}$Sr, these momentum-dependent
    effects are third order in the coupling constants $G_X$.  Just as
    the $\O\p{G^2}$ counter-terms do not affect $M$-body interactions
    until next-to-leading order (NLO), the $\O\p{G^3}$
    momentum-dependent terms do not come into play until
    next-next-leading order (NNLO).}

  We hope these changes and discussions help clarify our treatment of
  two-body $s$-wave interactions, in particular addressing any
  subtleties and potential concerns as soon as we introduce a two-body
  interaction Hamiltonian.


\item Concerning the effective theory corrections to many-body energy
  eigenstates:

  \blue{(v) On page 10: For the sentence ``Finally, the effective
    theory involves no corrections to the many-body energy
    eigenstates.'' the referee can again only guess, what the authors
    would like to express, but as it is written it is just plain
    wrong. The states between the non-interacting and the interacting
    theory are connected by a unitary transformation and are therefore
    not identical to the non-interacting ones. }

  We thank the referee for pointing out a subtlety that we need to
  clarify about effective theories constructed with a Schrieffer-Wolff
  transformation.  Between the non-interacting Hamiltonian $H_0$, the
  interacting Hamiltonian $H = H_0 + H_{\t{int}}$, and the effective
  Hamiltonian $H_{\t{eff}}$ constructed from $H$ with a unitary
  transformation, there are many different sets of energies and
  eigenstates to keep track of.

  To help clarify the relationship between $H_0$, $H$, and
  $H_{\t{eff}}$, as well as their respective eigenstates and spectra,
  we have re-written our opening to section III, replacing the
  following paragraphs (page 8, line 36):

  \red{As the bare two-body Hamiltonian $H_{\t{int}}$ is not diagonal
    with respect to single-particle motional states, the problem of
    determining many-body spectra and eigenstates nominally involves
    all motional degrees of freedom of the atoms involved.  If we are
    interested in addressing internal atomic states at zero
    temperature, however, in principle we need only to consider the
    motional ground state associated with each many-body internal
    atomic state.  We can then ignore all excited motional states, as
    they will be neither thermally occupied nor externally
    interrogated.  This procedure makes the many-body problem more
    tractable by greatly reducing the dimensionality of the relevant
    Hilbert space.}

  \red{We denote the space of all internal states of atoms in
    single-particle, non-interacting (multi-particle, interacting)
    motional ground states by $\H_{\t{ground}}^{\t{single}}$
    ($\H_{\t{ground}}^{\t{multi}}$).  For sufficiently weak
    interactions, one can construct a unitary transformation $U$
    acting on the full many-body Hilbert space which rotates
    $\H_{\t{ground}}^{\t{multi}}$ into $\H_{\t{ground}}^{\t{single}}$
    [43].  The unitary $U$ can then be used to construct an {\it
      effective Hamiltonian} $H_{\t{int}}^{\t{eff}}$ on
    $\H_{\t{ground}}^{\t{single}}$ that reproduces the spectrum of
    $H_{\t{int}}$ on $\H_{\t{ground}}^{\t{multi}}$, namely
    $H_{\t{int}}^{\t{eff}}=UH_{\t{int}}U^\dag$.  This method for
    constructing an effective theory is commonly known as the
    Schrieffer-Wolff transformation, named after the authors of its
    celebrated application in relating the Anderson and Kondo models
    of magnetic impurities in metals [44].}

  by:

  \green{The net Hamiltonian $H = H_0 + H_{\t{int}}$ for interacting
    AEAs on a lattice is not diagonal with respect to single-particle
    motional state indices.  The problem of determining interacting
    atoms' orbital excitation spectrum therefore nominally involves
    all atomic motional degrees of freedom.  At zero temperature,
    however, each orbital state of interacting atoms is associated
    with a single motional ground state.  In order to compute an
    orbital excitation spectrum at zero temperature, in principle we
    need to identify these motional ground states.  We can then ignore
    all excited motional states, which will be neither thermally
    occupied nor externally interrogated.  Such a procedure would
    drastically reduce the dimensionality of the Hilbert space
    necessary to describe the atoms, thereby greatly simplifying any
    description of the atoms' orbital spectrum and dynamics.  In
    practice, however, identifying the motional ground states of
    interacting atoms and writing down a Hamiltonian restricted to
    this subspace is a very difficult process to carry out
    analytically.}

  \green{We denote the motional ground-state subspace of the
    non-interacting Hamiltonian $H_0$ by
    $\H_{\t{ground}}^{\t{single}}$, and the motional ground-state
    subspace of the interacting Hamiltonian $H = H_0 + H_{\t{int}}$ by
    $\H_{\t{ground}}^{\t{multi}}$.  That is, all atomic wavefunctions
    for states within $\H_{\t{ground}}^{\t{single}}$ are described by
    $\phi_{i,0}$, while the atomic wavefunctions for states within
    $\H_{\t{ground}}^{\t{multi}}$ are generally unknown, and are in
    principle determined by minimizing the energy of a state with
    respect to its motional degrees of freedom.  Both
    $\H_{\t{ground}}^{\t{single}}$ and $\H_{\t{ground}}^{\t{multi}}$
    are subspaces of the full Hilbert space $\H_{\t{full}}$.  When
    interactions are sufficiently weak compared to the spectral gap
    $\Delta$ of the non-interacting Hamiltonian $H_0$ between
    $\H_{\t{ground}}^{\t{single}}$ and its orthogonal complement
    $\H_{\t{full}}\setminus\H_{\t{ground}}^{\t{single}}$, one can
    identify a particular unitary operator $U$ (acting on the full
    Hilbert space $\H_{\t{full}}$) which rotates
    $\H_{\t{ground}}^{\t{multi}}$ into $\H_{\t{ground}}^{\t{single}}$
    [49].  This unitary $U$ can be used to construct an {\it effective
      Hamiltonian} $H_{\t{eff}} = U H U^\dag$ with two key properties:
    (i) $H_{\t{eff}}$ is diagonal in the same (known) basis as the
    non-interacting Hamiltonian $H_0$, and (ii) the spectrum of
    $H_{\t{eff}}$ on $\H_{\t{ground}}^{\t{single}}$ is identical to
    that of the interacting Hamiltonian $H$ on
    $\H_{\t{ground}}^{\t{multi}}$.  The use of an effective
    Hamiltonian $H_{\t{eff}}$ thus side-steps the need to identify
    $\H_{\t{ground}}^{\t{multi}}$ in order to compute the orbital
    spectrum of $H$ at zero temperature.  This method for constructing
    an effective theory is commonly known as the Schrieffer-Wolff
    transformation, named after the authors of its celebrated
    application in relating the Anderson and Kondo models of magnetic
    impurities in metals [50].}

  We have also modified the text pointed out by the referee, changing
  (page 10, line 19):

  \red{Finally, the effective theory involves no corrections to the
    many-body energy eigenstates.  Including such corrections would
    invalidate the effective theory, which preserves the eigenstates
    of the non-interacting Hamiltonian $H_0$ on
    $\H_{\t{ground}}^{\t{single}}$.}

  to:

  \green{Finally, the effective theory involves no corrections to the
    non-interacting many-body energy eigenstates; the purpose of the
    effective Hamiltonian $H_{\t{eff}} = H_0 + H_{\t{int}}^{\t{eff}}$
    is to reproduce, on the eigenstates of the non-interacting
    Hamiltonian $H_0$ within $\H_{\t{ground}}^{\t{single}}$, the
    spectrum of the interacting Hamiltonian $H = H_0 + H_{\t{int}}$ on
    $\H_{\t{ground}}^{\t{multi}}$.  Modifying the eigenstates of the
    non-interacting Hamiltonian $H_0$ on
    $\H_{\t{ground}}^{\t{single}}$ thus invalidates the effective
    theory.}

  We hope that these changes clarify subtleties of the
  Schrieffer-Wolff transformation, and in particular the fact that the
  effective Hamiltonian $H_{\t{eff}}$ has the spectrum of the
  interacting Hamiltonian $H$ on the eigenstates of the
  non-interacting Hamiltonian $H_0$.


\item Concerning:

  \blue{In addition, the is also an important scientific question. The
    authors use renormalization of the theory by adding counter terms
    to remove the diverging contributions. This approach might provide
    the correct result however a microscopic derivation and motivation
    based on the pseudo-potential would be certainly help to justify
    this approach.}

  


\end{enumerate}

\end{document}