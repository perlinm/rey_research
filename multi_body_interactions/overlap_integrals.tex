\documentclass[aps,notitlepage,nofootinbib,11pt]{revtex4-1}

% linking references
\usepackage{hyperref}
\hypersetup{
  breaklinks=true,
  colorlinks=true,
  linkcolor=blue,
  filecolor=magenta,
  urlcolor=cyan,
}

%%% paragraph indentation and spacing
\setlength{\parindent}{0cm} % don't indent new paragraphs...
\parskip 6pt % ... place a space between paragraphs instead
\frenchspacing % only one space after periods

%%% header / footer
\usepackage{fancyhdr} % easier header and footer management
\pagestyle{fancy} % page formatting style
\fancyhf{}
\renewcommand{\headrulewidth}{0pt} % remove horizontal line in header
\usepackage{lastpage} % for referencing last page
\cfoot{\thepage~of \pageref{LastPage}} % "x of y" page labeling

% figures
\usepackage{hyperref} % for linking references
\usepackage{graphicx,float} % for figures
\usepackage{grffile} % help latex properly identify figure extensions
\graphicspath{{./figures/}} % set path for all graphics
\usepackage[caption=false]{subfig} % subfigures
\newcommand{\sref}[1]{\protect\subref{#1}}

%%% symbols, notations, etc.
\usepackage{physics,bm,braket,amssymb} % physics and math packages
\usepackage{accents} % for resolving some accent (e.g. tilde) issues
\renewcommand{\t}{\text} % text in math mode
\newcommand{\f}[2]{\dfrac{#1}{#2}} % shorthand
\newcommand{\p}[1]{\left(#1\right)} % parenthesis
\renewcommand{\sp}[1]{\left[#1\right]} % square parenthesis
\renewcommand{\set}[1]{\left\{#1\right\}} % curly parenthesis
\renewcommand{\v}{\bm} % bold vectors
\newcommand{\uv}[1]{\hat{\v{#1}}} % unit vectors
\renewcommand{\d}{\partial} % partial d
\renewcommand{\c}{\cdot} % inner product
\newcommand{\bk}{\Braket} % shorthand for braket notation

\usepackage{dsfont}
\newcommand{\1}{\mathds{1}}

\newcommand{\E}{\mathcal E}

\usepackage{listings}


\begin{document}

\title{Efficient calculation of Wannier functions and their overlap
  integrals}

\author{Michael A. Perlin}

\maketitle
\thispagestyle{fancy}

Solutions to the Mathieu equation (i.e. spatial eigenfunctions of an
optical lattice, or Bloch functions) with quasi-momentum $q$ in band
$n$ take the form
\begin{align}
  \tilde\phi_{qn}\p{x}
  = e^{iqx} \sum_{k=-\infty}^\infty c_{qk}^{(n)}e^{2ikx}
\end{align}
where all coefficients $c_{qk}^{(n)}\in\mathbb R$. Without loss of
generality, we will only consider on-site (Wannier) functions
$\phi_n\p{x}$ for band $n$ on a lattice site centered at $x=0$, which
are given by the discrete Fourier transform of the Bloch functions:
\begin{align}
  \phi_n\p{x}
  = \sum_q \tilde\phi_{qn}\p{x}
  = \sum_{q,k} c_{qk}^{(n)} e^{i\p{2k+q}x}
  = \sum_{q,k} c_{qk}^{(n)} \E_{qk}\p{x},
  \label{eq:wannier_first}
\end{align}
where $\E_{qk}\p{x} \equiv e^{i\p{2k+q}x}$. In practice, when
computing \eqref{eq:wannier_first} we must choose a finite number $Q$
of allowed quasi-momenta $q$ and a finite number $K$ of allowed values
of $k$ (centered on 0). We can then construct the $Q\times K$ matrices
$\v c_n\equiv \sum_{q,k}c_{qk}^{(n)}\uv q\otimes\uv k$ and
$\v\E\p{x}\equiv \sum_{q,k}\E_{qk}\p{x}\uv q\otimes\uv k$, and rewrite
\eqref{eq:wannier_first} as
\begin{align}
  \phi_n\p{x} = \v c_n \c \v\E\p{x}.
  \label{eq:wannier}
\end{align}
Constructing these matrices and computing their scalar product is
substantially faster to perform numerically than the double sum in
\eqref{eq:wannier_first}.

The Wannier functions in \eqref{eq:wannier} appear in e.g. the
(one-dimensional) four-function overlap integral
\begin{align}
  F_n
  \equiv \int dx~ \phi_n\p{x}^*\phi_0\p{x}^*\phi_0\p{x}\phi_0\p{x}.
  \label{eq:F}
\end{align}
As $\phi_n\p{x}$ has the same parity (i.e. even/odd) as $n$, the
entire integrand in \eqref{eq:F} likewise has the same parity as
$n$. It follows that $F_n$ vanishes if $n$ is odd. If $n$ is even,
then the integrand in \eqref{eq:F} is even, which means that we need
only integrate over half of the lattice. Furthermore, knowing that
$\phi_0\p{x}$ and $F_n$ are real\footnote{We will not spend time
  justifying this claim here, but can do so upon request.}, we can
generally make the replacement
$\v\E\p{x}\to\Re\sp{\v\E\p{x}}\equiv\v\E^R\p{x}$ for the purposes of
computing \eqref{eq:F}. For a lattice of length $L$, the integral in
\eqref{eq:F} then becomes
\begin{align}
  F_n
  = 2\int_0^{L/2} dx~ \Re\sp{\phi_n\p{x}} \phi_0\p{x}^3
  = 2\int_0^{L/2} dx~
  \sp{\v c_n\c\v\E^R\p{x}} \sp{\v c_0\c\v\E^R\p{x}}^3.
  \label{eq:F2}
\end{align}

In order to reduce the number of operations necessary to compute the
overlaps $F_n$, we should first numerically solve the Mathieu equation
to get the coefficients $c_{qn}^{(k)}$ for every quasi-momentum $q$ up
to some cutoff in band ($n$) and Fourier order ($k$). These
coefficients should be stored in a three-dimensional array which can
be sliced to extract the matrix $\v c_n$ from memory. Recycling the
coefficients $c_{qn}^{(k)}$ in this manner provides an enormous
speedup in the computation of $F_n$, as we would otherwise need to
re-compute all of these coefficients for every instance of the
integrand (as occurs when e.g. using Mathematica's Mathieu function
methods to compute $F_n$). Computing $\v\E^R\p{x}$ only once for each
instance of the integrand and recycling it for $\phi_n\p{x}$ and
$\phi_0\p{x}$ also also saves us an additional factor of $\sim2$ in
computation time.

Lastly, there is another trick we can use to efficiently compute the
matrices $\v\E\p{x}$. Rather than looping over all $q$ and $k$
individually to compute all elements of $\v\E\p{x}$, it is
substantially faster to compute the array $\sum_q e^{iqx}\uv q$ and
replicate it $K$ times; and likewise compute $\sum_k e^{2ikx}\uv k$
and replicate it $Q$ times. We can then multiply these matrices
element-wise to get $\v\E^R\p{x}$. Implemented in Python using the
NumPy library, this procedure would look something like:
\begin{lstlisting}[language=python,basicstyle=\small]
  q_phases = numpy.matlib.repmat(numpy.exp(i * quasi_momenta * x), K, 1).T
  k_phases = numpy.matlib.repmat(numpy.exp(i * k_values * x), Q, 1)
  E_x = q_phases * k_phases
\end{lstlisting}
where ``quasi\_momenta'' and ``k\_values'' are respectively the NumPy
arrays $\sum_q q\uv q$ and $\sum_k k\uv k$.

\end{document}
