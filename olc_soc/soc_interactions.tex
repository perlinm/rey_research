\documentclass[aps,notitlepage,nofootinbib,11pt]{revtex4-1}

% linking references
\usepackage{hyperref}
\hypersetup{
  breaklinks=true,
  colorlinks=true,
  linkcolor=blue,
  filecolor=magenta,
  urlcolor=cyan,
}

%%% symbols, notations, etc.
\usepackage{physics,braket,bm,commath,amssymb} % physics and math
\renewcommand{\t}{\text} % text in math mode
\newcommand{\f}[2]{\dfrac{#1}{#2}} % shorthand for fractions
\newcommand{\p}[1]{\left(#1\right)} % parenthesis
\renewcommand{\sp}[1]{\left[#1\right]} % square parenthesis
\renewcommand{\set}[1]{\left\{#1\right\}} % curly parenthesis
\renewcommand{\v}{\bm} % bold vectors
\newcommand{\uv}[1]{\hat{\v{#1}}} % unit vectors
\renewcommand{\d}{\partial} % partial d
\renewcommand{\c}{\cdot} % inner product
\newcommand{\bk}{\Braket} % shorthand for braket notation


\usepackage[inline]{enumitem} % for inline enumeration

% leave a note in the text, visible in the compiled document
\newcommand{\note}[1]{\textcolor{red}{#1}}
\usepackage{ulem} % for strikeout text


\usepackage{dsfont}
\newcommand{\1}{\mathds{1}}

\newcommand{\up}{\uparrow}
\newcommand{\dn}{\downarrow}
\newcommand{\E}{\mathcal{E}}
\renewcommand{\H}{\mathcal{H}}
\newcommand{\K}{\mathcal{K}}
\newcommand{\Z}{\mathbb{Z}}


\usepackage{accents}
\newcommand{\utilde}[1]{\underaccent{\tilde}{#1}}

\newcommand{\g}{\text{g}}
\newcommand{\e}{\text{e}}


\begin{document}

\title{Interactions in the 1-D optical lattice clock}

\author{Michael A. Perlin}

\maketitle

The interaction Hamiltonian for nuclear-spin polarized fermionic
alkali-earth(-like) atoms is
\begin{align}
  H_{\t{int}} = G \int d^3x~
  \psi_\e^\dag\p{x} \psi_\g^\dag\p{x} \psi_\g\p{x} \psi_\e\p{x},
  &&
  G \equiv \f{4\pi a_{\e\g^-}}{m_A},
\end{align}
where $\psi_\sigma$ is a fermionic field operator for atoms in
electronic state $s\in\set{\g,\e}$, $a_{eg}^-$ is a scattering length,
and $m_A$ is the mass of a single atom.  On a 1-D lattice with tight
transverse confinement, we can expand the field operators as
\begin{align}
  \psi_\sigma\p{x} = \sum_{q,n} \phi_{qn}\p{x} c_{qn\sigma},
\end{align}
for wavefunctions $\phi_{qn}$ and annihilation operators
$c_{qn\sigma}$ additionally indexed by quasi-momenta $q$ and band
indices $n$.  The interaction Hamiltonian then becomes
\begin{align}
  H_{\t{int}} = G \sum K^{pk;q\ell}_{rm;sn}
  c_{rm,\e}^\dag c_{sn,\g}^\dag c_{q\ell,\g} c_{pk,\e},
  &&
  K^{pk;q\ell}_{rm;sn} \equiv \int d^3x~
  \phi_{rm}\p{x}^* \phi_{sn}\p{x}^* \phi_{q\ell}\p{x} \phi_{pk}\p{x}.
\end{align}
By conservation of momentum, we know that $K^{pk;q\ell}_{rm;sn}=0$
unless $p+q+r+s=0$ (where we note that all quasi-momenta are defined
modulo 2 in units of the lattice wavenumber), so we can write
\begin{align}
  H_{\t{int}} = G \sum K^{pk;q\ell}_{p+r,m;q-r,n}
  c_{p+r,m,\e}^\dag c_{q-r,n,\g}^\dag c_{q\ell,\g} c_{pk,\e}
  \approx G \sum K^{k\ell}_{mn}
  c_{p+r,m,\e}^\dag c_{q-r,n,\g}^\dag c_{q\ell,\g} c_{pk,\e},
  \label{eq:H_int_full}
\end{align}
where we made the approximation that $K^{pk;q\ell}_{p+r,m;q-r,n}$ only
weakly depends on the quasi-momenta $\p{p,q,r}$.  We note that for the
cases we will be considering, this approximation breaks down at the
$\sim10\%$ level for lattices with depth $\lesssim 15 E_R$, where
$E_R$ is the lattice recoil energy.


\section{Single-band spin model}

We now restrict ourselves to considering only atoms a single band,
such that $K^{k\ell}_{mn}\to K$ and the single-particle energy (up to
a global shift) takes the form $E_q=-2J\cos\p{\pi q}$, where we measure
momenta $q$ in units of the lattice wavenumber.  If
$\abs{J}\gg\abs{GK}$, then by the secular approximation we can neglect
terms in $H_{\t{int}}$ which do not conserve the sum of
single-particle energies.  By solving the single-particle energy
conservation condition, one can find that this approximation amounts
to neglecting terms with $\set{p,q}\ne\set{p+r,q-r}$; that is, atoms
in different clock states can only interact via
\begin{enumerate*}[label=(\roman*)]
\item direct density-density terms, and
\item terms which exchange their momenta.
\end{enumerate*}
The surviving terms in \eqref{eq:H_int_full} are thus
\begin{align}
  H_{\t{int}}^{\t{single-band}}
  = G K \sum \p{c_{p,\e}^\dag c_{q,\g}^\dag c_{q,\g} c_{p,\e}
    + c_{q,\e}^\dag c_{p,\g}^\dag c_{q,\g} c_{p,\e}}
  = G K \sum \p{n_{q,\g} n_{p,\e}
    - c_{q,\e}^\dag c_{q,\g} c_{p,\g}^\dag c_{p,\e}}.
\end{align}
Defining
\begin{align}
  \1_p \equiv c_{p,\e}^\dag c_{p,\e} + c_{p,\g}^\dag c_{p,\g},
  &&
  \sigma_p^j \equiv \sum_{\alpha,\beta}
  c_{p\alpha}^\dag \sigma^j_{\alpha\beta} c_{p\beta},
\end{align}
for Pauli matrices $\sigma^j$, we can therefore write
\begin{align}
  H_{\t{int}}^{\t{single-band}} = G K \sum
  \sp{\p{\f{\1_q+\sigma_q^z}{2}} \p{\f{\1_p-\sigma_p^z}{2}}
    - \sigma_q^- \sigma_p^+}.
\end{align}
Up to a global shift in energy, this result is equivalently
\begin{align}
  H_{\t{int}}^{\t{single-band}}
  = - \f14 G K \sum \v\sigma_q\c\v\sigma_p
  = - G K \v S \c \v S,
  &&
  \v S \equiv \f12 \sum_p \v\sigma_p.
\end{align}
Note that the collective spin operators $\v S$ are extensive in the
number of particles, $N$, and that the overlap integral $K\sim L^{-1}$
for $L$ lattice sites, as
\begin{align}
  K = \int d^3x \abs{\phi_0\p{x}}^4
  = \int d^3x \abs{\f1{\sqrt{L}}\sum_j w_j\p{x}}^4
  \approx \f1{L^2} \sum_j \int d^3x \abs{w_j\p{x}}^4
  = \f1L \int d^3x \abs{w_0\p{x}}^4,
\end{align}
where $w_j\p{x}$ is a ground-band Wannier orbital localized at site
$j$, and the remaining overlap integral is independent of $L$.  The
approximation above is equivalent to neglecting inter-site
interactions and interaction-assisted hopping.  If $L\sim N$ then the
single-band interaction Hamiltonian is extensive in the number of
particles ($N$), as one should expect.


\section{Inter-band interactions}

We now consider interactions between atoms in two different bands, and
define $a_{p\sigma} \equiv c_{p,0,\sigma}$ (for atoms in band 0) and
$b_{p\sigma} \equiv c_{p,1,\sigma}$ (for atoms in band 1).  The
single-particle energies in the first two bands take the form
$E_{qn}=\bar E_n+2J_n\cos\p{\pi q}$ with
$\abs{\bar E_1-\bar E_0}\gg\abs{J_1}\gg\abs{J_0}$.  In the weakly
interacting limit, conserving single-particle energies requires
conserving particle number within each band, so the relevant
interactions are
\begin{align}
  H_{\t{int}}^{\t{inter-band}}
  = G K_{0,1} \sum
  \p{a_{p+r,\sigma}^\dag b_{q-r,\bar\sigma}^\dag
    + b_{p+r,\sigma}^\dag a_{q-r,\bar\sigma}^\dag}
  b_{q\bar\sigma} a_{p\sigma}
\end{align}
for $K_{0,1}\equiv K^{0,1}_{0,1}=-K^{0,0}_{1,1}>0$.  In words: atoms
in different bands may either remain in their respective clock states,
or swap a clock excitation.

The band curvatures from nonzero $J_0,J_1$ further restrict allowed
momentum changes (i.e. values of $r$) from inter-band interactions by
imposing additional constraints to conserve net single-particle
energy.  In the limit $\abs{J_1}\gg\abs{J_0}$, we find that the only
terms which conserve net single-particle energy are proportional to
$b_{q\tau}^\dag b_{q\sigma}$ and $b_{-q,\tau}^\dag b_{q\sigma}$, which
gives us
\begin{multline}
  H_{\t{int}}^{\t{inter-band}}
  = G K_{0,1} \sum
  \p{a_{p\sigma}^\dag b_{q\bar\sigma}^\dag
    + b_{q\sigma}^\dag a_{p\bar\sigma}^\dag
    + a_{p+2q,\sigma}^\dag b_{-q,\bar\sigma}^\dag
    + b_{-q,\sigma}^\dag a_{p+2q,\bar\sigma}^\dag}
  b_{q\bar\sigma} a_{p\sigma} \\
  = G K_{0,1} \sum
  \p{b_{q\bar\sigma}^\dag b_{q\bar\sigma} a_{p\sigma}^\dag a_{p\sigma}
    - b_{q\sigma}^\dag b_{q\bar\sigma} a_{p\bar\sigma}^\dag a_{p\sigma}
    + b_{-q,\bar\sigma}^\dag b_{q\bar\sigma} a_{p+2q,\sigma}^\dag a_{p\sigma}
    - b_{-q,\sigma}^\dag b_{q\bar\sigma} a_{p+2q,\bar\sigma}^\dag a_{p\sigma}}.
\end{multline}
The first two terms in this sum
($\sim b_{q\bar\sigma}^\dag b_{q\bar\sigma}$ and
$b_{q\sigma}^\dag b_{q\bar\sigma}$) correspond to density-density
terms and an exchange of clock state between atoms in different bands.
Unlike the single-band case, momentum-exchanging processes are now
disallowed due to unequal band curvatures: the fact that $J_0\ne J_1$
implies $E_{p,0}+E_{q,1}\ne E_{q,0}+E_{p,1}$.  The latter terms in the
sum ($\sim b_{-q,\bar\sigma}^\dag b_{q\bar\sigma}$ and
$b_{-q,\sigma}^\dag b_{q\bar\sigma}$) appear because
$E_{q,1}=E_{-q,1}$, i.e. the excited-band atom pays no energetic
penalty for the momentum change $q\to-q$.  To conserve net momentum in
such a process, the ground-band atom must simultaneously change
momentum as $p\to p+2q$.  At face value, this process does not
conserve energy: $E_{p,0}\ne E_{p+2q,0}$, and in turn
$E_{p,0}+E_{q,1}\ne E_{p+2q,0}+E_{-q,1}$.  For this reason, such a
process was not allowed in the single-band case.  In the two-band case
with $\abs{J_1}\gg\abs{J_0}$, however, small changes in the momentum
of the excited-band atom result in large changes in its
single-particle energy (i.e. large on the scale of conserved energy
violation, which is $\sim J_0$).  A small momentum shift
$-q\to-q+\delta q$ (together with a corresponding shift
$p+2q\to p+2q-\delta q$) is therefore sufficient to restore the
initial single-particle energy of the two atoms.

We can simplify the first set of terms of the inter-band Hamiltonian
similarly to the way that we did in the single-band case by defining
collective (pseudo-)spin operators $\v S_n$ for band $n$.  Up to a
global shift in energy, we can then take
\begin{align}
  \sum \p{b_{q\bar\sigma}^\dag b_{q\bar\sigma} a_{p\sigma}^\dag a_{p\sigma}
    - b_{q\sigma}^\dag b_{q\bar\sigma} a_{p\bar\sigma}^\dag a_{p\sigma}}
  \to -2\v S_0\c\v S_1,
\end{align}
which implies
\begin{align}
  H_{\t{int}}^{\t{inter-band}}
  = -2 G K_{0,1} \v S_0\c\v S_1
  + G K_{0,1} \sum
  \p{b_{-q,\bar\sigma}^\dag b_{q\bar\sigma} a_{p+2q,\sigma}^\dag a_{p\sigma}
    - b_{-q,\sigma}^\dag b_{q\bar\sigma} a_{p+2q,\bar\sigma}^\dag a_{p\sigma}}.
\end{align}


\section{Two-atom band-hopping}

If we use driving schemes to make $\bar E_1=\bar E_0$ (discussed in a
separate set of notes), we must also consider interaction terms which
do not conserve particle number within each band.  Such terms take the
form
\begin{align}
  H_{\t{int}}^{\t{band-hopping}} = - G K_{0,1}
  \sum \p{b_{p+r,\e}^\dag b_{q-r,\g}^\dag a_{q,\g} a_{p,\e} + \t{h.c.}}.
  \label{eq:H_int_band_hopping}
\end{align}
Energy conservation in the limit $\abs{J_1}\gg\abs{J_0}$ now forces
$r=\p{q-p}/2\pm1/2$, such that the single-particle energies of
excited-band states addressed by $H_{\t{int}}^{\t{band-hopping}}$ are
\begin{multline}
  2J_1\sp{\cos\p{\pi\sp{p+r}} + \cos\p{\pi\sp{q-r}}}
  = 2J_1\sp{\cos\p{\pi~\f{p+q}{2}+\f{\pi}{2}}
    + \cos\p{\pi~\f{p+q}{2}-\f{\pi}{2}}} \\
  = 2J_1\sp{-\sin\p{\pi~\f{p+q}{2}} + \sin\p{\pi~\f{p+q}{2}}}
  = 0.
\end{multline}
Similarly to the situation we encountered in the case of inter-band
interactions, at face value a constant single-particle energy of all
excited-band states generally implies that energy is not conserved by
the band-hopping processes in \eqref{eq:H_int_band_hopping}.  Due to
the strong dependence of excited-band single-particle energies
$E_{q,1}$ on quasi-momenta $q$, however, energy conservation can be
restored by a small shift $r\to r+\delta r$.  Keeping only terms in
\eqref{eq:H_int_band_hopping} with $r=\p{q-p}/2\pm1/2$, we thus find
\begin{align}
  H_{\t{int}}^{\t{band-hopping}}
  = - \sqrt{2} G K_{0,1} \sum
  \p{d_{\p{p+q}/2}^\dag a_{q,\g} a_{p,\e} + \t{h.c.}},
\end{align}
where
\begin{align}
  d_s^\dag \equiv \f1{\sqrt2}
  \p{b_{s+1/2,\e}^\dag b_{s-1/2,\g}^\dag
    + b_{s-1/2,\e}^\dag b_{s+1/2,\g}^\dag}.
\end{align}
For $s\ne t$, these two-body excited-band operators satisfy the
commutation relations of hard-core bosons, i.e.
\begin{align}
  \sp{d_s, d_t} = \sp{d_s^\dag, d_t^\dag} = \sp{d_s, d_t^\dag} = 0,
\end{align}
\begin{align}
  \sp{d_s, d_s^\dag}
  = 1 - \f12 \p{\tilde n_{s+1/2,\e}^\dag + \tilde n_{s-1/2,\e}^\dag
    + \tilde n_{s+1/2,\g}^\dag + \tilde n_{s-1/2,\g}^\dag},
  &&
  \tilde n_{p\sigma} \equiv b_{p\sigma}^\dag b_{p\sigma}.
\end{align}


\section{Net interaction Hamiltonian}

Putting everything together and letting $G_X\equiv G K_X$, our total
interaction Hamiltonian is
\begin{align}
  H_{\t{int}}
  &= - G_0\v S_0\c\v S_0 - G_1\v S_1\c\v S_1 - 2 G_{0,1}\v S_0\c\v S_1
  \tag*{} \\ &\quad + G_{0,1} \sum
  \p{b_{-q,\bar\sigma}^\dag b_{q\bar\sigma} a_{p+2q,\sigma}^\dag a_{p\sigma}
    - b_{-q,\sigma}^\dag b_{q\bar\sigma} a_{p+2q,\bar\sigma}^\dag a_{p\sigma}}
  \tag*{} \\ &\quad - \sqrt2 G_{0,1} \sum
  \p{d_{\p{p+q}/2}^\dag a_{q,\g} a_{p,\e} + \t{h.c.}},
\end{align}
where
\begin{align}
  d_s^\dag \equiv \f1{\sqrt2}
  \p{b_{s+1/2,\e}^\dag b_{s-1/2,\g}^\dag
    + b_{s-1/2,\e}^\dag b_{s+1/2,\g}^\dag}.
\end{align}
Notes:
\begin{enumerate}[label=(\roman*)]
\item The last set of terms ($\sim d_s^\dag$ and $d_s$) appear only if
  we use driving schemes to match the mean band energies, thereby
  effective setting $\bar E_1=\bar E_0$; these terms can be neglected
  in the absence of such driving.
\item The number of nonvanishing terms in $\v S_n$ is proportional to
  the occupation of band $n$.  If there is a large population of atoms
  in the ground band and a small population of atoms in the excited
  band, then we can safely neglect the $\sim\v S_1\c\v S_1$ term,
  which is doubly small compared to e.g. $\v S_0\c\v S_1$.
\item Similarly, if the ground band is nearly filled with atoms in one
  clock state and the excited band only sparsely populated with atoms
  in the opposite clock state, then we can neglect the terms
  $\sim b_{-q,\bar\sigma}^\dag b_{q\bar\sigma} a_{p+2q,\sigma}^\dag
  a_{p\sigma}$, as $b_{-q,\bar\sigma}^\dag b_{q\bar\sigma}$ manifestly
  addresses a small number of atoms while
  $a_{p+2q,\sigma}^\dag a_{p\sigma}$ is only nonzero for quasi-momenta
  $p$ with $p+2q$ unoccupied.
\item If there is a nonzero population of atoms in the excited band,
  the terms
  $\sim b_{-q,\sigma}^\dag b_{q\bar\sigma} a_{p+2q,\bar\sigma}^\dag
  a_{p\sigma}$ will generally cause delocalization of clock
  excitations across all quasi-momenta (on a time scale
  $\sim1/G_{0,1}$ which is extensive with $N$).
\end{enumerate}


\end{document}
