\documentclass[aps,notitlepage,nofootinbib,11pt]{revtex4-1}

% linking references
\usepackage{hyperref}
\hypersetup{
  breaklinks=true,
  colorlinks=true,
  linkcolor=blue,
  filecolor=magenta,
  urlcolor=cyan,
}

%%% header / footer
\usepackage{fancyhdr} % easier header and footer management
\pagestyle{fancy} % page formatting style
\fancyhf{}
\usepackage{lastpage} % for referencing last page
\cfoot{\thepage~of \pageref{LastPage}} % "x of y" page labeling
\renewcommand{\headrulewidth}{0pt} % remove horizontal line in header

% figures
\usepackage{hyperref} % for linking references
\usepackage{graphicx,float} % for figures
\usepackage{grffile} % help latex properly identify figure extensions
\graphicspath{{./figures/}} % set path for all graphics
\usepackage[caption=false]{subfig} % subfigures ("subfloat")
\newcommand{\sref}[1]{\protect\subref{#1}}

% inline lists
\usepackage[inline]{enumitem}
\setlist[enumerate,1]{label={(\roman*)}}

%%% symbols, notations, etc.
\usepackage{physics,braket,amssymb} % physics and math packages
\usepackage{accents} % for resolving some accent (e.g. tilde) issues
\renewcommand{\t}{\text} % text in math mode
\newcommand{\f}[2]{\dfrac{#1}{#2}} % shorthand
\newcommand{\p}[1]{\left(#1\right)} % parenthesis
\renewcommand{\sp}[1]{\left[#1\right]} % square parenthesis
\renewcommand{\set}[1]{\left\{#1\right\}} % curly parenthesis
\newcommand{\bk}{\Braket} % shorthand
\renewcommand{\d}{\partial} % partial d

\usepackage{dsfont}
\newcommand{\1}{\mathds{1}}

\newcommand{\Z}{\mathbb Z}
\renewcommand{\O}{\mathcal O}

\usepackage{accents}
\newcommand{\utilde}[1]{\underaccent{\tilde}{#1}}


\begin{document}

\title{Generating new forms of spin-orbit coupling with a 1-D optical
  lattice clock}

\author{Michael A. Perlin}

\maketitle
\thispagestyle{fancy}


The primary motivation of this work is to use the ${}^{87}$Sr optical
lattice clock (OLC) to study general forms of spin-orbit coupling.  To
accomplish this task, we will draw analogies between the OLC and
trapped ions, with the hope of facilitating an exchange of ideas and
techniques between the trapped ion and optical lattice communities.
We will map degrees of freedom between the two systems as follows: the
quasi-momentum of an atom in the OLC corresponds to the on-site
orbital occupied by a trapped ion in a Coulomb lattice; the band index
of an OLC atom corresponds to a vibrational (phonon) mode of an ion;
and the (electronic) clock states of an OLC atom constitute a
pseudo-spin degree of freedom which corresponds to two optical states
of an ion.  For brevity, we will sometimes refer to the clock states
as the ``spin'' of the OLC atoms; we will not address the nuclear spin
degree of freedom of OLC atoms, and will assume that they are
nuclear-spin-polarized.

We start by considering a two level atom in a one-dimensional (1-D)
optical lattice interrogated by a plane-wave clock laser.  After a
rotating wave approximation, the dynamics of this system is
effectively described by the Hamiltonian
\begin{align}
  H
  = \f{p^2}{2m} + V_0\sin^2\p{k_L z} - \f12\delta\sigma^z
  - \f12\Omega\p{e^{ikz}\sigma^+ + e^{-ikz}\sigma^-},
  \label{eq:start}
\end{align}
where $p$ and $m$ are respectively the momentum and mass of the atom,
$V_0$ and $k_L$ are the lattice depth and wavenumber; $z$ is the
position along the lattice; $\delta=\omega-\omega_0$ is the detuning
of the clock laser frequency $\omega$ from the atomic clock transition
energy $\omega_0$; $\sigma^j$ with $j\in\set{x,y,z}$ is a Pauli matrix
addressing the spin of the atom;
$\sigma^\pm\equiv\p{\sigma^x\pm i\sigma^y}/2$ are spin raising and
lowering operators; $\Omega$ is the bare Rabi frequency of the clock
laser, and $k$ is the wavenumber of the clock laser along the lattice
axis.


\section{Quasi-momentum expansion and diagonalization}

The first two terms of the Hamiltonian in \eqref{eq:start} have
discrete translational invariance, which means that they may be
diagonalized with quasi-momentum and band index as good quantum
numbers.  The spatial eigenfunctions of these terms correspond to
solutions of the Mathieu equation, whose eigenfunctions
$\bk{z|qn}=\phi_{qn}\p{z}$ and corresponding energy eigenvalues
$E_{qn}$ are, in a periodic lattice with $L$ sites, indexed by a
quasi-momentum $q$ and band index $n$, where $q/k_L\in\Z/L$ and
$n\in\mathbb N_0$.  Additionally introducing the index
$s\in\set{\pm 1}$ to label the ground ($s=-1$) and excited ($s=1$), we
define field operators $c_{qns}$ which act on the vacuum as
$c_{qns}^\dag\ket{\t{vacuum}}=\ket{qns}$ and obey the standard
fermionic commutation relations.  The Hamiltonian in \eqref{eq:start}
can then be written in the general form
\begin{align}
  H
  = \sum_{q,n,s}\p{E_{qn}-\f12s\delta}c_{qns}^\dag c_{qns}
  - \f12\sum_{\substack{q,n,s\\g,m,r}}
  \Omega^{qns}_{gmr} c_{gmr}^\dag c_{qns},
  \label{eq:field_operators}
\end{align}
where $\Omega^{qns}_{gmr}\equiv\Omega\bk{gmr|e^{-iskz}|qns}$ is a
coupling constant between states $\ket{qns}$ and $\ket{gmr}$.  This
coupling constant vanishes unless $r=\bar s\equiv-s$.  Furthermore, in
what amounts to a stationary phase approximation (see Appendix
\ref{sec:laser_coupling}), we can say that
$\Omega^{qns}_{gm\bar s} \approx \Omega^{qns}_{gm\bar s}
\delta_{g,q-sk}$, which is a result of a momentum kick $k$ imparted
into OLC atoms when they absorb or emit a photon; this approximation
becomes exact when the clock photon momentum $k$ is commensurate with
the lattice.  Making this substitution, the Hamiltonian in
\eqref{eq:field_operators} becomes
\begin{align}
  H
  = \sum_{q,n,s}\p{E_{qn}-\f12s\delta}c_{qns}^\dag c_{qns}
  - \f12\sum_{q,n,s,m}\Omega^{qns}_{q-sk,m\bar s}
  c_{q-sk,m,\bar s}^\dag c_{qns}.
  \label{eq:kick}
\end{align}
Further redefining field operators, energies, and coupling constants
\begin{align}
  b_{qns} \equiv c_{q+sk/2,ns},
  &&
  E_{qns} \equiv E_{q+sk/2,n},
  &&
  \Omega^{qs}_{nm} = \Omega^{q+sk/2,ns}_{q-sk/2,m\bar s},
  \label{eq:transformation}
\end{align}
diagonalizes the Hamiltonian in \eqref{eq:kick} with respect to
quasi-momentum $q$:
\begin{align}
  H
  = \sum_{q,n,s}\p{E_{qns}-\f12s\delta} b_{qns}^\dag b_{qns}
  - \f12\sum_{q,n,s,m} \Omega^{qs}_{nm} b_{qm\bar s}^\dag b_{qns}.
  \label{eq:full_H}
\end{align}
This redefinition amounts to a gauge transformation of the field
operators, which shifts the OLC band structure (i.e. dispersion
relation; see Figure \ref{fig:bands}).  Appendix
\ref{sec:laser_coupling} covers some useful symmetries of
$\Omega^{qs}_{nm}$.

\begin{figure}
  \subfloat[]{
    \includegraphics[width=0.45\textwidth]{qn_bands_V10.pdf}
    \label{fig:before}
  } \subfloat[]{
    \includegraphics[width=0.45\textwidth]{qns_bands_V10.pdf}
    \label{fig:after}
  }
  \caption{Band diagrams of the Sr-87 OLC with a lattice depth
    $V_0=10E_R$ (where $E_R=k_L^2/2m\approx22~\t{kHz}$ is the lattice
    recoil energy) \sref{fig:before} before, and \sref{fig:after}
    after the gauge transformation in \eqref{eq:transformation}. The
    legend in \sref{fig:after} indicates the clock state (spin) for
    each line.}
  \label{fig:bands}
\end{figure}

A few of observations about the Hamiltonian in \eqref{eq:full_H}.
First, there is no coupling between states with different
(gauge-transformed) quasi-momenta $q$.  Once initialized, therefore,
the single-particle physics of the this OLC is reducible to the
dynamics within each sub-manifold of states with fixed quasi-momentum
$q$.  This fact is in analogy with trapped ions which are pinned to a
single site of a Coulomb crystal.  Second, there is no direct coupling
between different states with the same spin $s$; such coupling only
occurs at second order in the coupling strength $\Omega$.  In
principle, it is possible to introduce first-order coupling between
states with different band index $n$ and the same spin $s$ by
modulating the lattice in either phase or amplitude, but for now we
leave this consideration for future work.  Third, the Hamiltonian in
\eqref{eq:full_H} exhibits the standard type of spin-orbit coupling
investigated in the $^{87}$Sr OLC at JILA, in which different
quasi-momenta $q$ are energetically favorable for different spins $s$.
This spin-orbit coupling is similar to that encountered in a condensed
matter setting, and can lead to interesting topological physics.
Finally, the 1-D OLC Hamiltonian exhibits a spin-phonon-like coupling
between spin $s$ and band index $n$, which we will see more clearly in
Section \ref{sec:ions}.


\section{Mapping onto trapped ions}
\label{sec:ions}

In this section, we illustrate more clearly the analogy between the
1-D optical lattice clock and a 1-D crystal of trapped ions.  To
simplify our analysis, assume single-particle occupation each
quasi-momentum in the OLC, which translates to having a single ion per
site in a Coulomb crystal.  We will also only consider transitions
between adjacent bands (i.e. vibrational modes). We now define the
``lowering operators'' $a_{qnm}\equiv\sum_s\op{q,n-1,s}{qns}$; in
addition to
\begin{align}
  \bar E_{qn} \equiv \f12\p{E_{qn,+} + E_{qn,-}},
  &&
  \Delta_{qn} \equiv E_{qn,+} - E_{qn,-},
\end{align}
which are the mean energy ($\bar E_{qn}$) and energy splitting
($\Delta_{qn}$) between spin states within one band (i.e. between
$\ket{qn,+}$ and $\ket{qn,-}$); and finally
\begin{align}
  \Omega_{qn} \equiv \Omega^{q,+}_{nn} = \Omega^{q,-}_{nn},
  &&
  \Lambda_{qn}^x
  \equiv \f12\p{\Omega^{q,+}_{n-1,n} + \Omega^{q,-}_{n-1,n}},
  &&
  \Lambda_{qn}^y
  \equiv \f12\p{\Omega^{q,+}_{n-1,n} - \Omega^{q,-}_{n-1,n}},
\end{align}
which are intra-band ($\Omega_{qn}$) and inter-band ($\Lambda_{qn}^x$,
$\Lambda_{qn}^y$) coupling coefficients, respectively corresponding to
coefficients for carrier and side-band transitions.  In terms of these
quantities, the Hamiltonian from \eqref{eq:full_H} can be written as
\begin{align}
  H
  = \sum_{q,n} \sp{\op{qn}
    \p{\bar E_{qn} + \sp{\Delta_{qn}-\delta}S^z - \Omega_{qn}S^x}
    - \Lambda_{qn}^x \p{a_{qn}^\dag + a_{qn}}S^x
    - \Lambda_{qn}^y i\p{a_{qn}^\dag - a_{qn}}S^y},
  \label{eq:separated}
\end{align}
where we have introduced the on-band spin operators
$S^j\equiv\sigma^j/2$. The full derivation of the Hamiltonian in
\eqref{eq:separated} from that in \eqref{eq:full_H} is provided in
Appendix \ref{sec:spin_separation}.

The operators $a_{qn}+a_{qn}^\dag$ and $i\p{a_{qn}^\dag-a_{qn}}$ in
\eqref{eq:separated} can be identified with position and momentum
operators in a trapped ion Hamiltonian. In this sense we are
engineering a second type of spin-orbit coupling in the OLC: between
spin and band modes, which is analogous to coupling between spin and
vibrational phonon modes in a trapped ion system.


\subsection{Deep lattice approximation}

As the lattice depth $V_0$ is increased relative to the lattice recoil
energy $E_R\equiv k_L^2/2m$, i.e. when $V_0\gg E_R$, the energies
$E_{qns}$ and coupling coefficients $\Omega^{qs}_{nm}$ lose their
dependence on quasi-momentum $q$, which implies that
\begin{align}
  \bar E_{qn} \to E_n \equiv \f1L\sum_q E_{qn},
  &&
  \Delta_{qn} \to 0,
\end{align}
\begin{align}
  \Omega_{qn} \to \Omega_n \equiv \f1L\sum_q \Omega_{qn},
  &&
  \Lambda_{qn}^x \to 0,
  &&
  \Lambda_{qn}^y \to
  \Lambda_n^y \equiv \f1L\sum_q \Lambda_{qn}^y,
\end{align}
where $L$ is the number of lattice sites in the OLC. The fact that
$\Lambda_{qn}^x$ vanishes while $\Lambda_{qn}^y$ is preserved is a
consequence of the symmetries of $\Omega^{qs}_{nm}$, which we derive
in Appendix \ref{sec:laser_coupling} and summarize in
\eqref{eq:symmetries}. With these simplifications, the single-particle
Hamiltonian in \eqref{eq:separated} reduces to the simpler form
\begin{align}
  H
  = \sum_{q,n}\sp{\op{qn}\p{E_n - \delta S^z - \Omega_n S^x}
    - \Lambda_n^y i\p{a_{qn}^\dag - a_{qn}}S^y}.
  \label{eq:deep}
\end{align}
This Hamiltonian has lost the standard type of spin-orbit coupling
investigated in the 1-D OLC, in that there is no dependence of any
dynamical variables on the crystal momentum $q$.  Nonetheless, the
spin-phonon like coupling is preserved through the
$a_{qn}^\dag-a_{qn}$ term.


\section{Dynamics}
\label{sec:dynamics}

We now consider the OLC Hamiltonian after diagonalization in
quasi-momentum, i.e. \eqref{eq:full_H}, in the interaction picture of
the bare energies $E_{qns}-s\delta/2$:
\begin{align}
  H_I
  = -\f12\sum_{q,n,s,m} \Omega^{qs}_{nm}
  \exp\sp{i\p{\Delta^{qs}_{nm}+s\delta}t}
  b_{qm\bar s}^\dag b_{qns},
  \label{eq:full_H_I}
\end{align}
where $\Delta^{qs}_{nm}\equiv E_{qm\bar s}-E_{qns}$ is the energy gap
between the single-particle states $\ket{qm\bar s}$ and $\ket{qns}$ at
no detuning ($\delta=0$). With the ${}^{87}$Sr OLC, in practice the
Rabi frequency $\Omega$ is much smaller than the energy difference
between two bands, such that
\begin{align}
  \abs{\Omega^{qs}_{nm}} < \Omega\ll\abs{\Delta^{qs}_{nm}}
  \label{eq:weak_laser}
\end{align}
for any $q,n,s$ and $m\ne n$. Without detuning, all inter-band
couplings in \eqref{eq:full_H_I} therefore vanish by the secular
approximation. In the trapped ion system, \eqref{eq:weak_laser}
corresponds to having an interrogation laser which is too weak to
excite vibrational modes of the ions. While it is possible to set a
static detuning $\delta$ which bridges the energy difference between
different bands, doing so will still ultimately result in dynamics
which are reducible to that of a two-level system. In order to induce
nontrivial multi-band dynamics, we must therefore introduce time
dependence into the Hamiltonian. A simple means of introducing time
dependence is to modulate the detuning $\delta$ or Rabi frequency
$\Omega$, which respectively correspond to frequency and amplitude
modulation of the clock-state interrogation laser.


\subsection{Frequency modulation}
\label{sec:freq_mod}

Modulating the detuning as
$\delta\p{t}=\delta_0-\tilde\delta\cos\p{\nu t}$ for some mean
$\delta_0$, modulation amplitude $\tilde\delta$, and modulation
frequency $\nu$ results in the Hamiltonian
\begin{align}
  H_I
  = -\f12\sum_{\substack{q,n,s\\m,\kappa}}
  J\p{\kappa,s\tilde\delta/\nu} \Omega^{qs}_{nm}
  \exp\sp{i\p{\Delta^{qs}_{nm}+s\delta_0-\kappa\nu}t}
  b_{qm\bar s}^\dag b_{qns},
  \label{eq:freq_mod_H}
\end{align}
where $\kappa\in\Z$ and $J\p{n,x}$ is the $n$-th order Bessel function
of the first kind evaluated at $x$. If $\nu\gg\Omega$ and the OLC is
initialized in the state $\ket\phi=\ket{q_0n_0s_0}$, then the dynamics
induced by \eqref{eq:freq_mod_H} can be equivalently realized by the
effective time-independent Hamiltonian
\begin{align}
  H_I^{\t{eff}}
  = \sum_{q,n,s} \epsilon_{qns}^\phi b_{qns}^\dag b_{qns}
  - \f12\sum_{q,n,s,m} J\p{\kappa^{qs}_{nm},s\tilde\delta/\nu}
  \Omega^{qs}_{nm} b_{qm\bar s}^\dag b_{qns},
  \label{eq:freq_mod_H_eff}
\end{align}
where $\kappa^{qs}_{nm}\in\Z$ minimizes
$\abs{\Delta^{qs}_{nm}+s\delta_0-\kappa^{qs}_{nm}\nu}$ and the reduced
energies $\epsilon_{qns}^\phi$ satisfy
\begin{align}
  E_{qns} - \f12s\delta_0
  = E_{q_0n_0s_0} - \f12s_0\delta_0
  + \ell_{qns}^\phi\nu + \epsilon_{qns}^\phi
  \label{eq:reduced_E}
\end{align}
with $\ell_{qns}^\phi\in\Z$ and $\abs{\epsilon_{qns}^\phi}<\nu/2$. The
derivations of \eqref{eq:freq_mod_H} and \eqref{eq:freq_mod_H_eff} are
provided in Appendix \ref{sec:freq_mod_derivation}.

\begin{figure}[hp]
  \captionsetup[subfloat]{farskip=1pt,captionskip=1pt}
  \subfloat[First excited band, $V_0=80E_R$]{
    \includegraphics[width=0.45\textwidth]{freq_mod_band_V80.pdf}
    \label{fig:freq_mod_band_V80}
  } \subfloat[Excited clock state, $V_0=80E_R$]{
    \includegraphics[width=0.45\textwidth]{freq_mod_spin_V80.pdf}
    \label{fig:freq_mod_spin_V80}
  }
  \\
  \subfloat[First excited band, $V_0=40E_R$]{
    \includegraphics[width=0.45\textwidth]{freq_mod_band_V40.pdf}
    \label{fig:freq_mod_band_V40}
  } \subfloat[Excited clock state, $V_0=40E_R$]{
    \includegraphics[width=0.45\textwidth]
    {freq_mod_spin_V40.pdf}
    \label{fig:freq_mod_spin_V40}
  }
  \\
  \subfloat[First excited band, $V_0=10E_R$]{
    \includegraphics[width=0.45\textwidth]
    {freq_mod_band_V10.pdf}
    \label{fig:freq_mod_band_V10}
  } \subfloat[Excited clock state, $V_0=10E_R$]{
    \includegraphics[width=0.45\textwidth]
    {freq_mod_spin_V10.pdf}
    \label{fig:freq_mod_spin_V10}
  }
  \caption{Time evolution of states initially in the lowest band
    ($n=0$) and ground clock state ($s=-1$) subject to the Hamiltonian
    in \eqref{eq:freq_mod_H_eff} with a detuning
    $\delta=\Delta\cos\p{\Delta t}$ for $\Delta$ equal to the mean
    energy gap between the lowest two bands.  Color indicates the
    population of the state specified in the captions, which also
    indicate the lattice depths.  The interaction strength
    $\Omega^{0,-}_{1,0}/2\pi$ is approximately $43~\t{Hz}$ in a deep
    lattice with $V_0=80E_R$, $51~\t{Hz}$ in a medium lattice with
    $V_0=40E_R$, and $58~\t{Hz}$ in a shallow lattice with
    $V_0=40E_R$.  In all cases, $\Omega=1~\t{kHz}$.}
  \label{fig:freq_mod}
\end{figure}

At zero mean detuning ($\delta_0=0$), any choice of
$\nu=\Delta^{qs}_{nm}/\ell$ for $\ell\in\Z$ can, in principle, induce
multi-band dynamics.  As an example, Figure \ref{fig:freq_mod} shows
the time evolution of atoms initially in the lowest band and ground
clock state (i.e. $\ket{q,0,-}$) subject to a detuning
$\delta\p{t}=\Delta\cos\p{\Delta t}$, where $\Delta$ is the mean
energy gap between the lowest two bands (vibrational modes).  The
plots in this figure show the expectation values
$\bk{\op{q}\otimes\O_B\otimes\O_S}$ over time for
$\O_B,\O_S\in\set{\1,\op{1}}$ and a variety of lattice depths $V_0$.
Here $\O_{B,S}=\1$ traces out over the corresponding degree of freedom
(i.e. band or spin), while $\O_{B,S}=\op{1}$ appropriately selects out
either the first excited band or the excited clock state.  The value
of $\bk{\op{q}\otimes\op{1}\otimes\1}$ thus measures the total
population of atoms in the first excited band (independent of clock
state), while $\bk{\op{q}\otimes\1\otimes\op{1}}$ measures the total
population of atoms in the excited clock state (independent of band).

In a deep lattice ($V_0=80E_R$), the $q$-dependence of the energy gaps
$\Delta^{qs}_{nm}$ vanishes. We therefore see no spin-orbit coupling
in a deep lattice (Figures \ref{fig:freq_mod_band_V80} and
\ref{fig:freq_mod_spin_V80}), as the modulation frequency $\nu$ is
resonant with these gaps for all $q$. In the medium depth lattice
($V_0=40E_R$; Figures \ref{fig:freq_mod_band_V40} and
\ref{fig:freq_mod_spin_V40}), $\nu$ is only approximately resonant
with the energy gaps for all $q$, leading to some spin-orbit
coupling. Finally, spin-orbit coupling is most prominent in the
shallow lattice ($V_0=10E_R$; Figures \ref{fig:freq_mod_band_V10} and
\ref{fig:freq_mod_spin_V10}), in which the energy gaps, and thus OLC
dynamics, are strongly selective on quasi-momentum. In all cases,
there is evidence of spin-phonon-like coupling in the form of common
features between the excited band and excited clock state populations.


\subsection{Amplitude modulation}
\label{sec:amp_mod}

We can alternately modulate the amplitude of the clock laser as
$\Omega\to\Omega\cos\p{\nu t}$ for some frequency $\nu$, which results
in the Hamiltonian
\begin{align}
  H_I
  = -\f14\sum_{\substack{q,n,s\\m,r}} \Omega^{qs}_{nm}
  \exp\sp{i\p{\Delta^{qs}_{nm}+s\delta+r\nu}t}
  b_{qm\bar s}^\dag b_{qns},
  \label{eq:amp_mod_H}
\end{align}
for $r\in\set{-1,1}$. At no detuning, the amplitude modulation
frequency $\nu$ can be chosen to bridge some particular energy gap
$\Delta^{qs}_{nm}$ to induce transitions between different
bands. Similarly to the restriction on dynamics induced by the
time-independent OLC Hamiltonian in \eqref{eq:full_H_I} due to the
limitation in \eqref{eq:weak_laser}, however, in the ${}^{87}$Sr OLC
the differences between the different band gaps $\Delta^{qs}_{nm}$ are
generally larger than the Rabi coupling $\Omega$. The dynamics of the
OLC with amplitude modulation of the clock laser at no detuning thus
generally reduces to that of a two-level system.

While most choices of detuning $\delta$ do not change the above
argument, the symmetric effect of $\delta$ on different clock states
means that it is possible to engineer four-level dynamics in the OLC
for quasi-momenta $q$ with
$\Delta^{q,-}_{nm}\approx\Delta^{q,+}_{nm}$. Letting
\begin{align}
  \utilde\Delta^{qs}_{nm}
  \equiv \Delta^{qs}_{nm} + s\delta
  = \p{E_{qm\bar s} - \f12\bar s\delta} - \p{E_{qns} - \f12s\delta}
\end{align}
be the total band gap between $\ket{qm\bar s}$ and $\ket{qns}$ at
detuning $\delta$, if $\Delta^{0,-}_{nm}=\Delta^{0,+}_{nm}$ and we set
$\delta\approx\Delta^{0,\pm}_{1,0}/2$, for example, then one can work
out that
\begin{align}
  \utilde\Delta^{0,+}_{0,0}
  = \utilde\Delta^{0,-}_{1,0}
  = \utilde\Delta^{0,+}_{1,1}
  = \Delta^{0,\pm}_{1,0}/2
  \approx \delta
  \label{eq:gaps}
\end{align}
Setting $\nu=\delta$ then simultaneously couples
$\ket{0,0,-}\leftrightarrow\ket{0,0,+}
\leftrightarrow\ket{0,1,-}\leftrightarrow\ket{0,1,+}$, while coupling
to all higher bands can be neglected by the secular approximation.

\begin{figure}
  \captionsetup[subfloat]{farskip=1pt,captionskip=1pt}
  \subfloat[$V_0=10E_R$]{
    \includegraphics[width=0.45\textwidth]{qns_bands_detuned_V10.pdf}
    \label{fig:test}
  } \subfloat[$V_0=40E_R$]{
    \includegraphics[width=0.45\textwidth]{qns_bands_detuned_V40.pdf}
    \label{fig:test2}
  }
  \caption{Band diagrams of the Sr-87 OLC with a detuning
    $\delta=\Delta/2$ for two lattice depths $V_0$.  Legends indicate
    the clock state (spin) for each line.}
  \label{fig:bands_detuned}
\end{figure}

Figure \ref{fig:bands_detuned} shows the band diagrams which results
from setting $\delta=\Delta/2$ with lattice depths of $V_0=10E_R$ and
$V_0=40E_R$. While the energy gaps between the first four energy
levels (corresponding to both clock states in the lowest two bands)
are identical (at $q=0$), the gap to the fifth level (in the third
band) is different by an amount sufficiently large to make the fifth
level off-resonant for state transition.

Figure \ref{fig:amp_mod} shows the time evolution of atoms initially
in the lowest band and ground clock state (i.e. $\ket{q,0,-}$) subject
to a static detuning $\delta=\Delta/2$ and amplitude modulation with
frequency $\nu=\Delta/2$. As before, spin-orbit coupling is most
prominent in a shallow lattice, and is suppressed in a deep lattice;
spin-phonon-like coupling is present for all lattice depths.

Note that similarly to the case in Section \ref{sec:freq_mod}, if
$\nu\gg\Omega$ and the OLC is initialized in the state
$\ket\phi=\ket{q_0n_0s_0}$, then dynamics induced by
\eqref{eq:amp_mod_H} may be equivalently realized by the effective
time-independent Hamiltonian
\begin{align}
  H_I^{\t{eff}}
  = \sum_{q,n,s} \epsilon_{qns}^\phi b_{qns}^\dag b_{qns}
  - \f14\sum_{q,n,s,m} C\p{\kappa^{qs}_{nm}} \Omega^{qs}_{nm}
  b_{qm\bar s}^\dag b_{qns}
\end{align}
where $\kappa^{qs}_{nm}\in\Z$ minimizes
$\abs{\Delta^{qs}_{nm}+s\delta_0-\kappa^{qs}_{nm}\nu}$;
$C\p{\kappa}\equiv1$ when $\abs{\kappa}=1$ and $C\p{\kappa}\equiv0$
otherwise; and the reduced energies $\epsilon_{qns}^\phi$ satisfy
\eqref{eq:reduced_E} with $\ell_{qns}^\phi\in\Z$ and
$\abs{\epsilon_{qns}^\phi}<\nu/2$.

\begin{figure}[h!]
  \captionsetup[subfloat]{farskip=1pt,captionskip=1pt}
  \subfloat[First excited band, $V_0=80E_R$]{
    \includegraphics[width=0.45\textwidth]{amp_mod_band_V80.pdf}
    \label{fig:amp_mod_band_V80}
  } \subfloat[Excited clock state, $V_0=80E_R$]{
    \includegraphics[width=0.45\textwidth]{amp_mod_spin_V80.pdf}
    \label{fig:amp_mod_spin_V80}
  }
  \\
  \subfloat[First excited band, $V_0=40E_R$]{
    \includegraphics[width=0.45\textwidth]{amp_mod_band_V40.pdf}
    \label{fig:amp_mod_band_V40}
  } \subfloat[Excited clock state, $V_0=40E_R$]{
    \includegraphics[width=0.45\textwidth]{amp_mod_spin_V40.pdf}
    \label{fig:amp_mod_spin_V40}
  }
  \\
  \subfloat[First excited band, $V_0=10E_R$]{
    \includegraphics[width=0.45\textwidth]{amp_mod_band_V10.pdf}
    \label{fig:amp_mod_band_V10}
  } \subfloat[Excited clock state, $V_0=10E_R$]{
    \includegraphics[width=0.45\textwidth]{amp_mod_spin_V10.pdf}
    \label{fig:amp_mod_spin_V10}
  }
  \caption{Plots corresponding to those in Figure \ref{fig:freq_mod},
    but for a static detuning $\delta=-\Delta/2$ and amplitude
    modulated clock laser which takes
    $\Omega\to\Omega\cos\p{\Delta t}$.}
  \label{fig:amp_mod}
\end{figure}


\newpage
\appendix

\section{Properties of the laser-induced coupling constants}
\label{sec:laser_coupling}

Spatial eigenfunctions of atoms on a 1-D periodic lattice are indexed
by quasi-momentum $q$ and band $n$, and take the form
\begin{align}
  \bk{z|qn}
  = \phi_{qn}\p{z}
  = e^{iqz} u_{qn}\p{z}
  = e^{iqz} \sum_{\kappa=-\infty}^\infty c_{qn}^{(\kappa)} e^{i2k_L z\kappa},
\end{align}
where $k_L$ is the wavenumber of the lattice photons, and all
coefficients $c_{qn}^{(\kappa)}\in\mathbb R$.  We can thus expand
\begin{align}
  \Omega^{qns}_{gm}
  \equiv \Omega\bk{gm|e^{-iskz}|qn}
  = \Omega \int dz~ e^{i\p{q-sk-g}z}
  \sum_{\kappa,\ell} e^{i2k_Lz\p{\kappa-\ell}}
  c_{gm}^{(\ell)} c_{qn}^{(\kappa)},
\end{align}
which vanishes unless $g\approx q-sk$\footnote{This claim is formally
  a stationary phase approximation, and becomes exact if $k$ is
  commensurate with the lattice.} and $\kappa=\ell$, so
$\Omega^{qns}_{gm}\approx\Omega^{qns}_{q-sk,m}\delta_{g,q-sk}$ only
couples the state $\ket{qns}$ to $\ket{q-sk,m\bar s}$.  Defining
\begin{align}
  \Omega^{qs}_{nm}
  \equiv \Omega^{q+sk/2,ns}_{q-sk/2,m}
  = \Omega \sum_\kappa c_{q-sk/2,m}^{(\kappa)} c_{q+sk/2,n}^{(\kappa)},
\end{align}
we can say that
\begin{align}
  \Omega^{qs}_{nm}
  = \Omega^{q+sk/2,ns}_{q-sk/2,m}
  = \p{\Omega^{q+sk/2,ns}_{q-sk/2,m}}^*
  = \Omega^{q-sk/2,m\bar s}_{q+sk/2,n}
  = \Omega^{q+\bar sk/2,m\bar s}_{q-\bar sk/2,n}
  = \Omega^{q\bar s}_{mn}.
  \label{eq:flip_both}
\end{align}
We can also use the fact that
$\phi_{qn}\p{z}=\p{-1}^n\phi_{-q,n}\p{z}^*$ to say
\begin{align}
  \Omega^{qs}_{nm}
  = \Omega\int dz~ \phi_{q-sk/2,m}\p{z}^* \phi_{q+sk/2,n}\p{z}
  = \p{-1}^{n+m}\Omega^{-q,s}_{mn}.
  \label{eq:flip_bands}
\end{align}
To summarize \eqref{eq:flip_both} and \eqref{eq:flip_bands}:
\begin{align}
  \Omega^{qs}_{nm}
  = \p{-1}^{n+m} \Omega^{-q,s}_{mn}
  = \p{-1}^{n+m} \Omega^{-q,\bar s}_{nm}
  = \Omega^{q\bar s}_{mn}.
  \label{eq:symmetries}
\end{align}


\section{Separating out the pseudo-spin degree of freedom}
\label{sec:spin_separation}

We start with the 1-D OLC Hamiltonian after diagonalization in
quasi-momentum, i.e.
\begin{align}
  H
  = \sum_{q,n,s}\p{E_{qns}-\f12s\delta} b_{qns}^\dag b_{qns}
  - \f12\sum_{q,n,s,m} \Omega^{qs}_{nm} b_{qm\bar s}^\dag b_{qns},
\end{align}
and define
\begin{align}
  \bar E_{qn} \equiv \f12\p{E_{qn,+} + E_{qn,-}},
  &&
  \Delta_{qn} \equiv E_{qn,+} - E_{qn,-},
\end{align}
\begin{align}
  \Omega_{qn} \equiv \Omega^{q,+}_{nn} = \Omega^{q,-}_{nn},
  &&
  \Lambda^{qs}_{nm}
  \equiv \Omega^{qs}_{n,n-m} = \Omega^{q\bar s}_{n-m,n},
\end{align}
in terms of which
\begin{multline}
  H = \sum_{q,n,s}\sp{\p{\bar E_{qn} + \f12 s \sp{\Delta_{qn}-\delta}}
    b_{qns}^\dag b_{qns}
    - \f12\Omega_{qn} b_{qn\bar s}^\dag b_{qns}} \\
  - \f12\sum_{\substack{q,n,s\\0<m\le n}}\Lambda^{qs}_{nm}
  \p{b_{q,n-m,\bar s}^\dag b_{qns} + b_{qns}^\dag b_{q,n-m,\bar s}}.
  \label{eq:lambda}
\end{multline}
In order to separate out the spin degree of freedom, we consider only
at most single-particle occupation of each quasi-momentum, and define
the ``lowering operators'' $a_{qnm}\equiv\op{q,n-m}{qn}$ in addition
to the on-band spin operators $S^j\equiv\sigma^j/2$ which act only on
the spin degree of freedom.  We can rewrite \eqref{eq:lambda} in terms
of these operators for single particles as
\begin{align}
  H
  = \sum_{q,n} \op{qn}\sp{\bar E_{qn}
    + \p{\Delta_{qn}-\delta} S^z - \Omega_{qn} S^x}
  - \sum_{\substack{q,n\\0<m\le n}}\Lambda^{qs}_{nm}
  \p{a_{qnm} S^{\bar s} + a_{qnm}^\dag S^s}.
\end{align}
We now expand
\begin{multline}
  \sum_s\Lambda^{qs}_{nm}\p{a_{qnm} S^{\bar s} + a_{qnm}^\dag S^s} =
  \Lambda^{q,+}_{nm}\p{a_{qnm}S^- + a_{qnm}^\dag S^+}
  + \Lambda^{q,-}_{nm}\p{a_{qnm}S^+ + a_{qnm}^\dag S^-} \\
  = \f12S^x \p{\Lambda^{q,+}_{nm} + \Lambda^{q,-}_{nm}} \p{a_{qnm} +
    a_{qnm}^\dag} + \f12 iS^y \p{\Lambda^{q,+}_{nm} -
    \Lambda^{q,-}_{nm}} \p{-a_{qnm} + a_{qnm}^\dag},
\end{multline}
which motivates the definitions
\begin{align}
  \Lambda_{qnm}^x
  \equiv \f12\p{\Lambda^{q,+}_{nm} + \Lambda^{q,-}_{nm}}
  = \f12\p{\Omega^{q,+}_{n-m,n} + \Omega^{q,-}_{n-m,n}}
  = \f12\p{\Omega^{q,+}_{n-m,n} + \p{-1}^m\Omega^{-q,+}_{n-m,n}}, \\
  \Lambda_{qnm}^y
  \equiv \f12\p{\Lambda^{q,+}_{nm} - \Lambda^{q,-}_{nm}}
  = \f12\p{\Omega^{q,+}_{n-m,n} - \Omega^{q,-}_{n-m,n}}
  = \f12\p{\Omega^{q,+}_{n-m,n} - \p{-1}^m\Omega^{-q,+}_{n-m,n}},
\end{align}
such that
\begin{multline}
  H = \sum_{q,n}\op{qn}
  \sp{\bar E_{qn} + \p{\Delta_{qn}-\delta}S^z - \Omega_{qn}S^x} \\
  - \sum_{\substack{q,n\\m>0}}
  \sp{\Lambda_{qnm}^x \p{a_{qnm}^\dag + a_{qnm}}S^x
    + \Lambda_{qnm}^y i\p{a_{qnm}^\dag - a_{qnm}}S^y}.
\end{multline}
Letting $a_{qn}\equiv a_{qn,1}$;
$\Lambda_{qn}^j\equiv\Lambda_{qn,1}^j$ for $j=x,y$; and neglecting all
inter-band (side-band) couplings beyond nearest bands, we arrive at
the Hamiltonian given in \eqref{eq:separated}:
\begin{align}
  H
  = \sum_{q,n} \sp{\op{qn}
    \p{\bar E_{qn} + \sp{\Delta_{qn}-\delta}S^z - \Omega_{qn}S^x}
    - \Lambda_{qn}^x \p{a_{qn}^\dag + a_{qn}}S^x
    - \Lambda_{qn}^y i\p{a_{qn}^\dag - a_{qn}}S^y}.
\end{align}


\section{Frequency modulation of the clock laser}
\label{sec:freq_mod_derivation}

In the interaction picture of the single-particle state energies
$E_{qns}-s\delta/2$, the Hamiltonian of the 1-D OLC is
\begin{align}
  H_I
  = - \f12\sum_{q,n,s,m} \Omega^{qs}_{nm}
  \exp\sp{i\p{\Delta^{qs}_{nm}+s\delta}t} b_{qm\bar s}^\dag b_{qns},
\end{align}
where $\Delta^{qs}_{nm}\equiv E_{qm\bar s}-E_{qns}$ is the energy gap
between the single-particle states $\ket{qm\bar s}$ and $\ket{qns}$ at
no detuning. If we modulate the detuning as
$\delta\p{t}=\delta_0-\tilde\delta\cos\p{\nu t}$ for some mean
$\delta_0$, modulation amplitude $\tilde\delta$, and modulation
frequency $\nu$, then the Hamiltonian becomes
\begin{align}
  H_I
  = - \f12\sum_{q,n,s,m} \Omega^{qs}_{nm}
  \exp\sp{i\utilde\Delta^{qs}_{nm}t-is\tilde\delta\sin\p{\nu t}/\nu}
  b_{qm\bar s}^\dag b_{qns},
\end{align}
where
\begin{align}
  \utilde\Delta^{qs}_{nm}
  \equiv \Delta^{qs}_{nm} + s\delta_0
  = \p{E_{qm\bar s}-\f12\bar s\delta_0} - \p{E_{qns}-\f12s\delta_0}
  = \utilde E_{qm\bar s} - \utilde E_{qns}.
\end{align}
is the total band gap between $\ket{qm\bar s}$ and $\ket{qns}$ at the
mean detuning $\delta_0$ and
$\utilde E_{qns}\equiv E_{qns}-\f12s\delta_0$.

Letting $\tau\equiv\nu t$ and $\beta\equiv\tilde\delta/\nu$, the
exponential $\exp\p{is\beta\sin\tau}$ is periodic in $\tau$ with
period $2\pi$, which means that we can expand it in a Fourier series
as
\begin{align}
  e^{-is\beta\sin\tau}
  = \sum_{\kappa=-\infty}^\infty e^{-i\kappa\tau}\f1{2\pi}
  \int_{-\pi}^\pi dx~ e^{i\kappa x - is\beta\sin x}
  = \sum_\kappa J\p{\kappa,s\beta} e^{-i\kappa\tau}
\end{align}
where $J\p{n,x}$ is the $n$-th order Bessel function of the first kind
evaluated at $x$. It follows that
\begin{align}
  H_I
  = -\f12\sum_{\substack{q,n,s\\m,\kappa}}
  J\p{\kappa,s\tilde\delta/\nu} \Omega^{qs}_{nm}
  \exp\sp{i\p{\utilde\Delta^{qs}_{nm}-\kappa\nu}t}
  b_{qm\bar s}^\dag b_{qns},
\end{align}
which is precisely the Hamiltonian given in
\eqref{eq:freq_mod_H}. Letting $\kappa^{qs}_{nm}\in\Z$ minimize
$\abs{\utilde\Delta^{qs}_{nm}-\kappa^{qs}_{nm}\nu}$ and assuming that
$\nu\gg\Omega$, by the secular approximation
\begin{align}
  H_I
  = -\f12\sum_{q,n,s,m}
  J\p{\kappa^{qs}_{nm},s\tilde\delta/\nu} \Omega^{qs}_{nm}
  \exp\sp{i\p{\utilde\Delta^{qs}_{nm}-\kappa^{qs}_{nm}\nu}t}
  b_{qm\bar s}^\dag b_{qns}.
  \label{eq:freq_mod_H_derived}
\end{align}
If an atom in the OLC is initially in the state
$\ket\phi=\ket{q_0n_0s_0}$, then the only relevant states for all OLC
dynamics will be those for which
\begin{align}
  \utilde E_{qns}
  = \utilde E_{q_0n_0s_0} + \ell_{qns}^\phi\nu + \epsilon_{qns}^\phi
  \label{eq:relevant_condition}
\end{align}
with $\ell_{qns}^\phi\in\Z$ and reduced energies $\epsilon_{qns}^\phi$
satisfying $\abs{\epsilon_{qns}^\phi}\lesssim\Omega\ll\nu$. For all
relevant states, therefore,
\begin{align}
  \utilde\Delta^{qs}_{nm} - \kappa^{qs}_{nm}\nu
  = \utilde E_{qm\bar s} - \utilde E_{qns} - \kappa^{qs}_{nm}\nu
  = \p{\ell_{qm\bar s}^\phi - \ell_{qns}^\phi - \kappa^{qs}_{nm}}\nu
  + \epsilon_{qm\bar s}^\phi - \epsilon_{qns}^\phi.
\end{align}
As $\abs{\epsilon_{qm\bar s}^\phi-\epsilon_{qns}^\phi}<\nu/2$ for all
relevant $q,n,s,m$, minimizing
$\abs{\utilde\Delta^{qs}_{nm}-\kappa^{qs}_{nm}\nu}$ forces
$\kappa^{qs}_{nm}=\ell_{qm\bar s}^\phi-\ell_{qns}^\phi$. The
Hamiltonian in \eqref{eq:freq_mod_H_derived} is then
\begin{align}
  H_I
  = -\f12\sum_{q,n,s,m}
  J\p{\kappa^{qs}_{nm},s\tilde\delta/\nu} \Omega^{qs}_{nm}
  \exp\sp{i\p{\epsilon_{qm\bar s}^\phi-\epsilon_{qns}^\phi}t}
  b_{qm\bar s}^\dag b_{qns},
\end{align}
and, with appropriate choice of interaction picture, is equivalent to
the effective Hamiltonian
\begin{align}
  H_I^{\t{eff}}
  = \sum_{q,n,s} \epsilon_{qns}^\phi b_{qns}^\dag b_{qns}
  - \f12\sum_{q,n,s,m} J\p{\kappa^{qs}_{nm},s\tilde\delta/\nu}
  \Omega^{qs}_{nm} b_{qns}^\dag b_{qm\bar s},
  \label{eq:freq_mod_H_eff_derived}
\end{align}
which has the great advantage over \eqref{eq:freq_mod_H_derived} of
lacking time dependence. In using \eqref{eq:freq_mod_H_eff_derived}
for e.g. computing a propagator or otherwise simulating OLC dynamics,
there is no need to actually keep track of which states are
``relevant'', as all irrelevant states will generally have reduced
energies $\epsilon_{qns}^\phi\sim\nu/2\gg\Omega$, and therefore
automatically decouple from all relevant states. A similar trick can
be used to effectively simulate OLC dynamics with amplitude modulation
as in \eqref{eq:amp_mod_H}, which results in the Hamiltonian
\begin{align}
  H_I^{\t{eff}}
  = \sum_{q,n,s} \epsilon_{qns}^\phi b_{qns}^\dag b_{qns}
  - \f14\sum_{q,n,s,m} C\p{\kappa^{qs}_{nm}} \Omega^{qs}_{nm}
  b_{qm\bar s}^\dag b_{qns},
\end{align}
for $C\p{\kappa}\equiv1$ when $\abs{\kappa}=1$ and
$C\p{\kappa}\equiv0$ otherwise.

\end{document}
